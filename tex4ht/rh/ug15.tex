% !! DO NOT MODIFY THIS FILE BY HAND !! Created by spool2tex.awk.
% Copyright (c) 1991-2002, The Numerical ALgorithms Group Ltd.
% All rights reserved.
%
% Redistribution and use in source and binary forms, with or without
% modification, are permitted provided that the following conditions are
% met:
%
%     - Redistributions of source code must retain the above copyright
%       notice, this list of conditions and the following disclaimer.
%
%     - Redistributions in binary form must reproduce the above copyright
%       notice, this list of conditions and the following disclaimer in
%       the documentation and/or other materials provided with the
%       distribution.
%
%     - Neither the name of The Numerical ALgorithms Group Ltd. nor the
%       names of its contributors may be used to endorse or promote products
%       derived from this software without specific prior written permission.
%
% THIS SOFTWARE IS PROVIDED BY THE COPYRIGHT HOLDERS AND CONTRIBUTORS "AS
% IS" AND ANY EXPRESS OR IMPLIED WARRANTIES, INCLUDING, BUT NOT LIMITED
% TO, THE IMPLIED WARRANTIES OF MERCHANTABILITY AND FITNESS FOR A
% PARTICULAR PURPOSE ARE DISCLAIMED. IN NO EVENT SHALL THE COPYRIGHT OWNER
% OR CONTRIBUTORS BE LIABLE FOR ANY DIRECT, INDIRECT, INCIDENTAL, SPECIAL,
% EXEMPLARY, OR CONSEQUENTIAL DAMAGES (INCLUDING, BUT NOT LIMITED TO,
% PROCUREMENT OF SUBSTITUTE GOODS OR SERVICES-- LOSS OF USE, DATA, OR
% PROFITS-- OR BUSINESS INTERRUPTION) HOWEVER CAUSED AND ON ANY THEORY OF
% LIABILITY, WHETHER IN CONTRACT, STRICT LIABILITY, OR TORT (INCLUDING
% NEGLIGENCE OR OTHERWISE) ARISING IN ANY WAY OUT OF THE USE OF THIS
% SOFTWARE, EVEN IF ADVISED OF THE POSSIBILITY OF SUCH DAMAGE.



% *********************************************************************
\head{chapter}{What's New in FriCAS}{ugWhatsNew}
% *********************************************************************

%------------------------------------------------------------------
\head{section}{Release Notes}{releaseNotes}
%------------------------------------------------------------------
\Language{} information can be found online at
{http://fricas.sourceforge.net}
%------------------------------------------------------------------
\subsubsection{FriCAS 1.3.1}
%------------------------------------------------------------------
\begin{itemize}

\item Categories with associative multiplication are now subcategories
      of categories with nonassociative multiplication.

\item Inlining optimization in now effective also in command line
      (interpreter) compiler.

\item Added conversions between finitely presented groups
      and permutation groups (Todd-Coxeter algorithm) and
      back.

\item Removed special handling of coercion of String to OutputForm
      from Spad compiler.

\item Former FramedModule is renamed to FractionalIdealAsModule.
      Added new FramedModule.

\item Whole interpreter is now included in executable (no need
      to load parts before use).

\end{itemize}

Bug fixes, in particular:

\begin{itemize}

\item Fixed build with sbcl-1.3.13.

\item Limits using the name of variable in limit point work
      now.

\item A few output fixes.

\item Several integrator fixes.

\item Removed wrond interpreter transform of '~='.

\item Fixed compilation of type parameters containing non-type values.

\item Plots sometimes used single precision.  Now they should
      always use double precision.

\end{itemize}
%------------------------------------------------------------------
\subsubsection{FriCAS 1.3.0}
%------------------------------------------------------------------
\begin{itemize}

\item Several domains and categories are more general,
      in particular matrices, indexed products and
      direct product.

\item ')show' now evaluates predicates.

\item Improved integrator, handles few more 'erf' cases and
      more algebraic functions.  Result should be
      simpler.

\item Added support for using FriCAS as ECL shared library.

\item Polynomial factorization uses Kaltofen-Shoup method when
      applicable.

\item '\$createLocalLibDb' defaults to false.

\item Simpler, more predictable equality for algebraic
      numbers (no longer uses 'trueEqual').

\item Renamed LinearlyExplicitRingOver to LinearlyExplicitOver.

\item Renamed 'length' in Tuple to '\#'.

\item Removed argumentless 'random'.

\end{itemize}

Bug fixes, in particular:

\begin{itemize}

\item Fixed several build problems.

\item Handle scripted symbols in DeRhamComplex.

\item Handle empty matrices in more places.

\item Fixed unparse of negative integers.

\item No longer crashes on quoted expressions in types.

\end{itemize}

%------------------------------------------------------------------
\subsubsection{FriCAS 1.2.7}
%------------------------------------------------------------------
\begin{itemize}

\item New package implementing van Hoej factorization algorithm
      for LODO-s.

\item Gcd over Expression(Integer) now uses modular method.

\item Improvements to integrator, in particular trigonometric
      functions are consistently integrated via transformation
      to complex exponentials.

\item Some categories and domains are more general.  In particular
      OrderedFreeMonoid is removed, as ordered case is handled
      by FreeMonoid.

\item Category Monad in renamed to Magma.  Domain Magma is
      renamed to FreeMagma.

\end{itemize}

Bug fixes, in particular:

\begin{itemize}

\item Coercion of square matrices to polynomials is fixed.

\item Problem with division by 0 in derivative of 'ellipticPi'
      is fixed.

\item Division in Ore algebras used to cause infinite loop
      when coefficients were power series.

\end{itemize}

%------------------------------------------------------------------
\subsubsection{FriCAS 1.2.6}
%------------------------------------------------------------------
\begin{itemize}

\item Polynomial factorization is available for larger class
      of base rings.

\item Improvements to integrator.

\item 'normalize' can be applied to list of expressions.

\item Eigenvalues can be computed over larger range of base fields.

\item Common denominator package handles now multivariate polynomials.

\item More uniform break (error) handling.

\end{itemize}

Bug fixes, in particular:

\begin{itemize}

\item 'distribute' handles 'box' operator.

\item Fixed problem with guessing over multivariate polynomials.

\item Fixed hashcode handling for Void in Aldor.

\end{itemize}

%------------------------------------------------------------------
\subsubsection{FriCAS 1.2.5}
%------------------------------------------------------------------
\begin{itemize}

\item Several improvements to integrator.

\item Improvements to handling of series, in particular new
      function 'prodiag' to compute infinite products, 'series' and
      'coefficients' for multivariate Taylor series, new 'laurent'
      function which builds Laurent series from order and stream of
      coefficients.

\item GMP should now work with sbcl on all platforms and with Closure CL
      on all platforms except for Power PC.

\item Added a few domains for discrete groups.

\item Extended GCD in Ore algebras can now return coefficients of
      both GCD and LCM.

\item New function for computing integrals of solutions of linear
      differential operators.

\item ')savesystem' command is now removed.

\item Continuation lines which begins like commands are no longer
      treated as commands.

\end{itemize}

Bug fixes, in particular:

\begin{itemize}

\item Fixed printing of scripted symbols.

\item Fixed 'totalDegreeSorted' (affected Groebner bases).

\item Fixed few problems with Hensel lifting (including SF bug 47).

\item Fixed 'series' in UnivariateLaurentSeriesConstructor.

\item Fixed 'order' in SparseUnivariatePowerSeries.

\item Printing of series now respect 'showall' setting, cyclic
      series are detected.

\item Fixed problem with interpreter preferring Union to base type.

\end{itemize}

%------------------------------------------------------------------
\subsubsection{FriCAS 1.2.4}
%------------------------------------------------------------------
\begin{itemize}

\item New cylindrical decomposition package.

\item New GnuDraw package for plotting via gnuplot.

\item Texmacs interface now handles Cork symbols.

\item Added double precision versions of several special functions
      (needed for plotting).

\item Nopile mode for Spad is changed to be more convenient.

\item 'stringMatch' is removed (was broken beyond repair).

\end{itemize}

Bug fixes, in particular:

\begin{itemize}

\item Fixed interpreter assignment to parts of nested aggregates
      (issue 376).

\item Fixed interpreter coercion from Equation to Boolean (issue 359).

\item Fix printing of '\%i' in types (issue 132).

\item Disabled incorrect shortcut during coercion (issue 29).

\item Difference of intervals now agrees with definition as interval
      operation.

\item Avoid overwriting loop limit and increment.

\item Fix a polynomial gcd failure due to bad reduction.

\item Avoid mangling unevaluated algebraic integrals.

\item Fix integration of unevaluated derivatives.

\item Restore parser handling of '\\/' and '/\\'.

\item Properly escape strings and symbols in TeXFormat.

\item Fix toplevel multiparameter macros.

\item Fix problem with missing parentheses around plexes.

\item Avoid crash when printing error message from '-eval'.

\item Redirect I/O when running programs from Closure CL.

\end{itemize}

%------------------------------------------------------------------
\subsubsection{FriCAS 1.2.3}
%------------------------------------------------------------------
\begin{itemize}

\item Improved integration in terms of 'Ei' and 'erf'.

\item Classical orthogonal polynomials may be used as expressions.

\item More cases of generalized indexing for two dimensional arrays.

\item Value of 'lambertW' at '-1/e' is now simplified.

\item FriCAS now knows that formal derivatives are commutative.

\item 'setelt' is renamed to 'setelt!'.

\item ')read' now creates intermediate files in current directory.

\item Continuation characters in comments are now respected.

\item In Spad '\$Lisp' calls now must have a type.

\item In Spad \spad{error} did only minimal checking of its argument.
  Now argument to \spad{error} must be a \spad{String} or \spad{OutputForm}
  or a literal list of \spad{OutputForm}-s.

\end{itemize}

Bug fixes, in particular:

\begin{itemize}

\item Input lines with empty continuation are no longer lost.

\item Types like "failed" now consistently use string quotes in output form.

\item Fixed pattern matching using \spad{%i} in patterns.

\item Fixed ')display op coerce'.

\item Fixed ')version' command.

\item Fixed crash when printing '\%'.

\item Fix a buffer overflow in HyperDoc.

\item Fixed HyperDoc errors in 'Dependants' and 'Users'.

\item HyperDoc browser better handles constructors with parameters.

\end{itemize}

%------------------------------------------------------------------
\subsubsection{FriCAS 1.2.2}
%------------------------------------------------------------------
\begin{itemize}
\item Improvements to 'integrate': better handling of algebraic
  integrals, new routine which handles some integrals containing
  'lambertW'.

\item Improvements to 'limit', now Gruntz algorithm knows about
  a few tractable functions.

\item Smith form of sparse integer matrices is now much more
  efficient.

\item Generalized indexing for two dimensional arrays.

\item Pile/nopile mode is now restored after ')read' or ')compile'.
  Piling rules now accept some forms of multiline lists.

\item Eliminated version checking in generated code. Note: this
  change means that Spad code compiled by earlier FriCAS versions
  will not run in FriCAS 1.2.2.

\item Updated Aldor interface to work with free Aldor.
\end{itemize}

Bug fixes, in particular:

\begin{itemize}
\item Interpreter can now handle complicated mutually recursive
  functions.

\item Spad compiler should now correctly handle 'has' inside a function.

\item Fixed derivatives of Whittaker functions.
\end{itemize}

%------------------------------------------------------------------
\subsubsection{FriCAS 1.2.1}
%------------------------------------------------------------------
\begin{itemize}
\item Improvements to 'integrate': a new routine for integration in
  terms of Ei, better handling of algebraic integrals.

\item Implemented 'erfi'.

\item Derivatives of 'asec', 'asech', 'acsc' and 'acsch' use different
  formula so that numeric evaluation of derivative will take correct
  branch on real axis.

\item Linear dependence package is changed to be consistent with
  linear solvers.

\item It is now possible to extract empty submatrices.

\item Changed default style of 3D graphics.

\item Support for building Mac OS application bundle.
\end{itemize}

Bug fixes, in particular:

\begin{itemize}
\item fixed few cases of wrong or unevaluated integrals.

\item better zero test during limit computation avoids division by
  zero.

\item fixed buffer overflow problems in view3D.

\item 'reducedSystem' on empty input returns basis of correct size.
\end{itemize}

%------------------------------------------------------------------
\subsubsection{FriCAS 1.2.0}
%------------------------------------------------------------------
\begin{itemize}
\item New MatrixManipulation package.

\item New ParallelIntegrationTools package.

\item Gruntz algorithm is now used also for finite one-sided limits.

\item FriCAS has now true 2-dimensional arrays (previously they were
emulated using vectors of vectors).

\item Speedups in some matrix operations and in arithmetic with
algebraic expressions.

\item FreeModule is now more general, it allows Comparable as second
argument.

\item Changed Spad parser, it now uses common scanner with
interpreter. Spad language is now closer to interpreter language and
Aldor. 'leave' is removed, 'free', 'generate' and 'goto' are now
keywords. Pile rules changed slightly, they should be more intuitive
now. Error messages from Spad parser should be slightly better.
\end{itemize}

Bug fixes, in particular:

\begin{itemize}
\item Fixed a few build problems.

\item Eliminated division by 0 during 'normalize'.

\item 'nthRootIfCan' removes leading zeros from generalized series
  (this avoids problems with power series expanders).

\item Fixed corruption of formal derivatives.

\item Fixed two problems with Fortran output.

\item Fixed ')untrace' and ')undo'. Fixed ')trace' with ECL.

\item Fixed problem with calling efricas if user's default shell is
  (t)csh.
\end{itemize}

%------------------------------------------------------------------
\subsubsection{FriCAS 1.1.8}
%------------------------------------------------------------------

\begin{itemize}
\item Improvements of pattern matching integrator, it can now
  integrate in terms of Fresnel integrals and better handles integrals
  in terms of Si and Ci.

\item Better integration of symbolic derivatives.

\item Better normalization of Liouvillian functions.

\item New package for computing limits using Gruntz algorithm.

\item Faster removal of roots from denominators.

\item New domains for multivariate Ore algebras and partial
  differential operators.

\item New package for noncommutative Groebner bases.

\item New domain for univariate power series with arbitrary exponents.

\item New special functions: Shi and Chi.

\item Several aggregates (in particular tables) allow more general
  parameter types.

\item New domain for hash tables using equality from underlying
  domain.
\end{itemize}

Bug fixes, in particular:

\begin{itemize}
\item Fixed problem with gcd failing due to bad reduction.

\item Fixed series of 'acot' and Puiseux series of several special
  functions.

\item Fixed wrong factorization of differential operators.

\item Fixed build problem on recent Mac OS X.
\end{itemize}

%------------------------------------------------------------------
\subsubsection{FriCAS 1.1.7}
%------------------------------------------------------------------

\begin{itemize}
\item Improved integration in terms of special functions.

\item Updated new graphics framework and graph theory package.

\item Added routines for numerical evaluation of several special
  functions.

\item Added modular method for computing polynomial gcd over algebraic
  extensions.

\item Derivatives of fresnelC and fresnelS are changed to agree with
  established convention.

\item When printing floats groups of digits are now separated by
  underscores (previously were separated by spaces).

\item Added C code for removing directories, this speeds up full build
  and should avoid build problems on Mac OSX.
\end{itemize}

Bug fixes, in particular:

\begin{itemize}
\item Series expansion now handle poles of Gamma.

\item Fixed derivatives of meijerG.
\end{itemize}

%------------------------------------------------------------------
\subsubsection{FriCAS 1.1.6}
%------------------------------------------------------------------

\begin{itemize}
\item Added experimental graph theory package.

\item Added power series expanders for Weierstrass elliptic functions
  at 0.

\item New functions: kroneckerProduct and kroneckerSum for matrices,
  numeric weierstrassInvariants and modularInvariantJ, symbolic Jacobi
  Zeta, double float numeric elliptic integrals.

\item New domains for vectors and matrices of unsigned 8 and 16 bit
  integers.

\item Changes to Spad compiler: underscores which are not needed as
  escape are now significant in Spad names and strings, macros with
  parameters are supported, added partial support for exceptions,
  braces can be used for grouping.

\item A few speedups.

\item Reduced disc space usage during build.
\end{itemize}

Bug fixes, in particular:

\begin{itemize}
\item Fixed eval of hypergeometricF at 0

\item Fixed problem with scope of macros.

\item Worked around problems with opening named pipes in several Lisp
  implementations.

\item Fixed a problem with searching documentation via HyperDoc.

\item Fixed build problem on Mac OSX.
\end{itemize}

%------------------------------------------------------------------
\subsubsection{FriCAS 1.1.5}
%------------------------------------------------------------------

\begin{itemize}
\item Added numeric version of lambertW.

\item New function 'rootFactor' which tries to write roots of products
  as products of roots.

\item 'try', 'catch' and 'finally' are now Spad keywords.

\item Experimental support for using gmp with Closure CL (64-bit
  Intel/Amd only).

\item New categories CoercibleFrom and ConvertibleFrom. New domain for
  ordinals up to epsilon0. New domain for matrices of machine
  integers. New package for solving linear equations written as
  expressions (faster then general expression solver).

\item Functions exported by Product() are now called 'construct',
  'first' and 'second' (instead of 'makeprod', 'selectfirst' and
  'selectsecond' respectively).

\item Some functions are now much faster, in particular bivariate
  factorization over small finite fields.

\item When using sbcl FriCAS now tries to preload statistical
  profiler.
\end{itemize}

Bug fixes, in particular:

\begin{itemize}
\item Fixed handling of Control-C in FriCAS compiled by recent sbcl.

\item Fixed HyperDoc crash due to bad handling of '\#'.

\item Fixed power series expanders for elliptic integrals.

\item Fixed 'possible wild ramification' problem with algebraic
  integrals.

\item 'has' in interpreter now correctly handles \spad{%}.

\item Spad compiler can now handle single \spad{=>} at top level of a
  function.

\item Fixed few problems with conditional types in Spad compiler.
\end{itemize}

%------------------------------------------------------------------
\subsubsection{FriCAS 1.1.4}
%------------------------------------------------------------------

\begin{itemize}
\item New domains for combinatorial probability theory by Franz
  Lehner.

\item Improved integration of algebraic functions.

\item Initial support for semirings.

\item Updated framework for theory of computations.

\item In Spad parser \spad{**, ^'} and \spad{->} are now
  right-associative.

\item Spad parser no longer transforms relational operators.

\item Join of categories is faster which speeds up Spad compiler.
\end{itemize}

Bug fixes, in particular:

\begin{itemize}
\item Retraction of 'rootOf' from Expression(Integer) to
  AlgebraicNumber works now.

\item Attempt to print error message about invalid type no longer
  crash (SF 2977357).

\item Fixed few problems in Spad compiler dealing with conditional
  exports.

\item HyperDoc now should find all function descriptions (previously
  it missed several).
\end{itemize}

%------------------------------------------------------------------
\subsubsection{FriCAS 1.1.3}
%------------------------------------------------------------------

\begin{itemize}
\item Added "jet bundle" framework by Werner Seiler and Joachim Schue,
  which includes completion procedure and symmetry analysis for PDE.

\item Better splitting of group representations (added Holt-Rees
  improvement to meatAxe).

\item Added numeric versions of some elliptic integrals and few more
  elliptic functions.

\item Speeded up FFCGP (finite fields via Zech logarithms).

\item New experimental flag (off by default, set via
  setSimplifyDenomsFlag) which if on causes removal of irrationalities
  from denominators. Usually it causes slowdown, but on some examples
  gives huge speedup. It may go away in future (when no longer
  needed).

\item Added experimental framework for theory of computations.
\end{itemize}

Bug fixes, in particular:

\begin{itemize}
\item Numerical solutions of polynomial systems have now required
  accuracy (SF 2418832).

\item Fixed problem with crashes during tracing.

\item Fixed a problem with nested iteration (SF 3016806).

\item Eliminated stack overflow when concatenating long lists.
\end{itemize}

%------------------------------------------------------------------
\subsubsection{FriCAS 1.1.2}
%------------------------------------------------------------------

\begin{itemize}
\item Experimental Texmacs interface and Texmacs format output.

\item Guessing package can now guess algebraic dependencies.

\item Expansion into Taylor series and limits now work for most
  special functions.

\item Spad to Aldor translator is removed.

\item Spad compiler no longer allows to denote sets using braces.
\end{itemize}

Bug fixes, in particular:

\begin{itemize}
\item Fixed few cases where elementary integrals were returned
  unevaluated or produced wrong results.

\item Unwanted numerical evaluation should be no longer a problem
  (FriCAS interpreter now very strongly prefers symbolic evaluation
  over numerical evaluation).

\item Fixed a truncation bug in guessing package which caused loss of
  some correct solutions.

\item TeX and MathML format should correctly put parentheses around
  and inside sums and products.

\item Fixed few problems with handling of Unicode.
\end{itemize}

%------------------------------------------------------------------
\subsubsection{FriCAS 1.1.1}
%------------------------------------------------------------------

\begin{itemize}
\item New graphics framework.

\item Support for using GMP with sbcl on 32/64 bit AMD/Intel
  processors (to activate it one must use '--with-gmp' option to
  configure).

\item Improvements to integration and normalization. In particular
  integrals containing multiple non-nested roots should now work much
  faster. Also FriCAS now can compute more integrals of Liouvillian
  functions.

\item Several new special functions.

\item Improvements to efricas.

\item Looking for default init file FriCAS now first tries to use
  '.fricas.input' and only if that fails it looks for '.axiom.input'.
\end{itemize}

Bug fixes, in particular:

\begin{itemize}
\item Numeric atan, asin and acos took wrong branch.

\item WeierstrassPreparation package did not work.

\item Saving and restoring history should be now more reliable.

\item Fixed two bugs in Spad compiler related to conditional
  compilation.

\item Fixed a problem with rational reconstruction which affected
  guessing package.
\end{itemize}

%------------------------------------------------------------------
\subsubsection{FriCAS 1.1.0}
%------------------------------------------------------------------

\begin{itemize}
\item New domains and packages: VectorSpaceBasis domain, DirichletRing
  domain, 3D graphic output in Wavefront .obj format, specialized
  machine precision numeric vectors and matrices (faster then general
  vectors and matrices), Html output.

\item Support Clifford algebras corresponding to non-diagonal matrix,
  added new operations.

\item 'normalize' now tries to simplify logarithms of algebraic
  constants.

\item New functions: Fresnel integrals, carmichaelLambda.

\item Speed improvements: several polynomial operations are faster,
  faster multiplication in Ore algebras, faster computation of strong
  generating set for permutation groups, faster coercions.

\item Several improvements to the guessing package (in particular new
  option Somos for restricting attention to Somos-like sequences
\end{itemize}

Bug fixes, in particular:

\begin{itemize}
\item FriCAS can now compute multiplicative inverse of a power series
  with constant term not equal to 1.

\item Fixed a problem with passing interpreter functions to algebra.

\item Two bugs causing crashes in HyperDoc interface are fixed.

\item FriCAS now ignores sign when deciding if number is prime.

\item A failing coercion that used to crash FriCAS is now detected.

\item 'has' test sometimes gave wrong result.

\item Plotting fixes.
\end{itemize}

%------------------------------------------------------------------
\subsubsection{FriCAS 1.0.9}
%------------------------------------------------------------------

\begin{itemize}
\item Speed improvements to polynomial multiplication, power series
  multiplication, guessing package and coercion of polynomials to
  expressions.

\item Domains for tensor products.

\item 'Complex(Integer)' is now UniqueFactorizationDomain.

\item Types in interpreter are now of type 'Type' (instead of
  'Domain') and categories in interpreter are of type 'Category'
  (instead of 'Subdomain(Domain)').

\item Interpreter functions can now return 'Type'.

\item New function for files: 'flush'.

\item Spad compiler: return in nested functions and nested functions
  returning functions.
\end{itemize}

Bug fixes, in particular:

\begin{itemize}
\item Several fixes to guessing package.

\item Avoid crash when unparsing equations.

\item Equation solver accepts more solutions.

\item Fixed handling of 'Tuple' in Spad parser.

\item Fixed miscompilation of record constructor by Spad compiler.
\end{itemize}

%------------------------------------------------------------------
\subsubsection{FriCAS 1.0.8}
%------------------------------------------------------------------

\begin{itemize}
\item Improved version of guessing package. It can now handle much
  larger problems than before. Added ability to guess functional
  substitution equations.

\item Experimental support for build using CMU CL

\item Various speed improvements including faster indexing for two
  dimensional arrays

\item By default FriCAS build tries to use sbcl.

\item Building no longer require patch.
\end{itemize}

Bug fixes, in particular:

\begin{itemize}
\item correct definition of random() for matrices

\item conditionals in .input files work again

\item Spad compiler now recognizes more types as equal

\item fixed problem with pattern-matching quote
\end{itemize}

%------------------------------------------------------------------
\subsubsection{FriCAS 1.0.7}
%------------------------------------------------------------------

\begin{itemize}
\item Comparisons between elements of the Expression domain are
  undefined. Earlier versions gave confusing results for expressions
  like '\%e < \%pi' -- now FriCAS will complain about '<' being
  undefined.

\item A domain for general quaternions was added.

\item Equality in Any is now more reasonable -- it uses equality from
  underlying domain if available.

\item Messages about loading of components are switched off by
  default.

\item Release build benefits from parallel make.

\item In Spad code a single quote now means that the following token
  is a symbol.

\item Reorganization of algebra sources, in particular several types
  have changed (this may affect users Spad code).
\end{itemize}

Bug fixes, in particular:

\begin{itemize}
\item Categories with default package can be used just after
  definition (fixes 1.0.6 regression).

\item Plots involving 0 or 1 work now.

\item Numbers in radix bigger than 10 appear correctly in TeX output.

\item Fixed browser crashes when displaying some domains.

\item Fix horizontal display of fractions.

\item Allow local domains in conditionals (in Spad code).

\item Fixed problem with splitting polynomials and nested extensions.
\end{itemize}

%------------------------------------------------------------------
\subsubsection{FriCAS 1.0.6}
%------------------------------------------------------------------

\begin{itemize}
\item the axiom script is no longer installed (use fricas script
  instead)

\item some undesirable simplification are no longer done by default,
  for example now asin(sin(3)) is left unevaluated

\item support lambda expressions using '+->' syntax and nested
  functions in Spad

\item better configure, support for Dragonfly BSD

\item faster bootstrap, also parallel (this does not affect speed of
  release build)
\end{itemize}

Several bug fixes, in particular:

\begin{itemize}
\item fixed a regression introduced in 1.0.4 which caused equality for
  nested products to sometimes give wrong result

\item corrected fixed output of floating point numbers,

\item operations on differential operators like symmetric power work
  now

\item fixed crashes related to coercing power series

\item functions returning Void can be traced
\end{itemize}

%------------------------------------------------------------------
\subsubsection{FriCAS 1.0.5}
%------------------------------------------------------------------

\begin{itemize}
\item improvement to normalize function, it performs now much stronger
  simplifications than before

\item better integration: due to improved normalize FriCAS can now
  integrate many functions that it previously considered unintegrable

\item improvement to Martin Rubey guessing package, for example it can
  now guess differential equation for the generating function of
  integer partitions

\item better support for using type valued functions

\item several bug fixes
\end{itemize}

%------------------------------------------------------------------
\subsubsection{FriCAS 1.0.4}
%------------------------------------------------------------------

\begin{itemize}
\item significant speedups for some operations (for example definite
  integration)

\item support for building algebra using user-defined optimization
  settings

\item support for mouse wheel in HyperDoc browser

\item included support for interfacing with Aldor

\item new optional Emacs mode and efricas script to run FriCAS inside
  emacs

\item better unparse

\item removed support for attributes (replaced by empty categories)
  and use of colon for type conversions in Spad code

\item a few bug fixes
\end{itemize}

%------------------------------------------------------------------
\subsubsection{FriCAS 1.0.3}
%------------------------------------------------------------------

\begin{itemize}
\item added multiple precision Gamma and logGamma functions

\item better line editing

\item removed some undocumented and confusing constructs from Spad
  language

\item added new categories for semiring and ordered semigroup, direct
  product of monoids is now a monoid

\item internal cleanups and restructurings

\item a few bug fixes
\end{itemize}

%------------------------------------------------------------------
\subsubsection{FriCAS 1.0.2}
%------------------------------------------------------------------

\begin{itemize}
\item ')nopiles' command gives conventional syntax

\item added pfaffian function

\item ECL support

\item Graphics and Hyperdoc work using openmcl or ECL

\item Output may be now delimited by user defined markers

\item Experimental support for using as a Lisp library

\item Spad compiler is now significantly faster

\item Several bug fixes
\end{itemize}

%------------------------------------------------------------------
\subsubsection{FriCAS 1.0.1}
%------------------------------------------------------------------

\begin{itemize}
\item Graphics and Hyperdoc work using sbcl or clisp

\item Builds under Cygwin (using Cygwin clisp)

\item MathML support contributed by Arthur C. Ralfs

\item Help files created by Tim Daly

\item Added SPADEDIT script

\item Full release caches all generated HyperDoc pages

\item Bug fixes, including implementing some missing functions and
  build fixes
\end{itemize}

%------------------------------------------------------------------
\subsubsection{FriCAS 1.0.0}
%------------------------------------------------------------------

The 1.0 release is the first release of FriCAS. Below we list main
differences compared to AXIOM September 2006.

Numerous bug fixes (in particular HyperDoc is now fully functional on
Unix systems).

FriCAS includes guessing package written by Martin Rubey. This package
provides unique ability to guess formulas for sequences of numbers or
polynomials.

Some computation, in particular involving Expression domain, should be
much faster. FriCAS to go trough its testsuite needs only half of the
time needed by AXIOM September 2006.

Spad compilation is faster (in some cases 2 times faster).

FriCAS is much more portable than AXIOM September 2006. It can be
build on Linux, many Unix systems (for example Mac OSX and Solaris 10)
and Windows. It can be build on top of gcl, sbcl, clisp or openmcl
(gcl and sbcl based FriCAS is fully functional, clisp or openmcl based
one lacks graphic support).

Many unused or non-working parts are removed from FriCAS. In
particular FriCAS does not contain support for NAG numerical library.

FriCAS can be build from sources using only a few pre-generated Lisp
files for bootstrap -- only to bootstrap Shoe translator. This means
that modifying FriCAS algebra is now much easier.

%------------------------------------------------------------------------
\head{section}{Changes to Spad language}{ugSpadChanges}
%------------------------------------------------------------------------
\begin{enumerate}
\item \spad{$} as name of current domain is no longer supported, use
  \spad{%} instead.

\item Attributes are no longer supported, use niladic categories with
  no exports instead.

\item Floating point numbers without leading zero are no longer
  supported, so instead of \spad{.01} use \spad{0.01}

\item Anonymous functions using \spad{#1}, \spad{#2}, etc. are no
  longer supported, to define anonymous functions use \textspadexpr{+->}.

\item Braces no longer construct sets. So instead of
  \textspadexpr{\{'sin, 'cos\}::Set(Symbol)} use
  \spad{set(['sin, 'cos])$Set(Symbol)}. %$

\item Old Spad used colon (\spad{:}) to denote conversion, like
  \spad{pretend} but performing even less checking. This is no longer
  supported, use \spad{::} or \spad{pretend} instead.

\item There was an alternative spelling for brackets and braces, in
  FriCAS this is no longer supported, so one has to write brackets and
  braces as is.

\item \spad{SubsetCategory} was handled in special way by the
  compiler. This is no longer supported.

\item Old Spad compiler used to transform relational operators
  \spad{~=,<=,>,>=} in ways which are correct for linear order, but
  may conflict with other uses (as partial order or when generating
  \spad{OutputForm}). FriCAS no longer performs this transformation.
  Similarely, Spad parser no longer treats \spadop{^} and \spadop{^=} in
  special way.

\item Quote in old Spad allowed to insert arbitrary literal Lisp data,
  FriCAS only allows symbols after quote. Code using old behavior
  needs to be rewritten, however it seems that this feature was almost
  unused, so this should be no problem.

\item Old Spad treated statement consisting just of constructor name
  (with arguments if needed) as request to import the constructor.
  FriCAS requires \spad{import} keyword.

\item In FriCAS \spad{**, ^, ->} are right associative. Also, right
  binding power of \textspadexpr{+->} is increased, which allows more natural
  writing of code.

\item Few non-working experimental features are removed, in particular
  partial support for APL-like syntax.

\item FriCAS implemented parametric macros in the Spad compiler.

\item FriCAS allows simplified form for exporting constants (without
  \spad{constant} keyword).

\item FriCAS added partial support for exception handling (currently
  only \spad{finally} part).

\item The \spad{leave} construct is removed from FriCAS. Use
  \spad{break} instead.

\item \spad{div} is no longer a keyword. \spad{free}, \spad{generate},
  \spad{goto} are FriCAS keywords.

\item '\$Lisp' calls now must have a type

\item \spad{error} did only minimal checking of its argument.  Now
  argument to \spad{error} must be a \spad{String} or \spad{OutputForm}
  or a literal list of \spad{OutputForm}-s.

\end{enumerate}

There are also library changes that affect user code:

\begin{enumerate}
\item \spad{**} lost its definition as exponentiation, use \spadop{^}
  instead.

\item \spadop{^} is no longer used as negation (it means exponentiation
  now) and \spadop{^=} no longer means inequality, use \spad{not} and
  \spad{~=} instead.

\item \spad{setelt} is renamed to \spad{setelt!}.

\item Operator properties are now symbols and not strings, so instead
  of \spad{has?(op, "even")} use \spad{has?(op, 'even)}

\item There is new category \spad{Comparable}, several constructors
  that asserted \spad{OrderedSet} now only assert \spad{Comparable}.
\end{enumerate}


%------------------------------------------------------------------
\head{section}{Online Information}{onlineInformation}
%------------------------------------------------------------------
\Language{} information can be found online at
\begin{itemize}
\item {http://fricas.sourceforge.net} -- The official homepage of
  FriCAS.
\item {http://axiom-wiki.newsynthesis.org} -- A wiki site related to
  FriCAS.
\item {http://sourceforge.net/p/fricas/code/HEAD/tree/} -- The
  official source code repository.
\item {https://github.com/fricas/fricas} -- A live git mirror of
  the official SVN repository.
\item {http://fricas.github.io} -- Documentation of FriCAS including
  the API of the FriCAS library.
\end{itemize}

% ------------------------------------------------------------------------
\head{section}{Old News about AXIOM Version 2.x}{ugWhatsNewNAG}
%------------------------------------------------------------------------

Many things have changed in this version of AXIOM and
we describe many of the more important topics here.

%------------------------------------------------------------------------
%\head{section}{The New \Language{} Library Compiler}{ugWhatsNewAsharp}
%------------------------------------------------------------------------
%
% A new compiler is now available for \Language{}.
% The programming language is referred to as the \Language{} Extension Language
% (or \aldor{} for short), and
% improves upon the old \Language{} language in many ways.
% The \spadsys{)compile} command has been upgraded to be able to
% invoke the new or old compilers.
% The language and the compiler are described in the hard-copy
% documentation which came with your \Language{} system.
%
% To ease the chore of upgrading your {\it .spad} files (old
% compiler) to {\it .as} files (new compiler), the
% \spadsys{)compile} command has been given a {\tt )translate}
% option. This invokes a special version of the old compiler which
% parses and analyzes your old code and produces augmented code
% using the new syntax.
% Please be aware that the translation is not necessarily one
% hundred percent complete or correct.
% You should attempt to compile the output with the \aldor{} compiler
% and make any necessary corrections.


% ----------------------------------------------------------------------
\head{subsection}{The NAG Library Link}{nagLinkIntro}
% ----------------------------------------------------------------------

Content removed, NAGLink is no longer included in FriCAS.


%------------------------------------------------------------------------
\head{subsection}{Interactive Front-end and Language}{ugWhatsNewLanguage}
%------------------------------------------------------------------------

The \spad{leave} keyword has been replaced by the
\spad{break} keyword for compatibility with the new AXIOM
extension language.
See section \spadref{ugLangLoopsBreak}
for more information.

Curly braces are no longer used to create sets. Instead, use
\spadfun{set} followed by a bracketed expression. For example,
\begin{xtc}
\begin{xtccomment}
\end{xtccomment}
\begin{spadsrc}
set [1,2,3,4]
\end{spadsrc}
\begin{TeXOutput}
\begin{fricasmath}{1}
\BRACE{1\COMMA 2\COMMA 3\COMMA 4}%
\end{fricasmath}
\end{TeXOutput}
\formatResultType{Set(PositiveInteger)}
\end{xtc}

Curly braces are now used to enclose a block (see section
\spadref{ugLangBlocks}
for more information). For compatibility, a block can still be
enclosed by parentheses as well.

% ``Free functions'' created by the \aldor{} compiler can now be
% loaded and used within the AXIOM interpreter. A {\it free
% function} is a library function that is implemented outside a
% domain or category constructor.

New coercions to and from type \spadtype{Expression} have been
added. For example, it is now possible to map a polynomial
represented as an expression to an appropriate polynomial type.

Various messages have been added or rewritten for clarity.

%------------------------------------------------------------------------
\head{subsection}{Library}{ugWhatsNewLibrary}
%------------------------------------------------------------------------

The \spadtype{FullPartialFractionExpansion}
domain has been added. This domain computes factor-free full
partial fraction expansions.
See section
\xmpref{FullPartialFractionExpansion}
for examples.

We have implemented the Bertrand/Cantor algorithm for integrals of
hyperelliptic functions. This brings a major speedup for some
classes of algebraic integrals.

We have implemented a new (direct) algorithm for integrating trigonometric
functions. This brings a speedup and an improvement in the answer
quality.

The {\sf SmallFloat} domain has been renamed
\spadtype{DoubleFloat} and {\sf SmallInteger} has been renamed
\spadtype{SingleInteger}. The new abbreviations as
\spadtype{DFLOAT} and \spadtype{SINT}, respectively.
We have defined the macro {\sf SF}, the old abbreviation for {\sf
SmallFloat}, to expand to \spadtype{DoubleFloat} and modified
the documentation and input file examples to use the new names
and abbreviations. You should do the same in any private \Language{}
files you have.

We have made improvements to the differential equation solvers
and there is a new facility for solving systems of first-order
linear differential equations.
In particular, an important fix was made to the solver for
inhomogeneous linear ordinary differential equations that
corrected the calculation of particular solutions.
We also made improvements to the polynomial
and transcendental equation solvers including the
ability to solve some classes of systems of transcendental
equations.

The efficiency of power series have been improved and left and right
expansions of \spad{tan(f(x))} at \spad{x =} a pole of \spad{f(x)}
can now be computed.
A number of power series bugs were fixed and the \spadtype{GeneralUnivariatePowerSeries}
domain was added.
The power series variable can appear in the coefficients and when this
happens, you cannot differentiate or integrate the series.  Differentiation
and integration with respect to other variables is supported.

A domain was added for representing asymptotic expansions of a
function at an exponential singularity.

For limits, the main new feature is the exponential expansion domain used
to treat certain exponential singularities.  Previously, such singularities
were treated in an {\it ad hoc} way and only a few cases were covered.  Now
AXIOM can do things like

\begin{verbatim}
limit( (x+1)^(x+1)/x^x - x^x/(x-1)^(x-1), x = %plusInfinity)
\end{verbatim}

in a systematic way.  It only does one level of nesting, though.  In other
words, we can handle \spad{exp(} some function with a pole \spad{)}, but not
\linebreak \spad{exp(exp(} some function with a pole \spad{)).}

The computation of integral bases has been improved through careful
use of Hermite row reduction. A P-adic algorithm
for function fields of algebraic curves in finite characteristic has also
been developed.

Miscellaneous:
There is improved conversion of definite and indefinite integrals to
\spadtype{InputForm};
binomial coefficients are displayed in a new way;
some new simplifications of radicals have been implemented;
the operation \spadfun{complexForm} for converting to rectangular coordinates
has been added;
symmetric product operations have been added to \spadtype{LinearOrdinaryDifferentialOperator}.

%------------------------------------------------------------------------
\head{subsection}{\HyperName}{ugWhatsNewHyperDoc}
%------------------------------------------------------------------------

The buttons on the titlebar and scrollbar have been replaced
with ones which have a 3D effect. You can change the foreground and
background colors of these ``controls'' by including and modifying
the following lines in your {\bf .Xdefaults} file.
\begin{verbatim}
Axiom.hyperdoc.ControlBackground: White
Axiom.hyperdoc.ControlForeground: Black
\end{verbatim}

For various reasons, \HyperName{} sometimes displays a
secondary window. You can control the size and placement of this
window by including and modifying
the following line in your {\bf .Xdefaults} file.
%
\begin{verbatim}
Axiom.hyperdoc.FormGeometry: =950x450+100+0
\end{verbatim}
%
This setting is a standard X Window System geometry specification:
you are requesting a window 950 pixels wide by 450 deep and placed in
the upper left corner.

Some key definitions have been changed to conform more closely
with the CUA guidelines. Press
F9
to see the current definitions.

Input boxes (for example, in the Browser) now accept paste-ins from
the X Window System. Use the second button to paste in something
you have previously copied or cut. An example of how you can use this
is that you can paste the type from an \Language{} computation
into the main Browser input box.


%------------------------------------------------------------------------
\head{subsection}{Documentation}{ugWhatsNewDocumentation}
%------------------------------------------------------------------------
We describe here a few additions to the on-line
version of the AXIOM book which you can read with
HyperDoc.

A section has been added to the graphics chapter, describing
how to build \twodim{} graphs from lists of points. An example is
given showing how to read the points from a file.
See section \spadref{ugGraphTwoDbuild}
for details.

A further section has been added to that same chapter, describing
how to add a \twodim{} graph to a viewport which already
contains other graphs.
See section
\spadref{ugGraphTwoDappend}
for details.

Chapter 3
and the on-line \HyperName{} help have been unified.

An explanation of operation names ending in ``?'' and ``!'' has
been added to the first chapter.
See the
end of the section
\spadref{ugIntroCallFun}
for details.

An expanded explanation of using predicates has
been added to the sixth chapter. See the
example involving \userfun{evenRule} in the middle of the section
\spadref{ugUserRules}
for details.

Documentation for the \spadsys{)compile}, \spadsys{)library} and
\spadsys{)load} commands has been greatly changed. This reflects
the ability of the \spadsys{)compile} to now invoke the \aldor{}
compiler, the impending deletion of the \spadsys{)load} command
and the new \spadsys{)library} command.
The \spadsys{)library} command replaces \spadsys{)load} and is
compatible with the compiled output from both the old and new
compilers.

%------------------------------------------------------------------------
\head{subsection}{\aldor{} compiler - Enhancements and Additions}{ugTwoTwoAldor}
%------------------------------------------------------------------------
Content removed - \aldor{} (now using name {\it Aldor}) is a separate
project.
%
%------------------------------------------------------------------------
\head{subsection}{New polynomial domains and algorithms}{ugTwoTwoPolynomials}
%------------------------------------------------------------------------
Univariate polynomial factorization over the integers has been
enhanced by updates to the \spadtype{GaloisGroupFactorizer} type
and friends from Frederic Lehobey (Frederic.Lehobey@lifl.fr, University of
Lille I, France).

The package constructor \spadtype{PseudoRemainderSequence}
provides efficient algorithms by Lionel Ducos
(Lionel.Ducos@mathlabo.univ-poitiers.fr, University of Poitiers, France)
for computing sub-resultants.
This leads to a speed up in many places in \Language{} where
sub-resultants are computed (polynomial system solving,
algebraic factorization, integration).

Based on this package, the domain constructor
\spadtype{NewSparseUnivariatePolynomial}
extends the constructor \spadtype{SparseUnivariatePolynomial}.
In a similar way, the \spadtype{NewSparseMultivariatePolynomial} extends
the constructor \spadtype{SparseUnivariatePolynomial};
it also provides some additional operations related
to polynomial system solving by means of triangular sets.

Several domain constructors implement
regular triangular sets (or regular chains).
Among them \spadtype{RegularTriangularSet}
and \spadtype{SquareFreeRegularTriangularSet}.
They also implement an algorithm by Marc Moreno Maza (marc@nag.co.uk, NAG)
for computing triangular decompositions of polynomial systems.
This method is refined in the package \spadtype{LazardSetSolvingPackage}
in order to produce decompositions by means of Lazard triangular sets.
For the case of polynomial systems with finitely many solutions,
these decompositions can also be computed by
the package \spadtype{LexTriangularPackage}.

The domain constructor \spadtype{RealClosure} by Renaud Rioboo
(Renaud.Rioboo@lip6.fr, University of Paris 6, France)
provides the real closure of an ordered field.
The implementation is based on interval arithmetic.
Moreover, the design of this constructor and its related
packages allows an easy use of other codings for real algebraic numbers.

Based on triangular decompositions and the \spadtype{RealClosure} constructor,
the package \spadtype{ZeroDimensionalSolvePackage}
provides operations for computing symbolically the real or complex roots
of polynomial systems with finitely many solutions.

Polynomial arithmetic with non-commutative variables
has been improved too by a contribution of Michel Petitot
(Michel.Petitot@lifl.fr, University of Lille I, France).
The domain constructors \spadtype{XRecursivePolynomial}
and \spadtype{XDistributedPolynomial} provide
recursive and distributed representations for these polynomials.
They are the non-commutative equivalents for
the \spadtype{SparseMultivariatePolynomial}
and \spadtype{DistributedMultivariatePolynomial} constructors.
The constructor \spadtype{LiePolynomial} implement Lie
polynomials in the Lyndon basis.
The constructor \spadtype{XPBWPolynomial} manage polynomials
with non-commutative variables in
the Poincar\'e-Birkhoff-Witt basis from the Lyndon basis.
This allows to compute in the Lie Group associated with a
free nilpotent Lie algebra by using the \spadtype{LieExponentials}
domain constructor.
%
%------------------------------------------------------------------------
\head{subsection}{Enhancements to HyperDoc and Graphics}{ugTwoTwoHyperdoc}
%------------------------------------------------------------------------
From this version of AXIOM onwards, the pixmap format used to save graphics
images in color and to display them in HyperDoc has been changed to the
industry-standard XPM format. See {\tt ftp://koala.inria.fr/pub/xpm}.
%
%------------------------------------------------------------------------
\head{subsection}{Enhancements to NAGLink}{ugTwoTwoNAGLink}
%------------------------------------------------------------------------
Content removed -  NAGLink is no longer included in FriCAS.
%
%------------------------------------------------------------------------
\head{subsection}{Enhancements to the Lisp system}{ugTwoTwoCCL}
%------------------------------------------------------------------------
Content removed - no longer relevant since FriCAS runs on different
Lisp systems.
