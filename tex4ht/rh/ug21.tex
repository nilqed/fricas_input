% !! DO NOT MODIFY THIS FILE BY HAND !! Created by spool2tex.awk.

% Copyright (c) 1991-2002, The Numerical ALgorithms Group Ltd.
% All rights reserved.
%
% Redistribution and use in source and binary forms, with or without
% modification, are permitted provided that the following conditions are
% met:
%
%     - Redistributions of source code must retain the above copyright
%       notice, this list of conditions and the following disclaimer.
%
%     - Redistributions in binary form must reproduce the above copyright
%       notice, this list of conditions and the following disclaimer in
%       the documentation and/or other materials provided with the
%       distribution.
%
%     - Neither the name of The Numerical ALgorithms Group Ltd. nor the
%       names of its contributors may be used to endorse or promote products
%       derived from this software without specific prior written permission.
%
% THIS SOFTWARE IS PROVIDED BY THE COPYRIGHT HOLDERS AND CONTRIBUTORS "AS
% IS" AND ANY EXPRESS OR IMPLIED WARRANTIES, INCLUDING, BUT NOT LIMITED
% TO, THE IMPLIED WARRANTIES OF MERCHANTABILITY AND FITNESS FOR A
% PARTICULAR PURPOSE ARE DISCLAIMED. IN NO EVENT SHALL THE COPYRIGHT OWNER
% OR CONTRIBUTORS BE LIABLE FOR ANY DIRECT, INDIRECT, INCIDENTAL, SPECIAL,
% EXEMPLARY, OR CONSEQUENTIAL DAMAGES (INCLUDING, BUT NOT LIMITED TO,
% PROCUREMENT OF SUBSTITUTE GOODS OR SERVICES-- LOSS OF USE, DATA, OR
% PROFITS-- OR BUSINESS INTERRUPTION) HOWEVER CAUSED AND ON ANY THEORY OF
% LIABILITY, WHETHER IN CONTRACT, STRICT LIABILITY, OR TORT (INCLUDING
% NEGLIGENCE OR OTHERWISE) ARISING IN ANY WAY OUT OF THE USE OF THIS
% SOFTWARE, EVEN IF ADVISED OF THE POSSIBILITY OF SUCH DAMAGE.

% *********************************************************************
\head{chapter}{Programs for FriCAS Images}{ugAppGraphics}
% *********************************************************************
%
This appendix contains the \Language{} programs used to generate
the images in the \Gallery{} color insert of this book.
All these input files are included
with the \Language{} system.
To produce the images
on page 6 of the \Gallery{} insert, for example, issue the command:
\begin{verbatim}
)read images6
\end{verbatim}

These images were produced on an IBM RS/6000 model 530 with a
standard color graphics adapter.  The smooth shaded images
were made from X Window System screen dumps.
The remaining images were produced with \Language{}-generated
PostScript output.  The images were reproduced from slides made on an Agfa
ChromaScript PostScript interpreter with a Matrix Instruments QCR camera.

% *********************************************************************
\head{section}{images1.input}{ugFimagesOne}
% *********************************************************************

\begin{xmpLinesPlain}
)read tknot                                              -- Read torus knot program.

torusKnot(15,17, 0.1, 6, 700)                            -- {\bf A (15,17) torus knot.}
\end{xmpLinesPlain}
\index{torus knot}

\newpage
% *********************************************************************
\head{section}{images2.input}{ugFimagesTwo}
% *********************************************************************

These images illustrate how Newton's method converges when computing the
\index{Newton iteration}
complex cube roots of 2.   Each point in the \smath{(x,y)}-plane represents the
complex number \smath{x + iy,} which is given as a starting point for Newton's
method.  The poles in these images represent bad starting values.
The flat areas are the regions of convergence to the three roots.

\begin{xmpLinesPlain}
)read newton                                             -- Read the programs from
)read vectors                                            -- \quad{}Chapter 10.
f := newtonStep(x^3 - 2)                                 -- Create a Newton's iteration
                                                         -- \quad{}function for $x^3 = 2$.
\end{xmpLinesPlain}

The function $f^n$ computes $n$ steps of Newton's method.

\begin{xmpLinesNoResetPlain}
clipValue := 4                                           -- Clip values with magnitude $> 4$.
drawComplexVectorField(f^3, -3..3, -3..3)                -- The vector field for $f^3$
drawComplex(f^3, -3..3, -3..3)                           -- The surface for $f^3$
drawComplex(f^4, -3..3, -3..3)                           -- The surface for $f^4$
\end{xmpLinesNoResetPlain}

% *********************************************************************
\head{section}{images3.input}{ugFimagesThree}
% *********************************************************************

\begin{xmpLinesPlain}
)r tknot
for i in 0..4 repeat torusKnot(2, 2 + i/4, 0.5, 25, 250)
\end{xmpLinesPlain}

% *********************************************************************
\head{section}{images5.input}{ugFimagesFive}
% *********************************************************************

The parameterization of the Etruscan Venus is due to George Frances.
\index{Etruscan Venus}

\begin{xmpLinesPlain}
venus(a,r,steps) ==
  surf := (u:DFLOAT, v:DFLOAT): Point DFLOAT +->
    cv := cos(v)
    sv := sin(v)
    cu := cos(u)
    su := sin(u)
    x := r * cos(2*u) * cv + sv * cu
    y := r * sin(2*u) * cv - sv * su
    z := a * cv
    point [x,y,z]
  draw(surf, 0..%pi, -%pi..%pi, var1Steps==steps,
       var2Steps==steps, title == "Etruscan Venus")

venus(5/2, 13/10, 50)                                    -- \textbf{The Etruscan Venus}
\end{xmpLinesPlain}

The Figure-8 Klein Bottle
\index{Klein bottle}
parameterization is from
``Differential Geometry and Computer Graphics'' by Thomas Banchoff,
in {\it Perspectives in Mathematics,} Anniversary of Oberwolfasch 1984,
Birkh\"{a}user-Verlag, Basel, pp. 43-60.

\begin{xmpLinesNoResetPlain}
klein(x,y) ==
  cx := cos(x)
  cy := cos(y)
  sx := sin(x)
  sy := sin(y)
  sx2 := sin(x/2)
  cx2 := cos(x/2)
  sq2 := sqrt(2.0@DFLOAT)
  point [cx * (cx2 * (sq2 + cy) + (sx2 * sy * cy)), _
         sx * (cx2 * (sq2 + cy) + (sx2 * sy * cy)), _
         -sx2 * (sq2 + cy) + cx2 * sy * cy]

draw(klein, 0..4*%pi, 0..2*%pi, var1Steps==50,           -- \textbf{Figure-8 Klein bottle}
     var2Steps==50,title=="Figure Eight Klein Bottle")
\end{xmpLinesNoResetPlain}

The next two images are examples of generalized tubes.

\begin{xmpLinesNoResetPlain}
)read ntube
rotateBy(p, theta) ==                                    -- Rotate a point $p$ by
  c := cos(theta)                                        -- \quad{}$\theta$ around the origin.
  s := sin(theta)
  point [p.1*c - p.2*s, p.1*s + p.2*c]

bcircle t ==                                             -- A circle in three-space.
  point [3*cos t, 3*sin t, 0]

twist(u, t) ==                                           -- An ellipse that twists
  theta := 4*t                                           -- \quad{}around four times as
  p := point [sin u, cos(u)/2]                           -- \quad{}\spad{t} revolves once.
  rotateBy(p, theta)

ntubeDrawOpt(bcircle, twist, 0..2*%pi, 0..2*%pi,         -- \bf{Twisted Torus}
             var1Steps == 70, var2Steps == 250)

twist2(u, t) ==                                          -- Create a twisting circle.
  theta := t
  p := point [sin u, cos(u)]
  rotateBy(p, theta)

cf(u,v) == sin(21*u)                                     -- Color function with \spad{21} stripes.

ntubeDrawOpt(bcircle, twist2, 0..2*%pi, 0..2*%pi,        -- \textbf{Striped Torus}
  colorFunction == cf, var1Steps == 168,
  var2Steps == 126)
\end{xmpLinesNoResetPlain}

% *********************************************************************
\head{section}{images6.input}{ugFimagesSix}
% *********************************************************************

\begin{xmpLinesPlain}
gam(x,y) ==                                              -- The height and color are the
  g := Gamma complex(x,y)                                -- \quad{}real and argument parts
  point [x,y,max(min(real g, 4), -4), argument g]        -- \quad{}of the Gamma function,
                                                         -- \quad{}respectively.

draw(gam, -%pi..%pi, -%pi..%pi,                          -- \textbf{The Gamma Function}
     title == "Gamma(x + %i*y)", _
     var1Steps == 100, var2Steps == 100)

b(x,y) == Beta(x,y)

draw(b, -3.1..3, -3.1 .. 3, title == "Beta(x,y)")        -- \textbf{The Beta Function}

atf(x,y) ==
  a := atan complex(x,y)
  point [x,y,real a, argument a]

draw(atf, -3.0..%pi, -3.0..%pi)                          -- \textbf{The Arctangent function}
\end{xmpLinesPlain}
\index{function!Gamma}
\index{function!Euler Beta}
\index{Euler!Beta function}


% *********************************************************************
\head{section}{images7.input}{ugFimagesSeven}
% *********************************************************************

First we look at the conformal
\index{conformal map}
map $z \mapsto z + 1/z$.

\begin{xmpLinesPlain}
)read conformal                                          -- Read program for drawing
                                                         -- \quad{}conformal maps.

f z == z                                                 -- The coordinate grid for the
                                                         -- \quad{}complex plane.
conformalDraw(f, -2..2, -2..2, 9, 9, "cartesian")        -- \textbf{Mapping 1: Source}

f z == z + 1/z                                           -- The map $z \mapsto z + 1/z$

conformalDraw(f, -2..2, -2..2, 9, 9, "cartesian")        -- \bf{Mapping 1: Target}
\end{xmpLinesPlain}

The map $z \mapsto -(z+1)/(z-1)$ maps
the unit disk to the right half-plane, as shown
\index{Riemann!sphere}
on the Riemann sphere.

\begin{xmpLinesNoResetPlain}
f z == z                                                 -- The unit disk.

riemannConformalDraw(f,0.1..0.99,0..2*%pi,7,11,"polar")  -- \textbf{Mapping 2: Source}

f z == -(z+1)/(z-1)                                      -- The map $x\mapsto -(z+1)/(z-1)$.
riemannConformalDraw(f,0.1..0.99,0..2*%pi,7,11,"polar")  -- \textbf{Mapping 2: Target}

riemannSphereDraw(-4..4, -4..4, 7, 7, "cartesian")       -- \textbf{Riemann Sphere Mapping}
\end{xmpLinesNoResetPlain}

% *********************************************************************
\head{section}{images8.input}{ugFimagesEight}
% *********************************************************************

\begin{xmpLinesPlain}
)read dhtri
)read tetra
drawPyramid 4                                            -- \textbf{Sierpinsky's Tetrahedron}

\index{Sierpinsky's Tetrahedron}
)read antoine
drawRings 2                                              -- \textbf{Antoine's Necklace}

\index{Antoine's Necklace}
)read scherk
drawScherk(3,3)                                          -- \textbf{Scherk's Minimal Surface}

\index{Scherk's minimal surface}
)read ribbonsNew
drawRibbons([x^i for i in 1..5], x=-1..1, y=0..2)        -- \bf{Ribbon Plot}
\end{xmpLinesPlain}


%\input{gallery/conformal.htex}
% *********************************************************************
\head{section}{conformal.input}{ugFconformal}
% *********************************************************************
%
The functions in this section draw conformal maps both on the
\index{conformal map}
plane and on the Riemann sphere.
\index{Riemann!sphere}

%   -- Compile, don't interpret functions.
%\xmpLine{)set fun comp on}{}
\begin{xmpLinesPlain}
C := Complex DoubleFloat                                 -- Complex Numbers
S := Segment DoubleFloat                                 -- Draw ranges
R3 := Point DFLOAT                                       -- Points in 3-space

\end{xmpLinesPlain}

\userfun{conformalDraw}{\it (f, rRange, tRange, rSteps, tSteps, coord)}
draws the image of the coordinate grid under {\it f} in the complex plane.
The grid may be given in either polar or Cartesian coordinates.
Argument {\it f} is the function to draw;
{\it rRange} is the range of the radius (in polar) or real (in Cartesian);
{\it tRange} is the range of $\theta$ (in polar) or imaginary (in Cartesian);
{\it tSteps, rSteps}, are the number of intervals in the {\it r} and
$\theta$ directions; and
{\it coord} is the coordinate system to use (either {\tt "polar"} or
{\tt "cartesian"}).

\begin{xmpLinesNoResetPlain}
conformalDraw: (C -> C, S, S, PI, PI, String) -> VIEW3D
conformalDraw(f,rRange,tRange,rSteps,tSteps,coord) ==
  transformC :=                                          -- Function for changing an \smath{(x,y)}
    coord = "polar" => polar2Complex                     -- \quad{}pair into a complex number.
    cartesian2Complex
  cm := makeConformalMap(f, transformC)
  sp := createThreeSpace()                               -- Create a fresh space.
  adaptGrid(sp, cm, rRange, tRange, rSteps, tSteps)      -- Plot the coordinate lines.
  makeViewport3D(sp, "Conformal Map")                    -- Draw the image.
\end{xmpLinesNoResetPlain}

\userfun{riemannConformalDraw}{\it (f, rRange, tRange, rSteps, tSteps, coord)}
draws the image of the coordinate grid under {\it f} on the Riemann sphere.
The grid may be given in either polar or Cartesian coordinates.
Its arguments are the same as those for \userfun{conformalDraw}.
\begin{xmpLinesNoResetPlain}
riemannConformalDraw:(C->C,S,S,PI,PI,String)->VIEW3D
riemannConformalDraw(f, rRange, tRange,
                     rSteps, tSteps, coord) ==
  transformC :=                                          -- Function for changing an \smath{(x,y)}
    coord = "polar" => polar2Complex                     -- \quad{}pair into a complex number.
    cartesian2Complex
  sp := createThreeSpace()                               -- Create a fresh space.
  cm := makeRiemannConformalMap(f, transformC)
  adaptGrid(sp, cm, rRange, tRange, rSteps, tSteps)      -- Plot the coordinate lines.
  curve(sp,[point [0,0,2.0@DFLOAT,0],point [0,0,2.0@DFLOAT,0]]) -- Add an invisible point at
  makeViewport3D(sp,"Map on the Riemann Sphere")         -- \quad{}the north pole for scaling.

adaptGrid(sp, f, uRange, vRange,  uSteps, vSteps) ==     -- Plot the coordinate grid
  delU := (high(uRange) - low(uRange))/uSteps            -- \quad{}using adaptive plotting for
  delV := (high(vRange) - low(vRange))/vSteps            -- \quad{}coordinate lines, and draw
  uSteps := uSteps + 1; vSteps := vSteps + 1             -- \quad{}tubes around the lines.
  u := low uRange
  for i in 1..uSteps repeat                              -- Draw coordinate lines in the \spad{v}
    c := curryLeft(f,u)                                  -- \quad{}direction; curve \spad{c} fixes the
    cf := (t:DFLOAT):DFLOAT +-> 0                        -- \quad{}current value of \spad{u}.
    makeObject(c,vRange::SEG Float,colorFunction==cf,    -- Draw the \spad{v} coordinate line.
      space == sp, tubeRadius == .02, tubePoints == 6)
    u := u + delU
  v := low vRange
  for i in 1..vSteps repeat                              -- Draw coordinate lines in the \spad{u}
    c := curryRight(f,v)                                 -- \quad{}direction; curve \spad{c} fixes the
    cf := (t:DFLOAT):DFLOAT +-> 1                        -- \quad{}current value of \spad{v}.
    makeObject(c,uRange::SEG Float,colorFunction==cf,    -- Draw the \spad{u} coordinate line.
      space == sp, tubeRadius == .02, tubePoints == 6)
    v := v + delV
  void()

riemannTransform(z) ==                                   -- Map a point in the complex
  r := sqrt norm z                                       -- \quad{}plane to the Riemann sphere.
  cosTheta := (real z)/r
  sinTheta := (imag z)/r
  cp := 4*r/(4+r^2)
  sp := sqrt(1-cp*cp)
  if r>2 then sp := -sp
  point [cosTheta*cp, sinTheta*cp, -sp + 1]

cartesian2Complex(r:DFLOAT, i:DFLOAT):C ==               -- Convert Cartesian coordinates to
  complex(r, i)                                          -- \quad{}complex Cartesian form.

polar2Complex(r:DFLOAT, th:DFLOAT):C ==                  -- Convert polar coordinates to
  complex(r*cos(th), r*sin(th))                          -- \quad{}complex Cartesian form.

makeConformalMap(f, transformC) ==                       -- Convert complex function \spad{f} to a
  (u:DFLOAT,v:DFLOAT):R3 +->                             -- \quad{}mapping: (\spadtype{DFLOAT},\spadtype{DFLOAT}) $\to$ \pspadtype{R3}
    z := f transformC(u, v)                              -- \quad{}in the complex plane.
    point [real z, imag z, 0.0@DFLOAT]

makeRiemannConformalMap(f, transformC) ==                -- Convert a complex function \spad{f} to a
  (u:DFLOAT, v:DFLOAT):R3 +->                            -- \quad{}mapping: (\spadtype{DFLOAT},\spadtype{DFLOAT}) $\to$ \spadtype{R3}
    riemannTransform f transformC(u, v)                  -- \quad{}on the Riemann sphere.

riemannSphereDraw: (S, S, PI, PI, String) -> VIEW3D      -- Draw a picture of the mapping
riemannSphereDraw(rRange,tRange,rSteps,tSteps,coord) ==  -- \quad{}of the complex plane to
  transformC :=                                          -- \quad{}the Riemann sphere.
    coord = "polar" => polar2Complex
    cartesian2Complex
  grid := (u:DFLOAT, v:DFLOAT): R3 +->                   -- Coordinate grid function.
    z1 := transformC(u, v)
    point [real z1, imag z1, 0]
  sp := createThreeSpace()                               -- Create a fresh space.
  adaptGrid(sp, grid, rRange, tRange, rSteps, tSteps)    -- Draw the flat grid.
  connectingLines(sp,grid,rRange,tRange,rSteps,tSteps)
  makeObject(riemannSphere,0..2*%pi,0..%pi,space==sp)    -- Draw the sphere.
  f := (z:C):C +-> z
  cm := makeRiemannConformalMap(f, transformC)
  adaptGrid(sp, cm, rRange, tRange, rSteps, tSteps)      -- Draw the sphere grid.
  makeViewport3D(sp, "Riemann Sphere")

connectingLines(sp,f,uRange,vRange,uSteps,vSteps) ==     -- Draw the lines that connect
  delU := (high(uRange) - low(uRange))/uSteps            -- \quad{}the points in the complex
  delV := (high(vRange) - low(vRange))/vSteps            -- \quad{}plane to the north pole
  uSteps := uSteps + 1; vSteps := vSteps + 1             -- \quad{}of the Riemann sphere.
  u := low uRange
  for i in 1..uSteps repeat                              -- For each u.
    v := low vRange
    for j in 1..vSteps repeat                            -- For each v.
      p1 := f(u,v)
      p2 := riemannTransform complex(p1.1, p1.2)         -- Project p1 onto the sphere.
      fun := lineFromTo(p1,p2)                           -- Create a line function.
      cf := (t:DFLOAT):DFLOAT +-> 3
      makeObject(fun, 0..1,space==sp,tubePoints==4,      -- Draw the connecting line.
                 tubeRadius==0.01,colorFunction==cf)
      v := v + delV
    u := u + delU
  void()

riemannSphere(u,v) ==                                    -- A sphere sitting on the
  sv := sin(v)                                           -- \quad{}complex plane, with radius 1.
  0.99@DFLOAT*(point [cos(u)*sv,sin(u)*sv,cos(v),0.0@DFLOAT])+
    point [0.0@DFLOAT, 0.0@DFLOAT, 1.0@DFLOAT, 4.0@DFLOAT]

lineFromTo(p1, p2) ==                                    -- Create a line function
  d := p2 - p1                                           -- \quad{}that goes from p1 to p2
  (t:DFLOAT):Point DFLOAT +->
    p1 + t*d
\end{xmpLinesNoResetPlain}

%\input{gallery/tknot.htex}
% *********************************************************************
\head{section}{tknot.input}{ugFtknot}
% *********************************************************************
%
Create a $(p,q)$ torus-knot with radius $r$ around the curve.
The formula was derived by Larry Lambe.

\begin{xmpLinesPlain}
)read ntube
torusKnot: (DFLOAT, DFLOAT, DFLOAT, PI, PI) -> VIEW3D
torusKnot(p, q ,r, uSteps, tSteps) ==
  knot := (t:DFLOAT):Point DFLOAT +->                    -- Function for the torus knot.
    fac := 4/(2.2@DFLOAT-sin(q*t))
    fac * point [cos(p*t), sin(p*t), cos(q*t)]
  circle := (u:DFLOAT, t:DFLOAT): Point DFLOAT +->       -- The cross section.
    r * point [cos u, sin u]
  ntubeDrawOpt(knot, circle, 0..2*%pi, 0..2*%pi,         -- Draw the circle around the knot.
               var1Steps == uSteps, var2Steps == tSteps)

\end{xmpLinesPlain}

%\input{gallery/ntube.htex}
% *********************************************************************
\head{section}{ntube.input}{ugFntube}
% *********************************************************************
%
The functions in this file create generalized tubes (also known as generalized
cylinders).
These functions draw a 2-d curve in the normal
planes around a 3-d curve.

\begin{xmpLinesPlain}
R3 := Point DFLOAT                                       -- Points in 3-Space
R2 := Point DFLOAT                                       -- Points in 2-Space
S := Segment Float                                       -- Draw ranges
                                                         -- Introduce types for functions for:
ThreeCurve := DFLOAT -> R3                               -- \quad{}---the space curve function
TwoCurve := (DFLOAT, DFLOAT) -> R2                       -- \quad{}---the plane curve function
Surface := (DFLOAT, DFLOAT) -> R3                        -- \quad{}---the surface function
                                                         -- Frenet frames define a
FrenetFrame :=                                           -- \quad{}coordinate system around a
   Record(value:R3,tangent:R3,normal:R3,binormal:R3)     -- \quad{}point on a space curve.
frame: FrenetFrame                                       -- The current Frenet frame
                                                         -- \quad{}for a point on a curve.
\end{xmpLinesPlain}

\userfun{ntubeDraw}{\it (spaceCurve, planeCurve,}
$u_0 .. u_1,$ $t_0 .. t_1)$
draws {\it planeCurve} in the normal planes of {\it spaceCurve.}
The parameter $u_0 .. u_1$ specifies
the parameter range for {\it planeCurve}
and $t_0 .. t_1$ specifies the parameter range for {\it spaceCurve}.
Additionally, the plane curve function takes
a second parameter: the current parameter of {\it spaceCurve}.
This allows the plane curve to change shape
as it goes around the space curve.
See \spadref{ugFimagesFive} for an example of this.
%
\begin{xmpLinesNoResetPlain}
ntubeDraw: (ThreeCurve,TwoCurve,S,S) -> VIEW3D
ntubeDraw(spaceCurve,planeCurve,uRange,tRange) ==
  ntubeDrawOpt(spaceCurve, planeCurve, uRange, _
               tRange, []$List DROPT)

ntubeDrawOpt: (ThreeCurve,TwoCurve,S,S,List DROPT)
    -> VIEW3D
ntubeDrawOpt(spaceCurve,planeCurve,uRange,tRange,l) ==   -- This function is similar
                                                         -- \quad{}to \userfun{ntubeDraw}, but takes
  delT:DFLOAT := (high(tRange) - low(tRange))/10000      -- \quad{}optional parameters that it
  oldT:DFLOAT := low(tRange) - 1                         -- \quad{}passes to the \spadfun{draw} command.
  fun := ngeneralTube(spaceCurve,planeCurve,delT,oldT)
  draw(fun, uRange, tRange, l)

\end{xmpLinesNoResetPlain}

\userfun{nfrenetFrame}{\it (c, t, delT)}
numerically computes the Frenet frame
about the curve {\it c} at {\it t}.
Parameter {\it delT} is a small number used to
compute derivatives.
\begin{xmpLinesNoResetPlain}
nfrenetFrame(c, t, delT) ==
  f0 := c(t)
  f1 := c(t+delT)
  t0 := f1 - f0                                          -- The tangent.
  n0 := f1 + f0
  b := cross(t0, n0)                                     -- The binormal.
  n := cross(b,t0)                                       -- The normal.
  ln := length n
  lb := length b
  ln = 0 or lb = 0 =>
      error "Frenet Frame not well defined"
  n := (1/ln)*n                                          -- Make into unit length vectors.
  b := (1/lb)*b
  [f0, t0, n, b]$FrenetFrame
\end{xmpLinesNoResetPlain}

\userfun{ngeneralTube}{\it (spaceCurve, planeCurve,}{\it  delT, oltT)}
creates a function that can be passed to the system \spadfun{draw} command.
The function is a parameterized surface for the general tube
around {\it spaceCurve}.  {\it delT} is a small number used to compute
derivatives. {\it oldT} is used to hold the current value of the
{\it t} parameter for {\it spaceCurve.}  This is an efficiency measure
to ensure that frames are only computed once for each value of {\it t}.
\begin{xmpLinesNoResetPlain}
ngeneralTube: (ThreeCurve, TwoCurve, DFLOAT, DFLOAT) -> Surface
ngeneralTube(spaceCurve, planeCurve, delT, oldT) ==
  free frame                                             -- Indicate that \spad{frame} is global.
  (v:DFLOAT, t: DFLOAT): R3 +->
    if (t ~= oldT) then                                  -- If not already computed,
      frame := nfrenetFrame(spaceCurve, t, delT)         -- \quad{}compute new frame.
      oldT := t
    p := planeCurve(v, t)
    frame.value + p.1*frame.normal + p.2*frame.binormal  -- Project \spad{p} into the normal plane.
\end{xmpLinesNoResetPlain}

%\input{gallery/dhtri.htex}
% *********************************************************************
\head{section}{dhtri.input}{ugFdhtri}
% *********************************************************************
%
Create affine transformations (DH matrices) that transform
a given triangle into another.

\begin{xmpLinesPlain}
tri2tri: (List Point DFLOAT, List Point DFLOAT) -> DHMATRIX(DFLOAT)
                                                         -- Compute a \spadtype{DHMATRIX} that
tri2tri(t1, t2) ==                                       -- \quad{}transforms \spad{t1} to \spad{t2,} where
  n1 := triangleNormal(t1)                               -- \quad{}\spad{t1} and \spad{t2} are the vertices
  n2 := triangleNormal(t2)                               -- \quad{}of two triangles in 3-space.
  tet2tet(concat(t1, n1), concat(t2, n2))

tet2tet: (List Point DFLOAT, List Point DFLOAT) -> DHMATRIX(DFLOAT)
                                                         -- Compute a \spadtype{DHMATRIX} that
tet2tet(t1, t2) ==                                       -- \quad{}transforms \spad{t1} to \spad{t2,}
  m1 := makeColumnMatrix t1                              -- \quad{}where \spad{t1} and \spad{t2} are the
  m2 := makeColumnMatrix t2                              -- \quad{}vertices of two tetrahedrons
  m2 * inverse(m1)                                       -- \quad{}in 3-space.

makeColumnMatrix(t) ==                                   -- Put the vertices of a tetra-
  m := new(4,4,0)$DHMATRIX(DFLOAT)                       -- \quad{}hedron into matrix form.
  for x in t for i in 1..repeat
    for j in 1..3 repeat
      m(j,i) := x.j
    m(4,i) := 1
  m

triangleNormal(t) ==                                     -- Compute a vector normal to
  a := triangleArea t                                    -- \quad{}the given triangle, whose
  p1 := t.2 - t.1                                        -- \quad{}length is the square root
  p2 := t.3 - t.2                                        -- \quad{}of the area of the triangle.
  c := cross(p1, p2)
  len := length(c)
  len = 0 => error "degenerate triangle!"
  c := (1/len)*c
  t.1 + sqrt(a) * c

triangleArea t ==                                        -- Compute the area of a
  a := length(t.2 - t.1)                                 -- \quad{}triangle using Heron's
  b := length(t.3 - t.2)                                 -- \quad{}formula.
  c := length(t.1 - t.3)
  s := (a+b+c)/2
  sqrt(s*(s-a)*(s-b)*(s-c))
\end{xmpLinesPlain}


% *********************************************************************
\head{section}{tetra.input}{ugFtetra}
% *********************************************************************
%
%\input{gallery/tetra.htex}
%\outdent{Sierpinsky's Tetrahedron}

\begin{xmpLinesPlain}
)set expose add con DenavitHartenbergMatrix              -- Bring DH matrices into the
                                                         -- \quad{}environment.
x1:DFLOAT := sqrt(2.0@DFLOAT/3.0@DFLOAT)                 -- Set up the coordinates of the
x2:DFLOAT := sqrt(3.0@DFLOAT)/6                          -- \quad{}corners of the tetrahedron.

z := 0.0@DFLOAT
h := 0.5@DFLOAT

p1 := point [-h, -x2, z]                                 -- Some needed points.
p2 := point [h, -x2, z]
p3 := point [z, 2*x2, z]
p4 := point [z, z, x1]

baseTriangle  := [p2, p1, p3]                            -- The base of the tetrahedron.

mt := [h*(p2+p1), h*(p1+p3), h*(p3+p2)]                  -- The ``middle triangle'' inscribed
                                                         -- \quad{}in the base of the tetrahedron.
bt1 := [mt.1, p1, mt.2]                                  -- The bases of the triangles of
bt2 := [p2, mt.1, mt.3]                                  -- \quad{}the subdivided tetrahedron.
bt3 := [mt.2, p3, mt.3]
bt4 := [h*(p2+p4), h*(p1+p4), h*(p3+p4)]

tt1 := tri2tri(baseTriangle, bt1)                        -- Create the transformations
tt2 := tri2tri(baseTriangle, bt2)                        -- \quad{}that bring the base of the
tt3 := tri2tri(baseTriangle, bt3)                        -- \quad{}tetrahedron to the bases of
tt4 := tri2tri(baseTriangle, bt4)                        -- \quad{}the subdivided tetrahedron.

drawPyramid(n) ==                                        -- Draw a Sierpinsky tetrahedron
  s := createThreeSpace()                                -- \quad{}with \spad{n} levels of recursive
  dh := rotatex(0.0@DFLOAT)                              -- \quad{}subdivision.
  drawPyramidInner(s, n, dh)
  makeViewport3D(s, "Sierpinsky Tetrahedron")

drawPyramidInner(s, n, dh) ==                            -- Recursively draw a Sierpinsky
  n = 0 => makeTetrahedron(s, dh, n)                     -- \quad{}tetrahedron.
  drawPyramidInner(s, n-1, dh * tt1)                     -- Draw the 4 recursive pyramids.
  drawPyramidInner(s, n-1, dh * tt2)
  drawPyramidInner(s, n-1, dh * tt3)
  drawPyramidInner(s, n-1, dh * tt4)

makeTetrahedron(sp, dh, color) ==                        -- Draw a tetrahedron into the
  w1 := dh*p1                                            -- \quad{}given space with the given
  w2 := dh*p2                                            -- \quad{}color, transforming it by
  w3 := dh*p3                                            -- \quad{}the given DH matrix.
  w4 := dh*p4
  polygon(sp, [w1, w2, w4])
  polygon(sp, [w1, w3, w4])
  polygon(sp, [w2, w3, w4])
  void()
\end{xmpLinesPlain}
\index{Sierpinsky's Tetrahedron}


%\input{gallery/antoine.htex}
% *********************************************************************
\head{section}{antoine.input}{ugFantoine}
% *********************************************************************
%
Draw Antoine's Necklace.
\index{Antoine's Necklace}
Thank you to Matthew Grayson at IBM's T.J Watson Research Center for the idea.

\begin{xmpLinesPlain}
)set expose add con DenavitHartenbergMatrix              -- Bring DH matrices into
                                                         -- \quad{}the environment.
torusRot: DHMATRIX(DFLOAT)                               -- The current transformation for
                                                         -- \quad{}drawing a sub ring.

drawRings(n) ==                                          -- Draw Antoine's Necklace with \spad{n}
  s := createThreeSpace()                                -- \quad{}levels of recursive subdivision.
  dh:DHMATRIX(DFLOAT) := identity()                      -- \quad{}The number of subrings is $10^n$.
  drawRingsInner(s, n, dh)                               -- Do the real work.
  makeViewport3D(s, "Antoine's Necklace")

\end{xmpLinesPlain}

In order to draw Antoine rings, we take one ring, scale it down to
a smaller size, rotate it around its central axis, translate it
to the edge of the larger ring and rotate it around the edge to
a point corresponding to its count (there are 10 positions around
the edge of the larger ring). For each of these new rings we
recursively perform the operations, each ring becoming 10 smaller
rings. Notice how the \spadtype{DHMATRIX} operations are used to build up
the proper matrix composing all these transformations.

\begin{xmpLinesNoResetPlain}
F ==> DFLOAT
drawRingsInner(s, n, dh) ==                              -- Recursively draw Antoine's
  n = 0 =>                                               -- \quad{}Necklace.
    drawRing(s, dh)
    void()
  t := 0.0@F                                             -- Angle around ring.
  p := 0.0@F                                             -- Angle of subring from plane.
  tr := 1.0@F                                            -- Amount to translate subring.
  inc := 0.1@F                                           -- The translation increment.
  for i in 1..10 repeat                                  -- Subdivide into 10 linked rings.
    tr := tr + inc
    inc := -inc
    dh' := dh*rotatez(t)*translate(tr,0.0@F,0.0@F)*      -- Transform ring in center
           rotatey(p)*scale(0.35@F, 0.48@F, 0.4@F)       -- \quad{}to a link.
    drawRingsInner(s, n-1, dh')
    t := t + 36.0@F
    p := p + 90.0@F
  void()

drawRing(s, dh) ==                                       -- Draw a single ring into
  free torusRot                                          -- \quad{}the given subspace,
  torusRot := dh                                         -- \quad{}transformed by the given
  makeObject(torus, 0..2*%pi, 0..2*%pi, var1Steps == 6,  -- \quad{}\spadtype{DHMATRIX}.
             space == s, var2Steps == 15)

torus(u ,v) ==                                           -- Parameterization of a torus,
  cu := cos(u)/6                                         -- \quad{}transformed by the
  torusRot*point [(1+cu)*cos(v),(1+cu)*sin(v),(sin u)/6] -- \quad{}\spadtype{DHMATRIX} in \spad{torusRot.}
\end{xmpLinesNoResetPlain}

%\input{gallery/scherk.htex}
% *********************************************************************
\head{section}{scherk.input}{ugFscherk}
% *********************************************************************
%

Scherk's minimal surface, defined by:
\index{Scherk's minimal surface}
$e^z \cos(x) = \cos(y)$.
See: {\it A Comprehensive Introduction to Differential Geometry,} Vol. 3,
by Michael Spivak, Publish Or Perish, Berkeley, 1979, pp. 249-252.

\begin{xmpLinesPlain}
(xOffset, yOffset):DFLOAT                                -- Offsets for a single piece
                                                         -- \quad{}of Scherk's minimal surface.

drawScherk(m,n) ==                                       -- Draw Scherk's minimal surface
  free xOffset, yOffset                                  -- \quad{}on an \spad{m} by \spad{n} patch.
  space := createThreeSpace()
  for i in 0..m-1 repeat
    xOffset := i*%pi
    for j in 0 .. n-1 repeat
      rem(i+j, 2) = 0 => 'iter                           -- Draw only odd patches.
      yOffset := j*%pi
      drawOneScherk(space)                               -- Draw a patch.
  makeViewport3D(space, "Scherk's Minimal Surface")

scherk1(u,v) ==                                          -- The first patch that makes
  x := cos(u)/exp(v)                                     -- \quad{}up a single piece of
  point [xOffset + acos(x), yOffset + u, v, abs(v)]      -- \quad{}Scherk's minimal surface.

scherk2(u,v) ==                                          -- The second patch.
  x := cos(u)/exp(v)
  point [xOffset - acos(x), yOffset + u, v, abs(v)]

scherk3(u,v) ==                                          -- The third patch.
  x := exp(v) * cos(u)
  point [xOffset + u, yOffset + acos(x), v, abs(v)]

scherk4(u,v) ==                                          -- The fourth patch.
  x := exp(v) * cos(u)
  point [xOffset + u, yOffset - acos(x), v, abs(v)]

drawOneScherk(s) ==                                      -- Draw the surface by
  makeObject(scherk1,-%pi/2..%pi/2,0..%pi/2,space==s,    -- \quad{}breaking it into four
             var1Steps == 28, var2Steps == 28)           -- \quad{}patches and then drawing
  makeObject(scherk2,-%pi/2..%pi/2,0..%pi/2,space==s,    -- \quad{}the patches.
             var1Steps == 28, var2Steps == 28)
  makeObject(scherk3,-%pi/2..%pi/2,-%pi/2..0,space==s,
             var1Steps == 28, var2Steps == 28)
  makeObject(scherk4,-%pi/2..%pi/2,-%pi/2..0,space==s,
             var1Steps == 28, var2Steps == 28)
  void()
\end{xmpLinesPlain}
\begin{SysCmdOutput}
\end{SysCmdOutput}
