% !! DO NOT MODIFY THIS FILE BY HAND !! Created by spool2tex.awk.

% Copyright (c) 1991-2002, The Numerical ALgorithms Group Ltd.
% All rights reserved.
%
% Redistribution and use in source and binary forms, with or without
% modification, are permitted provided that the following conditions are
% met:
%
%     - Redistributions of source code must retain the above copyright
%       notice, this list of conditions and the following disclaimer.
%
%     - Redistributions in binary form must reproduce the above copyright
%       notice, this list of conditions and the following disclaimer in
%       the documentation and/or other materials provided with the
%       distribution.
%
%     - Neither the name of The Numerical ALgorithms Group Ltd. nor the
%       names of its contributors may be used to endorse or promote products
%       derived from this software without specific prior written permission.
%
% THIS SOFTWARE IS PROVIDED BY THE COPYRIGHT HOLDERS AND CONTRIBUTORS "AS
% IS" AND ANY EXPRESS OR IMPLIED WARRANTIES, INCLUDING, BUT NOT LIMITED
% TO, THE IMPLIED WARRANTIES OF MERCHANTABILITY AND FITNESS FOR A
% PARTICULAR PURPOSE ARE DISCLAIMED. IN NO EVENT SHALL THE COPYRIGHT OWNER
% OR CONTRIBUTORS BE LIABLE FOR ANY DIRECT, INDIRECT, INCIDENTAL, SPECIAL,
% EXEMPLARY, OR CONSEQUENTIAL DAMAGES (INCLUDING, BUT NOT LIMITED TO,
% PROCUREMENT OF SUBSTITUTE GOODS OR SERVICES-- LOSS OF USE, DATA, OR
% PROFITS-- OR BUSINESS INTERRUPTION) HOWEVER CAUSED AND ON ANY THEORY OF
% LIABILITY, WHETHER IN CONTRACT, STRICT LIABILITY, OR TORT (INCLUDING
% NEGLIGENCE OR OTHERWISE) ARISING IN ANY WAY OUT OF THE USE OF THIS
% SOFTWARE, EVEN IF ADVISED OF THE POSSIBILITY OF SUCH DAMAGE.

% *********************************************************************
\head{chapter}{Domains}{ugDomains}
% *********************************************************************

We finally come to the \spadgloss{domain constructor}.
A few subtle differences between packages and
domains turn up some interesting issues.
We first discuss these differences then
describe the resulting issues by illustrating a program
for the \spadtype{QuadraticForm} constructor.
After a short example of an algebraic constructor,
\spadtype{CliffordAlgebra}, we show how you use domain constructors to build
a database query facility.

% *********************************************************************
\head{section}{Domains vs. Packages}{ugPackagesDoms}
% *********************************************************************
%
Packages are special cases of domains.
What is the difference between a package and a domain that is not a
package?  Internally, \Language{} makes no distinction.  However,
humans think differently about them, so we make the
following definition: a domain that is not a package
has the symbol \spadSyntax
denotes ``this domain.'' If the \spadSyntax, then
evidently there is nothing of interest to do with the objects of the
domain.
You might then say that a package is a ``boring'' domain!
But, as you saw in \chapref{ugPackages}, packages are a very useful
notion indeed.
The exported operations of a package depend solely on the parameters
to the package constructor and other explicit domains.

To summarize, domain constructors are versatile structures that serve two
distinct practical purposes:
Those like \spadtype{Polynomial} and \spadtype{List}
describe classes of computational objects;
others, like \pspadtype{SortPackage}, describe packages of useful
operations.
As in the last chapter, we focus here on the first kind.

% *********************************************************************
\head{section}{Definitions}{ugDomainsDefs}
% *********************************************************************
%

The syntax for defining a domain constructor is the same as for any
function in \Language{}:
\begin{center}
\frenchspacing{\tt {\it DomainForm} : {\it Exports} == {\it Implementation}}
\end{center}
As this definition usually extends over many lines, a
\spad{where} expression is generally used instead.
\spadkey{where}

% ----------------------------------------------------------------------
\beginImportant
A recommended format for the definition of a domain is:\newline
{\tt%
{\it DomainForm} : Exports  ==  Implementation where \newline
\hspace*{.75pc} {\it optional type declarations} \newline
\hspace*{.75pc} Exports  ==  [{\it Category Assertions}] with \newline
\hspace*{2.0pc}   {\it list of exported operations} \newline
\hspace*{.75pc} Implementation  ==  [{\it Add Domain}] add \newline
\hspace*{2.0pc}   [Rep := {\it Representation}] \newline
\hspace*{2.0pc}   {\it list of function definitions for exported operations}
}

Note: The brackets {\tt [ ]} here denote optionality.
\endImportant
% ----------------------------------------------------------------------

A complete domain constructor definition for
\spadtype{QuadraticForm} is shown in Figure \ref{fig-quadform}.
Interestingly, this little domain illustrates all the new
concepts you need to learn.

\begin{figXmpLines}[caption={The \textspadtype{QuadraticForm} domain.},label={fig-quadform}]
)abbrev domain QFORM QuadraticForm

++ Description:
++   This domain provides modest support for
++   quadratic forms.
QuadraticForm(n, K): Exports == Implementation where
    n: PositiveInteger
    K: Field

    Exports == AbelianGroup with                         -- The exports.
      quadraticForm: SquareMatrix(n,K) -> %              -- The export \spadfun{quadraticForm}.
        ++ \spad{quadraticForm(m)} creates a quadratic
        ++ quadratic form from a symmetric,
        ++ square matrix \spad{m}.
      matrix: % -> SquareMatrix(n,K)                     -- The export \spadfun{matrix}.
        ++ \spad{matrix(qf)} creates a square matrix
        ++ from the quadratic form \spad{qf}.
      elt: (%, DirectProduct(n,K)) -> K                  -- The export \spadfun{elt}.
        ++ \spad{qf(v)} evaluates the quadratic form
        ++ \spad{qf} on the vector \spad{v},
        ++ producing a scalar.

    Implementation == SquareMatrix(n,K) add              -- The definitions of the exports
      Rep := SquareMatrix(n,K)                           -- The ``representation.''
      quadraticForm m ==                                 -- The definition of
        not symmetric? m => error                        -- \spadfun{quadraticForm}.
          "quadraticForm requires a symmetric matrix"
        m :: %
      matrix q == q :: Rep                               -- The definition of \spadfun{matrix}.
      elt(q,v) == dot(v, (matrix q * v))                 -- The definition of \spadfun{elt}.
\end{figXmpLines}

A domain constructor can take any number and type of parameters.
\spadtype{QuadraticForm} takes a positive integer \spad{n} and a field
\spad{K} as arguments.
Like a package, a domain has a set of explicit exports and an
implementation described by a capsule.
Domain constructors are documented in the same way as package constructors.

Domain \spadtype{QuadraticForm(n, K)}, for a given positive integer
\spad{n} and domain \spad{K}, explicitly exports three operations:
%
\begin{itemize}
\item\spad{quadraticForm(A)} creates a quadratic form from a matrix
\spad{A}.
\item\spad{matrix(q)} returns the matrix \spad{A} used to create
the quadratic form \spad{q}.
\item\spad{q.v} computes the scalar $v^TAv$
for a given vector \spad{v}.
\end{itemize}

Compared with the corresponding syntax given for the definition of a
package, you see that a domain constructor has three optional parts to
its definition: {\it Category Assertions}, {\it Add Domain}, and
{\it Representation}.

% *********************************************************************
\head{section}{Category Assertions}{ugDomainsAssertions}
% *********************************************************************
%

The {\it Category Assertions} part of your domain constructor
definition lists those categories of which all domains created by
the constructor are unconditionally members.
The word ``unconditionally'' means that membership in a category
does not depend on the values of the parameters to the domain
constructor.
This part thus defines the link between the domains and the
category hierarchies given on the inside covers of this book.
As described in \spadref{ugCategoriesCorrectness}, it is this link
that makes it possible for you to pass objects of the domains as
arguments to other operations in \Language{}.

Every \spadtype{QuadraticForm} domain is declared
to be unconditionally a member of category \spadtype{AbelianGroup}.
An abelian group is a collection of elements closed under
addition.
Every object {\it x} of an abelian group has an additive inverse
{\it y} such that $x + y = 0$.
The exports of an abelian group include \spad{0},
\spadop{+}, \spadop{-}, and scalar multiplication by an integer.
After asserting that \spadtype{QuadraticForm} domains are abelian
groups, it is possible to pass quadratic forms to algorithms that
only assume arguments to have these abelian group
properties.

In \spadref{ugCategoriesConditionals}, you saw that
\spadtype{Fraction(R)}, a member of
\spadtype{QuotientFieldCategory(R)},
is a member of \spadtype{OrderedSet} if \spad{R}
is a member of \spadtype{OrderedSet}.
Likewise, from the {\tt Exports} part of the definition of
\spadtype{ModMonic(R, S)},
\begin{verbatim}
UnivariatePolynomialCategory(R) with
  if R has Finite then Finite
     ...
\end{verbatim}
you see that \spadtype{ModMonic(R, S)} is a member of
\spadtype{Finite} is \spad{R} is.

The {\tt Exports} part of a domain definition is
the same kind of
expression that can appear to the right of an
\spadSyntax{==} in a category definition.
If a domain constructor is unconditionally a member of two or more
categories, a \spad{Join} form is used.
\spadkey{Join}
The {\tt Exports} part of the definition of
\spadtype{FlexibleArray(S)} reads, for example:
\begin{verbatim}
Join(ExtensibleLinearAggregate(S),
     OneDimensionalArrayAggregate(S)) with...
\end{verbatim}

% *********************************************************************
\head{section}{A Demo}{ugDomainsDemo}
% *********************************************************************
%
Before looking at the {\it Implementation} part of \spadtype{QuadraticForm},
let's try some examples.

\vskip 2pc
\begin{xtc}
\begin{xtccomment}
Build a domain \spad{QF}.
\end{xtccomment}
\begin{spadsrc}
QF := QuadraticForm(2,Fraction Integer)
\end{spadsrc}
\begin{TeXOutput}
\begin{fricasmath}{1}
\STRING{QuadraticForm(2,Fraction(Integer))}%
\end{fricasmath}
\end{TeXOutput}
\formatResultType{Type}
\end{xtc}
\begin{xtc}
\begin{xtccomment}
Define a matrix to be used to construct
a quadratic form.
\end{xtccomment}
\begin{spadsrc}
A := matrix [[-1,1/2],[1/2,1]]
\end{spadsrc}
\begin{TeXOutput}
\begin{fricasmath}{2}
\begin{MATRIX}{2}-{1}&\frac{1}{2}\\\frac{1}{2}&1\end{MATRIX}%
\end{fricasmath}
\end{TeXOutput}
\formatResultType{Matrix(Fraction(Integer))}
\end{xtc}
\begin{xtc}
\begin{xtccomment}
Construct the quadratic form.
A package call \spad{$QF} is necessary since there
are other \spadtype{QuadraticForm} domains.
\end{xtccomment}
\begin{spadsrc}
q : QF := quadraticForm(A)
\end{spadsrc}
\begin{TeXOutput}
\begin{fricasmath}{3}
\begin{MATRIX}{2}-{1}&\frac{1}{2}\\\frac{1}{2}&1\end{MATRIX}%
\end{fricasmath}
\end{TeXOutput}
\formatResultType{QuadraticForm(2, Fraction(Integer))}
\end{xtc}
\begin{xtc}
\begin{xtccomment}
Looks like a matrix. Try computing
the number of rows.
\Language{} won't let you.
\end{xtccomment}
\begin{spadsrc}
nrows q
\end{spadsrc}
\begin{MessageOutput}
   There are 2 exposed and 1 unexposed library operations named nrows 
      having 1 argument(s) but none was determined to be applicable. 
      Use HyperDoc Browse, or issue
                              )display op nrows
      to learn more about the available operations. Perhaps 
      package-calling the operation or using coercions on the arguments
      will allow you to apply the operation.
\end{MessageOutput}
\begin{MessageOutput}
   Cannot find a definition or applicable library operation named nrows
      with argument type(s) 
                     QuadraticForm(2,Fraction(Integer))
      
      Perhaps you should use "@" to indicate the required return type, 
      or "$" to specify which version of the function you need.
\end{MessageOutput}
\end{xtc}
\begin{xtc}
\begin{xtccomment}
Create a direct product element \spad{v}.
A package call is again necessary, but \Language{}
understands your list as denoting a vector.
\end{xtccomment}
\begin{spadsrc}
v := directProduct([2,-1])$DirectProduct(2,Fraction Integer)
\end{spadsrc}
\begin{TeXOutput}
\begin{fricasmath}{4}
\BRACKET{2\COMMA -{1}}%
\end{fricasmath}
\end{TeXOutput}
\formatResultType{DirectProduct(2, Fraction(Integer))}
\end{xtc}
\begin{xtc}
\begin{xtccomment}
Compute the product $v^TAv$.
\end{xtccomment}
\begin{spadsrc}
q.v
\end{spadsrc}
\begin{TeXOutput}
\begin{fricasmath}{5}
-{5}%
\end{fricasmath}
\end{TeXOutput}
\formatResultType{Fraction(Integer)}
\end{xtc}
\begin{xtc}
\begin{xtccomment}
What is 3 times \spad{q} minus \spad{q} plus \spad{q}?
\end{xtccomment}
\begin{spadsrc}
3*q-q+q
\end{spadsrc}
\begin{TeXOutput}
\begin{fricasmath}{6}
\begin{MATRIX}{2}-{3}&\frac{3}{2}\\\frac{3}{2}&3\end{MATRIX}%
\end{fricasmath}
\end{TeXOutput}
\formatResultType{QuadraticForm(2, Fraction(Integer))}
\end{xtc}

% *********************************************************************
\head{section}{Browse}{ugDomainsBrowse}
% *********************************************************************

The \Browse{} facility of \HyperName{} is useful for
investigating
the properties of domains, packages, and categories.
From the main \HyperName{} menu, move your mouse to {\bf Browse} and
click on the left mouse button.
This brings up the \Browse{} first page.
Now, with your mouse pointer somewhere in this window, enter the
string ``quadraticform'' into the input area (all lower case
letters will do).
Move your mouse to {\bf Constructors} and click.
Up comes a page describing \spadtype{QuadraticForm}
that includes a part labeled by ``{\it
Description:}''.
You also see the types for arguments \spad{n} and \spad{K}
displayed as well as the fact that \spadtype{QuadraticForm}
returns an \spadtype{AbelianGroup}.

Select {\bf Operations} to get a list of operations for
\spadtype{QuadraticForm}.
You can select an operation by clicking on it
to get an individual page with information about that operation.
Or you can select the buttons along the bottom to see alternative
views or get additional information on the operations.
Eventually, use
\UpButton{}
to return to the first page on
\spadtype{QuadraticForm}.

You can go and experiment a bit by selecting {\bf Field} and {\bf n}
with your mouse.  Going back to {\bf Operations} you will see
that {\bf Implementations} view now works (it is disabled if
some domain parameter is unspecified).
Then return to the page on \spadtype{QuadraticForm}.

At the bottom the \spadtype{QuadraticForm} page has buttons
for {\bf Parents}, {\bf Ancestors}, and others.
Clicking on {\bf Parents}, you see that \spadtype{QuadraticForm}
has \spadtype{AbelianGroup} and \spadtype{ConvertibleTo} as
parents (note that \spadtype{QuadraticForm} distributed
with \Language{} is richer then the demo version presented
before).

% *********************************************************************
\head{section}{Representation}{ugDomainsRep}
% *********************************************************************
%
The {\tt Implementation} part of an \Language{} capsule for a
domain constructor uses the special variable \spad{Rep} to
\index{Rep @ {\tt Rep}}
identify the lower level data type used to represent the objects
\index{representation!of a domain}
of the domain.
\index{domain!representation}
The \spad{Rep} for quadratic forms is \spadtype{SquareMatrix(n, K)}.
This means that all objects of the domain are required to be
\spad{n} by \spad{n} matrices with elements from \spadtype{K}.

The code for \spadfun{quadraticForm} in Figure \ref{fig-quadform}
on page \pageref{fig-quadform}
checks that the matrix is symmetric and then converts it to
\spadSyntax on line 28 coerces \spad{m} to a
quadratic form.
In fact, the quadratic form you created in step (3) of
\spadref{ugDomainsDemo} is just the matrix you passed it in
disguise!
Without seeing this definition, you would not know that.
Nor can you take advantage of this fact now that you do know!
When we try in the next step of \spadref{ugDomainsDemo} to regard
\spad{q} as a matrix by asking for \spadfun{nrows}, the number of
its rows, \Language{} gives you an error message saying, in
effect, ``Good try, but this won't work!''

The definition for the \spadfunFrom{matrix}{QuadraticForm}
function could hardly be simpler:
it just returns its argument after explicitly
\spadglossSee{coercing}{coerce} its argument to a matrix.
Since the argument is already a matrix, this coercion does no computation.

Within the context of a capsule, an object of \spadSyntax and \spad{Rep}
have the same named operation available,
the one from \textspadsyntax{\%} takes precedence.
Thus, if you want the one from \textspadsyntax{Rep}, you must
package call it using a \textspadsyntax{\$Rep} suffix.}
This makes the definition of \spad{q.v} easy---it
just calls the \spadfunFrom{dot}{DirectProduct} product from
\spadtype{DirectProduct} to perform the indicated operation.

% *********************************************************************
\head{section}{Multiple Representations}{ugDomainsMultipleReps}
% *********************************************************************
%

To write functions that implement the operations of a domain, you
want to choose the most computationally efficient
data structure to represent the elements of your domain.

A classic problem in computer algebra is the optimal choice for an
internal representation of polynomials.
If you create a polynomial, say \mathOrSpad{3x^2+ 5}, how
does \Language{} hold this value internally?
There are many ways.
\Language{} has nearly a dozen different representations of
polynomials, one to suit almost any purpose.
Algorithms for solving polynomial equations work most
efficiently with polynomials represented one way, whereas those for
factoring polynomials are most efficient using another.
One often-used representation is a list of terms, each term
consisting of exponent-coefficient records written in the order
of decreasing exponents.
For example, the polynomial \mathOrSpad{3x^2+5} is
%>> I changed the k's in next line to e's as I thought that was
%>> clearer.
represented by the list \spad{[[e:2, c:3], [e:0, c:5]]}.

What is the optimal data structure for a matrix?
It depends on the application.
For large sparse matrices, a linked-list structure of records
holding only the non-zero elements may be optimal.
If the elements can be defined by a simple formula
\mathOrSpad{f(i,j)}, then a compiled function for
\spad{f} may be optimal.
Some programmers prefer to represent ordinary matrices as vectors
of vectors.
Others prefer to represent matrices by one big linear array where
elements are accessed with linearly computable indexes.

While all these simultaneous structures tend to be confusing,
\Language{} provides a helpful organizational tool for such a purpose:
categories.
\spadtype{PolynomialCategory}, for example, provides a uniform user
interface across all polynomial types.
Each kind of polynomial implements functions for
all these operations, each in its own way.
If you use only the top-level operations in
\spadtype{PolynomialCategory} you usually do not care what kind
of polynomial implementation is used.

%>> I've often thought, though, that it would be nice to be
%>> be able to use conditionals for representations.
Within a given domain, however, you define (at most) one
representation.\footnote{You can make that representation a
\pspadtype{Union} type, however.
See \spadref{ugTypesUnions} for examples of unions.}
If you want to have multiple representations (that is, several
domains, each with its own representation), use a category to
describe the {\tt Exports}, then define separate domains for each
representation.

% *********************************************************************
\head{section}{Add Domain}{ugDomainsAddDomain}
% *********************************************************************
%

The capsule part of {\tt Implementation} defines functions that
implement the operations exported by the domain---usually only
some of the operations.
In our demo in \spadref{ugDomainsDemo}, we asked for the value of
\spad{3*q-q+q}.
Where do the operations \spadop{*}, \spadop{+}, and
\spadop{-} come from?
There is no definition for them in the capsule!

The {\tt Implementation} part of a definition can
\index{domain!add}
optionally specify an ``add-domain'' to the left of an {\tt add}
\spadkey{add}
(for \spadtype{QuadraticForm}, defines
\spadtype{SquareMatrix(n,K)} is the add-domain).
The meaning of an add-domain is simply this: if the capsule part
of the {\tt Implementation} does not supply a function for an
operation, \Language{} goes to the add-domain to find the
function.
So do \spadopFrom{*}{QuadraticForm}, \spadopFrom{+}{QuadraticForm}
and \spadopFrom{-}{QuadraticForm} come from
\spadtype{SquareMatrix(n,K)}?
%Read on!

%*********************************************************************
\head{section}{Defaults}{ugDomainsDefaults}
% *********************************************************************
%
In \chapref{ugPackages}, we saw that categories can provide
default implementations for their operations.
How and when are they used?
When \Language{} finds that
\spadtype{QuadraticForm(2, Fraction Integer)}
does not implement the operations \spadop{*},
\spadop{+}, and \spadop{-}, it goes to
\spadtype{SquareMatrix(2,Fraction Integer)} to find it.
As it turns out, \spadtype{SquareMatrix(2, Fraction Integer)} does
not implement {\it any} of these operations!

What does \Language{} do then?
Here is its overall strategy.
First, \Language{} looks for a function in the capsule for the domain.
If it is not there, \Language{} looks in the add-domain for the
operation.
If that fails, \Language{} searches the add-domain of the add-domain,
and so on.
If all those fail, it then searches the default packages for the
categories of which the domain is a member.
In the case of \spadtype{QuadraticForm}, it searches
\spadtype{AbelianGroup}, then its parents, grandparents, and
so on.
If this fails, it then searches the default packages of the
add-domain.
Whenever a function is found, the search stops immediately and the
function is returned.
When all fails, the system calls \spadfun{error} to report this
unfortunate news to you.
To find out the actual order of constructors searched for
\spadtype{QuadraticForm}, consult \Browse{}: from the
\spadtype{QuadraticForm}, and click on {\bf Search Path}.

Let's apply this search strategy for our example \spad{3*q-q+q}.
The scalar multiplication comes first.
\Language{} finds a default implementation in
\spadtype{AbelianGroup&}.
Remember from \spadref{ugCategoriesDefaults} that
\spadtype{SemiGroup} provides a default definition for
\mathOrSpad{x^n} by repeated squaring?
\spadtype{AbelianGroup} similarly provides a definition for
\mathOrSpad{n*x} by repeated doubling.

But the search of the defaults for \spadtype{QuadraticForm} fails
to find any \spadop{+} or \spadop{*} in the default packages for
the ancestors of \spadtype{QuadraticForm}.
So it now searches among those for \spadtype{SquareMatrix}.
Category \spadtype{MatrixCategory}, which provides a uniform interface
for all matrix domains,
is a grandparent of \spadtype{SquareMatrix} and
has a capsule defining many functions for matrices, including
matrix addition, subtraction, and scalar multiplication.
The default package \spadtype{MatrixCategory&} is where the
functions for \spadopFrom{+}{QuadraticForm} and
\spadopFrom{-}{QuadraticForm} come from.

You can use \Browse{} to discover where the operations for
\spadtype{QuadraticForm} are implemented.
First, get the page describing \spadtype{QuadraticForm}.
With your mouse somewhere in this window, type a ``2'', press the
\fbox{\bf Tab} key, and then enter ``Fraction
Integer'' to indicate that you want the domain
\spadtype{QuadraticForm(2, Fraction Integer)}.
Now click on {\bf Operations} to get a table of operations and on
\spadop{*} to get a page describing the \spadop{*} operation.
Finally, click on {\bf implementation} at the bottom.

% *********************************************************************
\head{section}{Origins}{ugDomainsOrigins}
% *********************************************************************
%

Aside from the notion of where an operation is implemented,
\index{operation!origin}
a useful notion is  the {\it origin} or ``home'' of an operation.
When an operation (such as
\spadfunFrom{quadraticForm}{QuadraticForm}) is explicitly exported by
a domain (such as \spadtype{QuadraticForm}), you can say that the
origin of that operation is that domain.
If an operation is not explicitly exported from a domain, it is inherited
from, and has as origin, the (closest) category that explicitly exports it.
The operations \spadopFrom{+}{AbelianMonoid} and
\spadopFrom{-}{AbelianMonoid} of \spadtype{QuadraticForm},
for example, are inherited from \spadtype{AbelianMonoid}.
As it turns out, \spadtype{AbelianMonoid} is the origin of virtually every
\spadop{+} operation in \Language{}!

Again, you can use \Browse{} to discover the origins of
operations.
From the \Browse{} page on \spadtype{QuadraticForm}, click on {\bf
Operations}, then on {\bf origins} at the bottom of the page.

The origin of the operation is the {\it only} place where on-line
documentation is given.
However, you can re-export an operation to give it special
documentation.
Suppose you have just invented the world's fastest algorithm for
inverting matrices using a particular internal representation for
matrices.
If your matrix domain just declares that it exports
\spadtype{MatrixCategory}, it exports the \spadfun{inverse}
operation, but the documentation the user gets from \Browse{} is
the standard one from \spadtype{MatrixCategory}.
To give your version of \spadfun{inverse} the attention it
deserves, simply export the operation explicitly with new
documentation.
This redundancy gives \spadfun{inverse} a new origin and tells
\Browse{} to present your new documentation.

% *********************************************************************
\head{section}{Short Forms}{ugDomainsShortForms}
% *********************************************************************
%
In \Language{}, a domain could be defined using only an add-domain
and no capsule.
Although we talk about rational numbers as quotients of integers,
there is no type \pspadtype{RationalNumber} in \Language{}.
To create such a type, you could compile the following
``short-form'' definition:

\begin{xmpLines}
RationalNumber() == Fraction(Integer)
\end{xmpLines}

The {\tt Exports} part of this definition is missing and is taken
to be equivalent to that of \spadtype{Fraction(Integer)}.
Because of the add-domain philosophy, you get precisely
what you want.
The effect is to create a little stub of a domain.
When a user asks to add two rational numbers, \Language{} would
ask \pspadtype{RationalNumber} for a function implementing this
\spadop{+}.
Since the domain has no capsule, the domain then immediately
sends its request to \spadtype{Fraction (Integer)}.

The short form definition for domains is used to
define such domains as \spadtype{MultivariatePolynomial}:
\exptypeindex{MultivariatePolynomial}

\begin{xmpLines}
MultivariatePolynomial(vl: List Symbol, R: Ring) ==
   SparseMultivariatePolynomial(R,
      OrderedVariableList vl)
\end{xmpLines}

%% *********************************************************************
\head{section}{Example 1: Clifford Algebra}{ugDomainsClifford}
% *********************************************************************
%

Now that we have \spadtype{QuadraticForm} available,
let's put it to use.
Given some quadratic form \smath{Q} described by an
\smath{n} by \smath{n} matrix over a field \smath{K}, the domain
\spadtype{CliffordAlgebra(n, K, Q)} defines a vector space of
dimension \mathOrSpad{2^n} over \smath{K}.
This is an interesting domain since complex numbers, quaternions,
exterior algebras and spin algebras are all examples of Clifford
algebras.

The basic idea is this:
the quadratic form \spad{Q} defines a basis
$e_1,e_2\ldots,e_n$ for the
vector space \mathOrSpad{K^n}---the direct product of \spad{K}
with itself \spad{n} times.
From this, the Clifford algebra generates a basis of
\mathOrSpad{2^n} elements given by all the possible products
of the $e_i$ in order without duplicates, that is,
1,
$e_1$,
$e_2$,
$e_1e_2$,
$e_3$,
$e_1e_3$,
$e_2e_3$,
$e_1e_2,e_3$,
and so on.

The algebra is defined by the relations
$$
\begin{array}{lclc}
e_i \  e_i & = & Q(e_i) \\
e_i \  e_j & = & -e_j \  e_i & \hbox{for } i \neq j
\end{array}
$$

Now look at the snapshot of its definition given in Figure
\ref{fig-clifalg}.
Lines 9-10 show part of the definitions of the {\tt Exports}.
A Clifford algebra over a field \spad{K} is asserted to be a ring,
an algebra over \spad{K}, and a vector space over \spad{K}.
Its explicit exports include
\spad{e(n),} which returns the \eth{n} unit element.

\begin{figXmpLines}[caption={Part of the \textspadtype{CliffordAlgebra} domain.},label={fig-clifalg}]
NNI ==> NonNegativeInteger
PI  ==> PositiveInteger

CliffordAlgebra(n,K,q): Exports == Implementation where
    n: PI
    K: Field
    q: QuadraticForm(n, K)

    Exports == Join(Ring,Algebra(K),VectorSpace(K)) with
      e: PI -> %
          ...

    Implementation == add
      Qeelist :=
        [q.unitVector(i::PI) for i in 1..n]
      dim     :=  2^n
      Rep     := PrimitiveArray K
      New ==> new(dim, 0$K)$Rep
      x + y ==
        z := New
        for i in 0..dim-1 repeat z.i := x.i + y.i
        z
      addMonomProd: (K, NNI, K, NNI, %) -> %
      addMonomProd(c1, b1, c2, b2, z) ==  ...
      x * y ==
        z := New
        for ix in 0..dim-1 repeat
          if x.ix ~= 0 then for iy in 0..dim-1 repeat
            if y.iy ~= 0
            then addMonomProd(x.ix,ix,y.iy,iy,z)
          z
           ...
\end{figXmpLines}

The {\tt Implementation} part begins by defining a local variable
\spad{Qeelist} to hold the list of all \spad{q.v} where \spad{v}
runs over the unit vectors from 1 to the dimension \spad{n}.
Another local variable \spad{dim} is set to \mathOrSpad{2^n},
computed once and for all.
The representation for the domain is
\spadtype{PrimitiveArray(K)},
which is a basic array of elements from domain \spad{K}.
Line 18 defines \spad{New} as shorthand for the more lengthy
expression \spad{new(dim, 0$K)$Rep}, which computes a primitive
array of length \mathOrSpad{2^n} filled with \spad{0}'s from
domain \spad{K}.

Lines 19-22 define the sum of two elements \spad{x} and \spad{y}
straightforwardly.
First, a new array of all \spad{0}'s is created, then filled with
the sum of the corresponding elements.
Indexing for primitive arrays starts at 0.
The definition of the product of \spad{x} and \spad{y} first requires
the definition of a local function \userfun{addMonomProd}.
\Language{} knows it is local since it is not an exported function.
The types of all local functions must be declared.

For a demonstration of \spadtype{CliffordAlgebra}, see
\xmpref{CliffordAlgebra}.

% *********************************************************************
\head{section}{Example 2: Building A Query Facility}{ugDomsinsDatabase}
% *********************************************************************
%
We now turn to an entirely different kind of application,
building a query language for a database.

Here is the practical problem to solve.
The \Browse{} facility of \Language{} has a
database for all operations and constructors which is
stored on disk and accessed by \HyperName{}.
For our purposes here, we regard each line of this file as having
eight fields:
{\tt class, name, type, nargs, exposed, kind, origin,} and {\tt condition.}
Here is an example entry:

\begin{verbatim}
o`determinant`$->R`1`x`d`Matrix(R)`has(R,commutative("*"))
\end{verbatim}

In English, the entry means:
\begin{quotation}\raggedright
The operation \spadfun{determinant}: \spad{% -> R}
with {\it 1} argument, is
{\it exposed} and is exported by {\it domain} \spadtype{Matrix(R)}
if {\tt R has commutative("*")}.
\end{quotation}

Our task is to create a little query language that allows us
to get useful information from this database.

% *********************************************************************
\head{subsection}{A Little Query Language}{ugDomainsQueryLanguage}
% *********************************************************************

First we design a simple language for accessing information from
the database.
We have the following simple model in mind for its design.
Think of the database as a box of index cards.
There is only one search operation---it
takes the name of a field and a predicate
\index{predicate}
(a boolean-valued function) defined on the fields of the
index cards.
When applied, the search operation goes through the entire box
selecting only those index cards for which the predicate is true.
The result of a search is a new box of index cards.
This process can be repeated again and again.

The predicates all have a particularly simple form: {\it symbol}
{\tt =} {\it pattern}, where {\it symbol} designates one of the
fields, and {\it pattern} is a ``search string''---a string
that may contain a ``{\tt *}'' as a
wildcard.
Wildcards match any substring, including the empty string.
Thus the pattern {\tt "*ma*t"} matches
{\tt "mat", "doormat"} and {\tt "smart".}

To illustrate how queries are given, we give you a sneak preview
of the facility we are about to create.

\begin{xtc}
\begin{xtccomment}
Extract the database of all \Language{} operations.
\end{xtccomment}
\begin{spadsrc}
ops := getDatabase("o")
\end{spadsrc}
\begin{TeXOutput}
\begin{fricasmath}{1}
8002%
\end{fricasmath}
\end{TeXOutput}
\formatResultType{Database(IndexCard)}
\end{xtc}
\begin{xtc}
\begin{xtccomment}
How many exposed three-argument \spadfun{map} operations involving streams?
\end{xtccomment}
\begin{spadsrc}
ops.(name="map").(nargs="3").(type="*Stream*")
\end{spadsrc}
\begin{TeXOutput}
\begin{fricasmath}{2}
3%
\end{fricasmath}
\end{TeXOutput}
\formatResultType{Database(IndexCard)}
\end{xtc}

As usual, the arguments of \spadfun{elt} (\spadSyntax{.})
associate to the left.
The first \spadfun{elt} produces the set of all operations with
name {\tt map}.
The second \spadfun{elt} produces the set of all map operations
with three arguments.
The third \spadfun{elt} produces the set of all three-argument map
operations having a type mentioning \spadtype{Stream}.

Another thing we'd like to do is to extract one field from each of
the index cards in the box and look at the result.
Here is an example of that kind of request.

\begin{xtc}
\begin{xtccomment}
What constructors explicitly export a \spadfun{determinant} operation?
\end{xtccomment}
\begin{spadsrc}
elt(elt(elt(elt(ops,name="determinant"),origin),sort),unique)
\end{spadsrc}
\begin{TeXOutput}
\begin{fricasmath}{3}
\BRACKET{\STRING{"InnerMatrixLinearAlgebraFunctions"}\COMMA \STRING{%
"MatrixCategory"}\COMMA \STRING{"MatrixLinearAlgebraFunctions"}\COMMA \STRING%
{"SquareMatrixCategory"}}%
\end{fricasmath}
\end{TeXOutput}
\formatResultType{DataList(String)}
\end{xtc}

The first \spadfun{elt} produces the set of all index cards with
name {\tt determinant}.
The second \spadfun{elt} extracts the {\tt origin} component from
each index card. Each origin component
is the name of a constructor which directly
exports the operation represented by the index card.
Extracting a component from each index card produces what we call
a {\it datalist}.
The third \spadfun{elt}, {\tt sort}, causes the datalist of
origins to be sorted in alphabetic
order.
The fourth, {\tt unique}, causes duplicates to be removed.

Before giving you a more extensive demo of this facility,
we now build the necessary domains and packages to implement it.
%We will introduce a few of our minor conveniences.

% *********************************************************************
\head{subsection}{The Database Constructor}{ugDomainsDatabaseConstructor}
% *********************************************************************

We work from the top down. First, we define a database,
our box of index cards, as an abstract datatype.
For sake of illustration and generality,
we assume that an index card is some type \spad{S}, and
that a database is a box of objects of type \spad{S}.
Here is the \Language{} program defining the \pspadtype{Database}
domain.

\begin{xmpLines}
PI ==> PositiveInteger
Database(S): Exports == Implementation where
  S: Object with
    elt: (%, Symbol) -> String
    display: % -> Void
    fullDisplay: % -> Void

  Exports == with
    elt: (%,QueryEquation) -> %                          -- Select by an equation.
    elt: (%, Symbol) -> DataList String                  -- Select by a field name.
    "+": (%,%) -> %                                      -- Combine two databases.
    "-": (%,%) -> %                                      -- Subtract one from another.
    display: % -> Void                                   -- A brief database display.
    fullDisplay: % -> Void                               -- A full database display.
    fullDisplay: (%,PI,PI) -> Void                       -- A selective display.
    coerce: % -> OutputForm                              -- Display a database.
  Implementation == add
      ...
\end{xmpLines}

The domain constructor takes a parameter \spad{S}, which
stands for the class of index cards.
We describe an index card later.
Here think of an index card as a string which has
the eight fields mentioned above.

First, we tell \Language{} what operations we are going to require
from index cards.
We need an \spadfun{elt} to extract the contents of a field
(such as {\tt name} and {\tt type}) as a string.
For example,
\spad{c.name} returns a string that is the content of the
\spad{name} field on the index card \spad{c}.
We need to display an index card in two ways:
\pspadfun{display} shows only the name and type of an
operation;
\pspadfun{fullDisplay} displays all fields.
The display operations return no useful information and thus have
return type \spadtype{Void}.

Next, we tell \Language{} what operations the user can apply
to the database.
This part defines our little query language.
The most important operation is
{\frenchspacing\tt db . field = pattern} which
returns a new database, consisting of all index
cards of {\tt db} such that the \spad{field} part of the index
card is matched by the string pattern called \spad{pattern}.
The expression {\tt field = pattern} is an object of type
\spadtype{QueryEquation} (defined in the next section).

Another \spadfun{elt} is needed to produce a \pspadtype{DataList}
object.
Operation \spadop{+} is to merge two databases together;
\spadop{-} is used to subtract away common entries in a second
database from an initial database.
There are three display functions.
The \pspadfun{fullDisplay} function has two versions: one
that prints all the records, the other that prints only a fixed
number of records.
A \spadfun{coerce} to \spadtype{OutputForm} creates a display
object.

The {\tt Implementation} part of \spadtype{Database} is straightforward.
\begin{xmpLines}
  Implementation == add
    s: Symbol
    Rep := List S
    elt(db,equation) == ...
    elt(db,key) == [x.key for x in db]::DataList(String)
    display(db) ==  for x in db repeat display x
    fullDisplay(db) == for x in db repeat fullDisplay x
    fullDisplay(db, n, m) == for x in db for i in 1..m
      repeat
        if i >= n then fullDisplay x
    x+y == removeDuplicates! merge(x,y)
    x-y == mergeDifference(copy(x::Rep),
                           y::Rep)$MergeThing(S)
    coerce(db): OutputForm == (#db):: OutputForm
\end{xmpLines}

The database is represented by a list of elements of \spad{S} (index cards).
We leave the definition of the first \spadfun{elt} operation
(on line 4) until the next section.
The second \spadfun{elt} collects all the strings with field name
{\it key} into a list.
The \spadfun{display} function and first \spadfun{fullDisplay} function
simply call the corresponding functions from \spad{S}.
The second \spadfun{fullDisplay} function provides an efficient way of
printing out a portion of a large list.
The \spadop{+} is defined by using the existing
\spadfunFrom{merge}{List} operation defined on lists, then
removing duplicates from the result.
The \spadop{-} operation requires writing a corresponding
subtraction operation.
A package \spadtype{MergeThing} (not shown) provides this.

The \spadfun{coerce} function converts the database to an
\spadtype{OutputForm} by computing the number of index cards.
This is a good example of the independence of
the representation of an \Language{} object from how it presents
itself to the user. We usually do not want to look at a database---but
do care how many ``hits'' we get for a given query.
So we define the output representation of a database to be simply
the number of index cards our query finds.
% *********************************************************************
\head{subsection}{Query Equations}{ugDomainsQueryEquations}
% *********************************************************************

The predicate for our search is given by an object of type
\pspadtype{QueryEquation}.
\Language{} does not have such an object yet so we
have to invent it.

\begin{xmpLines}
QueryEquation(): Exports == Implementation where
  Exports == with
    equation: (Symbol, String) -> %
    variable: % -> Symbol
    value:    % -> String

  Implementation == add
    Rep := Record(var:Symbol, val:String)
    equation(x, s) == [x, s]
    variable q == q.var
    value    q == q.val
\end{xmpLines}

\Language{} converts an input expression of the form
\spad{a = b} to \spad{equation(a, b)}.
Our equations always have a symbol on the left and a string
on the right.
The {\tt Exports} part thus specifies an operation
\spadfun{equation} to create a query equation, and
\pspadfun{variable} and \pspadfun{value} to select the left- and
right-hand sides.
The {\tt Implementation} part uses \pspadtype{Record} for a
space-efficient representation of an equation.

Here is the missing definition for the \spadfun{elt} function of
\pspadtype{Database} in the last section:

\begin{xmpLines}
    elt(db,eq) ==
      field  := variable eq
      value := value eq
      [x for x in db | matches?(value,x.field)]
\end{xmpLines}

Recall that a database is represented by a list.
Line 4 simply runs over that list collecting all elements
such that the pattern (that is, \spad{value})
matches the selected field of the element.

% *********************************************************************
\head{subsection}{DataLists}{ugDomainsDataLists}
% *********************************************************************

Type \pspadtype{DataList} is a new type invented to hold the result
of selecting one field from each of the index cards in the box.
It is useful to make datalists extensions of lists---lists that
have special \spadfun{elt} operations defined on them for
sorting and removing duplicates.

\begin{xmpLines}
DataList(S:OrderedSet) : Exports == Implementation where
  Exports == ListAggregate(S) with
    elt: (%,"unique") -> %
    elt: (%,"sort") -> %
    elt: (%,"count") -> NonNegativeInteger
    coerce: List S -> %

  Implementation ==  List(S) add
    Rep := List S
    elt(x,"unique") == removeDuplicates(x)
    elt(x,"sort") == sort(x)
    elt(x,"count") == #x
    coerce(x:List S) == x :: %
\end{xmpLines}

The {\tt Exports} part asserts that datalists belong to the
category \spadtype{ListAggregate}.
Therefore, you can use all the usual list operations on datalists,
such as \spadfunFrom{first}{List}, \spadfunFrom{rest}{List}, and
\spadfunFrom{concat}{List}.
In addition, datalists have four explicit operations.
Besides the three \spadfun{elt} operations, there is a
\spadfun{coerce} operation that creates datalists from lists.

The {\tt Implementation} part needs only to define four functions.
All the rest are obtained from \spadtype{List(S)}.

% *********************************************************************
\head{subsection}{Index Cards}{ugDomainsDatabase}
% *********************************************************************

An index card comes from a file as one long string.
We define functions that extract substrings from the long
string.
Each field has a name that
is passed as a second argument to \spadfun{elt}.

\begin{xmpLines}
IndexCard() == Implementation where
  Exports == with
    elt: (%, Symbol) -> String
    display: % -> Void
    fullDisplay: % -> Void
    coerce: String -> %
  Implementation == String add ...
\end{xmpLines}

We leave the {\tt Implementation} part to the reader.
All operations involve straightforward string manipulations.

% *********************************************************************
\head{subsection}{Creating a Database}{ugDomainsCreating}
% *********************************************************************

We must not forget one important operation: one that builds the database in the
first place!
We'll name it \pspadfun{getDatabase} and put it in a package.
This function is implemented by calling the \Lisp{} function
\spad{getBrowseDatabase(s)} to get appropriate information from
\Browse{}.
This operation takes a string indicating which lines you
want from the database: \spad{"o"} gives you all operation
lines, and \spad{"k"}, all constructor lines.
Similarly, \spad{"c"}, \spad{"d"}, and \spad{"p"} give
you all category, domain and package lines respectively.
%
\begin{xmpLines}
OperationsQuery(): Exports == Implementation where
  Exports == with
    getDatabase: String -> Database(IndexCard)

  Implementation == add
    getDatabase(s) == getBrowseDatabase(s)$Lisp
\end{xmpLines}

We do not bother creating a special name for databases of index
cards.
\pspadtype{Database (IndexCard)} will do.
Notice that we used the package \pspadtype{OperationsQuery} to
create, in effect,
a new kind of domain: \pspadtype{Database(IndexCard)}.

% *********************************************************************
\head{subsection}{Putting It All Together}{ugDomainsPutting}
% *********************************************************************

To create the database facility, you put all these constructors
into one file.\footnote{You could use separate files, but we
are putting them all together because, organizationally, that is
the logical thing to do.}
At the top of the file put \spadsys{)abbrev} commands, giving the
constructor abbreviations you created.

\begin{xmpLines}
)abbrev domain  ICARD   IndexCard
)abbrev domain  QEQUAT  QueryEquation
)abbrev domain  MTHING  MergeThing
)abbrev domain  DLIST   DataList
)abbrev domain  DBASE   Database
)abbrev package OPQUERY OperationsQuery
\end{xmpLines}

With all this in {\bf alql.spad}, for example, compile it using
\syscmdindex{compile}
\begin{verbatim}
)compile alql
\end{verbatim}
and then load each of the constructors:
\begin{verbatim}
)load ICARD QEQUAT MTHING DLIST DBASE OPQUERY
\end{verbatim}
\syscmdindex{load}
You are ready to try some sample queries.

% *********************************************************************
\head{subsection}{Example Queries}{ugDomainsExamples}
% *********************************************************************

Our first set of queries give some statistics on constructors in
the current \Language{} system.

\begin{xtc}
\begin{xtccomment}
How many constructors does \Language{} have?
\end{xtccomment}
\begin{spadsrc}
ks := getDatabase "k"
\end{spadsrc}
\begin{TeXOutput}
\begin{fricasmath}{1}
1212%
\end{fricasmath}
\end{TeXOutput}
\formatResultType{Database(IndexCard)}
\end{xtc}
\begin{xtc}
\begin{xtccomment}
Break this down into the number of categories, domains, and packages.
\end{xtccomment}
\begin{spadsrc}
[ks.(kind=k) for k in ["c","d","p"]]
\end{spadsrc}
\begin{TeXOutput}
\begin{fricasmath}{2}
\BRACKET{264\COMMA 413\COMMA 535}%
\end{fricasmath}
\end{TeXOutput}
\formatResultType{List(Database(IndexCard))}
\end{xtc}
\begin{xtc}
\begin{xtccomment}
What are all the domain constructors that take 5 parameters?
\end{xtccomment}
\begin{spadsrc}
elt(ks.(kind="d").(nargs="5"),name)
\end{spadsrc}
\begin{TeXOutput}
\begin{fricasmath}{3}
\BRACKET{\STRING{"FractionalIdealAsModule"}\COMMA \STRING{"IndexedJetBundle"}%
\COMMA \STRING{"InnerIndexedTwoDimensionalArray"}\COMMA \STRING{%
"ModularField"}\COMMA \STRING{"ModularRing"}\COMMA \STRING{%
"RadicalFunctionField"}\COMMA \STRING{"ResidueRing"}\COMMA \STRING{%
"TensorProduct"}}%
\end{fricasmath}
\end{TeXOutput}
\formatResultType{DataList(String)}
\end{xtc}
\begin{xtc}
\begin{xtccomment}
How many constructors have ``Matrix'' in their name?
\end{xtccomment}
\begin{spadsrc}
mk := ks.(name="*Matrix*")
\end{spadsrc}
\begin{TeXOutput}
\begin{fricasmath}{4}
34%
\end{fricasmath}
\end{TeXOutput}
\formatResultType{Database(IndexCard)}
\end{xtc}
\begin{xtc}
\begin{xtccomment}
What are the names of those that are domains?
\end{xtccomment}
\begin{spadsrc}
elt(mk.(kind="d"),name)
\end{spadsrc}
\begin{TeXOutput}
\begin{fricasmath}{5}
\BRACKET{\STRING{"ComplexDoubleFloatMatrix"}\COMMA \STRING{%
"DenavitHartenbergMatrix"}\COMMA \STRING{"DirectProductMatrixModule"}\COMMA %
\STRING{"DoubleFloatMatrix"}\COMMA \STRING{"IndexedMatrix"}\COMMA \STRING{%
"LieSquareMatrix"}\COMMA \STRING{"Matrix"}\COMMA \STRING{"RectangularMatrix"}%
\COMMA \STRING{"SparseEchelonMatrix"}\COMMA \STRING{"SquareMatrix"}\COMMA %
\STRING{"ThreeDimensionalMatrix"}\COMMA \STRING{"U16Matrix"}\COMMA \STRING{%
"U32Matrix"}\COMMA \STRING{"U8Matrix"}}%
\end{fricasmath}
\end{TeXOutput}
\formatResultType{DataList(String)}
\end{xtc}
\begin{xtc}
\begin{xtccomment}
How many operations are there in the library?
\end{xtccomment}
\begin{spadsrc}
o := getDatabase "o"
\end{spadsrc}
\begin{TeXOutput}
\begin{fricasmath}{6}
8002%
\end{fricasmath}
\end{TeXOutput}
\formatResultType{Database(IndexCard)}
\end{xtc}
\begin{xtc}
\begin{xtccomment}
Break this down into categories, domains, and packages.
\end{xtccomment}
\begin{spadsrc}
[o.(kind=k) for k in ["c","d","p"]]
\end{spadsrc}
\begin{TeXOutput}
\begin{fricasmath}{7}
\BRACKET{2011\COMMA 2791\COMMA 3200}%
\end{fricasmath}
\end{TeXOutput}
\formatResultType{List(Database(IndexCard))}
\end{xtc}

The query language is helpful in getting information about a
particular operation you might like to apply.
While this information can be obtained with
\Browse{}, the use of the query database gives you data that you
can manipulate in the workspace.

\begin{xtc}
\begin{xtccomment}
How many operations have ``eigen'' in the name?
\end{xtccomment}
\begin{spadsrc}
eigens := o.(name="*eigen*")
\end{spadsrc}
\begin{TeXOutput}
\begin{fricasmath}{8}
9%
\end{fricasmath}
\end{TeXOutput}
\formatResultType{Database(IndexCard)}
\end{xtc}
\begin{xtc}
\begin{xtccomment}
What are their names?
\end{xtccomment}
\begin{spadsrc}
elt(eigens,name)
\end{spadsrc}
\begin{TeXOutput}
\begin{fricasmath}{9}
\BRACKET{\STRING{"eigenMatrix"}\COMMA \STRING{"eigenvalues"}\COMMA \STRING{%
"eigenvalues"}\COMMA \STRING{"eigenvalues"}\COMMA \STRING{"eigenvector"}%
\COMMA \STRING{"eigenvector"}\COMMA \STRING{"eigenvectors"}\COMMA \STRING{%
"eigenvectors"}\COMMA \STRING{"eigenvectors"}}%
\end{fricasmath}
\end{TeXOutput}
\formatResultType{DataList(String)}
\end{xtc}
\begin{xtc}
\begin{xtccomment}
Where do they come from?
\end{xtccomment}
\begin{spadsrc}
elt(elt(elt(eigens,origin),sort),unique) 
\end{spadsrc}
\begin{TeXOutput}
\begin{fricasmath}{10}
\BRACKET{\STRING{"EigenPackage"}\COMMA \STRING{"InnerEigenPackage"}\COMMA %
\STRING{"RadicalEigenPackage"}}%
\end{fricasmath}
\end{TeXOutput}
\formatResultType{DataList(String)}
\end{xtc}

The operations \spadop{+} and \spadop{-} are useful for
constructing small databases and combining them.
However, remember that the only matching you can do is string
matching.
Thus a pattern such as {\tt "*Matrix*"} on the type field
matches
any type containing \spadtype{Matrix}, \spadtype{MatrixCategory},
\spadtype{SquareMatrix}, and so on.

\begin{xtc}
\begin{xtccomment}
How many operations mention ``Matrix'' in their type?
\end{xtccomment}
\begin{spadsrc}
tm := o.(type="*Matrix*")
\end{spadsrc}
\begin{TeXOutput}
\begin{fricasmath}{11}
329%
\end{fricasmath}
\end{TeXOutput}
\formatResultType{Database(IndexCard)}
\end{xtc}
\begin{xtc}
\begin{xtccomment}
How many operations come from constructors with ``Matrix'' in
their name?
\end{xtccomment}
\begin{spadsrc}
fm := o.(origin="*Matrix*")
\end{spadsrc}
\begin{TeXOutput}
\begin{fricasmath}{12}
245%
\end{fricasmath}
\end{TeXOutput}
\formatResultType{Database(IndexCard)}
\end{xtc}
\begin{xtc}
\begin{xtccomment}
How many operations are in \spad{fm} but not in \spad{tm}?
\end{xtccomment}
\begin{spadsrc}
fm-tm 
\end{spadsrc}
\begin{TeXOutput}
\begin{fricasmath}{13}
204%
\end{fricasmath}
\end{TeXOutput}
\formatResultType{Database(IndexCard)}
\end{xtc}
\begin{xtc}
\begin{xtccomment}
Display the operations that both mention ``Matrix'' in their type
and come from a constructor having ``Matrix'' in their name.
\end{xtccomment}
\begin{spadsrc}
fullDisplay(fm-%) 
\end{spadsrc}
\end{xtc}
\begin{xtc}
\begin{xtccomment}
How many operations involve matrices?
\end{xtccomment}
\begin{spadsrc}
m := tm+fm 
\end{spadsrc}
\begin{TeXOutput}
\begin{fricasmath}{15}
525%
\end{fricasmath}
\end{TeXOutput}
\formatResultType{Database(IndexCard)}
\end{xtc}
\begin{xtc}
\begin{xtccomment}
Display 4 of them.
\end{xtccomment}
\begin{spadsrc}
fullDisplay(m, 202, 205) 
\end{spadsrc}
\end{xtc}
\begin{xtc}
\begin{xtccomment}
How many distinct names of operations involving matrices are there?
\end{xtccomment}
\begin{spadsrc}
elt(elt(elt(m,name),unique),count) 
\end{spadsrc}
\begin{TeXOutput}
\begin{fricasmath}{17}
284%
\end{fricasmath}
\end{TeXOutput}
\formatResultType{PositiveInteger}
\end{xtc}
