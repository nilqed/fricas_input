% Generated by Sphinx.
\def\sphinxdocclass{report}
\newif\ifsphinxKeepOldNames \sphinxKeepOldNamestrue
\documentclass[letterpaper,10pt,english]{sphinxmanual}
\usepackage{iftex}

\ifPDFTeX
  \usepackage[utf8]{inputenc}
\fi
\ifdefined\DeclareUnicodeCharacter
  \DeclareUnicodeCharacter{00A0}{\nobreakspace}
\fi
\usepackage{cmap}
\usepackage[T1]{fontenc}
\usepackage{amsmath,amssymb,amstext}
\usepackage{babel}
\usepackage{times}
\usepackage[Bjarne]{fncychap}
\usepackage{longtable}
\usepackage{sphinx}
\usepackage{multirow}
\usepackage{eqparbox}


\addto\captionsenglish{\renewcommand{\figurename}{Fig.\@ }}
\addto\captionsenglish{\renewcommand{\tablename}{Table }}
\SetupFloatingEnvironment{literal-block}{name=Listing }

\addto\extrasenglish{\def\pageautorefname{page}}

\setcounter{tocdepth}{2}


\title{DifferentialForms Documentation}
\date{Dec 17, 2016}
\release{1.1.0}
\author{Kurt Pagani \textless{}nilqed@gmail.com\textgreater{}}
\newcommand{\sphinxlogo}{}
\renewcommand{\releasename}{Release}
\makeindex

\makeatletter
\def\PYG@reset{\let\PYG@it=\relax \let\PYG@bf=\relax%
    \let\PYG@ul=\relax \let\PYG@tc=\relax%
    \let\PYG@bc=\relax \let\PYG@ff=\relax}
\def\PYG@tok#1{\csname PYG@tok@#1\endcsname}
\def\PYG@toks#1+{\ifx\relax#1\empty\else%
    \PYG@tok{#1}\expandafter\PYG@toks\fi}
\def\PYG@do#1{\PYG@bc{\PYG@tc{\PYG@ul{%
    \PYG@it{\PYG@bf{\PYG@ff{#1}}}}}}}
\def\PYG#1#2{\PYG@reset\PYG@toks#1+\relax+\PYG@do{#2}}

\expandafter\def\csname PYG@tok@gd\endcsname{\def\PYG@tc##1{\textcolor[rgb]{0.63,0.00,0.00}{##1}}}
\expandafter\def\csname PYG@tok@gu\endcsname{\let\PYG@bf=\textbf\def\PYG@tc##1{\textcolor[rgb]{0.50,0.00,0.50}{##1}}}
\expandafter\def\csname PYG@tok@gt\endcsname{\def\PYG@tc##1{\textcolor[rgb]{0.00,0.27,0.87}{##1}}}
\expandafter\def\csname PYG@tok@gs\endcsname{\let\PYG@bf=\textbf}
\expandafter\def\csname PYG@tok@gr\endcsname{\def\PYG@tc##1{\textcolor[rgb]{1.00,0.00,0.00}{##1}}}
\expandafter\def\csname PYG@tok@cm\endcsname{\let\PYG@it=\textit\def\PYG@tc##1{\textcolor[rgb]{0.25,0.50,0.56}{##1}}}
\expandafter\def\csname PYG@tok@vg\endcsname{\def\PYG@tc##1{\textcolor[rgb]{0.73,0.38,0.84}{##1}}}
\expandafter\def\csname PYG@tok@vi\endcsname{\def\PYG@tc##1{\textcolor[rgb]{0.73,0.38,0.84}{##1}}}
\expandafter\def\csname PYG@tok@mh\endcsname{\def\PYG@tc##1{\textcolor[rgb]{0.13,0.50,0.31}{##1}}}
\expandafter\def\csname PYG@tok@cs\endcsname{\def\PYG@tc##1{\textcolor[rgb]{0.25,0.50,0.56}{##1}}\def\PYG@bc##1{\setlength{\fboxsep}{0pt}\colorbox[rgb]{1.00,0.94,0.94}{\strut ##1}}}
\expandafter\def\csname PYG@tok@ge\endcsname{\let\PYG@it=\textit}
\expandafter\def\csname PYG@tok@vc\endcsname{\def\PYG@tc##1{\textcolor[rgb]{0.73,0.38,0.84}{##1}}}
\expandafter\def\csname PYG@tok@il\endcsname{\def\PYG@tc##1{\textcolor[rgb]{0.13,0.50,0.31}{##1}}}
\expandafter\def\csname PYG@tok@go\endcsname{\def\PYG@tc##1{\textcolor[rgb]{0.20,0.20,0.20}{##1}}}
\expandafter\def\csname PYG@tok@cp\endcsname{\def\PYG@tc##1{\textcolor[rgb]{0.00,0.44,0.13}{##1}}}
\expandafter\def\csname PYG@tok@gi\endcsname{\def\PYG@tc##1{\textcolor[rgb]{0.00,0.63,0.00}{##1}}}
\expandafter\def\csname PYG@tok@gh\endcsname{\let\PYG@bf=\textbf\def\PYG@tc##1{\textcolor[rgb]{0.00,0.00,0.50}{##1}}}
\expandafter\def\csname PYG@tok@ni\endcsname{\let\PYG@bf=\textbf\def\PYG@tc##1{\textcolor[rgb]{0.84,0.33,0.22}{##1}}}
\expandafter\def\csname PYG@tok@nl\endcsname{\let\PYG@bf=\textbf\def\PYG@tc##1{\textcolor[rgb]{0.00,0.13,0.44}{##1}}}
\expandafter\def\csname PYG@tok@nn\endcsname{\let\PYG@bf=\textbf\def\PYG@tc##1{\textcolor[rgb]{0.05,0.52,0.71}{##1}}}
\expandafter\def\csname PYG@tok@no\endcsname{\def\PYG@tc##1{\textcolor[rgb]{0.38,0.68,0.84}{##1}}}
\expandafter\def\csname PYG@tok@na\endcsname{\def\PYG@tc##1{\textcolor[rgb]{0.25,0.44,0.63}{##1}}}
\expandafter\def\csname PYG@tok@nb\endcsname{\def\PYG@tc##1{\textcolor[rgb]{0.00,0.44,0.13}{##1}}}
\expandafter\def\csname PYG@tok@nc\endcsname{\let\PYG@bf=\textbf\def\PYG@tc##1{\textcolor[rgb]{0.05,0.52,0.71}{##1}}}
\expandafter\def\csname PYG@tok@nd\endcsname{\let\PYG@bf=\textbf\def\PYG@tc##1{\textcolor[rgb]{0.33,0.33,0.33}{##1}}}
\expandafter\def\csname PYG@tok@ne\endcsname{\def\PYG@tc##1{\textcolor[rgb]{0.00,0.44,0.13}{##1}}}
\expandafter\def\csname PYG@tok@nf\endcsname{\def\PYG@tc##1{\textcolor[rgb]{0.02,0.16,0.49}{##1}}}
\expandafter\def\csname PYG@tok@si\endcsname{\let\PYG@it=\textit\def\PYG@tc##1{\textcolor[rgb]{0.44,0.63,0.82}{##1}}}
\expandafter\def\csname PYG@tok@s2\endcsname{\def\PYG@tc##1{\textcolor[rgb]{0.25,0.44,0.63}{##1}}}
\expandafter\def\csname PYG@tok@nt\endcsname{\let\PYG@bf=\textbf\def\PYG@tc##1{\textcolor[rgb]{0.02,0.16,0.45}{##1}}}
\expandafter\def\csname PYG@tok@nv\endcsname{\def\PYG@tc##1{\textcolor[rgb]{0.73,0.38,0.84}{##1}}}
\expandafter\def\csname PYG@tok@s1\endcsname{\def\PYG@tc##1{\textcolor[rgb]{0.25,0.44,0.63}{##1}}}
\expandafter\def\csname PYG@tok@ch\endcsname{\let\PYG@it=\textit\def\PYG@tc##1{\textcolor[rgb]{0.25,0.50,0.56}{##1}}}
\expandafter\def\csname PYG@tok@m\endcsname{\def\PYG@tc##1{\textcolor[rgb]{0.13,0.50,0.31}{##1}}}
\expandafter\def\csname PYG@tok@gp\endcsname{\let\PYG@bf=\textbf\def\PYG@tc##1{\textcolor[rgb]{0.78,0.36,0.04}{##1}}}
\expandafter\def\csname PYG@tok@sh\endcsname{\def\PYG@tc##1{\textcolor[rgb]{0.25,0.44,0.63}{##1}}}
\expandafter\def\csname PYG@tok@ow\endcsname{\let\PYG@bf=\textbf\def\PYG@tc##1{\textcolor[rgb]{0.00,0.44,0.13}{##1}}}
\expandafter\def\csname PYG@tok@sx\endcsname{\def\PYG@tc##1{\textcolor[rgb]{0.78,0.36,0.04}{##1}}}
\expandafter\def\csname PYG@tok@bp\endcsname{\def\PYG@tc##1{\textcolor[rgb]{0.00,0.44,0.13}{##1}}}
\expandafter\def\csname PYG@tok@c1\endcsname{\let\PYG@it=\textit\def\PYG@tc##1{\textcolor[rgb]{0.25,0.50,0.56}{##1}}}
\expandafter\def\csname PYG@tok@o\endcsname{\def\PYG@tc##1{\textcolor[rgb]{0.40,0.40,0.40}{##1}}}
\expandafter\def\csname PYG@tok@kc\endcsname{\let\PYG@bf=\textbf\def\PYG@tc##1{\textcolor[rgb]{0.00,0.44,0.13}{##1}}}
\expandafter\def\csname PYG@tok@c\endcsname{\let\PYG@it=\textit\def\PYG@tc##1{\textcolor[rgb]{0.25,0.50,0.56}{##1}}}
\expandafter\def\csname PYG@tok@mf\endcsname{\def\PYG@tc##1{\textcolor[rgb]{0.13,0.50,0.31}{##1}}}
\expandafter\def\csname PYG@tok@err\endcsname{\def\PYG@bc##1{\setlength{\fboxsep}{0pt}\fcolorbox[rgb]{1.00,0.00,0.00}{1,1,1}{\strut ##1}}}
\expandafter\def\csname PYG@tok@mb\endcsname{\def\PYG@tc##1{\textcolor[rgb]{0.13,0.50,0.31}{##1}}}
\expandafter\def\csname PYG@tok@ss\endcsname{\def\PYG@tc##1{\textcolor[rgb]{0.32,0.47,0.09}{##1}}}
\expandafter\def\csname PYG@tok@sr\endcsname{\def\PYG@tc##1{\textcolor[rgb]{0.14,0.33,0.53}{##1}}}
\expandafter\def\csname PYG@tok@mo\endcsname{\def\PYG@tc##1{\textcolor[rgb]{0.13,0.50,0.31}{##1}}}
\expandafter\def\csname PYG@tok@kd\endcsname{\let\PYG@bf=\textbf\def\PYG@tc##1{\textcolor[rgb]{0.00,0.44,0.13}{##1}}}
\expandafter\def\csname PYG@tok@mi\endcsname{\def\PYG@tc##1{\textcolor[rgb]{0.13,0.50,0.31}{##1}}}
\expandafter\def\csname PYG@tok@kn\endcsname{\let\PYG@bf=\textbf\def\PYG@tc##1{\textcolor[rgb]{0.00,0.44,0.13}{##1}}}
\expandafter\def\csname PYG@tok@cpf\endcsname{\let\PYG@it=\textit\def\PYG@tc##1{\textcolor[rgb]{0.25,0.50,0.56}{##1}}}
\expandafter\def\csname PYG@tok@kr\endcsname{\let\PYG@bf=\textbf\def\PYG@tc##1{\textcolor[rgb]{0.00,0.44,0.13}{##1}}}
\expandafter\def\csname PYG@tok@s\endcsname{\def\PYG@tc##1{\textcolor[rgb]{0.25,0.44,0.63}{##1}}}
\expandafter\def\csname PYG@tok@kp\endcsname{\def\PYG@tc##1{\textcolor[rgb]{0.00,0.44,0.13}{##1}}}
\expandafter\def\csname PYG@tok@w\endcsname{\def\PYG@tc##1{\textcolor[rgb]{0.73,0.73,0.73}{##1}}}
\expandafter\def\csname PYG@tok@kt\endcsname{\def\PYG@tc##1{\textcolor[rgb]{0.56,0.13,0.00}{##1}}}
\expandafter\def\csname PYG@tok@sc\endcsname{\def\PYG@tc##1{\textcolor[rgb]{0.25,0.44,0.63}{##1}}}
\expandafter\def\csname PYG@tok@sb\endcsname{\def\PYG@tc##1{\textcolor[rgb]{0.25,0.44,0.63}{##1}}}
\expandafter\def\csname PYG@tok@k\endcsname{\let\PYG@bf=\textbf\def\PYG@tc##1{\textcolor[rgb]{0.00,0.44,0.13}{##1}}}
\expandafter\def\csname PYG@tok@se\endcsname{\let\PYG@bf=\textbf\def\PYG@tc##1{\textcolor[rgb]{0.25,0.44,0.63}{##1}}}
\expandafter\def\csname PYG@tok@sd\endcsname{\let\PYG@it=\textit\def\PYG@tc##1{\textcolor[rgb]{0.25,0.44,0.63}{##1}}}

\def\PYGZbs{\char`\\}
\def\PYGZus{\char`\_}
\def\PYGZob{\char`\{}
\def\PYGZcb{\char`\}}
\def\PYGZca{\char`\^}
\def\PYGZam{\char`\&}
\def\PYGZlt{\char`\<}
\def\PYGZgt{\char`\>}
\def\PYGZsh{\char`\#}
\def\PYGZpc{\char`\%}
\def\PYGZdl{\char`\$}
\def\PYGZhy{\char`\-}
\def\PYGZsq{\char`\'}
\def\PYGZdq{\char`\"}
\def\PYGZti{\char`\~}
% for compatibility with earlier versions
\def\PYGZat{@}
\def\PYGZlb{[}
\def\PYGZrb{]}
\makeatother

\renewcommand\PYGZsq{\textquotesingle}

\begin{document}

\maketitle
\tableofcontents
\phantomsection\label{index::doc}


Contents:


\chapter{1 Theory}
\label{section-1.0:introduction}\label{section-1.0:fricas-differentialforms}\label{section-1.0::doc}\label{section-1.0:theory}

\section{1.0 Introduction}
\label{section-1.0:id1}
The package \sphinxcode{DifferentialForms} (in file \sphinxcode{dform.spad}) builds on the
domain \sphinxcode{DeRhamComplex}. In the following section we give a brief overview
of the functions that are going to be implemented. The focus is on precise
definitions of the notions, since those may be varying in the literature.
In section (2) we will describe the exported functions and how they work,
in section (3) some short implementation notes will be given and finally
the last section is devoted to some examples.


\section{1.1 Definitions}
\label{section-1.0:definitions}
Let \(\mathcal{M}\) be a n-dimensional manifold (sufficiently smooth and
orientable). To each point \(P \in \mathcal{M}\) there is a neighborhood
which can be diffeomorphically mapped to some region in \(\mathbb{R}^n\),
with coordinates
\begin{equation*}
\begin{split}x_1 (P'), \ldots, x_n (P')\end{split}
\end{equation*}
for all \(P' \in \mathcal{U} (P) \subset \mathcal{M}\). The tangent space
\(T_{P'} (\mathcal{M})\) at the point \(P'\) is a vector space, that
is spanned by the basis
\begin{equation*}
\begin{split}e_1 (P'), \ldots, e_n (P')\end{split}
\end{equation*}
which also is often denoted by
\begin{equation*}
\begin{split}\partial_1, \ldots, \partial_n =
 \frac{\partial}{\partial x_1}, \ldots,
 \frac{\partial}{\partial x_n}.\end{split}
\end{equation*}
A tangent vector \(v\) has the form
\begin{equation*}
\begin{split}v = \sum_{j = 1}^n v^j e_j .\end{split}
\end{equation*}
The cotangent space \(T_{P'}^{} (\mathcal{M})^{\star}\) is the vector space
of linear functionals
\begin{equation*}
\begin{split}\alpha : T_{P'} (\mathcal{M}) \rightarrow \mathbb{R},\end{split}
\end{equation*}
spanned by the basis \(e^1 (P'), \ldots, e^n (P')\)
which (corresponding to the basis \(\partial_j\)) is also denoted by
\begin{equation*}
\begin{split}d x^1,\ldots, d x^n.\end{split}
\end{equation*}
The latter notation indicates the dependency on the moving
point \(P'\). The dual basis is by definition comprised of those linear
functionals such that
\begin{equation*}
\begin{split}e^j (e_k) = \delta^j_k .\end{split}
\end{equation*}
Therefore we have
\begin{equation*}
\begin{split}\alpha (v) = \alpha \left( \sum_{j = 1}^n v^j e_j \right) =
\sum_{j = 1}^n v^j \alpha (e_j) = \sum_{j = 1}^n v^j \alpha_j,\end{split}
\end{equation*}
where \(\alpha = \sum_{j = 1}^n \alpha_j e^j\).


\subsection{1.1.1 Inner product of differential forms (\textbf{dot})}
\label{section-1.0:inner-product-of-differential-forms-dot}
Let \(g_x\) be a symmetric \(n \times n\) matrix which is nondegenerate
(i.e. \(\det (g_x) \neq 0\)). The index x indicates that this matrix
depends on the coordinates \(x_1 (P), \ldots, x_n (P)\) and may be varying
from point to point. If this dependency is smooth (enough) we speak of a
(pseudo-) Riemannian metric (locally). This way we get an isomorphism between
tangent vectors and 1-forms (= covectors):
\begin{equation*}
\begin{split}\alpha_j = g_{j k} v^k, \hspace{1.2em} v^j = g^{j k} \alpha_j .\end{split}
\end{equation*}
Clearly, \(\sum_k g^{j k} g_{k l} = \delta^j_l\), in other words
\((g^{j k})\) is the inverse of g. The metric g defines an \emph{inner product}
of vectors
\begin{equation*}
\begin{split}g (v, w) = \langle v, w \rangle : = g_{i j} v^i w^j\end{split}
\end{equation*}
and by duality also on 1-forms:
\begin{equation*}
\begin{split}g^{- 1} (\alpha, \beta) = \langle \alpha, \beta \rangle : = g^{i j}
 \alpha_i \beta_j .\end{split}
\end{equation*}
Now, this inner product is extended to arbitrary p-forms by
\begin{equation*}
\begin{split}\langle \alpha_1 \wedge \ldots \wedge \alpha_p , \beta_1 \wedge
\ldots \wedge \beta_p \rangle : = \det (\langle \alpha_i, \beta_j \rangle)
, \hspace{1.8em} (1 \leqslant i, j \leqslant p), \label{dot}\end{split}
\end{equation*}
and linearity.


\subsection{1.1.2. The volume form \protect\(\eta\protect\) (\textbf{volumeForm})}
\label{section-1.0:the-volume-form-volumeform}
The Riemannian \emph{volume form} \(\eta\) is (by definition) given by the
n-form
\begin{equation*}
\begin{split}\eta = \sqrt{| \det g |} e^1 \wedge \ldots \wedge e^n = \sqrt{| \det\,g |} d
x^1 \wedge \ldots \wedge d x^n . \label{vol}\end{split}
\end{equation*}
This definition makes sense because by a (orientation preserving) change of
coordinates \(\sqrt{\mathrm{det} g}\) transforms like the component of
a n-form.


\subsection{1.1.3. Hodge dual (\textbf{hodgeStar})}
\label{section-1.0:hodge-dual-hodgestar}
The \emph{Hodge dual} of a differential p-form \(\beta\) is the (n - p)-form
\(\star \beta\) such that
\begin{equation*}
\begin{split}\alpha \wedge \star \beta = \langle \alpha, \beta \rangle \eta \label{hodge}\end{split}
\end{equation*}
holds, for all p-forms \(\alpha\). The linear operator \((\star)\) is
called the \emph{Hodge star operator}. By the \emph{Riesz representation theorem} the
Hodge dual is uniquely defined by the expression above.

\textbf{Warning}

Flanders \footnote[4]{\sphinxAtStartFootnote%
Harley Flanders and Mathematics. Differential Forms with Applications to
the Physical Sciences. Dover Pubn Inc, Auflage: Revised. edition.
} defines the Hodge dual by the equality
\begin{equation*}
\begin{split}\lambda \wedge \mu = \langle \star \lambda, \mu \rangle \eta\end{split}
\end{equation*}
where \(\lambda\) is a p-form and \(\mu\) a (n - p)-form.
This can result in different signs (actually \(\star_F = s(g)\star\),
where \(s(g)\) is the sign of the determinant of \(g\)).

The generally adopted definition (2016) is the one given at the beginning
of this subsection.

The components of \(\star \beta\) are
\begin{equation*}
\begin{split}(\star \beta)_{j_1, \ldots, j_{n - p}} = \frac{1}{p!} \varepsilon_{i_1,
 \ldots, i_p, j_1, \ldots, j_{n - p}}  \sqrt{| \det g |} g^{i_1 k_1} \ldots
 g^{i_p k_p} \beta_{k_1, \ldots, k_p}\end{split}
\end{equation*}
what is equal to
\begin{equation*}
\begin{split}\frac{1}{p! \sqrt{| \det g |}} \varepsilon_{}^{k_1, \ldots, k_p, l_1,
 \ldots, l_{n - p}} g_{j_1 l_1} \ldots g_{j_{n - p}, l_{n - p}} \beta_{k_1,
 \ldots, k_p} .\end{split}
\end{equation*}

\subsection{1.1.4 Interior product (\textbf{interiorProduct})}
\label{section-1.0:interior-product-interiorproduct}
The \emph{interior product} of a vectorfield \(v\) and a p-form \(\alpha\)
is a (p -1)-form \(i_v (\alpha)\) such that
\begin{equation*}
\begin{split}i_v (\alpha) (v_1, \ldots, v_{p - 1}) = \alpha (v, v_1, \ldots, v_{p - 1})\end{split}
\end{equation*}
holds, for all vectorfields \(v_1, \ldots, v_{p - 1}\). Therefore, the
components of \(i_v (\alpha)\) are
\begin{equation*}
\begin{split}i_v (\alpha)_{j_1, \ldots, j_{p - 1}} = v^j \alpha_{j, j_1, \ldots, j_{p -
  1} .}\end{split}
\end{equation*}
One can express the interior product by using the \(\star\)-operator.
Let \(\alpha\) be the 1-form defined by the equation
\begin{equation*}
\begin{split}\alpha (w) = g (v, w), \forall w.\end{split}
\end{equation*}
That means in components: \(\alpha_j = g_{j k} v^k\),
thus we have
\begin{equation*}
\begin{split}i_v (\beta) = (-)^{p - 1} \star^{- 1} (\alpha \wedge \star \beta) .\end{split}
\end{equation*}
Clearly, the interior product is independent of any metric, whereas the
Hodge operator is \textbf{not}! So, usually one should not use the Hodge
operator to compute the interior product.

We will use the fact that the interior product is an \emph{antiderivation},
which allows a recursive implementation.


\subsection{1.1.5 The Lie derivative (\textbf{lieDerivative})}
\label{section-1.0:the-lie-derivative-liederivative}
The \emph{Lie derivative} with respect to a vector field \(v\) can be
calculated (and defined) using Cartan's formula:
\begin{equation*}
\begin{split}\mathcal{L}_v \alpha = d i_v (\alpha) + i_v (d \alpha).\end{split}
\end{equation*}
There are other ways to define \(\mathcal{L}_v \alpha\), however,
it is convenient to compute it this way when \(d\) and \(i_v\) are
already at hand.


\subsection{1.1.6 The CoDifferential \protect\(\delta\protect\) (\textbf{codifferential})}
\label{section-1.0:the-codifferential-codifferential}
The \emph{codifferential} \(\delta\) is defined on a p-form as follows:
\begin{equation*}
\begin{split}\delta = (-1)^{n(p-1)+1}\,s(g) \star\,d\,\star\end{split}
\end{equation*}
where \(g\) is the metric and \(s(g)\) is related to the
\textbf{signature} of \(s(g)\) as described next.


\subsection{1.1.7 The sign of a metric  \protect\(s(g)\protect\) (\textbf{s})}
\label{section-1.0:the-sign-of-a-metric-s}
The signature of a metric \(g\) is defined as the difference of
the number of positive (p) and negative (q) eigenvalues, i.e:
\begin{equation*}
\begin{split}\mathrm{signature(g)} = p - q\end{split}
\end{equation*}
and the \textbf{sign} functions \textbf{s} is defined as:
\begin{equation*}
\begin{split}s(g) = (-1)^{\frac{n -\mathrm{signature(g)}}{2}}\end{split}
\end{equation*}
Since, as we always assume, \(g\)  is non-degenerate, we have
\(p+q=n\), and consequently:
\begin{equation*}
\begin{split}s(g) = (-1)^{q} = \mathrm{sign}\, \mathrm{det}(g)\end{split}
\end{equation*}

\subsection{1.1.8 The inverse Hodge star \protect\(\star^{-1}\protect\) (\textbf{invHodgeStar})}
\label{section-1.0:the-inverse-hodge-star-invhodgestar}
Applying the Hodge star operator twice on a p-form twice we get
the identity up to sign:
\begin{equation*}
\begin{split}\star\circ\star\, \omega_p = (-1)^{p(n-p)}\,s(g)\,\omega_p.\end{split}
\end{equation*}
Therefore
\begin{equation*}
\begin{split}\star^{-1}\,\omega_p = (-1)^{p(n-p)}\,s(g)\,\star\omega_p.\end{split}
\end{equation*}

\subsection{1.1.9 The Hodge-Laplacian \protect\(\Delta_g\protect\) (\textbf{hodgeLaplacian})}
\label{section-1.0:the-hodge-laplacian-hodgelaplacian}
The \textbf{Hodge-Laplacian} also known as \emph{Laplace-de Rham operator} is
defined on any manifold equipped with a (pseudo-) Riemannian
metric \(g\) and is given by
\begin{equation*}
\begin{split}\Delta_g = d\circ\delta + \delta\circ d\end{split}
\end{equation*}
Note that in the Euclidean case \(\Delta_g = - \Delta\), where
latter is the ordinary Laplacian.


\section{Bibliography}
\label{section-1.0:bibliography}\paragraph{References}


\chapter{2 Export}
\label{section-2.0:export}\label{section-2.0::doc}

\section{2.0 Package Details}
\label{section-2.0:package-details}\begin{quote}\begin{description}
\item[{Package Name}] \leavevmode
DifferentialForms

\item[{Abbreviation}] \leavevmode
DFORM

\item[{Source files}] \leavevmode
dform.spad

\item[{Dependent on}] \leavevmode
DeRhamComplex (DERHAM)

\end{description}\end{quote}

\begin{Verbatim}[commandchars=\\\{\}]
\PYG{n}{DifferentialForms}\PYG{p}{(}\PYG{n}{R}\PYG{p}{,}\PYG{n}{v}\PYG{p}{)}

\PYG{n}{R}\PYG{p}{:} \PYG{n}{Join}\PYG{p}{(}\PYG{n}{Ring}\PYG{p}{,}\PYG{n}{Comparable}\PYG{p}{)}   \PYG{o}{\PYGZhy{}}\PYG{o}{\PYGZhy{}} \PYG{n}{e}\PYG{o}{.}\PYG{n}{g}\PYG{o}{.} \PYG{n}{Integer}
\PYG{n}{v}\PYG{p}{:} \PYG{n}{List} \PYG{n}{Symbol}             \PYG{o}{\PYGZhy{}}\PYG{o}{\PYGZhy{}} \PYG{n}{e}\PYG{o}{.}\PYG{n}{g}\PYG{o}{.} \PYG{p}{[}\PYG{n}{x}\PYG{p}{,}\PYG{n}{y}\PYG{p}{,}\PYG{n}{z}\PYG{p}{]} \PYG{o+ow}{or} \PYG{p}{[}\PYG{n}{x}\PYG{p}{[}\PYG{l+m+mi}{0}\PYG{p}{]}\PYG{p}{,}\PYG{n}{x}\PYG{p}{[}\PYG{l+m+mi}{1}\PYG{p}{]}\PYG{p}{,}\PYG{n}{x}\PYG{p}{[}\PYG{l+m+mi}{2}\PYG{p}{]}\PYG{p}{]}

\PYG{n}{X} \PYG{o}{==}\PYG{o}{\PYGZgt{}} \PYG{n}{Expression} \PYG{n}{R}         \PYG{o}{\PYGZhy{}}\PYG{o}{\PYGZhy{}} \PYG{n}{function} \PYG{n}{Ring}
\end{Verbatim}

For the examples following, we choose \sphinxcode{R=Integer} and
\sphinxcode{v={[}x{[}0{]},x{[}1{]},x{[}2{]},x{[}3{]}{]}}, and the abbreviation \sphinxcode{M:=DFORM(R,v)}.


\section{2.1 The metric g}
\label{section-2.0:the-metric-g}
Some functions expect the metric \sphinxcode{g} as a parameter. Generally this
will be provided by an invertible square matrix \sphinxcode{g:SquareMatrix(\#v,X)}.

For the examples following, we choose the Minkowski metric::

\begin{Verbatim}[commandchars=\\\{\}]
\PYG{n}{g} \PYG{p}{:}\PYG{o}{=} \PYG{n}{diagonalMatrix}\PYG{p}{(}\PYG{p}{[}\PYG{o}{\PYGZhy{}}\PYG{l+m+mi}{1}\PYG{p}{,}\PYG{l+m+mi}{1}\PYG{p}{,}\PYG{l+m+mi}{1}\PYG{p}{,}\PYG{l+m+mi}{1}\PYG{p}{]}\PYG{p}{)}\PYG{n+nd}{@SquareMatrix}\PYG{p}{(}\PYG{l+m+mi}{4}\PYG{p}{,}\PYG{n}{Integer}\PYG{p}{)}
\end{Verbatim}
\begin{equation*}
\begin{split}\left[
\begin{array}{cccc}
-1 & 0 & 0 & 0 \\
0 & 1 & 0 & 0 \\
0 & 0 & 1 & 0 \\
0 & 0 & 0 & 1
\end{array}
\right]\end{split}
\end{equation*}
$_{\text{Type: SquareMatrix(4,Integer)}}$


\section{2.2 Exported Functions}
\label{section-2.0:exported-functions}
The function \textbf{baseForms} return the basis one forms, while \textbf{coordVector}
returns a list of the coordinates. The function \textbf{coordSymbols} also returns
the coordinates, however, as symbols only (convenient when used by \textbf{D}).

\begin{Verbatim}[commandchars=\\\{\}]
SMR ==\PYGZgt{} SquareMatrix(\PYGZsh{}v,X)
DRC ==\PYGZgt{} DeRhamComplex(R,v)

dx:=baseForms()\PYGZdl{}M     \PYGZhy{}\PYGZhy{} [dx[0],...,dx[3]]
x:=coordVector()\PYGZdl{}M    \PYGZhy{}\PYGZhy{} [x[0],...,x[3]]
xs:=coordSymbols()\PYGZdl{}M  \PYGZhy{}\PYGZhy{} as above but as List Symbol (for differentiation)
\end{Verbatim}


\subsection{2.2.1 Volume Form}
\label{section-2.0:volume-form}\begin{description}
\item[{\textbf{volumeForm}}] \leavevmode
Given a metric \(g\) the function returns the corresponding volume
element of the Riemannian (pseudo-) manifold.

\end{description}

\begin{Verbatim}[commandchars=\\\{\}]
volumeForm : SquareMatrix(\PYGZsh{}v,X) \PYGZhy{}\PYGZgt{} DeRhamComplex(R,v)

volumeForm(g)\PYGZdl{}M
\end{Verbatim}
\begin{equation*}
\begin{split}{dx _ {0}} \  {dx _ {1}} \  {dx _ {2}} \  {dx _ {3}}\end{split}
\end{equation*}
$_{\text{Type: DeRhamComplex(Integer,{[}x{[}0{]},x{[}1{]},x{[}2{]},x{[}3{]}{]})}}$


\subsection{2.2.1 Scalar Product}
\label{section-2.0:scalar-product}\begin{description}
\item[{\textbf{dot}}] \leavevmode
Compute the inner product of two differential forms with respect to the
metric \sphinxcode{g}.

\end{description}

\begin{Verbatim}[commandchars=\\\{\}]
dot : (SMR,DRC,DRC) \PYGZhy{}\PYGZgt{} X

dot(g,dx.1*dx.2,dx.1*dx.2)\PYGZdl{}M   \PYGZhy{}\PYGZhy{} note dx.1 corresponds to dx[0].
\end{Verbatim}
\begin{equation*}
\begin{split}- 1\end{split}
\end{equation*}
$_{\text{Type: Expression(Integer)}}$


\subsection{2.2.2 Hodge Star Operator}
\label{section-2.0:hodge-star-operator}\begin{description}
\item[{\textbf{hodgeStar}}] \leavevmode
Compute the Hodge dual form of a differential form with respect
to a metric \sphinxcode{g}.

\end{description}

\begin{Verbatim}[commandchars=\\\{\}]
\PYG{n}{hodgeStar} \PYG{p}{:} \PYG{p}{(}\PYG{n}{SMR}\PYG{p}{,}\PYG{n}{DRC}\PYG{p}{)} \PYG{o}{\PYGZhy{}}\PYG{o}{\PYGZgt{}} \PYG{n}{DRC}

\PYG{n}{hodgeStar}\PYG{p}{(}\PYG{n}{g}\PYG{p}{,}\PYG{n}{dx}\PYG{o}{.}\PYG{l+m+mi}{2} \PYG{o}{*} \PYG{n}{dx}\PYG{o}{.}\PYG{l+m+mi}{3}\PYG{p}{)}
\end{Verbatim}
\begin{equation*}
\begin{split}{dx _ {0}} \  {dx _ {3}}\end{split}
\end{equation*}
$_{\text{Type: DeRhamComplex(Integer,{[}x{[}0{]},x{[}1{]},x{[}2{]},x{[}3{]}{]})}}$


\subsection{2.2.3 Interior Product}
\label{section-2.0:interior-product}\begin{description}
\item[{\textbf{interiorProduct}}] \leavevmode
Calculate the interior product \(i_X(a)\) of the vector field X
with the differential form a.

\end{description}

\begin{Verbatim}[commandchars=\\\{\}]
interiorProduct : (Vector(X),DRC) \PYGZhy{}\PYGZgt{} DRC

interiorProduct(vector x, dx.1*dx.3)\PYGZdl{}M
\end{Verbatim}
\begin{equation*}
\begin{split}{{x _ {0}} \  {dx _ {2}}} -{{x _ {2}} \  {dx _ {0}}}\end{split}
\end{equation*}
$_{\text{Type: DeRhamComplex(Integer,{[}x{[}0{]},x{[}1{]},x{[}2{]},x{[}3{]}{]})}}$


\subsection{2.2.4 Lie Derivative}
\label{section-2.0:lie-derivative}\begin{description}
\item[{\textbf{lieDerivative}}] \leavevmode
Calculates the Lie derivative \(\mathcal{L}_X(a)\) of the differential
form a with respect to the vector field X.

\end{description}

\begin{Verbatim}[commandchars=\\\{\}]
\PYG{n}{lieDerivative} \PYG{p}{:} \PYG{p}{(}\PYG{n}{Vector}\PYG{p}{(}\PYG{n}{X}\PYG{p}{)}\PYG{p}{,}\PYG{n}{DRC}\PYG{p}{)} \PYG{o}{\PYGZhy{}}\PYG{o}{\PYGZgt{}} \PYG{n}{DRC}

\PYG{n}{lieDerivative}\PYG{p}{(}\PYG{n}{vector} \PYG{n}{x}\PYG{p}{,} \PYG{n}{dx}\PYG{o}{.}\PYG{l+m+mi}{1} \PYG{o}{*} \PYG{n}{dx}\PYG{o}{.}\PYG{l+m+mi}{3} \PYG{o}{*} \PYG{n}{dx}\PYG{o}{.}\PYG{l+m+mi}{4}\PYG{p}{)}
\end{Verbatim}
\begin{equation*}
\begin{split}3 \  {dx _ {0}} \  {dx _ {2}} \  {dx _ {3}}\end{split}
\end{equation*}
$_{\text{Type: DeRhamComplex(Integer,{[}x{[}0{]},x{[}1{]},x{[}2{]},x{[}3{]}{]})}}$


\subsection{2.2.5 Projection}
\label{section-2.0:projection}\begin{description}
\item[{\textbf{proj}}] \leavevmode
Project to homogeneous terms of degree p.

\end{description}

\begin{Verbatim}[commandchars=\\\{\}]
\PYG{n}{NNI} \PYG{o}{==}\PYG{o}{\PYGZgt{}} \PYG{n}{NonNegativeInteger}
\PYG{n}{proj} \PYG{p}{:} \PYG{p}{(}\PYG{n}{NNI}\PYG{p}{,}\PYG{n}{DRC}\PYG{p}{)} \PYG{o}{\PYGZhy{}}\PYG{o}{\PYGZgt{}} \PYG{n}{DRC}

\PYG{n}{proj}\PYG{p}{(}\PYG{l+m+mi}{2}\PYG{p}{,} \PYG{l+m+mi}{2}\PYG{o}{*}\PYG{n}{dx}\PYG{o}{.}\PYG{l+m+mi}{1} \PYG{o}{+} \PYG{n}{dx}\PYG{o}{.}\PYG{l+m+mi}{2}\PYG{o}{*}\PYG{n}{dx}\PYG{o}{.}\PYG{l+m+mi}{3} \PYG{o}{\PYGZhy{}} \PYG{n}{dx}\PYG{o}{.}\PYG{l+m+mi}{3}\PYG{o}{*}\PYG{n}{dx}\PYG{o}{.}\PYG{l+m+mi}{4}\PYG{p}{)}
\end{Verbatim}
\begin{equation*}
\begin{split}-{{dx _ {2}} \  {dx _ {3}}}+{{dx _ {1}} \  {dx _ {2}}}\end{split}
\end{equation*}
$_{\text{Type: DeRhamComplex(Integer,{[}x{[}0{]},x{[}1{]},x{[}2{]},x{[}3{]}{]})}}$


\subsection{2.2.6 Monomials}
\label{section-2.0:monomials}\begin{description}
\item[{\textbf{monomials}}] \leavevmode
List all monomials of degree p (p in 1..n).
This is a basis for \(\Lambda_p^n\).

\end{description}

\begin{Verbatim}[commandchars=\\\{\}]
monomials : NNI \PYGZhy{}\PYGZgt{} List DRC

monomials(3)\PYGZdl{}M
\end{Verbatim}
\begin{equation*}
\begin{split}\left[
{{dx _ {0}} \  {dx _ {1}} \  {dx _ {2}}}, \: {{dx _ {0}} \  {dx _
{1}} \  {dx _ {3}}}, \: {{dx _ {0}} \  {dx _ {2}} \  {dx _ {3}}}, \:
{{dx _ {1}} \  {dx _ {2}} \  {dx _ {3}}}
\right]\end{split}
\end{equation*}
$_{\text{Type: List DeRhamComplex(Integer,{[}x{[}0{]},x{[}1{]},x{[}2{]},x{[}3{]}{]})}}$


\subsection{2.2.7 Atomize Basis Term}
\label{section-2.0:atomize-basis-term}\begin{description}
\item[{\textbf{atomizeBasisTerm}}] \leavevmode
Given a basis term, return a list of the generators (atoms).

\end{description}

\begin{Verbatim}[commandchars=\\\{\}]
\PYG{n}{atomizeBasisTerm} \PYG{p}{:} \PYG{n}{DRC} \PYG{o}{\PYGZhy{}}\PYG{o}{\PYGZgt{}} \PYG{n}{List} \PYG{n}{DRC}

\PYG{n}{atomizeBasisTerm}\PYG{p}{(}\PYG{n}{dx}\PYG{o}{.}\PYG{l+m+mi}{1} \PYG{o}{*} \PYG{n}{dx}\PYG{o}{.}\PYG{l+m+mi}{2} \PYG{o}{*} \PYG{n}{dx}\PYG{o}{.}\PYG{l+m+mi}{4}\PYG{p}{)}
\end{Verbatim}
\begin{equation*}
\begin{split}\left[
 {dx _ {0}}, \: {dx _ {1}}, \: {dx _ {3}}
\right]\end{split}
\end{equation*}
$_{\text{Type: List(DeRhamComplex(Integer,{[}x{[}0{]},x{[}1{]},x{[}2{]},x{[}3{]}{]}))}}$


\subsection{2.2.8 Conjugate Basis Term}
\label{section-2.0:conjugate-basis-term}\begin{description}
\item[{\textbf{conjBasisTerm}}] \leavevmode
Return the complement of a basis term with respect to the Euclidean
volume form.

\end{description}

\begin{Verbatim}[commandchars=\\\{\}]
\PYG{n}{conjBasisTerm} \PYG{p}{:} \PYG{n}{DRC} \PYG{o}{\PYGZhy{}}\PYG{o}{\PYGZgt{}} \PYG{n}{DRC}

\PYG{n}{conjBasisTerm} \PYG{n}{dx}\PYG{o}{.}\PYG{l+m+mi}{4}
\end{Verbatim}
\begin{equation*}
\begin{split}{dx _ {0}} \  {dx _ {1}} \  {dx _ {2}}\end{split}
\end{equation*}
$_{\text{Type: DeRhamComplex(Integer,{[}x{[}0{]},x{[}1{]},x{[}2{]},x{[}3{]}{]})}}$


\subsection{2.2.9 Scalar and Vector Field}
\label{section-2.0:scalar-and-vector-field}\begin{description}
\item[{\textbf{vectorField}}] \leavevmode
Generate a generic vector field named by a given symbol.

\item[{\textbf{covectorField}}] \leavevmode
Generate a generic co-vector field named by a given symbol.

\item[{\textbf{scalarField}}] \leavevmode
Generate a generic scalar field named by a given symbol.

\end{description}

\begin{Verbatim}[commandchars=\\\{\}]
vectorField   : Symbol \PYGZhy{}\PYGZgt{} List X
covectorField : Symbol \PYGZhy{}\PYGZgt{} List DRC
scalarField   : Symbol \PYGZhy{}\PYGZgt{} X

vectorField(Q)\PYGZdl{}M
scalarField(f)\PYGZdl{}M
\end{Verbatim}
\begin{equation*}
\begin{split}\left[
{{Q _ {1}}
\left(
{{x _ {0}}, \: {x _ {1}}, \: {x _ {2}}, \: {x _ {3}}}
\right)},
\: {{Q _ {2}}
\left(
{{x _ {0}}, \: {x _ {1}}, \: {x _ {2}}, \: {x _ {3}}}
\right)},
\: {{Q _ {3}}
\left(
{{x _ {0}}, \: {x _ {1}}, \: {x _ {2}}, \: {x _ {3}}}
\right)},
\: {{Q _ {4}}
\left(
{{x _ {0}}, \: {x _ {1}}, \: {x _ {2}}, \: {x _ {3}}}
\right)}
\right]\end{split}
\end{equation*}
$_{\text{Type: List(Expression(Integer))}}$
\begin{equation*}
\begin{split}f
\left(
 {{x _ {0}}, \: {x _ {1}}, \: {x _ {2}}, \: {x _ {3}}}
\right)\end{split}
\end{equation*}
$_{\text{Type: Expression(Integer)}}$


\subsection{2.2.10 Miscellaneous Functions}
\label{section-2.0:miscellaneous-functions}
A \emph{zero form} with symbol s can be generated by

\begin{Verbatim}[commandchars=\\\{\}]
zeroForm : Symbol \PYGZhy{}\PYGZgt{} DRC

zeroForm(s)\PYGZdl{}M
\end{Verbatim}
\begin{equation*}
\begin{split}s
\left(
 {{x _ {0}}, \: {x _ {1}}, \: {x _ {2}}, \: {x _ {3}}}
\right)\end{split}
\end{equation*}
$_{\text{Type: DeRhamComplex(Integer,{[}x{[}0{]},x{[}1{]},x{[}2{]},x{[}3{]}{]})}}$

A synonym for the \textbf{exteriorDerivative} is the common operator \textbf{d}:

\begin{Verbatim}[commandchars=\\\{\}]
d : DRC \PYGZhy{}\PYGZgt{} DRC

d zeroForm(f)\PYGZdl{}M
\end{Verbatim}
\begin{equation*}
\begin{split}{{{f _ {{,4}}}
\left(
{{x _ {0}}, \: {x _ {1}}, \: {x _ {2}}, \: {x _ {3}}}
\right)}
\  {dx _ {3}}}+{{{f _ {{,3}}}
\left(
{{x _ {0}}, \: {x _ {1}}, \: {x _ {2}}, \: {x _ {3}}}
\right)}
\  {dx _ {2}}}+ \\ {{{f _ {{,2}}}
\left(
{{x _ {0}}, \: {x _ {1}}, \: {x _ {2}}, \: {x _ {3}}}
\right)}
\  {dx _ {1}}}+{{{f _ {{,1}}}
\left(
{{x _ {0}}, \: {x _ {1}}, \: {x _ {2}}, \: {x _ {3}}}
\right)}
\  {dx _ {0}}}\end{split}
\end{equation*}
The special zero forms \textbf{0} and \textbf{1} can be generated by

\begin{Verbatim}[commandchars=\\\{\}]
one :  \PYGZhy{}\PYGZgt{} DRC
zero : \PYGZhy{}\PYGZgt{} DRC

zero()\PYGZdl{}M
one()\PYGZdl{}M
\end{Verbatim}

There are also some special multiplication operators which allow to deal
with a kind \emph{vector valued} forms (actually lists):

\begin{Verbatim}[commandchars=\\\{\}]
\PYG{n}{\PYGZus{}}\PYG{o}{*} \PYG{p}{:} \PYG{p}{(}\PYG{n}{List} \PYG{n}{X}\PYG{p}{,} \PYG{n}{List} \PYG{n}{DRC}\PYG{p}{)}   \PYG{o}{\PYGZhy{}}\PYG{o}{\PYGZgt{}} \PYG{n}{DRC}
\PYG{n}{\PYGZus{}}\PYG{o}{*} \PYG{p}{:} \PYG{p}{(}\PYG{n}{List} \PYG{n}{DRC}\PYG{p}{,} \PYG{n}{List} \PYG{n}{DRC}\PYG{p}{)} \PYG{o}{\PYGZhy{}}\PYG{o}{\PYGZgt{}} \PYG{n}{DRC}

\PYG{n}{Note}\PYG{p}{:} \PYG{n}{the} \PYG{n}{lists} \PYG{n}{must} \PYG{n}{have} \PYG{n}{dimension} \PYG{c+c1}{\PYGZsh{}v.}

\PYG{n}{For} \PYG{n}{instance}\PYG{p}{:}

\PYG{n}{x} \PYG{o}{*} \PYG{n}{dx}
\end{Verbatim}
\begin{equation*}
\begin{split}{{x _ {3}} \  {dx _ {3}}}+{{x _ {2}} \  {dx _ {2}}}+{{x _ {1}} \
{dx _ {1}}}+{{x _ {0}} \  {dx _ {0}}}\end{split}
\end{equation*}
An example for the second case:

\begin{Verbatim}[commandchars=\\\{\}]
dx*[hodgeStar(g,dx.j)\PYGZdl{}M for j in 1..4]
\end{Verbatim}
\begin{equation*}
\begin{split}2 \  {dx _ {0}} \  {dx _ {1}} \  {dx _ {2}} \  {dx _ {3}}\end{split}
\end{equation*}
$_{\text{Type: DeRhamComplex(Integer,{[}x{[}0{]},x{[}1{]},x{[}2{]},x{[}3{]}{]})}}$


\chapter{3 Implementation}
\label{section-3.0:implementation}\label{section-3.0::doc}

\section{3.0 Implementation Notes}
\label{section-3.0:implementation-notes}
In this section some implementation details will be described. This is a
ongoing process and might be improved.


\subsection{3.1 Internal Representation}
\label{section-3.0:internal-representation}
Differential forms are represented as \sphinxcode{List Record(k:EAB, c:R)} where each
basic term is represented as \sphinxcode{Record(k:EAB, c:R)}, for instance:
\begin{equation*}
\begin{split}h (x, y, z)\ dz + g (x, y, z)\ dy + f (x, y, z)\ dx\end{split}
\end{equation*}
maps to

\begin{Verbatim}[commandchars=\\\{\}]
\PYG{p}{[}\PYG{p}{[}\PYG{n}{k}\PYG{o}{=} \PYG{p}{[}\PYG{l+m+mi}{0}\PYG{p}{,}\PYG{l+m+mi}{0}\PYG{p}{,}\PYG{l+m+mi}{1}\PYG{p}{]}\PYG{p}{,}\PYG{n}{c}\PYG{o}{=} \PYG{n}{h}\PYG{p}{(}\PYG{n}{x}\PYG{p}{,}\PYG{n}{y}\PYG{p}{,}\PYG{n}{z}\PYG{p}{)}\PYG{p}{]}\PYG{p}{,}\PYG{p}{[}\PYG{n}{k}\PYG{o}{=} \PYG{p}{[}\PYG{l+m+mi}{0}\PYG{p}{,}\PYG{l+m+mi}{1}\PYG{p}{,}\PYG{l+m+mi}{0}\PYG{p}{]}\PYG{p}{,}\PYG{n}{c}\PYG{o}{=} \PYG{n}{g}\PYG{p}{(}\PYG{n}{x}\PYG{p}{,}\PYG{n}{y}\PYG{p}{,}\PYG{n}{z}\PYG{p}{)}\PYG{p}{]}\PYG{p}{,}\PYG{p}{[}\PYG{n}{k}\PYG{o}{=} \PYG{p}{[}\PYG{l+m+mi}{1}\PYG{p}{,}\PYG{l+m+mi}{0}\PYG{p}{,}\PYG{l+m+mi}{0}\PYG{p}{]}\PYG{p}{,}\PYG{n}{c}\PYG{o}{=} \PYG{n}{f}\PYG{p}{(}\PYG{n}{x}\PYG{p}{,}\PYG{n}{y}\PYG{p}{,}\PYG{n}{z}\PYG{p}{)}\PYG{p}{]}\PYG{p}{]}
\end{Verbatim}

or, another example:
\begin{equation*}
\begin{split}c (x, y, z)\ dy\ dz + b (x, y, z)\ dx\ dz + a (x, y, z)\ dx\ dy\end{split}
\end{equation*}
goes to

\begin{Verbatim}[commandchars=\\\{\}]
\PYG{p}{[}\PYG{p}{[}\PYG{n}{k}\PYG{o}{=} \PYG{p}{[}\PYG{l+m+mi}{0}\PYG{p}{,}\PYG{l+m+mi}{1}\PYG{p}{,}\PYG{l+m+mi}{1}\PYG{p}{]}\PYG{p}{,}\PYG{n}{c}\PYG{o}{=} \PYG{n}{c}\PYG{p}{(}\PYG{n}{x}\PYG{p}{,}\PYG{n}{y}\PYG{p}{,}\PYG{n}{z}\PYG{p}{)}\PYG{p}{]}\PYG{p}{,}\PYG{p}{[}\PYG{n}{k}\PYG{o}{=} \PYG{p}{[}\PYG{l+m+mi}{1}\PYG{p}{,}\PYG{l+m+mi}{0}\PYG{p}{,}\PYG{l+m+mi}{1}\PYG{p}{]}\PYG{p}{,}\PYG{n}{c}\PYG{o}{=} \PYG{n}{b}\PYG{p}{(}\PYG{n}{x}\PYG{p}{,}\PYG{n}{y}\PYG{p}{,}\PYG{n}{z}\PYG{p}{)}\PYG{p}{]}\PYG{p}{,}\PYG{p}{[}\PYG{n}{k}\PYG{o}{=} \PYG{p}{[}\PYG{l+m+mi}{1}\PYG{p}{,}\PYG{l+m+mi}{1}\PYG{p}{,}\PYG{l+m+mi}{0}\PYG{p}{]}\PYG{p}{,}\PYG{n}{c}\PYG{o}{=} \PYG{n}{a}\PYG{p}{(}\PYG{n}{x}\PYG{p}{,}\PYG{n}{y}\PYG{p}{,}\PYG{n}{z}\PYG{p}{)}\PYG{p}{]}\PYG{p}{]}
\end{Verbatim}

It is easily seen that for n generators \(x_1, \ldots, x_n\) the term
\(d x_{j_p} \wedge \ldots \wedge d x_{j_q}\) is represented by
\([k = [a_{i_1}, \ldots, a_{i_n}], c = \pm 1]\)
where \(a_{i_s} \in \{ 0, 1 \}\) depending on whether
\(d x_{i_s}\) is contained in the term or not. The (local) function
\textbf{terms} sends a differential form \(\omega\) to the representation
\(r (\omega)\).
Note that the operations are destructive, this means that one has to copy the
objects in order to get new ones (mere assignment inherits all previous
changes).
\begin{description}
\item[{\textbf{Note}}] \leavevmode
The interpreter normalizes the basic terms according to increasing
generators, i.e for example: \(d x_3 \wedge d x_2\) will be stored
as \(- d x_2 \wedge d x_3\), whereby the signum of the permutation is
calculated and transferred to the \sphinxcode{c}-field in the record.

\end{description}

Example:

\begin{Verbatim}[commandchars=\\\{\}]
\PYG{n}{terms}\PYG{p}{(}\PYG{n}{dx3}\PYG{o}{*}\PYG{n}{dx2}\PYG{p}{)} \PYG{o}{\PYGZhy{}}\PYG{o}{\PYGZgt{}} \PYG{p}{[}\PYG{p}{[}\PYG{n}{k}\PYG{o}{=} \PYG{p}{[}\PYG{l+m+mi}{0}\PYG{p}{,}\PYG{l+m+mi}{1}\PYG{p}{,}\PYG{l+m+mi}{1}\PYG{p}{]}\PYG{p}{,}\PYG{n}{c}\PYG{o}{=} \PYG{o}{\PYGZhy{}} \PYG{l+m+mi}{1}\PYG{p}{]}\PYG{p}{]}
\end{Verbatim}


\subsection{3.2 dot :: inner product}
\label{section-3.0:dot-inner-product}
Given a (pseudo)-Riemannian metric g, the scalar product of two basic terms of
the same degree is given by
\begin{equation*}
\begin{split}\langle d x_{i_1} \ldots d x_{i_p}, d x_{j_1} \ldots d x_{j_p} \rangle =
  \det (\langle d x_{i_k} , d x_{j_l} \rangle) = \det (g^{- 1} (i_k,
  j_l)),\end{split}
\end{equation*}
whereby \(1 \leqslant k, l \leqslant p\). Note that
\(g^{- 1} (i_k, j_l) = g^{i_k j_l}\) is the inverse of
\(g_{i_k j_l}\) (raised indexes as usual). In other words, the scalar
product is inherited from the dual vector space of the space where the
coordinates \((x_1, \ldots, x_n)\) live, and is continued by linearity.
By the way, terms of different \textbf{degree} are considered to be
\emph{orthogonal} to each other.

Example:

\begin{Verbatim}[commandchars=\\\{\}]
\PYG{p}{[}\PYG{n}{x}\PYG{p}{,}\PYG{n}{y}\PYG{p}{,}\PYG{n}{z}\PYG{p}{]}\PYG{p}{,} \PYG{n}{G} \PYG{o}{=} \PYG{n}{matrix}\PYG{p}{(}\PYG{n}{G}\PYG{p}{[}\PYG{n}{i}\PYG{p}{,}\PYG{n}{j}\PYG{p}{]}\PYG{p}{)} \PYG{o}{=} \PYG{n}{g}\PYG{o}{\PYGZca{}}\PYG{p}{\PYGZob{}}\PYG{o}{\PYGZhy{}} \PYG{l+m+mi}{1}\PYG{p}{\PYGZcb{}}\PYG{p}{,}
\end{Verbatim}
\begin{equation*}
\begin{split}\langle d x \wedge d y, d y \wedge d z \rangle = \langle d x, d y \rangle
  \langle d y, d z \rangle - \langle d x, d z \rangle \langle d y, d y
  \rangle = \\ {G[1,2] G[2, 3] - G[1, 3] G[2, 2]}\end{split}
\end{equation*}
The corresponding EAB's are \sphinxcode{{[}1,1,0{]}} and \sphinxcode{{[}0,1,1{]}}. If we define a
function \textbf{pos} which gives the positions of 0 or 1 respectively, the example
tells us:

\begin{Verbatim}[commandchars=\\\{\}]
\PYG{n}{pos}\PYG{p}{(}\PYG{p}{[}\PYG{l+m+mi}{1}\PYG{p}{,}\PYG{l+m+mi}{1}\PYG{p}{,}\PYG{l+m+mi}{0}\PYG{p}{]}\PYG{p}{,}\PYG{l+m+mi}{1}\PYG{p}{)}\PYG{o}{=}\PYG{p}{[}\PYG{l+m+mi}{1}\PYG{p}{,}\PYG{l+m+mi}{2}\PYG{p}{]} \PYG{o+ow}{and} \PYG{n}{pos}\PYG{p}{(}\PYG{p}{[}\PYG{l+m+mi}{0}\PYG{p}{,}\PYG{l+m+mi}{1}\PYG{p}{,}\PYG{l+m+mi}{1}\PYG{p}{]}\PYG{p}{,}\PYG{l+m+mi}{1}\PYG{p}{)}\PYG{o}{=}\PYG{p}{[}\PYG{l+m+mi}{2}\PYG{p}{,}\PYG{l+m+mi}{3}\PYG{p}{]}
\end{Verbatim}

so that the direct product of the two resulting lists gives the desired minor:

\begin{Verbatim}[commandchars=\\\{\}]
\PYG{p}{[}\PYG{l+m+mi}{1}\PYG{p}{,}\PYG{l+m+mi}{2}\PYG{p}{]}\PYG{n}{x}\PYG{p}{[}\PYG{l+m+mi}{2}\PYG{p}{,}\PYG{l+m+mi}{3}\PYG{p}{]}\PYG{o}{=}\PYG{p}{[}\PYG{p}{[}\PYG{l+m+mi}{1}\PYG{p}{,}\PYG{l+m+mi}{2}\PYG{p}{]}\PYG{p}{,}\PYG{p}{[}\PYG{l+m+mi}{1}\PYG{p}{,}\PYG{l+m+mi}{3}\PYG{p}{]}\PYG{p}{,}\PYG{p}{[}\PYG{l+m+mi}{2}\PYG{p}{,}\PYG{l+m+mi}{2}\PYG{p}{]}\PYG{p}{,}\PYG{p}{[}\PYG{l+m+mi}{2}\PYG{p}{,}\PYG{l+m+mi}{3}\PYG{p}{]}\PYG{p}{]} \PYG{o}{=}\PYG{o}{\PYGZgt{}}
\end{Verbatim}
\begin{equation*}
\begin{split} \left|\begin{array}{c}
  G_{12} G_{1 3}\\
  G_{2 2} G_{2 3}
\end{array}\right| .\end{split}
\end{equation*}
This essentially comprises the method we will use to compute the scalar product
w.r.t symmetric matrices g and two basic terms of equal degree.

Local function: \textbf{dot2}
\begin{itemize}
\item {} 
compute the inverse of tmverbatim\{g\}.

\item {} 
build the tmverbatim\{pos\} lists.

\item {} 
build the minor and apply \sphinxcode{determinant}.

\end{itemize}

Actually there are two functions \textbf{dot1} and \textbf{dot2}, where the former
is used when the metric g is \emph{diagonal} (which is equivalent to the
basis vectors being orthogonal) because the performance might be better if the
dimension of the space is huge.


\subsection{3.3 hodgeStar :: Hodge dual}
\label{section-3.0:hodgestar-hodge-dual}
In this new version we have removed the first method (3.3.1) because the
performance difference is not as significant as we thought in the first
place, at least not for \(n\leq 7\). However, we save the method in
case someone has to deal with really high space dimensions.


\subsubsection{3.3.1 Diagonal, non-degenerated g}
\label{section-3.0:diagonal-non-degenerated-g}
If g is a diagonal matrix then the components of \(\star \beta\) reduce to
\begin{equation*}
\begin{split}(\star \beta)_{j_1, \ldots, j_{n - p}} = \frac{1}{p!} \varepsilon_{k_1,
  \ldots, k_p, j_1, \ldots, j_{n - p}}  \sqrt{| \det g |}  [g^{k_1 k_1}
  \ldots g^{k_p k_p}] \beta_{k_1, \ldots, k_p}\end{split}
\end{equation*}
which implies that \((j_1, \ldots, j_{n - p})\) must be the complement of
\((k_1,\ldots, k_p)\) in \(\{ 1, 2, \ldots, n \}\) and
\begin{equation*}
\begin{split}\star (d x_{k_1} \wedge \ldots \wedge d x_{k_p}) = C d x_{j_1} \wedge
 \ldots \wedge d x_{j_{n - p}}\end{split}
\end{equation*}
for some (yet unknown) factor \sphinxcode{C} which must actually be equal to the right
hand side of the component formula above. When we recollect the internal
representation of \(d x_{k_1} \wedge \ldots \wedge d x_{k_p}\) as
EAB then it is easy to get the complement by flipping the
0 and 1. Define a function \textbf{flip} such that
\begin{equation*}
\begin{split}\mathrm{flip} (d x_{k_1} \wedge \ldots \wedge d x_{k_p}) = d x_{j_1}
\wedge \ldots \wedge d x_{j_{n - p}}\end{split}
\end{equation*}
then by using the Hodge formula with
\(\alpha = \beta = dx_{k_1} \wedge \ldots \wedge d x_{k_p}\) we get using
\(\star \alpha = C\ \mathrm{flip} (\alpha)\):
\begin{equation*}
\begin{split}C \alpha \wedge \mathrm{flip} (\alpha) = \langle \alpha, \alpha \rangle \eta
 = \langle \alpha, \alpha \rangle \sqrt{| \det (g) |} d x_1 \wedge \ldots
 \wedge d x_n .\end{split}
\end{equation*}
Since \(\alpha \wedge \mathrm{flip} (\alpha)\) is a n-form, the function
\textbf{leadingCoefficient} returns the one and only coefficient. Thus we
can calculate \sphinxcode{C} to
\begin{equation*}
\begin{split}C = \frac{\langle \alpha, \alpha \rangle \sqrt{| \det (g)
|}}{\mathtt{leadingCoefficient} (\alpha \wedge \mathrm{flip} (\alpha))} .\end{split}
\end{equation*}
In \textbf{SPAD} syntax this looks like:
\begin{equation*}
\begin{split}\mathtt{C =}  \frac{\mathtt{dot} (\alpha, \alpha) \star
 \mathtt{sqrt\left(abs\left(\right.determinant\left(g\right)\right)}}{\mathtt{leadingCoefficient}
 \left( \alpha \star \mathtt{flip} (\alpha) \right)} .\end{split}
\end{equation*}
This way the interpreter saved us the tedious computation of the permutation
signatures. Moreover, we have not to care whether the metric g is positive or
negative definite.


\subsubsection{3.3.2 General case}
\label{section-3.0:general-case}
Let \(J\) denote an ordered multi-index and \(J_\sharp\) its dual.
Then a generic p-vector may be written as
\begin{equation*}
\begin{split}\beta = \sum_{|J|=p} b^J \ e_J.\end{split}
\end{equation*}
Thus by definition we obtain:
\begin{equation*}
\begin{split}\alpha\wedge\star\beta=(\alpha,\beta)\,\eta \Rightarrow
e_J\wedge\star\beta=(e_J,\beta)\,\eta\end{split}
\end{equation*}
Since \(\star\beta\) is a (n-p)-form, we get:
\begin{equation*}
\begin{split}\star\beta=\sum_{|K|=n-p} a^K e_K \Rightarrow
\sum_{|K|=n-p} a^K e_J\wedge e_K=\sum_{|I|=p} b^I (e_J,e_I)=
\sum_{|I|=p} g_{JI} b^I \eta = b_J \eta.\end{split}
\end{equation*}
Now the term \(e_J\wedge e_K\) is non-zero only if \(K=J_\sharp\),
therefore
\begin{equation*}
\begin{split}a^{J_\sharp} =\sqrt{g}\, \epsilon(J)\, \sum_{|I|=p} g_{JI} b^I\end{split}
\end{equation*}
where \(e_J\wedge e_{J_\sharp}=\epsilon(J)\, \eta\ \) defines
\(\epsilon\).

If we choose \(\beta=e_M\) we finally get
\begin{equation*}
\begin{split}\star e_M = \sqrt{g} \sum_{|J|=p} \epsilon(J)\, g_{JM}\, e_{J_\sharp}.\end{split}
\end{equation*}
This formula will be used to compute the Hodge dual for \emph{monomials}. We define
a function \textbf{hodgeBT}, in pseudo-code:

\begin{Verbatim}[commandchars=\\\{\}]
\PYG{n}{hodgeStarBT}\PYG{p}{(}\PYG{n}{dx}\PYG{p}{[}\PYG{n}{M}\PYG{p}{]}\PYG{p}{)}\PYG{o}{=} \PYG{n}{sqrt}\PYG{p}{(}\PYG{n}{g}\PYG{p}{)}\PYG{o}{*}
     \PYG{n}{SUM}\PYG{p}{[}\PYG{n}{J}\PYG{p}{]} \PYG{p}{\PYGZob{}}\PYG{n}{eps}\PYG{p}{(}\PYG{n}{dx}\PYG{p}{[}\PYG{n}{J}\PYG{p}{]}\PYG{p}{)}\PYG{o}{*}\PYG{n}{dot}\PYG{p}{(}\PYG{n}{g}\PYG{p}{,}\PYG{n}{dx}\PYG{p}{[}\PYG{n}{J}\PYG{p}{]}\PYG{p}{,}\PYG{n}{dx}\PYG{p}{[}\PYG{n}{M}\PYG{p}{]}\PYG{p}{)}\PYG{o}{*}\PYG{n}{conjBasisTerm}\PYG{p}{(}\PYG{n}{dx}\PYG{p}{[}\PYG{n}{j}\PYG{p}{]}\PYG{p}{)}\PYG{p}{\PYGZcb{}}
\end{Verbatim}

which then allows to compute the Hodge dual of any form by simple recursion:

\begin{Verbatim}[commandchars=\\\{\}]
hodgeStar(g:SMR,x:DRC):DRC ==
  x=0\PYGZdl{}DRC =\PYGZgt{} x
  leadingCoefficient(x) * hodgeStarBT(g,leadingBasisTerm(x)) + \PYGZus{}
    hodgeStar(g, reductum(x))
\end{Verbatim}


\subsection{3.4 interiorProduct :: Interior product}
\label{section-3.0:interiorproduct-interior-product}
In this newer version we have replaced the method which uses the Hodge
operator. Instead we used the fact that the interior product is
an \emph{antiderivation}, actually the unique antiderivation of degree
\(-1\) on the exterior algebra such that \(i_X(\alpha)=\alpha(X)\):
\begin{equation*}
\begin{split}i_X(\beta\wedge\gamma)=i_X(\beta)\wedge\gamma)+
 (-1)^{{\mathtt deg}\, \beta}\ \beta\wedge i_X(\gamma)\end{split}
\end{equation*}
This also allows an easy implementation by recursion.


\subsection{3.5 lieDerivative :: Lie derivative}
\label{section-3.0:liederivative-lie-derivative}
Here we use \emph{Cartan's formula} (see 1.1.5), so that there is not much to
say.

\begin{Verbatim}[commandchars=\\\{\}]
\PYG{n}{lieDerivative}\PYG{p}{(}\PYG{n}{w}\PYG{p}{:}\PYG{n}{Vector} \PYG{n}{X}\PYG{p}{,}\PYG{n}{x}\PYG{p}{:}\PYG{n}{DRC}\PYG{p}{)}\PYG{p}{:}\PYG{n}{DRC} \PYG{o}{==}
  \PYG{n}{a} \PYG{p}{:}\PYG{o}{=} \PYG{n}{exteriorDifferential}\PYG{p}{(}\PYG{n}{interiorProduct}\PYG{p}{(}\PYG{n}{w}\PYG{p}{,}\PYG{n}{x}\PYG{p}{)}\PYG{p}{)}
  \PYG{n}{b} \PYG{p}{:}\PYG{o}{=} \PYG{n}{interiorProduct}\PYG{p}{(}\PYG{n}{w}\PYG{p}{,} \PYG{n}{exteriorDifferential}\PYG{p}{(}\PYG{n}{x}\PYG{p}{)}\PYG{p}{)}
  \PYG{n}{a}\PYG{o}{+}\PYG{n}{b}
\end{Verbatim}


\subsection{3.6 proj :: Projection}
\label{section-3.0:proj-projection}
Since the elements of \(\mathtt{DeRhamComplex}\) are in
\begin{equation*}
\begin{split}X = \bigoplus_{p = 0}^n \Omega^p (V)\end{split}
\end{equation*}
it is convenient to have a function
\(\mathtt{proj}:\{ 0, \ldots,n \}\times X \rightarrow X\) which
returns the projection on the homogeneous component
\(\Omega^p (V)\). The implementation is straightforward when using the
internals of EAB. Probably there are better ways to do this,
especially by using exported functions only.

\begin{Verbatim}[commandchars=\\\{\}]
** deprecated **
proj(x,p) ==
  t:List REA := x::List REA
  idx := [j for j in 1..\PYGZsh{}t \textbar{} \PYGZsh{}pos(t.j.k,1)=p]
  s := [copy(t.j) for j in idx::List(NNI)]
  convert(s)\PYGZdl{}DRC
\end{Verbatim}

\textbf{NEW}

In the new version we actually replaced the function above by the following
recursive one:

\begin{Verbatim}[commandchars=\\\{\}]
proj(p,x) ==
  x=0 =\PYGZgt{} x
  homogeneous? x and degree(x)=p =\PYGZgt{} x
  a:=leadingBasisTerm(x)
  if degree(a)=p then
    leadingCoefficient(x)*a + proj(p, reductum x)
  else
    proj(p, reductum x)
\end{Verbatim}

\textbf{NOTE}
We have changed the order of arguments from (DRC,NNI) to (NNI,DRC) because
this corresponds more to the usual nomenclature of projections.


\chapter{4 Usage}
\label{section-4.0:usage}\label{section-4.0::doc}

\section{4.0 Examples}
\label{section-4.0:examples}
In this chapter some examples are provided.


\subsection{4.1 Calculus in \protect\(\mathbb{R}^3\protect\)}
\label{section-4.0:calculus-in}
We will prove the following identities (see the summary, 4.1.4, for details):
\begin{equation*}
\begin{split}d\,f = [\mathtt{grad}\,f]_1 \\
d\,[T]_1 = [\mathtt{curl}\,T]_2 \\
d\,[T]_2 = [\mathtt{div}\,T]_3\end{split}
\end{equation*}
Let M denote our differential graded algebra on \(\mathbb{R}^3\). In
FriCAS we can express this as

\begin{Verbatim}[commandchars=\\\{\}]
\PYG{n}{M} \PYG{o}{==}\PYG{o}{\PYGZgt{}} \PYG{n}{DFORM}\PYG{p}{(}\PYG{n}{INT}\PYG{p}{,}\PYG{p}{[}\PYG{n}{x}\PYG{p}{,}\PYG{n}{y}\PYG{p}{,}\PYG{n}{z}\PYG{p}{]}\PYG{p}{)}
\end{Verbatim}
\begin{equation*}
\begin{split}\mathtt{DifferentialForms(Integer,[x,y,z])}\end{split}
\end{equation*}
$_{\text{Type: Type}}$

The list of available methods can be seen by

\begin{Verbatim}[commandchars=\\\{\}]
\PYG{p}{)}\PYG{n}{show} \PYG{n}{M}

\PYG{n}{DifferentialForms}\PYG{p}{(}\PYG{n}{Integer}\PYG{p}{,}\PYG{p}{[}\PYG{n}{x}\PYG{p}{,}\PYG{n}{y}\PYG{p}{,}\PYG{n}{z}\PYG{p}{]}\PYG{p}{)} \PYG{o+ow}{is} \PYG{n}{a} \PYG{n}{package} \PYG{n}{constructor}\PYG{o}{.}
\PYG{n}{Abbreviation} \PYG{k}{for} \PYG{n}{DifferentialForms} \PYG{o+ow}{is} \PYG{n}{DFORM}
\PYG{n}{This} \PYG{n}{constructor} \PYG{o+ow}{is} \PYG{n}{exposed} \PYG{o+ow}{in} \PYG{n}{this} \PYG{n}{frame}\PYG{o}{.}
\PYG{o}{\PYGZhy{}}\PYG{o}{\PYGZhy{}}\PYG{o}{\PYGZhy{}}\PYG{o}{\PYGZhy{}}\PYG{o}{\PYGZhy{}}\PYG{o}{\PYGZhy{}}\PYG{o}{\PYGZhy{}}\PYG{o}{\PYGZhy{}}\PYG{o}{\PYGZhy{}}\PYG{o}{\PYGZhy{}}\PYG{o}{\PYGZhy{}}\PYG{o}{\PYGZhy{}}\PYG{o}{\PYGZhy{}}\PYG{o}{\PYGZhy{}}\PYG{o}{\PYGZhy{}}\PYG{o}{\PYGZhy{}}\PYG{o}{\PYGZhy{}}\PYG{o}{\PYGZhy{}}\PYG{o}{\PYGZhy{}}\PYG{o}{\PYGZhy{}}\PYG{o}{\PYGZhy{}}\PYG{o}{\PYGZhy{}}\PYG{o}{\PYGZhy{}}\PYG{o}{\PYGZhy{}}\PYG{o}{\PYGZhy{}}\PYG{o}{\PYGZhy{}}\PYG{o}{\PYGZhy{}}\PYG{o}{\PYGZhy{}}\PYG{o}{\PYGZhy{}}\PYG{o}{\PYGZhy{}}\PYG{o}{\PYGZhy{}} \PYG{n}{Operations} \PYG{o}{\PYGZhy{}}\PYG{o}{\PYGZhy{}}\PYG{o}{\PYGZhy{}}\PYG{o}{\PYGZhy{}}\PYG{o}{\PYGZhy{}}\PYG{o}{\PYGZhy{}}\PYG{o}{\PYGZhy{}}\PYG{o}{\PYGZhy{}}\PYG{o}{\PYGZhy{}}\PYG{o}{\PYGZhy{}}\PYG{o}{\PYGZhy{}}\PYG{o}{\PYGZhy{}}\PYG{o}{\PYGZhy{}}\PYG{o}{\PYGZhy{}}\PYG{o}{\PYGZhy{}}\PYG{o}{\PYGZhy{}}\PYG{o}{\PYGZhy{}}\PYG{o}{\PYGZhy{}}\PYG{o}{\PYGZhy{}}\PYG{o}{\PYGZhy{}}\PYG{o}{\PYGZhy{}}\PYG{o}{\PYGZhy{}}\PYG{o}{\PYGZhy{}}\PYG{o}{\PYGZhy{}}\PYG{o}{\PYGZhy{}}\PYG{o}{\PYGZhy{}}\PYG{o}{\PYGZhy{}}\PYG{o}{\PYGZhy{}}\PYG{o}{\PYGZhy{}}\PYG{o}{\PYGZhy{}}\PYG{o}{\PYGZhy{}}\PYG{o}{\PYGZhy{}}

\PYG{n}{coordSymbols} \PYG{p}{:} \PYG{p}{(}\PYG{p}{)} \PYG{o}{\PYGZhy{}}\PYG{o}{\PYGZgt{}} \PYG{n}{List}\PYG{p}{(}\PYG{n}{Symbol}\PYG{p}{)}
\PYG{o}{.}\PYG{o}{.}\PYG{o}{.}
\PYG{n}{baseForms} \PYG{p}{:} \PYG{p}{(}\PYG{p}{)} \PYG{o}{\PYGZhy{}}\PYG{o}{\PYGZgt{}} \PYG{n}{List}\PYG{p}{(}\PYG{n}{DeRhamComplex}\PYG{p}{(}\PYG{n}{Integer}\PYG{p}{,}\PYG{p}{[}\PYG{n}{x}\PYG{p}{,}\PYG{n}{y}\PYG{p}{,}\PYG{n}{z}\PYG{p}{]}\PYG{p}{)}\PYG{p}{)}
\PYG{n}{coordVector} \PYG{p}{:} \PYG{p}{(}\PYG{p}{)} \PYG{o}{\PYGZhy{}}\PYG{o}{\PYGZgt{}} \PYG{n}{List}\PYG{p}{(}\PYG{n}{Expression}\PYG{p}{(}\PYG{n}{Integer}\PYG{p}{)}\PYG{p}{)}
\PYG{o}{.}
\PYG{o}{.}
\PYG{o}{.}
\end{Verbatim}

The position vector \(P=(x,y,z)\) and the basis of one forms can be
obtained by

\begin{Verbatim}[commandchars=\\\{\}]
P:=coordVector()\PYGZdl{}M
\end{Verbatim}
\begin{equation*}
\begin{split}[x,y,z]\end{split}
\end{equation*}
$_{\text{Type: List(Expression(Integer))}}$

and

\begin{Verbatim}[commandchars=\\\{\}]
dP:=baseForms()\PYGZdl{}M
\end{Verbatim}
\begin{equation*}
\begin{split}[dx,dy,dz]\end{split}
\end{equation*}
$_{\text{Type: List(DeRhamComplex(Integer,{[}x,y,z{]}))}}$

This way we can call the coordinates as \(P.i\) and the basis one forms
as \(dP.i\). Of course, we can also use \(dx,dy,dz\) directly when
we set

\begin{Verbatim}[commandchars=\\\{\}]
[dx,dy,dz]:=baseForms()\PYGZdl{}M
\end{Verbatim}

or when we use the generators of the domain \sphinxcode{DeRhamComplex} itself:

\begin{Verbatim}[commandchars=\\\{\}]
dx:=generator(1)\PYGZdl{}DERHAM(INT,[x,y,z])
dy:= ...
\end{Verbatim}

The first method, however, is quite convenient when using indexed coordinates
and also because we can form expressions like

\begin{Verbatim}[commandchars=\\\{\}]
\PYG{n}{P} \PYG{o}{*} \PYG{n}{dP}
\end{Verbatim}
\begin{equation*}
\begin{split}z\ dz + y\ dy + x\ dx\end{split}
\end{equation*}
$_{\text{Type: DeRhamComplex(Integer,{[}x,y,z{]})}}$.


\subsubsection{4.1.1 Gradient}
\label{section-4.0:gradient}
There are many ways to create a zero form, one of those is

\begin{Verbatim}[commandchars=\\\{\}]
f := zeroForm(f)\PYGZdl{}M
\end{Verbatim}
\begin{equation*}
\begin{split}f(x,y,z)\end{split}
\end{equation*}
$_{\text{Type: DeRhamComplex(Integer,{[}x,y,z{]})}}$

Now we apply the exterior differential operator to f:

\begin{Verbatim}[commandchars=\\\{\}]
\PYG{n}{d} \PYG{n}{f}
\end{Verbatim}
\begin{equation*}
\begin{split}{{{f _ {{,3}}}
\left(
{x, \: y, \: z}
\right)}
\  dz}+{{{f _ {{,2}}}
\left(
{x, \: y, \: z}
\right)}
\  dy}+{{{f _ {{,1}}}
\left(
{x, \: y, \: z}
\right)}
\  dx}\end{split}
\end{equation*}
$_{\text{Type: DeRhamComplex(Integer,{[}x,y,z{]})}}$

The coefficients of \(df\) are just

\begin{Verbatim}[commandchars=\\\{\}]
\PYG{p}{[}\PYG{n}{coefficient}\PYG{p}{(}\PYG{n}{d} \PYG{n}{f}\PYG{p}{,} \PYG{n}{dP}\PYG{o}{.}\PYG{n}{j}\PYG{p}{)} \PYG{k}{for} \PYG{n}{j} \PYG{o+ow}{in} \PYG{l+m+mf}{1.}\PYG{o}{.}\PYG{l+m+mi}{3}\PYG{p}{]}
\end{Verbatim}
\begin{equation*}
\begin{split} \left[
 {{f _ {{,1}}}
 \left(
 {x, \: y, \: z}
 \right)},
 \: {{f _ {{,2}}}
 \left(
 {x, \: y, \: z}
 \right)},
 \: {{f _ {{,3}}}
 \left(
 {x, \: y, \: z}
 \right)}
\right]\end{split}
\end{equation*}
$_{\text{Type: List(Expression(Integer))}}$

the components of the gradient vector \(\nabla f\) of \(f\).


\subsubsection{4.1.2 Curl}
\label{section-4.0:curl}
Let T be a generic vector field on \(M=\mathbb{R}^3\):

\begin{Verbatim}[commandchars=\\\{\}]
T := vectorField(T)\PYGZdl{}M
\end{Verbatim}
\begin{equation*}
\begin{split}\left[
{{T _ {1}}
\left(
{x, \: y, \: z}
\right)},
\: {{T _ {2}}
\left(
{x, \: y, \: z}
\right)},
\: {{T _ {3}}
\left(
{x, \: y, \: z}
\right)}
\right]\end{split}
\end{equation*}
$_{\text{Type: List(Expression(Integer))}}$

Then we build the general one form \(\tau\):

\begin{Verbatim}[commandchars=\\\{\}]
\PYG{n}{tau} \PYG{p}{:}\PYG{o}{=} \PYG{n}{T} \PYG{o}{*} \PYG{n}{dP}
\end{Verbatim}

Now we apply the exterior differential operator \(d\):

\begin{Verbatim}[commandchars=\\\{\}]
\PYG{n}{d} \PYG{n}{tau}
\end{Verbatim}
\begin{equation*}
\begin{split}\scriptstyle{
{{\left( {{{T _ {3}} _ {{,2}}}
\left(
{x, \: y, \: z}
\right)}
-{{{T _ {2}} _ {{,3}}}
\left(
{x, \: y, \: z}
\right)}
\right)}
\  dy \  dz}+{{\left( {{{T _ {3}} _ {{,1}}}
\left(
{x, \: y, \: z}
\right)}
-{{{T _ {1}} _ {{,3}}}
\left(
{x, \: y, \: z}
\right)}
\right)}
\  dx \  dz}+  \\ {{\left( {{{T _ {2}} _ {{,1}}}
\left(
{x, \: y, \: z}
\right)}
-{{{T _ {1}} _ {{,2}}}
\left(
{x, \: y, \: z}
\right)}
\right)}
\  dx \  dy}
}\end{split}
\end{equation*}
$_{\text{Type: DeRhamComplex(Integer,{[}x,y,z{]})}}$

Next, we want to extract the coefficients:

\begin{Verbatim}[commandchars=\\\{\}]
[coefficient(d tau, m) for m in monomials(2)\PYGZdl{}M]
\end{Verbatim}
\begin{equation*}
\begin{split}\scriptstyle{
\left[
{{{{T _ {2}} _ {{,1}}}
\left(
{x, \: y, \: z}
\right)}
-{{{T _ {1}} _ {{,2}}}
\left(
{x, \: y, \: z}
\right)}},
\: {{{{T _ {3}} _ {{,1}}}
\left(
{x, \: y, \: z}
\right)}
-{{{T _ {1}} _ {{,3}}}
\left(
{x, \: y, \: z}
\right)}},
\: {{{{T _ {3}} _ {{,2}}}
\left(
{x, \: y, \: z}
\right)}
-{{{T _ {2}} _ {{,3}}}
\left(
{x, \: y, \: z}
\right)}}
\right]}\end{split}
\end{equation*}
The (well known) \textbf{curl} is defined as
\begin{equation*}
\begin{split}\mathtt{curl}(T) =\nabla\times T= \scriptstyle{
\left(
\frac{\partial T_3}{\partial y} - \frac{\partial T_2}{\partial z},
\frac{\partial T_1}{\partial z} - \frac{\partial T_3}{\partial x},
\frac{\partial T_2}{\partial x} - \frac{\partial T_1}{\partial y}
\right)}\end{split}
\end{equation*}
\begin{Verbatim}[commandchars=\\\{\}]
\PYG{n}{curl}\PYG{p}{(}\PYG{n}{V}\PYG{p}{)} \PYG{o}{==} \PYG{p}{[}\PYG{n}{D}\PYG{p}{(}\PYG{n}{V}\PYG{o}{.}\PYG{l+m+mi}{3}\PYG{p}{,}\PYG{n}{y}\PYG{p}{)}\PYG{o}{\PYGZhy{}}\PYG{n}{D}\PYG{p}{(}\PYG{n}{V}\PYG{o}{.}\PYG{l+m+mi}{2}\PYG{p}{,}\PYG{n}{z}\PYG{p}{)}\PYG{p}{,}\PYG{n}{D}\PYG{p}{(}\PYG{n}{V}\PYG{o}{.}\PYG{l+m+mi}{1}\PYG{p}{,}\PYG{n}{z}\PYG{p}{)}\PYG{o}{\PYGZhy{}}\PYG{n}{D}\PYG{p}{(}\PYG{n}{V}\PYG{o}{.}\PYG{l+m+mi}{3}\PYG{p}{,}\PYG{n}{x}\PYG{p}{)}\PYG{p}{,}\PYG{n}{D}\PYG{p}{(}\PYG{n}{V}\PYG{o}{.}\PYG{l+m+mi}{2}\PYG{p}{,}\PYG{n}{x}\PYG{p}{)}\PYG{o}{\PYGZhy{}}\PYG{n}{D}\PYG{p}{(}\PYG{n}{V}\PYG{o}{.}\PYG{l+m+mi}{1}\PYG{p}{,}\PYG{n}{y}\PYG{p}{)}\PYG{p}{]}
\end{Verbatim}

We now \textbf{claim} that the following identity holds:
\begin{equation*}
\begin{split}d (T\, dP) =  \star(\mathtt{curl}(V)\, dP)\end{split}
\end{equation*}
where \sphinxcode{*} denotes the Hodge star operator with respect to the Euclidean
metric

\begin{Verbatim}[commandchars=\\\{\}]
\PYG{n}{g}\PYG{p}{:}\PYG{o}{=}\PYG{n}{diagonalMatrix}\PYG{p}{(}\PYG{p}{[}\PYG{l+m+mi}{1}\PYG{p}{,}\PYG{l+m+mi}{1}\PYG{p}{,}\PYG{l+m+mi}{1}\PYG{p}{]}\PYG{p}{)}
\end{Verbatim}
\begin{equation*}
\begin{split}\left[
\begin{array}{ccc}
1 & 0 & 0 \\
0 & 1 & 0 \\
0 & 0 & 1
\end{array}
\right]\end{split}
\end{equation*}
To prove it we just have to test:

\begin{Verbatim}[commandchars=\\\{\}]
test( d(T*dP) = hodgeStar(g,curl(T)*dP)\PYGZdl{}M )
\end{Verbatim}
\begin{equation*}
\begin{split}\mathtt{true}\end{split}
\end{equation*}
$_{\text{Type: Boolean}}$


\subsubsection{4.1.3 Divergence}
\label{section-4.0:divergence}
Again, let T be a generic vector field on \(M=\mathbb{R}^3\), then the
divergence is defined by
\begin{equation*}
\begin{split}\mathtt{div}(T) = \nabla \bullet T =
\scriptstyle{
\frac{\partial T_1}{\partial x} +
\frac{\partial T_2}{\partial y} +
\frac{\partial T_3}{\partial z}}.\end{split}
\end{equation*}
When we calculate

\begin{Verbatim}[commandchars=\\\{\}]
d hodgeStar(g, T*dP)\PYGZdl{}M
\end{Verbatim}

we get the 3-form
\begin{equation*}
\begin{split}{\left( {{{T _ {3}} _ {{,3}}}
\left(
{x, \: y, \: z}
\right)}+{{{T
_ {2}} _ {{,2}}}
\left(
{x, \: y, \: z}
\right)}+{{{T
_ {1}} _ {{,1}}}
\left(
{x, \: y, \: z}
\right)}
\right)}
\  dx \  dy \  dz\end{split}
\end{equation*}

\subsubsection{4.1.4 Summary}
\label{section-4.0:summary}
Let us summarize what we have obtained above. We use the following notation
for the mapping of scalar functions and vector fields to differential forms:
\begin{equation*}
\begin{split}f \rightarrow [f]_0 \\
T \rightarrow [T]_1\end{split}
\end{equation*}
where the index denotes the degree of the form. Moreover, we define another
pair of forms by applying the Hodge operator:
\begin{equation*}
\begin{split}[T]_2 = \star [T]_1 \\
[f]_3 = \star [f]_0\end{split}
\end{equation*}
So we can state the general identities:
\begin{equation*}
\begin{split}d\,f = [\nabla\,f]_1 \\
d\,[T]_1 = [\mathtt{curl}\,T]_2 \\
d\,[T]_2 = [\mathtt{div}\,T]_3\end{split}
\end{equation*}

\subsubsection{4.1.5 Hodge duals}
\label{section-4.0:hodge-duals}
To conclude this example, we are going to calculate a table for the Hodge
duals of the monomials.

\begin{Verbatim}[commandchars=\\\{\}]
g:=diagonalMatrix([1,1,1])::SquareMatrix(3,INT)

[[hodgeStar(g,m)\PYGZdl{}M for m in monomials(j)\PYGZdl{}M] for j in 0..3]
\end{Verbatim}
\begin{equation*}
\begin{split}\left[
{\left[ {dx \  dy \  dz}
\right]},
\: {\left[ {dy \  dz}, \: -{dx \  dz}, \: {dx \  dy}
\right]},
\: {\left[ dz, \: -dy, \: dx
\right]},
\: {\left[ 1
\right]}
\right]\end{split}
\end{equation*}
$_{\text{Type: List(List(DeRhamComplex(Integer,{[}x,y,z{]})))}}$

Thus we get the following table:

\noindent\begin{tabulary}{\linewidth}{|L|L|L|}
\hline

\(\alpha\)
&
\(\star\alpha\)
&
\(\star\star\alpha\)
\\
\hline
\(1\)
&
\(dx\wedge dy \wedge dz\)
&
\(1\)
\\
\hline
\(dx\)
&
\(dy \wedge dz\)
&
\(dx\)
\\
\hline
\(dy\)
&
\(-dx \wedge dz\)
&
\(dy\)
\\
\hline
\(dz\)
&
\(dx \wedge dy\)
&
\(dz\)
\\
\hline\end{tabulary}


By the way, this method can be applied in any dimension for any metric.


\subsection{4.2 Faraday 2-form}
\label{section-4.0:faraday-2-form}
The free electromagnetic field can be described by a 2-form \textbf{F} in
Minkowski space. This form - also known as Faraday 2-form - is given by
\begin{equation*}
\begin{split}\scriptstyle{
F=B_1\ dy\wedge dz + B_2\ dz\wedge dx + B_3\ dx\wedge dy +
  E_1\ dx\wedge dt + E_2\ dy\wedge dt + E_3\ dz\wedge dt
}\end{split}
\end{equation*}
where we here use the \textbf{cgs} system and \textbf{E}, \textbf{B} denote the classical
fields (see the example in the documentation of \sphinxcode{DeRhamComplex}).

To represent \textbf{F} in FriCAS we have to choose space-time variables
\(x,y,z,t\), in the correct order, and \(g\) will be the
Minkowski metric:

\begin{Verbatim}[commandchars=\\\{\}]
\PYG{n}{v} \PYG{p}{:}\PYG{o}{=} \PYG{p}{[}\PYG{n}{x}\PYG{p}{,}\PYG{n}{y}\PYG{p}{,}\PYG{n}{z}\PYG{p}{,}\PYG{n}{t}\PYG{p}{]}

\PYG{n}{g} \PYG{p}{:}\PYG{o}{=} \PYG{n}{diagonalMatrix}\PYG{p}{(}\PYG{p}{[}\PYG{o}{\PYGZhy{}}\PYG{l+m+mi}{1}\PYG{p}{,}\PYG{o}{\PYGZhy{}}\PYG{l+m+mi}{1}\PYG{p}{,}\PYG{o}{\PYGZhy{}}\PYG{l+m+mi}{1}\PYG{p}{,}\PYG{l+m+mi}{1}\PYG{p}{]}\PYG{p}{)}\PYG{p}{:}\PYG{p}{:}\PYG{n}{SquareMatrix}\PYG{p}{(}\PYG{l+m+mi}{4}\PYG{p}{,}\PYG{n}{INT}\PYG{p}{)}

\PYG{n}{M} \PYG{p}{:}\PYG{o}{=} \PYG{n}{DFORM}\PYG{p}{(}\PYG{n}{INT}\PYG{p}{,}\PYG{n}{v}\PYG{p}{)}

\PYG{n}{R} \PYG{o}{==}\PYG{o}{\PYGZgt{}} \PYG{n}{EXPR}\PYG{p}{(}\PYG{n}{INT}\PYG{p}{)}
\end{Verbatim}

Instead of \(x,y,z,t\) we also could have chosen \(x_0,x_1,x_2,x_3\)
for instance. Now we need the coordinates and basis one forms:
\begin{description}
\item[{\textbf{Important}}] \leavevmode
The order of the variables must coincide with that in the metric g.
That means for example, for \(t,x,y,z\) the positive \sphinxcode{1} comes
first.

\end{description}

\begin{Verbatim}[commandchars=\\\{\}]
X := coordVector()\PYGZdl{}M

dX := baseForms()\PYGZdl{}M
\end{Verbatim}

We also need the field \textbf{E} and \textbf{B}, but this time we will not choose the
\sphinxcode{vectorField} function because we only need three components:

\begin{Verbatim}[commandchars=\\\{\}]
\PYG{n}{E} \PYG{p}{:}\PYG{o}{=} \PYG{p}{[}\PYG{n}{operator} \PYG{n}{E}\PYG{p}{[}\PYG{n}{i}\PYG{p}{]} \PYG{k}{for} \PYG{n}{i} \PYG{o+ow}{in} \PYG{l+m+mf}{1.}\PYG{o}{.}\PYG{l+m+mi}{3}\PYG{p}{]}
\PYG{n}{B} \PYG{p}{:}\PYG{o}{=} \PYG{p}{[}\PYG{n}{operator} \PYG{n}{B}\PYG{p}{[}\PYG{n}{i}\PYG{p}{]} \PYG{k}{for} \PYG{n}{i} \PYG{o+ow}{in} \PYG{l+m+mf}{1.}\PYG{o}{.}\PYG{l+m+mi}{3}\PYG{p}{]}
\end{Verbatim}

Eventually we can build \textbf{F}:

\begin{Verbatim}[commandchars=\\\{\}]
\PYG{n}{F} \PYG{p}{:}\PYG{o}{=} \PYG{p}{(}\PYG{n}{B}\PYG{o}{.}\PYG{l+m+mi}{1} \PYG{n}{X}\PYG{p}{)}\PYG{o}{*}\PYG{n}{dX}\PYG{o}{.}\PYG{l+m+mi}{2}\PYG{o}{*}\PYG{n}{dX}\PYG{o}{.}\PYG{l+m+mi}{3} \PYG{o}{+} \PYG{p}{(}\PYG{n}{B}\PYG{o}{.}\PYG{l+m+mi}{2} \PYG{n}{X}\PYG{p}{)}\PYG{o}{*}\PYG{n}{dX}\PYG{o}{.}\PYG{l+m+mi}{3}\PYG{o}{*}\PYG{n}{dX}\PYG{o}{.}\PYG{l+m+mi}{1} \PYG{o}{+} \PYG{p}{(}\PYG{n}{B}\PYG{o}{.}\PYG{l+m+mi}{3} \PYG{n}{X}\PYG{p}{)}\PYG{o}{*}\PYG{n}{dX}\PYG{o}{.}\PYG{l+m+mi}{1}\PYG{o}{*}\PYG{n}{dX}\PYG{o}{.}\PYG{l+m+mi}{2} \PYG{o}{+}\PYG{n}{\PYGZus{}}
     \PYG{p}{(}\PYG{n}{E}\PYG{o}{.}\PYG{l+m+mi}{1} \PYG{n}{X}\PYG{p}{)}\PYG{o}{*}\PYG{n}{dX}\PYG{o}{.}\PYG{l+m+mi}{1}\PYG{o}{*}\PYG{n}{dX}\PYG{o}{.}\PYG{l+m+mi}{4} \PYG{o}{+} \PYG{p}{(}\PYG{n}{E}\PYG{o}{.}\PYG{l+m+mi}{2} \PYG{n}{X}\PYG{p}{)}\PYG{o}{*}\PYG{n}{dX}\PYG{o}{.}\PYG{l+m+mi}{2}\PYG{o}{*}\PYG{n}{dX}\PYG{o}{.}\PYG{l+m+mi}{4} \PYG{o}{+} \PYG{p}{(}\PYG{n}{E}\PYG{o}{.}\PYG{l+m+mi}{3} \PYG{n}{X}\PYG{p}{)}\PYG{o}{*}\PYG{n}{dX}\PYG{o}{.}\PYG{l+m+mi}{3}\PYG{o}{*}\PYG{n}{dX}\PYG{o}{.}\PYG{l+m+mi}{4}
\end{Verbatim}
\begin{equation*}
\begin{split}\scriptstyle{
{{{E _ {3}}
\left(
{x, \: y, \: z, \: t}
\right)}
\  dz \  dt}+{{{E _ {2}}
\left(
{x, \: y, \: z, \: t}
\right)}
\  dy \  dt}+{{{B _ {1}}
\left(
{x, \: y, \: z, \: t}
\right)}
\  dy \  dz} + \\
{{{E _ {1}}
\left(
{x, \: y, \: z, \: t}
\right)}
\  dx \  dt} -{{{B _ {2}}
\left(
{x, \: y, \: z, \: t}
\right)}
\  dx \  dz}+{{{B _ {3}}
\left(
{x, \: y, \: z, \: t}
\right)}
\  dx \  dy}
}\end{split}
\end{equation*}
$_{\text{Type: DeRhamComplex(Integer,{[}x,y,z,t{]})}}$

We apply the exterior differential operator \textbf{d} to \textbf{F}:

\begin{Verbatim}[commandchars=\\\{\}]
\PYG{n}{d} \PYG{n}{F}
\end{Verbatim}
\begin{equation*}
\begin{split}\scriptstyle{
  {{\left( {{{E _ {3}} _ {{,2}}}
  \left(
  {x, \: y, \: z, \: t}
  \right)}
  -{{{E _ {2}} _ {{,3}}}
  \left(
  {x, \: y, \: z, \: t}
  \right)}+{{{B
  _ {1}} _ {{,4}}}
  \left(
  {x, \: y, \: z, \: t}
  \right)}
  \right)}
  \  dy \  dz \  dt}\, + \\
  {{\left( {{{E _ {3}} _ {{,1}}}
  \left(
  {x, \: y, \: z, \: t}
  \right)}
  -{{{E _ {1}} _ {{,3}}}
  \left(
  {x, \: y, \: z, \: t}
  \right)}
  -{{{B _ {2}} _ {{,4}}}
  \left(
  {x, \: y, \: z, \: t}
  \right)}
  \right)}
  \  dx \  dz \  dt}\, + \\
  {{\left( {{{E _ {2}} _ {{,1}}}
  \left(
  {x, \: y, \: z, \: t}
  \right)}
  -{{{E _ {1}} _ {{,2}}}
  \left(
  {x, \: y, \: z, \: t}
  \right)}+{{{B
  _ {3}} _ {{,4}}}
  \left(
  {x, \: y, \: z, \: t}
  \right)}
  \right)}
  \  dx \  dy \  dt}\, + \\
  {{\left( {{{B _ {3}} _ {{,3}}}
  \left(
  {x, \: y, \: z, \: t}
  \right)}+{{{B
  _ {2}} _ {{,2}}}
  \left(
  {x, \: y, \: z, \: t}
  \right)}+{{{B
  _ {1}} _ {{,1}}}
  \left(
  {x, \: y, \: z, \: t}
  \right)}
  \right)}
  \  dx \  dy \  dz}
  }\end{split}
\end{equation*}
$_{\text{Type: DeRhamComplex(Integer,{[}x,y,z,t{]})}}$

We see at once that the first three terms of the sum correspond to the
vector
\begin{equation*}
\begin{split}\nabla\times\mathbf{E}+\frac{\partial\mathbf{B}}{\partial t}\end{split}
\end{equation*}
and the fourth term is
\begin{equation*}
\begin{split}\nabla\bullet\mathbf{B}.\end{split}
\end{equation*}
Actually, all terms are zero by two of the \emph{Maxwell} equations. Consequently
we have shown (the well known fact)
\begin{equation*}
\begin{split}d\mathbf{F} = 0\end{split}
\end{equation*}
Now let us apply the \(\star\)-operator to \textbf{F}, which is also a 2-form:

\begin{Verbatim}[commandchars=\\\{\}]
\PYGZpc{}F := hodgeStar(g,F)\PYGZdl{}M
\end{Verbatim}
\begin{equation*}
\begin{split}\scriptstyle{
 {{{B _ {3}}
 \left(
 {x, \: y, \: z, \: t}
 \right)}
 \  dz \  dt} + {{{B _ {2}}
 \left(
 {x, \: y, \: z, \: t}
 \right)}
 \  dy \  dt} -{{{E _ {1}}
 \left(
 {x, \: y, \: z, \: t}
 \right)}
 \  dy \  dz}+ \\
 {{{B _ {1}}
 \left(
 {x, \: y, \: z, \: t}
 \right)}
 \  dx \  dt} + {{{E _ {2}}
 \left(
 {x, \: y, \: z, \: t}
 \right)}
 \  dx \  dz}- {{{E _ {3}}
 \left(
 {x, \: y, \: z, \: t}
 \right)}
 \  dx \  dy}
 }\end{split}
\end{equation*}
$_{\text{Type: DeRhamComplex(Integer,{[}x,y,z,t{]})}}$

Now, as before:

\begin{Verbatim}[commandchars=\\\{\}]
\PYG{n}{d} \PYG{o}{\PYGZpc{}}\PYG{n}{F}
\end{Verbatim}
\begin{equation*}
\begin{split}\scriptstyle{
{{\left( -{{{E _ {1}} _ {{,4}}}
\left(
{x, \: y, \: z, \: t}
\right)}+{{{B
_ {3}} _ {{,2}}}
\left(
{x, \: y, \: z, \: t}
\right)}
-{{{B _ {2}} _ {{,3}}}
\left(
{x, \: y, \: z, \: t}
\right)}
\right)}
\  dy \  dz \  dt}+ \\
{{\left( {{{E _ {2}} _ {{,4}}}
\left(
{x, \: y, \: z, \: t}
\right)}+{{{B
_ {3}} _ {{,1}}}
\left(
{x, \: y, \: z, \: t}
\right)}
-{{{B _ {1}} _ {{,3}}}
\left(
{x, \: y, \: z, \: t}
\right)}
\right)}
\  dx \  dz \  dt}+ \\
{{\left( -{{{E _ {3}} _ {{,4}}}
\left(
{x, \: y, \: z, \: t}
\right)}+{{{B
_ {2}} _ {{,1}}}
\left(
{x, \: y, \: z, \: t}
\right)}
-{{{B _ {1}} _ {{,2}}}
\left(
{x, \: y, \: z, \: t}
\right)}
\right)}
\  dx \  dy \  dt}+ \\
{{\left( -{{{E _ {3}} _ {{,3}}}
\left(
{x, \: y, \: z, \: t}
\right)}
-{{{E _ {2}} _ {{,2}}}
\left(
{x, \: y, \: z, \: t}
\right)}
-{{{E _ {1}} _ {{,1}}}
\left(
{x, \: y, \: z, \: t}
\right)}
\right)}
\  dx \  dy \  dz}
}\end{split}
\end{equation*}
$_{\text{Type: DeRhamComplex(Integer,{[}x,y,z,t{]})}}$

Again, we see that the first three terms correspond to
\begin{equation*}
\begin{split}-\frac{\partial\mathbf{E}}{\partial t}+ \nabla\times\mathbf{B}\end{split}
\end{equation*}
while the last one corresponds to:
\begin{equation*}
\begin{split}-\,\nabla\bullet\mathbf{E}\end{split}
\end{equation*}
Thus, in vacuum, these are the second pair of \emph{Maxwell's} equation and we
have:
\begin{equation*}
\begin{split}d \star\mathbf{F} = 0\end{split}
\end{equation*}
To conclude this example we will compute the quantities (4-forms):
\begin{equation*}
\begin{split}\mathbf{F} \wedge \mathbf{F} \ \ \mathrm{and} \ \
\mathbf{F} \wedge \star\mathbf{F}.\end{split}
\end{equation*}
Recalling the definition of the Hodge dual it is sufficient (in principle)
to compute the scalar product \(\langle F,F\rangle\):

\begin{Verbatim}[commandchars=\\\{\}]
dot(g,F,F)\PYGZdl{}M
\end{Verbatim}
\begin{equation*}
\begin{split}\scriptstyle{
 -{{{{E _ {3}}
 \left(
 {x, \: y, \: z, \: t}
 \right)}}
 ^ {2}} -{{{{E _ {2}}
 \left(
 {x, \: y, \: z, \: t}
 \right)}}
 ^ {2}} -{{{{E _ {1}}
 \left(
 {x, \: y, \: z, \: t}
 \right)}}
 ^ {2}}+ \\
 {{{{B _ {3}}
 \left(
 {x, \: y, \: z, \: t}
 \right)}}
 ^ {2}}+{{{{B _ {2}}
 \left(
 {x, \: y, \: z, \: t}
 \right)}}
 ^ {2}}+{{{{B _ {1}}
 \left(
 {x, \: y, \: z, \: t}
 \right)}}
 ^ {2}}
 }\end{split}
\end{equation*}
$_{\text{Type: Expression(Integer)}}$

and \(\langle F,\star F\rangle\):

\begin{Verbatim}[commandchars=\\\{\}]
dot(g,F,\PYGZpc{}F)\PYGZdl{}M
\end{Verbatim}
\begin{equation*}
\begin{split}\scriptstyle{
 -{2 \  {{B _ {3}}
 \left(
 {x, \: y, \: z, \: t}
 \right)}
 \  {{E _ {3}}
 \left(
 {x, \: y, \: z, \: t}
 \right)}}
 -{2 \  {{B _ {2}}
 \left(
 {x, \: y, \: z, \: t}
 \right)}
 \  {{E _ {2}}
 \left(
 {x, \: y, \: z, \: t}
 \right)}}
 -{2 \  {{B _ {1}}
 \left(
 {x, \: y, \: z, \: t}
 \right)}
 \  {{E _ {1}}
 \left(
 {x, \: y, \: z, \: t}
 \right)}}
 }\end{split}
\end{equation*}
$_{\text{Type: Expression(Integer)}}$

Indeed, we can \emph{test} the defining identity, e.g. for the first case:

\begin{Verbatim}[commandchars=\\\{\}]
test(F * \PYGZpc{}F = dot(g,F,F)\PYGZdl{}M * volumeForm(g)\PYGZdl{}M)
\end{Verbatim}
\begin{equation*}
\begin{split}\mathtt{true}\end{split}
\end{equation*}
$_{\text{Type: Boolean}}$


\subsection{4.3 Some Examples from \emph{Maple}}
\label{section-4.0:some-examples-from-maple}
Examples from \href{http://www.maplesoft.com/support/help/Maple/view.aspx?path=DifferentialGeometry/Tensor/HodgeStar}{Maple}.


\subsubsection{4.3.1 5-dimensional Manifold}
\label{section-4.0:dimensional-manifold}\label{section-4.0:maple}
First create a 5-dimensional manifold M and define a metric tensor g on the
tangent space of M:

\begin{Verbatim}[commandchars=\\\{\}]
v:=[x[j] for j in 1..5]
M:=DFORM(INT,v)
g:=diagonalMatrix([1,1,1,1,1])::SquareMatrix(5,INT)
dX:=baseForms()\PYGZdl{}M
\end{Verbatim}

\begin{Verbatim}[commandchars=\\\{\}]
hodgeStar(g,dX.1)\PYGZdl{}M
\end{Verbatim}
\begin{equation*}
\begin{split}{dx _ {2}} \  {dx _ {3}} \  {dx _ {4}} \  {dx _ {5}}\end{split}
\end{equation*}
$_{\text{Type: DeRhamComplex(Integer,{[}x{[}1{]},x{[}2{]},x{[}3{]},x{[}4{]},x{[}5{]}{]})}}$

\begin{Verbatim}[commandchars=\\\{\}]
hodgeStar(g,dX.2)\PYGZdl{}M
\end{Verbatim}
\begin{equation*}
\begin{split}-{{dx _ {1}} \  {dx _ {3}} \  {dx _ {4}} \  {dx _ {5}}}\end{split}
\end{equation*}
$_{\text{Type: DeRhamComplex(Integer,{[}x{[}1{]},x{[}2{]},x{[}3{]},x{[}4{]},x{[}5{]}{]})}}$

\begin{Verbatim}[commandchars=\\\{\}]
hodgeStar(g,dX.2*dX.3)\PYGZdl{}M
\end{Verbatim}
\begin{equation*}
\begin{split}{dx _ {1}} \  {dx _ {4}} \  {dx _ {5}}\end{split}
\end{equation*}
$_{\text{Type: DeRhamComplex(Integer,{[}x{[}1{]},x{[}2{]},x{[}3{]},x{[}4{]},x{[}5{]}{]})}}$

\begin{Verbatim}[commandchars=\\\{\}]
hodgeStar(g,dX.2*dX.4)\PYGZdl{}M
\end{Verbatim}
\begin{equation*}
\begin{split}-{{dx _ {1}} \  {dx _ {3}} \  {dx _ {5}}}\end{split}
\end{equation*}
$_{\text{Type: DeRhamComplex(Integer,{[}x{[}1{]},x{[}2{]},x{[}3{]},x{[}4{]},x{[}5{]}{]})}}$

\begin{Verbatim}[commandchars=\\\{\}]
hodgeStar(g,dX.2*dX.3*dX.4)\PYGZdl{}M
\end{Verbatim}
\begin{equation*}
\begin{split}-{{dx _ {1}} \  {dx _ {5}}}\end{split}
\end{equation*}
$_{\text{Type: DeRhamComplex(Integer,{[}x{[}1{]},x{[}2{]},x{[}3{]},x{[}4{]},x{[}5{]}{]})}}$

We see an exact match with the published results.


\subsubsection{4.3.2 General metric (2-dim)}
\label{section-4.0:general-metric-2-dim}
To show the dependence of the Hodge star operator upon the metric, we consider
a general metric g on a 2-dimensional manifold.

\begin{Verbatim}[commandchars=\\\{\}]
v:=[x,y]
M:=DFORM(INT,v)
R ==\PYGZgt{} EXPR INT
g:=matrix([[a::R,b],[b,c]])::SquareMatrix(2,R)
[dx,dy]:=baseForms()\PYGZdl{}M
\end{Verbatim}

\begin{Verbatim}[commandchars=\\\{\}]
hodgeStar(g,dx)\PYGZdl{}M
\end{Verbatim}
\begin{equation*}
\begin{split}{{{c \  {\sqrt {{abs
\left(
{{{a \  c} -{{b} ^ {2}}}}
\right)}}}}
\over {{a \  c} -{{b} ^ {2}}}} \  dy}+{{{b \  {\sqrt {{abs
\left(
{{{a \  c} -{{b} ^ {2}}}}
\right)}}}}
\over {{a \  c} -{{b} ^ {2}}}} \  dx}\end{split}
\end{equation*}
$_{\text{Type: DeRhamComplex(Integer,{[}x,y{]})}}$

\begin{Verbatim}[commandchars=\\\{\}]
hodgeStar(g,dy)\PYGZdl{}M
\end{Verbatim}
\begin{equation*}
\begin{split}-{{{b \  {\sqrt {{abs
\left(
{{{a \  c} -{{b} ^ {2}}}}
\right)}}}}
\over {{a \  c} -{{b} ^ {2}}}} \  dy} -{{{a \  {\sqrt {{abs
\left(
{{{a \  c} -{{b} ^ {2}}}}
\right)}}}}
\over {{a \  c} -{{b} ^ {2}}}} \  dx}\end{split}
\end{equation*}
$_{\text{Type: DeRhamComplex(Integer,{[}x,y{]})}}$

\begin{Verbatim}[commandchars=\\\{\}]
f := hodgeStar(g,dx*dy)\PYGZdl{}M
\end{Verbatim}
\begin{equation*}
\begin{split}{\sqrt {{abs
 \left(
 {{{a \  c} -{{b} ^ {2}}}}
 \right)}}}
 \over {{a \  c} -{{b} ^ {2}}}\end{split}
\end{equation*}
$_{\text{Type: DeRhamComplex(Integer,{[}x,y{]})}}$

\begin{Verbatim}[commandchars=\\\{\}]
hodgeStar(g,f)\PYGZdl{}M
\end{Verbatim}
\begin{equation*}
\begin{split}{{abs
\left(
{{{a \  c} -{{b} ^ {2}}}}
\right)}
\over {{a \  c} -{{b} ^ {2}}}} \  dx \  dy\end{split}
\end{equation*}
$_{\text{Type: DeRhamComplex(Integer,{[}x,y{]})}}$


\subsubsection{4.3.3 Laplacian}
\label{section-4.0:laplacian}
The Laplacian of a function with respect to a metric g can be calculated
using the exterior derivative and the Hodge star operator. Generally, the
following identity holds:
\begin{equation*}
\begin{split}\Delta = d \circ \delta + \delta \circ d\end{split}
\end{equation*}
where \(\delta:=(-1)^p\, \star^{-1}\,d \,\star\) is the \textbf{codifferential}
to be applied on a p-form (resulting in a (p-1)-form). Therefore, the
Laplacian applied to a function f (zero form) is:
\begin{equation*}
\begin{split}\Delta f = \delta \circ df = \star^{-1}\, d \,\star df=
\star\, d \, \star df.\end{split}
\end{equation*}
\begin{Verbatim}[commandchars=\\\{\}]
v:=[r,u]  \PYGZhy{}\PYGZhy{} polar coordinates
M:=DFORM(INT,v)
R ==\PYGZgt{} EXPR INT
g:=matrix([[1,0],[0,r\PYGZca{}2]])::SquareMatrix(2,R)
[dr,du]:=baseForms()\PYGZdl{}M
\end{Verbatim}

A function on M can easiliy be defined by

\begin{Verbatim}[commandchars=\\\{\}]
f:=zeroForm(f)\PYGZdl{}M
\end{Verbatim}
\begin{equation*}
\begin{split}f\left({r, \: u}\right)\end{split}
\end{equation*}
$_{\text{Type: DeRhamComplex(Integer,{[}r,u{]})}}$

We translate the formula:

\begin{Verbatim}[commandchars=\\\{\}]
hodgeStar(g, d hodgeStar(g,d f)\PYGZdl{}M)\PYGZdl{}M
\end{Verbatim}
\begin{equation*}
\begin{split}{{{abs
\left(
{{{r} ^ {2}}}
\right)}
\  {{f _ {{{,2}{,2}}}}
\left(
{r, \: u}
\right)}}+{{{r}
^ {2}} \  {abs
\left(
{{{r} ^ {2}}}
\right)}
\  {{f _ {{{,1}{,1}}}}
\left(
{r, \: u}
\right)}}+{r
\  {abs
\left(
{{{r} ^ {2}}}
\right)}
\  {{f _ {{,1}}}
\left(
{r, \: u}
\right)}}}
\over {{r} ^ {4}}\end{split}
\end{equation*}
$_{\text{Type: DeRhamComplex(Integer,{[}r,u{]})}}$

Simplifying yields for M:
\begin{equation*}
\begin{split}\Delta_M f = \frac{\partial^2 f}{\partial r^2} +
           \frac{1}{r} \frac{\partial f}{\partial r} +
           \frac{1}{r^2} \frac{\partial^2 f}{\partial u^2}\end{split}
\end{equation*}

\subsubsection{4.3.4 Lie derivative}
\label{section-4.0:lie-derivative}
\begin{Verbatim}[commandchars=\\\{\}]
v:=[x[i] for i in 1..3]
M:=DFORM(INT,v)
dX:=baseForms()\PYGZdl{}M
V:=vectorField(V)\PYGZdl{}M
f:=scalarField(f)\PYGZdl{}M
\end{Verbatim}

\begin{Verbatim}[commandchars=\\\{\}]
\PYG{n}{lieDerivative}\PYG{p}{(}\PYG{n}{V}\PYG{p}{,}\PYG{n}{dX}\PYG{o}{.}\PYG{l+m+mi}{1}\PYG{p}{)}
\end{Verbatim}
\begin{equation*}
\begin{split}\scriptstyle{
{{{{V _ {1}} _ {{,3}}}
\left(
{{x _ {1}}, \: {x _ {2}}, \: {x _ {3}}}
\right)}
\  {dx _ {3}}}+{{{{V _ {1}} _ {{,2}}}
\left(
{{x _ {1}}, \: {x _ {2}}, \: {x _ {3}}}
\right)}
\  {dx _ {2}}}+{{{{V _ {1}} _ {{,1}}}
\left(
{{x _ {1}}, \: {x _ {2}}, \: {x _ {3}}}
\right)}
\  {dx _ {1}}}}\end{split}
\end{equation*}
$_{\text{Type: DeRhamComplex(Integer,{[}x{[}1{]},x{[}2{]},x{[}3{]}{]})}}$

\begin{Verbatim}[commandchars=\\\{\}]
\PYG{n}{lieDerivative}\PYG{p}{(}\PYG{n}{V}\PYG{p}{,}\PYG{n}{f}\PYG{o}{*}\PYG{n}{dX}\PYG{o}{.}\PYG{l+m+mi}{1}\PYG{p}{)}
\end{Verbatim}
\begin{equation*}
\begin{split}\cal{L}_V\,f\,dx_1\end{split}
\end{equation*}
$_{\text{Type: DeRhamComplex(Integer,{[}x{[}1{]},x{[}2{]},x{[}3{]}{]})}}$

\begin{Verbatim}[commandchars=\\\{\}]
\PYG{n}{lieDerivative}\PYG{p}{(}\PYG{n}{V}\PYG{p}{,}\PYG{n}{f}\PYG{o}{*}\PYG{n}{dX}\PYG{o}{.}\PYG{l+m+mi}{1}\PYG{o}{*}\PYG{n}{dX}\PYG{o}{.}\PYG{l+m+mi}{2}\PYG{p}{)}
\end{Verbatim}
\begin{equation*}
\begin{split}\cal{L}_V\,f\,dx_1\wedge dx_2\end{split}
\end{equation*}
$_{\text{Type: DeRhamComplex(Integer,{[}x{[}1{]},x{[}2{]},x{[}3{]}{]})}}$


\subsection{4.4 More examples (way of working)}
\label{section-4.0:more-examples-way-of-working}\begin{quote}\begin{description}
\item[{Setup}] \leavevmode
\end{description}\end{quote}

\begin{Verbatim}[commandchars=\\\{\}]
\PYG{p}{)}\PYG{n}{clear} \PYG{n+nb}{all}

  \PYG{n}{All} \PYG{n}{user} \PYG{n}{variables} \PYG{o+ow}{and} \PYG{n}{function} \PYG{n}{definitions} \PYG{n}{have} \PYG{n}{been} \PYG{n}{cleared}\PYG{o}{.}
\end{Verbatim}

\begin{Verbatim}[commandchars=\\\{\}]
n:=4 \PYGZhy{}\PYGZhy{} dim of base space (n\PYGZgt{}=2 !)
R:=Integer  \PYGZhy{}\PYGZhy{} Ring

v:=[subscript(x,[j::OutputForm]) for j in 1..n] \PYGZhy{}\PYGZhy{} (x\PYGZus{}1,..,x\PYGZus{}n)

M:=DFORM(R,v)

\PYGZhy{}\PYGZhy{} basis 1\PYGZhy{}forms and coordinate vector
dx:=baseForms()\PYGZdl{}M     \PYGZhy{}\PYGZhy{} [dx[1],...,dx[n]]
x:=coordVector()\PYGZdl{}M    \PYGZhy{}\PYGZhy{} [x[1],...,x[n]]
xs:=coordSymbols()\PYGZdl{}M  \PYGZhy{}\PYGZhy{} as above but as List Symbol (for differentiate, D)

\PYGZhy{}\PYGZhy{} operator, vector field, scalar field, symbol
a:=operator \PYGZsq{}a         \PYGZhy{}\PYGZhy{} operator
b:=vectorField(b)\PYGZdl{}M    \PYGZhy{}\PYGZhy{} generic vector field [b1(x1..xn),...,bn(x1..xn)]
c:=vectorField(c)\PYGZdl{}M
P:=scalarField(P)\PYGZdl{}M \PYGZhy{}\PYGZhy{} scalar field P(x1,..,xn)

\PYGZhy{}\PYGZhy{} metric
g:=diagonalMatrix([1 for i in 1..n])\PYGZdl{}SquareMatrix(n,EXPR R)  \PYGZhy{}\PYGZhy{} Euclidean
h:=diagonalMatrix(c)\PYGZdl{}SquareMatrix(n,EXPR R)

\PYGZhy{}\PYGZhy{} vector field (R)
vf:=vector b
\end{Verbatim}
\begin{quote}\begin{description}
\item[{Macros}] \leavevmode
\end{description}\end{quote}

\begin{Verbatim}[commandchars=\\\{\}]
\PYGZhy{}\PYGZhy{} macros
dV(g) ==\PYGZgt{} volumeForm(g)\PYGZdl{}M
i(X,w) ==\PYGZgt{} interiorProduct(X,w)\PYGZdl{}M
L(X,w) ==\PYGZgt{} lieDerivative(X,w)\PYGZdl{}M
** w ==\PYGZgt{} hodgeStar(g,w)\PYGZdl{}M  \PYGZhy{}\PYGZhy{} don\PYGZsq{}t use * instead of ** !
\end{Verbatim}

\begin{Verbatim}[commandchars=\\\{\}]
\PYG{n}{w}\PYG{p}{:}\PYG{o}{=}\PYG{n}{x}\PYG{o}{.}\PYG{l+m+mi}{1}\PYG{o}{*}\PYG{n}{dx}\PYG{o}{.}\PYG{l+m+mi}{2}\PYG{o}{\PYGZhy{}}\PYG{n}{x}\PYG{o}{.}\PYG{l+m+mi}{2}\PYG{o}{*}\PYG{n}{dx}\PYG{o}{.}\PYG{l+m+mi}{1}
\end{Verbatim}
\begin{equation*}
\begin{split}{{x _ {1}} \  {dx _ {2}}} -{{x _ {2}} \  {dx _ {1}}}\end{split}
\end{equation*}
$_{\text{Type: DeRhamComplex(Integer,{[}x{[}1{]},x{[}2{]},x{[}3{]},x{[}4{]}{]})}}$

\begin{Verbatim}[commandchars=\\\{\}]
\PYG{n}{d} \PYG{n}{w}
\end{Verbatim}
\begin{equation*}
\begin{split}2 \  {dx _ {1}} \  {dx _ {2}}\end{split}
\end{equation*}
$_{\text{Type: DeRhamComplex(Integer,{[}x{[}1{]},x{[}2{]},x{[}3{]},x{[}4{]}{]})}}$

\begin{Verbatim}[commandchars=\\\{\}]
\PYG{n}{w}\PYG{o}{*}\PYG{n}{w}
\end{Verbatim}
\begin{equation*}
\begin{split}0\end{split}
\end{equation*}
$_{\text{Type: DeRhamComplex(Integer,{[}x{[}1{]},x{[}2{]},x{[}3{]},x{[}4{]}{]})}}$

\begin{Verbatim}[commandchars=\\\{\}]
\PYG{n}{i}\PYG{p}{(}\PYG{n}{vf}\PYG{p}{,}\PYG{n}{w}\PYG{p}{)}
\end{Verbatim}
\begin{equation*}
\begin{split}{{x _ {1}} \  {{b _ {2}}
\left(
{{x _ {1}}, \: {x _ {2}}, \: {x _ {3}}, \: {x _ {4}}}
\right)}}
-{{x _ {2}} \  {{b _ {1}}
\left(
{{x _ {1}}, \: {x _ {2}}, \: {x _ {3}}, \: {x _ {4}}}
\right)}}\end{split}
\end{equation*}
$_{\text{Type: DeRhamComplex(Integer,{[}x{[}1{]},x{[}2{]},x{[}3{]},x{[}4{]}{]})}}$

\begin{Verbatim}[commandchars=\\\{\}]
\PYG{n}{L}\PYG{p}{(}\PYG{n}{vf}\PYG{p}{,}\PYG{n}{w}\PYG{p}{)}
\end{Verbatim}
\begin{equation*}
\begin{split}{{\left( {{x _ {1}} \  {{{b _ {2}} _ {{,4}}}
\left(
{{x _ {1}}, \: {x _ {2}}, \: {x _ {3}}, \: {x _ {4}}}
\right)}}
-{{x _ {2}} \  {{{b _ {1}} _ {{,4}}}
\left(
{{x _ {1}}, \: {x _ {2}}, \: {x _ {3}}, \: {x _ {4}}}
\right)}}
\right)}
\  {dx _ {4}}}+ \\ {{\left( {{x _ {1}} \  {{{b _ {2}} _ {{,3}}}
\left(
{{x _ {1}}, \: {x _ {2}}, \: {x _ {3}}, \: {x _ {4}}}
\right)}}
-{{x _ {2}} \  {{{b _ {1}} _ {{,3}}}
\left(
{{x _ {1}}, \: {x _ {2}}, \: {x _ {3}}, \: {x _ {4}}}
\right)}}
\right)}
\  {dx _ {3}}}+ \\ {{\left( {{x _ {1}} \  {{{b _ {2}} _ {{,2}}}
\left(
{{x _ {1}}, \: {x _ {2}}, \: {x _ {3}}, \: {x _ {4}}}
\right)}}
-{{x _ {2}} \  {{{b _ {1}} _ {{,2}}}
\left(
{{x _ {1}}, \: {x _ {2}}, \: {x _ {3}}, \: {x _ {4}}}
\right)}}+{{b
_ {1}}
\left(
{{x _ {1}}, \: {x _ {2}}, \: {x _ {3}}, \: {x _ {4}}}
\right)}
\right)}
\  {dx _ {2}}}+ \\ {{\left( {{x _ {1}} \  {{{b _ {2}} _ {{,1}}}
\left(
{{x _ {1}}, \: {x _ {2}}, \: {x _ {3}}, \: {x _ {4}}}
\right)}}
-{{x _ {2}} \  {{{b _ {1}} _ {{,1}}}
\left(
{{x _ {1}}, \: {x _ {2}}, \: {x _ {3}}, \: {x _ {4}}}
\right)}}
-{{b _ {2}}
\left(
{{x _ {1}}, \: {x _ {2}}, \: {x _ {3}}, \: {x _ {4}}}
\right)}
\right)}
\  {dx _ {1}}}\end{split}
\end{equation*}
$_{\text{Type: DeRhamComplex(Integer,{[}x{[}1{]},x{[}2{]},x{[}3{]},x{[}4{]}{]})}}$

\begin{Verbatim}[commandchars=\\\{\}]
\PYG{n}{d} \PYG{n}{i}\PYG{p}{(}\PYG{n}{vf}\PYG{p}{,}\PYG{n}{w}\PYG{p}{)} \PYG{o}{+} \PYG{n}{i}\PYG{p}{(}\PYG{n}{vf}\PYG{p}{,}\PYG{n}{d} \PYG{n}{w}\PYG{p}{)}
\end{Verbatim}
\begin{equation*}
\begin{split}{{\left( {{x _ {1}} \  {{{b _ {2}} _ {{,4}}}
\left(
{{x _ {1}}, \: {x _ {2}}, \: {x _ {3}}, \: {x _ {4}}}
\right)}}
-{{x _ {2}} \  {{{b _ {1}} _ {{,4}}}
\left(
{{x _ {1}}, \: {x _ {2}}, \: {x _ {3}}, \: {x _ {4}}}
\right)}}
\right)}
\  {dx _ {4}}}+ \\ {{\left( {{x _ {1}} \  {{{b _ {2}} _ {{,3}}}
\left(
{{x _ {1}}, \: {x _ {2}}, \: {x _ {3}}, \: {x _ {4}}}
\right)}}
-{{x _ {2}} \  {{{b _ {1}} _ {{,3}}}
\left(
{{x _ {1}}, \: {x _ {2}}, \: {x _ {3}}, \: {x _ {4}}}
\right)}}
\right)}
\  {dx _ {3}}}+ \\ {{\left( {{x _ {1}} \  {{{b _ {2}} _ {{,2}}}
\left(
{{x _ {1}}, \: {x _ {2}}, \: {x _ {3}}, \: {x _ {4}}}
\right)}}
-{{x _ {2}} \  {{{b _ {1}} _ {{,2}}}
\left(
{{x _ {1}}, \: {x _ {2}}, \: {x _ {3}}, \: {x _ {4}}}
\right)}}+{{b
_ {1}}
\left(
{{x _ {1}}, \: {x _ {2}}, \: {x _ {3}}, \: {x _ {4}}}
\right)}
\right)}
\  {dx _ {2}}}+ \\ {{\left( {{x _ {1}} \  {{{b _ {2}} _ {{,1}}}
\left(
{{x _ {1}}, \: {x _ {2}}, \: {x _ {3}}, \: {x _ {4}}}
\right)}}
-{{x _ {2}} \  {{{b _ {1}} _ {{,1}}}
\left(
{{x _ {1}}, \: {x _ {2}}, \: {x _ {3}}, \: {x _ {4}}}
\right)}}
-{{b _ {2}}
\left(
{{x _ {1}}, \: {x _ {2}}, \: {x _ {3}}, \: {x _ {4}}}
\right)}
\right)}
\  {dx _ {1}}}\end{split}
\end{equation*}
$_{\text{Type: DeRhamComplex(Integer,{[}x{[}1{]},x{[}2{]},x{[}3{]},x{[}4{]}{]})}}$

\begin{Verbatim}[commandchars=\\\{\}]
\PYG{o}{\PYGZpc{}} \PYG{o}{\PYGZhy{}} \PYG{n}{L}\PYG{p}{(}\PYG{n}{vf}\PYG{p}{,}\PYG{n}{w}\PYG{p}{)}
\end{Verbatim}
\begin{equation*}
\begin{split}0\end{split}
\end{equation*}
$_{\text{Type: DeRhamComplex(Integer,{[}x{[}1{]},x{[}2{]},x{[}3{]},x{[}4{]}{]})}}$

\begin{Verbatim}[commandchars=\\\{\}]
dot(g,w,w)\PYGZdl{}M
\end{Verbatim}
\begin{equation*}
\begin{split}{{{x _ {2}}} ^ {2}}+{{{x _ {1}}} ^ {2}}\end{split}
\end{equation*}
$_{\text{Type: Expression(Integer)}}$

\begin{Verbatim}[commandchars=\\\{\}]
\PYG{n}{d} \PYG{n}{i}\PYG{p}{(}\PYG{n}{vf}\PYG{p}{,}\PYG{n}{dV}\PYG{p}{(}\PYG{n}{g}\PYG{p}{)}\PYG{p}{)} \PYG{o}{\PYGZhy{}}\PYG{o}{\PYGZhy{}} \PYG{n}{div}\PYG{p}{(}\PYG{n}{b}\PYG{p}{)} \PYG{n}{dV}
\end{Verbatim}
\begin{equation*}
\begin{split}\def\sp{^}\def\sb{_}\def\leqno(#1){}
\def\erf{\mathrm{erf}}\def\sinh{\mathrm{sinh}}
\def\zag#1#2{{{\left.{#1}\right|}\over{\left|{#2}\right.}}}
\def\csch{\mathrm{csch}}
{\left( {{{b \sb {4}} \sb {{,4}}}
\left(
{{x \sb {1}}, \: {x \sb {2}}, \: {x \sb {3}}, \: {x \sb {4}}}
\right)}+{{{b
\sb {3}} \sb {{,3}}}
\left(
{{x \sb {1}}, \: {x \sb {2}}, \: {x \sb {3}}, \: {x \sb {4}}}
\right)}+{{{b
\sb {2}} \sb {{,2}}}
\left(
{{x \sb {1}}, \: {x \sb {2}}, \: {x \sb {3}}, \: {x \sb {4}}}
\right)}+{{{b
\sb {1}} \sb {{,1}}}
\left(
{{x \sb {1}}, \: {x \sb {2}}, \: {x \sb {3}}, \: {x \sb {4}}}
\right)}
\right)}
\  {dx \sb {1}} \  {dx \sb {2}} \  {dx \sb {3}} \  {dx \sb {4}}\end{split}
\end{equation*}
$_{\text{Type: DeRhamComplex(Integer,{[}x{[}1{]},x{[}2{]},x{[}3{]},x{[}4{]}{]})}}$

\begin{Verbatim}[commandchars=\\\{\}]
d (P*one()\PYGZdl{}M) \PYGZhy{}\PYGZhy{} One()?
\end{Verbatim}
\begin{equation*}
\begin{split}{{{P _ {{,4}}}
\left(
{{x _ {1}}, \: {x _ {2}}, \: {x _ {3}}, \: {x _ {4}}}
\right)}
\  {dx _ {4}}}+{{{P _ {{,3}}}
\left(
{{x _ {1}}, \: {x _ {2}}, \: {x _ {3}}, \: {x _ {4}}}
\right)}
\  {dx _ {3}}}+ \\ {{{P _ {{,2}}}
\left(
{{x _ {1}}, \: {x _ {2}}, \: {x _ {3}}, \: {x _ {4}}}
\right)}
\  {dx _ {2}}}+{{{P _ {{,1}}}
\left(
{{x _ {1}}, \: {x _ {2}}, \: {x _ {3}}, \: {x _ {4}}}
\right)}
\  {dx _ {1}}}\end{split}
\end{equation*}
$_{\text{Type: DeRhamComplex(Integer,{[}x{[}1{]},x{[}2{]},x{[}3{]},x{[}4{]}{]})}}$

\begin{Verbatim}[commandchars=\\\{\}]
\PYG{n}{i}\PYG{p}{(}\PYG{n}{vf}\PYG{p}{,}\PYG{o}{\PYGZpc{}}\PYG{p}{)}
\end{Verbatim}
\begin{equation*}
\begin{split}{{{b _ {1}}
\left(
{{x _ {1}}, \: {x _ {2}}, \: {x _ {3}}, \: {x _ {4}}}
\right)}
\  {{P _ {{,1}}}
\left(
{{x _ {1}}, \: {x _ {2}}, \: {x _ {3}}, \: {x _ {4}}}
\right)}}+ \\ {{{b
_ {2}}
\left(
{{x _ {1}}, \: {x _ {2}}, \: {x _ {3}}, \: {x _ {4}}}
\right)}
\  {{P _ {{,2}}}
\left(
{{x _ {1}}, \: {x _ {2}}, \: {x _ {3}}, \: {x _ {4}}}
\right)}}+ \\ {{{b
_ {3}}
\left(
{{x _ {1}}, \: {x _ {2}}, \: {x _ {3}}, \: {x _ {4}}}
\right)}
\  {{P _ {{,3}}}
\left(
{{x _ {1}}, \: {x _ {2}}, \: {x _ {3}}, \: {x _ {4}}}
\right)}}+ \\ {{{b
_ {4}}
\left(
{{x _ {1}}, \: {x _ {2}}, \: {x _ {3}}, \: {x _ {4}}}
\right)}
\  {{P _ {{,4}}}
\left(
{{x _ {1}}, \: {x _ {2}}, \: {x _ {3}}, \: {x _ {4}}}
\right)}}\end{split}
\end{equation*}
$_{\text{Type: DeRhamComplex(Integer,{[}x{[}1{]},x{[}2{]},x{[}3{]},x{[}4{]}{]})}}$

\begin{Verbatim}[commandchars=\\\{\}]
1/dot(g,w,w)\PYGZdl{}M*w
\end{Verbatim}
\begin{equation*}
\begin{split}{{{x _ {1}} \over {{{{x _ {2}}} ^ {2}}+{{{x _ {1}}} ^ {2}}}} \  {dx
_ {2}}} -{{{x _ {2}} \over {{{{x _ {2}}} ^ {2}}+{{{x _ {1}}} ^
{2}}}} \  {dx _ {1}}}\end{split}
\end{equation*}
$_{\text{Type: DeRhamComplex(Integer,{[}x{[}1{]},x{[}2{]},x{[}3{]},x{[}4{]}{]})}}$

\begin{Verbatim}[commandchars=\\\{\}]
\PYG{n}{d} \PYG{o}{\PYGZpc{}}
\end{Verbatim}
\begin{equation*}
\begin{split}0\end{split}
\end{equation*}
$_{\text{Type: DeRhamComplex(Integer,{[}x{[}1{]},x{[}2{]},x{[}3{]},x{[}4{]}{]})}}$

\begin{Verbatim}[commandchars=\\\{\}]
\PYG{n}{s}\PYG{p}{:}\PYG{o}{=}\PYG{n}{zeroForm}\PYG{p}{(}\PYG{l+s+s1}{\PYGZsq{}}\PYG{l+s+s1}{s)\PYGZdl{}M}
\end{Verbatim}
\begin{equation*}
\begin{split}s
\left(
{{x _ {1}}, \: {x _ {2}}, \: {x _ {3}}, \: {x _ {4}}}
\right)\end{split}
\end{equation*}
$_{\text{Type: DeRhamComplex(Integer,{[}x{[}1{]},x{[}2{]},x{[}3{]},x{[}4{]}{]})}}$

\begin{Verbatim}[commandchars=\\\{\}]
\PYG{n}{d} \PYG{n}{s}
\end{Verbatim}
\begin{equation*}
\begin{split}{{{s _ {{,4}}}
\left(
{{x _ {1}}, \: {x _ {2}}, \: {x _ {3}}, \: {x _ {4}}}
\right)}
\  {dx _ {4}}}+{{{s _ {{,3}}}
\left(
{{x _ {1}}, \: {x _ {2}}, \: {x _ {3}}, \: {x _ {4}}}
\right)}
\  {dx _ {3}}}+ \\ {{{s _ {{,2}}}
\left(
{{x _ {1}}, \: {x _ {2}}, \: {x _ {3}}, \: {x _ {4}}}
\right)}
\  {dx _ {2}}}+{{{s _ {{,1}}}
\left(
{{x _ {1}}, \: {x _ {2}}, \: {x _ {3}}, \: {x _ {4}}}
\right)}
\  {dx _ {1}}}\end{split}
\end{equation*}
$_{\text{Type: DeRhamComplex(Integer,{[}x{[}1{]},x{[}2{]},x{[}3{]},x{[}4{]}{]})}}$

\begin{Verbatim}[commandchars=\\\{\}]
\PYG{n}{d} \PYG{p}{(}\PYG{o}{*}\PYG{o}{*} \PYG{n}{s}\PYG{p}{)}
\end{Verbatim}
\begin{equation*}
\begin{split}0\end{split}
\end{equation*}
$_{\text{Type: DeRhamComplex(Integer,{[}x{[}1{]},x{[}2{]},x{[}3{]},x{[}4{]}{]})}}$

\begin{Verbatim}[commandchars=\\\{\}]
\PYG{o}{*}\PYG{o}{*} \PYG{p}{(} \PYG{n}{d} \PYG{n}{s}\PYG{p}{)}
\end{Verbatim}
\begin{equation*}
\begin{split}{{{s _ {{,1}}}
\left(
{{x _ {1}}, \: {x _ {2}}, \: {x _ {3}}, \: {x _ {4}}}
\right)}
\  {dx _ {2}} \  {dx _ {3}} \  {dx _ {4}}} -{{{s _ {{,2}}}
\left(
{{x _ {1}}, \: {x _ {2}}, \: {x _ {3}}, \: {x _ {4}}}
\right)}
\  {dx _ {1}} \  {dx _ {3}} \  {dx _ {4}}}+ \\ {{{s _ {{,3}}}
\left(
{{x _ {1}}, \: {x _ {2}}, \: {x _ {3}}, \: {x _ {4}}}
\right)}
\  {dx _ {1}} \  {dx _ {2}} \  {dx _ {4}}} -{{{s _ {{,4}}}
\left(
{{x _ {1}}, \: {x _ {2}}, \: {x _ {3}}, \: {x _ {4}}}
\right)}
\  {dx _ {1}} \  {dx _ {2}} \  {dx _ {3}}}\end{split}
\end{equation*}
$_{\text{Type: DeRhamComplex(Integer,{[}x{[}1{]},x{[}2{]},x{[}3{]},x{[}4{]}{]})}}$

\begin{Verbatim}[commandchars=\\\{\}]
\PYG{n}{d} \PYG{p}{(}\PYG{o}{*}\PYG{o}{*} \PYG{p}{(} \PYG{n}{d} \PYG{n}{s}\PYG{p}{)}\PYG{p}{)} \PYG{o}{\PYGZhy{}}\PYG{o}{\PYGZhy{}} \PYG{n}{Laplacian}\PYG{p}{(}\PYG{n}{s}\PYG{p}{)} \PYG{n}{dV}
\end{Verbatim}
\begin{equation*}
\begin{split}{\left( {{s \sb {{{,1}{,1}}}}
\left(
{{x \sb {1}}, \: {x \sb {2}}, \: {x \sb {3}}, \: {x \sb {4}}}
\right)}+{{s
\sb {{{,2}{,2}}}}
\left(
{{x \sb {1}}, \: {x \sb {2}}, \: {x \sb {3}}, \: {x \sb {4}}}
\right)}+{{s
\sb {{{,3}{,3}}}}
\left(
{{x \sb {1}}, \: {x \sb {2}}, \: {x \sb {3}}, \: {x \sb {4}}}
\right)}+{{s
\sb {{{,4}{,4}}}}
\left(
{{x \sb {1}}, \: {x \sb {2}}, \: {x \sb {3}}, \: {x \sb {4}}}
\right)}
\right)}
\  {dx \sb {1}} \  {dx \sb {2}} \  {dx \sb {3}} \  {dx \sb {4}}\end{split}
\end{equation*}
$_{\text{Type: DeRhamComplex(Integer,{[}x{[}1{]},x{[}2{]},x{[}3{]},x{[}4{]}{]})}}$

\begin{Verbatim}[commandchars=\\\{\}]
r:=sin(x.1*x.2)*one()\PYGZdl{}M
\end{Verbatim}
\begin{equation*}
\begin{split}\sin
\left(
{{{x _ {1}} \  {x _ {2}}}}
\right)\end{split}
\end{equation*}
$_{\text{Type: DeRhamComplex(Integer,{[}x{[}1{]},x{[}2{]},x{[}3{]},x{[}4{]}{]})}}$

\begin{Verbatim}[commandchars=\\\{\}]
\PYG{n}{d} \PYG{n}{r}
\end{Verbatim}
\begin{equation*}
\begin{split}{{x _ {1}} \  {\cos
\left(
{{{x _ {1}} \  {x _ {2}}}}
\right)}
\  {dx _ {2}}}+{{x _ {2}} \  {\cos
\left(
{{{x _ {1}} \  {x _ {2}}}}
\right)}
\  {dx _ {1}}}\end{split}
\end{equation*}
$_{\text{Type: DeRhamComplex(Integer,{[}x{[}1{]},x{[}2{]},x{[}3{]},x{[}4{]}{]})}}$

\begin{Verbatim}[commandchars=\\\{\}]
\PYG{n}{d} \PYG{p}{(}\PYG{o}{*}\PYG{o}{*} \PYG{p}{(} \PYG{n}{d} \PYG{n}{r}\PYG{p}{)}\PYG{p}{)}
\end{Verbatim}
\begin{equation*}
\begin{split}{\left( -{{{x _ {2}}} ^ {2}} -{{{x _ {1}}} ^ {2}}
\right)}
\  {\sin
\left(
{{{x _ {1}} \  {x _ {2}}}}
\right)}
\  {dx _ {1}} \  {dx _ {2}} \  {dx _ {3}} \  {dx _ {4}}\end{split}
\end{equation*}
$_{\text{Type: DeRhamComplex(Integer,{[}x{[}1{]},x{[}2{]},x{[}3{]},x{[}4{]}{]})}}$

\begin{Verbatim}[commandchars=\\\{\}]
\PYG{o}{*}\PYG{o}{*} \PYG{p}{(}\PYG{n}{d} \PYG{p}{(}\PYG{o}{*}\PYG{o}{*} \PYG{p}{(} \PYG{n}{d} \PYG{n}{r}\PYG{p}{)}\PYG{p}{)}\PYG{p}{)}
\end{Verbatim}
\begin{equation*}
\begin{split}{\left( -{{{x _ {2}}} ^ {2}} -{{{x _ {1}}} ^ {2}}
\right)}
\  {\sin
\left(
{{{x _ {1}} \  {x _ {2}}}}
\right)}\end{split}
\end{equation*}
$_{\text{Type: DeRhamComplex(Integer,{[}x{[}1{]},x{[}2{]},x{[}3{]},x{[}4{]}{]})}}$

\begin{Verbatim}[commandchars=\\\{\}]
\PYG{o}{*}\PYG{o}{*} \PYG{p}{(}\PYG{n}{d} \PYG{p}{(}\PYG{o}{*}\PYG{o}{*} \PYG{p}{(} \PYG{n}{d} \PYG{n}{r}\PYG{p}{)}\PYG{p}{)}\PYG{p}{)}\PYG{p}{:}\PYG{p}{:}\PYG{n}{EXPR} \PYG{n}{INT}
\end{Verbatim}
\begin{equation*}
\begin{split}{\left( -{{{x _ {2}}} ^ {2}} -{{{x _ {1}}} ^ {2}}
\right)}
\  {\sin
\left(
{{{x _ {1}} \  {x _ {2}}}}
\right)}\end{split}
\end{equation*}
$_{\text{Type: Expression(Integer)}}$

\begin{Verbatim}[commandchars=\\\{\}]
\PYG{n+nb}{eval}\PYG{p}{(}\PYG{o}{\PYGZpc{}}\PYG{p}{,}\PYG{n}{xs}\PYG{o}{.}\PYG{l+m+mi}{1}\PYG{o}{=}\PYG{o}{\PYGZpc{}}\PYG{n}{pi}\PYG{p}{)}
\end{Verbatim}
\begin{equation*}
\begin{split}{\left( -{{\pi} ^ {2}} -{{{x _ {2}}} ^ {2}}
\right)}
\  {\sin
\left(
{{{x _ {2}} \  \pi}}
\right)}\end{split}
\end{equation*}
$_{\text{Type: Expression(Integer)}}$

\begin{Verbatim}[commandchars=\\\{\}]
\PYG{n+nb}{eval}\PYG{p}{(}\PYG{o}{\PYGZpc{}}\PYG{p}{,}\PYG{n}{xs}\PYG{o}{.}\PYG{l+m+mi}{2}\PYG{o}{=}\PYG{o}{\PYGZpc{}}\PYG{n}{pi}\PYG{o}{/}\PYG{l+m+mi}{3}\PYG{p}{)}
\end{Verbatim}
\begin{equation*}
\begin{split}-{{{10} \  {{\pi} ^ {2}} \  {\sin
\left(
{{{{\pi} ^ {2}} \over 3}}
\right)}}
\over 9}\end{split}
\end{equation*}
$_{\text{Type: Expression(Integer)}}$

\begin{Verbatim}[commandchars=\\\{\}]
a(P)*one()\PYGZdl{}M
\end{Verbatim}
\begin{equation*}
\begin{split}a
\left(
{{P
\left(
{{x _ {1}}, \: {x _ {2}}, \: {x _ {3}}, \: {x _ {4}}}
\right)}}
\right)\end{split}
\end{equation*}
$_{\text{Type: DeRhamComplex(Integer,{[}x{[}1{]},x{[}2{]},x{[}3{]},x{[}4{]}{]})}}$

\begin{Verbatim}[commandchars=\\\{\}]
d (a(P)*one()\PYGZdl{}M) \PYGZhy{}\PYGZhy{} chain diff
\end{Verbatim}
\begin{equation*}
\begin{split}{{{P _ {{,4}}}
\left(
{{x _ {1}}, \: {x _ {2}}, \: {x _ {3}}, \: {x _ {4}}}
\right)}
\  {{a _ {{\ }} ^ {,}}
\left(
{{P
\left(
{{x _ {1}}, \: {x _ {2}}, \: {x _ {3}}, \: {x _ {4}}}
\right)}}
\right)}
\  {dx _ {4}}}+ \\ {{{P _ {{,3}}}
\left(
{{x _ {1}}, \: {x _ {2}}, \: {x _ {3}}, \: {x _ {4}}}
\right)}
\  {{a _ {{\ }} ^ {,}}
\left(
{{P
\left(
{{x _ {1}}, \: {x _ {2}}, \: {x _ {3}}, \: {x _ {4}}}
\right)}}
\right)}
\  {dx _ {3}}}+ \\ {{{P _ {{,2}}}
\left(
{{x _ {1}}, \: {x _ {2}}, \: {x _ {3}}, \: {x _ {4}}}
\right)}
\  {{a _ {{\ }} ^ {,}}
\left(
{{P
\left(
{{x _ {1}}, \: {x _ {2}}, \: {x _ {3}}, \: {x _ {4}}}
\right)}}
\right)}
\  {dx _ {2}}}+ \\ {{{P _ {{,1}}}
\left(
{{x _ {1}}, \: {x _ {2}}, \: {x _ {3}}, \: {x _ {4}}}
\right)}
\  {{a _ {{\ }} ^ {,}}
\left(
{{P
\left(
{{x _ {1}}, \: {x _ {2}}, \: {x _ {3}}, \: {x _ {4}}}
\right)}}
\right)}
\  {dx _ {1}}}\end{split}
\end{equation*}
$_{\text{Type: DeRhamComplex(Integer,{[}x{[}1{]},x{[}2{]},x{[}3{]},x{[}4{]}{]})}}$


\chapter{Indices and tables}
\label{index:indices-and-tables}\begin{itemize}
\item {} 
\DUrole{xref,std,std-ref}{genindex}

\item {} 
\DUrole{xref,std,std-ref}{modindex}

\item {} 
\DUrole{xref,std,std-ref}{search}

\end{itemize}



\renewcommand{\indexname}{Index}
\printindex
\end{document}
