\documentclass[11pt,a4paper]{article}
\usepackage[utf8]{inputenc}
\usepackage[english]{babel}
\usepackage{amsmath}
\usepackage{amsfonts}
\usepackage{amssymb}
\usepackage{amsthm}
\usepackage{stmaryrd}
\usepackage[margin=0.8in]{geometry}

\begin{document}
\newcommand{\Norm}[1]{\vert\vert#1\vert\vert} % Norm
\newcommand{\Abs}[1]{\vert#1\vert} % Abs
\newcommand{\Dotp}[2]{\langle#1,#2\rangle} % Scalar product
\newcommand{\RR}{\mathbb{R}} % Reals
\newcommand{\FT}[1]{\int_{\RR^n} #1 e^{-i\Dotp{k}{x}} dm(x)} % Fourier transform
\newcommand{\IFT}[1]{\int_{\RR^n}\hat#1 e^{i\Dotp{k}{x}} dm(k)} % Inverse Fourier transform

\newtheorem{theorem}{Theorem}
\newtheorem{proposition}{Proposition}
\newtheorem{corollary}{Corollary}
\newtheorem{definition}{Definition}
\newcommand{\LL}{{\cal L}} % Lagrange

\newcommand{\Current}[1]{\llbracket#1\rrbracket}
\newcommand{\DD}{{\cal D}}

\title{Weakly calibrated currents}
\author{Kurt Pagani\\{\tt pagani@scios.ch}}
\date{\today}
\maketitle
\abstract{We show}

\section{Introduction}
Denote by $\DD_p(\RR^n)$ the space of infinitely differentiable $p$-forms on $\RR^n$ 
with compact support, and by
$\DD'_p(\RR^n)$ the corresponding space of $p$-currents (DeRham currents to be precise). 
One may think of Distributions with values in the finite dimensional Grassmann space
$\wedge_p \RR^n$, whose elements are the $p$-vectors
\[
   v = \tfrac{1}{p!}\sum_{\alpha} v^\alpha e_\alpha
     = \sum_{1\leq\alpha_1<\ldots<\alpha_p\leq n} v^{\alpha_1 \ldots \alpha_p} 
          e_1\wedge\ldots\wedge e_p,
\] 
where $e_1,\ldots,e_n$ denotes the standard basis of $\RR^n$.
For any $C\subset \DD_p(\RR^n)$ we define
\begin{equation} \label{PhiC}
\Phi_{C}(T)={\tt sup}\{T(\omega): \omega\in C\},
\end{equation}
for all $T\in \DD'_p(\RR^n)$. The set $\{T\in\DD'_p(\RR^n): |\Phi_C(T)|<\infty \}$ is 
called the domain of $\Phi_C$, as it is usual in convexity theory. It is easily seen that
$\Phi_C$ is a positively homogeneous, lower semicontinuous and convex function. Moreover,
it is the support function of the set $C$, thus we could assume that $C$ is a 
closed convex subset since the the support functions of $C$ and $\overline{{\tt conv}(C)}$ coincide. Recall that the boundary of a $p$-current is defined as the $(p-1)$-current
\[
 \partial T(\phi) = T(d\phi),
\]
so that for smooth objects the Stokes theorem results:
\[
   \int_{\partial T} \omega = \int_T d\omega.
\]
For such smooth objects that we can integrate over, we use the notation $\Current{.}$. For
instance, let $\gamma$ be a smooth curve joining the points $a,b\in \RR^n$, then
\[
   \Current{\gamma}(\omega)=\int_I \Dotp{\gamma'(t)}{(\omega\circ\gamma)(t)} \,dt
\]
and
\[
   \partial \Current{\gamma}(\phi)=\Current{\gamma}(d\phi)=\Current{b}(\phi)-\Current{a}(\phi)
   = \phi(b) - \phi(a).
\]
With that definition we have $ \partial \Current{\gamma}= \Current{\partial \gamma}$
in this special case, what
is not true generally, of course. Furthermore, the integrand $\Dotp{\gamma'(t)}{(\omega\circ\gamma)(t)} \,dt$ is just the pull-back of the one form $\omega$ by the mapping 
$\gamma:I \rightarrow \RR^n$, where $I=[a,b]$. Therefore we may also write
\[
   \Current{\gamma}(\omega)=(\gamma_\#\Current{I})(\omega)=\Current{I}(\gamma^{\#}\omega),
\]
where $\gamma_\#$ and $\gamma^{\#}$ denotes the push-forward and pull-back respectively.
We adopt this notation for any mapping, so that, for example, $G_{u,\#}\Current{\Omega}$
means integration over the graph of the mapping $u:\Omega \rightarrow \RR^n$.
\newpage
\begin{definition}
Let $1\leq p\leq n$, $C\subset {\DD}_p(\RR^n)$, $S\in  {\DD}'_{p-1}(\RR^n)$, then set
  \begin{equation}
    \alpha(C,S)= {\tt inf}\{\Phi_C(T): T\in {\DD}'_p(\RR^n), \partial T = S\},
  \end{equation}   
where $\Phi_C$ as in $(\ref{PhiC})$.
\end{definition}
By the well known saddle point property we have
\[
   \alpha(C,S)= \inf_{\partial T=S} \sup_C T(\omega) \geq \sup_C\inf_{\partial T=S} T(\omega),
\]
and if there is a pair\footnote{a saddle point actually}  $(T_0,\omega_0)$ such that
\[
       T_0(\omega) \leq T_0(\omega_0)\leq T(\omega_0)
\]
for all $(T,\omega)\in \{\partial T=S\}\times C$, then
\[
   \alpha(C,S)= \inf_{\partial T=S} \sup_C T(\omega) = 
      \sup_C\inf_{\partial T=S} T(\omega) = T_0(\omega_0).
\]
\begin{definition}
Let $1\leq p\leq n$, $C\subset {\DD}_p(\RR^n)$, $S\in  {\DD}'_{p-1}(\RR^n)$, then set
  \begin{equation}
    \beta(C,S)= {\tt sup}\{S(\phi): \phi\in {\DD}_{p-1}(\RR^n), d\phi\in C \}.
  \end{equation}   
\end{definition}
It may be the case that $C$ does not contain any exact forms, therefore, and because we use the extended real number system, let us agree to set
$\sup \emptyset=-\infty$ and $\inf \emptyset=\infty$. Note that $\beta$ is the support
function of the set $d^{-1}C$, thus $\beta(C,S)=\Phi_{d^{-1}C}(S)$. 
\begin{definition}
 A $p$-form $\omega_0$ is called a calibration for $C\subset {\DD}_p(\RR^n)$ if 
  \begin{itemize}
    \item[1.] $\omega_0\in C$,
    \item[2.] $d\omega_0=0$, and
    \item[3.] $\int_{\RR^n} \omega_0=0$ if $p=n$.
  \end{itemize}
 A $p$-current $T_0$ is said to be calibrated by $\omega_0$ if $T_0(\omega_0)=\Phi_C(T_0)$.
\end{definition}
Now, we can come to the main point: suppose $T_0$ is calibrated by $\omega_0$, then 
\[
   \Phi_C(T)-\Phi_C(T_0)=\Phi_C(T) - T_0(\omega_0)\geq T(\omega_0) - T_0(\omega_0)
     =\partial B(\omega_0)=B(d\omega_0)=0,
\]
thus
\[
   \Phi_C(T)\geq \Phi_C(T_0)
\]
for all $T$ such that $\partial T=\partial T_0$. Recall that the compactly supported
de Rham cohomology groups for $\RR^n$ are given by $H_c^p(\RR^n)=\delta_{p,n}{\RR}$, 
therefore we had to add item 3 in the definition of a calibration for $p=n$ in order
to guarantee that the difference $T-T_0$ is a boundary $\partial B$ when $p=n$. It can be 
omitted if one uses ${{\cal E}_p}(\RR^n)$ instead of ${\DD}_p(\RR^n)$.

\begin{definition}
We call a $p$-current $T_0$ weakly calibrated on $C\subset {\DD}_p(\RR^n)$ if there is
a sequence $\{d\phi_j\}_{j\in{\mathbb N}}\subset C$ of exact forms such that
\[
    \lim_{j\rightarrow\infty} T_0(d\phi_j)=\Phi_C(T_0).
\]
\end{definition} 
Obviously, every calibrated current is also weakly calibrated (choose $d\phi_j=\omega_0$ for
all $j$, since $\omega_0$ has to be exact on $\RR^n$). 

\newpage
\section{Results}
\begin{proposition}
Suppose $T_0\in {\DD}'_p(\RR^n)$ with $\partial T_0=S$ is weakly calibrated on 
 $C\subset {\DD}_p(\RR^n)$, then
 \begin{equation}
   \Phi_C(T_0)=\alpha(C,S)=\beta(C,S),
 \end{equation}
that is
\[
    \Phi_C(T_0) \leq \Phi_C(T)
\]
for all $T$ with the same boundary $S$.
\end{proposition}
\begin{proof}
For all $T$ with boundary $S$ holds $T(d\phi_j)=S(\phi_j)$ and we always have 
$\beta(C,S)\leq \alpha(C,S)$ because of
\[
       S(\phi)=\partial T(\phi)=T(d\phi)\leq \Phi_C(T),
\]
therefore
  $$\Phi_C(T)-\Phi_C(T_0)=\Phi_C(T) - \lim_{j\rightarrow\infty}T_0(d\phi_j)
  \geq \liminf_{j\rightarrow\infty}(T(d\phi_j)-T_0(d\phi_j))=0.$$
Now, since $\Phi_C(T_0)=\alpha(C,S)$ and 
$\lim_{j\rightarrow\infty}T_0(d\phi_j)=\lim_{j\rightarrow\infty}S(\phi_j)=\alpha(C,S)
\leq \beta(C,S)$ it follows necessarily that $\alpha=\beta.$
\end{proof}

The proof above shows that a necessary condition for a calibrated current (weakly or not) is
given by $\alpha(C,S)=\beta(C,S)$. Recalling the saddle point property mentioned in the 
introduction, it holds
\[
   \alpha(C,S)= \inf_{\partial T=S} \sup_C T(\omega) 
   \geq \sup_C\inf_{\partial T=S} T(\omega)\geq \beta(C,S),
\]
where all inequalities may be strict.




\begin{thebibliography}{9}

\bibitem{PP}
  L.E. Payne, G.A. Philippin
  \emph{On Maximum Principles for a Class of Nonlinear Second-Order Elliptic Equations}.
  Journal of Differential Equations,37,39-48,1980.

\bibitem{PP2}
  L.E. Payne
  \emph{"Best Possible" Maximum Principles}.
  Banach Center Publications,Vol. 11,
  PWN Polish Scientific Publishers, Warsaw 1985.
  
\bibitem{Sperb}
  R.P. Sperb
  \emph{Maximum Principles and their Applications}.
  Academic Press, 1981.

\bibitem{GT}
  D. Gilbarg, N.S. Trudinger
  \emph{Elliptic Partial Differential Equations of Second Order}.
  Classics in Mathematics,
  Springer Verlag.	
  
\bibitem{Rellich}
  F. Rellich
  \emph{Darstellung der Eigenwerte von $\Delta u + \lambda u=0$ durch ein Randintegral}.
  Math. Zeitschrift, 46,
  635-636, 1940.	

\end{thebibliography}




\end{document}