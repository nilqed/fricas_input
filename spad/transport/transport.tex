\documentclass[12pt,a4paper,draft]{article}
\usepackage[utf8]{inputenc}
\usepackage[english]{babel}
\usepackage{amsmath}
\usepackage{amsfonts}
\usepackage{amssymb}
\usepackage{amsthm}
\usepackage{graphicx}
\usepackage{stmaryrd}
\author{Kurt Pagani}
\title{Transport}
\newcommand{\curspace}[2]{\mathcal{D}_#1(\mathbb{R}^#2)}
\newcommand{\dfspace} [2]{\mathcal{D}^#1(\mathbb{R}^#2)}
\newcommand{\abs}[1]{\vert#1\vert}
\newcommand{\norm}[1]{\left\lVert#1\right\rVert}
\newcommand{\RR}[1]{\mathbb{R}^#1}
\newcommand{\Lp}[2]{L^#1(#2)}
\newcommand{\LL}[1]{L^2(\RR{#1})}
\newcommand{\dotp}[2]{\langle #1,#2 \rangle}
\newcommand{\Prob}[1]{\mathcal{P}({\RR #1})}
\newcommand{\Probc}[1]{\mathcal{P}({\RR #1 \times \RR #1})}
\newtheorem{theorem}{Theorem}
\newtheorem{lemma}{Lemma}
\newtheorem{prop}{Proposition}
\newtheorem{definition}{Definition}
\newtheorem{corollary}{Corollary}
\begin{document}
\maketitle
\section{Introduction and Notation}
In the sequel we are going to use a lot of results and terminology of
{\sl mass transportation theory}, whereby \cite{villani_topics_2003} serves
as the main reference.
\\ \\
\noindent
Let $\Prob{n}$ denote the space of probability measures on $\RR n$ and for
$\varphi\in\LL{n}$ let $\hat\varphi$ denote its (unitary) Fourier transform.
Each normalized $\varphi\in\LL{n}$ gives rise to a measure
\begin{displaymath}
   \nu_{\varphi}(f)=\int_{\RR n} f(x) |\varphi(x)|^2 \,dx,
\end{displaymath}
where $f\in C_{0}(\RR n)$, the continuous functions with compact support. Then
we define the mapping
\begin{equation}
   \mu : \LL{n}\cap\{||\varphi||=1\} \longrightarrow \Probc{n} 
\end{equation}
\begin{displaymath}
   \varphi \longmapsto \nu_{\varphi}\otimes\nu_{\hat\varphi},
\end{displaymath}
that means $\mu_{\varphi}$ is the (unique) product measure with marginals 
$\nu_{\varphi}$ and $\nu_{\hat\varphi}$. Furthermore we denote by $\Gamma(\varphi)$
the subset of $\Probc{n}$ whose elements have the aforementioned marginals.

Let $H:\RR{n}\times\RR{n}\longrightarrow\mathbb{R}$ be lower semi-continuous
and bounded below then we call
\begin{equation}
       K_H(\varphi) = \inf_{\gamma\in\Gamma(\varphi)} 
          \int_{\RR{n} \times \RR{n}} H(q,p) \,d\gamma(q,p)
\end{equation}\label{KantEnergy}
the {\sl Kantorovich} energy of $\varphi$. Similarly we call
\begin{equation}
      E_H(\varphi) = \int_{\RR{n} \times \RR{n}} H(q,p) \, d\mu_{\varphi}(q,p)
\end{equation}\label{SchroedEnergy}
the {\sl Schr\"odinger} energy, for reasons that will be enlightened further below. 
Monge's formulation of the {\sl optimal transport problem} reads in our case as
%a map $T:{\RR n}\rightarrow \RR{n}$ which minimizes the functional
\begin{equation}
   M_H(\varphi)=\inf_{T} \left\{\int_{\RR n} H(q,T(q) d\nu_{\varphi} : 
       T_\# \nu_{\varphi} = \nu_{\hat\varphi} \right\},
\end{equation}
which means to find a minimizing map $T:{\RR n}\rightarrow \RR{n}$ that 
transports the measure $\nu_\varphi$ to $\nu_{\hat\varphi}$ by pushing forward:
\begin{displaymath}
     T_\# \nu_{\varphi}(f):=\nu_{\varphi}(f\circ T)=
        \int_{\RR n} f(T(q)) |\varphi(q)|^2 \,dq 
          = \int_{\RR n} f(p) |\hat\varphi(p)|^2 \,dp.
\end{displaymath}
If all quantities involved were smooth enough and $T$ one to one then we would get the condition
\begin{equation}\label{MAEQ}
     |\varphi(q)|^2 = |\hat\varphi(T(q))|^2\, |\det{DT(q)}|
\end{equation}
by a simple change of coordinates. So it is by no means for sure whether 
admissible maps exist at all. 
%
\subsection{Schrödinger Energy}
Suppose $H$ has the familiar form $H(x,k)= \frac{\hbar^2 |k|^2}{2m}+V(x)$,
then we easily calculate that 
\begin{displaymath}
     K_H(\varphi) = E_H(\varphi)=
        \frac{\hbar^2}{2m} \int_{\RR n} |k|^2 |\hat\varphi(k)|^2  dk +   
        \int_{\RR n} V(x) |\varphi(x)|^2  dx
\end{displaymath}
holds. Furthermore, if $\partial_j\varphi\in L^2(\RR n)$  
then the above expression reduces to 
\begin{equation}
          K_H(\varphi) = E_H(\varphi)= \int_{\RR n}
              \left(  \frac{\hbar^2}{2m} |\nabla\varphi(x)|^2
              + V(x) |\varphi(x)|^2 \right) dx.
\end{equation}

%
Whether the energy is finite or not will also depend on the behavior of $V$ of
course. 
In a similar way the above deduction holds whenever the {\sl cost function} $H$
has the form $H(q,p)=T(p)+V(q)$ or even $H=T(p) V(q)$, that is the 
{\sl Kantorovich energy} coincides with $E_H$ which in turn means that the 
{\sl transference plan} $\mu_\varphi=\nu_{\varphi}\otimes\nu_{\hat\varphi}$ 
is optimal. Villani notes with reference to the sand pile example \cite{villani_topics_2003} 
\begin{verse}
    \ldots this corresponds to the most stupid transportation plan that one may 
    imagine: any piece of sand, regardless of its location, is distributed
    over the entire hole, proportionally to the depth.
\end{verse} 
He certainly would not claim that quantum mechanics is stupid, however, we 
recognize that the procedure mentioned is just another formulation of the
uncertainty principle (replacing sand pile/hole by position/momentum). This is in
strong contrast to the corresponding {\sl Monge} problem 
(omitting the factor $\hbar^2/2m$ from now on),
\begin{displaymath}
       M_H(\varphi)=\inf_{T} \left\{\int_{\RR n}
           \left( |T(x)|^2+V(x) \right) d\nu_{\varphi(x)} : 
       T_\# \nu_{\varphi} = \nu_{\hat\varphi}\right\},
\end{displaymath}
where, since $T$ is a map, there is no such distribution ({\sl mass} 
cannot be split by Monge transport). By a theorem of Brenier-McCann \cite{mccann_existence_1995},
there is a convex function $\phi$ on $\RR n$ such that
\begin{displaymath}
                  (\nabla\phi)_{\#}\nu_\varphi = \nu_{\hat\varphi},
\end{displaymath}
whence we have \footnote{notice that $K_H$ usually is a relaxation of $M_H$ since an admissible transport map $T$ always gives raise to a transference plan
$(id\times T)_{\#} \nu_{\varphi}\in\Gamma(\varphi)$} 
\begin{displaymath}
       K_H(\varphi)\leq M_H(\varphi) \leq \int_{\RR n}
           \left( |\nabla\phi(x)|^2+V(x) \right) |\varphi(x)|^2 dx. 
\end{displaymath}
Now, if we suppose for the moment the existence of a ground state $\varphi_0>0$
to $E_H$ (more precisely to the self-adjoint operator corresponding to $H$),
we would find the identities
\begin{displaymath}
    E_H(\varphi_0)=K_H(\varphi_0)= 
       \int_{\RR n}
           \left( |\nabla\log{\varphi_0(x)}|^2+V(x) \right) |\varphi_0(x)|^2 dx
           \leq M_H(\varphi_0). 
\end{displaymath}
This leads to the question: 
\begin{verse}
     Can $\nabla\phi = - \nabla\log{\varphi_0}$ be a {\sl Brenier} map?
\end{verse}
In the first place $\phi=- \log{\varphi_0}$ had to be convex, or equivalently,
the ground state $\varphi_0(x)=C e^{-\phi(x)}$ should be {\tt log-}concave,
a property that is not uncommon for certain potentials $V$. A far more
stringent condition, however, is the requirement 
$(-\nabla\log{\varphi_0})_{\#}\nu_{\varphi_0} = \nu_{\hat\varphi_0}$, which,
assuming some smoothness and recalling $(\ref{MAEQ})$, reads as
\begin{equation}
           |\varphi_0(q)|^2 = |\hat\varphi_0(\nabla\phi(q))|^2\, |\det{D^2\phi(q)}|.
\end{equation}
Actually, the ground state of the harmonic oscillator  
$\varphi_{ho}(x)=C e^{-\frac{1}{2}|x|^2 }$ satisfies the above equation and 
consequently in this particular case we have 
$K_H(\varphi_{ho})=E_H(\varphi_{ho})=M_H(\varphi_{ho})$ with transport map
$T(x)=\nabla\phi(x) = -\nabla\log{\varphi_{ho}(x)}=x$, the identity map.
Are there others? We do not know.
\subsection{General $H$}
If the Hamilton function $H$ does not split up as above then we only have
$K_H(\varphi)\leq E_H(\varphi)$ and since the infimum in (\ref{KantEnergy})
is always attained \footnote{under the conditions given at the beginning} , there is a $\gamma_{\varphi}\in\Gamma(\varphi)$ such that
$\gamma_{\varphi}(H) \leq \mu_{\varphi}(H)$. In the following let us denote
by $\Gamma_n = {\RR n}\oplus {\RR n}$ a $2n$-dimensional phase space, where
there should be no confusion among the meanings of $\Gamma$, e.g. we have $\Gamma(\varphi)\subset{\cal P}(\Gamma_n)$.
The minimization problem
\begin{equation}
                 \lambda_0 = \inf \left\{ \int_{\Gamma_n} H d\mu_\varphi: 
                       \varphi\in L^2(\RR n), ||\varphi||_2 = 1  \right\}
\end{equation}
may have or may not have a solution. This depends on the function $H$ under
consideration and even if there is a solution it is by no means granted that
it would be a ground state to a corresponding self-adjoint {\sl Hamiltonian}.
Existence questions will not be our concern at this point, therefore we will
take the existence of a minimizer $\varphi_0\in L^2(\RR n)$ for granted. Since
we have assumed the function $H$ to be bounded below (and l.s.c) it is obvious
that 
\begin{displaymath}
                  \lambda_0 \geq \inf_{\Gamma_n} H > - \infty
\end{displaymath} 
and moreover it holds that $\lambda_0=E_H(\varphi_0)\geq K_H(\varphi_0)$. When we define (assuming $H$ fixed) 
\begin{displaymath}
        F_{\varphi}(x) = \int_{\RR n} H(x,k)\, |\hat\varphi(k)|^2 dk
\end{displaymath} 
as well
\begin{displaymath}
        G_{\varphi}(k) = \int_{\RR n} H(x,k)\, |\varphi(x)|^2 dx
\end{displaymath}
and  recall that $\mu_{\varphi}=\nu_{\varphi}\otimes\nu_{\hat\varphi}$ holds by
definition, we obtain 
\begin{displaymath}
        E_H(\varphi)=\int_{\Gamma_n} H(x,k)\, d\mu_{\varphi}(x,k) =
           \int_{\RR n} F_{\varphi}(x) d\nu_{\varphi}(x) =
           \int_{\RR n} G_{\varphi}(k) d\nu_{\hat\varphi}(k). 
\end{displaymath} 
Now we may state the {\sl Euler equations} which a minimizer had to satisfy.
\begin{prop}
Let $\varphi_0\in L^2(\RR n)$ be a critical point of $E_H(\varphi)$, then
it satisfies the equation (in ${\cal D}'(\RR n)$)
\begin{equation}\label{EulerEq}
     \left(2 E_0 - F_{\varphi_0}(x)\right) \varphi_0(x) = 
         \int_{\RR n} G_{\varphi_0}(k)\, \hat\varphi_0(k)\, 
            e^{i\langle k,x \rangle}\, dm_n(k),
\end{equation} 
where $E_0 = E_H(\varphi_0)$ and $dm_n(k):=(2\pi)^{-\frac{n}{2}}\,dk$.
\end{prop}

Whether (\ref{EulerEq}) is valid almost everywhere w.r.t. Lebesgue measure 
depends (here again) on the function $H$. The inverse Fourier transform on
the right hand side should be understood symbolically, unless 
$G_{\varphi_0}\hat{\varphi_0}\in L^1(\RR n)$. In a compact notation the
equation for a critical point of $E_H$ is 
\begin{displaymath}
  F_{\varphi}\,\varphi+ \widehat{G_{\varphi}} \star\varphi = 2\lambda\,\varphi                     
\end{displaymath}
where the convolution is defined here as 
$(f\star g)(x)=\int_{\RR n} f(x-y)\,g(y)\,dm_n(y)$. It is easily checked that
in case of $H(x,k)=|k|^2+V(x)$, (\ref{EulerEq}) reduces to 
$(-\Delta+V(x)) \varphi_0=E_0 \varphi_0$. Our main interest, however, is
$H$ being the indicator function of an open subset of $\Gamma_n$ which is
obviously bounded, measurable and lower semi-continuous. This is, as will
be outlined farther below, connected to the question:
\begin{verse}
    How big can we make 
    \begin{equation}
             \int_{\Lambda} |\varphi(x)|^2 \, |\hat\varphi(k)|^2 dx\,dk,
    \end{equation}
    given a compact subset $\Lambda$ of phase space $\Gamma_n$?
\end{verse} 
Actually the question may be posed for $\Lambda\subset\Gamma_n$ having finite
Lebesgue measure.
\subsection{Duality}
One of the corner stones of mass transportation theory certainly is 
{\sl Kantorovich's} duality formula (\cite{villani_topics_2003}, Theorem 1.3)
which, translated to our needs, says
\begin{equation}
   K_H(\varphi) = \sup_{T(k)+V(x) \leq H(x,k)} 
              \left\{ \int_{\RR n} T(k) |\hat\varphi(k)|^2 \,dk +
                               \int_{\RR n} V(x) |\varphi(x)|^2 \,dx                         
                                      \right\},
\end{equation}
where the functions $T,V$ may either be any bounded continuous functions on $\RR n$
or by extension $(T,V)\in L^1(\nu_{\varphi})\times L^1(\nu_{\hat\varphi})$,
satisfying the inequality $T+V\leq H$ point-wise in the first case and almost
everywhere (with respect to the measures) in the second case. 
We cite one other result from \cite{villani_topics_2003} which will be needed
later on (a precursor of Strassen's theorem, Theorem 1.27): 
Let $U$ be a nonempty open subset of $\Gamma_n$, then
\begin{equation}\label{Strassen}
     \inf_{\gamma\in\Gamma(\varphi)} \int_{U} d\gamma =
       \sup_{A\subset{\RR n}} 
          \left\{ 
            \int_{A} |\varphi(x)|^2\,dx -
            \int_{A_U} |\hat\varphi(k)|^2\, dk : A\,\; closed 
          \right\},                                                 
\end{equation}
where $A_U:=\{k\in{\RR n}: \exists x\in A , (x,k)\notin U \}$. This result
implies, setting $H=\chi_U$,
\begin{displaymath}
  E_{\chi_U}(\varphi)\geq K_{\chi_U}(\varphi)=
        \sup\{ \nu_{\varphi}(A)-\nu_{\hat\varphi}(A_U)
            : A\subset{\RR n},A\,\;closed \}. 
\end{displaymath}
Note that we use the notation $\nu(A)$ and $\nu(\chi_A)$ interchangeably when
there is no danger of confusion (i.e. we identify a set with its indicator 
function).
%
\subsection{Symplectic transformations}
Let $M:\Gamma_n \rightarrow \Gamma_n$ be a symplectic transformation, represented
by a matrix of the form (we use the same symbol)
\begin{displaymath}
          M=M^{A,B,C,D}:=\begin{bmatrix}
          A & B \\ 
          C & D
          \end{bmatrix} 
\end{displaymath}
where the $n\times n$ block matrices $A,B,C,D$ satisfy the equations:
\begin{displaymath}
         \begin{array}{c}
          A^T D - C^T B = I \\
          A^T C = C^T A \\ 
          D^T B = B^T D. 
         \end{array} 
\end{displaymath}
Then we obtain for any $f\in C_0({\Gamma_n})$:
\begin{displaymath}
      M_{\#}\mu_{\varphi}(f) = \mu_{\varphi}(f\circ M) = 
         \int_{\Gamma_n} f(A x+B k, C x + D k) d\mu_{\varphi}(x,k).
\end{displaymath}
The inverse $M^{-1}$ of $M$ is easily calculated using the symplectic
condition $M^T J M = J$ to
\begin{displaymath}
           M^{-1}=J^{-1} M^T J = \begin{bmatrix}
          D^T & -B^T \\ 
          -C^T & A^T
          \end{bmatrix}
\end{displaymath}
which implies:
\begin{displaymath}
      M_{\#}\mu_{\varphi}(f) =  \int_{\Gamma_n} f(\xi,\eta)\,
      |\varphi(D^T \xi - B^T \eta)|^2 \,
      |\hat\varphi(-C^T \xi + A^T \eta)|^2 d\xi d\eta.       
\end{displaymath}
Simple examples (e.g. $n=1$ and $\phi(x)=C \exp(-\alpha |x|)$) show that 
we cannot expect the image measure $M_{\#}\mu_{\varphi}$ being an element of some
$\Gamma(\psi)$. However, two special cases immediately spring to mind:
\begin{displaymath}
          M^{A,0,0,D}_{\#}\mu_{\varphi}(f) =  \int_{\Gamma_n} f(\xi,\eta)\,
      |\varphi(D^T \xi|^2 \,
      |\hat\varphi(A^T \eta)|^2 d\xi d\eta. 
\end{displaymath}
and
\begin{displaymath}
       M^{0,B,C,0}_{\#}\mu_{\varphi}(f) =  \int_{\Gamma_n} f(\xi,\eta)\,
      |\varphi( - B^T \eta)|^2 \,
      |\hat\varphi(-C^T \xi)|^2 d\xi d\eta.
\end{displaymath}
In the first case we have $B=C=0$, so that $A^T D = I$ by the symplectic
conditions above. The second case requires $-C^T B=I$ by the same reasoning
since $A=D=0$. Hence there are two subgroups generated by matrices of the form
\begin{displaymath}
      \begin{bmatrix}
      A & 0 \\ 
      0 & A^{-T}
      \end{bmatrix}
and 
      \begin{bmatrix}
      0 & B \\ 
     -B^{-T} & 0
      \end{bmatrix}.        
\end{displaymath}
For these, the image measures are indeed of the form $d\mu_{\psi}$. If we use
the notation $\varphi_A(x)=\varphi(A x)$ we may state:
\begin{equation}
        M^{A,0,0,A^{-T}}_{\#}\mu_{\varphi} = \mu_{\varphi_{A^{-1}}}
\end{equation}
and
\begin{equation}
       M^{0,B,-B^{-T},0}_{\#}\mu_{\varphi} = \mu_{\widehat{\varphi_{B^{-1}}}}.
\end{equation}
This follows by straightforward computation. Finally we want to mention the 
special case $B=I_n$ (the identity matrix in $\RR n$), giving M=J, thus
\begin{displaymath}
        J_{\#}\mu_{\varphi}(H) = \mu_{\varphi}(H\circ J)=\mu_{\hat\varphi}(H),
\end{displaymath}
which is equivalent to $E_{H\circ J}(\varphi)=E_H({\hat\varphi})$. In other
words, if $H$ is invariant under the canonical transformation 
$x'=k, k'=-x$ and if $\varphi_0$ is a unique positive minimum of $E_H$, then 
$\varphi_0(x)=C \exp(-|x|^2/2)$.  
\subsection{Orthonormal Sequences in $L^2(\RR d)$}
Let $\{\varphi_j\}_{j\in J}$ be an orthonormal sequence in $L^2(\mathbb{R})$, then
a result by H. S. Shapiro, meanwhile known as {\sl Shapiro's Umbrella Theorem},
states that if given two functions $f(x)$ and $g(k)$ in $L^2(\mathbb{R})$ such that
\begin{displaymath}
         |\varphi_j(x)| \leq |f(x)|,\,\, |\hat\varphi_j(k)|\leq |g(k)|
\end{displaymath}
for all $j\in J$ and for almost all $x,k$ in $\mathbb{R}$ then $J$ must be
finite. We refer to \cite{jaming:hal-00080455} and the references therein for background information
and more details.
Recently. E. Mallinikova (\cite{malinnikova:2010},Th. 1.2) showed the following localization property of a
orthonormal sequecne $\{\varphi_j\}_{j=1}^N$: 
\begin{equation}\label{LocProp}
                N - |A| |B| \leq \frac{3}{2} \sum_{j=1}^N 
                   \left( \sqrt{\nu_{\varphi_j}({\RR d} \backslash A)} 
                      + \sqrt{\nu_{\hat\varphi_j}({\RR d} \backslash B)}                 
                   \right)
\end{equation}
where $A,B\subset {\RR d}$ are arbitrary measurable sets with finite Lebesgue
measure (i.e. $|A|,|B|<\infty$). Remembering the definition of the Radon measures
$\nu_{\varphi}$ at the beginning, $\nu_{\varphi}({\RR d} \backslash A)$ is
just 
\begin{displaymath}
          \int_{{\RR d} \backslash A} |\varphi(x)|^2\,dx.
\end{displaymath}  
The inequality (\ref{LocProp}) immediatley leads to a quantitative version of the 
{\sl Umbrella theorem} ([EM,Th. 4]) as well as to the general inequality
\begin{equation}
      \sum_{_j=1}^N \int_{\Gamma_d} \left( |x|^p+|k|^p  \right)\, 
               d\mu_{\varphi_j}(x,k) \geq C\, N^{1+\frac{p}{2 d}},
\end{equation} 
where $C$ depends only on $p>0$ and $d$. Moreover, it is also shown that the inequality is sharp up to a multiplicative constant. 
\subsection{The Nazarov-Jaming Inrequality}
Another important result we shall need is the following inequality obtained by
Nazarov for the case $d=1$ and extended by Jaming 
\cite{jaming:hal-00120268} to $d\geq 1$.

Let $A,B\subset {\RR d}$, each having finite Lebesgue measure, then there are
positive constants \footnote{Clearly, the constants may depend on the dimension 
$d$, although we do not explicitly outline this point by notation.}
$\alpha,\beta$ and $\eta(A,B)$ such that
\begin{equation}\label{JamingNazarov}
     \nu_{\varphi}({\RR d}\backslash A)+
     \hat\nu_{\varphi}({\RR d}\backslash B) \geq \alpha e^{-\beta \eta(A,B)}
\end{equation}
holds  for all $\varphi\in L^2(\RR d),\,||\varphi||=1$. The constant $\eta$
is given by
\begin{displaymath}
    \eta(A,B) = \left\{ \begin{array}{lr} |A| |B| & : d = 1
     \\ \min(|A|\,|B|,|A|^{1/d}\, w(B), w(A)\, |B|^{1/d}) & : d \ge 1 
     \end{array} \right.
\end{displaymath}
with $w(A)$ the {\sl average width} of $A$ (see \cite{jaming:hal-00120268} 
for the precise definition).

\subsection{Scaling}
For $\lambda>0$ let $\varphi_{\lambda}(x)$ denote the scaled function 
$\lambda^n \varphi(\lambda x)$, then $||\varphi_{\lambda}||=1$ whenever 
$\varphi\in L^2(\RR n)$ and $||\varphi||=1$. The Fourier transform 
$\widehat{\varphi_{\lambda}}$ of $\varphi_{\lambda}$ is easily calculated
to be equal to $\hat{\varphi}_{1/\lambda}$, therefore 
\begin{equation}
    \mu_{\varphi_{\lambda}}(f) = \int_{\Gamma_n} f(x,k) 
       \,|\varphi_{\lambda}(x)|^2 \, |\hat{\varphi}_{1/\lambda}(k)|^2 \,
       dx\,dk,
\end{equation}
for all $f\in C_0(\Gamma_n)$. The coordinate change $\xi=\lambda x, 
\eta=k/\lambda$ yields
\begin{equation}
    \mu_{\varphi_{\lambda}}(f) = \int_{\Gamma_n} f(\frac{\xi}{\lambda},
      \lambda\,\eta) 
       \,|\varphi(\xi)|^2 \, |\hat{\varphi}(\eta)|^2 \,
       d\xi\,d\eta = \int_{\Gamma_n}  f(\frac{\xi}{\lambda},
      \lambda\,\eta) d\mu_{\varphi}(\xi,\eta).
\end{equation}
Recall that $\mathrm{supp}(\mu_{\varphi})=
  \mathrm{supp}(\nu_{\varphi})\times \mathrm{supp}(\nu_{\hat\varphi})$, so we
get 
\begin{equation}
     \lim_{\lambda\rightarrow 0}  \mu_{\varphi_{\lambda}}(f)  =
       \int_{\RR n} f(x,0)\,dx
\end{equation}
and correspondingly
\begin{equation}
      \lim_{\lambda\rightarrow \infty}  \mu_{\varphi_{\lambda}}(f)  =
       \int_{\RR n} f(0,k)\,dk,
\end{equation}
that is $\mu_{\varphi_{1/j}}$ converges {\sl vaguely} 
to $(1\times\delta_0)\,dx\,dk$
while $\mu_{\varphi_{j}}$ tends to $(\delta_0\times 1)\,dx\,dk$ for 
$j\rightarrow \infty$. This behavior is best visualized when taking a function
$f\in C_0(\Gamma_n)$ whose support is the ball $\{|x|^2+|k|^2\leq 1\}$, then
the scaled support degenerates like an {\sl ellipse} with one half-axis equal to 
$\lambda$ and the other equal to $1/\lambda$ (which indeed is the case for $n=1$). 

Note: we say that a sequence of measures $\mu_{j}$ converges \sl vaguely
to a measure $\mu$ if $\mu_{j}(f)\rightarrow \mu(f)$ for all continuous
functions $f$ with compact support, whereas we speak of {\sl weak convergence}
if the functions $f$ are continuous and bounded.

\section{Maximal Probability of Closed Sets}
\begin{definition}
Let $\Lambda$ be a closed subset of $\Gamma_n$, then we define
\begin{equation}
     e(\Lambda) = \sup \left\{ \int_{\Lambda} d\mu_{\varphi} : 
                     \varphi\in L^2(\RR n), \, ||\varphi|| = 1 \right\}.
\end{equation}
\end{definition}

\begin{lemma}
Let $A,B$ be subsets of $\RR n$ having finite Lebesgue measure, that is $|A|+|B|<\infty$, then exists a $\psi$ such that 
\begin{equation}
          \nu_{\psi}(A)+\nu_{\hat\psi}(B) = 0.
\end{equation}
\end{lemma}

\begin{proof}
This follows by Corollary 2.5.A in \cite{havin_uncertainty_2012}. Actually it is
shown that there always is a $\varphi\in L^2(\RR n)$ such that for any given pair
$g,h$ of functions in $L^2(\RR n)$ the restriction of $\varphi$ to $A$ and that
of $\hat\varphi$ to $B$ coincides with the restriction of $g$ to $A$ and
$h$ to $B$ respectively. 
\end{proof}

\begin{prop}
Let each of $A,B$ be the complement of a bounded open subset in $\RR n$, then 
\begin{displaymath}
       e(A\times B) = 1.
\end{displaymath}
\end{prop}

\begin{proof}
Set $U={\RR n}\backslash A,\, V={\RR n}\backslash B$, then for each $\varphi$ we 
have $\mu_{\varphi}(A\times B)=\nu_{\varphi}(A)\,
\nu_{\hat\varphi}(B)=(1-\nu_{\varphi}(U))\, (1-\nu_{\hat\varphi}(V))$. By 
the lemma above we may choose a $\psi$ such that 
$\nu_{\psi}(A)=\nu_{\hat\psi}(B) = 0$, thus $\mu_{\psi}(A\times B)=1$.
\end{proof}

\begin{prop}
Let $A,B$ be subsets of $\RR n$ such that $|A|+|B|<\infty$, then for every 
normalized $\varphi\in L^2(\RR n)$
\begin{equation}\label{eUbound}
   \mu_{\varphi}(\chi_{A\times B}) \leq 
      \left( 1-\frac{\alpha}{2} e^{-\beta\, \eta(A,B)} \right)^2
\end{equation}
with constants $\alpha,\beta$ and $\eta$ as in (\ref{JamingNazarov}). 
\end{prop}

\begin{proof}
Using (\ref{JamingNazarov})  we get 
$2-\left(\nu_{\varphi}(A)+\nu_{\hat\varphi}(B)\right)\geq \alpha 
  e^{-\beta\,\eta(A,B)}$. Dividing both sides by two and applying the
  arithmetic-geometric mean inequality yields 
  $ \sqrt{\nu_{\varphi}(A)\nu_{\hat\varphi}(B)}\leq 
     1-\frac{\alpha}{2} e^{-\beta\, \eta(A,B)}$, what implies (\ref{eUbound}).
\end{proof}

\begin{corollary}
Let $\Lambda\subset\Gamma_n$ be compact, then
\begin{equation}
             e(\Lambda) \leq 
      \left( 1-\frac{\alpha}{2} e^{-\beta\, 
          \eta(\pi_1(\Lambda),\pi_2(\Lambda))} \right)^2,
\end{equation}
where $\pi_1,\pi_2:\Gamma_n \rightarrow {\RR n}$ are the standard projections
and the constants $\alpha,\beta,\eta$ are as in (\ref{JamingNazarov}). 
\end{corollary}

\begin{proof}
The images of the projections $\pi_1,\pi_2$ are again compact, thus measurable
and of finite Lebesgue measure. 
\end{proof}
\rm
If we replace compactness by finite Lebesgue measure or closed only then
we have to deal with analytic sets. 
Since 
\begin{displaymath}
    e(\Lambda)=\sup_{\varphi} \mu_{\varphi}(\Lambda)=
      1-\inf_{\varphi} \mu_{\varphi}(\Gamma_n\backslash\Lambda)
\end{displaymath}
we have the relation to $E_H$ with $H=\chi_{\Gamma_n\backslash\Lambda}$. Since
$U=\Gamma_n\backslash\Lambda$ is open we can also apply (\ref{Strassen}). 

\bibliographystyle{plain}
\bibliography{omt}
\end{document}