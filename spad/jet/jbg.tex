%==============================================================================
% Documentation (pdflatex this_file.spad
% Note: ensure that the code above is between \iffalse ... )if false\fi.
%==============================================================================
\documentclass[12pt,a4paper]{article}
\usepackage[utf8]{inputenc}
\usepackage[english]{babel}
\usepackage{amsmath}
\usepackage{amsfonts}
\usepackage{amssymb}
\usepackage{makeidx}
\usepackage{graphicx}
\usepackage{listings}
\usepackage{color}
\usepackage{epstopdf}
\usepackage{pgfplots}
%
\pgfplotsset{width=10cm,compat=1.9}
%
\definecolor{dkgreen}{rgb}{0,0.6,0}
\definecolor{gray}{rgb}{0.5,0.5,0.5}
\definecolor{mauve}{rgb}{0.58,0,0.82}
\definecolor{orange}{rgb}{1.0,0.44,0}
%
\lstdefinelanguage{SPAD}
{keywords={if,then,else,for,in,repeat},
keywords=[2]{Fraction, Integer, Polynomial, Expression, Float, 
             DoubleFloat, SurfaceComplex, CellMap, 
             List},
keywords=[3]{cellMap, bdry, tangentSpace, normalSpace, normalField,
             metricTensor, jacobianMatrix, faces, coords, coordSymbols},
keywordstyle=[2]\color{red},
keywordstyle=[3]\color{orange},
sensitive=true,%
alsoletter={\$},%
comment=[l]{--},%
string=[b]",%
string=[b]'%
}

\lstset{frame=tb,
  language=SPAD,
  aboveskip=3mm,
  belowskip=3mm,
  showstringspaces=false,
  columns=flexible,
  basicstyle={\small\ttfamily},
  numbers=none,
  numberstyle=\tiny\color{gray},
  keywordstyle=\color{blue},
  commentstyle=\color{dkgreen},
  stringstyle=\color{mauve},
  breaklines=true,
  breakatwhitespace=true,
  tabsize=3
}
\author{Kurt Pagani \\ {\tt nilqed@gmail.com}}
\date{December 20, 2016}
\title{JetBundleGeometry \\ {\small\tt Domain: JBG}}
%
\newcommand{\CAD}{{\tt CAD}}
\newcommand{\QE}{{\tt QE}}
\newcommand{\RR}[1]{\mathbb{R}^{#1}}
\newcommand{\QQ}[1]{\mathbb{Q}^{#1}}
\newcommand{\ZZ}[1]{\mathbb{Z}^{#1}}
\newcommand{\KK}[1]{\mathbb{K}^{#1}}
\newcommand{\spadfun}[1]{\textcolor{magenta}{\tt #1}}
\newcommand{\spadbold}[1]{{\tt\bf #1}}
\newcommand{\type}[1]{\textcolor{blue}{\tt\tiny #1}}
\newtheorem{definition}{Definition}
%
\begin{document}
\maketitle
%
\begin{abstract}
This manual decribes the FriCAS domain {\bf JetBundleGeometry}.
This domain extends the JET series and provides methods to compute
{\bf pull backs}, {\bf integrals} and other quantities of 
differential forms living on a Jet bundle. The domain relies on
{\tt JET} (\cite{wms:jet}) and some other domains and packages. 
\end{abstract}
%
\tableofcontents
\section{Introduction}
Recall that a bundle is a triple $(M,X,\pi)$, where $M,X$ are manifolds
and $\pi:M\rightarrow X$ is the projection, i.e. a continuous surjective
mapping from the total space ($M$) to the base space ($X$). A bundle is 
called trivial if $M$ is homeomorphic to $X\times U$, where
$U$ denotes another manifold, called the fibre. A locally trivial bundle
is called a fibre bundle, which is what we here are concerned with.
We will use the following nomenclature: 
\begin{displaymath}
   x=(x_1,\ldots,x_p), \ \ u=(u^1,\ldots,u^n)
\end{displaymath}
where $x$ are the (local) coordinates of the base manifold $X$ and $u$
the the ones on the fibre $U$. Locally the projection takes the form
\begin{displaymath}
 \pi:\begin{cases} 
      & X\times U \\
      & (x,u) \mapsto x.
   \end{cases}
\end{displaymath}
A section of the fibre bundle then has the form
\begin{displaymath}
 \Phi_f:\begin{cases} 
      & X\times U \\
      & x \mapsto (x,f(x)).
\end{cases}
\end{displaymath}
such that $\pi\circ \Phi_f = \mathcal{I}d$.
A jet bundle now may be considered as an iteration of that construction,
that is we will speak of a $n$-th order jet bundle $M^{(n)}$ if
\begin{displaymath}
 M^{(n)} = X \times \left( U\times U_1\times\ldots\times U_n  \right)
\end{displaymath}  
where the $U_j$ are the jet (Euclidean) spaces. We will just write $M$
for $M^{(0)}$.
\subsection{FriCAS JET} 
JET in FriCAS is a sophisticated series of domains and packages written
by Joachim Schue and Werner M. Seiler (\cite{wms:aci},\cite{wms:axiom},
\cite{wms:diss},\cite{wms:jet}). Everything is contained in the
source file {\tt jet.spad} ($7000$+ loc). At the moment JET comprises
the following types or packages:
\begin{itemize}
\item[1] \spadfun{CartanKuranishi} is a package for the completion of a 
given differential equation to an involutive equation. Procedures for 
Cartan characters and Hilbert polynomial are also provided. Based on the 
{\em Cartan-Kuranishi} theorem as it is used in formal theory.
%
\item[2] \spadfun{DistributedJetBundlePolynomial} implements polynomials 
in a distributed representation. The unknows come from a finite list of 
jet variables. The implementation is basically a copy of the one of 
{\em GeneralDistributedMultivariatePolynomial}. 
%
\item[3] \spadfun{IndexedJetBundle} provides the standard implementation 
for a jet bundle with a given number of dependent and independent 
variables where the variables are given as symbols with upper bounds 
on the indexes (that's why the prefix {\it Indexed}. Otherwise it is
the same as \spadfun{JetBundle}.
%
\item[4] \spadfun{JetBundle} implements a jet bundle of arbitrary order 
with given names for the independent and dependent variables. 
It supports only repeated index notation.
%
\item[5] \spadfun{JetBundleBaseFunctionCategory} defines the category of 
functions (local sections) of the base space of a jet bundle, i.e. 
functions depending only on the independent variables. Such a category 
is needed e.g. for the representation of solutions.
%
\item[6] \spadfun{JetBundleCategory} provides basic data structures and 
procedures for jet bundles. Nearly all necessary functions are implemented 
already here. Only the representation and functions which direct access 
to it must be implemented in a domain. Two notations of derivatives are 
supported. Default is multi-index notation, where the i-th entry of the 
index denotes the number of differentiations taken with respect to $x^i$. 
In repeated index notation each entry $i$ in the index denotes a 
differentiation with respect to $x^i$. The choice affects, however, 
only in- and output. Internally, multi-index notation is used throughout.
%
\item[7] \spadfun{JetBundleExpression} defines expressions over a jet 
bundle based on Expression Integer. It allows all kind of algebraic 
operations. simplify is implemented using Groebner bases in polynomials 
over kernels. Thus it might not work correctly for general expressions. 
This also affects dimension.
%
\item[8] \spadfun{JetBundleFunctionCategory} defines the category of 
functions (local sections) over a jet bundle. The formal derivative is 
defined already here. It uses the Jacobi matrix of the functions. The 
columns of the matrices are enumerated by jet variables. Thus they are 
represented as a Record of the matrix and a list of the jet variables. 
Several simplification routines are implemented already here.
%
\item[9] \spadfun{JetBundleLinearFunction} implements linear functions 
over a jet bundle. The coefficients are functions of the independent 
variables only.
%
\item[10] \spadfun{JetBundlePolynomial} implements polynomial sections 
over a jet bundle. The order is not fixed, thus jet variables of any 
order can appear.
%
\item[11] \spadfun{JetBundleSymAna} is only necessary to have a valid 
return type for some procedures in SymmetryAnalysis. It is essentially 
identical with JetBundle but computes its parameters in a more 
complicated way.
%
\item[12] \spadfun{JetBundleXExpression} implements arbitrary functions in 
a jet bundle which depend only on the independent variables $x$. Otherwise 
it is identical with {\tt JetBundleExpression}. Such a domain is needed 
for {\tt JetLinearFunction}.
%
\item[13] \spadfun{JetCoordinateTransformation} implements changes of 
local coordinates. Given are the changes of the coordinates of the base 
space, i.e. the independent and dependent variables. The transformations 
of the derivatives are computed via the chain rule. $Y (W)$ contains 
expressions for the old variables in terms of the new ones.
%
\item[14] \spadfun{JetDifferential} implements differentials (one-forms) 
over a jet bundle.The differentials operate on {\tt JetVectorField}.
%
\item[15] \spadfun{JetDifferentialEquation} provides the basic data 
structures and procedures for differential equations as needed in the 
geometric theory. Differential equation means here always a submanifold 
in the jet bundle. The concrete equations which define this submanifold 
are called system. In an object of the type JetDifferentialEquation much 
more than only the system is stored. $D$ denotes the class of functions 
allowed as equations. It is assumed that the simplify procedure of $D$ 
returns only independent equations and a system with symbol in row 
echelon form.
%
\item[16] \spadfun{JetGroebner} provides a procedure to compute 
{\em Groebner} bases for arbitrary domains of jet polynomials. 
Two internal procedures transform to and from 
DistributedJetBundlePolynomial where the actual computation is done. 
The argument LJV contains all jet variables effectively occurring in 
the polynomials. The ordering is determined by the ordering in $P$.
%
\item[17] \spadfun{JetLazyFunction} takes as argument a domain in 
JetBundleFunctionCategory and returns another domain in the same 
category. This domain has basically the same properties as the argument 
domain, but there is a lazy evaluation mechanism for derivatives. 
This means that differentiations are not immediately performed. 
Instead a pointer is established to the function to be differentiated. 
Only when the exact value of the derivative is needed, the differentiation 
is executed. Special care is taken for leading derivatives and jet 
variables to avoid as much as possible the need to evaluate expressions. 
This entails that the result of {\em jetVariables} may contain spurious 
variables. 
Furthermore many functions in {\em JetLazyFunction} destructively change 
their arguments. This affects, however, only their internal representation, 
not the value obtained after full evaluation.
%
\item[18] \spadfun{JetVectorField} implements vector fields over the 
jet bundle $JB$ with coefficients from $D$. The fields operate on functions 
from $D$.
%
\item[19] \spadfun{LUDecomposition} contains procedures to solve linear 
systems of equations or to compute inverses using a {\tt LU} decomposition.
%
\item[20] \spadfun{SparseEchelonMatrix} implements sparse matrices whose 
columns are enumerated by the {\tt OrderedSet} and whose entries belong 
to a {\tt Gcd} domain. The basic operation of this domain is the computation
of an row echelon form. The used algorithm tries to maintain the sparsity 
and is especially adapted to matrices who are already close to a row 
echelon form.
%
\item[21] \spadfun{SymmetryAnalysis} provides procedures for the symmetry 
analysis of differential equations over a given jet bundle. 
Currently there exist only some procedures to set up the determining 
system for the symmetry generators of {\em Lie} point symmetries.
\end{itemize}
The exported functions and how all plays together is best documented
in \cite{wms:axiom}. There you will also find comprehensive 
implementation notes as well as some examples.
\subsection{Motivation}
Let us recall the following domains/packages and their functionality:
\begin{itemize}
\item[1] \spadfun{DeRhamComplex} provides differential forms as a
graded ring in a given set of variables:
\begin{displaymath}
  \mathtt{DeRhamComplex(R,[x_1,\ldots,x_n])} =
      \bigoplus_{p=0}^n \Lambda^p([x_1,\ldots,x_n])
\end{displaymath}
where $x_j\in \mathtt{Expression(R)}$, $R$ a ring, meaning that the
coefficients are from {\tt Expression R}.
%
\item[2] \spadfun{DifferentialForms} is a package extending 
{\tt DeRhamComplex} by {\em Riemannian metrics} $g$ and
the connected standard operations like {\em Hodge star} $\star_g$,
$\delta_g$, $\Delta_g$, $i_X$, $\mathcal{L}_X$ and 
$\langle\cdot,\cdot\rangle_g$.
%
\item[3] \spadfun{CellMap} and \spadfun{SurfaceComplex} are domains
which loosely spoken will provide parametric surface patches and
their formal sums (free group with integer coefficients):
\begin{displaymath}
  \mathtt{SurfaceComplex(R,n)} =
      \mathbb{Z}\cdot\bigoplus_{p=0}^n \Sigma^p([\$_1,\ldots,\$_n])  
\end{displaymath}
where $\Sigma_p$ is the collection of $p-$cells, that is mappings
from a $p-$cell into $n-$space.

Consequently we can define the boundary operator $\partial$ such 
that $T(d\varphi)=\partial T(\varphi)$ holds, where $T$ is an
element of SurfaceComplex and $\varphi$ one of DeRhamComplex.
\end{itemize}
But note that the code of these domains is unrelated, that is
both can be used independently. So we are in need of a package
or domain which brings them together in order to utilize their 
mutual duality.

A second point is the need of computing {\em pull backs} of
differential forms and - closely related - of integration of
forms over {\em affine chains}. To achieve this we have
to incorporate two (usually) different spaces of differential
forms.

Eventually, the following quotations from \cite{wms:axiom} together
with the remarks above induced the usage of {\tt JetBundle}
instead of creating a new domain:
%
\begin{verse}
\ldots
The implementation of both is somewhat rudimentary, as we hope that some
day {\tt AXIOM} will contain a reasonable environment for differential 
geometric calculations and then it should be used instead of some 
special domains like the two mentioned. 
\ldots
The implementation of the differential geometric domains  {\tt VectorField} 
(abbreviated by {\tt VF}) and {\tt Differential (DIFF)} are fairly primitive. 
They should be considered as a temporary hack, until AXIOM contains 
a better environment for differential geometric calculations.
\ldots
\end{verse}  
%
We may interpret {\tt JetBundle} in the first place as a local
coordinate patch (chart) with independent variables $x$ and
dependent ones $u$. Then we will be able to pull forms in $u-$
space back to $x$-space. In other words we will only use a 
zero order jet bundle for the moment. In a second step the
interplay might be extended to higher order bundles such that
the deficiencies mentioned in the above quotes may be mitigated.
%
\section{Appendix}
The following is a verbatim inclusion of the extended abstract of
W.M.\,Seiler's paper.

\begin{center}
{\bf JET --- An AXIOM Environment for Geometric Computations with
Differential Equations} \\
{ W.M.~Seiler and J.~Calmet\\
        Institut f\"ur Algorithmen und Kognitive Systeme\\
        Universit\"at Karlsruhe\\
        76128 Karlsruhe, Germany\\
        Email: seilerw@ira.uka.de} \\
\bf Extended Abstract
\end{center}

Geometric methods play an important role in the analysis of nonlinear
differential equations. For example, symmetry methods provide the more or less
only systematic approach to the construction of solutions. However, most
geometric computations tend to be very tedious. Thus the use of computer algebra
systems considerably helps in the application of these methods.

{\small JET} is an environment within the computer algebra system {\small AXIOM}
to perform such computations. The current implementation emphasizes the two key
concepts involution and symmetry. It provides some packages for the completion
of a given system of differential equations to an equivalent involutive one
based on the Cartan-Kuranishi theorem and for setting up the determining
equations for classical and non-classical point symmetries.

We stress that {\small JET} is an {\em environment\/} for such computations and
not simply a collection of some special purpose algorithms. Thus it contains
general data structures for the jet bundle formalism which can also be used for
other tasks than the two above mentioned. Using the generic programming
facilities of {\small AXIOM} it is possible to provide several representations
for jet bundles and for different classes of differential equations. The main
application packages are independent of such details.

Involution has important applications in symmetry theory. One should e.g.\
mention that involutive systems are locally solvable and only for such systems
the two widely used definitions of a symmetry coincide. The in calculations
applied definition as a transformation leaving the differential equation
invariant yields for not locally solvable systems usually less symmetries then
the definition as a transformation mapping solutions into solutions.  This can
easily be seen with the system $u_{z}+yu_{x}=u_{y}=0$. Obviously it is not
invariant under $y$-translations, but its solution space $u=const$ has this
symmetry. This effect has especially implications on the nonclassical method of
Bluman and Cole.

Other applications include computing the size of the symmetry group of a
differential equation without solving the determining equations. Furthermore it
is possible to ``correct'' the result by subtracting some unwanted effects like
e.g.\ the trivial superposition symmetry of linear equations. In the case of
gauge theory the concept of involution leads to an new intrinsic definition for
the number of degrees of freedom based on a similar formal correction.

A brief description of an earlier version of {\small JET} can be found in
Ref.~\cite{wms:aci}. The current version is described in much detail in
Ref.~\cite{wms:axiom}. For more information about the underlying mathematical
theory we refer to Ref.~\cite{wms:diss}. Applications of the concept of
involution in symmetry theory are discussed in Ref.~\cite{wms:mcm}. Finally
we mention Ref.~\cite{wms:con1} for some applications in physics. All these
publications can be obtained via {\small WWW} at the {\small URL}:\\
{\tt
http://www.mathematik.uni-kassel.de/\textasciitilde  seiler/}. 
\footnote{link updated.}
%
%
\begin{thebibliography}{1}
%
\bibitem{ST} Singer, I. M., and Thorpe, J. A.{\em Lecture Notes on 
  Elementary Topology and Geometry}, Scott, Foresman and Company. 
\bibitem{SPV} Spivak, M. {\em Calculus on Manifolds}, W. A. Benjamin, 
  Inc., New York, 1965. 

\bibitem{wms:aci}
J.~Sch\"u, W.M. Seiler, and J.~Calmet.
\newblock Algorithmic methods for {L}ie pseudogroups.
\newblock In N.~Ibragimov, M.~Torrisi, and A.~Valenti, editors, {\em Proc.
  Modern Group Analysis: Advanced Analytical and Computational Methods in
  Mathematical Physics}, pages 337--344, Acireale (Italy), 1992. Kluwer,
  Dordrecht~1993.

\bibitem{wms:axiom}
W.M. Seiler.
\newblock Applying {AXIOM} to partial differential equations.
\newblock Internal Report 95-17, Universit\"at Karlsruhe, Fakult\"at f\"ur
  Informatik, 1995. Available from:
\newblock {\tt http://citeseerx.ist.psu.edu}

\bibitem{wms:diss}
W.M. Seiler.
\newblock {\em Analysis and Application of the Formal Theory of Partial
  Differential Equations}.
\newblock PhD thesis, School of Physics and Materials, Lancaster University,
  1994.

\bibitem{wms:mcm}
W.M. Seiler.
\newblock Involution and symmetry reductions.
\newblock Preprint~1995.

\bibitem{wms:con1}
W.M. Seiler and R.W. Tucker.
\newblock Involution and constrained dynamics~{I}: The {D}irac approach.
\newblock {\em J.\ Phys.~A}, to appear, 1995.

\bibitem{wms:jet}
W.M. Seiler and  Jacques Calmet
\newblock {\em JET - An AXIOM Environment for Geometric Computations 
     with Differential Equations}
Electronic Proc. IMACS Conference on Applications of Computer Algebra, 
Albuquerque 1995, M. Wester, S. Steinberg, M. Jahn (eds.)
\end{thebibliography}
%
\end{document}
% -----------------------------------------------------------------------------
% END DOCUMENTATION
% -----------------------------------------------------------------------------