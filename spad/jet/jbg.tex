%==============================================================================
% Documentation (pdflatex this_file.spad
% Note: ensure that the code above is between \iffalse ... )if false\fi.
%==============================================================================
\documentclass[12pt,a4paper]{article}
\usepackage[utf8]{inputenc}
\usepackage[english]{babel}
\usepackage{amsmath}
\usepackage{amsfonts}
\usepackage{amssymb}
\usepackage{makeidx}
\usepackage{graphicx}
\usepackage{listings}
\usepackage{color}
\usepackage{epstopdf}
\usepackage{pgfplots}
%
\pgfplotsset{width=10cm,compat=1.9}
%
\definecolor{dkgreen}{rgb}{0,0.6,0}
\definecolor{gray}{rgb}{0.5,0.5,0.5}
\definecolor{mauve}{rgb}{0.58,0,0.82}
\definecolor{orange}{rgb}{1.0,0.44,0}
%
\lstdefinelanguage{SPAD}
{keywords={if,then,else,for,in,repeat},
keywords=[2]{Fraction, Integer, Polynomial, Expression, Float, 
             DoubleFloat, SurfaceComplex, CellMap, 
             List},
keywords=[3]{cellMap, bdry, tangentSpace, normalSpace, normalField,
             metricTensor, jacobianMatrix, faces, coords, coordSymbols},
keywordstyle=[2]\color{red},
keywordstyle=[3]\color{orange},
sensitive=true,%
alsoletter={\$},%
comment=[l]{--},%
string=[b]",%
string=[b]'%
}

\lstset{frame=tb,
  language=SPAD,
  aboveskip=3mm,
  belowskip=3mm,
  showstringspaces=false,
  columns=flexible,
  basicstyle={\small\ttfamily},
  numbers=none,
  numberstyle=\tiny\color{gray},
  keywordstyle=\color{blue},
  commentstyle=\color{dkgreen},
  stringstyle=\color{mauve},
  breaklines=true,
  breakatwhitespace=true,
  tabsize=3
}
\author{Kurt Pagani \\ {\tt nilqed@gmail.com}}
\date{December 20, 2016}
\title{JetBundleGeometry \\ {\small\tt Domain: JBG}}
%
\newcommand{\CAD}{{\tt CAD}}
\newcommand{\QE}{{\tt QE}}
\newcommand{\RR}[1]{\mathbb{R}^{#1}}
\newcommand{\QQ}[1]{\mathbb{Q}^{#1}}
\newcommand{\ZZ}[1]{\mathbb{Z}^{#1}}
\newcommand{\KK}[1]{\mathbb{K}^{#1}}
\newcommand{\spadfun}[1]{\textcolor{magenta}{\tt #1}}
\newcommand{\spadbold}[1]{{\tt\bf #1}}
\newcommand{\type}[1]{\textcolor{blue}{\tt\tiny #1}}
\newtheorem{definition}{Definition}
%
\begin{document}
\maketitle
%
\begin{abstract}
This manual decribes the FriCAS domain {\bf JetBundleGeometry}.
This domain extends the JET series and provides methods to compute
{\bf pull backs}, {\bf integrals} and other quantities of 
differential forms living on a Jet bundle. The domain relies on
{\tt JET} (\cite{wms:jet}) and some other domains and packages. 
\end{abstract}
%
\tableofcontents
\section{Introduction}
Recall that a bundle is a triple $(M,X,\pi)$, where $M,X$ are manifolds
and $\pi:M\rightarrow X$ is the projection, i.e. a continuous surjective
mapping from the total space ($M$) to the base space ($X$). A bundle is 
called trivial if $M$ is homeomorphic to $X\times U$, where
$U$ denotes another manifold, called the fibre. A locally trivial bundle
is called a fibre bundle, which is what we here are concerned with.
We will use the following nomenclature: 
\begin{displaymath}
   x=(x_1,\ldots,x_p), \ \ u=(u^1,\ldots,u^n)
\end{displaymath}
where $x$ are the (local) coordinates of the base manifold $X$ and $u$
the the ones on the fibre $U$. Locally the projection takes the form
\begin{displaymath}
 \pi:\begin{cases} 
      & X\times U \\
      & (x,u) \mapsto x.
   \end{cases}
\end{displaymath}
A section of the fibre bundle then has the form
\begin{displaymath}
 \Phi_f:\begin{cases} 
      & X\times U \\
      & x \mapsto (x,f(x)).
\end{cases}
\end{displaymath}
such that $\pi\circ \Phi_f = \mathcal{I}d$.
A jet bundle now may be considered as an iteration of that construction,
that is we will speak of a $n$-th order jet bundle $M^{(n)}$ if
\begin{displaymath}
 M^{(n)} = X \times \left( U\times U_1\times\ldots\times U_n  \right)
\end{displaymath}  
where the $U_j$ are the jet (Euclidean) spaces. We will just write $M$
for $M^{(0)}$. 

\section{Appendix}
The following is a verbatim inclusion of the extended abstract of
W.M.\,Seiler's paper.

\begin{center}
{\bf JET --- An AXIOM Environment for Geometric Computations with
Differential Equations} \\
{ W.M.~Seiler and J.~Calmet\\
        Institut f\"ur Algorithmen und Kognitive Systeme\\
        Universit\"at Karlsruhe\\
        76128 Karlsruhe, Germany\\
        Email: seilerw@ira.uka.de} \\
\bf Extended Abstract
\end{center}

Geometric methods play an important role in the analysis of nonlinear
differential equations. For example, symmetry methods provide the more or less
only systematic approach to the construction of solutions. However, most
geometric computations tend to be very tedious. Thus the use of computer algebra
systems considerably helps in the application of these methods.

{\small JET} is an environment within the computer algebra system {\small AXIOM}
to perform such computations. The current implementation emphasizes the two key
concepts involution and symmetry. It provides some packages for the completion
of a given system of differential equations to an equivalent involutive one
based on the Cartan-Kuranishi theorem and for setting up the determining
equations for classical and non-classical point symmetries.

We stress that {\small JET} is an {\em environment\/} for such computations and
not simply a collection of some special purpose algorithms. Thus it contains
general data structures for the jet bundle formalism which can also be used for
other tasks than the two above mentioned. Using the generic programming
facilities of {\small AXIOM} it is possible to provide several representations
for jet bundles and for different classes of differential equations. The main
application packages are independent of such details.

Involution has important applications in symmetry theory. One should e.g.\
mention that involutive systems are locally solvable and only for such systems
the two widely used definitions of a symmetry coincide. The in calculations
applied definition as a transformation leaving the differential equation
invariant yields for not locally solvable systems usually less symmetries then
the definition as a transformation mapping solutions into solutions.  This can
easily be seen with the system $u_{z}+yu_{x}=u_{y}=0$. Obviously it is not
invariant under $y$-translations, but its solution space $u=const$ has this
symmetry. This effect has especially implications on the nonclassical method of
Bluman and Cole.

Other applications include computing the size of the symmetry group of a
differential equation without solving the determining equations. Furthermore it
is possible to ``correct'' the result by subtracting some unwanted effects like
e.g.\ the trivial superposition symmetry of linear equations. In the case of
gauge theory the concept of involution leads to an new intrinsic definition for
the number of degrees of freedom based on a similar formal correction.

A brief description of an earlier version of {\small JET} can be found in
Ref.~\cite{wms:aci}. The current version is described in much detail in
Ref.~\cite{wms:axiom}. For more information about the underlying mathematical
theory we refer to Ref.~\cite{wms:diss}. Applications of the concept of
involution in symmetry theory are discussed in Ref.~\cite{wms:mcm}. Finally
we mention Ref.~\cite{wms:con1} for some applications in physics. All these
publications can be obtained via {\small WWW} at the {\small URL}:\\
{\tt
http://www.mathematik.uni-kassel.de/\textasciitilde  seiler/}. 
\footnote{link updated.}
%
%
\begin{thebibliography}{1}
%
\bibitem{ST} Singer, I. M., and Thorpe, J. A.{\em Lecture Notes on 
  Elementary Topology and Geometry}, Scott, Foresman and Company. 
\bibitem{SPV} Spivak, M. {\em Calculus on Manifolds}, W. A. Benjamin, 
  Inc., New York, 1965. 

\bibitem{wms:aci}
J.~Sch\"u, W.M. Seiler, and J.~Calmet.
\newblock Algorithmic methods for {L}ie pseudogroups.
\newblock In N.~Ibragimov, M.~Torrisi, and A.~Valenti, editors, {\em Proc.
  Modern Group Analysis: Advanced Analytical and Computational Methods in
  Mathematical Physics}, pages 337--344, Acireale (Italy), 1992. Kluwer,
  Dordrecht~1993.

\bibitem{wms:axiom}
W.M. Seiler.
\newblock Applying {AXIOM} to partial differential equations.
\newblock Internal Report 95-17, Universit\"at Karlsruhe, Fakult\"at f\"ur
  Informatik, 1995.

\bibitem{wms:diss}
W.M. Seiler.
\newblock {\em Analysis and Application of the Formal Theory of Partial
  Differential Equations}.
\newblock PhD thesis, School of Physics and Materials, Lancaster University,
  1994.

\bibitem{wms:mcm}
W.M. Seiler.
\newblock Involution and symmetry reductions.
\newblock Preprint~1995.

\bibitem{wms:con1}
W.M. Seiler and R.W. Tucker.
\newblock Involution and constrained dynamics~{I}: The {D}irac approach.
\newblock {\em J.\ Phys.~A}, to appear, 1995.

\bibitem{wms:jet}
W.M. Seiler and  Jacques Calmet
\newblock {\em JET - An AXIOM Environment for Geometric Computations 
     with Differential Equations}
Electronic Proc. IMACS Conference on Applications of Computer Algebra, 
Albuquerque 1995, M. Wester, S. Steinberg, M. Jahn (eds.)
\end{thebibliography}
%
\end{document}
% -----------------------------------------------------------------------------
% END DOCUMENTATION
% -----------------------------------------------------------------------------