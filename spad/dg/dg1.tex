%==============================================================================
% Documentation (pdflatex this_file.spad
% Note: ensure that the code above is between \iffalse ... )if false\fi.
%==============================================================================
\documentclass[12pt,a4paper]{article}
\usepackage[utf8]{inputenc}
\usepackage[english]{babel}
\usepackage{amsmath}
\usepackage{amsfonts}
\usepackage{amssymb}
\usepackage{makeidx}
\usepackage{graphicx}
\usepackage{listings}
\usepackage{color}
\usepackage{epstopdf}
\usepackage{pgfplots}
%
\pgfplotsset{width=10cm,compat=1.9}
%
\definecolor{dkgreen}{rgb}{0,0.6,0}
\definecolor{gray}{rgb}{0.5,0.5,0.5}
\definecolor{mauve}{rgb}{0.58,0,0.82}
\definecolor{orange}{rgb}{1.0,0.44,0}
%
\lstdefinelanguage{SPAD}
{keywords={if,then,else,for,in,repeat},
keywords=[2]{Fraction, Integer, Polynomial, Expression, Float, 
             DoubleFloat, SurfaceComplex, CellMap, 
             List},
keywords=[3]{cellMap, bdry, tangentSpace, normalSpace, normalField,
             metricTensor, jacobianMatrix, faces, coords, coordSymbols,
             pullBack, integrate, multipleIntegral,DG1},
keywordstyle=[2]\color{red},
keywordstyle=[3]\color{orange},
sensitive=true,%
alsoletter={\$},%
comment=[l]{--},%
string=[b]",%
string=[b]'%
}

\lstset{frame=tb,
  language=SPAD,
  aboveskip=3mm,
  belowskip=3mm,
  showstringspaces=false,
  columns=flexible,
  basicstyle={\small\ttfamily},
  numbers=none,
  numberstyle=\tiny\color{gray},
  keywordstyle=\color{blue},
  commentstyle=\color{dkgreen},
  stringstyle=\color{mauve},
  breaklines=true,
  breakatwhitespace=true,
  tabsize=3
}
\author{Kurt Pagani \\ {\tt nilqed@gmail.com}}
\date{December 20, 2016}
\title{DifferentialGeometry1 \\ {\small\tt Domain: DG1}}
%
\newcommand{\CAD}{{\tt CAD}}
\newcommand{\QE}{{\tt QE}}
\newcommand{\RR}[1]{\mathbb{R}^{#1}}
\newcommand{\QQ}[1]{\mathbb{Q}^{#1}}
\newcommand{\ZZ}[1]{\mathbb{Z}^{#1}}
\newcommand{\KK}[1]{\mathbb{K}^{#1}}
\newcommand{\spadfun}[1]{\textcolor{magenta}{\tt #1}}
\newcommand{\spadbold}[1]{{\tt\bf #1}}
\newcommand{\type}[1]{\textcolor{blue}{\tt\tiny #1}}
\newtheorem{definition}{Definition}
%
\begin{document}
\maketitle
%
\begin{abstract}
This manual describes the FriCAS package {\bf DifferentialGeometry1}.
This package combines differential forms and cell mappings to provide 
methods which compute {\bf pull backs}, {\bf integrals} as well as
some other quantities of differential forms living on a surface 
complex.
\end{abstract}
%
\tableofcontents
\section{Introduction}
Initially, it was planned to build this package by reusing
{\tt JET} (included in FriCAS), a sophisticated series of domains and 
packages written by Joachim Schue and Werner M. Seiler (\cite{wms:aci},
\cite{wms:axiom},\cite{wms:diss},\cite{wms:jet}). Actually, most of
the functionality of this package has already been implemented in
a new package {\tt JetBundleGeometry} when we recognized that jet
bundles in FriCAS are much more powerful than just serving as a 
helper package for local coordinate patches. Continuing the process
would have us led to unfortunate dependencies on cell mappings and
other domains, so to make it difficult to use higher order jet bundles
in connection with differential forms. Consequently we extracted 
everything useful into this package. Because parts of the manual
also had been written the reader will certainly recognize this
here and there, although we tried to remove most unrelated topics.
The package {\tt JetBundleGeometry} will be based on a new domain,
{\tt JetDifferentialForm}\footnote{work in progress} and will only 
depend on {\tt JET}. The rest of this section may be skipped, however, 
it is assumed the reader is acquainted with the manuals of 
{\tt DifferentialForm} and {\tt SurfaceComplex}. 

Recall that a bundle is a triple $(M,X,\pi)$, where $M,X$ are manifolds
and $\pi:M\rightarrow X$ is the projection, i.e. a continuous surjective
mapping from the total space ($M$) to the base space ($X$). A bundle is 
called trivial if $M$ is homeomorphic to $X\times U$, where
$U$ denotes another manifold, called the fibre. A locally trivial bundle
is called a fibre bundle, which is what we here are concerned with.
We will use the following nomenclature: 
\begin{displaymath}
   x=(x_1,\ldots,x_p), \ \ u=(u^1,\ldots,u^n)
\end{displaymath}
where $x$ are the (local) coordinates of the base manifold $X$ and $u$
the the ones on the fibre $U$. Locally the projection takes the form
\begin{displaymath}
 \pi:\begin{cases} 
      & X\times U \\
      & (x,u) \mapsto x.
   \end{cases}
\end{displaymath}
A section of the fibre bundle then has the form
\begin{displaymath}
 \Phi_f:\begin{cases} 
      & X\times U \\
      & x \mapsto (x,f(x)).
\end{cases}
\end{displaymath}
such that $\pi\circ \Phi_f = \mathcal{I}d$.
A jet bundle now may be considered as an iteration of that construction,
that is we will speak of a $n$-th order jet bundle $M^{(n)}$ if
\begin{displaymath}
 M^{(n)} = X \times \left( U\times U_1\times\ldots\times U_n  \right)
\end{displaymath}  
where the $U_j$ are the jet (Euclidean) spaces. We will just write $M$
for $M^{(0)}$.
%
\subsection{Motivation}
Let us recall the following domains/packages and their functionality:
\begin{itemize}
\item[1] \spadfun{DeRhamComplex} provides differential forms as a
graded ring in a given set of variables:
\begin{displaymath}
  \mathtt{DeRhamComplex(R,[x_1,\ldots,x_n])} =
      \bigoplus_{p=0}^n \Lambda^p([x_1,\ldots,x_n])
\end{displaymath}
where $x_j\in \mathtt{Expression(R)}$, $R$ a ring, meaning that the
coefficients are from {\tt Expression R}.
%
\item[2] \spadfun{DifferentialForms} is a package extending 
{\tt DeRhamComplex} by {\em Riemannian metrics} $g$ and
the connected standard operations like {\em Hodge star} $\star_g$,
$\delta_g$, $\Delta_g$, $i_X$, $\mathcal{L}_X$ and 
$\langle\cdot,\cdot\rangle_g$.
%
\item[3] \spadfun{CellMap} and \spadfun{SurfaceComplex} are domains
which loosely spoken will provide parametric surface patches and
their formal sums (free group with integer coefficients):
\begin{displaymath}
  \mathtt{SurfaceComplex(R,n)} =
      \mathbb{Z}\cdot\bigoplus_{p=0}^n \Sigma^p([\$_1,\ldots,\$_n])  
\end{displaymath}
where $\Sigma_p$ is the collection of $p-$cells, that is mappings
from a $p-$cell into $n-$space.

Consequently we can define the boundary operator $\partial$ such 
that $T(d\varphi)=\partial T(\varphi)$ holds, where $T$ is an
element of SurfaceComplex and $\varphi$ one of DeRhamComplex.
\end{itemize}
But note that the code of these domains is unrelated, that is
both can be used independently. So we are in need of a package
or domain which brings them together in order to utilize their 
mutual duality.

A second point is the need of computing {\em pull backs} of
differential forms and - closely related - of integration of
forms over {\em affine chains}. To achieve this we have
to incorporate two (usually) different spaces of differential
forms.

Eventually, the following quotations from \cite{wms:axiom} together
with the remarks above induced the usage of {\tt JetBundle}
instead of creating a new domain:
%
\begin{verse}
\ldots
The implementation of both is somewhat rudimentary, as we hope that some
day {\tt AXIOM} will contain a reasonable environment for differential 
geometric calculations and then it should be used instead of some 
special domains like the two mentioned. 
\ldots
The implementation of the differential geometric domains  {\tt VectorField} 
(abbreviated by {\tt VF}) and {\tt Differential (DIFF)} are fairly primitive. 
They should be considered as a temporary hack, until AXIOM contains 
a better environment for differential geometric calculations.
\ldots
\end{verse}  
%
We may interpret {\tt JetBundle} in the first place as a local
coordinate patch (chart) with independent variables $x$ and
dependent ones $u$. Then we will be able to pull forms in $u-$
space back to $x$-space. In other words we will only use a 
zero order jet bundle for the moment. In a second step the
interplay might be extended to higher order bundles such that
the deficiencies mentioned in the above quotes may be mitigated.
%
\section{Theory}
For the general theory about pulling back - and integration of
differential forms we refer to one of the excellent books, e.g.
(\cite{FLAN},\cite{SPV},\cite{MTAA},\cite{CART}), cited in the 
bibliography.
We recall two important facts about pulling back differential
forms $\alpha,\beta$ by smooth maps $f$:
\begin {eqnarray*}
   f^\star (\alpha\wedge\beta) &=& f^\star\alpha\wedge f^\star\beta,\\
   d\, f^\star\alpha &=& f^\star (d\,\alpha). 
\end {eqnarray*} 
The implementation of \spadfun{pullBack} is based on this.
Furthermore, we have for $f:Q \rightarrow M$ the identity
\begin{displaymath}
   \int_{M} \omega = \int_Q f^\star \omega,
\end{displaymath}
on which \spadfun{integrate} is based on. Usually $Q$ is a 
$p-$cell and $M$ a $p-$surface (or surface complex). In other
words, $f$ is {\tt S.map} and $Q$ is {\tt S.dom} when $S$ is
of type {\tt CellMap}.
Since we have the restriction {\tt \#(S.dom)}$\leq n$, where $n$
is the dimension of the range of {\tt S.map}, the mapping $f$
may also be described by a set of equations (see further below).
This essentially is all we need to compute most of the usual
differential quantities that occur in connection with local and
global surface theories. As remarked in the introduction, this
package will only serve as a simple link between the domain of
affine chains {\tt SurfaceComplex} and the domain of co-chains
{\tt DeRhamComplex}. More sophisticated methods will be 
implemented in the forthcoming package {\tt JetBundleGeometry}. 
%
\subsection{Local Coordinates}
To set up the package we have to provide two lists of symbols
which will denote the dependent- and the independent coordinates, for
example
\begin{lstlisting}
M ==> DG1([x,y],[u,v,w])
\end{lstlisting}
means that $u,v,w$ are {\em dependent} coordinates, that is functions
of the {\em independent} coordinates $x,y$. Note that $M$ is just
a package selector without any information how this dependencies
are defined. However, $M$ will provide two spaces of differential
forms, namely
\begin{lstlisting}
DeRhamComplex(Integer,[x,y])    -- e.g. a(x,y) dx dy
DeRhamComplex(Integer,[u,v,w])  -- e.g. b(u,v,w) du dv.
\end{lstlisting}
When we - for instance - think of a parametric surface 
\begin{displaymath}
   (x,y) \mapsto (u(x,y),v(x,y),w(x,y))
\end{displaymath} 
we immediately see how the pull back of the $2-$form $b(u,v,w)\,du\wedge dv$
to the parameter domain looks like:
\begin{displaymath}
  b(u(x,y),v(x,y),w(x,y)) du(x,y)\wedge dv(x,y),
\end{displaymath}  
where $du(x,y)\wedge dv(x,y)$ expands to
\begin{displaymath}
 (u_x dx + u_y dy)\wedge (v_x dx + v_y dy) = (u_x v_y - v_x u_y) dx\wedge dy
\end{displaymath}
of course. 
%
\subsection{$p-$Surfaces}
Recall that a $p-$surface is a mapping from a compact cell $Q$ of
dimension $p$ into $n-$space, where we always assume $p\leq n$.
Staying with the parametric surface picture above, we have $p=2$,
and $n=3$, and we may speak of a $2-$surface. Since $p=n-1$ it 
actually is a hyper-surface but this is in no way a must. Usually
$n$ is chosen as big as necessary in order to obtain an 
{\em embedding} of the (generalized) surface. Remembering {\em Nash's}
embedding theorem, we can (almost) always isometrically embed a Riemannian 
manifold into Euclidean space if $n$ is sufficiently big.

Back to the example, if we want to integrate our form over the $2-$surface,
we will specify a cell mapping of type $\tt CellMap(Integer,3)$ as
follows:
\begin{lstlisting}
CM3 ==>  CellMap(Integer,3)
Q:=[a..b,c..d]  -- 2-cell
S:=cellMap(Q,q+->[F(q.1,q.2),G(q.1,q.2),H(q.1,q.2)])$CM3
\end{lstlisting}
We see that $S$ is formulated independent of the coordinates in $M$,
only if we apply $S$ to $(x,y)$, we will get the connection between
the variables $u,v,w$ and $x,y$. Now we have 
\begin{displaymath}
   \int_{S(Q)} b(u,v,w)\,du\wedge dv = \int_{Q} S^{\star}(b(u,v,w)\,du\wedge dv)
\end{displaymath}
or, using the results above,
\begin{displaymath}
   = \int_{Q} b(u(x,y),v(x,y),w(x,y))  (u_x v_y - v_x u_y) dx\wedge dy
\end{displaymath}
what reduces to an ordinary double integral over the cell $Q$:
\begin{displaymath}
\int_c^d \int_a^b b(u(x,y),v(x,y),w(x,y))  (u_x v_y - v_x u_y) dx dy.
\end{displaymath}
This is just how these integrals over cell maps are computed:
\begin{lstlisting}
  integrate(b(u,v,w)*du*dv,S)$M
\end{lstlisting}


\section{Appendix}
%
\begin{thebibliography}{1}
%
\bibitem{ST} Singer, I. M., and Thorpe, J. A.{\em Lecture Notes on 
  Elementary Topology and Geometry}, Scott, Foresman and Company. 

\bibitem{SPV} Spivak, M. {\em Calculus on Manifolds}, W. A. Benjamin, 
  Inc., New York, 1965. 

\bibitem{MTAA} Ralph Abraham, Jerrold E.Marsden and Tudor Ratiu.Manifolds, 
       {\em Tensor Analysis, and Applications}. Springer,
       Auflage: 2nd Corrected ed. 1988. Corr. 2nd printing 1993 edition.
%       
\bibitem{CART} Henri Cartan. {\em Differential Forms}. Dover Pubn Inc.
%
\bibitem{GMT} Herbert Federer. {\em Geometric Measure Theory}. Springer, 
       Reprint of the 1st ed. Berlin, Heidelberg, New York 1969 edition.
%
\bibitem{FLAN} Harley Flanders, {\em Differential Forms with Applications to 
       the Physical Sciences}. Dover Pubn Inc, Revised. edition.
%
\bibitem{QYBEQ} L. A. Lambe and D. E. Radford. {\em Introduction to the 
       Quantum Yang-Baxter Equation and Quantum Groups:An Algebraic Approach}.
       Springer, 1997 edition.
%
\bibitem{PMA} Walter Rudin and RudinWalter.{\em Principles of Mathematical
       Analysis}. McGraw Hill Book Co, Revised. edition.
%
\bibitem{GIT} Hassler Whitney. {\em Geometric Integration Theory}, 
       Princeton Mathematical Series, No. 21. Literary Licensing, LLC.


\bibitem{wms:aci}
J.~Sch\"u, W.M. Seiler, and J.~Calmet.
\newblock Algorithmic methods for {L}ie pseudogroups.
\newblock In N.~Ibragimov, M.~Torrisi, and A.~Valenti, editors, {\em Proc.
  Modern Group Analysis: Advanced Analytical and Computational Methods in
  Mathematical Physics}, pages 337--344, Acireale (Italy), 1992. Kluwer,
  Dordrecht~1993.

\bibitem{wms:axiom}
W.M. Seiler.
\newblock Applying {AXIOM} to partial differential equations.
\newblock Internal Report 95-17, Universit\"at Karlsruhe, Fakult\"at f\"ur
  Informatik, 1995. Available from:
\newblock {\tt http://citeseerx.ist.psu.edu}

\bibitem{wms:diss}
W.M. Seiler.
\newblock {\em Analysis and Application of the Formal Theory of Partial
  Differential Equations}.
\newblock PhD thesis, School of Physics and Materials, Lancaster University,
  1994.

\bibitem{wms:mcm}
W.M. Seiler.
\newblock Involution and symmetry reductions.
\newblock Preprint~1995.

\bibitem{wms:con1}
W.M. Seiler and R.W. Tucker.
\newblock Involution and constrained dynamics~{I}: The {D}irac approach.
\newblock {\em J.\ Phys.~A}, to appear, 1995.

\bibitem{wms:jet}
W.M. Seiler and  Jacques Calmet
\newblock {\em JET - An AXIOM Environment for Geometric Computations 
     with Differential Equations}
Electronic Proc. IMACS Conference on Applications of Computer Algebra, 
Albuquerque 1995, M. Wester, S. Steinberg, M. Jahn (eds.)
\end{thebibliography}
%
\end{document}
% -----------------------------------------------------------------------------
% END DOCUMENTATION
% -----------------------------------------------------------------------------