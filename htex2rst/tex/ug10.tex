% !! DO NOT MODIFY THIS FILE BY HAND !! Created by spool2tex.awk.

% Copyright (c) 1991-2002, The Numerical ALgorithms Group Ltd.
% All rights reserved.
%
% Redistribution and use in source and binary forms, with or without
% modification, are permitted provided that the following conditions are
% met:
%
%     - Redistributions of source code must retain the above copyright
%       notice, this list of conditions and the following disclaimer.
%
%     - Redistributions in binary form must reproduce the above copyright
%       notice, this list of conditions and the following disclaimer in
%       the documentation and/or other materials provided with the
%       distribution.
%
%     - Neither the name of The Numerical ALgorithms Group Ltd. nor the
%       names of its contributors may be used to endorse or promote products
%       derived from this software without specific prior written permission.
%
% THIS SOFTWARE IS PROVIDED BY THE COPYRIGHT HOLDERS AND CONTRIBUTORS "AS
% IS" AND ANY EXPRESS OR IMPLIED WARRANTIES, INCLUDING, BUT NOT LIMITED
% TO, THE IMPLIED WARRANTIES OF MERCHANTABILITY AND FITNESS FOR A
% PARTICULAR PURPOSE ARE DISCLAIMED. IN NO EVENT SHALL THE COPYRIGHT OWNER
% OR CONTRIBUTORS BE LIABLE FOR ANY DIRECT, INDIRECT, INCIDENTAL, SPECIAL,
% EXEMPLARY, OR CONSEQUENTIAL DAMAGES (INCLUDING, BUT NOT LIMITED TO,
% PROCUREMENT OF SUBSTITUTE GOODS OR SERVICES-- LOSS OF USE, DATA, OR
% PROFITS-- OR BUSINESS INTERRUPTION) HOWEVER CAUSED AND ON ANY THEORY OF
% LIABILITY, WHETHER IN CONTRACT, STRICT LIABILITY, OR TORT (INCLUDING
% NEGLIGENCE OR OTHERWISE) ARISING IN ANY WAY OUT OF THE USE OF THIS
% SOFTWARE, EVEN IF ADVISED OF THE POSSIBILITY OF SUCH DAMAGE.

% *********************************************************************
\head{chapter}{Interactive Programming}{ugIntProg}
% *********************************************************************

Programming in the interpreter is easy.
So is the use of \Language{}'s graphics facility.
Both are rather flexible and allow you to use them for many
interesting applications.
However, both require learning some basic ideas and skills.

All graphics examples in the \Gallery{} section are either
produced directly by interactive commands or by interpreter
programs.
Four of these programs are introduced here.
By the end of this chapter you will know enough about graphics and
programming in the interpreter to not only understand all these
examples, but to tackle interesting and difficult problems on your
own.
\appxref{ugAppGraphics} lists all the remaining commands and
programs used to create these images.

% *********************************************************************
\head{section}{Drawing Ribbons Interactively}{ugIntProgDrawing}
% *********************************************************************
%

We begin our discussion of interactive graphics with the creation
of a useful facility: plotting ribbons of two-graphs in
three-space.
Suppose you want to draw the \twodim{} graphs of \smath{n}
functions
$f_i(x), 1 \leq i \leq n,$
all over some fixed range of \smath{x}.
One approach is to create a \twodim{} graph for each one, then
superpose one on top of the other.
What you will more than likely get is a jumbled mess.
Even if you make each function a different color, the result is
likely to be confusing.

A better approach is to display each of the \smath{f_i(x)} in three
\index{ribbon}
dimensions as a ``ribbon'' of some appropriate width along the
\smath{y}-direction, laying down each  ribbon next to the
previous one.
A ribbon is simply a function of \smath{x} and \smath{y} depending
only on \smath{x.}

We illustrate this for \smath{f_i(x)} defined as simple powers of
\smath{x} for \smath{x} ranging between \smath{-1} and \smath{1}.

\begin{psXtc}
\begin{xtccomment}
Draw the ribbon for \mathOrSpad{z=x ^ 2}.
\end{xtccomment}
\begin{spadsrc}
draw(x^2,x=-1..1,y=0..1)
\end{spadsrc}
\epsffile[0 0 295 295]{ribbon1.ps}
\end{psXtc}

Now that was easy!
What you get is a ``wire-mesh'' rendition of the ribbon.
That's fine for now.
Notice that the mesh-size is small in both the \smath{x} and the
\smath{y} directions.
\Language{} normally computes points in both these directions.
This is unnecessary.
One step is all we need in the \smath{y}-direction.
To have \Language{} economize on \spad{y}-points, we re-draw the
ribbon with option \spad{var2Steps == 1}.

\begin{psXtc}
\begin{xtccomment}
Re-draw the ribbon, but with option \spad{var2Steps == 1}
so that only \spad{1} step is computed in the
\smath{y} direction.
\end{xtccomment}
\begin{spadsrc}
vp := draw(x^2,x=-1..1,y=0..1,var2Steps==1) 
\end{spadsrc}
\epsffile[0 0 295 295]{ribbon2.ps}
\end{psXtc}

The operation has created a viewport, that is, a graphics window
on your screen.
We assigned the viewport to \spad{vp} and now we manipulate
its contents.


Graphs are objects, like numbers and algebraic expressions.
You may want to do some experimenting with graphs.
For example, say
\begin{verbatim}
showRegion(vp, "on")
\end{verbatim}
to put a bounding box around the ribbon.
Try it!
Issue \spad{rotate(vp, -45, 90)} to rotate the
figure \smath{-45} longitudinal degrees and \smath{90} latitudinal
degrees.

\begin{psXtc}
\begin{xtccomment}
Here is a different rotation.
This turns the graph so you can view it along the \smath{y}-axis.
\end{xtccomment}
\begin{spadsrc}
rotate(vp, 0, -90)
\end{spadsrc}
\epsffile[0 0 295 295]{ribbon2r.ps}
\end{psXtc}

There are many other things you can do.
In fact, most everything you can do interactively using the
\threedim{} control panel (such as translating, zooming, resizing,
coloring, perspective and lighting selections) can also be done
directly by operations (see \chapref{ugGraph} for more details).

When you are done experimenting, say \spad{reset(vp)} to restore the
picture to its original position and settings.


Let's add another ribbon to our picture---one
for \mathOrSpad{x^3}.
Since \smath{y} ranges from \smath{0} to \smath{1} for the
first ribbon, now let \smath{y} range from \smath{1} to
\smath{2.}
This puts the second ribbon next to the first one.

How do you add a second ribbon to the viewport?
One method is
to extract the ``space'' component from the
viewport using the operation
\spadfunFrom{subspace}{ThreeDimensionalViewport}.
You can think of the space component as the object inside the
window (here, the ribbon).
Let's call it \spad{sp}.
To add the second ribbon, you draw the second ribbon using the
option \spad{space == sp}.

\begin{xtc}
\begin{xtccomment}
Extract the space component of \spad{vp}.
\end{xtccomment}
\begin{spadsrc}
sp := subspace(vp)
\end{spadsrc}
\begin{MessageOutput}
   There are 2 exposed and 0 unexposed library operations named 
      subspace having 1 argument(s) but none was determined to be 
      applicable. Use HyperDoc Browse, or issue
                            )display op subspace
      to learn more about the available operations. Perhaps 
      package-calling the operation or using coercions on the arguments
      will allow you to apply the operation.
\end{MessageOutput}
\begin{MessageOutput}
   Cannot find a definition or applicable library operation named 
      subspace with argument type(s) 
                                Variable(vp)
      
      Perhaps you should use "@" to indicate the required return type, 
      or "$" to specify which version of the function you need.
\end{MessageOutput}
\end{xtc}

\begin{psXtc}
\begin{xtccomment}
Add the ribbon for
\mathOrSpad{x^3} alongside that for
\mathOrSpad{x^2}.
\end{xtccomment}
\begin{spadsrc}
vp := draw(x^3,x=-1..1,y=1..2,var2Steps==1, space==sp)
\end{spadsrc}
\epsffile[0 0 295 295]{ribbons.ps}
\end{psXtc}

Unless you moved the original viewport, the new viewport covers
the old one.
You might want to check that the old object is still there by
moving the top window.

Let's show quadrilateral polygon outlines on the ribbons and then
enclose the ribbons in a box.

\begin{psXtc}
\begin{xtccomment}
Show quadrilateral polygon outlines.
\end{xtccomment}
\begin{spadsrc}
drawStyle(vp,"shade");outlineRender(vp,"on")
\end{spadsrc}
\epsffile[0 0 295 295]{ribbons2.ps}
\end{psXtc}
\begin{psXtc}
\begin{xtccomment}
Enclose the ribbons in a box.
\end{xtccomment}
\begin{spadsrc}
rotate(vp,20,-60); showRegion(vp,"on")
\end{spadsrc}
\epsffile[0 0 295 295]{ribbons2b.ps}
\end{psXtc}

This process has become tedious!
If we had to add two or three more ribbons, we would have to
repeat the above steps several more times.
It is time to write an interpreter program to help us take care of
the details.

% *********************************************************************
\head{section}{A Ribbon Program}{ugIntProgRibbon}
% *********************************************************************
%

The above approach creates a new viewport for each additional
ribbon.
A better approach is to build one object composed of all ribbons
before creating a viewport.
To do this, use \spadfun{makeObject} rather than \spadfun{draw}.
The operations have similar formats, but
\spadfun{draw} returns a viewport and
\spadfun{makeObject} returns a space object.

We now create a function \userfun{drawRibbons} of two arguments:
\spad{flist}, a list of formulas for the ribbons you want to draw,
and \spad{xrange}, the range over which you want them drawn.
Using this function, you can just say
\begin{verbatim}
drawRibbons([x^2, x^3], x=-1..1)
\end{verbatim}
to do all of the work required in the last section.
Here is the \userfun{drawRibbons} program.
Invoke your favorite editor and create a file called {\bf ribbon.input}
containing the following program.

\begin{figXmpLines}[caption={The first \protect\pspadfun{drawRibbons} function.}]
drawRibbons(flist, xrange) ==
  sp := createThreeSpace()               -- Create empty space \spad{sp}.
  y0 := 0                                -- The initial ribbon position.
  for f in flist repeat                  -- For each function \spad{f},
    makeObject(f, xrange, y=y0..y0+1,    -- \quad{}create and add a ribbon
       space==sp, var2Steps == 1)        -- \quad{}for \spad{f} to the space \spad{sp}.
    y0 := y0 + 1                         -- The next ribbon position.
  vp := makeViewport3D(sp, "Ribbons")    -- Create viewport.
  drawStyle(vp, "shade")                 -- Select shading style.
  outlineRender(vp, "on")                -- Show polygon outlines.
  showRegion(vp,"on")                    -- Enclose in a box.
  n := # flist                           -- The number of ribbons
  zoom(vp,n,1,n)                         -- Zoom in x- and z-directions.
  rotate(vp,0,75)                        -- Change the angle of view.
  vp                                     -- Return the viewport.
\end{figXmpLines}

Here are some remarks on the syntax used in the \pspadfun{drawRibbons} function
(consult \chapref{ugUser} for more details).
Unlike most other programming languages which use semicolons,
parentheses, or {\it begin}--{\it end} brackets to delineate the
structure of programs, the structure of an \Language{} program is
determined by indentation.
The first line of the function definition always begins in column 1.
All other lines of the function are indented with respect to the first
line and form a \spadgloss{pile} (see \spadref{ugLangBlocks}).

The definition of \userfun{drawRibbons}
consists of a pile of expressions to be executed one after
another.
Each expression of the pile is indented at the same level.
Lines 4-7 designate one single expression:
since lines 5-7 are indented with respect to the others, these
lines are treated as a continuation of line 4.
Also since lines 5 and 7 have the same indentation level, these
lines designate a pile within the outer pile.

The last line of a pile usually gives the value returned by the
pile.
Here it is also the value returned by the function.
\Language{} knows this is the last line of the function because it
is the last line of the file.
In other cases, a new expression beginning in column one signals
the end of a function.

The line \spad{drawStyle(vp,"shade")} is given after the viewport
has been created to select the draw style.
We have also used the \spadfunFrom{zoom}{ThreeDimensionalViewport}
option.
Without the zoom, the viewport region would be scaled equally in
all three coordinate directions.

Let's try the function \userfun{drawRibbons}.
First you must read the file to give \Language{} the function definition.

\begin{xtc}
\begin{xtccomment}
Read the input file.
\end{xtccomment}
\begin{spadsrc}
)read ribbon 
\end{spadsrc}
\end{xtc}
\begin{psXtc}
\begin{xtccomment}
Draw ribbons for $x, x^2,\dots, x^5$
for $-1 \leq x \leq 1$
\end{xtccomment}
\begin{spadsrc}
drawRibbons([x^i for i in 1..5],x=-1..1) 
\end{spadsrc}
\epsffile[0 0 295 295]{ribbons5.ps}
\end{psXtc}


% *********************************************************************
\head{section}{Coloring and Positioning Ribbons}{ugIntProgColor}
% *********************************************************************
%

Before leaving the ribbon example, we  make two improvements.
Normally, the color given to each point in the space is a
function of its height within a bounding box.
The points at the bottom of the
box are red, those at the top are purple.

To change the normal coloring, you can give
an option \textspadexpr{colorFunction == {\it function}}.
When \Language{} goes about displaying the data, it
determines the range of colors used for all points within the box.
\Language{} then distributes these numbers uniformly over the number of hues.
Here we use the simple color function
$(x,y) \mapsto i$ for the
\eth{i} ribbon.

Also, we add an argument \spad{yrange} so you can give the range of
\spad{y} occupied by the ribbons.
For example, if the \spad{yrange} is given as
\spad{y=0..1} and there are \smath{5} ribbons to be displayed, each
ribbon would have width \smath{0.2} and would appear in the
range $0 \leq y \leq 1$.

Refer to lines 4-9.
Line 4 assigns to \spad{yVar} the variable part of the
\spad{yrange} (after all, it need not be \spad{y}).
Suppose that \spad{yrange} is given as \spad{t = a..b} where \spad{a} and
\spad{b} have numerical values.
Then line 5 assigns the value of \spad{a} to the variable \spad{y0}.
Line 6 computes the width of the ribbon by dividing the difference of
\spad{a} and \spad{b} by the number, \spad{num}, of ribbons.
The result is assigned to the variable \spad{width}.
Note that in the for-loop in line 7, we are iterating in parallel; it is
not a nested loop.

\begin{figXmpLines}[caption={The final \protect\pspadfun{drawRibbons} function.}]
drawRibbons(flist, xrange, yrange) ==
  sp := createThreeSpace()                     -- Create empty space \spad{sp}.
  num := # flist                               -- The number of ribbons.
  yVar := variable yrange                      -- The ribbon variable.
  y0:Float    := low segment yrange             - The first ribbon coordinate.
  width:Float := (high segment yrange - y0)/num   The width of a ribbon.
  for f in flist for color in 1..num repeat    -- For each function \spad{f},
    makeObject(f, xrange, yVar = y0..y0+width, -- \quad{}create and add ribbon to
      var2Steps == 1, _                        -- \quad{}\spad{sp} of a different color.
      colorFunction == (x,y) +-> color, _
      space == sp)
    y0 := y0 + width                           -- The next ribbon coordinate.
  vp := makeViewport3D(sp, "Ribbons")          -- Create viewport.
  drawStyle(vp, "shade")                       -- Select shading style.
  outlineRender(vp, "on")                      -- Show polygon outlines.
  showRegion(vp, "on")                         -- Enclose -- in a box.
  vp                                           -- Return the viewport.
\end{figXmpLines}


% *********************************************************************
\head{section}{Points, Lines, and Curves}{ugIntProgPLC}
% *********************************************************************
%
What you have seen so far is a high-level program using the
graphics facility.
We now turn to the more basic notions of points, lines, and curves
in \threedim{} graphs.
These facilities use small floats (objects
of type \spadtype{DoubleFloat}) for data.
Let us first give names to the small float values \smath{0} and
\smath{1}.
\begin{xtc}
\begin{xtccomment}
The small float 0.
\end{xtccomment}
\begin{spadsrc}
zero := 0.0@DFLOAT 
\end{spadsrc}
\begin{TeXOutput}
\begin{fricasmath}{1}
\STRING{0.0}%
\end{fricasmath}
\end{TeXOutput}
\formatResultType{DoubleFloat}
\end{xtc}
\begin{xtc}
\begin{xtccomment}
The small float 1.
\end{xtccomment}
\begin{spadsrc}
one  := 1.0@DFLOAT 
\end{spadsrc}
\begin{TeXOutput}
\begin{fricasmath}{2}
\STRING{1.0}%
\end{fricasmath}
\end{TeXOutput}
\formatResultType{DoubleFloat}
\end{xtc}
The \spadSyntax{@} sign means ``of the type.'' Thus \spad{zero} is
\smath{0.0} of the type \spadtype{DoubleFloat}.
You can also say \spad{0.0::DFLOAT}.

Points can have four small float components: \smath{x, y, z} coordinates and an
optional color.
A ``curve'' is simply a list of points connected by straight line
segments.
\begin{xtc}
\begin{xtccomment}
Create the point \spad{origin} with color zero, that is, the lowest color
on the color map.
\end{xtccomment}
\begin{spadsrc}
origin := point [zero,zero,zero,zero] 
\end{spadsrc}
\begin{TeXOutput}
\begin{fricasmath}{3}
\BRACKET{\STRING{0.0}\COMMA \STRING{0.0}\COMMA \STRING{0.0}\COMMA \STRING{0.0%
}}%
\end{fricasmath}
\end{TeXOutput}
\formatResultType{Point(DoubleFloat)}
\end{xtc}
\begin{xtc}
\begin{xtccomment}
Create the point \spad{unit} with color zero.
\end{xtccomment}
\begin{spadsrc}
unit := point [one,one,one,zero] 
\end{spadsrc}
\begin{TeXOutput}
\begin{fricasmath}{4}
\BRACKET{\STRING{1.0}\COMMA \STRING{1.0}\COMMA \STRING{1.0}\COMMA \STRING{0.0%
}}%
\end{fricasmath}
\end{TeXOutput}
\formatResultType{Point(DoubleFloat)}
\end{xtc}
\begin{xtc}
\begin{xtccomment}
Create the curve (well, here, a line) from
\spad{origin} to \spad{unit}.
\end{xtccomment}
\begin{spadsrc}
line := [origin, unit] 
\end{spadsrc}
\begin{TeXOutput}
\begin{fricasmath}{5}
\BRACKET{\BRACKET{\STRING{0.0}\COMMA \STRING{0.0}\COMMA \STRING{0.0}\COMMA %
\STRING{0.0}}\COMMA \BRACKET{\STRING{1.0}\COMMA \STRING{1.0}\COMMA \STRING{%
1.0}\COMMA \STRING{0.0}}}%
\end{fricasmath}
\end{TeXOutput}
\formatResultType{List(Point(DoubleFloat))}
\end{xtc}

We make this line segment into an arrow by adding an arrowhead.
The arrowhead extends to,
say, \spad{p3} on the left, and to, say, \spad{p4} on the right.
To describe an arrow, you tell \Language{} to draw the two curves
\spad{[p1, p2, p3]} and \spad{[p2, p4].}
We also decide through experimentation on
values for \spad{arrowScale}, the ratio of the size of
the arrowhead to the stem of the arrow, and \spad{arrowAngle},
the angle between the arrowhead and the arrow.

Invoke your favorite editor and create
an input file called {\bf arrows.input}.
This input file first defines the values of
%\spad{origin},\spad{unit},
\spad{arrowAngle} and \spad{arrowScale}, then
defines the function \userfun{makeArrow}$(p_1, p_2)$ to
draw an arrow from point $p_1$ to $p_2$.

%origin := point [0.0@DFLOAT,0.0@DFLOAT,0.0@DFLOAT,0.0@DFLOAT]       -- The point 0 with color 0.
%unit := point [1.0@DFLOAT,1.0@DFLOAT,1.0@DFLOAT,0.0@DFLOAT]         -- A second point with color 0.
%\xmpLine{}{}
\begin{xmpLines}
arrowAngle := %pi-%pi/10.0@DFLOAT              -- The angle of the arrowhead.
arrowScale := 0.2@DFLOAT                       -- The size of the arrowhead
                                               -- \quad{}relative to the stem.
makeArrow(p1, p2) ==
  delta := p2 - p1                             -- The arrow.
  len := arrowScale * length delta             -- The length of the arrowhead.
  theta := atan(delta.1, delta.2)              -- The angle from the x-axis
  c1 := len*cos(theta + arrowAngle)            -- The x-coord of left endpoint.
  s1 := len*sin(theta + arrowAngle)            -- The y-coord of left endpoint.
  c2 := len*cos(theta - arrowAngle)            -- The x-coord of right endpoint.
  s2 := len*sin(theta - arrowAngle)            -- The y-coord of right endpoint.
  z  := p2.3*(1 - arrowScale)                  -- The z-coord of both endpoints.
  p3 := point [p2.1 + c1, p2.2 + s1, z, p2.4]  -- The left endpoint of head.
  p4 := point [p2.1 + c2, p2.2 + s2, z, p2.4]  -- The right endpoint of head.
  [[p1, p2, p3], [p2, p4]]                     -- The arrow as a list of curves.
\end{xmpLines}

Read the file and then create
an arrow from the point \spad{origin} to the point \spad{unit}.
\begin{xtc}
\begin{xtccomment}
Read the input file defining \userfun{makeArrow}.
\end{xtccomment}
\begin{spadsrc}
)read arrows
\end{spadsrc}
\begin{TeXOutput}
\begin{fricasmath}{6}
\STRING{2.827433388230814}%
\end{fricasmath}
\end{TeXOutput}
\formatResultType{DoubleFloat}
\begin{TeXOutput}
\begin{fricasmath}{7}
\STRING{0.2}%
\end{fricasmath}
\end{TeXOutput}
\formatResultType{DoubleFloat}
\end{xtc}
\begin{xtc}
\begin{xtccomment}
Construct the arrow (a list of two curves).
\end{xtccomment}
\begin{spadsrc}
arrow := makeArrow(origin,unit)
\end{spadsrc}
\begin{MessageOutput}
   Compiling function makeArrow with type (Point(DoubleFloat),Point(
      DoubleFloat)) -> List(List(Point(DoubleFloat))) 
\end{MessageOutput}
\begin{TeXOutput}
\begin{fricasmath}{9}
\BRACKET{\BRACKET{\BRACKET{\STRING{0.0}\COMMA \STRING{0.0}\COMMA \STRING{0.0}%
\COMMA \STRING{0.0}}\COMMA \BRACKET{\STRING{1.0}\COMMA \STRING{1.0}\COMMA %
\STRING{1.0}\COMMA \STRING{0.0}}\COMMA \BRACKET{\STRING{0.6913462860460797}%
\COMMA \STRING{0.842733077659504}\COMMA \STRING{0.8}\COMMA \STRING{0.0}}}%
\COMMA \BRACKET{\BRACKET{\STRING{1.0}\COMMA \STRING{1.0}\COMMA \STRING{1.0}%
\COMMA \STRING{0.0}}\COMMA \BRACKET{\STRING{0.842733077659504}\COMMA \STRING{%
0.6913462860460797}\COMMA \STRING{0.8}\COMMA \STRING{0.0}}}}%
\end{fricasmath}
\end{TeXOutput}
\formatResultType{List(List(Point(DoubleFloat)))}
\end{xtc}
\begin{xtc}
\begin{xtccomment}
Create an empty object \spad{sp} of type \spad{ThreeSpace}.
\end{xtccomment}
\begin{spadsrc}
sp := createThreeSpace()
\end{spadsrc}
\begin{TeXOutput}
\begin{fricasmath}{10}
\STRING{3-Space\ with\ }0\STRING{\ components}%
\end{fricasmath}
\end{TeXOutput}
\formatResultType{ThreeSpace(DoubleFloat)}
\end{xtc}
\begin{xtc}
\begin{xtccomment}
Add each curve of the arrow to the space \spad{sp}.
\end{xtccomment}
\begin{spadsrc}
for a in arrow repeat sp := curve(sp,a)
\end{spadsrc}
\end{xtc}
\begin{psXtc}
\begin{xtccomment}
Create a \threedim{} viewport containing that space.
\end{xtccomment}
\begin{spadsrc}
vp := makeViewport3D(sp,"Arrow")
\end{spadsrc}
\epsffile[0 0 295 295]{arrow.ps}
\end{psXtc}
\begin{psXtc}
\begin{xtccomment}
Here is a better viewing angle.
\end{xtccomment}
\begin{spadsrc}
rotate(vp,200,-60)
\end{spadsrc}
\epsffile[0 0 295 295]{arrowr.ps}
\end{psXtc}


% *********************************************************************
\head{section}{A Bouquet of Arrows}{ugIntProgColorArr}
% *********************************************************************

%\Language{} gathers up all the points of a graph and looks at the range
%of color values given as integers.
%If theses color values range from a minimum value of \spad{a} to a maximum
%value of \spad{b}, then the \spad{a} values are colored red (the
%lowest color in our spectrum), and \spad{b} values are colored
%purple (the highest color), and those in the middle are colored
%green.
%When all the points are the same color as above, \Language{}
%chooses green.

Let's draw a ``bouquet'' of arrows.
Each arrow is identical. The arrowheads are
uniformly placed on a circle parallel to the \smath{xy}-plane.
Thus the position of each arrow differs only
by the angle $\theta$,
$0 \leq \theta < 2\pi$,
between the arrow and
the \smath{x}-axis on the \smath{xy}-plane.

Our bouquet is rather special: each arrow has a different
color (which won't be evident here, unfortunately).
This is arranged by letting the color of each successive arrow be
denoted by $\theta$.
In this way, the color of arrows ranges from red to green to violet.
Here is a program to draw a bouquet of \smath{n} arrows.

\begin{xmpLines}
drawBouquet(n,title) ==
  z := 0.0@DFLOAT
  e := 1.0@DFLOAT
  angle := z                                        -- The initial angle.
  sp := createThreeSpace()                          -- Create empty space \spad{sp}.
  for i in 0..n-1 repeat                            -- For each index i, create:
    start := point [z,z,z,angle]                    -- point at base of arrow;
    end   := point [cos angle, sin angle, e, angle] -- point at tip of arrow;
    arrow := makeArrow(start,end)                   -- \spad{i}th arrow.
    for a in makeArrow(start,end) repeat            -- For each arrow component,
      curve(sp,a)                                   -- \quad{}add the component to \spad{sp}.
    angle := angle + 2*%pi/n                        -- The next angle.
  makeViewport3D(sp,title)                          -- Create the viewport from \spad{sp}.
\end{xmpLines}

\begin{xtc}
\begin{xtccomment}
Read the input file.
\end{xtccomment}
\begin{spadsrc}
)read bouquet
\end{spadsrc}
relative size of the arrow head compared to the length of the arrow
\begin{TeXOutput}
\begin{fricasmath}{1}
\STRING{0.2}%
\end{fricasmath}
\end{TeXOutput}
\formatResultType{DoubleFloat}
angle of the arrow head
\begin{TeXOutput}
\begin{fricasmath}{2}
\STRING{2.827433388230814}%
\end{fricasmath}
\end{TeXOutput}
\formatResultType{DoubleFloat}
Add an arrow head to a line segment, which starts at 'p1', ends at 'p2',
has length 'len', and and angle 'arg'.  We pass 'len' and 'arg' as
arguments since they were already computed by the calling program
\end{xtc}
\begin{psXtc}
\begin{xtccomment}
A bouquet of a dozen arrows.
\end{xtccomment}
\begin{spadsrc}
drawBouquet(12,"A Dozen Arrows")
\end{spadsrc}
\epsffile[0 0 295 295]{bouquet.ps}
\end{psXtc}
\

% *********************************************************************
%\head{section}{Diversion: When Things Go Wrong}{ugIntProgDivTwo}
% *********************************************************************
%
%Up to now, if you have typed in all the programs exactly as they are in
%the book, you have encountered no errors.
%In practice, however, it is easy to make mistakes.
%Computers are unforgiving: your program must be letter-for-letter correct
%or you will encounter some error.
%
%One thing that can go wrong is that you can create a syntactically
%incorrect program.
%As pointed out in Diversion 1, the meaning of \Language{} programs is
%affected by indentation.
%
%The \Language{} parser will ensure that all parentheses, brackets, and
%braces balance, and that commas and operators appear in the correct
%context.
%For example, change line ??
%to ??
%and run.
%
%A common mistake is to misspell an identifier or operation name.
%These are generally easy to spot since the interpreter will tell you the
%name of the operation together with the type and number of arguments which
%it is trying to find.
%
%Another mistake is to either to omit an argument or to give too many.
%Again \Language{} will notify you of the offending operation.
%
%Indentation makes your programs more readable.
%However there are several ways to create a syntactically valid program.
%A most common problem occurs when a line is either indented improperly.
%% either or what?
%If this is a first line of a pile, then all the other lines will act as an
%inner pile to the first line.
%If it is a line of the pile other than the first line, \Language{} then
%thinks that this line is a continuation of the previous line.
%More frequently than not, a syntactically correct expression is created.
%Almost never however will this be a semantically correct.
%Only when the program is run will an error be discovered.
%For example, change line ??
%to ??
%and run.

% *********************************************************************
\head{section}{Drawing Complex Vector Fields}{ugIntProgVecFields}
% *********************************************************************

We now put our arrows to good use drawing complex vector fields.
These vector fields give a representation of complex-valued
functions of complex variables.
Consider a Cartesian coordinate grid of points \smath{(x, y)} in
the plane, and some complex-valued function \smath{f} defined on
this grid.
At every point on this grid, compute the value of
$f(x + iy)$ and call it \smath{z}.
Since \smath{z} has both a real and imaginary value for a given
\smath{(x,y)} grid point, there are four dimensions to plot.
What do we do?
We represent the values of \smath{z} by arrows planted at each
grid point.
Each arrow represents the value of \smath{z} in polar coordinates
$(r,\theta)$.
The length of the arrow is proportional to \smath{r}.
Its direction is given by $\theta$.

The code for drawing vector fields is in the file {\bf vectors.input}.
We discuss its contents from top to bottom.

Before showing you the code, we have two small
matters to take care of.
First, what if the function has large spikes, say, ones that go off
to infinity?
We define a variable \spad{clipValue} for this purpose. When
\spad{r} exceeds the value of \spad{clipValue}, then the value of
\spad{clipValue} is used instead of that for \spad{r}.
For convenience, we define a function \spad{clipFun(x)} which uses
\spad{clipValue} to ``clip'' the value of \spad{x}.

%
\begin{xmpLines}
clipValue : DFLOAT := 6                              -- Maximum value allowed.
clipFun(x) == min(max(x,-clipValue),clipValue)
\end{xmpLines}

Notice that we identify \spad{clipValue} as a small float but do
not declare the type of the function \userfun{clipFun}.
As it turns out, \userfun{clipFun} is called with a
small float value.
This declaration ensures that \userfun{clipFun} never does a
conversion when it is called.

The second matter concerns the possible ``poles'' of a
function, the actual points where the spikes have infinite
values.
\Language{} uses normal \spadtype{DoubleFloat} arithmetic  which
does not directly handle infinite values.
If your function has poles, you must adjust your step size to
avoid landing directly on them (\Language{} calls \spadfun{error}
when asked to divide a value by \spad{0}, for example).

We set the variables \spad{realSteps} and \spad{imagSteps} to
hold the number of steps taken in the real and imaginary
directions, respectively.
Most examples will have ranges centered around the origin.
To avoid a pole at the origin, the number of points is taken
to be odd.

\begin{xmpLinesNoReset}
realSteps: INT := 25                                 -- Number of real steps.
imagSteps: INT := 25                                 -- Number of imaginary steps.
)read arrows
\end{xmpLinesNoReset}

Now define the function \userfun{drawComplexVectorField} to draw the arrows.
It is good practice to declare the type of the main function in
the file.
This one declaration is usually sufficient to ensure that other
lower-level functions are compiled with the correct types.

\begin{xmpLinesNoReset}
C := Complex DoubleFloat
S := Segment DoubleFloat
drawComplexVectorField: (C -> C, S, S) -> VIEW3D
\end{xmpLinesNoReset}

The first argument is a function mapping complex small floats into
complex small floats.
The second and third arguments give the range of real and
imaginary values as segments like \spad{a..b}.
The result is a \threedim{} viewport.
Here is the full function definition:

\begin{xmpLinesNoReset}
drawComplexVectorField(f, realRange,imagRange) ==
  delReal := (high(realRange)-low(realRange))/realSteps The real step size.
  delImag := (high(imagRange)-low(imagRange))/imagSteps The imaginary step size.
  sp := createThreeSpace()                           -- Create empty space \spad{sp}.
  real := low(realRange)                             -- The initial real value.
  for i in 1..realSteps+1 repeat                     -- Begin real iteration.
    imag := low(imagRange)                           -- The initial imaginary value.
    for j in 1..imagSteps+1 repeat                   -- Begin imaginary iteration.
      z := f complex(real,imag)                      -- The value of \spad{f} at the point.
      arg := argument z                              -- The direction of the arrow.
      len := clipFun sqrt norm z                     -- The length of the arrow.
      p1 :=  point [real, imag, 0.0@DFLOAT, arg]     -- The base point of the arrow.
      scaleLen := delReal * len                      -- The scaled length of the arrow.
      p2 := point [p1.1 + scaleLen*cos(arg),         -- The tip point of the arrow.
                   p1.2 + scaleLen*sin(arg),0.0@DFLOAT, arg]
      arrow := makeArrow(p1, p2)                     -- Create the arrow.
      for a in arrow repeat curve(sp, a)             -- Add arrow to the space \spad{sp}.
      imag := imag + delImag                         -- The next imaginary value.
    real := real + delReal                           -- The next real value.
  makeViewport3D(sp, "Complex Vector Field")         -- Draw it!
\end{xmpLinesNoReset}

As a first example, let us draw \spad{f(z) == sin(z)}.
There is no need to create a user function: just pass the
\spadfunFrom{sin}{Complex DoubleFloat} from \spadtype{Complex DoubleFloat}.
\begin{xtc}
\begin{xtccomment}
Read the file.
\end{xtccomment}
\begin{spadsrc}
)read vectors 
\end{spadsrc}
\begin{TeXOutput}
\begin{fricasmath}{1}
\STRING{2.827433388230814}%
\end{fricasmath}
\end{TeXOutput}
\formatResultType{DoubleFloat}
\begin{TeXOutput}
\begin{fricasmath}{2}
\STRING{0.2}%
\end{fricasmath}
\end{TeXOutput}
\formatResultType{DoubleFloat}
\begin{TeXOutput}
\begin{fricasmath}{4}
\STRING{6.0}%
\end{fricasmath}
\end{TeXOutput}
\formatResultType{DoubleFloat}
\begin{TeXOutput}
\begin{fricasmath}{6}
25%
\end{fricasmath}
\end{TeXOutput}
\formatResultType{Integer}
\begin{TeXOutput}
\begin{fricasmath}{7}
25%
\end{fricasmath}
\end{TeXOutput}
\formatResultType{Integer}
\begin{TeXOutput}
\begin{fricasmath}{8}
\STRING{Complex(DoubleFloat)}%
\end{fricasmath}
\end{TeXOutput}
\formatResultType{Type}
\begin{TeXOutput}
\begin{fricasmath}{9}
\STRING{Segment(DoubleFloat)}%
\end{fricasmath}
\end{TeXOutput}
\formatResultType{Type}
\end{xtc}
\begin{psXtc}
\begin{xtccomment}
Draw the complex vector field of \spad{sin(x)}.
\end{xtccomment}
\begin{spadsrc}
drawComplexVectorField(sin,-2..2,-2..2) 
\end{spadsrc}
\epsffile[0 0 295 295]{vectorSin.ps}
\end{psXtc}
\

% *********************************************************************
\head{section}{Drawing Complex Functions}{ugIntProgCompFuns}
% *********************************************************************

Here is another way to graph a complex function of complex
arguments.
For each complex value \smath{z}, compute \smath{f(z)}, again
expressing the value in polar coordinates \smath{(r,\theta{})}.
We draw the complex valued function, again considering the
\smath{(x,y)}-plane as the complex plane, using \smath{r} as the
height (or \smath{z}-coordinate) and \smath{\theta} as the color.
This is a standard plot---we learned how to do this in
\chapref{ugGraph}---but here we write a new program to illustrate
the creation of polygon meshes, or grids.

Call this function \userfun{drawComplex}.
It displays the points using the ``mesh'' of points.
The function definition is in three parts.

\begin{xmpLines}
drawComplex: (C -> C, S, S) -> VIEW3D
drawComplex(f, realRange, imagRange) ==              -- The first part.
  delReal := (high(realRange)-low(realRange))/realSteps The real step size.
  delImag := (high(imagRange)-low(imagRange))/imagSteps The imaginary step size.
  llp:List List Point DFLOAT := []                   -- Initial list of list of points \spad{llp}.
\end{xmpLines}

Variables \spad{delReal} and \spad{delImag} give the step
sizes along the real and imaginary directions as computed by the values
of the global variables \spad{realSteps} and \spad{imagSteps}.
The mesh is represented by a list of lists of points \spad{llp},
initially empty.
Now \spad{[ ]} alone is ambiguous, so
to set this initial value
you have to tell \Language{} what type of empty list it is.
Next comes the loop which builds \spad{llp}.

\begin{xmpLinesNoReset}
  real := low(realRange)                              - The initial real value.
  for i in 1..realSteps+1 repeat                     -- Begin real iteration.
    imag := low(imagRange)                           -- The initial imaginary value.
    lp := []$(List Point DFLOAT)                     -- The initial list of points \spad{lp}.
    for j in 1..imagSteps+1 repeat                   -- Begin imaginary iteration.
      z := f complex(real,imag)                      -- The value of \spad{f} at the point.
      pt := point [real,imag, clipFun sqrt norm z,   -- Create a point.
                   argument z]
      lp := cons(pt,lp)                              -- Add the point to \spad{lp}.
      imag := imag + delImag                         -- The next imaginary value.
    real := real + delReal                           -- The next real value.
    llp := cons(lp, llp)                             -- Add \spad{lp} to \spad{llp}.
\end{xmpLinesNoReset}

The code consists of both an inner and outer loop.
Each pass through the inner loop adds one list \spad{lp} of points
to the list of lists of points \spad{llp}.
The elements of \spad{lp} are collected in reverse order.

\begin{xmpLinesNoReset}
  makeViewport3D(mesh(llp), "Complex Function")      -- Create a mesh and display.
\end{xmpLinesNoReset}

The operation \spadfun{mesh} then creates an object of type
\spadtype{ThreeSpace(DoubleFloat)} from the list of lists of points.
This is then passed to \spadfun{makeViewport3D} to display the
image.

Now add this function directly to your {\bf vectors.input}
file and re-read the file using \spad{)read vectors}.
We try \userfun{drawComplex} using
a user-defined function \spad{f}.

\begin{xtc}
\begin{xtccomment}
Read the file.
\end{xtccomment}
\begin{spadsrc}
)read vectors 
\end{spadsrc}
\begin{TeXOutput}
\begin{fricasmath}{1}
\STRING{2.827433388230814}%
\end{fricasmath}
\end{TeXOutput}
\formatResultType{DoubleFloat}
\begin{TeXOutput}
\begin{fricasmath}{2}
\STRING{0.2}%
\end{fricasmath}
\end{TeXOutput}
\formatResultType{DoubleFloat}
\begin{TeXOutput}
\begin{fricasmath}{4}
\STRING{6.0}%
\end{fricasmath}
\end{TeXOutput}
\formatResultType{DoubleFloat}
\begin{TeXOutput}
\begin{fricasmath}{6}
25%
\end{fricasmath}
\end{TeXOutput}
\formatResultType{Integer}
\begin{TeXOutput}
\begin{fricasmath}{7}
25%
\end{fricasmath}
\end{TeXOutput}
\formatResultType{Integer}
\begin{TeXOutput}
\begin{fricasmath}{8}
\STRING{Complex(DoubleFloat)}%
\end{fricasmath}
\end{TeXOutput}
\formatResultType{Type}
\begin{TeXOutput}
\begin{fricasmath}{9}
\STRING{Segment(DoubleFloat)}%
\end{fricasmath}
\end{TeXOutput}
\formatResultType{Type}
\end{xtc}
\begin{xtc}
\begin{xtccomment}
This one has a pole at \smath{z=0}.
\end{xtccomment}
\begin{spadsrc}
f(z) == exp(1/z)
\end{spadsrc}
\end{xtc}
\begin{psXtc}
\begin{xtccomment}
Draw it with an odd number of steps to avoid the pole.
\end{xtccomment}
\begin{spadsrc}
drawComplex(f,-2..2,-2..2)
\end{spadsrc}
\epsffile[0 0 295 295]{complexExp.ps}
\end{psXtc}
\

% *********************************************************************
\head{section}{Functions Producing Functions}{ugIntProgFunctions}
% *********************************************************************

In \spadref{ugUserMake}, you learned how to use the operation
\spadfun{function} to create a function from symbolic formulas.
Here we introduce a similar operation which not only
creates functions, but functions from functions.

The facility we need is provided by the package
\spadtype{MakeUnaryCompiledFunction(E,S,T)}.
\exptypeindex{MakeUnaryCompiledFunction}
This package produces a unary (one-argument) compiled
function from some symbolic data
generated by a previous computation.\footnote{%
\spadtype{MakeBinaryCompiledFunction} is available for binary
functions.}
\exptypeindex{MakeBinaryCompiledFunction}
The \spad{E} tells where the symbolic data comes from;
the \spad{S} and \spad{T} give \Language{} the
source and target type of the function, respectively.
The compiled function produced  has type
\spad{S -> T}.
To produce a compiled function with definition \spad{p(x) == expr}, call
\spad{compiledFunction(expr, x)} from this package.
The function you get has no name.
You must to assign the function to the variable \spad{p} to give it that name.
%
\begin{xtc}
\begin{xtccomment}
Do some computation.
\end{xtccomment}
\begin{spadsrc}
(x+1/3)^5
\end{spadsrc}
\begin{TeXOutput}
\begin{fricasmath}{1}
\SUPER{\SYMBOL{x}}{5}+\frac{5}{3}\TIMES \SUPER{\SYMBOL{x}}{4}+\frac{10}{9}%
\TIMES \SUPER{\SYMBOL{x}}{3}+\frac{10}{27}\TIMES \SUPER{\SYMBOL{x}}{2}+\frac{%
5}{81}\TIMES \SYMBOL{x}+\frac{1}{243}%
\end{fricasmath}
\end{TeXOutput}
\formatResultType{Polynomial(Fraction(Integer))}
\end{xtc}
\begin{xtc}
\begin{xtccomment}
Convert this to an anonymous function of \spad{x}.
Assign it to the variable \spad{p} to give the function a name.
\end{xtccomment}
\begin{spadsrc}
p := compiledFunction(%,x)$MakeUnaryCompiledFunction(POLY FRAC INT,DFLOAT,DFLOAT)
\end{spadsrc}
\begin{MessageOutput}
   Compiling function %A with type DoubleFloat -> DoubleFloat 
\end{MessageOutput}
\begin{TeXOutput}
\begin{fricasmath}{2}
\theMap{unaryFunction}%
\end{fricasmath}
\end{TeXOutput}
\formatResultType{(DoubleFloat -> DoubleFloat)}
\end{xtc}
\begin{xtc}
\begin{xtccomment}
Apply the function.
\end{xtccomment}
\begin{spadsrc}
p(sin(1.3))
\end{spadsrc}
\begin{TeXOutput}
\begin{fricasmath}{3}
\STRING{3.668751115057229}%
\end{fricasmath}
\end{TeXOutput}
\formatResultType{DoubleFloat}
\end{xtc}

For a more sophisticated application, read on.

% *********************************************************************
\head{section}{Automatic Newton Iteration Formulas}{ugIntProgNewton}
% *********************************************************************

%--rhx: TODO: Check this carefully.
% This setting is needed to get Newton's iterations to converge.
% \spadcommand{)set streams calculate 10}

We resume
our continuing saga of arrows and complex functions.
Suppose we want to investigate the behavior of Newton's iteration function
\index{Newton iteration}
in the complex plane.
Given a function \smath{f}, we want to find the complex values
\smath{z} such that \smath{f(z) = 0}.

The first step is to produce a Newton iteration formula for
a given \smath{f}:
$x_{n+1} = x_n - \frac{f(x_n)}{f'(x_n)}.$
We represent this formula by a function \smath{g}
that performs the computation on the right-hand side, that is,
$x_{n+1} = {g}(x_n)$.

The type \spadtype{Expression Integer}
(abbreviated \spadtype{EXPR INT})
is used to represent general symbolic expressions in
\Language{}.
\exptypeindex{Expression}
To make our facility as general as possible, we assume
\smath{f} has this type.
Given \smath{f}, we want
to produce a Newton iteration function \spad{g} which,
given a complex point $x_n$, delivers the next
Newton iteration point $x_{n+1}$.

This time we write an input file called {\bf newton.input}.
We need to import \spadtype{MakeUnaryCompiledFunction} (discussed
in the last section), call it with appropriate types, and then define
the function \spad{newtonStep} which references it.
Here is the function \spad{newtonStep}:

\begin{xmpLines}
C := Complex DoubleFloat                             -- The complex numbers.
E := Expression Integer                              -- The expression domain.
complexFunPack:=MakeUnaryCompiledFunction(E,C,C)     -- Package for making functions.

newtonStep(f) ==                                     -- Newton's iteration function.
  fun  := complexNumericFunction f                   -- Function for $f$.
  deriv := complexDerivativeFunction(f,1)            -- Function for $f'$.
  (x:C):C +->                                        -- Return the iterator function.
    x - fun(x)/deriv(x)

complexNumericFunction f ==                          -- Turn an expression \spad{f} into a
  v := theVariableIn f                               -- \quad{}function.
  compiledFunction(f, v)$complexFunPack

complexDerivativeFunction(f,n) ==                    -- Create an nth derivative
  v := theVariableIn f                               -- \quad{}function.
  df := D(f,v,n)
  compiledFunction(df, v)$complexFunPack

theVariableIn f ==                                   -- Returns the variable in $f$.
  vl := variables f                                  -- The list of variables.
  nv := # vl                                         -- The number of variables.
  nv > 1 => error "Expression is not univariate."
  nv = 0 => 'x                                       -- Return a dummy variable.
  first vl
\end{xmpLines}

Do you see what is going on here?
A formula \spad{f} is passed into the function \userfun{newtonStep}.
First, the function turns \spad{f} into a compiled program mapping
complex numbers into complex numbers.  Next, it does the same thing
for the derivative of \spad{f}.  Finally, it returns a function which
computes a single step of Newton's iteration.

The function \userfun{complexNumericFunction} extracts the variable
from the expression \spad{f} and then turns \spad{f} into a function
which maps complex numbers into complex numbers. The function
\userfun{complexDerivativeFunction} does the same thing for the
derivative of \spad{f}.  The function \userfun{theVariableIn}
extracts the variable from the expression \spad{f}, calling the function
\spadfun{error} if \spad{f} has more than one variable.
It returns the dummy variable \spad{x} if \spad{f} has no variables.

Let's now apply \userfun{newtonStep} to the formula for computing
cube roots of two.
%
\begin{noOutputXtc}
\begin{xtccomment}
Read the input file with the definitions.
\end{xtccomment}
\begin{spadsrc}
)read newton
\end{spadsrc}
Newton's Iteration function
newtonStep(f) returns a newton's iteration function for the
expression f.
create complex numeric functions from an expression
\begin{TeXOutput}
\begin{fricasmath}{2}
\STRING{%
MakeUnaryCompiledFunction(Expression(Integer),Complex(DoubleFloat),Complex(DoubleFloat))%
}%
\end{fricasmath}
\end{TeXOutput}
\formatResultType{Type}
create a complex numeric function from an expression
create a complex numeric derivatiave function from an expression
return the unique variable in x, or an error if it is multivariate
\end{noOutputXtc}
\begin{noOutputXtc}
\begin{xtccomment}
\end{xtccomment}
\begin{spadsrc}
)read vectors 
\end{spadsrc}
\begin{TeXOutput}
\begin{fricasmath}{6}
\STRING{2.827433388230814}%
\end{fricasmath}
\end{TeXOutput}
\formatResultType{DoubleFloat}
\begin{TeXOutput}
\begin{fricasmath}{7}
\STRING{0.2}%
\end{fricasmath}
\end{TeXOutput}
\formatResultType{DoubleFloat}
\begin{TeXOutput}
\begin{fricasmath}{9}
\STRING{6.0}%
\end{fricasmath}
\end{TeXOutput}
\formatResultType{DoubleFloat}
\begin{TeXOutput}
\begin{fricasmath}{11}
25%
\end{fricasmath}
\end{TeXOutput}
\formatResultType{Integer}
\begin{TeXOutput}
\begin{fricasmath}{12}
25%
\end{fricasmath}
\end{TeXOutput}
\formatResultType{Integer}
\begin{TeXOutput}
\begin{fricasmath}{13}
\STRING{Complex(DoubleFloat)}%
\end{fricasmath}
\end{TeXOutput}
\formatResultType{Type}
\begin{TeXOutput}
\begin{fricasmath}{14}
\STRING{Segment(DoubleFloat)}%
\end{fricasmath}
\end{TeXOutput}
\formatResultType{Type}
\end{noOutputXtc}

\begin{xtc}
\begin{xtccomment}
The cube root of two.
\end{xtccomment}
\begin{spadsrc}
f := x^3 - 2
\end{spadsrc}
\begin{TeXOutput}
\begin{fricasmath}{19}
\SUPER{\SYMBOL{x}}{3}-{2}%
\end{fricasmath}
\end{TeXOutput}
\formatResultType{Polynomial(Integer)}
\end{xtc}
\begin{xtc}
\begin{xtccomment}
Get Newton's iteration formula.
\end{xtccomment}
\begin{spadsrc}
g := newtonStep f
\end{spadsrc}
\begin{MessageOutput}
   Compiling function theVariable with type Polynomial(Integer) -> 
      Symbol 
\end{MessageOutput}
\begin{MessageOutput}
   Compiling function complexNumericFunction with type Polynomial(
      Integer) -> (Complex(DoubleFloat) -> Complex(DoubleFloat)) 
\end{MessageOutput}
\begin{MessageOutput}
   Compiling function complexDerivativeFunction with type (Polynomial(
      Integer),PositiveInteger) -> (Complex(DoubleFloat) -> Complex(
      DoubleFloat)) 
\end{MessageOutput}
\begin{MessageOutput}
   Compiling function newtonStep with type Polynomial(Integer) -> (
      Complex(DoubleFloat) -> Complex(DoubleFloat)) 
\end{MessageOutput}
\begin{MessageOutput}
   Compiling function %B with type Complex(DoubleFloat) -> Complex(
      DoubleFloat) 
\end{MessageOutput}
\begin{MessageOutput}
   Compiling function %C with type Complex(DoubleFloat) -> Complex(
      DoubleFloat) 
\end{MessageOutput}
\begin{TeXOutput}
\begin{fricasmath}{20}
\theMap{?}%
\end{fricasmath}
\end{TeXOutput}
\formatResultType{(Complex(DoubleFloat) -> Complex(DoubleFloat))}
\end{xtc}
\begin{xtc}
\begin{xtccomment}
Let \spad{a} denote the result of
applying Newton's iteration once to the complex number \spad{1 + %i}.
\end{xtccomment}
\begin{spadsrc}
a := g(1.0 + %i)
\end{spadsrc}
\begin{TeXOutput}
\begin{fricasmath}{21}
\STRING{0.6666666666666667}+\STRING{0.33333333333333337}\TIMES \ImaginaryI %
\end{fricasmath}
\end{TeXOutput}
\formatResultType{Complex(DoubleFloat)}
\end{xtc}
\begin{xtc}
\begin{xtccomment}
Now apply it repeatedly. How fast does it converge?
\end{xtccomment}
\begin{spadsrc}
[(a := g(a)) for i in 1..]
\end{spadsrc}
\begin{TeXOutput}
\begin{fricasmath}{22}
\BRACKET{\STRING{1.1644444444444444}-{\STRING{0.7377777777777778}\TIMES %
\ImaginaryI }\COMMA \STRING{0.9261400469716478}-{\STRING{0.17463006425584393}%
\TIMES \ImaginaryI }\COMMA \STRING{1.3164444838140228}+\STRING{%
0.15690694583015852}\TIMES \ImaginaryI \COMMA \STRING{1.2462991025761463}+%
\STRING{0.015454763610132094}\TIMES \ImaginaryI \COMMA \STRING{%
1.259872529653208}-{\STRING{3.382716205931127e-4}\TIMES \ImaginaryI }\COMMA %
\STRING{1.259920960928212}+\STRING{2.602353465342268e-8}\TIMES \ImaginaryI %
\COMMA \STRING{1.259921049894879}-{\STRING{3.6751942591616685e-15}\TIMES %
\ImaginaryI }\COMMA \STRING{...}}%
\end{fricasmath}
\end{TeXOutput}
\formatResultType{Stream(Complex(DoubleFloat))}
\end{xtc}
\begin{xtc}
\begin{xtccomment}
Check the accuracy of the last iterate.
\end{xtccomment}
\begin{spadsrc}
a^3
\end{spadsrc}
\begin{TeXOutput}
\begin{fricasmath}{23}
\STRING{2.0000000000000275}-{\STRING{1.7502021699542322e-14}\TIMES %
\ImaginaryI }%
\end{fricasmath}
\end{TeXOutput}
\formatResultType{Complex(DoubleFloat)}
\end{xtc}

In \xmpref{MappingPackage1}, we show how functions can be
manipulated as objects in \Language{}.
A useful operation to consider here is \spadop{*}, which means
composition.
For example \spad{g*g} causes the Newton iteration formula
to be applied twice.
Correspondingly, \spad{g^n} means to apply the iteration formula
\spad{n} times.

%
\begin{xtc}
\begin{xtccomment}
Apply \spad{g} twice to the point \spad{1 + %i}.
\end{xtccomment}
\begin{spadsrc}
(g*g) (1.0 + %i)
\end{spadsrc}
\begin{TeXOutput}
\begin{fricasmath}{24}
\STRING{1.1644444444444444}-{\STRING{0.7377777777777778}\TIMES \ImaginaryI }%
\end{fricasmath}
\end{TeXOutput}
\formatResultType{Complex(DoubleFloat)}
\end{xtc}
\begin{xtc}
\begin{xtccomment}
Apply \spad{g} 11 times.
\end{xtccomment}
\begin{spadsrc}
(g^11) (1.0 + %i)
\end{spadsrc}
\begin{TeXOutput}
\begin{fricasmath}{25}
\STRING{1.2599210498948732}%
\end{fricasmath}
\end{TeXOutput}
\formatResultType{Complex(DoubleFloat)}
\end{xtc}

Look now at the vector field and surface generated
after two steps of Newton's formula for the cube root of two.
The poles in these pictures represent bad starting values, and the
flat areas are the regions of convergence to the three roots.
%
\begin{psXtc}
\begin{xtccomment}
The vector field.
\end{xtccomment}
\begin{spadsrc}
drawComplexVectorField(g^3,-3..3,-3..3)
\end{spadsrc}
\epsffile[0 0 295 295]{vectorRoot.ps}
\end{psXtc}
\begin{psXtc}
\begin{xtccomment}
The surface.
\end{xtccomment}
\begin{spadsrc}
drawComplex(g^3,-3..3,-3..3)
\end{spadsrc}
\epsffile[0 0 295 295]{complexRoot.ps}
\end{psXtc}
