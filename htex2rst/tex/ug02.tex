% !! DO NOT MODIFY THIS FILE BY HAND !! Created by spool2tex.awk.

% Copyright (c) 1991-2002, The Numerical ALgorithms Group Ltd.
% All rights reserved.
%
% Redistribution and use in source and binary forms, with or without
% modification, are permitted provided that the following conditions are
% met:
%
%     - Redistributions of source code must retain the above copyright
%       notice, this list of conditions and the following disclaimer.
%
%     - Redistributions in binary form must reproduce the above copyright
%       notice, this list of conditions and the following disclaimer in
%       the documentation and/or other materials provided with the
%       distribution.
%
%     - Neither the name of The Numerical ALgorithms Group Ltd. nor the
%       names of its contributors may be used to endorse or promote products
%       derived from this software without specific prior written permission.
%
% THIS SOFTWARE IS PROVIDED BY THE COPYRIGHT HOLDERS AND CONTRIBUTORS "AS
% IS" AND ANY EXPRESS OR IMPLIED WARRANTIES, INCLUDING, BUT NOT LIMITED
% TO, THE IMPLIED WARRANTIES OF MERCHANTABILITY AND FITNESS FOR A
% PARTICULAR PURPOSE ARE DISCLAIMED. IN NO EVENT SHALL THE COPYRIGHT OWNER
% OR CONTRIBUTORS BE LIABLE FOR ANY DIRECT, INDIRECT, INCIDENTAL, SPECIAL,
% EXEMPLARY, OR CONSEQUENTIAL DAMAGES (INCLUDING, BUT NOT LIMITED TO,
% PROCUREMENT OF SUBSTITUTE GOODS OR SERVICES-- LOSS OF USE, DATA, OR
% PROFITS-- OR BUSINESS INTERRUPTION) HOWEVER CAUSED AND ON ANY THEORY OF
% LIABILITY, WHETHER IN CONTRACT, STRICT LIABILITY, OR TORT (INCLUDING
% NEGLIGENCE OR OTHERWISE) ARISING IN ANY WAY OUT OF THE USE OF THIS
% SOFTWARE, EVEN IF ADVISED OF THE POSSIBILITY OF SUCH DAMAGE.

% *********************************************************************
\head{chapter}{Using Types and Modes}{ugTypes}
% *********************************************************************

In this chapter we look at the key notion of \spadgloss{type} and its
generalization \spadgloss{mode}.
We show that every \Language{} object has a type that
determines what you can do with the object.
In particular, we explain how to use types to call specific functions
from particular parts of the library and how types and modes can be used
to create new objects from old.
We also look at \pspadtype{Record} and \pspadtype{Union} types and
the special type \spadtype{Any}.
Finally, we give you an idea of how \Language{} manipulates types and
modes internally to resolve ambiguities.

% *********************************************************************
\head{section}{The Basic Idea}{ugTypesBasic}
% *********************************************************************

The \Language{} world deals with many kinds of objects.
There are mathematical objects such as numbers and polynomials,
data structure objects such as lists and arrays, and graphics
objects such as points and graphic images.
Functions are objects too.

\Language{} organizes objects using the notion of \spadglossSee{domain of
computation}{domain}, or simply \spadgloss{domain}.
Each domain denotes a class of objects.
The class of objects it denotes is usually given by the name of the
domain: \spadtype{Integer} for the integers, \spadtype{Float} for
floating-point numbers, and so on.
The convention is that the first letter of a domain name is capitalized.
Similarly, the domain \spadtype{Polynomial(Integer)} denotes ``polynomials
with integer coefficients.''
Also, \spadtype{Matrix(Float)} denotes ``matrices with floating-point
entries.''

Every basic \Language{} object belongs to a unique domain.
The integer \spad{3} belongs to the domain \spadtype{Integer} and
the polynomial \spad{x + 3} belongs to the domain
\spadtype{Polynomial(Integer)}.
The domain of an object is also called its \spadgloss{type}.
Thus we speak of ``the type \spadtype{Integer}''
and ``the type \spadtype{Polynomial(Integer)}.''

\begin{xtc}
\begin{xtccomment}
After an \Language{} computation, the type is displayed toward the
right-hand side of the page (or screen).
\end{xtccomment}
\begin{spadsrc}
-3
\end{spadsrc}
\begin{TeXOutput}
\begin{fricasmath}{1}
-{3}%
\end{fricasmath}
\end{TeXOutput}
\formatResultType{Integer}
\end{xtc}
\begin{xtc}
\begin{xtccomment}
Here we create a rational number but it looks like the last result.
The type however tells you it is different.
You cannot identify the type of an object by how \Language{}
displays the object.
\end{xtccomment}
\begin{spadsrc}
-3/1
\end{spadsrc}
\begin{TeXOutput}
\begin{fricasmath}{2}
-{3}%
\end{fricasmath}
\end{TeXOutput}
\formatResultType{Fraction(Integer)}
\end{xtc}
\begin{xtc}
\begin{xtccomment}
When a computation produces a result of a simpler type, \Language{} leaves
the type unsimplified.
Thus no information is lost.
\end{xtccomment}
\begin{spadsrc}
x + 3 - x 
\end{spadsrc}
\begin{TeXOutput}
\begin{fricasmath}{3}
3%
\end{fricasmath}
\end{TeXOutput}
\formatResultType{Polynomial(Integer)}
\end{xtc}
\begin{xtc}
\begin{xtccomment}
This seldom matters since \Language{} retracts the answer to the
simpler type if it is necessary.
\end{xtccomment}
\begin{spadsrc}
factorial(%) 
\end{spadsrc}
\begin{TeXOutput}
\begin{fricasmath}{4}
6%
\end{fricasmath}
\end{TeXOutput}
\formatResultType{Expression(Integer)}
\end{xtc}
\begin{xtc}
\begin{xtccomment}
When you issue a positive number, the type \spadtype{PositiveInteger} is
printed.
Surely, \spad{3} also has type \spadtype{Integer}!
The curious reader may now have two questions.
First, is the type of an object not unique?
Second, how is \spadtype{PositiveInteger} related to \spadtype{Integer}?
Read on!
\end{xtccomment}
\begin{spadsrc}
3
\end{spadsrc}
\begin{TeXOutput}
\begin{fricasmath}{5}
3%
\end{fricasmath}
\end{TeXOutput}
\formatResultType{PositiveInteger}
\end{xtc}

Any domain can be refined to a \spadgloss{subdomain} by a membership
\spadgloss{predicate}.\footnote{A predicate is a function that,
when applied to an object of the domain, returns either
\spad{true} or \spad{false}.}
For example, the domain \spadtype{Integer} can be refined to the
subdomain \spadtype{PositiveInteger}, the set of integers
\spad{x} such that \spad{x > 0}, by giving the \Language{}
predicate \spad{x +-> x > 0}.
Similarly, \Language{} can define subdomains such as ``the
subdomain of diagonal matrices,'' ``the subdomain of lists of length
two,'' ``the subdomain of monic irreducible polynomials in
\spad{x},'' and so on.
Trivially, any domain is a subdomain of itself.

While an object belongs to a unique domain, it can belong to any
number of subdomains.
Any subdomain of the domain of an object can be used as the
{\it type} of that object.
The type of \spad{3} is indeed both \spadtype{Integer} and
\spadtype{PositiveInteger} as well as any other subdomain of
integer whose predicate is satisfied, such as ``the prime
integers,'' ``the odd positive integers between 3 and 17,'' and so
on.

% *********************************************************************
\head{subsection}{Domain Constructors}{ugTypesBasicDomainCons}
% *********************************************************************

In \Language{}, domains are objects.
You can create them, pass them to functions, and, as we'll see later, test
them for certain properties.

In \Language{}, you ask for a value of a function by applying its name
to a set of arguments.

\begin{xtc}
\begin{xtccomment}
To ask for ``the factorial of 7'' you enter this expression to
\Language{}.
This applies the function \spad{factorial} to the value \spad{7}
to compute the result.
\end{xtccomment}
\begin{spadsrc}
factorial(7)
\end{spadsrc}
\begin{TeXOutput}
\begin{fricasmath}{1}
5040%
\end{fricasmath}
\end{TeXOutput}
\formatResultType{PositiveInteger}
\end{xtc}
\begin{xtc}
\begin{xtccomment}
Enter the type \spadtype{Polynomial (Integer)} as an expression to
\Language{}.
This looks much like a function call as well.
It is!
The result is appropriately stated to be of type
\spadtype{Domain}, which
according to our usual convention, denotes the class of all domains.
\end{xtccomment}
\begin{spadsrc}
Polynomial(Integer)
\end{spadsrc}
\begin{TeXOutput}
\begin{fricasmath}{2}
\STRING{Polynomial(Integer)}%
\end{fricasmath}
\end{TeXOutput}
\formatResultType{Type}
\end{xtc}

The most basic operation involving domains is that of building a new
domain from a given one.
To create the domain of ``polynomials over the integers,'' \Language{}
applies the function \spadtype{Polynomial} to the domain
\spadtype{Integer}.
A function like \spadtype{Polynomial} is called a \spadgloss{domain
constructor} or,
\index{constructor!domain}
more simply, a
\spadgloss{constructor}.
A domain constructor is a function that creates a domain.
An argument to a domain constructor can be another domain or, in general,
an arbitrary kind of object.
\spadtype{Polynomial} takes a single domain argument while
\spadtype{SquareMatrix} takes a positive integer as an argument
to give its dimension and
a domain argument to give the type of its components.

What kinds of domains can you use as the argument to
\spadtype{Polynomial} or \spadtype{SquareMatrix} or
\spadtype{List}?
Well, the first two are mathematical in nature.
You want to be able to perform algebraic operations like
\spadop{+} and \spadop{*} on polynomials and square matrices,
and operations such as \spadfun{determinant} on square matrices.
So you want to allow polynomials of integers {\it and} polynomials
of square matrices with complex number coefficients and, in
general, anything that ``makes sense.'' At the same time, you
don't want \Language{} to be able to build nonsense domains such
as ``polynomials of strings!''

In contrast to algebraic structures, data structures can hold any
kind of object.
Operations on lists such as \spadfunFrom{insert}{List},
\spadfunFrom{delete}{List}, and \spadfunFrom{concat}{List} just
manipulate the list itself without changing or operating on its
elements.
Thus you can build \spadtype{List} over almost any datatype,
including itself.
\begin{xtc}
\begin{xtccomment}
Create a complicated algebraic domain.
\end{xtccomment}
\begin{spadsrc}
List (List (Matrix (Polynomial (Complex (Fraction (Integer))))))
\end{spadsrc}
\begin{TeXOutput}
\begin{fricasmath}{3}
\STRING{List(List(Matrix(Polynomial(Complex(Fraction(Integer))))))}%
\end{fricasmath}
\end{TeXOutput}
\formatResultType{Type}
\end{xtc}
\begin{xtc}
\begin{xtccomment}
Try to create a meaningless domain.
\end{xtccomment}
\begin{spadsrc}
Polynomial(String)
\end{spadsrc}
\begin{MessageOutput}
   Polynomial(String) is not a valid type.
\end{MessageOutput}
\end{xtc}
Evidently from our last example, \Language{} has some mechanism
that tells what a constructor can use as an argument.
This brings us to the notion of \spadgloss{category}.
As domains are objects, they too have a domain.
The domain of a domain is a category.
A category is simply a type whose members are domains.

A common algebraic category is \spadtype{Ring}, the class of all domains
that are ``rings.''
A ring is an algebraic structure with constants \spad{0} and \spad{1} and
operations \spadopFrom{+}{Ring}, \spadopFrom{-}{Ring}, and
\spadopFrom{*}{Ring}.
These operations are assumed ``closed'' with respect to the domain,
meaning that they take two objects of the domain and produce a result
object also in the domain.
The operations are understood to satisfy certain ``axioms,'' certain
mathematical principles providing the algebraic foundation for rings.
For example, the {\it additive inverse axiom} for rings states:
\begin{center}
Every element \spad{x} has an additive inverse \spad{y} such
that \spad{x + y = 0}.
\end{center}
The prototypical example of a domain that is a ring is the integers.
Keep them in mind whenever we mention \spadtype{Ring}.

Many algebraic domain constructors such as \spadtype{Complex},
\spadtype{Polynomial}, \spadtype{Fraction}, take rings as
arguments and return rings as values.
You can use the infix operator ``\spad{has}''
\spadkey{has}
to ask a domain if it belongs to a particular category.

\begin{xtc}
\begin{xtccomment}
All numerical types are rings.
Domain constructor \spadtype{Polynomial} builds ``the ring of polynomials
over any other ring.''
\end{xtccomment}
\begin{spadsrc}
Polynomial(Integer) has Ring
\end{spadsrc}
\begin{TeXOutput}
\begin{fricasmath}{4}
\STRING{true}%
\end{fricasmath}
\end{TeXOutput}
\formatResultType{Boolean}
\end{xtc}
\begin{xtc}
\begin{xtccomment}
Constructor \spadtype{List} never produces a ring.
\end{xtccomment}
\begin{spadsrc}
List(Integer) has Ring
\end{spadsrc}
\begin{TeXOutput}
\begin{fricasmath}{5}
\STRING{false}%
\end{fricasmath}
\end{TeXOutput}
\formatResultType{Boolean}
\end{xtc}
\begin{xtc}
\begin{xtccomment}
The constructor \spadtype{Matrix(R)} builds ``the domain of all matrices
over the ring \spad{R}.'' This domain is never a ring since the operations
\spadSyntax{+}, \spadSyntax{-}, and \spadSyntax{*} on matrices of arbitrary
shapes are undefined.
\end{xtccomment}
\begin{spadsrc}
Matrix(Integer) has Ring
\end{spadsrc}
\begin{TeXOutput}
\begin{fricasmath}{6}
\STRING{false}%
\end{fricasmath}
\end{TeXOutput}
\formatResultType{Boolean}
\end{xtc}
\begin{xtc}
\begin{xtccomment}
Thus you can never build polynomials over matrices.
\end{xtccomment}
\begin{spadsrc}
Polynomial(Matrix(Integer))
\end{spadsrc}
\begin{MessageOutput}
   Polynomial(Matrix(Integer)) is not a valid type.
\end{MessageOutput}
\end{xtc}
\begin{xtc}
\begin{xtccomment}
Use \spadtype{SquareMatrix(n,R)} instead.
For any positive integer \spad{n}, it builds ``the ring of \spad{n} by
\spad{n} matrices over \spad{R}.''
\end{xtccomment}
\begin{spadsrc}
Polynomial(SquareMatrix(7,Complex(Integer)))
\end{spadsrc}
\begin{TeXOutput}
\begin{fricasmath}{7}
\STRING{Polynomial(SquareMatrix(7,Complex(Integer)))}%
\end{fricasmath}
\end{TeXOutput}
\formatResultType{Type}
\end{xtc}

Another common category is \spadtype{Field}, the class of all fields.
\index{field}
A field is a ring with additional operations.
For example, a field has commutative multiplication and
a closed operation \spadopFrom{/}{Field} for the
division of two elements.
\spadtype{Integer} is not a field since, for example, \spad{3/2} does not
have an integer result.
The prototypical example of a field is the rational numbers, that is, the
domain \spadtype{Fraction(Integer)}.
In general, the constructor \spadtype{Fraction} takes a ring as an
argument and returns a field.\footnote{Actually,
the argument domain must have some additional
properties so as to belong to category \spadtype{IntegralDomain}.}
Other domain constructors, such as \spadtype{Complex}, build fields only if
their argument domain is a field.

\begin{xtc}
\begin{xtccomment}
The complex integers (often called the ``Gaussian integers'') do not form
a field.
\end{xtccomment}
\begin{spadsrc}
Complex(Integer) has Field
\end{spadsrc}
\begin{TeXOutput}
\begin{fricasmath}{8}
\STRING{false}%
\end{fricasmath}
\end{TeXOutput}
\formatResultType{Boolean}
\end{xtc}
\begin{xtc}
\begin{xtccomment}
But fractions of complex integers do.
\end{xtccomment}
\begin{spadsrc}
Fraction(Complex(Integer)) has Field
\end{spadsrc}
\begin{TeXOutput}
\begin{fricasmath}{9}
\STRING{true}%
\end{fricasmath}
\end{TeXOutput}
\formatResultType{Boolean}
\end{xtc}
\begin{xtc}
\begin{xtccomment}
The algebraically equivalent domain of complex rational numbers is a field
since domain constructor \spadtype{Complex} produces a field whenever its
argument is a field.
\end{xtccomment}
\begin{spadsrc}
Complex(Fraction(Integer)) has Field
\end{spadsrc}
\begin{TeXOutput}
\begin{fricasmath}{10}
\STRING{true}%
\end{fricasmath}
\end{TeXOutput}
\formatResultType{Boolean}
\end{xtc}

The most basic category is \spadtype{Type}.
\exptypeindex{Type}
It denotes the class of all domains and
subdomains.\footnote{\spadtype{Type} does not denote the class of
all types.
The type of all categories is \spadtype{Category}.
The type of \spadtype{Type} itself is undefined.}
Domain constructor \spadtype{List} is able to build ``lists of
elements from domain \spad{D}'' for arbitrary \spad{D} simply by
requiring that \spad{D} belong to category \spadtype{Type}.

Now, you may ask, what exactly is a category?
\index{category}
Like domains, categories can be defined in the \Language{} language.
A category is defined by three components:
%
\begin{enumerate}
\item a name (for example, \spadtype{Ring}),
used to refer to the class of domains that the category represents;
\item a set of operations, used to refer to the operations that
the domains of this class support
(for example, \spadop{+}, \spadop{-}, and \spadop{*} for rings); and
\item an optional list of other categories that this category extends.
\end{enumerate}
%
This last component is a new idea.
And it is key to the design of \Language{}!
Because categories can extend one another, they form hierarchies.
Detailed charts showing the category hierarchies in \Language{} are
displayed in the endpages of this book.
There you see that all categories are extensions of \spadtype{Type} and that
\spadtype{Field} is an extension of \spadtype{Ring}.

The operations supported by the domains of a category are called the
\spadglossSee{exports}{export} of that category because these are the
operations made available for system-wide use.
The exports of a domain of a given category are not only the ones
explicitly mentioned by the category.
Since a category extends other categories, the operations of these other
categories---and all categories these other categories extend---are also
exported by the domains.

For example, polynomial domains belong to \spadtype{PolynomialCategory}.
This category explicitly mentions some twenty-nine
operations on polynomials, but it
extends eleven other categories (including \spadtype{Ring}).
As a result, the current system has over one hundred operations on polynomials.

If a domain belongs to a category that extends, say, \spadtype{Ring}, it
is convenient to say that the domain exports \spadtype{Ring}.
The name of the category thus provides a convenient shorthand for the list
of operations exported by the category.
Rather than listing operations such as \spadopFrom{+}{Ring} and
\spadopFrom{*}{Ring} of \spadtype{Ring} each time they are needed, the
definition of a type simply asserts that it exports category
\spadtype{Ring}.

The category name, however, is more than a shorthand.
The name \spadtype{Ring}, in fact, implies that the operations exported by
rings are required to satisfy a set of ``axioms'' associated with the name
\spadtype{Ring}.\footnote{This subtle
but important feature distinguishes \Language{} from
other abstract datatype designs.}

Why is it not correct to assume that some type is a ring if it exports all
of the operations of \spadtype{Ring}?
Here is why.
Some languages such as {\bf APL}
\index{APL}
denote the \spadtype{Boolean} constants \spad{true} and
\spad{false} by the integers \spad{1} and \spad{0}
respectively, then use \spadop{+} and \spadop{*} to denote the
logical operators \spadfun{or} and \spadfun{and}.
But with these definitions
\spadtype{Boolean} is not a ring since the additive inverse
axiom is violated.\footnote{There is no inverse element \spad{a}
such that \spad{1 + a = 0}, or, in the usual terms:
\spad{true or a = false}.}
This alternative definition of \spadtype{Boolean} can be easily
and correctly implemented in \Language{}, since
\spadtype{Boolean} simply does not assert that it is of category
\spadtype{Ring}.
This prevents the system from building meaningless domains such as
\spadtype{Polynomial(Boolean)} and then wrongfully applying
algorithms that presume that the ring axioms hold.


Enough on categories. To learn more about them, see
\chapref{ugCategories}.
We now return to our discussion of domains.

Domains \spadgloss{export} a set of operations to make them
available for system-wide use.
\spadtype{Integer}, for example, exports the operations
\spadopFrom{+}{Integer} and \spadopFrom{=}{Integer} given by
the \spadglossSee{signatures}{signature}
\spadopFrom{+}{Integer}: \spad{(Integer,Integer)->Integer} and
\spadopFrom{=}{Integer}: \spad{(Integer,Integer)->Boolean},
respectively.
Each of these operations takes two \spadtype{Integer} arguments.
The \spadopFrom{+}{Integer} operation also returns an \spadtype{Integer} but
\spadopFrom{=}{Integer} returns a \spadtype{Boolean}: \spad{true} or
\spad{false}.
The operations exported by a domain usually manipulate objects of
the domain---but not always.

The operations of a domain may actually take as arguments, and return as
values, objects from any domain.
For example, \spadtype{Fraction (Integer)} exports the operations
\spadopFrom{/}{Fraction}: \spad{(Integer,Integer)->Fraction(Integer)}
and \spadfunFrom{characteristic}{Fraction}:
\spad{->NonNegativeInteger}.

Suppose all operations of a domain take as arguments and return
as values, only objects from {\it other} domains.
\index{package}
This kind of domain
\index{constructor!package}
is what \Language{} calls a \spadgloss{package}.

A package does not designate a class of objects at all.
Rather, a package is just a collection of operations.
Actually the bulk of the \Language{} library of algorithms consists
of packages.
The facilities for factorization; integration; solution of linear,
polynomial, and differential equations; computation of limits; and so
on, are all defined in packages.
Domains needed by algorithms can be passed to a package as arguments or
used by name if they are not ``variable.''
Packages are useful for defining operations that convert objects of one
type to another, particularly when these types have different
parameterizations.
As an example, the package \spadtype{PolynomialFunction2(R,S)} defines
operations that convert polynomials over a domain \spad{R} to polynomials
over \spad{S}.
To convert an object from \spadtype{Polynomial(Integer)} to
\spadtype{Polynomial(Float)}, \Language{} builds the package
\spadtype{PolynomialFunctions2(Integer,Float)} in order to create the
required conversion function.
(This happens ``behind the scenes'' for you: see \spadref{ugTypesConvert}
for details on how to convert objects.)

\Language{} categories, domains and packages and all their contained
functions are written in the \Language{} programming language and have
been compiled into machine code.
This is what comprises the \Language{} \spadgloss{library}.
In the rest of this book we show you how to use these domains and
their functions and how to write your own functions.

% *********************************************************************
\head{section}{Writing Types and Modes}{ugTypesWriting}
% *********************************************************************
%

We have already seen in
the last section
several examples of types.
Most of these examples had either no arguments (for example,
\spadtype{Integer}) or one argument (for example,
\spadtype{Polynomial (Integer)}).
In this section we give details about writing arbitrary types.
We then define \spadglossSee{modes}{mode} and discuss how to write
them.
We conclude the section with a discussion on constructor
abbreviations.

\begin{xtc}
\begin{xtccomment}
When might you need to write a type or mode?
You need to do so when you declare variables.
\end{xtccomment}
\begin{spadsrc}
a : PositiveInteger
\end{spadsrc}
\end{xtc}
\begin{xtc}
\begin{xtccomment}
You need to do so when you declare functions
(\spadref{ugTypesDeclare}),
\end{xtccomment}
\begin{spadsrc}
f : Integer -> String
\end{spadsrc}
\end{xtc}
\begin{xtc}
\begin{xtccomment}
You need to do so when you convert an object from one type to another
(\spadref{ugTypesConvert}).
\end{xtccomment}
\begin{spadsrc}
factor(2 :: Complex(Integer))
\end{spadsrc}
\begin{TeXOutput}
\begin{fricasmath}{3}
-{\ImaginaryI \TIMES \SUPER{\PAREN{1+\ImaginaryI }}{2}}%
\end{fricasmath}
\end{TeXOutput}
\formatResultType{Factored(Complex(Integer))}
\end{xtc}
\begin{xtc}
\begin{xtccomment}
\end{xtccomment}
\begin{spadsrc}
(2 = 3)$Integer
\end{spadsrc}
\begin{TeXOutput}
\begin{fricasmath}{4}
\STRING{false}%
\end{fricasmath}
\end{TeXOutput}
\formatResultType{Boolean}
\end{xtc}
\begin{xtc}
\begin{xtccomment}
You need to do so when you give computation target type information
(\spadref{ugTypesPkgCall}).
\end{xtccomment}
\begin{spadsrc}
(2 = 3)@Boolean
\end{spadsrc}
\begin{TeXOutput}
\begin{fricasmath}{5}
\STRING{false}%
\end{fricasmath}
\end{TeXOutput}
\formatResultType{Boolean}
\end{xtc}

% *********************************************************************
\head{subsection}{Types with No Arguments}{ugTypesWritingZero}
% *********************************************************************

A constructor with no arguments can be written either
\index{type!using parentheses}
with or without
\index{parentheses!using with types}
trailing opening and closing parentheses (\spadSyntax{()}).
\begin{center}
\begin{tabular}{ccc}
\spadtype{Boolean()} is the same as \spadtype{Boolean} & \quad &
\spadtype{Integer()} is the same as \spadtype{Integer} \\
\spadtype{String()} is the same as \spadtype{String} & \quad &
\spadtype{Void()} is the same as \spadtype{Void} \\
\end{tabular}
\end{center}
and so on.
It is customary to omit the parentheses.

% *********************************************************************
\head{subsection}{Types with One Argument}{ugTypesWritingOne}
% *********************************************************************

A constructor with one argument can frequently be
\index{type!using parentheses}
written with no
\index{parentheses!using with types}
parentheses.
Types nest from right to left so that
\spadtype{Complex Fraction Polynomial Integer} is the same as
\spadtype{Complex (Fraction (Polynomial (Integer)))}.
You need to use parentheses to force the application of a constructor
to the correct argument, but you need not use any more than is
necessary to remove ambiguities.

Here are some guidelines for using parentheses (they are possibly slightly
more restrictive than they need to be).
\begin{xtc}
\begin{xtccomment}
If the argument is an expression like \spad{2 + 3}
then you must enclose the argument in parentheses.
\end{xtccomment}
\begin{spadsrc}
e : PrimeField(2 + 3)
\end{spadsrc}
\end{xtc}
%
\begin{xtc}
\begin{xtccomment}
If the type is to be used with package calling
then you must enclose the argument in parentheses.
\end{xtccomment}
\begin{spadsrc}
content(2)$Polynomial(Integer)
\end{spadsrc}
\begin{TeXOutput}
\begin{fricasmath}{2}
2%
\end{fricasmath}
\end{TeXOutput}
\formatResultType{Integer}
\end{xtc}
\begin{xtc}
\begin{xtccomment}
Alternatively, you can write the type without parentheses
then enclose the whole type expression with parentheses.
\end{xtccomment}
\begin{spadsrc}
content(2)$(Polynomial Complex Fraction Integer)
\end{spadsrc}
\begin{TeXOutput}
\begin{fricasmath}{3}
2%
\end{fricasmath}
\end{TeXOutput}
\formatResultType{Complex(Fraction(Integer))}
\end{xtc}
\begin{xtc}
\begin{xtccomment}
If you supply computation target type information
(\spadref{ugTypesPkgCall})
then you should enclose the argument in parentheses.
\end{xtccomment}
\begin{spadsrc}
(2/3)@Fraction(Polynomial(Integer))
\end{spadsrc}
\begin{TeXOutput}
\begin{fricasmath}{4}
\frac{2}{3}%
\end{fricasmath}
\end{TeXOutput}
\formatResultType{Fraction(Polynomial(Integer))}
\end{xtc}
%
\begin{xtc}
\begin{xtccomment}
If the type itself has parentheses around it and we are
not in the case of the first example above,
then the parentheses can usually be omitted.
\end{xtccomment}
\begin{spadsrc}
(2/3)@Fraction(Polynomial Integer)
\end{spadsrc}
\begin{TeXOutput}
\begin{fricasmath}{5}
\frac{2}{3}%
\end{fricasmath}
\end{TeXOutput}
\formatResultType{Fraction(Polynomial(Integer))}
\end{xtc}
%
\begin{xtc}
\begin{xtccomment}
If the type is used in a declaration and the argument is a single-word
type, integer or symbol,
then the parentheses can usually be omitted.
\end{xtccomment}
\begin{spadsrc}
(d,f,g) : Complex Polynomial Integer
\end{spadsrc}
\end{xtc}

% *********************************************************************
\head{subsection}{Types with More Than One Argument}{ugTypesWritingMore}
% *********************************************************************

If a constructor
\index{type!using parentheses}
has more than
\index{parentheses!using with types}
one argument, you must use parentheses.
Some examples are
\begin{center}
\spadtype{UnivariatePolynomial(x, Float)} \\
\spadtype{MultivariatePolynomial([z,w,r], Complex Float)} \\
\spadtype{SquareMatrix(3, Integer)} \\
\spadtype{FactoredFunctions2(Integer,Fraction Integer)}
\end{center}

% *********************************************************************
\head{subsection}{Modes}{ugTypesWritingModes}
% *********************************************************************

A \spadgloss{mode} is a type that possibly is a
question mark (\spadSyntax{?}) or contains one in an argument
position.
For example, the following are all modes.
\begin{center}
\begin{tabular}{ccc}
\spadtype{?} & \quad &
\spadtype{Polynomial ?} \\
\spadtype{Matrix Polynomial ?} & \quad &
\spadtype{SquareMatrix(3,?)} \\
\spadtype{Integer} & \quad &
\spadtype{OneDimensionalArray(Float)}
\end{tabular}
\end{center}

As is evident from these examples, a mode is a type with a
part that is not specified (indicated by a question mark).
Only one \spadSyntax{?} is allowed per mode and it must appear in the
most deeply nested argument that is a type. Thus
\nonLibAxiomType{?(Integer)},
\nonLibAxiomType{Matrix(? (Polynomial))},
\nonLibAxiomType{SquareMatrix(?, Integer)} and
\nonLibAxiomType{SquareMatrix(?, ?)} are all invalid.
The question mark must take the place of a domain, not data (for example,
the integer that is the dimension of a square matrix).
This rules out, for example, the two \spadtype{SquareMatrix}
expressions.

Modes can be used for declarations
(\spadref{ugTypesDeclare})
and conversions
(\spadref{ugTypesConvert}).
However, you cannot use a mode for package calling or giving target
type information.

% *********************************************************************
\head{subsection}{Abbreviations}{ugTypesWritingAbbr}
% *********************************************************************

Every constructor has an abbreviation that
\index{abbreviation!constructor}
you can freely
\index{constructor!abbreviation}
substitute for the constructor name.
In some cases, the abbreviation is nothing more than the
capitalized version of the constructor name.

\beginImportant
Aside from allowing types to be written more concisely,
abbreviations are used by \Language{} to name various system
files for constructors (such as library filenames, test input
files and example files).
Here are some common abbreviations.
\begin{center}
\begin{tabular}{ll}
\small\spadtype{COMPLEX}   abbreviates \spadtype{Complex}             &
\small\spadtype{DFLOAT}    abbreviates \spadtype{DoubleFloat}         \\
\small\spadtype{EXPR}      abbreviates \spadtype{Expression}          &
\small\spadtype{FLOAT}     abbreviates \spadtype{Float}               \\
\small\spadtype{FRAC}      abbreviates \spadtype{Fraction}            &
\small\spadtype{INT}       abbreviates \spadtype{Integer}             \\
\small\spadtype{MATRIX}    abbreviates \spadtype{Matrix}              &
\small\spadtype{NNI}       abbreviates \spadtype{NonNegativeInteger}  \\
\small\spadtype{PI}        abbreviates \spadtype{PositiveInteger}     &
\small\spadtype{POLY}      abbreviates \spadtype{Polynomial}          \\
\small\spadtype{STRING}    abbreviates \spadtype{String}              &
\small\spadtype{UP}        abbreviates \spadtype{UnivariatePolynomial}\\
\end{tabular}
\end{center}
\endImportant

You can combine both full constructor names and abbreviations
in a type expression.
Here are some types using abbreviations.
\begin{center}
\spadtype{POLY INT} is the same as \spadtype{Polynomial(INT)} \\
\spadtype{POLY(Integer)} is the same as \spadtype{Polynomial(Integer)} \\
\spadtype{POLY(Integer)} is the same as \spadtype{Polynomial(INT)} \\
\spadtype{FRAC(COMPLEX(INT))} is the same as \spadtype{Fraction Complex Integer} \\
\spadtype{FRAC(COMPLEX(INT))} is the same as \spadtype{FRAC(Complex Integer)} \\
\end{center}

There are several ways of finding the names of constructors and
their abbreviations.
For a specific constructor, use \spadsys{)abbreviation query}.
\syscmdindex{abbreviation}
You can also use the \spadsys{)what} system command to see the names
and abbreviations of constructors.
\syscmdindex{what}
For more information about \spadsys{)what}, see
\spadref{ugSysCmdwhat}.
\begin{xtc}
\begin{xtccomment}
\spadsys{)abbreviation query} can be
abbreviated (no pun intended) to \spadsys{)abb q}.
\end{xtccomment}
\begin{spadsrc}
)abb q Integer
\end{spadsrc}
\begin{SysCmdOutput}
   INT abbreviates domain Integer 
\end{SysCmdOutput}
\end{xtc}
\begin{xtc}
\begin{xtccomment}
The \spadsys{)abbreviation query} command lists
the constructor name if you give the abbreviation.
Issue \spadsys{)abb q} if you want to see the names and abbreviations
of all \Language{} constructors.
\end{xtccomment}
\begin{spadsrc}
)abb q DMP
\end{spadsrc}
\begin{SysCmdOutput}
   DMP abbreviates domain DistributedMultivariatePolynomial 
\end{SysCmdOutput}
\end{xtc}
\begin{xtc}
\begin{xtccomment}
Issue this to see all packages whose names contain the string ``ode''.
\syscmdindex{what packages}
\end{xtccomment}
\begin{spadsrc}
)what packages ode
\end{spadsrc}
\begin{SysCmdOutput}
-------------------------------- Packages ---------------------------------

Packages with names matching patterns:
     ode 

 COMPCODE compCode                     EXPRODE  ExpressionSpaceODESolver
 FCPAK1   FortranCodePackage1          FCTOOL   FortranCodeTools
 GRAY     GrayCode                     LODEEF   ElementaryFunctionLODESolver
 NODE1    NonLinearFirstOrderODESolver  ODECONST ConstantLODE
 ODEEF    ElementaryFunctionODESolver  ODEINT   ODEIntegration
 ODEPAL   PureAlgebraicLODE            ODERAT   RationalLODE
 ODERED   ReduceLODE                   ODESYS   SystemODESolver
 ODETOOLS ODETools
 UTSODE   UnivariateTaylorSeriesODESolver
 UTSODETL UTSodetools
\end{SysCmdOutput}
\end{xtc}

% *********************************************************************
\head{section}{Declarations}{ugTypesDeclare}
% *********************************************************************
%
A \spadgloss{declaration} is an expression used
to restrict the type of values that can be assigned to variables.
A colon (\spadSyntax{:}) is always used after a variable or
list of variables to be declared.

\beginImportant
For a single variable, the syntax for declaration is
\begin{center}
{\it variableName \spad{:} typeOrMode}
\end{center}
For multiple variables, the syntax is
\begin{center}
{\tt (\subscriptIt{variableName}{1}, \subscriptIt{variableName}{2}, \ldots \subscriptIt{variableName}{N}): {\it typeOrMode}}
\end{center}
\endImportant

You can always combine a declaration with an assignment.
When you do, it is equivalent to first giving a declaration statement,
then giving an assignment.
For more information on assignment, see
\spadref{ugIntroAssign} and
\spadref{ugLangAssign}.
To see how to declare your own functions, see
\spadref{ugUserDeclare}.

\begin{xtc}
\begin{xtccomment}
This declares one variable to have a type.
\end{xtccomment}
\begin{spadsrc}
a : Integer 
\end{spadsrc}
\end{xtc}
\begin{xtc}
\begin{xtccomment}
This declares several variables to have a type.
\end{xtccomment}
\begin{spadsrc}
(b,c) : Integer 
\end{spadsrc}
\end{xtc}
\begin{xtc}
\begin{xtccomment}
\spad{a, b} and \spad{c} can only hold integer values.
\end{xtccomment}
\begin{spadsrc}
a := 45 
\end{spadsrc}
\begin{TeXOutput}
\begin{fricasmath}{3}
45%
\end{fricasmath}
\end{TeXOutput}
\formatResultType{Integer}
\end{xtc}
\begin{xtc}
\begin{xtccomment}
If a value cannot be converted to a declared type,
an error message is displayed.
\end{xtccomment}
\begin{spadsrc}
b := 4/5 
\end{spadsrc}
\begin{MessageOutput}
   Cannot convert right-hand side of assignment
   4
   -
   5

      to an object of the type Integer of the left-hand side.
\end{MessageOutput}
\end{xtc}
\begin{xtc}
\begin{xtccomment}
This declares a variable with a mode.
\end{xtccomment}
\begin{spadsrc}
n : Complex ? 
\end{spadsrc}
\end{xtc}
\begin{xtc}
\begin{xtccomment}
This declares several variables with a mode.
\end{xtccomment}
\begin{spadsrc}
(p,q,r) : Matrix Polynomial ? 
\end{spadsrc}
\end{xtc}
\begin{xtc}
\begin{xtccomment}
This complex object has integer real and imaginary parts.
\end{xtccomment}
\begin{spadsrc}
n := -36 + 9 * %i 
\end{spadsrc}
\begin{TeXOutput}
\begin{fricasmath}{6}
-{36}+9\TIMES \ImaginaryI %
\end{fricasmath}
\end{TeXOutput}
\formatResultType{Complex(Integer)}
\end{xtc}
\begin{xtc}
\begin{xtccomment}
This complex object has fractional symbolic real and imaginary parts.
\end{xtccomment}
\begin{spadsrc}
n := complex(4/(x + y),y/x) 
\end{spadsrc}
\begin{TeXOutput}
\begin{fricasmath}{7}
\frac{4}{\SYMBOL{y}+\SYMBOL{x}}+\frac{\SYMBOL{y}}{\SYMBOL{x}}\TIMES %
\ImaginaryI %
\end{fricasmath}
\end{TeXOutput}
\formatResultType{Complex(Fraction(Polynomial(Integer)))}
\end{xtc}
\begin{xtc}
\begin{xtccomment}
This matrix has entries that are polynomials with integer
coefficients.
\end{xtccomment}
\begin{spadsrc}
p := [[1,2],[3,4],[5,6]] 
\end{spadsrc}
\begin{TeXOutput}
\begin{fricasmath}{8}
\begin{MATRIX}{2}1&2\\3&4\\5&6\end{MATRIX}%
\end{fricasmath}
\end{TeXOutput}
\formatResultType{Matrix(Polynomial(Integer))}
\end{xtc}
\begin{xtc}
\begin{xtccomment}
This matrix has a single entry that is a polynomial with
rational number coefficients.
\end{xtccomment}
\begin{spadsrc}
q := [[x - 2/3]] 
\end{spadsrc}
\begin{TeXOutput}
\begin{fricasmath}{9}
\begin{MATRIX}{1}\SYMBOL{x}-{\frac{2}{3}}\end{MATRIX}%
\end{fricasmath}
\end{TeXOutput}
\formatResultType{Matrix(Polynomial(Fraction(Integer)))}
\end{xtc}
\begin{xtc}
\begin{xtccomment}
This matrix has entries that are polynomials with complex integer
coefficients.
\end{xtccomment}
\begin{spadsrc}
r := [[1-%i*x,7*y+4*%i]] 
\end{spadsrc}
\begin{TeXOutput}
\begin{fricasmath}{10}
\begin{MATRIX}{2}-{\ImaginaryI \TIMES \SYMBOL{x}}+1&7\TIMES \SYMBOL{y}+4%
\TIMES \ImaginaryI \end{MATRIX}%
\end{fricasmath}
\end{TeXOutput}
\formatResultType{Matrix(Polynomial(Complex(Integer)))}
\end{xtc}
%
\begin{xtc}
\begin{xtccomment}
Note the difference between this and the next example.
This is a complex object with polynomial real and imaginary parts.
\end{xtccomment}
\begin{spadsrc}
f : COMPLEX POLY ? := (x + y*%i)^2
\end{spadsrc}
\begin{TeXOutput}
\begin{fricasmath}{11}
-{\SUPER{\SYMBOL{y}}{2}}+\SUPER{\SYMBOL{x}}{2}+2\TIMES \SYMBOL{x}\TIMES %
\SYMBOL{y}\TIMES \ImaginaryI %
\end{fricasmath}
\end{TeXOutput}
\formatResultType{Complex(Polynomial(Integer))}
\end{xtc}
\begin{xtc}
\begin{xtccomment}
This is a polynomial with complex integer coefficients.
The objects are convertible from one to the other.
See \spadref{ugTypesConvert} for more information.
\end{xtccomment}
\begin{spadsrc}
g : POLY COMPLEX ? := (x + y*%i)^2
\end{spadsrc}
\begin{TeXOutput}
\begin{fricasmath}{12}
-{\SUPER{\SYMBOL{y}}{2}}+2\TIMES \ImaginaryI \TIMES \SYMBOL{x}\TIMES \SYMBOL{%
y}+\SUPER{\SYMBOL{x}}{2}%
\end{fricasmath}
\end{TeXOutput}
\formatResultType{Polynomial(Complex(Integer))}
\end{xtc}

% *********************************************************************
\head{section}{Records}{ugTypesRecords}
% *********************************************************************
%
A \pspadtype{Record} is an object composed of one or more other objects,
\index{Record@\protect\nonLibAxiomType{Record}}
each of which is referenced
\index{selector!record}
with
\index{record!selector}
a \spadgloss{selector}.
Components can all belong to the same type or each can have a different type.

% ----------------------------------------------------------------------
\beginImportant
The syntax for writing a \pspadtype{Record} type is
\begin{center}
{\tt Record(\subscriptIt{selector}{1}:\subscriptIt{type}{1}, \subscriptIt{selector}{2}:\subscriptIt{type}{2}, \ldots, \subscriptIt{selector}{N}:\subscriptIt{type}{N})}
\end{center}
You must be careful if a selector has the same name as a variable in the
workspace.
If this occurs, precede the selector name by a single
\index{quote}
quote.
\endImportant
% ----------------------------------------------------------------------

Record components are implicitly ordered.
All the components of a record can
be set at once by assigning the record a
bracketed \spadgloss{tuple} of values of the proper length
(for example, \spad{r : Record(a: Integer, b: String) := [1, "two"]}).
To access a component of a record \spad{r},
write the name \spad{r}, followed by a period, followed by a selector.

%
\begin{xtc}
\begin{xtccomment}
The object returned by this computation is a record with two components: a
\spad{quotient} part and a \spad{remainder} part.
\end{xtccomment}
\begin{spadsrc}
u := divide(5,2) 
\end{spadsrc}
\begin{TeXOutput}
\begin{fricasmath}{1}
\BRACKET{\SYMBOL{quotient}=2\COMMA \SYMBOL{remainder}=1}%
\end{fricasmath}
\end{TeXOutput}
\formatResultType{Record(quotient: Integer, remainder: Integer)}
\end{xtc}
%
\begin{xtc}
\begin{xtccomment}
This is the quotient part.
\end{xtccomment}
\begin{spadsrc}
u.quotient 
\end{spadsrc}
\begin{TeXOutput}
\begin{fricasmath}{2}
2%
\end{fricasmath}
\end{TeXOutput}
\formatResultType{PositiveInteger}
\end{xtc}
\begin{xtc}
\begin{xtccomment}
This is the remainder part.
\end{xtccomment}
\begin{spadsrc}
u.remainder 
\end{spadsrc}
\begin{TeXOutput}
\begin{fricasmath}{3}
1%
\end{fricasmath}
\end{TeXOutput}
\formatResultType{PositiveInteger}
\end{xtc}
%
\begin{xtc}
\begin{xtccomment}
You can use selector expressions on the left-hand side of an assignment
to change destructively the components of a record.
\end{xtccomment}
\begin{spadsrc}
u.quotient := 8978 
\end{spadsrc}
\begin{TeXOutput}
\begin{fricasmath}{4}
8978%
\end{fricasmath}
\end{TeXOutput}
\formatResultType{PositiveInteger}
\end{xtc}
\begin{xtc}
\begin{xtccomment}
The selected component \spad{quotient} has the value \spad{8978},
which is what is returned by the assignment.
Check that the value of \spad{u} was modified.
\end{xtccomment}
\begin{spadsrc}
u 
\end{spadsrc}
\begin{TeXOutput}
\begin{fricasmath}{5}
\BRACKET{\SYMBOL{quotient}=8978\COMMA \SYMBOL{remainder}=1}%
\end{fricasmath}
\end{TeXOutput}
\formatResultType{Record(quotient: Integer, remainder: Integer)}
\end{xtc}
\begin{xtc}
\begin{xtccomment}
Selectors are evaluated.
Thus you can use variables that evaluate to selectors instead of the
selectors themselves.
\end{xtccomment}
\begin{spadsrc}
s := 'quotient 
\end{spadsrc}
\begin{TeXOutput}
\begin{fricasmath}{6}
\SYMBOL{quotient}%
\end{fricasmath}
\end{TeXOutput}
\formatResultType{Variable(quotient)}
\end{xtc}
\begin{xtc}
\begin{xtccomment}
Be careful!
A selector could have the same name as a variable in the workspace.
If this occurs, precede the selector name by a single quote, as in
\spad{u.'quotient}.
\index{selector!quoting}
\end{xtccomment}
\begin{spadsrc}
divide(5,2).s 
\end{spadsrc}
\begin{TeXOutput}
\begin{fricasmath}{7}
2%
\end{fricasmath}
\end{TeXOutput}
\formatResultType{PositiveInteger}
\end{xtc}
\begin{xtc}
\begin{xtccomment}
Here we declare that the value of \spad{bd}
has two components: a string,
to be accessed via \spad{name}, and an integer,
to be accessed via \spad{birthdayMonth}.
\end{xtccomment}
\begin{spadsrc}
bd : Record(name : String, birthdayMonth : Integer) 
\end{spadsrc}
\end{xtc}
\begin{xtc}
\begin{xtccomment}
You must initially set the value of the entire \pspadtype{Record}
at once.
\end{xtccomment}
\begin{spadsrc}
bd := ["Judith", 3] 
\end{spadsrc}
\begin{TeXOutput}
\begin{fricasmath}{9}
\BRACKET{\SYMBOL{name}=\STRING{"Judith"}\COMMA \SYMBOL{birthdayMonth}=3}%
\end{fricasmath}
\end{TeXOutput}
\formatResultType{Record(name: String, birthdayMonth: Integer)}
\end{xtc}
\begin{xtc}
\begin{xtccomment}
Once set, you can change any of the individual components.
\end{xtccomment}
\begin{spadsrc}
bd.name := "Katie" 
\end{spadsrc}
\begin{TeXOutput}
\begin{fricasmath}{10}
\STRING{"Katie"}%
\end{fricasmath}
\end{TeXOutput}
\formatResultType{String}
\end{xtc}
\begin{xtc}
\begin{xtccomment}
Records may be nested and the selector names can be shared at
different levels.
\end{xtccomment}
\begin{spadsrc}
r : Record(a : Record(b: Integer, c: Integer), b: Integer) 
\end{spadsrc}
\end{xtc}
\begin{xtc}
\begin{xtccomment}
The record \spad{r} has a \spad{b} selector at two different levels.
Here is an initial value for \spad{r}.
\end{xtccomment}
\begin{spadsrc}
r := [[1,2],3] 
\end{spadsrc}
\begin{TeXOutput}
\begin{fricasmath}{12}
\BRACKET{\SYMBOL{a}=\BRACKET{\SYMBOL{b}=1\COMMA \SYMBOL{c}=2}\COMMA \SYMBOL{b%
}=3}%
\end{fricasmath}
\end{TeXOutput}
\formatResultType{Record(a: Record(b: Integer, c: Integer), b: Integer)}
\end{xtc}
\begin{xtc}
\begin{xtccomment}
This extracts the \spad{b} component from the \spad{a} component of \spad{r}.
\end{xtccomment}
\begin{spadsrc}
r.a.b 
\end{spadsrc}
\begin{TeXOutput}
\begin{fricasmath}{13}
1%
\end{fricasmath}
\end{TeXOutput}
\formatResultType{PositiveInteger}
\end{xtc}
\begin{xtc}
\begin{xtccomment}
This extracts the \spad{b} component from \spad{r}.
\end{xtccomment}
\begin{spadsrc}
r.b 
\end{spadsrc}
\begin{TeXOutput}
\begin{fricasmath}{14}
3%
\end{fricasmath}
\end{TeXOutput}
\formatResultType{PositiveInteger}
\end{xtc}
%
\begin{xtc}
\begin{xtccomment}
You can also use spaces or parentheses to refer to \pspadtype{Record}
components.
This is the same as \spad{r.a}.
\end{xtccomment}
\begin{spadsrc}
r(a) 
\end{spadsrc}
\begin{TeXOutput}
\begin{fricasmath}{15}
\BRACKET{\SYMBOL{b}=1\COMMA \SYMBOL{c}=2}%
\end{fricasmath}
\end{TeXOutput}
\formatResultType{Record(b: Integer, c: Integer)}
\end{xtc}
\begin{xtc}
\begin{xtccomment}
This is the same as \spad{r.b}.
\end{xtccomment}
\begin{spadsrc}
r b 
\end{spadsrc}
\begin{TeXOutput}
\begin{fricasmath}{16}
3%
\end{fricasmath}
\end{TeXOutput}
\formatResultType{PositiveInteger}
\end{xtc}
\begin{xtc}
\begin{xtccomment}
This is the same as \spad{r.b := 10}.
\end{xtccomment}
\begin{spadsrc}
r(b) := 10 
\end{spadsrc}
\begin{TeXOutput}
\begin{fricasmath}{17}
10%
\end{fricasmath}
\end{TeXOutput}
\formatResultType{PositiveInteger}
\end{xtc}
\begin{xtc}
\begin{xtccomment}
Look at \spad{r} to make sure it was modified.
\end{xtccomment}
\begin{spadsrc}
r 
\end{spadsrc}
\begin{TeXOutput}
\begin{fricasmath}{18}
\BRACKET{\SYMBOL{a}=\BRACKET{\SYMBOL{b}=1\COMMA \SYMBOL{c}=2}\COMMA \SYMBOL{b%
}=10}%
\end{fricasmath}
\end{TeXOutput}
\formatResultType{Record(a: Record(b: Integer, c: Integer), b: Integer)}
\end{xtc}

% *********************************************************************
\head{section}{Unions}{ugTypesUnions}
% *********************************************************************
%
Type \pspadtype{Union} is used for objects that
can be of any of a specific finite set of types.
\index{Union@\protect\nonLibAxiomType{Union}}
Two versions of unions are available,
one with selectors (like records) and one without.
\index{union}

% *********************************************************************
\head{subsection}{Unions Without Selectors}{ugTypesUnionsWOSel}
% *********************************************************************

The declaration \spad{x : Union(Integer, String, Float)}
states that \spad{x} can have values that are integers,
strings or ``big'' floats.
If, for example, the \pspadtype{Union} object is an integer, the object is
said to belong to the \spadtype{Integer} {\it branch}
of the \pspadtype{Union}.\footnote{
Note that we are being a bit careless with the language here.
Technically, the type of \spad{x} is always
\pspadtype{Union(Integer, String, Float)}.
If it belongs to the \spadtype{Integer} branch, \spad{x}
may be converted to an object of type \spadtype{Integer}.}

% ----------------------------------------------------------------------
\beginImportant
The syntax for writing a \pspadtype{Union} type without selectors is
\begin{center}
{\tt Union(\subscriptIt{type}{1}, \subscriptIt{type}{2}, \ldots, \subscriptIt{type}{N})}
\end{center}
The types in a union without selectors must be distinct.
\endImportant
% ----------------------------------------------------------------------

It is possible to create unions like
\pspadtype{Union(Integer, PositiveInteger)} but they are
difficult to work with because of the overlap in the branch
types.
See below for the rules \Language{} uses for converting something
into a union object.

The \spad{case} infix
\spadkey{case}
operator returns a \spadtype{Boolean}
and can be used to determine the branch in which an object lies.

\begin{xtc}
\begin{xtccomment}
This function displays a message stating in which
branch of the \pspadtype{Union} the object (defined as \spad{x}
above) lies.
\end{xtccomment}
\begin{spadsrc}
sayBranch(x : Union(Integer,String,Float)) : Void  ==
  output
    x case Integer => "Integer branch"
    x case String  => "String branch"
    "Float branch"
\end{spadsrc}
\begin{MessageOutput}
   Function declaration sayBranch : Union(Integer,String,Float) -> Void
      has been added to workspace.
\end{MessageOutput}
\end{xtc}
%
\begin{xtc}
\begin{xtccomment}
This tries \userfun{sayBranch} with an integer.
\end{xtccomment}
\begin{spadsrc}
sayBranch 1 
\end{spadsrc}
\begin{MessageOutput}
   Compiling function sayBranch with type Union(Integer,String,Float)
       -> Void 
\end{MessageOutput}
\end{xtc}
\begin{xtc}
\begin{xtccomment}
This tries \userfun{sayBranch} with a string.
\end{xtccomment}
\begin{spadsrc}
sayBranch "hello" 
\end{spadsrc}
\end{xtc}
\begin{xtc}
\begin{xtccomment}
This tries \userfun{sayBranch} with a floating-point number.
\end{xtccomment}
\begin{spadsrc}
sayBranch 2.718281828 
\end{spadsrc}
\end{xtc}
%

There are two things of interest about this particular
example to which we would like to draw your attention.
\begin{enumerate}
%
\item \Language{} normally converts a result to the target value
before passing it to the function.
If we left the declaration information out of this function definition
then the \spad{sayBranch} call would have been attempted with an
\spadtype{Integer} rather than a \pspadtype{Union}, and an error would have
resulted.
%
\item The types in a \pspadtype{Union} are searched in the order given.
So if the type were given as

\noindent
{\small\spad{sayBranch(x: Union(String,Integer,Float,Any)): Void}}

\noindent
then the result would have been ``String branch'' because there
is a conversion from \spadtype{Integer} to \spadtype{String}.
\end{enumerate}

Sometimes \pspadtype{Union} types can have extremely
long names.
\Language{} therefore abbreviates the names of unions by printing
the type of the branch first within the \pspadtype{Union} and then
eliding the remaining types with an ellipsis (\spadSyntax{...}).

\begin{xtc}
\begin{xtccomment}
Here the \spadtype{Integer} branch is displayed first.
Use \spadSyntax{::} to create a \pspadtype{Union} object from an object.
\end{xtccomment}
\begin{spadsrc}
78 :: Union(Integer,String)
\end{spadsrc}
\begin{TeXOutput}
\begin{fricasmath}{5}
78%
\end{fricasmath}
\end{TeXOutput}
\formatResultType{Union(Integer, ...)}
\end{xtc}
\begin{xtc}
\begin{xtccomment}
Here the \spadtype{String} branch is displayed first.
\end{xtccomment}
\begin{spadsrc}
s := "string" :: Union(Integer,String) 
\end{spadsrc}
\begin{TeXOutput}
\begin{fricasmath}{6}
\STRING{"string"}%
\end{fricasmath}
\end{TeXOutput}
\formatResultType{Union(String, ...)}
\end{xtc}
\begin{xtc}
\begin{xtccomment}
Use \spad{typeOf} to see the full and actual \pspadtype{Union} type.
\spadkey{typeOf}
\end{xtccomment}
\begin{spadsrc}
typeOf s
\end{spadsrc}
\begin{TeXOutput}
\begin{fricasmath}{7}
\STRING{Union(Integer,String)}%
\end{fricasmath}
\end{TeXOutput}
\formatResultType{Type}
\end{xtc}
\begin{xtc}
\begin{xtccomment}
A common operation that returns a union is \spadfunFrom{exquo}{Integer}
which returns the ``exact quotient'' if the quotient is exact,...
\end{xtccomment}
\begin{spadsrc}
three := exquo(6,2) 
\end{spadsrc}
\begin{TeXOutput}
\begin{fricasmath}{8}
3%
\end{fricasmath}
\end{TeXOutput}
\formatResultType{Union(Integer, ...)}
\end{xtc}
\begin{xtc}
\begin{xtccomment}
and \spad{"failed"} if the quotient is not exact.
\end{xtccomment}
\begin{spadsrc}
exquo(5,2)
\end{spadsrc}
\begin{TeXOutput}
\begin{fricasmath}{9}
\STRING{"failed"}%
\end{fricasmath}
\end{TeXOutput}
\formatResultType{Union("failed", ...)}
\end{xtc}
\begin{xtc}
\begin{xtccomment}
A union with a \spad{"failed"} is frequently used to indicate the failure
or lack of applicability of an object.
As another example, assign an integer a variable \spad{r} declared to be a
rational number.
\end{xtccomment}
\begin{spadsrc}
r: FRAC INT := 3 
\end{spadsrc}
\begin{TeXOutput}
\begin{fricasmath}{10}
3%
\end{fricasmath}
\end{TeXOutput}
\formatResultType{Fraction(Integer)}
\end{xtc}
\begin{xtc}
\begin{xtccomment}
The operation \spadfunFrom{retractIfCan}{Fraction} tries to retract the
fraction to the underlying domain \spadtype{Integer}.
It produces a union object.
Here it succeeds.
\end{xtccomment}
\begin{spadsrc}
retractIfCan(r) 
\end{spadsrc}
\begin{TeXOutput}
\begin{fricasmath}{11}
3%
\end{fricasmath}
\end{TeXOutput}
\formatResultType{Union(Integer, ...)}
\end{xtc}
\begin{xtc}
\begin{xtccomment}
Assign it a rational number.
\end{xtccomment}
\begin{spadsrc}
r := 3/2 
\end{spadsrc}
\begin{TeXOutput}
\begin{fricasmath}{12}
\frac{3}{2}%
\end{fricasmath}
\end{TeXOutput}
\formatResultType{Fraction(Integer)}
\end{xtc}
\begin{xtc}
\begin{xtccomment}
Here the retraction fails.
\end{xtccomment}
\begin{spadsrc}
retractIfCan(r) 
\end{spadsrc}
\begin{TeXOutput}
\begin{fricasmath}{13}
\STRING{"failed"}%
\end{fricasmath}
\end{TeXOutput}
\formatResultType{Union("failed", ...)}
\end{xtc}

% *********************************************************************
\head{subsection}{Unions With Selectors}{ugTypesUnionsWSel}
% *********************************************************************

Like records (\spadref{ugTypesRecords}),
you can write \pspadtype{Union} types
\index{selector!union}
with selectors.
\index{union!selector}

% ----------------------------------------------------------------------
\beginImportant
The syntax for writing a \pspadtype{Union} type with selectors is
\begin{center}
{\tt Union(\subscriptIt{selector}{1}:\subscriptIt{type}{1}, \subscriptIt{selector}{2}:\subscriptIt{type}{2}, \ldots, \subscriptIt{selector}{N}:\subscriptIt{type}{N})}
\end{center}
You must be careful if a selector has the same name as a variable in the
workspace.
If this occurs, precede the selector name by a single
\index{quote}
quote.
\index{selector!quoting}
It is an error to use a selector that does not correspond to the branch of
the \pspadtype{Union} in which the element actually lies.
\endImportant
% ----------------------------------------------------------------------

Be sure to understand the difference between records and unions
with selectors.
\index{union!difference from record}
Records can have more than one component and the selectors are
used to refer to the components.
\index{record!difference from union}
Unions always have one component but the type of that one
component can vary.
An object of type \pspadtype{Record(a: Integer, b: Float, c: String)}
contains an integer {\it and} a float  {\it and} a
string.
An object of type \pspadtype{Union(a: Integer, b: Float, c: String)}
contains an integer {\it or} a float  {\it or} a
string.

Here is a version of the \userfun{sayBranch} function (cf.
\spadref{ugTypesUnionsWOSel}) that works with a union with selectors.
It displays a message stating in which branch of the \pspadtype{Union} the
object lies.
\begin{verbatim}
sayBranch(x:Union(i:Integer,s:String,f:Float)):Void==
  output
    x case i => "Integer branch"
    x case s  => "String branch"
    "Float branch"
\end{verbatim}
Note that \spad{case} uses the selector name as its right-hand argument.
\spadkey{case}
If you accidentally use the branch type on the right-hand side of
\spad{case}, \spad{false} will be returned.

\begin{xtc}
\begin{xtccomment}
Declare variable \spad{u} to have a union type with selectors.
\end{xtccomment}
\begin{spadsrc}
u : Union(i : Integer, s : String) 
\end{spadsrc}
\end{xtc}
\begin{xtc}
\begin{xtccomment}
Give an initial value to \spad{u}.
\end{xtccomment}
\begin{spadsrc}
u := "good morning" 
\end{spadsrc}
\begin{TeXOutput}
\begin{fricasmath}{2}
\STRING{"good\ morning"}%
\end{fricasmath}
\end{TeXOutput}
\formatResultType{Union(s: String, ...)}
\end{xtc}
\begin{xtc}
\begin{xtccomment}
Use \spad{case} to determine in which
branch of a \pspadtype{Union} an object lies.
\end{xtccomment}
\begin{spadsrc}
u case i 
\end{spadsrc}
\begin{TeXOutput}
\begin{fricasmath}{3}
\STRING{false}%
\end{fricasmath}
\end{TeXOutput}
\formatResultType{Boolean}
\end{xtc}
\begin{xtc}
\begin{xtccomment}
\end{xtccomment}
\begin{spadsrc}
u case s 
\end{spadsrc}
\begin{TeXOutput}
\begin{fricasmath}{4}
\STRING{true}%
\end{fricasmath}
\end{TeXOutput}
\formatResultType{Boolean}
\end{xtc}
\begin{xtc}
\begin{xtccomment}
To access the element in a particular branch, use the selector.
\end{xtccomment}
\begin{spadsrc}
u.s 
\end{spadsrc}
\begin{TeXOutput}
\begin{fricasmath}{5}
\STRING{"good\ morning"}%
\end{fricasmath}
\end{TeXOutput}
\formatResultType{String}
\end{xtc}

% *********************************************************************
\head{section}{The ``Any'' Domain}{ugTypesAnyNone}
% *********************************************************************

With the exception of objects of type \pspadtype{Record}, all \Language{}
data structures are homogenous, that is, they hold objects all of the same
type.
\exptypeindex{Any}
If you need to get around this, you can use type \spadtype{Any}.
Using \spadtype{Any}, for example, you can create lists whose
elements are integers, rational numbers, strings, and even other lists.

\begin{xtc}
\begin{xtccomment}
Declare \spad{u} to have type \spadtype{Any}.
\end{xtccomment}
\begin{spadsrc}
u: Any
\end{spadsrc}
\end{xtc}
\begin{xtc}
\begin{xtccomment}
Assign a list of mixed type values to \spad{u}
\end{xtccomment}
\begin{spadsrc}
u := [1, 7.2, 3/2, x^2, "wally"]
\end{spadsrc}
\begin{TeXOutput}
\begin{fricasmath}{2}
\BRACKET{1\COMMA \STRING{7.2}\COMMA \frac{3}{2}\COMMA \SUPER{\SYMBOL{x}}{2}%
\COMMA \STRING{"wally"}}%
\end{fricasmath}
\end{TeXOutput}
\formatResultType{List(Any)}
\end{xtc}
\begin{xtc}
\begin{xtccomment}
When we ask for the elements, \Language{} displays these types.
\end{xtccomment}
\begin{spadsrc}
u.1 
\end{spadsrc}
\begin{TeXOutput}
\begin{fricasmath}{3}
1%
\end{fricasmath}
\end{TeXOutput}
\formatResultType{PositiveInteger}
\end{xtc}
\begin{xtc}
\begin{xtccomment}
Actually, these objects belong to \spadtype{Any} but \Language{}
automatically converts them to their natural types for you.
\end{xtccomment}
\begin{spadsrc}
u.3 
\end{spadsrc}
\begin{TeXOutput}
\begin{fricasmath}{4}
\frac{3}{2}%
\end{fricasmath}
\end{TeXOutput}
\formatResultType{Fraction(Integer)}
\end{xtc}
\begin{xtc}
\begin{xtccomment}
Since type \spadtype{Any} can be anything,
it can only belong to type \spadtype{Type}.
Therefore it cannot be used in algebraic domains.
\end{xtccomment}
\begin{spadsrc}
v : Matrix(Any)
\end{spadsrc}
\begin{MessageOutput}
   Matrix(Any) is not a valid type.
\end{MessageOutput}
\end{xtc}

Perhaps you are wondering how \Language{} internally represents
objects of type \spadtype{Any}.
An object of type \spadtype{Any} consists not only a data part
representing its normal value, but also a type part (a {\it badge}) giving
\index{badge}
its type.
For example, the value \spad{1} of type \spadtype{PositiveInteger} as an
object of type \spadtype{Any} internally looks like
\spad{[1,PositiveInteger()]}.

%When should you use \spadtype{Any} instead of a \pspadtype{Union} type?
%Can you plan ahead?
%For a \pspadtype{Union}, you must know in advance exactly which types you
%are
%\index{union!vs. Any@{vs. \protect\nonLibAxiomType{Any}}}
%going to allow.
%For \spadtype{Any}, anything that comes along can be accommodated.

% *********************************************************************
\head{section}{Conversion}{ugTypesConvert}
% *********************************************************************
%
\beginImportant
\spadglossSee{Conversion}{conversion}
is the process of changing an object of one type
into an object of another type.
The syntax for conversion is:
\begin{center}
{\it object} {\tt ::} {\it newType}
\end{center}
\endImportant

\begin{xtc}
\begin{xtccomment}
By default, \spad{3} has the type \spadtype{PositiveInteger}.
\end{xtccomment}
\begin{spadsrc}
3
\end{spadsrc}
\begin{TeXOutput}
\begin{fricasmath}{1}
3%
\end{fricasmath}
\end{TeXOutput}
\formatResultType{PositiveInteger}
\end{xtc}
\begin{xtc}
\begin{xtccomment}
We can change this into an object of type \spadtype{Fraction Integer}
by using \spadSyntax{::}.
\end{xtccomment}
\begin{spadsrc}
3 :: Fraction Integer
\end{spadsrc}
\begin{TeXOutput}
\begin{fricasmath}{2}
3%
\end{fricasmath}
\end{TeXOutput}
\formatResultType{Fraction(Integer)}
\end{xtc}

A \spadgloss{coercion} is a special kind of conversion that \Language{} is
allowed to do automatically when you enter an expression.
Coercions are usually somewhat safer than more general conversions.
The \Language{} library contains operations called
\spadfun{coerce} and \spadfun{convert}.
Only the \spadfun{coerce} operations can be used by the
interpreter to change an object into an object of another type unless
you explicitly use a \spadSyntax{::}.

By now you will be quite familiar with what types and modes look like.
It is useful to think of a type or mode as a pattern
for what you want the result to be.
\begin{xtc}
\begin{xtccomment}
Let's start with a square matrix of polynomials with complex rational number
coefficients.
\exptypeindex{SquareMatrix}
\end{xtccomment}
\begin{spadsrc}
m : SquareMatrix(2,POLY COMPLEX FRAC INT) 
\end{spadsrc}
\end{xtc}
\begin{xtc}
\begin{xtccomment}
\end{xtccomment}
\begin{spadsrc}
m := matrix [[x-3/4*%i,z*y^2+1/2],[3/7*%i*y^4 - x,12-%i*9/5]] 
\end{spadsrc}
\begin{TeXOutput}
\begin{fricasmath}{4}
\begin{MATRIX}{2}\SYMBOL{x}-{\frac{3}{4}\TIMES \ImaginaryI }&\SUPER{\SYMBOL{y%
}}{2}\TIMES \SYMBOL{z}+\frac{1}{2}\\\frac{3}{7}\TIMES \ImaginaryI \TIMES %
\SUPER{\SYMBOL{y}}{4}-{\SYMBOL{x}}&12-{\frac{9}{5}\TIMES \ImaginaryI }%
\end{MATRIX}%
\end{fricasmath}
\end{TeXOutput}
\formatResultType{SquareMatrix(2, Polynomial(Complex(Fraction(Integer))))}
\end{xtc}
\begin{xtc}
\begin{xtccomment}
We first want to interchange the \spadtype{Complex} and
\spadtype{Fraction} layers.
We do the conversion by doing the interchange in the type expression.
\end{xtccomment}
\begin{spadsrc}
m1 := m :: SquareMatrix(2,POLY FRAC COMPLEX INT) 
\end{spadsrc}
\begin{TeXOutput}
\begin{fricasmath}{5}
\begin{MATRIX}{2}\SYMBOL{x}-{\frac{3\TIMES \ImaginaryI }{4}}&\SUPER{\SYMBOL{y%
}}{2}\TIMES \SYMBOL{z}+\frac{1}{2}\\\frac{3\TIMES \ImaginaryI }{7}\TIMES %
\SUPER{\SYMBOL{y}}{4}-{\SYMBOL{x}}&\frac{60-{9\TIMES \ImaginaryI }}{5}%
\end{MATRIX}%
\end{fricasmath}
\end{TeXOutput}
\formatResultType{SquareMatrix(2, Polynomial(Fraction(Complex(Integer))))}
\end{xtc}
\begin{xtc}
\begin{xtccomment}
Interchange the \spadtype{Polynomial} and the
\spadtype{Fraction} levels.
\end{xtccomment}
\begin{spadsrc}
m2 := m1 :: SquareMatrix(2,FRAC POLY COMPLEX INT) 
\end{spadsrc}
\begin{TeXOutput}
\begin{fricasmath}{6}
\begin{MATRIX}{2}\frac{4\TIMES \SYMBOL{x}-{3\TIMES \ImaginaryI }}{4}&\frac{2%
\TIMES \SUPER{\SYMBOL{y}}{2}\TIMES \SYMBOL{z}+1}{2}\\\frac{3\TIMES %
\ImaginaryI \TIMES \SUPER{\SYMBOL{y}}{4}-{7\TIMES \SYMBOL{x}}}{7}&\frac{60-{9%
\TIMES \ImaginaryI }}{5}\end{MATRIX}%
\end{fricasmath}
\end{TeXOutput}
\formatResultType{SquareMatrix(2, Fraction(Polynomial(Complex(Integer))))}
\end{xtc}
\begin{xtc}
\begin{xtccomment}
Interchange the \spadtype{Polynomial} and the
\spadtype{Complex} levels.
\end{xtccomment}
\begin{spadsrc}
m3 := m2 :: SquareMatrix(2,FRAC COMPLEX POLY INT) 
\end{spadsrc}
\begin{TeXOutput}
\begin{fricasmath}{7}
\begin{MATRIX}{2}\frac{4\TIMES \SYMBOL{x}-{3\TIMES \ImaginaryI }}{4}&\frac{2%
\TIMES \SUPER{\SYMBOL{y}}{2}\TIMES \SYMBOL{z}+1}{2}\\\frac{-{7\TIMES \SYMBOL{%
x}}+3\TIMES \SUPER{\SYMBOL{y}}{4}\TIMES \ImaginaryI }{7}&\frac{60-{9\TIMES %
\ImaginaryI }}{5}\end{MATRIX}%
\end{fricasmath}
\end{TeXOutput}
\formatResultType{SquareMatrix(2, Fraction(Complex(Polynomial(Integer))))}
\end{xtc}

All the entries have changed types, although in comparing the
last two results only the entry in the lower left corner looks different.
We did all the intermediate steps to show you what \Language{} can do.

\begin{xtc}
\begin{xtccomment}
In fact, we could have combined all these into one conversion.
\end{xtccomment}
\begin{spadsrc}
m :: SquareMatrix(2,FRAC COMPLEX POLY INT) 
\end{spadsrc}
\begin{TeXOutput}
\begin{fricasmath}{8}
\begin{MATRIX}{2}\frac{4\TIMES \SYMBOL{x}-{3\TIMES \ImaginaryI }}{4}&\frac{2%
\TIMES \SUPER{\SYMBOL{y}}{2}\TIMES \SYMBOL{z}+1}{2}\\\frac{-{7\TIMES \SYMBOL{%
x}}+3\TIMES \SUPER{\SYMBOL{y}}{4}\TIMES \ImaginaryI }{7}&\frac{60-{9\TIMES %
\ImaginaryI }}{5}\end{MATRIX}%
\end{fricasmath}
\end{TeXOutput}
\formatResultType{SquareMatrix(2, Fraction(Complex(Polynomial(Integer))))}
\end{xtc}

There are times when \Language{} is not be able to do the conversion
in one step.
You may need to break up the transformation into several conversions
in order to get an object of the desired type.

We cannot move either \spadtype{Fraction} or \spadtype{Complex}
above (or to the left of, depending on how you look at it)
\spadtype{SquareMatrix} because each of these levels requires that its
argument type have commutative multiplication, whereas
\spadtype{SquareMatrix} does not.\footnote{\spadtype{Fraction} requires
that its argument belong to the category \spadtype{IntegralDomain} and
\index{category}
\spadtype{Complex} requires that its argument belong to
\spadtype{CommutativeRing}. See
\spadref{ugTypesBasic}
for a brief discussion of categories.}
The \spadtype{Integer} level did not move anywhere
because it does not allow any arguments.
We also did not move the \spadtype{SquareMatrix} part anywhere, but
we could have.
\begin{xtc}
\begin{xtccomment}
Recall that \spad{m} looks like this.
\end{xtccomment}
\begin{spadsrc}
m 
\end{spadsrc}
\begin{TeXOutput}
\begin{fricasmath}{9}
\begin{MATRIX}{2}\SYMBOL{x}-{\frac{3}{4}\TIMES \ImaginaryI }&\SUPER{\SYMBOL{y%
}}{2}\TIMES \SYMBOL{z}+\frac{1}{2}\\\frac{3}{7}\TIMES \ImaginaryI \TIMES %
\SUPER{\SYMBOL{y}}{4}-{\SYMBOL{x}}&12-{\frac{9}{5}\TIMES \ImaginaryI }%
\end{MATRIX}%
\end{fricasmath}
\end{TeXOutput}
\formatResultType{SquareMatrix(2, Polynomial(Complex(Fraction(Integer))))}
\end{xtc}
\begin{xtc}
\begin{xtccomment}
If we want a polynomial with matrix coefficients rather than a matrix
with polynomial entries, we can just do the conversion.
\end{xtccomment}
\begin{spadsrc}
m :: POLY SquareMatrix(2,COMPLEX FRAC INT) 
\end{spadsrc}
\begin{TeXOutput}
\begin{fricasmath}{10}
\begin{MATRIX}{2}0&1\\0&0\end{MATRIX}\TIMES \SUPER{\SYMBOL{y}}{2}\TIMES %
\SYMBOL{z}+\begin{MATRIX}{2}0&0\\\frac{3}{7}\TIMES \ImaginaryI &0\end{MATRIX}%
\TIMES \SUPER{\SYMBOL{y}}{4}+\begin{MATRIX}{2}1&0\\-{1}&0\end{MATRIX}\TIMES %
\SYMBOL{x}+\begin{MATRIX}{2}-{\frac{3}{4}\TIMES \ImaginaryI }&\frac{1}{2}\\0&%
12-{\frac{9}{5}\TIMES \ImaginaryI }\end{MATRIX}%
\end{fricasmath}
\end{TeXOutput}
\formatResultType{Polynomial(SquareMatrix(2, Complex(Fraction(Integer))))}
\end{xtc}
\begin{xtc}
\begin{xtccomment}
We have not yet used modes for any conversions.
Modes are a great shorthand for indicating the type of the
object you want.
Instead of using the long type expression in the
last example, we could have simply said this.
\end{xtccomment}
\begin{spadsrc}
m :: POLY ? 
\end{spadsrc}
\begin{TeXOutput}
\begin{fricasmath}{11}
\begin{MATRIX}{2}0&1\\0&0\end{MATRIX}\TIMES \SUPER{\SYMBOL{y}}{2}\TIMES %
\SYMBOL{z}+\begin{MATRIX}{2}0&0\\\frac{3}{7}\TIMES \ImaginaryI &0\end{MATRIX}%
\TIMES \SUPER{\SYMBOL{y}}{4}+\begin{MATRIX}{2}1&0\\-{1}&0\end{MATRIX}\TIMES %
\SYMBOL{x}+\begin{MATRIX}{2}-{\frac{3}{4}\TIMES \ImaginaryI }&\frac{1}{2}\\0&%
12-{\frac{9}{5}\TIMES \ImaginaryI }\end{MATRIX}%
\end{fricasmath}
\end{TeXOutput}
\formatResultType{Polynomial(SquareMatrix(2, Complex(Fraction(Integer))))}
\end{xtc}
\begin{xtc}
\begin{xtccomment}
We can also indicate more structure if we want the entries
of the matrices to be fractions.
\end{xtccomment}
\begin{spadsrc}
m :: POLY SquareMatrix(2,FRAC ?) 
\end{spadsrc}
\begin{TeXOutput}
\begin{fricasmath}{12}
\begin{MATRIX}{2}0&1\\0&0\end{MATRIX}\TIMES \SUPER{\SYMBOL{y}}{2}\TIMES %
\SYMBOL{z}+\begin{MATRIX}{2}0&0\\\frac{3\TIMES \ImaginaryI }{7}&0\end{MATRIX}%
\TIMES \SUPER{\SYMBOL{y}}{4}+\begin{MATRIX}{2}1&0\\-{1}&0\end{MATRIX}\TIMES %
\SYMBOL{x}+\begin{MATRIX}{2}-{\frac{3\TIMES \ImaginaryI }{4}}&\frac{1}{2}\\0&%
\frac{60-{9\TIMES \ImaginaryI }}{5}\end{MATRIX}%
\end{fricasmath}
\end{TeXOutput}
\formatResultType{Polynomial(SquareMatrix(2, Fraction(Complex(Integer))))}
\end{xtc}

% *********************************************************************
\head{section}{Subdomains Again}{ugTypesSubdomains}
% *********************************************************************

A \spadgloss{subdomain} \spad{S} of a domain \spad{D} is a domain
consisting of
\begin{enumerate}
\item those elements of \spad{D} that satisfy some
\spadgloss{predicate} (that is, a test that returns \spad{true} or
\spad{false}) and
\item a subset of the operations of \spad{D}.
\end{enumerate}
Every domain is a subdomain of itself, trivially satisfying the
membership test: \spad{true}.

Currently, there are only two system-defined subdomains in \Language{} that receive
substantial use.
\spadtype{PositiveInteger} and
\spadtype{NonNegativeInteger} are subdomains of \spadtype{Integer}.
An element \spad{x} of \spadtype{NonNegativeInteger} is an integer
that is greater than or equal to zero, that is, satisfies
\spad{x >= 0.}
An element \spad{x} of \spadtype{PositiveInteger} is a nonnegative integer
that is, in fact, greater than zero, that is, satisfies \spad{x > 0.}
Not all operations from \spadtype{Integer} are available for these
subdomains.
For example, negation and subtraction are not provided since the subdomains
are not closed under those operations.
When you use an integer in an expression, \Language{} assigns to it the
type that is the most specific subdomain whose predicate is satisfied.
\begin{xtc}
\begin{xtccomment}
This is a positive integer.
\end{xtccomment}
\begin{spadsrc}
5
\end{spadsrc}
\begin{TeXOutput}
\begin{fricasmath}{1}
5%
\end{fricasmath}
\end{TeXOutput}
\formatResultType{PositiveInteger}
\end{xtc}
\begin{xtc}
\begin{xtccomment}
This is a nonnegative integer.
\end{xtccomment}
\begin{spadsrc}
0
\end{spadsrc}
\begin{TeXOutput}
\begin{fricasmath}{2}
0%
\end{fricasmath}
\end{TeXOutput}
\formatResultType{NonNegativeInteger}
\end{xtc}
\begin{xtc}
\begin{xtccomment}
This is neither of the above.
\end{xtccomment}
\begin{spadsrc}
-5
\end{spadsrc}
\begin{TeXOutput}
\begin{fricasmath}{3}
-{5}%
\end{fricasmath}
\end{TeXOutput}
\formatResultType{Integer}
\end{xtc}
\begin{xtc}
\begin{xtccomment}
Furthermore, unless you are assigning an integer to a declared variable
or using a conversion, any integer result has as type the most
specific subdomain.
\end{xtccomment}
\begin{spadsrc}
(-2) - (-3)
\end{spadsrc}
\begin{TeXOutput}
\begin{fricasmath}{4}
1%
\end{fricasmath}
\end{TeXOutput}
\formatResultType{PositiveInteger}
\end{xtc}
\begin{xtc}
\begin{xtccomment}
\end{xtccomment}
\begin{spadsrc}
0 :: Integer
\end{spadsrc}
\begin{TeXOutput}
\begin{fricasmath}{5}
0%
\end{fricasmath}
\end{TeXOutput}
\formatResultType{Integer}
\end{xtc}
\begin{xtc}
\begin{xtccomment}
\end{xtccomment}
\begin{spadsrc}
x : NonNegativeInteger := 5
\end{spadsrc}
\begin{TeXOutput}
\begin{fricasmath}{6}
5%
\end{fricasmath}
\end{TeXOutput}
\formatResultType{NonNegativeInteger}
\end{xtc}

When necessary, \Language{} converts an integer object into one belonging
to a less specific subdomain.
For example, in \spad{3-2}, the arguments to \spadopFrom{-}{Integer} are both
elements of \spadtype{PositiveInteger}, but this type does not provide
a subtraction operation.
Neither does \spadtype{NonNegativeInteger}, so \spad{3} and \spad{2}
are viewed as elements of \spadtype{Integer}, where their difference
can be calculated.
The result is \spad{1}, which \Language{} then automatically assigns
the type \spadtype{PositiveInteger}.

\begin{xtc}
\begin{xtccomment}
Certain operations are very sensitive to the subdomains to which their
arguments belong.
This is an element of \spadtype{PositiveInteger}.
\end{xtccomment}
\begin{spadsrc}
2 ^ 2
\end{spadsrc}
\begin{TeXOutput}
\begin{fricasmath}{7}
4%
\end{fricasmath}
\end{TeXOutput}
\formatResultType{PositiveInteger}
\end{xtc}
\begin{xtc}
\begin{xtccomment}
This is an element of \spadtype{Fraction Integer}.
\end{xtccomment}
\begin{spadsrc}
2 ^ (-2)
\end{spadsrc}
\begin{TeXOutput}
\begin{fricasmath}{8}
\frac{1}{4}%
\end{fricasmath}
\end{TeXOutput}
\formatResultType{Fraction(Integer)}
\end{xtc}
\begin{xtc}
\begin{xtccomment}
It makes sense then that this
is a list of elements of \spadtype{PositiveInteger}.
\end{xtccomment}
\begin{spadsrc}
[10^i for i in 2..5]
\end{spadsrc}
\begin{TeXOutput}
\begin{fricasmath}{9}
\BRACKET{100\COMMA 1000\COMMA 10000\COMMA 100000}%
\end{fricasmath}
\end{TeXOutput}
\formatResultType{List(PositiveInteger)}
\end{xtc}
What should the type of \spad{[10^(i-1) for i in 2..5]} be?
On one hand, \spad{i-1} is always an integer greater than zero
as \spad{i} ranges from \spad{2} to \spad{5} and so \spad{10^i}
is also always a positive integer.
On the other, \spad{i-1} is a very simple function of \spad{i}.
\Language{} does not try to analyze every such function over the
index's range of values to determine whether it is always positive
or nowhere negative.
For an arbitrary \Language{} function, this analysis is not possible.

\begin{xtc}
\begin{xtccomment}
So, to be consistent no such analysis is done and we get this.
\end{xtccomment}
\begin{spadsrc}
[10^(i-1) for i in 2..5]
\end{spadsrc}
\begin{TeXOutput}
\begin{fricasmath}{10}
\BRACKET{10\COMMA 100\COMMA 1000\COMMA 10000}%
\end{fricasmath}
\end{TeXOutput}
\formatResultType{List(Fraction(Integer))}
\end{xtc}
\begin{xtc}
\begin{xtccomment}
To get a list of elements of \spadtype{PositiveInteger} instead, you
have two choices.
You can use a conversion.
\end{xtccomment}
\begin{spadsrc}
[10^((i-1) :: PI) for i in 2..5]
\end{spadsrc}
\begin{MessageOutput}
   Compiling function G755 with type Integer -> Boolean 
\end{MessageOutput}
\begin{MessageOutput}
   Compiling function G757 with type NonNegativeInteger -> Boolean 
\end{MessageOutput}
\begin{TeXOutput}
\begin{fricasmath}{11}
\BRACKET{10\COMMA 100\COMMA 1000\COMMA 10000}%
\end{fricasmath}
\end{TeXOutput}
\formatResultType{List(PositiveInteger)}
\end{xtc}
\begin{xtc}
\begin{xtccomment}
Or you can use \spad{pretend}.
\spadkey{pretend}
\end{xtccomment}
\begin{spadsrc}
[10^((i-1) pretend PI) for i in 2..5]
\end{spadsrc}
\begin{TeXOutput}
\begin{fricasmath}{12}
\BRACKET{10\COMMA 100\COMMA 1000\COMMA 10000}%
\end{fricasmath}
\end{TeXOutput}
\formatResultType{List(PositiveInteger)}
\end{xtc}

The operation \spad{pretend} is used to defeat the \Language{}
type system.
The expression \spad{object pretend D} means ``make a new object
(without copying) of type \spad{D} from \spad{object}.''
If \spad{object} were an integer and you told \Language{}
to pretend it was a list, you would probably see a message about a
fatal error being caught and memory possibly being damaged.
Lists do not have the same internal representation as integers!

You use \spad{pretend} at your peril.
\index{peril}

\begin{xtc}
\begin{xtccomment}
Use \spad{pretend} with great care!
\Language{} trusts you that the value is of the specified type.
\end{xtccomment}
\begin{spadsrc}
(2/3) pretend Complex Integer
\end{spadsrc}
\begin{TeXOutput}
\begin{fricasmath}{13}
2+3\TIMES \ImaginaryI %
\end{fricasmath}
\end{TeXOutput}
\formatResultType{Complex(Integer)}
\end{xtc}

% *********************************************************************
\head{section}{Package Calling and Target Types}{ugTypesPkgCall}
% *********************************************************************

\Language{} works hard to figure out what you mean by an
expression without your having to qualify it with type
information.
Nevertheless, there are times when you need to help it along by
providing hints (or even orders!) to get \Language{} to do what
you want.

We saw in \spadref{ugTypesDeclare} that declarations using types
and modes control the type of the results produced.
For example, we can either produce a complex object with
polynomial real and imaginary parts or a polynomial with complex
integer coefficients, depending on the declaration.

\spadglossSee{Package calling}{package call} is how you tell
\Language{} to use a particular function from a particular part of
the library.

\begin{xtc}
\begin{xtccomment}
Use the \spadopFrom{/}{Fraction} from \spadtype{Fraction Integer}
to create a fraction of two integers.
\end{xtccomment}
\begin{spadsrc}
2/3
\end{spadsrc}
\begin{TeXOutput}
\begin{fricasmath}{1}
\frac{2}{3}%
\end{fricasmath}
\end{TeXOutput}
\formatResultType{Fraction(Integer)}
\end{xtc}
\begin{xtc}
\begin{xtccomment}
If we wanted a floating point number, we can say ``use the
\spadopFrom{/}{Float} in \spadtype{Float}.''
\end{xtccomment}
\begin{spadsrc}
(2/3)$Float
\end{spadsrc}
\begin{TeXOutput}
\begin{fricasmath}{2}
\STRING{0.66666666666666666667}%
\end{fricasmath}
\end{TeXOutput}
\formatResultType{Float}
\end{xtc}
\begin{xtc}
\begin{xtccomment}
Perhaps we actually wanted a fraction of complex integers.
\end{xtccomment}
\begin{spadsrc}
(2/3)$Fraction(Complex Integer)
\end{spadsrc}
\begin{TeXOutput}
\begin{fricasmath}{3}
\frac{2}{3}%
\end{fricasmath}
\end{TeXOutput}
\formatResultType{Fraction(Complex(Integer))}
\end{xtc}

In each case, \Language{} used the indicated operations, sometimes
first needing to convert the two integers into objects of an
appropriate type.
In these examples, \spadopFrom{/}{Fraction} is written as an
infix operator.

\beginImportant
To use package calling with an infix operator, use the
following syntax:
\begin{center}
{\tt ( \subscriptIt{arg}{1} {\it op} \subscriptIt{arg}{1} )\${\it type} }
\end{center}
\endImportant

We used, for example, \spad{(2/3)$Float}.
The expression \spad{2 + 3 + 4} is equivalent to \spad{(2+3) + 4.}
Therefore in the expression
\spad{(2 + 3 + 4)$Float} the second
\spadop{+} comes from the \spadtype{Float} domain.
Can you guess whether the first \spadop{+} comes from
\spadtype{Integer} or \spadtype{Float}?\footnote{\spadtype{Float},
because the package call causes \Language{} to convert
\spad{(2 + 3)} and \spad{4} to type \spadtype{Float}.
Before the sum is converted, it is given a target type (see below) of
\spadtype{Float} by \Language{} and then evaluated.
The target type causes the \spadop{+} from \spadtype{Float} to be used.}

\beginImportant
For an operator written before its arguments, you must use
parentheses around the arguments (even if there is only one),
and follow the closing parenthesis by a \spadSyntax{$}
and then the type.
\begin{center}
{\tt {\it fun} ( \subscriptIt{arg}{1}, \subscriptIt{arg}{1}, \ldots, \subscriptIt{arg}{N} )\${\it type}}
\end{center}
\endImportant

For example, to call the ``minimum'' function from \spadtype{DoubleFloat}
on two integers, you could write \spad{min(4,89)$DoubleFloat}.
Another use of package calling is to tell \Language{} to use a library
function rather than a function you defined.
We discuss this in \spadref{ugUserUse}.

Sometimes rather than specifying where an operation comes from, you just
want to say what type the result should be.
We say that you provide
\index{type!target}
a
\spadglossSee{target type}{target} for the expression.
\index{target type}
Instead of using a \spadSyntax{$}, use a \spadSyntax{@} to specify
the requested target type.
Otherwise, the syntax is the same.
Note that giving a target type is not the same as explicitly doing a
conversion.
The first says ``try to pick operations so that the result has
such-and-such a type.''
The second says ``compute the result and then convert to an object of
such-and-such a type.''

\begin{xtc}
\begin{xtccomment}
Sometimes it makes sense, as in this expression,
to say ``choose the operations in this expression so that
the final result is a \spadtype{Float}.''
\end{xtccomment}
\begin{spadsrc}
(2/3)@Float
\end{spadsrc}
\begin{TeXOutput}
\begin{fricasmath}{4}
\STRING{0.66666666666666666667}%
\end{fricasmath}
\end{TeXOutput}
\formatResultType{Float}
\end{xtc}

Here we used \spadSyntax{@} to say that the target type of the
left-hand side was \spadtype{Float}.
In this simple case, there was no real difference
between using \spadSyntax{$} and \spadSyntax{@}.
You can see the difference if you try the following.
\begin{xtc}
\begin{xtccomment}
This says to try to choose \spadop{+} so that the result is
a string.
\Language{} cannot do this.
\end{xtccomment}
\begin{spadsrc}
(2 + 3)@String
\end{spadsrc}
\begin{MessageOutput}
   An expression involving @ String actually evaluated to one of type 
      PositiveInteger . Perhaps you should use :: String .
\end{MessageOutput}
\end{xtc}
\begin{xtc}
\begin{xtccomment}
This says to get the \spadop{+} from \spadtype{String} and apply
it to the two integers.
\Language{} also cannot do this because there is no \spadop{+}
exported by \spadtype{String}.
\end{xtccomment}
\begin{spadsrc}
(2 + 3)$String
\end{spadsrc}
\begin{MessageOutput}
   The function + is not implemented in String .
\end{MessageOutput}
\end{xtc}
(By the way, the operation \spadfunFrom{concat}{String} or juxtaposition
is used to concatenate two strings.)
\exptypeindex{String}

When we have more than one operation in an expression, the
difference is even more evident.
The following two expressions show that \Language{} uses the
target type to create different objects.
The \spadop{+}, \spadop{*} and \spadop{^} operations are all
chosen so that an object of the correct final type is created.

\begin{xtc}
\begin{xtccomment}
This says that the operations should be chosen so
that the result is a \spadtype{Complex} object.
\end{xtccomment}
\begin{spadsrc}
((x + y * %i)^2)@(Complex Polynomial Integer)
\end{spadsrc}
\begin{TeXOutput}
\begin{fricasmath}{5}
-{\SUPER{\SYMBOL{y}}{2}}+\SUPER{\SYMBOL{x}}{2}+2\TIMES \SYMBOL{x}\TIMES %
\SYMBOL{y}\TIMES \ImaginaryI %
\end{fricasmath}
\end{TeXOutput}
\formatResultType{Complex(Polynomial(Integer))}
\end{xtc}
\begin{xtc}
\begin{xtccomment}
This says that the operations should be chosen so
that the result is a \spadtype{Polynomial} object.
\end{xtccomment}
\begin{spadsrc}
((x + y * %i)^2)@(Polynomial Complex Integer)
\end{spadsrc}
\begin{TeXOutput}
\begin{fricasmath}{6}
-{\SUPER{\SYMBOL{y}}{2}}+2\TIMES \ImaginaryI \TIMES \SYMBOL{x}\TIMES \SYMBOL{%
y}+\SUPER{\SYMBOL{x}}{2}%
\end{fricasmath}
\end{TeXOutput}
\formatResultType{Polynomial(Complex(Integer))}
\end{xtc}
\begin{xtc}
\begin{xtccomment}
What do you think might happen if we left off all
target type and package call information in this last example?
\end{xtccomment}
\begin{spadsrc}
(x + y * %i)^2 
\end{spadsrc}
\begin{TeXOutput}
\begin{fricasmath}{7}
-{\SUPER{\SYMBOL{y}}{2}}+2\TIMES \ImaginaryI \TIMES \SYMBOL{x}\TIMES \SYMBOL{%
y}+\SUPER{\SYMBOL{x}}{2}%
\end{fricasmath}
\end{TeXOutput}
\formatResultType{Polynomial(Complex(Integer))}
\end{xtc}
\begin{xtc}
\begin{xtccomment}
We can convert it to \spadtype{Complex} as an afterthought.
But this is more work than just saying making what we want in the first
place.
\end{xtccomment}
\begin{spadsrc}
% :: Complex ? 
\end{spadsrc}
\begin{TeXOutput}
\begin{fricasmath}{8}
-{\SUPER{\SYMBOL{y}}{2}}+\SUPER{\SYMBOL{x}}{2}+2\TIMES \SYMBOL{x}\TIMES %
\SYMBOL{y}\TIMES \ImaginaryI %
\end{fricasmath}
\end{TeXOutput}
\formatResultType{Complex(Polynomial(Integer))}
\end{xtc}

Finally, another use of package calling is to qualify fully an
operation that is passed as an argument to a function.

\begin{xtc}
\begin{xtccomment}
Start with a small matrix of integers.
\end{xtccomment}
\begin{spadsrc}
h := matrix [[8,6],[-4,9]] 
\end{spadsrc}
\begin{TeXOutput}
\begin{fricasmath}{9}
\begin{MATRIX}{2}8&6\\-{4}&9\end{MATRIX}%
\end{fricasmath}
\end{TeXOutput}
\formatResultType{Matrix(Integer)}
\end{xtc}
%
\begin{xtc}
\begin{xtccomment}
We want to produce a new matrix that has for entries the multiplicative
inverses of the entries of \spad{h}.
One way to do this is by calling
\spadfunFrom{map}{MatrixCategoryFunctions2} with the
\spadfunFrom{inv}{Fraction} function from \spadtype{Fraction (Integer)}.
\end{xtccomment}
\begin{spadsrc}
map(inv$Fraction(Integer),h) 
\end{spadsrc}
\begin{TeXOutput}
\begin{fricasmath}{10}
\begin{MATRIX}{2}\frac{1}{8}&\frac{1}{6}\\-{\frac{1}{4}}&\frac{1}{9}%
\end{MATRIX}%
\end{fricasmath}
\end{TeXOutput}
\formatResultType{Matrix(Fraction(Integer))}
\end{xtc}
\begin{xtc}
\begin{xtccomment}
We could have been a bit less verbose and used abbreviations.
\end{xtccomment}
\begin{spadsrc}
map(inv$FRAC(INT),h) 
\end{spadsrc}
\begin{TeXOutput}
\begin{fricasmath}{11}
\begin{MATRIX}{2}\frac{1}{8}&\frac{1}{6}\\-{\frac{1}{4}}&\frac{1}{9}%
\end{MATRIX}%
\end{fricasmath}
\end{TeXOutput}
\formatResultType{Matrix(Fraction(Integer))}
\end{xtc}
%
\begin{xtc}
\begin{xtccomment}
As it turns out, \Language{} is smart enough to know what we mean
anyway.
We can just say this.
\end{xtccomment}
\begin{spadsrc}
map(inv,h) 
\end{spadsrc}
\begin{TeXOutput}
\begin{fricasmath}{12}
\begin{MATRIX}{2}\frac{1}{8}&\frac{1}{6}\\-{\frac{1}{4}}&\frac{1}{9}%
\end{MATRIX}%
\end{fricasmath}
\end{TeXOutput}
\formatResultType{Matrix(Fraction(Integer))}
\end{xtc}

% *********************************************************************
\head{section}{Resolving Types}{ugTypesResolve}
% *********************************************************************

In this section we briefly describe an internal process by which
\index{resolve}
\Language{} determines a type to which two objects of possibly
different types can be converted.
We do this to give you further insight into how \Language{} takes
your input, analyzes it, and produces a result.

What happens when you enter \spad{x + 1} to \Language{}?
Let's look at what you get from the two terms of this expression.

\begin{xtc}
\begin{xtccomment}
This is a symbolic object whose type indicates the name.
\end{xtccomment}
\begin{spadsrc}
x
\end{spadsrc}
\begin{TeXOutput}
\begin{fricasmath}{1}
\SYMBOL{x}%
\end{fricasmath}
\end{TeXOutput}
\formatResultType{Variable(x)}
\end{xtc}
\begin{xtc}
\begin{xtccomment}
This is a positive integer.
\end{xtccomment}
\begin{spadsrc}
1
\end{spadsrc}
\begin{TeXOutput}
\begin{fricasmath}{2}
1%
\end{fricasmath}
\end{TeXOutput}
\formatResultType{PositiveInteger}
\end{xtc}

There are no operations in \spadtype{PositiveInteger} that add
positive integers to objects of type \spadtype{Variable(x)} nor
are there any in \spadtype{Variable(x)}.
Before it can add the two parts, \Language{} must come up with
a common type to which both \spad{x} and \spad{1} can be
converted.
We say that \Language{} must {\it resolve} the two types
into a common type.
In this example, the common type is \spadtype{Polynomial(Integer)}.

\begin{xtc}
\begin{xtccomment}
Once this is determined, both parts are converted into polynomials,
and the addition operation from \spadtype{Polynomial(Integer)} is used to
get the answer.
\end{xtccomment}
\begin{spadsrc}
x + 1
\end{spadsrc}
\begin{TeXOutput}
\begin{fricasmath}{3}
\SYMBOL{x}+1%
\end{fricasmath}
\end{TeXOutput}
\formatResultType{Polynomial(Integer)}
\end{xtc}
\begin{xtc}
\begin{xtccomment}
\Language{} can always resolve two types: if nothing resembling
the original types can be found, then \spadtype{Any} is be used.
\exptypeindex{Any}
This is fine and useful in some cases.
\end{xtccomment}
\begin{spadsrc}
["string",3.14159]
\end{spadsrc}
\begin{TeXOutput}
\begin{fricasmath}{4}
\BRACKET{\STRING{"string"}\COMMA \STRING{3.14159}}%
\end{fricasmath}
\end{TeXOutput}
\formatResultType{List(Any)}
\end{xtc}
\begin{xtc}
\begin{xtccomment}
In other cases objects of type \spadtype{Any} can't be used
by the operations you specified.
\end{xtccomment}
\begin{spadsrc}
"string" + 3.14159
\end{spadsrc}
\begin{MessageOutput}
   There are 14 exposed and 9 unexposed library operations named + 
      having 2 argument(s) but none was determined to be applicable. 
      Use HyperDoc Browse, or issue
                                )display op +
      to learn more about the available operations. Perhaps 
      package-calling the operation or using coercions on the arguments
      will allow you to apply the operation.
\end{MessageOutput}
\begin{MessageOutput}
   Cannot find a definition or applicable library operation named + 
      with argument type(s) 
                                   String
                                    Float
      
      Perhaps you should use "@" to indicate the required return type, 
      or "$" to specify which version of the function you need.
\end{MessageOutput}
\end{xtc}
Although this example was contrived, your expressions may need
to be qualified slightly to help \Language{} resolve the
types involved.
You may need to declare a few variables, do some package calling,
provide some target type information or do some explicit
conversions.

We suggest that you just enter the expression you want evaluated and
see what \Language{} does.
We think you will be impressed with its ability to ``do what I
mean.''
If \Language{} is still being obtuse, give it some hints.
As you work with \Language{}, you will learn where it needs a
little help to analyze quickly and perform your computations.

% *********************************************************************
\head{section}{Exposing Domains and Packages}{ugTypesExpose}
% *********************************************************************

In this section we discuss how \Language{} makes some operations
available to you while hiding others that are meant to be used by
developers or only in rare cases.
If you are a new user of \Language{}, it is likely that everything
you need is available by default and you may want
to skip over this section on first reading.

Every
\index{constructor!exposed}
domain and package in the \Language{} library
\index{constructor!hidden}
is
\index{exposed!constructor}
either
\spadglossSee{exposed}{expose} (meaning that you can use its operations without doing
anything special) or it is {\it hidden} (meaning you have to either
package call
(see \spadref{ugTypesPkgCall})
the operations it contains or explicitly expose it to use the
operations).
The initial exposure status for a constructor is set in the
file {\bf exposed.lsp} (see the {\it Installer's Note}
\index{exposed.lsp @{\bf exposed.lsp}}
for \Language{}
\index{file!exposed.lsp @{\bf exposed.lsp}}
if you need to know the location of this file).
Constructors are collected together in
\index{group!exposure}
{\it exposure groups}.
\index{exposure!group}
Categories are all in the exposure group ``categories'' and the
bulk of the basic set of packages and domains that are exposed
are in the exposure group ``basic.''
Here is an abbreviated sample of the file (without the Lisp parentheses):
\begin{verbatim}
basic
        AlgebraicNumber                          AN
        AlgebraGivenByStructuralConstants        ALGSC
        Any                                      ANY
        AnyFunctions1                            ANY1
        BinaryExpansion                          BINARY
        Boolean                                  BOOLEAN
        CardinalNumber                           CARD
        CartesianTensor                          CARTEN
        Character                                CHAR
        CharacterClass                           CCLASS
        CliffordAlgebra                          CLIF
        Color                                    COLOR
        Complex                                  COMPLEX
        ContinuedFraction                        CONTFRAC
        DecimalExpansion                         DECIMAL
        ...
\end{verbatim}
\begin{verbatim}
categories
        AbelianGroup                             ABELGRP
        AbelianMonoid                            ABELMON
        AbelianMonoidRing                        AMR
        AbelianSemiGroup                         ABELSG
        Aggregate                                AGG
        Algebra                                  ALGEBRA
        AlgebraicallyClosedField                 ACF
        AlgebraicallyClosedFunctionSpace         ACFS
        ArcHyperbolicFunctionCategory            AHYP
        ...
\end{verbatim}
For each constructor in a group, the full name and the abbreviation
is given.
There are other groups in {\bf exposed.lsp} but initially only the
constructors in exposure groups ``basic'' and ``categories'' are exposed.

As an interactive user of \Language{}, you do not need to modify
this file.
Instead, use \spadsys{)set expose} to expose, hide or query the exposure
status of an individual constructor or exposure group.
\syscmdindex{set expose}
The reason for having exposure groups is to be able to expose or hide
multiple constructors with a single command.
For example, you might group together into exposure group ``quantum'' a
number of domains and packages useful for quantum mechanical computations.
These probably should not be available to every user, but you want an easy
way to make the whole collection visible to \Language{} when it is looking
for operations to apply.

If you wanted to hide all the basic constructors available by default, you
would issue \spadsys{)set expose drop group basic}.
\syscmdindex{set expose drop group} We do not recommend that you do this.
If, however, you discover that you have hidden all the basic constructors,
you should issue \spadsys{)set expose add group basic} to restore your
default environment.
\syscmdindex{set expose add group}

It is more likely that you would want to expose or hide individual
constructors.
In \spadref{ugUserTriangle} we use several operations from
\spadtype{OutputForm}, a domain usually hidden.
To avoid package calling every operation from \spadtype{OutputForm}, we
expose the domain and let \Language{} conclude that those operations should
be used.
Use \spadsys{)set expose add constructor} and \spadsys{)set expose drop
constructor} to expose and hide a constructor, respectively.
\syscmdindex{set expose drop constructor}
You should use the constructor name, not the abbreviation.
The \spadsys{)set expose} command guides you through these options.
\syscmdindex{set expose add constructor}

If you expose a previously hidden constructor, \Language{}
exhibits new behavior (that was your intention) though you might not
expect the results that you get.
\spadtype{OutputForm} is, in fact, one of the worst offenders in this
regard.
\exptypeindex{OutputForm}
This domain is meant to be used by other domains for creating a
structure that \Language{} knows how to display.
It has functions like \spadopFrom{+}{OutputForm} that form output
representations rather than do mathematical calculations.
Because of the order in which \Language{} looks at constructors
when it is deciding what operation to apply, \spadtype{OutputForm}
might be used instead of what you expect.
\begin{xtc}
\begin{xtccomment}
This is a polynomial.
\end{xtccomment}
\begin{spadsrc}
x + x
\end{spadsrc}
\begin{TeXOutput}
\begin{fricasmath}{1}
2\TIMES \SYMBOL{x}%
\end{fricasmath}
\end{TeXOutput}
\formatResultType{Polynomial(Integer)}
\end{xtc}
\begin{xtc}
\begin{xtccomment}
Expose \spadtype{OutputForm}.
\end{xtccomment}
\begin{spadsrc}
)set expose add constructor OutputForm 
\end{spadsrc}
\begin{SysCmdOutput}
   OutputForm is now explicitly exposed in frame initial 
\end{SysCmdOutput}
\end{xtc}
\begin{xtc}
\begin{xtccomment}
This is what we get when \spadtype{OutputForm} is automatically
available.
\end{xtccomment}
\begin{spadsrc}
x + x 
\end{spadsrc}
\begin{TeXOutput}
\begin{fricasmath}{2}
\SYMBOL{x}+\SYMBOL{x}%
\end{fricasmath}
\end{TeXOutput}
\formatResultType{OutputForm}
\end{xtc}
\begin{xtc}
\begin{xtccomment}
Hide \spadtype{OutputForm} so we don't run into problems
with any later examples!
\end{xtccomment}
\begin{spadsrc}
)set expose drop constructor OutputForm 
\end{spadsrc}
\begin{SysCmdOutput}
   OutputForm is now explicitly hidden in frame initial 
\end{SysCmdOutput}
\end{xtc}

Finally, exposure is done on a frame-by-frame basis.
A \spadgloss{frame} (see \spadref{ugSysCmdframe})
\index{frame!exposure and}
is one of possibly several
logical \Language{} workspaces within a physical one, each having
its own environment (for example, variables and function definitions).
If you have several \Language{} workspace windows on your screen, they
are all different frames, automatically created for you by \HyperName{}.
Frames can be manually created, made active and destroyed by the
\spadsys{)frame} system command.
\syscmdindex{frame}
They do not share exposure information, so you need to use
\spadsys{)set expose} in each one to add or drop constructors from view.

% *********************************************************************
\head{section}{Commands for Snooping}{ugAvailSnoop}
% *********************************************************************

To conclude this chapter, we introduce you to some system commands
that you can use for getting more information about domains,
packages, categories, and operations.
The most powerful \Language{} facility for getting information about
constructors and operations is the \Browse{} component of \HyperName{}.
This is discussed in \chapref{ugBrowse}.

Use the \spadsys{)what} system command to see lists of system objects
whose name contain a particular substring (uppercase or lowercase is
not significant).
\syscmdindex{what}

\begin{xtc}
\begin{xtccomment}
Issue this to see a list of all operations with
``{\tt complex}'' in their names.
\syscmdindex{what operation}
\end{xtccomment}
\begin{spadsrc}
)what operation complex
\end{spadsrc}
\begin{SysCmdOutput}

Operations whose names satisfy the above pattern(s):

chainComplex                      coChainComplex                    
complex                           complex?                          
complexEigenvalues                complexEigenvectors               
complexElementary                 complexExpand                     
complexForm                       complexIntegrate                  
complexLimit                      complexNormalize                  
complexNumeric                    complexNumericIfCan               
complexRoots                      complexSolve                      
complexZeros                      createLowComplexityNormalBasis    
createLowComplexityTable          cubicalComplex                    
deltaComplex                      doubleComplex?                    
drawComplex                       drawComplexVectorField            
fortranComplex                    fortranDoubleComplex              
pmComplexintegrate                simplicialComplex                 
simplicialComplexIfCan            testComplexEquals                 
testComplexEqualsAux              xftestComplexEquals               
xftestComplexEqualsAux            
   
      To get more information about an operation such as complexForm , 
      issue the command )display op complexForm 
\end{SysCmdOutput}
\end{xtc}
\begin{xtc}
\begin{xtccomment}
If you want to see all domains with ``{\tt matrix}'' in their names, issue
this.
\syscmdindex{what domain}
\end{xtccomment}
\begin{spadsrc}
)what domain matrix
\end{spadsrc}
\begin{SysCmdOutput}
--------------------------------- Domains ---------------------------------

Domains with names matching patterns:
     matrix 

 CDFMAT   ComplexDoubleFloatMatrix     DFMAT    DoubleFloatMatrix
 DHMATRIX DenavitHartenbergMatrix      DPMM     DirectProductMatrixModule
 IMATRIX  IndexedMatrix                LSQM     LieSquareMatrix
 M3D      ThreeDimensionalMatrix       MATCAT-  MatrixCategory&
 MATRIX   Matrix                       RMATCAT- RectangularMatrixCategory&
 RMATRIX  RectangularMatrix            SEM      SparseEchelonMatrix
 SMATCAT- SquareMatrixCategory&        SQMATRIX SquareMatrix
 U16MAT   U16Matrix                    U32MAT   U32Matrix
 U8MAT    U8Matrix
\end{SysCmdOutput}
\end{xtc}
\begin{xtc}
\begin{xtccomment}
Similarly, if you wish to see all packages whose names contain
``{\tt gauss}'', enter this.
\syscmdindex{what packages}
\end{xtccomment}
\begin{spadsrc}
)what package gauss
\end{spadsrc}
\begin{SysCmdOutput}
-------------------------------- Packages ---------------------------------

Packages with names matching patterns:
     gauss 

 FFFG     FractionFreeFastGaussian
 FFFGF    FractionFreeFastGaussianFractions
 GAUSSFAC GaussianFactorizationPackage  UGAUSS   UnitGaussianElimination
\end{SysCmdOutput}
\end{xtc}
\begin{xtc}
\begin{xtccomment}
This command shows all
the operations that \spadtype{Any} provides.
Wherever \spadSyntax{$} appears, it means ``\spadtype{Any}''.
\syscmdindex{show}
\end{xtccomment}
\begin{spadsrc}
)show Any
\end{spadsrc}
\begin{SysCmdOutput}
 Any is a domain constructor
 Abbreviation for Any is ANY 
 This constructor is exposed in this frame.
------------------------------- Operations --------------------------------
 ?=? : (%,%) -> Boolean                any : (SExpression,None) -> %
 coerce : % -> OutputForm              dom : % -> SExpression
 domainOf : % -> OutputForm            hash : % -> SingleInteger
 latex : % -> String                   obj : % -> None
 objectOf : % -> OutputForm            ?~=? : (%,%) -> Boolean
 hashUpdate! : (HashState,%) -> HashState
 showTypeInOutput : Boolean -> String

\end{SysCmdOutput}
\end{xtc}
\begin{xtc}
\begin{xtccomment}
This displays all operations with the name \spadfun{complex}.
\syscmdindex{display operation}
\end{xtccomment}
\begin{spadsrc}
)display operation complex
\end{spadsrc}
\begin{SysCmdOutput}

There is one exposed function called complex :
   [1] (D1,D1) -> D from D if D has COMPCAT(D1) and D1 has COMRING
\end{SysCmdOutput}
\end{xtc}
Let's analyze this output.
\begin{xtc}
\begin{xtccomment}
First we find out what some of the abbreviations mean.
\end{xtccomment}
\begin{spadsrc}
)abbreviation query COMPCAT
\end{spadsrc}
\begin{SysCmdOutput}
   COMPCAT abbreviates category ComplexCategory 
\end{SysCmdOutput}
\end{xtc}
\begin{xtc}
\begin{xtccomment}
\end{xtccomment}
\begin{spadsrc}
)abbreviation query COMRING
\end{spadsrc}
\begin{SysCmdOutput}
   COMRING abbreviates category CommutativeRing 
\end{SysCmdOutput}
\end{xtc}

So if \spad{D1} is a commutative ring (such as the integers or
floats) and \spad{D} belongs to
\spadtype{ComplexCategory D1},
then there is an operation called \spadfun{complex} that
takes two elements of \spad{D1} and creates an element of
\spad{D}.
The primary example of a constructor implementing domains
belonging to \spadtype{ComplexCategory} is \spadtype{Complex}.
See \xmpref{Complex} for more information on that and see
\spadref{ugUserDeclare} for more information on function types.
