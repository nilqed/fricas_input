\documentclass[twoside,a4paper]{book}
\usepackage{fricas}

\begin{document}
%
% \thispagestyle{empty}
% \vspace*{1in}
% {\Large\sf Richard D. Jenks \hfill Robert S. Sutor \newline
% \vskip 1in
% \begin{center}
% \LARGE\sf AXIOM \\
% \quad \\ \quad \\
% \Large\sf The Scientific Computation System
% \quad \\ \quad \\ \vskip 1.5in
% \normalsize\rm Draft: \today
% \end{center}}
% \newpage
% \thispagestyle{empty}
% \quad
%
\pagenumbering{roman}
\setcounter{page}{0}

\begin{quote}
\color{red}\Large
  This is an attempt to make the content of the Axiom book (by Jenks
  and Sutor) available again in order to describe the FriCAS project
  (which is a fork of the original Axiom code). The following
  material is mainly taken from the original Axiom book, but is partly
  tailored to new developments in FriCAS.

  \vspace{1cm}

  \textbf{WARNING: This is work-in-progress! There might be errors and
    even false statements.}

  Plan is as follows:
  \begin{enumerate}
  \item Make LaTeX compilation work and include (generated) pictures
    and generated output of algebra commands.
  \item Simplify and remove redundant \LaTeX{} commands.
  \item Build on modern advanced latex packages.
  \item Generate indices and hyperrefs.
  \end{enumerate}

  All of the above will be done by taking care of still generating the
  same \texttt{.ht} files although the technicalities might look
  different. The contents of the files (no matter how wrong it
  actually is) will only be changed marginally. Rewriting of the
  contents will only happen after the technical part of producing the
  PDF file has been finished.

  \hspace*{\fill}Ralf Hemmecke
\end{quote}

%\input{dedicate}
%\input{foreword}
\clearpage
\tableofcontents
%\input{bios}
%
\pagenumbering{arabic}
\setcounter{chapter}{-1}
% !! DO NOT MODIFY THIS FILE BY HAND !! Created by spool2tex.awk.
\head{chapter}{Introduction to \Language{}}{ugNewIntro}

Welcome to the world of \Language{}.
We call \Language{} a scientific computation system:
a self-contained toolbox designed to meet
your scientific programming needs,
from symbolics, to numerics, to graphics.

This introduction is a quick overview of what \Language{} offers.

\head{section}{Symbolic computation}{ugNewIntroSymbolic}

\Language{} provides a wide range of
simple commands for
symbolic mathematical problem solving.
Do you need to solve an equation,
to expand a series,
or to obtain an integral?
If so, just ask \Language{} to do it.

%
\begin{xtc}
\begin{xtccomment}
Integrate $\frac{1}{(x^3 \  {(a+b x)}^{1/3})}$ with
respect to \smath{x}.
\end{xtccomment}
\begin{spadsrc}
integrate(1/(x^3 * (a+b*x)^(1/3)),x)
\end{spadsrc}
\begin{TeXOutput}
\begin{fricasmath}{1}
\frac{-{2\TIMES \SUPER{\SYMBOL{b}}{2}\TIMES \SUPER{\SYMBOL{x}}{2}\TIMES \sqrt%
{3}\TIMES \log{\PAREN{\nthroot{\SYMBOL{a}}{3}\TIMES \SUPER{\nthroot{\SYMBOL{b%
}\TIMES \SYMBOL{x}+\SYMBOL{a}}{3}}{2}+\SUPER{\nthroot{\SYMBOL{a}}{3}}{2}%
\TIMES \nthroot{\SYMBOL{b}\TIMES \SYMBOL{x}+\SYMBOL{a}}{3}+\SYMBOL{a}}}}+4%
\TIMES \SUPER{\SYMBOL{b}}{2}\TIMES \SUPER{\SYMBOL{x}}{2}\TIMES \sqrt{3}%
\TIMES \log{\PAREN{\SUPER{\nthroot{\SYMBOL{a}}{3}}{2}\TIMES \nthroot{\SYMBOL{%
b}\TIMES \SYMBOL{x}+\SYMBOL{a}}{3}-{\SYMBOL{a}}}}+12\TIMES \SUPER{\SYMBOL{b}%
}{2}\TIMES \SUPER{\SYMBOL{x}}{2}\TIMES \arctan{\PAREN{\frac{2\TIMES \sqrt{3}%
\TIMES \SUPER{\nthroot{\SYMBOL{a}}{3}}{2}\TIMES \nthroot{\SYMBOL{b}\TIMES %
\SYMBOL{x}+\SYMBOL{a}}{3}+\SYMBOL{a}\TIMES \sqrt{3}}{3\TIMES \SYMBOL{a}}}}+%
\PAREN{12\TIMES \SYMBOL{b}\TIMES \SYMBOL{x}-{9\TIMES \SYMBOL{a}}}\TIMES \sqrt%
{3}\TIMES \nthroot{\SYMBOL{a}}{3}\TIMES \SUPER{\nthroot{\SYMBOL{b}\TIMES %
\SYMBOL{x}+\SYMBOL{a}}{3}}{2}}{18\TIMES \SUPER{\SYMBOL{a}}{2}\TIMES \SUPER{%
\SYMBOL{x}}{2}\TIMES \sqrt{3}\TIMES \nthroot{\SYMBOL{a}}{3}}%
\end{fricasmath}
\end{TeXOutput}
\formatResultType{Union(Expression(Integer), ...)}
\end{xtc}

\Language{} provides state-of-the-art algebraic machinery to
handle your most advanced symbolic problems.
For example, \Language{}'s integrator gives you the answer
when an answer exists.
If one does not, it provides a proof that
there is no answer.
Integration is just one of a multitude of symbolic operations that
\Language{} provides.

\head{section}{Numeric computation}{ugNewIntroNumeric}

\Language{} has a numerical library that includes operations for
linear algebra, solution of equations, and special functions.
For many of these operations, you can select any number of
floating point digits to be carried out in the computation.

%
\begin{xtc}
\begin{xtccomment}
Solve $x^{49}-49x^4+9$ to
49 digits of accuracy.
\end{xtccomment}
\begin{spadsrc}
solve(x^49-49*x^4+9 = 0,1.e-49)
\end{spadsrc}
\begin{TeXOutput}
\begin{fricasmath}{1}
\BRACKET{\SYMBOL{x}=-{\STRING{0.65465367069042711367}}\COMMA \SYMBOL{x}=%
\STRING{1.0869213956538595085}\COMMA \SYMBOL{x}=\STRING{%
0.65465367072552717397}}%
\end{fricasmath}
\end{TeXOutput}
\formatResultType{List(Equation(Polynomial(Float)))}
\end{xtc}

The output of a computation can be converted to FORTRAN to be used
in a later numerical computation.
Besides floating point numbers, \Language{} provides literally
dozens of kinds of numbers to compute with.
These range from various kinds of integers, to fractions, complex
numbers, quaternions, continued fractions, and to numbers represented
with an arbitrary base.

%
\begin{xtc}
\begin{xtccomment}
What is \spad{10} to
the \eth{100} power in base \spad{32}?
\end{xtccomment}
\begin{spadsrc}
radix(10^100,32)
\end{spadsrc}
\begin{TeXOutput}
\begin{fricasmath}{2}
4\STRING{I}9\STRING{L}\STRING{K}\STRING{I}\STRING{P}9\STRING{G}\STRING{R}%
\STRING{S}\STRING{T}\STRING{C}5\STRING{I}\STRING{F}164\STRING{P}\STRING{O}5%
\STRING{V}72\STRING{M}\STRING{E}827226\STRING{J}\STRING{S}\STRING{L}\STRING{A%
}\STRING{P}462585\STRING{Q}7\STRING{H}00000000000000000000%
\end{fricasmath}
\end{TeXOutput}
\formatResultType{RadixExpansion(32)}
\end{xtc}


\head{section}{Graphics}{ugNewIntroGraphics}

You may often want to visualize a symbolic formula or draw
a graph from a set of numerical values.
To do this, you can call upon the \Language{}
graphics capability.

\begin{psXtc}
\begin{xtccomment}
Draw $J_0(\sqrt{x^2+y^2})$ for
$-20 \leq x,y \leq 20$.
\end{xtccomment}
\begin{spadsrc}
draw(5*besselJ(0,sqrt(x^2+y^2)), x=-20..20, y=-20..20)
\end{spadsrc}
\epsffile[0 0 295 295]{bessintr.ps}
\end{psXtc}

Graphs in \Language{} are interactive objects you can manipulate with
your mouse.
Just click on the graph, and a control panel pops up.
Using this mouse and the control panel, you can translate,
rotate, zoom,
change the coloring, lighting, shading, and
perspective on the picture.
You can also generate a PostScript copy of your graph to produce
hard-copy output.

\head{section}{\HyperName{}}{ugNewIntroFriCAS}

\HyperName{} presents you windows on the world of \Language{}, offering
on-line help,
examples, tutorials, a browser, and reference material.
\HyperName{} gives you on-line access to this book in a ``hypertext'' format.
Words that appear in a different font
(for example, \spadtype{Matrix}, \spadfun{factor}, and
\spadgloss{category}) are generally mouse-active; if you click on one
with your mouse, \HyperName{} shows you a new window for that word.

As another example of a \HyperName{} facility,
suppose that you want to compute the roots of
$x^{49} - 49x^4 + 9$ to \smath{49} digits (as in our previous example)
and you don't know how to tell \Language{} to do this.
The ``basic command'' facility of \HyperName{} leads the way.
Through the series of \HyperName{} windows shown in Figure \ref{fig-intro-br}
and the specified mouse clicks, you and \HyperName{}
generate the correct command to issue to compute the answer.


\begin{figure}[thb]
\vspace*{5.10in}
\caption{Computing the roots of $x^{49}-49x^4+9.$}\label{fig-intro-br}
\vskip .5\baselineskip
\end{figure}

\head{section}{Interactive Programming}{ugNewIntroInteractive}

\Language{}'s interactive programming language lets you
define your own functions.
A simple example of a user-defined function is one
that computes the successive Legendre polynomials.
\Language{} lets you define these polynomials in a piece-wise way.

%
\begin{xtc}
\begin{xtccomment}
The first Legendre polynomial.
\end{xtccomment}
\begin{spadsrc}
p(0) == 1
\end{spadsrc}
\end{xtc}
%
\begin{xtc}
\begin{xtccomment}
The second Legendre polynomial.
\end{xtccomment}
\begin{spadsrc}
p(1) == x
\end{spadsrc}
\end{xtc}
%
\begin{xtc}
\begin{xtccomment}
The \eth{n} Legendre polynomial for $(n > 1)$.
\end{xtccomment}
\begin{spadsrc}
p(n) == ((2*n-1)*x*p(n-1) - (n-1) * p(n-2))/n
\end{spadsrc}
\end{xtc}

In addition to letting you define simple functions like this,
the interactive language can be used
to create entire application packages.
All the graphs in the \Gallery{} section
in the center of the book, for example,
were created by programs written in the interactive language.

The above definitions for \spad{p} do no computation---they simply
tell \Language{} how to compute \spad{p(k)} for some positive
integer \spad{k}.
To actually get a value of a Legendre polynomial, you ask for it.
\index{Legendre polynomials}

%
\begin{xtc}
\begin{xtccomment}
What is the tenth Legendre polynomial?
\end{xtccomment}
\begin{spadsrc}
p(10)
\end{spadsrc}
\begin{MessageOutput}
   Compiling function p with type Integer -> Polynomial(Fraction(
      Integer)) 
\end{MessageOutput}
\begin{MessageOutput}
   Compiling function p as a recurrence relation.
\end{MessageOutput}
\begin{TeXOutput}
\begin{fricasmath}{4}
\frac{46189}{256}\TIMES \SUPER{\SYMBOL{x}}{10}-{\frac{109395}{256}\TIMES %
\SUPER{\SYMBOL{x}}{8}}+\frac{45045}{128}\TIMES \SUPER{\SYMBOL{x}}{6}-{\frac{%
15015}{128}\TIMES \SUPER{\SYMBOL{x}}{4}}+\frac{3465}{256}\TIMES \SUPER{%
\SYMBOL{x}}{2}-{\frac{63}{256}}%
\end{fricasmath}
\end{TeXOutput}
\formatResultType{Polynomial(Fraction(Integer))}
\end{xtc}

\Language{} applies the above pieces for
\spad{p} to obtain the value of \spad{p(10)}.
But it does more:
it creates an optimized, compiled function for \spad{p}.
The function is formed by putting the pieces together into
a single piece of code.
By {\it compiled}, we mean that the function is translated into
basic machine-code.
By {\it optimized}, we mean that certain transformations are
performed on that code to make it run faster.
For \spad{p}, \Language{} actually translates the original definition
that is recursive (one that calls itself)
to one that is iterative (one that consists of a simple loop).

%
\begin{xtc}
\begin{xtccomment}
What is the coefficient of \smath{x^{90}} in \spad{p(90)}?
\end{xtccomment}
\begin{spadsrc}
coefficient(p(90),x,90)
\end{spadsrc}
\begin{TeXOutput}
\begin{fricasmath}{5}
\frac{56882655420520178222 23458237426581853561 497449095175}{%
77371252455336267181 195264}%
\end{fricasmath}
\end{TeXOutput}
\formatResultType{Polynomial(Fraction(Integer))}
\end{xtc}

In general, a user function is type-analyzed and compiled on first use.
Later, if you use it with a different kind of object, the function
is recompiled if necessary.

\head{section}{Data Structures}{ugNewIntroData}

A variety of data structures are available for interactive use.
These include strings, lists, vectors, sets, multisets, and hash
tables.
A particularly useful structure for interactive use is the
infinite stream:

%
\begin{xtc}
\begin{xtccomment}
Create the infinite stream of derivatives of Legendre
polynomials
\end{xtccomment}
\begin{spadsrc}
[D(p(i),x) for i in 1..]
\end{spadsrc}
\begin{MessageOutput}
   There are no library operations named p 
      Use HyperDoc Browse or issue
                                 )what op p
      to learn if there is any operation containing " p " in its name.
\end{MessageOutput}
\begin{MessageOutput}
   Cannot find a definition or applicable library operation named p 
      with argument type(s) 
                               PositiveInteger
      
      Perhaps you should use "@" to indicate the required return type, 
      or "$" to specify which version of the function you need.
\end{MessageOutput}
\begin{MessageOutput}
   FriCAS will attempt to step through and interpret the code.
\end{MessageOutput}
\begin{MessageOutput}
   Interpret-Code mode is not supported for stream bodies.
\end{MessageOutput}
\end{xtc}

Streams display only a few of their initial elements.
Otherwise, they are ``lazy'': they
only compute elements when you ask for them.

Data structures are an important component for building
application software. Advanced users can
represent data for applications in optimal fashion.
In all, \Language{} offers
over forty kinds of aggregate data structures, ranging
from mutable structures (such as cyclic lists and flexible arrays) to
storage efficient structures (such as bit vectors).
As an example, streams are used as the internal data structure
for power series.

%
\begin{xtc}
\begin{xtccomment}
What is the series expansion
of $\log(\cot(x))$
about $x=\pi/2$?
\end{xtccomment}
\begin{spadsrc}
series(log(cot(x)),x = %pi/2)
\end{spadsrc}
\begin{TeXOutput}
\begin{fricasmath}{1}
\log{\PAREN{\frac{-{2\TIMES \SYMBOL{x}}+\pi }{2}}}+\frac{1}{3}\TIMES \SUPER{%
\PAREN{\SYMBOL{x}-{\frac{\pi }{2}}}}{2}+\frac{7}{90}\TIMES \SUPER{\PAREN{%
\SYMBOL{x}-{\frac{\pi }{2}}}}{4}+\frac{62}{2835}\TIMES \SUPER{\PAREN{\SYMBOL{%
x}-{\frac{\pi }{2}}}}{6}+\FUN{O}\PAREN{\SUPER{\PAREN{\SYMBOL{x}-{\frac{\pi }{%
2}}}}{8}}%
\end{fricasmath}
\end{TeXOutput}
\formatResultType{GeneralUnivariatePowerSeries(Expression(Integer), x, \%pi/2)}
\end{xtc}

Series and streams make no attempt to compute {\it all} their elements!
Rather, they stand ready to deliver elements on demand.

%
\begin{xtc}
\begin{xtccomment}
What is the coefficient of the \eth{50}
term of this series?
\end{xtccomment}
\begin{spadsrc}
coefficient(%,50)
\end{spadsrc}
\begin{TeXOutput}
\begin{fricasmath}{2}
\frac{44590788901016030052 44724230085655096564 4}{%
71314692864386691115 84090881309360354581 359130859375}%
\end{fricasmath}
\end{TeXOutput}
\formatResultType{Expression(Integer)}
\end{xtc}

\head{section}{Mathematical Structures}{ugNewIntroMathematical}

\Language{} also has many kinds of mathematical structures.
These range from simple ones (like polynomials and matrices) to
more esoteric ones (like ideals and Clifford algebras).
Most structures allow the construction of arbitrarily complicated
``types.''

%
\begin{xtc}
\begin{xtccomment}
Even a simple input expression can
result in a type with several levels.
\end{xtccomment}
\begin{spadsrc}
matrix [[x + %i,0], [1,-2]]
\end{spadsrc}
\begin{TeXOutput}
\begin{fricasmath}{1}
\begin{MATRIX}{2}\SYMBOL{x}+\ImaginaryI &0\\1&-{2}\end{MATRIX}%
\end{fricasmath}
\end{TeXOutput}
\formatResultType{Matrix(Polynomial(Complex(Integer)))}
\end{xtc}

The \Language{} interpreter builds types in response to user
input.
Often, the type of the result is changed in order to be applicable
to an operation.

%
\begin{xtc}
\begin{xtccomment}
The inverse operation requires that elements of the above matrices
are fractions.
\end{xtccomment}
\begin{spadsrc}
inverse(%)
\end{spadsrc}
\begin{TeXOutput}
\begin{fricasmath}{2}
\begin{MATRIX}{2}\frac{1}{\SYMBOL{x}+\ImaginaryI }&0\\\frac{1}{2\TIMES %
\SYMBOL{x}+2\TIMES \ImaginaryI }&-{\frac{1}{2}}\end{MATRIX}%
\end{fricasmath}
\end{TeXOutput}
\formatResultType{Union(Matrix(Fraction(Polynomial(Complex(Integer)))), ...)}
\end{xtc}

\head{section}{Pattern Matching}{ugNewIntroPattern}

A convenient facility for symbolic computation is ``pattern
matching.''
Suppose you have a trigonometric expression and you want to
transform it to some equivalent form.
Use a \spad{rule} command to describe the transformation rules you
\spadkey{rule}
need.
Then give the rules a name and apply that name as a function to
your trigonometric expression.

%
\begin{xtc}
\begin{xtccomment}
Introduce two rewrite rules.
\end{xtccomment}
\begin{spadsrc}
sinCosExpandRules := rule
  sin(x+y) == sin(x)*cos(y) + sin(y)*cos(x)
  cos(x+y) == cos(x)*cos(y) - sin(x)*sin(y)
  sin(2*x) == 2*sin(x)*cos(x)
  cos(2*x) == cos(x)^2 - sin(x)^2
\end{spadsrc}
\begin{TeXOutput}
\begin{fricasmath}{1}
\BRACE{\sin{\PAREN{\SYMBOL{y}+\SYMBOL{x}}}\SYMBOL{\ ==\ }\cos{\SYMBOL{x}}%
\TIMES \sin{\SYMBOL{y}}+\cos{\SYMBOL{y}}\TIMES \sin{\SYMBOL{x}}\COMMA \cos{%
\PAREN{\SYMBOL{y}+\SYMBOL{x}}}\SYMBOL{\ ==\ }-{\sin{\SYMBOL{x}}\TIMES \sin{%
\SYMBOL{y}}}+\cos{\SYMBOL{x}}\TIMES \cos{\SYMBOL{y}}\COMMA \sin{\PAREN{2%
\TIMES \SYMBOL{x}}}\SYMBOL{\ ==\ }2\TIMES \cos{\SYMBOL{x}}\TIMES \sin{\SYMBOL%
{x}}\COMMA \cos{\PAREN{2\TIMES \SYMBOL{x}}}\SYMBOL{\ ==\ }-{\SUPER{\PAREN{%
\sin{\SYMBOL{x}}}}{2}}+\SUPER{\PAREN{\cos{\SYMBOL{x}}}}{2}}%
\end{fricasmath}
\end{TeXOutput}
\formatResultType{Ruleset(Integer, Integer, Expression(Integer))}
\end{xtc}

%
\begin{xtc}
\begin{xtccomment}
Apply the rules to a simple trigonometric expression.
\end{xtccomment}
\begin{spadsrc}
sinCosExpandRules(sin(a+2*b+c))
\end{spadsrc}
\begin{TeXOutput}
\begin{fricasmath}{2}
\PAREN{-{\cos{\SYMBOL{a}}\TIMES \SUPER{\PAREN{\sin{\SYMBOL{b}}}}{2}}-{2%
\TIMES \cos{\SYMBOL{b}}\TIMES \sin{\SYMBOL{a}}\TIMES \sin{\SYMBOL{b}}}+\cos{%
\SYMBOL{a}}\TIMES \SUPER{\PAREN{\cos{\SYMBOL{b}}}}{2}}\TIMES \sin{\SYMBOL{c}}%
-{\cos{\SYMBOL{c}}\TIMES \sin{\SYMBOL{a}}\TIMES \SUPER{\PAREN{\sin{\SYMBOL{b}%
}}}{2}}+2\TIMES \cos{\SYMBOL{a}}\TIMES \cos{\SYMBOL{b}}\TIMES \cos{\SYMBOL{c}%
}\TIMES \sin{\SYMBOL{b}}+\SUPER{\PAREN{\cos{\SYMBOL{b}}}}{2}\TIMES \cos{%
\SYMBOL{c}}\TIMES \sin{\SYMBOL{a}}%
\end{fricasmath}
\end{TeXOutput}
\formatResultType{Expression(Integer)}
\end{xtc}

Using input files, you can create your own library of
transformation rules relevant to your applications, then
selectively apply the rules you need.

\head{section}{Polymorphic Algorithms}{ugNewIntroPolymorphic}

All components of the \Language{} algebra library are written in
the \Language{} library language.
This language is similar to the interactive language
except for protocols that authors are obliged to follow.
The library language permits you to write ``polymorphic
algorithms,'' algorithms defined to work in
their most natural settings and over a variety of types.

%
\begin{xtc}
\begin{xtccomment}
Define a system of polynomial equations \spad{S}.
\end{xtccomment}
\begin{spadsrc}
S := [3*x^3 + y + 1 = 0,y^2 = 4]
\end{spadsrc}
\begin{TeXOutput}
\begin{fricasmath}{1}
\BRACKET{\SYMBOL{y}+3\TIMES \SUPER{\SYMBOL{x}}{3}+1=0\COMMA \SUPER{\SYMBOL{y}%
}{2}=4}%
\end{fricasmath}
\end{TeXOutput}
\formatResultType{List(Equation(Polynomial(Integer)))}
\end{xtc}
%
\begin{xtc}
\begin{xtccomment}
Solve the system \spad{S} using rational number arithmetic and
30 digits of accuracy.
\end{xtccomment}
\begin{spadsrc}
solve(S,1/10^30)
\end{spadsrc}
\begin{TeXOutput}
\begin{fricasmath}{2}
\BRACKET{\BRACKET{\SYMBOL{y}=-{2}\COMMA \SYMBOL{x}=\frac{%
57602201066248085349 435651342568509}{83076749736557242056 487941267521536}}%
\COMMA \BRACKET{\SYMBOL{y}=2\COMMA \SYMBOL{x}=-{\frac{%
18707220957835557353 00716585876842265159 59365500929}{%
18707220957835557353 00716585876842265159 59365500928}}}}%
\end{fricasmath}
\end{TeXOutput}
\formatResultType{List(List(Equation(Polynomial(Fraction(Integer)))))}
\end{xtc}
%
\begin{xtc}
\begin{xtccomment}
Solve \spad{S} with the solutions expressed in radicals.
\end{xtccomment}
\begin{spadsrc}
radicalSolve(S)
\end{spadsrc}
\begin{TeXOutput}
\begin{fricasmath}{3}
\BRACKET{\BRACKET{\SYMBOL{y}=2\COMMA \SYMBOL{x}=-{1}}\COMMA \BRACKET{\SYMBOL{%
y}=2\COMMA \SYMBOL{x}=\frac{-{\sqrt{-{3}}}+1}{2}}\COMMA \BRACKET{\SYMBOL{y}=2%
\COMMA \SYMBOL{x}=\frac{\sqrt{-{3}}+1}{2}}\COMMA \BRACKET{\SYMBOL{y}=-{2}%
\COMMA \SYMBOL{x}=\frac{1}{\nthroot{3}{3}}}\COMMA \BRACKET{\SYMBOL{y}=-{2}%
\COMMA \SYMBOL{x}=\frac{\sqrt{-{1}}\TIMES \sqrt{3}-{1}}{2\TIMES \nthroot{3}{3%
}}}\COMMA \BRACKET{\SYMBOL{y}=-{2}\COMMA \SYMBOL{x}=\frac{-{\sqrt{-{1}}%
\TIMES \sqrt{3}}-{1}}{2\TIMES \nthroot{3}{3}}}}%
\end{fricasmath}
\end{TeXOutput}
\formatResultType{List(List(Equation(Expression(Integer))))}
\end{xtc}

While these solutions look very different, the results were
produced by the same internal algorithm!
The internal algorithm actually works with equations over any ``field.''
Examples of fields are the rational numbers, floating point
numbers, rational functions, power series, and general expressions
involving radicals.

\head{section}{Extensibility}{ugNewIntroExtensibility}

Users and system developers alike can augment the \Language{}
library, all using one common language.
Library code, like interpreter code, is compiled into machine
binary code for run-time efficiency.

Using this language, you can create new computational types and
new algorithmic packages.
All library code is polymorphic, described in terms of a database
of algebraic properties.
By following the language protocols, there is an automatic,
guaranteed interaction between your code and that of colleagues
and system implementers.

\input{tecintro}  % This ha no chapter number.

\part{Basic Features of \Language{}}
%
% !! DO NOT MODIFY THIS FILE BY HAND !! Created by spool2tex.awk.

% Copyright (c) 1991-2002, The Numerical ALgorithms Group Ltd.
% All rights reserved.
%
% Redistribution and use in source and binary forms, with or without
% modification, are permitted provided that the following conditions are
% met:
%
%     - Redistributions of source code must retain the above copyright
%       notice, this list of conditions and the following disclaimer.
%
%     - Redistributions in binary form must reproduce the above copyright
%       notice, this list of conditions and the following disclaimer in
%       the documentation and/or other materials provided with the
%       distribution.
%
%     - Neither the name of The Numerical ALgorithms Group Ltd. nor the
%       names of its contributors may be used to endorse or promote products
%       derived from this software without specific prior written permission.
%
% THIS SOFTWARE IS PROVIDED BY THE COPYRIGHT HOLDERS AND CONTRIBUTORS "AS
% IS" AND ANY EXPRESS OR IMPLIED WARRANTIES, INCLUDING, BUT NOT LIMITED
% TO, THE IMPLIED WARRANTIES OF MERCHANTABILITY AND FITNESS FOR A
% PARTICULAR PURPOSE ARE DISCLAIMED. IN NO EVENT SHALL THE COPYRIGHT OWNER
% OR CONTRIBUTORS BE LIABLE FOR ANY DIRECT, INDIRECT, INCIDENTAL, SPECIAL,
% EXEMPLARY, OR CONSEQUENTIAL DAMAGES (INCLUDING, BUT NOT LIMITED TO,
% PROCUREMENT OF SUBSTITUTE GOODS OR SERVICES-- LOSS OF USE, DATA, OR
% PROFITS-- OR BUSINESS INTERRUPTION) HOWEVER CAUSED AND ON ANY THEORY OF
% LIABILITY, WHETHER IN CONTRACT, STRICT LIABILITY, OR TORT (INCLUDING
% NEGLIGENCE OR OTHERWISE) ARISING IN ANY WAY OUT OF THE USE OF THIS
% SOFTWARE, EVEN IF ADVISED OF THE POSSIBILITY OF SUCH DAMAGE.

% *********************************************************************
\head{chapter}{An Overview of \Language{}}{ugIntro}
% *********************************************************************

Welcome to the \Language{} environment for interactive computation
and problem solving.
Consider this chapter a brief, whirlwind tour of the \Language{}
world.
We introduce you to \Language{}'s graphics and the \Language{}
language.
Then we give a sampling of the large variety of facilities
in the \Language{} system, ranging from the various kinds of
numbers, to data types (like lists, arrays, and sets) and
mathematical objects (like matrices, integrals, and differential
equations).
We conclude with the discussion of system commands and an
interactive ``undo.''

Before embarking on the tour, we need to brief those readers
working interactively with \Language{} on some details.
Others can skip right immediately to
\spadref{ugIntroTypo}.

% *********************************************************************
\head{section}{Starting Up and Winding Down}{ugIntroStart}
% *********************************************************************
%

You need to know how to start the \Language{} system and how to stop it.
We assume that \Language{} has been correctly installed on your
machine (as described in another \Language{} document).

To begin using \Language{}, issue the command {\bf axiom} to the
\index{starting @{starting \Language{}}}
operating system shell.
\index{axiom @{\bf axiom}}
There is a brief pause, some start-up messages, and then one
or more windows appear.

If you are not running \Language{} under the X Window System, there is
only one window (the console).
At the lower left of the screen there is a prompt that
\index{prompt}
looks like
\begin{verbatim}
(1) ->
\end{verbatim}
%%--> do you want to talk about equation numbers on the right, etc.
When you want to enter input to \Language{}, you do so on the same line
after the prompt.
The ``1'' in ``(1)'' is the computation step number and is incremented
\index{step number}
after you enter \Language{} statements.
Note, however, that a system command
such as \spadsys{)clear all}
may change the step number in other ways.
We talk about step numbers more when we discuss system commands
and the workspace history facility.

If you are running \Language{} under the X Window System, there may be two
\index{X Window System}
windows: the console window (as just described) and the \HyperName{}
main menu.
\index{Hyper @{\HyperName{}}}
\HyperName{} is a multiple-window hypertext system that lets you
\index{window}
view \Language{} documentation and examples on-line,
execute \Language{} expressions, and generate graphics.
If you are in a graphical windowing environment,
it is usually started automatically when \Language{} begins.
If it is not running, issue \spadsys{)hd} to start it.
We discuss the basics of \HyperName{} in \chapref{ugHyper}.
\syscmdindex{hd}

To interrupt an \Language{} computation, hold down the
\index{interrupt}
\fbox{\bf Ctrl} (control) key and press
\fbox{\bf c}.
This  brings you back to the \Language{} prompt.

\beginImportant
To exit from \Language{},  move to the console window,
\index{stopping @{stopping \Language{}}}
type \spadsys{)quit}
\index{exiting @{exiting \Language{}}}
at the input prompt and press the \fbox{\bf Enter} key.
\syscmdindex{quit}
You will probably be prompted with the following message:
\begin{center}
Please enter {\bf y} or {\bf yes} if you really want to leave the \\
interactive environment and return to the operating system
\end{center}
You should respond {\bf yes}, for example, to exit \Language{}.
\endImportant

We are purposely vague in describing exactly what your screen
looks like or what messages \Language{} displays.
\Language{} runs on a number of different machines, operating
systems and window environments, and these differences all affect
the physical look of the system.
You can also change the way that \Language{} behaves via
\spadgloss{system commands} described later in this chapter and in
\appxref{ugSysCmd}.
System commands are special commands, like \spadsys{)set}, that begin
with a closing parenthesis and are used to change your
environment.
For example, you can set a system variable so that you are not
prompted for confirmation when you want to leave \Language{}.

% *********************************************************************
\head{subsection}{\Clef{}}{ugAvailCLEF}
% *********************************************************************
%
If you are using \Language{} under the X Window System, the
\index{Clef@{\Clef{}}}
\index{command line editor}
\Clef{} command line editor is probably available and installed.
With this editor you can recall previous lines with the up and
down arrow keys.

To move forward and backward on a line, use the right and
left arrows.
You can use the
\fbox{\bf Insert}
key to toggle insert mode on or off.
When you are in insert mode,
the cursor appears as a large block and if you type
anything, the characters are inserted into the line without
deleting the previous ones.

If you press the
\fbox{\bf Home}
key, the cursor moves to the beginning of the line and if you press the
\fbox{\bf End}
key, the cursor moves to the end of the line.
Pressing
\fbox{\bf Ctrl}--\fbox{\bf End}
deletes all the text from the cursor to the end of the line.

\Clef{} also provides \Language{} operation name completion for
\index{operation name completion}
a limited set of operations.
If you enter a few letters and then press the
\fbox{\bf Tab} key,
\Clef{} tries to use those letters as the prefix of an \Language{}
operation name.
If a name appears and it is not what you want, press
\fbox{\bf Tab} again
to see another name.

You are ready to begin your journey into the world of \Language{}.
Proceed to the first stop.

% *********************************************************************
\head{section}{Typographic Conventions}{ugIntroTypo}
% *********************************************************************

In this book we have followed these typographical conventions:
\begin{itemize}
%
\item Categories, domains and packages are displayed in
a sans-serif typeface:
\spadtype{Ring}, \spadtype{Integer}, \spadtype{DiophantineSolutionPackage}.
%
\item Prefix operators, infix operators, and punctuation symbols in the \Language{}
language are displayed in the text like this:
\spadop{+}, \spadSyntax{$}, \spadSyntax{+->}.
%
\item \Language{} expressions or expression fragments are displayed in
a mon\-o\-space typeface:
\spad{inc(x) == x + 1}.
%
\item For clarity of presentation, \TeX{} is often
used to format expressions: $g(x)=x^2+1.$
%
\item Function names and \HyperName{} button names
are displayed in the text in
a bold typeface:
\spadfun{factor}, \spadfun{integrate},  {\bf Lighting}.
%
\item Italics are used for emphasis and for words defined in the
glossary: \spadgloss{category}.
\end{itemize}

This book contains over 2500 examples of \Language{} input and output.
All examples were run though \Language{} and their output was
created in \TeX{} form for this book by the \Language{}
\spadtype{TexFormat} package.
We have deleted system messages from the example output if those
messages are not important for the discussions in which the examples
appear.

% *********************************************************************
\head{section}{The \Language{} Language}{ugIntroExpressions}
% *********************************************************************
%

The \Language{} language is a rich language for performing
interactive computations and for building components of the
\Language{} library.
Here we present only some basic aspects of the language that you
need to know for the rest of this chapter.
Our discussion here is intentionally informal, with details
unveiled on an ``as needed'' basis.
For more information on a particular construct, we suggest you
consult the index at the back of the book.

% *********************************************************************
\head{subsection}{Arithmetic Expressions}{ugIntroArithmetic}
% *********************************************************************

For arithmetic expressions, use the \spadop{+} and \spadop{-}
\spadglossSee{operators}{operator} as in mathematics.
Use \spadop{*} for multiplication, and \spadop{^} for
exponentiation.
To create a fraction, use \spadop{/}.
When an expression contains several operators, those of highest
\spadgloss{precedence} are evaluated first.
For arithmetic operators, \spadop{^} has highest precedence,
\spadop{*} and \spadop{/} have the next highest
precedence, and \spadop{+} and \spadop{-} have the lowest
precedence.

\begin{xtc}
\begin{xtccomment}
\Language{} puts implicit parentheses around operations of higher
precedence, and groups those of equal precedence from left to right.
\end{xtccomment}
\begin{spadsrc}
1 + 2 - 3 / 4 * 3 ^ 2 - 1
\end{spadsrc}
\begin{TeXOutput}
\begin{fricasmath}{1}
-{\frac{19}{4}}%
\end{fricasmath}
\end{TeXOutput}
\formatResultType{Fraction(Integer)}
\end{xtc}
\begin{xtc}
\begin{xtccomment}
The above expression is equivalent to this.
\end{xtccomment}
\begin{spadsrc}
((1 + 2) - ((3 / 4) * (3 ^ 2))) - 1
\end{spadsrc}
\begin{TeXOutput}
\begin{fricasmath}{2}
-{\frac{19}{4}}%
\end{fricasmath}
\end{TeXOutput}
\formatResultType{Fraction(Integer)}
\end{xtc}
\begin{xtc}
\begin{xtccomment}
If an expression contains subexpressions enclosed in parentheses,
the parenthesized subexpressions are evaluated first (from left to
right, from inside out).
\end{xtccomment}
\begin{spadsrc}
1 + 2 - 3/ (4 * 3 ^ (2 - 1))
\end{spadsrc}
\begin{TeXOutput}
\begin{fricasmath}{3}
\frac{11}{4}%
\end{fricasmath}
\end{TeXOutput}
\formatResultType{Fraction(Integer)}
\end{xtc}

% *********************************************************************
\head{subsection}{Previous Results}{ugIntroPrevious}
% *********************************************************************

Use the percent sign (\spadSyntax{%}) to refer to the last
result.
\index{result!previous}
Also, use \spadSyntax to refer to previous results.
\index{percentpercent@{\%\%}}
\spad{%%(-1)} is equivalent to \spadSyntax{%},
\spad{%%(-2)} returns the next to the last result, and so on.
\spad{%%(1)} returns the result from step number 1,
\spad{%%(2)} returns the result from step number 2, and so on.
\spad{%%(0)} is not defined.

\begin{xtc}
\begin{xtccomment}
This is ten to the tenth power.
\end{xtccomment}
\begin{spadsrc}
10 ^ 10 
\end{spadsrc}
\begin{TeXOutput}
\begin{fricasmath}{1}
10000000000%
\end{fricasmath}
\end{TeXOutput}
\formatResultType{PositiveInteger}
\end{xtc}
\begin{xtc}
\begin{xtccomment}
This is the last result minus one.
\end{xtccomment}
\begin{spadsrc}
% - 1 
\end{spadsrc}
\begin{TeXOutput}
\begin{fricasmath}{2}
9999999999%
\end{fricasmath}
\end{TeXOutput}
\formatResultType{PositiveInteger}
\end{xtc}
\begin{xtc}
\begin{xtccomment}
This is the last result.
\end{xtccomment}
\begin{spadsrc}
%%(-1) 
\end{spadsrc}
\begin{TeXOutput}
\begin{fricasmath}{3}
9999999999%
\end{fricasmath}
\end{TeXOutput}
\formatResultType{PositiveInteger}
\end{xtc}
\begin{xtc}
\begin{xtccomment}
This is the result from step number 1.
\end{xtccomment}
\begin{spadsrc}
%%(1) 
\end{spadsrc}
\begin{TeXOutput}
\begin{fricasmath}{4}
10000000000%
\end{fricasmath}
\end{TeXOutput}
\formatResultType{PositiveInteger}
\end{xtc}

% *********************************************************************
\head{subsection}{Some Types}{ugIntroTypes}
% *********************************************************************

Everything in \Language{} has a type.
The type determines what operations you can perform on an object and
how the object can be used.
An entire chapter of this book (\chapref{ugTypes}) is dedicated to
the interactive use of types.
Several of the final chapters discuss how types are built and how
they are organized in the \Language{} library.

\begin{xtc}
\begin{xtccomment}
Positive integers are given type \spadtype{PositiveInteger}.
\end{xtccomment}
\begin{spadsrc}
8
\end{spadsrc}
\begin{TeXOutput}
\begin{fricasmath}{1}
8%
\end{fricasmath}
\end{TeXOutput}
\formatResultType{PositiveInteger}
\end{xtc}
\begin{xtc}
\begin{xtccomment}
Negative ones are given type \spadtype{Integer}.
This fine distinction is helpful to the
\Language{} interpreter.
\end{xtccomment}
\begin{spadsrc}
-8
\end{spadsrc}
\begin{TeXOutput}
\begin{fricasmath}{2}
-{8}%
\end{fricasmath}
\end{TeXOutput}
\formatResultType{Integer}
\end{xtc}
\begin{xtc}
\begin{xtccomment}
Here a positive integer exponent gives a polynomial result.
\end{xtccomment}
\begin{spadsrc}
x^8
\end{spadsrc}
\begin{TeXOutput}
\begin{fricasmath}{3}
\SUPER{\SYMBOL{x}}{8}%
\end{fricasmath}
\end{TeXOutput}
\formatResultType{Polynomial(Integer)}
\end{xtc}
\begin{xtc}
\begin{xtccomment}
Here a negative integer exponent produces a fraction.
\end{xtccomment}
\begin{spadsrc}
x^(-8)
\end{spadsrc}
\begin{TeXOutput}
\begin{fricasmath}{4}
\frac{1}{\SUPER{\SYMBOL{x}}{8}}%
\end{fricasmath}
\end{TeXOutput}
\formatResultType{Fraction(Polynomial(Integer))}
\end{xtc}

% *********************************************************************
\head{subsection}{Symbols, Variables, Assignments, and Declarations}{ugIntroAssign}
% *********************************************************************

A \spadgloss{symbol} is a literal used for the input of things like
the ``variables'' in polynomials and power series.

\begin{xtc}
\begin{xtccomment}
We use the three symbols \spad{x}, \spad{y}, and \spad{z} in
entering this polynomial.
\end{xtccomment}
\begin{spadsrc}
(x - y*z)^2
\end{spadsrc}
\begin{TeXOutput}
\begin{fricasmath}{1}
\SUPER{\SYMBOL{y}}{2}\TIMES \SUPER{\SYMBOL{z}}{2}-{2\TIMES \SYMBOL{x}\TIMES %
\SYMBOL{y}\TIMES \SYMBOL{z}}+\SUPER{\SYMBOL{x}}{2}%
\end{fricasmath}
\end{TeXOutput}
\formatResultType{Polynomial(Integer)}
\end{xtc}
A symbol has a name beginning with an uppercase or lowercase alphabetic
\index{symbol!naming}
character, \spadSyntax{%}, or \spadSyntax{!}.
Successive characters (if any) can be any of the above, digits, or
\spadSyntax{?}.
Case is distinguished: the symbol \spad{points} is different
from the symbol \spad{Points}.

A symbol can also be used in \Language{} as a \spadgloss{variable}.
A variable refers to a value.
To \spadglossSee{assign}{assignment}
a value to a variable,
\index{variable!naming}
the operator \spadSyntax{:=}
\index{assignment}
is used.\footnote{\Language{} actually has two forms of assignment:
{\it immediate} assignment, as discussed here,
and {\it delayed assignment}. See \spadref{ugLangAssign} for details.}
A variable initially has no restrictions on the kinds of
\index{declaration}
values to which it can refer.

\begin{xtc}
\begin{xtccomment}
This assignment gives the value \spad{4} (an integer) to
a variable named \spad{x}.
\end{xtccomment}
\begin{spadsrc}
x := 4
\end{spadsrc}
\begin{TeXOutput}
\begin{fricasmath}{2}
4%
\end{fricasmath}
\end{TeXOutput}
\formatResultType{PositiveInteger}
\end{xtc}
\begin{xtc}
\begin{xtccomment}
This gives the value \spad{z + 3/5} (a polynomial)  to \spad{x}.
\end{xtccomment}
\begin{spadsrc}
x := z + 3/5
\end{spadsrc}
\begin{TeXOutput}
\begin{fricasmath}{3}
\SYMBOL{z}+\frac{3}{5}%
\end{fricasmath}
\end{TeXOutput}
\formatResultType{Polynomial(Fraction(Integer))}
\end{xtc}
\begin{xtc}
\begin{xtccomment}
To restrict the types of objects that can be assigned to a variable,
use a \spadgloss{declaration}
\end{xtccomment}
\begin{spadsrc}
y : Integer 
\end{spadsrc}
\end{xtc}
\begin{xtc}
\begin{xtccomment}
After a variable is declared to be of some type, only values
of that type can be assigned to that variable.
\end{xtccomment}
\begin{spadsrc}
y := 89
\end{spadsrc}
\begin{TeXOutput}
\begin{fricasmath}{5}
89%
\end{fricasmath}
\end{TeXOutput}
\formatResultType{Integer}
\end{xtc}
\begin{xtc}
\begin{xtccomment}
The declaration for \spad{y} forces values assigned to \spad{y} to
be converted to integer values.
\end{xtccomment}
\begin{spadsrc}
y := sin %pi
\end{spadsrc}
\begin{TeXOutput}
\begin{fricasmath}{6}
0%
\end{fricasmath}
\end{TeXOutput}
\formatResultType{Integer}
\end{xtc}
\begin{xtc}
\begin{xtccomment}
If no such conversion is possible,
\Language{} refuses to assign a value to \spad{y}.
\end{xtccomment}
\begin{spadsrc}
y := 2/3
\end{spadsrc}
\begin{MessageOutput}
   Cannot convert right-hand side of assignment
   2
   -
   3

      to an object of the type Integer of the left-hand side.
\end{MessageOutput}
\end{xtc}
\begin{xtc}
\begin{xtccomment}
A type declaration can also be given together with an assignment.
The declaration can assist \Language{} in choosing the correct
operations to apply.
\end{xtccomment}
\begin{spadsrc}
f : Float := 2/3
\end{spadsrc}
\begin{TeXOutput}
\begin{fricasmath}{7}
\STRING{0.66666666666666666667}%
\end{fricasmath}
\end{TeXOutput}
\formatResultType{Float}
\end{xtc}

Any number of expressions can be given on input line.
Just separate them by semicolons.
Only the result of evaluating the last expression is displayed.

\begin{xtc}
\begin{xtccomment}
These two expressions have the same effect as
the previous single expression.
\end{xtccomment}
\begin{spadsrc}
f : Float; f := 2/3 
\end{spadsrc}
\begin{TeXOutput}
\begin{fricasmath}{8}
\STRING{0.66666666666666666667}%
\end{fricasmath}
\end{TeXOutput}
\formatResultType{Float}
\end{xtc}

The type of a symbol is either \spadtype{Symbol}
\exptypeindex{Symbol}
or \spadtype{Variable}({\it name}) where {\it name} is the name
of the symbol.

\begin{xtc}
\begin{xtccomment}
By default, the interpreter
\exptypeindex{Variable}
gives this symbol the type \spadtype{Variable(q)}.
\end{xtccomment}
\begin{spadsrc}
q
\end{spadsrc}
\begin{TeXOutput}
\begin{fricasmath}{9}
\SYMBOL{q}%
\end{fricasmath}
\end{TeXOutput}
\formatResultType{Variable(q)}
\end{xtc}
\begin{xtc}
\begin{xtccomment}
When multiple symbols are involved, \spadtype{Symbol} is used.
\end{xtccomment}
\begin{spadsrc}
[q, r]
\end{spadsrc}
\begin{TeXOutput}
\begin{fricasmath}{10}
\BRACKET{\SYMBOL{q}\COMMA \SYMBOL{r}}%
\end{fricasmath}
\end{TeXOutput}
\formatResultType{List(OrderedVariableList([q, r]))}
\end{xtc}

\begin{xtc}
\begin{xtccomment}
What happens when you try to use a symbol that is the name of a variable?
\end{xtccomment}
\begin{spadsrc}
f 
\end{spadsrc}
\begin{TeXOutput}
\begin{fricasmath}{11}
\STRING{0.66666666666666666667}%
\end{fricasmath}
\end{TeXOutput}
\formatResultType{Float}
\end{xtc}
\begin{xtc}
\begin{xtccomment}
Use a single quote (\spadSyntax{'}) before
\index{quote}
the name to get the symbol.
\end{xtccomment}
\begin{spadsrc}
'f
\end{spadsrc}
\begin{TeXOutput}
\begin{fricasmath}{12}
\SYMBOL{f}%
\end{fricasmath}
\end{TeXOutput}
\formatResultType{Variable(f)}
\end{xtc}

Quoting a name creates a symbol by
preventing evaluation of the name as a variable.
Experience will teach you when you are most likely going to need to use
a quote.
We try to point out the location of such trouble spots.

% *********************************************************************
\head{subsection}{Conversion}{ugIntroConversion}
% *********************************************************************

Objects of one type can usually be ``converted'' to objects of several
other types.
To \spadglossSee{convert}{conversion}
an object to a new type, use the \spadSyntax{::} infix
operator.\footnote{Conversion is discussed in detail in \spadref{ugTypesConvert}.}
For example, to display an object, it is necessary to
convert the object to type \spadtype{OutputForm}.

\begin{xtc}
\begin{xtccomment}
This produces a polynomial with rational number coefficients.
\end{xtccomment}
\begin{spadsrc}
p := r^2 + 2/3 
\end{spadsrc}
\begin{TeXOutput}
\begin{fricasmath}{1}
\SUPER{\SYMBOL{r}}{2}+\frac{2}{3}%
\end{fricasmath}
\end{TeXOutput}
\formatResultType{Polynomial(Fraction(Integer))}
\end{xtc}
\begin{xtc}
\begin{xtccomment}
Create a quotient of polynomials with integer coefficients
by using \spadSyntax{::}.
\end{xtccomment}
\begin{spadsrc}
p :: Fraction Polynomial Integer 
\end{spadsrc}
\begin{TeXOutput}
\begin{fricasmath}{2}
\frac{3\TIMES \SUPER{\SYMBOL{r}}{2}+2}{3}%
\end{fricasmath}
\end{TeXOutput}
\formatResultType{Fraction(Polynomial(Integer))}
\end{xtc}

Some conversions can be performed automatically when
\Language{} tries to evaluate your input.
Others conversions must be explicitly requested.

% *********************************************************************
\head{subsection}{Calling Functions}{ugIntroCallFun}
% *********************************************************************

As we saw earlier, when you want to add or subtract two values,
you place the arithmetic operator \spadop{+}
or \spadop{-} between the two
\spadglossSee{arguments}{argument} denoting the values.
To use most other \Language{} operations, however, you use another syntax:
\index{function!calling}
write the name
of the operation first, then an open parenthesis, then each of the
arguments separated by commas, and, finally, a closing parenthesis.
If the operation takes only one argument and the argument is a number
or a symbol, you can omit the parentheses.

\begin{xtc}
\begin{xtccomment}
This calls the operation \spadfun{factor} with the single
integer argument \spad{120}.
\end{xtccomment}
\begin{spadsrc}
factor(120)
\end{spadsrc}
\begin{TeXOutput}
\begin{fricasmath}{1}
\SUPER{2}{3}\TIMES 3\TIMES 5%
\end{fricasmath}
\end{TeXOutput}
\formatResultType{Factored(Integer)}
\end{xtc}
\begin{xtc}
\begin{xtccomment}
This is a call to \spadfun{divide} with the two integer arguments
\spad{125} and \spad{7}.
\end{xtccomment}
\begin{spadsrc}
divide(125,7)
\end{spadsrc}
\begin{TeXOutput}
\begin{fricasmath}{2}
\BRACKET{\SYMBOL{quotient}=17\COMMA \SYMBOL{remainder}=6}%
\end{fricasmath}
\end{TeXOutput}
\formatResultType{Record(quotient: Integer, remainder: Integer)}
\end{xtc}
\begin{xtc}
\begin{xtccomment}
This calls \spadfun{quatern} with four floating-point arguments.
\end{xtccomment}
\begin{spadsrc}
quatern(3.4,5.6,2.9,0.1)
\end{spadsrc}
\begin{TeXOutput}
\begin{fricasmath}{3}
\STRING{3.4}+\STRING{5.6}\TIMES \SYMBOL{i}+\STRING{2.9}\TIMES \SYMBOL{j}+%
\STRING{0.1}\TIMES \SYMBOL{k}%
\end{fricasmath}
\end{TeXOutput}
\formatResultType{Quaternion(Float)}
\end{xtc}
\begin{xtc}
\begin{xtccomment}
This is the same as \spad{factorial(10)}.
\end{xtccomment}
\begin{spadsrc}
factorial 10
\end{spadsrc}
\begin{TeXOutput}
\begin{fricasmath}{4}
3628800%
\end{fricasmath}
\end{TeXOutput}
\formatResultType{PositiveInteger}
\end{xtc}

An operations that returns a \spadtype{Boolean} value (that is,
\spad{true} or \spad{false}) frequently has a name suffixed with
a question mark (``?'').  For example, the \spadfun{even?}
operation returns \spad{true} if its integer argument is an even
number, \spad{false} otherwise.

An operation that can be destructive on one or more arguments
usually has a name ending in a exclamation point (``!'').
This actually means that it is {\it allowed} to update its
arguments but it is not {\it required} to do so. For example,
the underlying representation of a collection type may not allow
the very last element to removed and so an empty object may be
returned instead. Therefore, it is important that you use the
object returned by the operation and not rely on a physical
change having occurred within the object. Usually, destructive
operations are provided for efficiency reasons.

% *********************************************************************
\head{subsection}{Some Predefined Macros}{ugIntroMacros}
% *********************************************************************

\Language{} provides several \spadglossSee{macros}{macro}
for your convenience.\footnote{See \spadref{ugUserMacros}
for a discussion on how to write your own macros.}
Macros are names
\index{macro!predefined}
(or forms) that expand to larger expressions for commonly used values.

\begin{center}
\begin{tabular}{ll}
\spadgloss{\%i}             &  The square root of -1. \\
\spadgloss{\%e}             &  The base of the natural logarithm. \\
\spadgloss{\%pi}            &  $\pi$. \\
\spadgloss{\%infinity}      &  $\infty$. \\
\spadgloss{\%plusInfinity}  &  $+\infty$. \\
\spadgloss{\%minusInfinity} &  $-\infty$.
\end{tabular}
\end{center}
\index{\%i}
\index{\%e}
\index{\%pi}
\index{pi@{$\pi$ (= \%pi)}}
\index{\%infinity}
\index{infinity@{$\infty$ (= \%infinity)}}
\index{\%plusInfinity}
\index{\%minusInfinity}

%To display all the macros (along with anything you have
%defined in the workspace), issue the system command \spadsys{)display all}.

% *********************************************************************
\head{subsection}{Long Lines}{ugIntroLong}
% *********************************************************************

When you enter \Language{} expressions from your keyboard, there
will be times when they are too long to fit on one line.
\Language{} does not care how long your lines are, so you can let
them continue from the right margin to the left side of the
next line.

Alternatively, you may want to enter several shorter lines and
have \Language{} glue them together.
To get this glue, put an underscore (\_) at the end of
each line you wish to continue.
\begin{verbatim}
2_
+_
3
\end{verbatim}
is the same as if you had entered
\begin{verbatim}
2+3
\end{verbatim}

If you are putting your \Language{} statements in an input file
(see \spadref{ugInOutIn}),
you can use indentation to indicate the structure of your program.
(see \spadref{ugLangBlocks}).

% *********************************************************************
\head{subsection}{Comments}{ugIntroComments}
% *********************************************************************

Comment statements begin with two consecutive hyphens or two
consecutive plus signs and continue until the end of the line.

\begin{xtc}
\begin{xtccomment}
The comment beginning with {\tt --} is ignored by \Language{}.
\end{xtccomment}
\begin{spadsrc}
2 + 3   -- this is rather simple, no?
\end{spadsrc}
\begin{TeXOutput}
\begin{fricasmath}{1}
5%
\end{fricasmath}
\end{TeXOutput}
\formatResultType{PositiveInteger}
\end{xtc}

There is no way to write long multi-line comments
other than starting each line with \spadSyntax{--} or
\spadSyntax{++}.

% *********************************************************************
\head{section}{Graphics}{ugIntroGraphics}
% *********************************************************************
%

\Language{} has a two- and three-dimensional drawing and rendering
\index{graphics}
package that allows you to draw, shade, color, rotate, translate, map,
clip, scale and combine graphic output of \Language{} computations.
The graphics interface is capable of plotting functions of one or more
variables and plotting parametric surfaces.
Once the graphics figure appears in a window,
move your mouse to the window and click.
A control panel appears immediately  and allows you to
interactively transform the object.

\begin{psXtc}
\begin{xtccomment}
This is an example of \Language{}'s two-dimensional plotting.
From the 2D Control Panel you can rescale the plot, turn axes and units
on and off and save the image, among other things.
This PostScript image was produced by clicking on the
\fbox{\bf PS} 2D Control Panel button.
\end{xtccomment}
\begin{spadsrc}
draw(cos(5*t/8), t=0..16*%pi, coordinates==polar)
\end{spadsrc}
% window was 256 x 256
\epsffile[72 72 300 300]{rose-1.ps}
\end{psXtc}

\begin{psXtc}
\begin{xtccomment}
This is an example of \Language{}'s three-dimensional plotting.
It is a monochrome graph of the complex arctangent
function.
The image displayed was rotated and had the ``shade'' and ``outline''
display options set from the 3D Control Panel.
The PostScript output was produced by clicking on the
\fbox{\bf save} 3D Control Panel button and then
clicking on the \fbox{\bf PS} button.
See \spadref{ugProblemNumeric} for more details and examples
of \Language{}'s numeric and graphics capabilities.
\end{xtccomment}
\begin{spadsrc}
draw((x,y) +-> real atan complex(x,y), -%pi..%pi, -%pi..%pi, colorFunction == (x,y) +-> argument atan complex(x,y))
\end{spadsrc}
% window was 256 x 256
\epsffile[72 72 285 285]{atan-1.ps}
\end{psXtc}

An exhibit of \Gallery{} is given in the
center section of this book.
For a description of the commands and programs that
produced these figures, see \appxref{ugAppGraphics}.
PostScript
\index{PostScript}
output is available so that \Language{} images can be
printed.\footnote{PostScript is a trademark of Adobe Systems Incorporated,
registered in the United States.}
See \chapref{ugGraph} for more examples and details about using
\Language{}'s graphics facilities.

% *********************************************************************
\head{section}{Numbers}{ugIntroNumbers}
% *********************************************************************
%

\Language{} distinguishes very carefully between different kinds
of numbers, how they are represented and what their properties
are.
Here are a sampling of some of these kinds of numbers and some
things you can do with them.

\begin{xtc}
\begin{xtccomment}
Integer arithmetic is always exact.
\end{xtccomment}
\begin{spadsrc}
11^13 * 13^11 * 17^7 - 19^5 * 23^3
\end{spadsrc}
\begin{TeXOutput}
\begin{fricasmath}{1}
25387751112538918594 666224484237298%
\end{fricasmath}
\end{TeXOutput}
\formatResultType{PositiveInteger}
\end{xtc}
\begin{xtc}
\begin{xtccomment}
Integers can be represented in factored form.
\end{xtccomment}
\begin{spadsrc}
factor 643238070748569023720594412551704344145570763243 
\end{spadsrc}
\begin{TeXOutput}
\begin{fricasmath}{2}
\SUPER{11}{13}\TIMES \SUPER{13}{11}\TIMES \SUPER{17}{7}\TIMES \SUPER{19}{5}%
\TIMES \SUPER{23}{3}\TIMES \SUPER{29}{2}%
\end{fricasmath}
\end{TeXOutput}
\formatResultType{Factored(Integer)}
\end{xtc}
\begin{xtc}
\begin{xtccomment}
Results stay factored when you do arithmetic.
Note that the \spad{12} is automatically factored for you.
\end{xtccomment}
\begin{spadsrc}
% * 12 
\end{spadsrc}
\begin{TeXOutput}
\begin{fricasmath}{3}
\SUPER{2}{2}\TIMES 3\TIMES \SUPER{11}{13}\TIMES \SUPER{13}{11}\TIMES \SUPER{%
17}{7}\TIMES \SUPER{19}{5}\TIMES \SUPER{23}{3}\TIMES \SUPER{29}{2}%
\end{fricasmath}
\end{TeXOutput}
\formatResultType{Factored(Integer)}
\end{xtc}
\index{radix}
\begin{xtc}
\begin{xtccomment}
Integers can also be displayed to bases other than 10.
This is an integer in base 11.
\end{xtccomment}
\begin{spadsrc}
radix(25937424601,11)
\end{spadsrc}
\begin{TeXOutput}
\begin{fricasmath}{4}
10000000000%
\end{fricasmath}
\end{TeXOutput}
\formatResultType{RadixExpansion(11)}
\end{xtc}
\begin{xtc}
\begin{xtccomment}
Roman numerals are also available for those special occasions.
\index{Roman numerals}
\end{xtccomment}
\begin{spadsrc}
roman(1992)
\end{spadsrc}
\begin{TeXOutput}
\begin{fricasmath}{5}
\SYMBOL{MCMXCII}%
\end{fricasmath}
\end{TeXOutput}
\formatResultType{RomanNumeral}
\end{xtc}
\begin{xtc}
\begin{xtccomment}
Rational number arithmetic is also exact.
\end{xtccomment}
\begin{spadsrc}
r := 10 + 9/2 + 8/3 + 7/4 + 6/5 + 5/6 + 4/7 + 3/8 + 2/9
\end{spadsrc}
\begin{TeXOutput}
\begin{fricasmath}{6}
\frac{55739}{2520}%
\end{fricasmath}
\end{TeXOutput}
\formatResultType{Fraction(Integer)}
\end{xtc}
\begin{xtc}
\begin{xtccomment}
To factor fractions, you have to
map \spadfun{factor} onto the numerator and denominator.
\end{xtccomment}
\begin{spadsrc}
map(factor,r) 
\end{spadsrc}
\begin{TeXOutput}
\begin{fricasmath}{7}
\frac{139\TIMES 401}{\SUPER{2}{3}\TIMES \SUPER{3}{2}\TIMES 5\TIMES 7}%
\end{fricasmath}
\end{TeXOutput}
\formatResultType{Fraction(Factored(Integer))}
\end{xtc}
\begin{xtc}
\begin{xtccomment}
Type \spadtype{SingleInteger} refers to machine word-length
integers.
\exptypeindex{SingleInteger}
In English, this expression means ``\spad{11} as a small
integer''.
\end{xtccomment}
\begin{spadsrc}
11@SingleInteger
\end{spadsrc}
\begin{TeXOutput}
\begin{fricasmath}{8}
11%
\end{fricasmath}
\end{TeXOutput}
\formatResultType{SingleInteger}
\end{xtc}
\begin{xtc}
\begin{xtccomment}
Machine double-precision floating-point numbers are also
available for numeric and graphical
applications.
\exptypeindex{DoubleFloat}
\end{xtccomment}
\begin{spadsrc}
123.21@DoubleFloat
\end{spadsrc}
\begin{TeXOutput}
\begin{fricasmath}{9}
\STRING{123.21000000000001}%
\end{fricasmath}
\end{TeXOutput}
\formatResultType{DoubleFloat}
\end{xtc}

The normal floating-point type in \Language{}, \spadtype{Float},
is a software implementation of floating-point numbers in which
the exponent and the mantissa may have any number of
digits.\footnote{See \xmpref{Float} and \xmpref{DoubleFloat} for
additional information on floating-point types.}
The types \spadtype{Complex(Float)} and
\spadtype{Complex(DoubleFloat)} are the corresponding software
implementations of complex floating-point numbers.

\begin{xtc}
\begin{xtccomment}
This is a floating-point approximation to about twenty digits.
\index{floating point}
The \spadSyntax{::}
is used here to change from one kind of object
(here, a rational number) to another (a floating-point number).
\end{xtccomment}
\begin{spadsrc}
r :: Float 
\end{spadsrc}
\begin{TeXOutput}
\begin{fricasmath}{10}
\STRING{22.118650793650793651}%
\end{fricasmath}
\end{TeXOutput}
\formatResultType{Float}
\end{xtc}
\begin{xtc}
\begin{xtccomment}
Use \spadfunFrom{digits}{Float} to change the number of digits in
the representation.
This operation returns the previous value so you can reset it
later.
\end{xtccomment}
\begin{spadsrc}
digits(22) 
\end{spadsrc}
\begin{TeXOutput}
\begin{fricasmath}{11}
20%
\end{fricasmath}
\end{TeXOutput}
\formatResultType{PositiveInteger}
\end{xtc}
\begin{xtc}
\begin{xtccomment}
To \spad{22} digits of precision, the number
$e^{\pi {\sqrt {163.0}}}$
appears to be an integer.
\end{xtccomment}
\begin{spadsrc}
exp(%pi * sqrt 163.0) 
\end{spadsrc}
\begin{TeXOutput}
\begin{fricasmath}{12}
\STRING{262537412640768744.0}%
\end{fricasmath}
\end{TeXOutput}
\formatResultType{Float}
\end{xtc}
\begin{xtc}
\begin{xtccomment}
Increase the precision to forty digits and try again.
\end{xtccomment}
\begin{spadsrc}
digits(40);  exp(%pi * sqrt 163.0) 
\end{spadsrc}
\begin{TeXOutput}
\begin{fricasmath}{13}
\STRING{262537412640768743.9999999999992500725976}%
\end{fricasmath}
\end{TeXOutput}
\formatResultType{Float}
\end{xtc}
\begin{xtc}
\begin{xtccomment}
Here are complex numbers with rational numbers as real and
\index{complex numbers}
imaginary parts.
\end{xtccomment}
\begin{spadsrc}
(2/3 + %i)^3 
\end{spadsrc}
\begin{TeXOutput}
\begin{fricasmath}{14}
-{\frac{46}{27}}+\frac{1}{3}\TIMES \ImaginaryI %
\end{fricasmath}
\end{TeXOutput}
\formatResultType{Complex(Fraction(Integer))}
\end{xtc}
\begin{xtc}
\begin{xtccomment}
The standard operations on complex numbers are available.
\end{xtccomment}
\begin{spadsrc}
conjugate % 
\end{spadsrc}
\begin{TeXOutput}
\begin{fricasmath}{15}
-{\frac{46}{27}}-{\frac{1}{3}\TIMES \ImaginaryI }%
\end{fricasmath}
\end{TeXOutput}
\formatResultType{Complex(Fraction(Integer))}
\end{xtc}
\begin{xtc}
\begin{xtccomment}
You can factor complex integers.
\end{xtccomment}
\begin{spadsrc}
factor(89 - 23 * %i)
\end{spadsrc}
\begin{TeXOutput}
\begin{fricasmath}{16}
-{\PAREN{1+\ImaginaryI }\TIMES \SUPER{\PAREN{2+\ImaginaryI }}{2}\TIMES \SUPER%
{\PAREN{3+2\TIMES \ImaginaryI }}{2}}%
\end{fricasmath}
\end{TeXOutput}
\formatResultType{Factored(Complex(Integer))}
\end{xtc}
\begin{xtc}
\begin{xtccomment}
Complex numbers with floating point parts are also available.
\end{xtccomment}
\begin{spadsrc}
exp(%pi/4.0 * %i)
\end{spadsrc}
\begin{TeXOutput}
\begin{fricasmath}{17}
\STRING{0.7071067811865475244008443621048490392849}+\STRING{%
0.7071067811865475244008443621048490392848}\TIMES \ImaginaryI %
\end{fricasmath}
\end{TeXOutput}
\formatResultType{Complex(Float)}
\end{xtc}
%%--> These are not numbers:
%\xtc{
%The real and imaginary parts can be symbolic.
%}{
%\spadcommand{complex(u,v) \bound{cuv}}
%}
%\xtc{
%Of course, you can do complex arithmetic with these also.
%See \xmpref{Complex} for more information.
%}{
%\spadcommand{\% ^ 2 \free{cuv}}
%}
\begin{xtc}
\begin{xtccomment}
Every rational number has an exact representation as a
repeating decimal expansion
(see \xmpref{DecimalExpansion}).
\end{xtccomment}
\begin{spadsrc}
decimal(1/352)
\end{spadsrc}
\begin{TeXOutput}
\begin{fricasmath}{18}
0\STRING{.}00284\overline{09}%
\end{fricasmath}
\end{TeXOutput}
\formatResultType{DecimalExpansion}
\end{xtc}
\begin{xtc}
\begin{xtccomment}
A rational number can also be expressed as a continued fraction (see
\index{continued fraction}
\xmpref{ContinuedFraction}).
\index{fraction!continued}
\end{xtccomment}
\begin{spadsrc}
continuedFraction(6543/210)
\end{spadsrc}
\begin{TeXOutput}
\begin{fricasmath}{19}
31+\ZAG{1}{6}+\ZAG{1}{2}+\ZAG{1}{1}+\ZAG{1}{3}%
\end{fricasmath}
\end{TeXOutput}
\formatResultType{ContinuedFraction(Integer)}
\end{xtc}
\begin{xtc}
\begin{xtccomment}
Also, partial fractions can be used and can be displayed in a
\index{partial fraction}
compact \ldots
\index{fraction!partial}
\end{xtccomment}
\begin{spadsrc}
partialFraction(1,factorial(10)) 
\end{spadsrc}
\begin{TeXOutput}
\begin{fricasmath}{20}
\frac{159}{\SUPER{2}{8}}-{\frac{23}{\SUPER{3}{4}}}-{\frac{12}{\SUPER{5}{2}}}+%
\frac{1}{7}%
\end{fricasmath}
\end{TeXOutput}
\formatResultType{PartialFraction(Integer)}
\end{xtc}
\begin{xtc}
\begin{xtccomment}
or expanded format (see \xmpref{PartialFraction}).
\end{xtccomment}
\begin{spadsrc}
padicFraction(%) 
\end{spadsrc}
\begin{TeXOutput}
\begin{fricasmath}{21}
\frac{1}{2}+\frac{1}{\SUPER{2}{4}}+\frac{1}{\SUPER{2}{5}}+\frac{1}{\SUPER{2}{%
6}}+\frac{1}{\SUPER{2}{7}}+\frac{1}{\SUPER{2}{8}}-{\frac{2}{\SUPER{3}{2}}}-{%
\frac{1}{\SUPER{3}{3}}}-{\frac{2}{\SUPER{3}{4}}}-{\frac{2}{5}}-{\frac{2}{%
\SUPER{5}{2}}}+\frac{1}{7}%
\end{fricasmath}
\end{TeXOutput}
\formatResultType{PartialFraction(Integer)}
\end{xtc}
\begin{xtc}
\begin{xtccomment}
Like integers, bases (radices) other than ten can be used for rational
numbers (see \xmpref{RadixExpansion}).
Here we use base eight.
\end{xtccomment}
\begin{spadsrc}
radix(4/7, 8)
\end{spadsrc}
\begin{TeXOutput}
\begin{fricasmath}{22}
0\STRING{.}\overline{4}%
\end{fricasmath}
\end{TeXOutput}
\formatResultType{RadixExpansion(8)}
\end{xtc}
\begin{xtc}
\begin{xtccomment}
Of course, there are complex versions of these as well.
\Language{} decides to make the result a complex rational number.
\end{xtccomment}
\begin{spadsrc}
% + 2/3*%i
\end{spadsrc}
\begin{TeXOutput}
\begin{fricasmath}{23}
\frac{4}{7}+\frac{2}{3}\TIMES \ImaginaryI %
\end{fricasmath}
\end{TeXOutput}
\formatResultType{Complex(Fraction(Integer))}
\end{xtc}
\begin{xtc}
\begin{xtccomment}
You can also use \Language{} to manipulate fractional powers.
\index{radical}
\end{xtccomment}
\begin{spadsrc}
(5 + sqrt 63 + sqrt 847)^(1/3)
\end{spadsrc}
\begin{TeXOutput}
\begin{fricasmath}{24}
\nthroot{14\TIMES \sqrt{7}+5}{3}%
\end{fricasmath}
\end{TeXOutput}
\formatResultType{AlgebraicNumber}
\end{xtc}
\begin{xtc}
\begin{xtccomment}
You can also compute with integers modulo a prime.
\end{xtccomment}
\begin{spadsrc}
x : PrimeField 7 := 5 
\end{spadsrc}
\begin{TeXOutput}
\begin{fricasmath}{25}
5%
\end{fricasmath}
\end{TeXOutput}
\formatResultType{PrimeField(7)}
\end{xtc}
\begin{xtc}
\begin{xtccomment}
Arithmetic is then done modulo \mathOrSpad{7}.
\end{xtccomment}
\begin{spadsrc}
x^3 
\end{spadsrc}
\begin{TeXOutput}
\begin{fricasmath}{26}
6%
\end{fricasmath}
\end{TeXOutput}
\formatResultType{PrimeField(7)}
\end{xtc}
\begin{xtc}
\begin{xtccomment}
Since \mathOrSpad{7} is prime, you can invert nonzero values.
\end{xtccomment}
\begin{spadsrc}
1/x 
\end{spadsrc}
\begin{TeXOutput}
\begin{fricasmath}{27}
3%
\end{fricasmath}
\end{TeXOutput}
\formatResultType{PrimeField(7)}
\end{xtc}
\begin{xtc}
\begin{xtccomment}
You can also compute modulo an integer that is not a prime.
\end{xtccomment}
\begin{spadsrc}
y : IntegerMod 6 := 5 
\end{spadsrc}
\begin{TeXOutput}
\begin{fricasmath}{28}
5%
\end{fricasmath}
\end{TeXOutput}
\formatResultType{IntegerMod(6)}
\end{xtc}
\begin{xtc}
\begin{xtccomment}
All of the usual arithmetic operations are available.
\end{xtccomment}
\begin{spadsrc}
y^3 
\end{spadsrc}
\begin{TeXOutput}
\begin{fricasmath}{29}
5%
\end{fricasmath}
\end{TeXOutput}
\formatResultType{IntegerMod(6)}
\end{xtc}
\begin{xtc}
\begin{xtccomment}
Inversion is not available if the modulus is not a prime
number.
Modular arithmetic and prime fields are discussed in
\spadref{ugxProblemFinitePrime}.
\end{xtccomment}
\begin{spadsrc}
1/y 
\end{spadsrc}
\begin{MessageOutput}
   There are 12 exposed and 12 unexposed library operations named / 
      having 2 argument(s) but none was determined to be applicable. 
      Use HyperDoc Browse, or issue
                                )display op /
      to learn more about the available operations. Perhaps 
      package-calling the operation or using coercions on the arguments
      will allow you to apply the operation.
\end{MessageOutput}
\begin{MessageOutput}
   Cannot find a definition or applicable library operation named / 
      with argument type(s) 
                               PositiveInteger
                                IntegerMod(6)
      
      Perhaps you should use "@" to indicate the required return type, 
      or "$" to specify which version of the function you need.
\end{MessageOutput}
\end{xtc}
\begin{xtc}
\begin{xtccomment}
This defines \spad{a} to be an algebraic number, that is,
a root of a polynomial equation.
\end{xtccomment}
\begin{spadsrc}
a := rootOf(a^5 + a^3 + a^2 + 3,a) 
\end{spadsrc}
\begin{TeXOutput}
\begin{fricasmath}{30}
\SYMBOL{a}%
\end{fricasmath}
\end{TeXOutput}
\formatResultType{Expression(Integer)}
\end{xtc}
\begin{xtc}
\begin{xtccomment}
Computations with \spad{a} are reduced according
to the polynomial equation.
\end{xtccomment}
\begin{spadsrc}
(a + 1)^10
\end{spadsrc}
\begin{TeXOutput}
\begin{fricasmath}{31}
-{85\TIMES \SUPER{\SYMBOL{a}}{4}}-{264\TIMES \SUPER{\SYMBOL{a}}{3}}-{378%
\TIMES \SUPER{\SYMBOL{a}}{2}}-{458\TIMES \SYMBOL{a}}-{287}%
\end{fricasmath}
\end{TeXOutput}
\formatResultType{Expression(Integer)}
\end{xtc}
\begin{xtc}
\begin{xtccomment}
Define \spad{b} to be an algebraic number involving \spad{a}.
\end{xtccomment}
\begin{spadsrc}
b := rootOf(b^4 + a,b) 
\end{spadsrc}
\begin{TeXOutput}
\begin{fricasmath}{32}
\SYMBOL{b}%
\end{fricasmath}
\end{TeXOutput}
\formatResultType{Expression(Integer)}
\end{xtc}
\begin{xtc}
\begin{xtccomment}
Do some arithmetic.
\end{xtccomment}
\begin{spadsrc}
2/(b - 1) 
\end{spadsrc}
\begin{TeXOutput}
\begin{fricasmath}{33}
\frac{2}{\SYMBOL{b}-{1}}%
\end{fricasmath}
\end{TeXOutput}
\formatResultType{Expression(Integer)}
\end{xtc}
\begin{xtc}
\begin{xtccomment}
To expand and simplify this, call \spadfun{ratDenom}
to rationalize the denominator.
\end{xtccomment}
\begin{spadsrc}
ratDenom(%) 
\end{spadsrc}
\begin{TeXOutput}
\begin{fricasmath}{34}
\PAREN{\SUPER{\SYMBOL{a}}{4}-{\SUPER{\SYMBOL{a}}{3}}+2\TIMES \SUPER{\SYMBOL{a%
}}{2}-{\SYMBOL{a}}+1}\TIMES \SUPER{\SYMBOL{b}}{3}+\PAREN{\SUPER{\SYMBOL{a}}{4%
}-{\SUPER{\SYMBOL{a}}{3}}+2\TIMES \SUPER{\SYMBOL{a}}{2}-{\SYMBOL{a}}+1}%
\TIMES \SUPER{\SYMBOL{b}}{2}+\PAREN{\SUPER{\SYMBOL{a}}{4}-{\SUPER{\SYMBOL{a}%
}{3}}+2\TIMES \SUPER{\SYMBOL{a}}{2}-{\SYMBOL{a}}+1}\TIMES \SYMBOL{b}+\SUPER{%
\SYMBOL{a}}{4}-{\SUPER{\SYMBOL{a}}{3}}+2\TIMES \SUPER{\SYMBOL{a}}{2}-{\SYMBOL%
{a}}+1%
\end{fricasmath}
\end{TeXOutput}
\formatResultType{Expression(Integer)}
\end{xtc}
\begin{xtc}
\begin{xtccomment}
If we do this, we should get \spad{b}.
\end{xtccomment}
\begin{spadsrc}
2/%+1 
\end{spadsrc}
\begin{TeXOutput}
\begin{fricasmath}{35}
\frac{\PAREN{\SUPER{\SYMBOL{a}}{4}-{\SUPER{\SYMBOL{a}}{3}}+2\TIMES \SUPER{%
\SYMBOL{a}}{2}-{\SYMBOL{a}}+1}\TIMES \SUPER{\SYMBOL{b}}{3}+\PAREN{\SUPER{%
\SYMBOL{a}}{4}-{\SUPER{\SYMBOL{a}}{3}}+2\TIMES \SUPER{\SYMBOL{a}}{2}-{\SYMBOL%
{a}}+1}\TIMES \SUPER{\SYMBOL{b}}{2}+\PAREN{\SUPER{\SYMBOL{a}}{4}-{\SUPER{%
\SYMBOL{a}}{3}}+2\TIMES \SUPER{\SYMBOL{a}}{2}-{\SYMBOL{a}}+1}\TIMES \SYMBOL{b%
}+\SUPER{\SYMBOL{a}}{4}-{\SUPER{\SYMBOL{a}}{3}}+2\TIMES \SUPER{\SYMBOL{a}}{2}%
-{\SYMBOL{a}}+3}{\PAREN{\SUPER{\SYMBOL{a}}{4}-{\SUPER{\SYMBOL{a}}{3}}+2%
\TIMES \SUPER{\SYMBOL{a}}{2}-{\SYMBOL{a}}+1}\TIMES \SUPER{\SYMBOL{b}}{3}+%
\PAREN{\SUPER{\SYMBOL{a}}{4}-{\SUPER{\SYMBOL{a}}{3}}+2\TIMES \SUPER{\SYMBOL{a%
}}{2}-{\SYMBOL{a}}+1}\TIMES \SUPER{\SYMBOL{b}}{2}+\PAREN{\SUPER{\SYMBOL{a}}{4%
}-{\SUPER{\SYMBOL{a}}{3}}+2\TIMES \SUPER{\SYMBOL{a}}{2}-{\SYMBOL{a}}+1}%
\TIMES \SYMBOL{b}+\SUPER{\SYMBOL{a}}{4}-{\SUPER{\SYMBOL{a}}{3}}+2\TIMES %
\SUPER{\SYMBOL{a}}{2}-{\SYMBOL{a}}+1}%
\end{fricasmath}
\end{TeXOutput}
\formatResultType{Expression(Integer)}
\end{xtc}
\begin{xtc}
\begin{xtccomment}
But we need to rationalize the denominator again.
\end{xtccomment}
\begin{spadsrc}
ratDenom(%) 
\end{spadsrc}
\begin{TeXOutput}
\begin{fricasmath}{36}
\SYMBOL{b}%
\end{fricasmath}
\end{TeXOutput}
\formatResultType{Expression(Integer)}
\end{xtc}
\begin{xtc}
\begin{xtccomment}
Types \spadtype{Quaternion} and \spadtype{Octonion} are also available.
Multiplication of quaternions is non-commutative, as expected.
\end{xtccomment}
\begin{spadsrc}
q:=quatern(1,2,3,4)*quatern(5,6,7,8) - quatern(5,6,7,8)*quatern(1,2,3,4)
\end{spadsrc}
\begin{TeXOutput}
\begin{fricasmath}{37}
-{8\TIMES \SYMBOL{i}}+16\TIMES \SYMBOL{j}-{8\TIMES \SYMBOL{k}}%
\end{fricasmath}
\end{TeXOutput}
\formatResultType{Quaternion(Integer)}
\end{xtc}

% *********************************************************************
\head{section}{Data Structures}{ugIntroCollect}
% *********************************************************************
%

\Language{} has a large variety of data structures available.
Many data structures are particularly useful for interactive
computation and others are useful for building applications.
The data structures of \Language{} are organized into
\spadglossSee{category hierarchies}{hierarchy} as shown on
the inside back cover.

A \spadgloss{list} is the most commonly used data structure in
\Language{} for holding objects all of the same
type.\footnote{Lists are discussed in \xmpref{List} and in
\spadref{ugLangIts}.}
The name {\it list} is short for ``linked-list of nodes.'' Each
node consists of a value (\spadfunFrom{first}{List}) and a link
(\spadfunFrom{rest}{List}) that
\spadglossSee{points}{pointer} to the next node, or to a
distinguished value denoting the empty list.
To get to, say, the third element, \Language{} starts at the front
of the list, then traverses across two links to the third node.

\begin{xtc}
\begin{xtccomment}
Write a list of elements using
square brackets with commas separating the elements.
\end{xtccomment}
\begin{spadsrc}
u := [1,-7,11] 
\end{spadsrc}
\begin{TeXOutput}
\begin{fricasmath}{1}
\BRACKET{1\COMMA -{7}\COMMA 11}%
\end{fricasmath}
\end{TeXOutput}
\formatResultType{List(Integer)}
\end{xtc}
\begin{xtc}
\begin{xtccomment}
This is the value at the third node.
Alternatively, you can say \spad{u.3}.
\end{xtccomment}
\begin{spadsrc}
first rest rest u
\end{spadsrc}
\begin{TeXOutput}
\begin{fricasmath}{2}
11%
\end{fricasmath}
\end{TeXOutput}
\formatResultType{PositiveInteger}
\end{xtc}

Many operations are defined on lists, such as:
\spadfun{empty?}, to test that a list has no elements;
\spadfun{cons}\spad{(x,l)}, to create a new list with
\spadfun{first} element \spad{x} and \spadfun{rest} \spad{l};
\spadfun{reverse}, to create a new list with elements in reverse
order; and \spadfun{sort}, to arrange elements in order.

An important point about lists is that they are ``mutable'': their
constituent elements and links can be changed ``in place.''
To do this, use any of the operations whose names end with the
character \spadSyntax{!}.

\begin{xtc}
\begin{xtccomment}
The operation \spadfunFromX{concat}{List}\spad{(u,v)}
replaces the last link of the list
\spad{u} to point to some other list \spad{v}.
Since \spad{u} refers to the original list,
this change is seen by \spad{u}.
\end{xtccomment}
\begin{spadsrc}
concat!(u,[9,1,3,-4]); u
\end{spadsrc}
\begin{TeXOutput}
\begin{fricasmath}{3}
\BRACKET{1\COMMA -{7}\COMMA 11\COMMA 9\COMMA 1\COMMA 3\COMMA -{4}}%
\end{fricasmath}
\end{TeXOutput}
\formatResultType{List(Integer)}
\end{xtc}
\begin{xtc}
\begin{xtccomment}
A {\it cyclic list} is a list with a ``cycle'':
\index{list!cyclic}
a link pointing back to an earlier node of the list.
\index{cyclic list}
To create a cycle, first get a node somewhere down
the list.
\end{xtccomment}
\begin{spadsrc}
lastnode := rest(u,3)
\end{spadsrc}
\begin{TeXOutput}
\begin{fricasmath}{4}
\BRACKET{9\COMMA 1\COMMA 3\COMMA -{4}}%
\end{fricasmath}
\end{TeXOutput}
\formatResultType{List(Integer)}
\end{xtc}
\begin{xtc}
\begin{xtccomment}
Use \spadfunFromX{setrest}{List} to
change the link emanating from that node to point back to an
earlier part of the list.
\end{xtccomment}
\begin{spadsrc}
setrest!(lastnode,rest(u,2)); u
\end{spadsrc}
\begin{TeXOutput}
\begin{fricasmath}{5}
\BRACKET{1\COMMA -{7}\COMMA \overline{11\COMMA 9}}%
\end{fricasmath}
\end{TeXOutput}
\formatResultType{List(Integer)}
\end{xtc}

A \spadgloss{stream}
is a structure that (potentially) has an infinite number of
distinct elements.\footnote{Streams are discussed in
\xmpref{Stream} and in \spadref{ugLangIts}.}
Think of a stream as an ``infinite list'' where elements are
computed successively.

\begin{xtc}
\begin{xtccomment}
Create an infinite stream of factored integers.
Only a certain number of initial elements are computed
and displayed.
\end{xtccomment}
\begin{spadsrc}
[factor(i) for i in 2.. by 2] 
\end{spadsrc}
\begin{TeXOutput}
\begin{fricasmath}{6}
\BRACKET{2\COMMA \SUPER{2}{2}\COMMA 2\TIMES 3\COMMA \SUPER{2}{3}\COMMA 2%
\TIMES 5\COMMA \SUPER{2}{2}\TIMES 3\COMMA 2\TIMES 7\COMMA \STRING{...}}%
\end{fricasmath}
\end{TeXOutput}
\formatResultType{Stream(Factored(Integer))}
\end{xtc}
\begin{xtc}
\begin{xtccomment}
\Language{} represents streams by a collection of already-computed
elements together with a function to compute the next element
``on demand.''
Asking for the \eth{n} element causes elements \spad{1} through
\spad{n} to be evaluated.
\end{xtccomment}
\begin{spadsrc}
%.36 
\end{spadsrc}
\begin{TeXOutput}
\begin{fricasmath}{7}
\SUPER{2}{3}\TIMES \SUPER{3}{2}%
\end{fricasmath}
\end{TeXOutput}
\formatResultType{Factored(Integer)}
\end{xtc}

Streams can also be finite or cyclic.
They are implemented by a linked list structure similar to lists
and have many of the same operations.
For example, \spadfun{first} and \spadfun{rest} are used to access
elements and successive nodes of a stream.
%%> reverse and sort do not exist for streams
%%Don't try to reverse or sort a stream: the
%%operation will generally run forever!

A \spadgloss{one-dimensional array} is another data structure used
to hold objects of the same type.\footnote{See \xmpref{OneDimensionalArray} for
details.}
Unlike lists, one-dimensional arrays are inflexible---they are
\index{array!one-dimensional}
implemented using a fixed block of storage.
Their advantage is that they give quick and equal access time to
any element.

\begin{xtc}
\begin{xtccomment}
A simple way to create a one-dimensional array is to apply the
operation \spadfun{oneDimensionalArray} to a list of elements.
\end{xtccomment}
\begin{spadsrc}
a := oneDimensionalArray [1, -7, 3, 3/2]
\end{spadsrc}
\begin{TeXOutput}
\begin{fricasmath}{8}
\BRACKET{1\COMMA -{7}\COMMA 3\COMMA \frac{3}{2}}%
\end{fricasmath}
\end{TeXOutput}
\formatResultType{OneDimensionalArray(Fraction(Integer))}
\end{xtc}
\begin{xtc}
\begin{xtccomment}
One-dimensional arrays are also mutable:
you can change their constituent elements ``in place.''
\end{xtccomment}
\begin{spadsrc}
a.3 := 11; a
\end{spadsrc}
\begin{TeXOutput}
\begin{fricasmath}{9}
\BRACKET{1\COMMA -{7}\COMMA 11\COMMA \frac{3}{2}}%
\end{fricasmath}
\end{TeXOutput}
\formatResultType{OneDimensionalArray(Fraction(Integer))}
\end{xtc}
\begin{xtc}
\begin{xtccomment}
However, one-dimensional arrays are not flexible structures.
You cannot destructively \spadfunX{concat} them together.
\end{xtccomment}
\begin{spadsrc}
concat!(a,oneDimensionalArray [1,-2])
\end{spadsrc}
\begin{MessageOutput}
   There are 5 exposed and 0 unexposed library operations named concat!
      having 2 argument(s) but none was determined to be applicable. 
      Use HyperDoc Browse, or issue
                             )display op concat!
      to learn more about the available operations. Perhaps 
      package-calling the operation or using coercions on the arguments
      will allow you to apply the operation.
\end{MessageOutput}
\begin{MessageOutput}
   Cannot find a definition or applicable library operation named 
      concat! with argument type(s) 
                   OneDimensionalArray(Fraction(Integer))
                        OneDimensionalArray(Integer)
      
      Perhaps you should use "@" to indicate the required return type, 
      or "$" to specify which version of the function you need.
\end{MessageOutput}
\end{xtc}

Examples of datatypes similar to \spadtype{OneDimensionalArray}
are: \spadtype{Vector} (vectors are mathematical structures
implemented by one-dimensional arrays), \spadtype{String} (arrays
of ``characters,'' represented by byte vectors), and
\spadtype{Bits} (represented by ``bit vectors'').

\begin{xtc}
\begin{xtccomment}
A vector of 32 bits, each representing the \spadtype{Boolean} value \spad{true}.
\end{xtccomment}
\begin{spadsrc}
bits(32,true)
\end{spadsrc}
\begin{TeXOutput}
\begin{fricasmath}{10}
\STRING{"11111111111111111111111111111111"}%
\end{fricasmath}
\end{TeXOutput}
\formatResultType{Bits}
\end{xtc}

A \spadgloss{flexible array} is a cross between a list
\index{array!flexible}
and a one-dimensional array.\footnote{See \xmpref{FlexibleArray} for
details.}
Like a one-dimensional array, a flexible array occupies a fixed
block of storage.
Its block of storage, however, has room to expand!
When it gets full, it grows (a new, larger block of storage is
allocated); when it has too much room, it contracts.

\begin{xtc}
\begin{xtccomment}
Create a flexible array of three elements.
\end{xtccomment}
\begin{spadsrc}
f := flexibleArray [2, 7, -5]
\end{spadsrc}
\begin{TeXOutput}
\begin{fricasmath}{11}
\BRACKET{2\COMMA 7\COMMA -{5}}%
\end{fricasmath}
\end{TeXOutput}
\formatResultType{FlexibleArray(Integer)}
\end{xtc}
\begin{xtc}
\begin{xtccomment}
Insert some elements between the second and third elements.
\end{xtccomment}
\begin{spadsrc}
insert!(flexibleArray [11, -3],f,2)
\end{spadsrc}
\begin{TeXOutput}
\begin{fricasmath}{12}
\BRACKET{2\COMMA 11\COMMA -{3}\COMMA 7\COMMA -{5}}%
\end{fricasmath}
\end{TeXOutput}
\formatResultType{FlexibleArray(Integer)}
\end{xtc}

Flexible arrays are used to implement ``heaps.'' A
\spadgloss{heap} is an example of a data structure called a
\spadgloss{priority queue}, where elements are ordered with
respect to one another.\footnote{See \xmpref{Heap} for more details.
Heaps are also examples of data structures called
\spadglossSee{bags}{bag}.
Other bag data structures are \spadtype{Stack}, \spadtype{Queue},
and \spadtype{Dequeue}.}
A heap is organized
so as to optimize insertion and extraction of maximum elements.
The \spadfunX{extract} operation
returns the maximum element of the heap, after destructively
removing that element and
reorganizing the heap
so that the next maximum element is ready to be delivered.

\begin{xtc}
\begin{xtccomment}
An easy way to create a heap is to apply the
operation \spadfun{heap} to a list of values.
\end{xtccomment}
\begin{spadsrc}
h := heap [-4,7,11,3,4,-7]
\end{spadsrc}
\begin{TeXOutput}
\begin{fricasmath}{13}
\BRACKET{11\COMMA 7\COMMA -{4}\COMMA 3\COMMA 4\COMMA -{7}}%
\end{fricasmath}
\end{TeXOutput}
\formatResultType{Heap(Integer)}
\end{xtc}
\begin{xtc}
\begin{xtccomment}
This loop extracts elements one-at-a-time from \spad{h}
until the heap is exhausted, returning the elements
as a list in the order they were extracted.
\end{xtccomment}
\begin{spadsrc}
[extract!(h) while not empty?(h)]
\end{spadsrc}
\begin{TeXOutput}
\begin{fricasmath}{14}
\BRACKET{11\COMMA 7\COMMA 4\COMMA 3\COMMA -{4}\COMMA -{7}}%
\end{fricasmath}
\end{TeXOutput}
\formatResultType{List(Integer)}
\end{xtc}

A \spadgloss{binary tree} is a ``tree'' with at most two branches
\index{tree}
per node: it is either empty, or else is a node consisting of a
value, and a left and right subtree (again, binary trees).\footnote{Example of binary tree types are
\spadtype{BinarySearchTree} (see \xmpref{BinarySearchTree},
\spadtype{PendantTree}, \spadtype{TournamentTree},
and \spadtype{BalancedBinaryTree} (see \xmpref{BalancedBinaryTree}).}

\begin{xtc}
\begin{xtccomment}
A {\it binary search tree} is a binary tree such that,
\index{tree!binary search}
for each node, the value of the node is
\index{binary search tree}
greater than all values (if any) in the left subtree,
and less than or equal all values (if any) in the right subtree.
\end{xtccomment}
\begin{spadsrc}
binarySearchTree [5,3,2,9,4,7,11]
\end{spadsrc}
\begin{TeXOutput}
\begin{fricasmath}{15}
\BRACKET{\BRACKET{2\COMMA 3\COMMA 4}\COMMA 5\COMMA \BRACKET{7\COMMA 9\COMMA %
11}}%
\end{fricasmath}
\end{TeXOutput}
\formatResultType{BinarySearchTree(PositiveInteger)}
\end{xtc}

\begin{xtc}
\begin{xtccomment}
A {\it balanced binary tree} is useful for doing modular computations.
\index{balanced binary tree}
Given a list \spad{lm} of moduli,
\index{tree!balanced binary}
\spadfun{modTree}\spad{(a,lm)} produces a balanced binary
tree with the values $a \bmod m$
at its leaves.
\end{xtccomment}
\begin{spadsrc}
modTree(8,[2,3,5,7])
\end{spadsrc}
\begin{TeXOutput}
\begin{fricasmath}{16}
\BRACKET{0\COMMA 2\COMMA 3\COMMA 1}%
\end{fricasmath}
\end{TeXOutput}
\formatResultType{List(Integer)}
\end{xtc}

A \spadgloss{set} is a collection of elements where duplication
and order is irrelevant.\footnote{See \xmpref{Set} for more
details.}
Sets are always finite and have no corresponding
structure like streams for infinite collections.

\begin{xtc}
\begin{xtccomment}
Create sets by using the \spadfun{set} function.
\end{xtccomment}
\begin{spadsrc}
fs := set[1/3,4/5,-1/3,4/5] 
\end{spadsrc}
\begin{TeXOutput}
\begin{fricasmath}{17}
\BRACE{-{\frac{1}{3}}\COMMA \frac{1}{3}\COMMA \frac{4}{5}}%
\end{fricasmath}
\end{TeXOutput}
\formatResultType{Set(Fraction(Integer))}
\end{xtc}

A \spadgloss{multiset}
is a set that keeps track of the number
of duplicate values.\footnote{See \xmpref{Multiset} for details.}
\begin{xtc}
\begin{xtccomment}
For all the primes \spad{p} between 2 and 1000, find the
distribution of $p \bmod 5$.
\end{xtccomment}
\begin{spadsrc}
multiset [x rem 5 for x in primes(2,1000)]
\end{spadsrc}
\begin{TeXOutput}
\begin{fricasmath}{18}
\BRACE{47\STRING{:\ }2\COMMA 42\STRING{:\ }3\COMMA 0\COMMA 40\STRING{:\ }1%
\COMMA 38\STRING{:\ }4}%
\end{fricasmath}
\end{TeXOutput}
\formatResultType{Multiset(Integer)}
\end{xtc}

A \spadgloss{table}
is conceptually a set of ``key--value'' pairs and
is a generalization of a multiset.\footnote{For examples of tables, see
\spadtype{AssociationList} (\xmpref{AssociationList}),
\spadtype{HashTable},
\spadtype{KeyedAccessFile} (\xmpref{KeyedAccessFile}),
\spadtype{Library} (\xmpref{Library}),
\spadtype{SparseTable} (\xmpref{SparseTable}),
\spadtype{StringTable} (\xmpref{StringTable}),
and \spadtype{Table} (\xmpref{Table}).}
The domain \spadtype{Table(Key, Entry)} provides a general-purpose
type for tables with {\it values} of type \spad{Entry} indexed
by {\it keys} of type \spad{Key}.

\begin{xtc}
\begin{xtccomment}
Compute the above distribution of primes using tables.
First, let \spad{t} denote an empty table of keys and values,
each of type \spadtype{Integer}.
\end{xtccomment}
\begin{spadsrc}
t : Table(Integer,Integer) := empty()
\end{spadsrc}
\begin{TeXOutput}
\begin{fricasmath}{19}
\STRING{table}\PAREN{}%
\end{fricasmath}
\end{TeXOutput}
\formatResultType{Table(Integer, Integer)}
\end{xtc}

We define a function \userfun{howMany} to return the number
of values of a given modulus \spad{k} seen so far.
It calls \spadfun{search}\spad{(k,t)} which returns the number of
values stored under the key \spad{k} in table \spad{t}, or
\spad{"failed"} if no such value is yet stored in \spad{t} under
\spad{k}.

\begin{xtc}
\begin{xtccomment}
In English, this says ``Define \spad{howMany(k)} as follows.
First, let \smath{n} be the value of \spadfun{search}\smath{(k,t)}.
Then, if \smath{n} has the value \smath{"failed"}, return the value
\smath{1}; otherwise return \smath{n + 1}.''
\end{xtccomment}
\begin{spadsrc}
howMany(k) == (n:=search(k,t); n case "failed" => 1; n+1)
\end{spadsrc}
\end{xtc}
\begin{xtc}
\begin{xtccomment}
Run through the primes to create the table, then print the table.
The expression \spad{t.m := howMany(m)} updates the value in table \spad{t}
stored under key \spad{m}.
\end{xtccomment}
\begin{spadsrc}
for p in primes(2,1000) repeat (m:= p rem 5; t.m:= howMany(m)); t
\end{spadsrc}
\begin{MessageOutput}
   Compiling function howMany with type Integer -> Integer 
\end{MessageOutput}
\begin{TeXOutput}
\begin{fricasmath}{21}
\STRING{table}\PAREN{4=38,1=40,0=1,3=42,2=47}%
\end{fricasmath}
\end{TeXOutput}
\formatResultType{Table(Integer, Integer)}
\end{xtc}

A {\it record}
is an example of an inhomogeneous collection
of objects.\footnote{See \spadref{ugTypesRecords} for details.}
A record consists of a set of named {\it selectors} that
can be used to access its components.
\index{Record@{\sf Record}}

\begin{xtc}
\begin{xtccomment}
Declare that \spad{daniel} can only be
assigned a record with two prescribed fields.
\end{xtccomment}
\begin{spadsrc}
daniel : Record(age : Integer, salary : Float) 
\end{spadsrc}
\end{xtc}
\begin{xtc}
\begin{xtccomment}
Give \spad{daniel} a value, using square brackets to enclose the values of
the fields.
\end{xtccomment}
\begin{spadsrc}
daniel := [28, 32005.12] 
\end{spadsrc}
\begin{TeXOutput}
\begin{fricasmath}{23}
\BRACKET{\SYMBOL{age}=28\COMMA \SYMBOL{salary}=\STRING{32005.12}}%
\end{fricasmath}
\end{TeXOutput}
\formatResultType{Record(age: Integer, salary: Float)}
\end{xtc}
\begin{xtc}
\begin{xtccomment}
Give \spad{daniel} a raise.
\end{xtccomment}
\begin{spadsrc}
daniel.salary := 35000; daniel 
\end{spadsrc}
\begin{TeXOutput}
\begin{fricasmath}{24}
\BRACKET{\SYMBOL{age}=28\COMMA \SYMBOL{salary}=\STRING{35000.0}}%
\end{fricasmath}
\end{TeXOutput}
\formatResultType{Record(age: Integer, salary: Float)}
\end{xtc}

A {\it union}
is a data structure used when objects
have multiple types.\footnote{See \spadref{ugTypesUnions} for details.}
\index{Union@{\sf Union}}

\begin{xtc}
\begin{xtccomment}
Let \spad{dog} be either an integer or a string value.
\end{xtccomment}
\begin{spadsrc}
dog: Union(licenseNumber: Integer, name: String)
\end{spadsrc}
\end{xtc}
\begin{xtc}
\begin{xtccomment}
Give \spad{dog} a name.
\end{xtccomment}
\begin{spadsrc}
dog := "Whisper"
\end{spadsrc}
\begin{TeXOutput}
\begin{fricasmath}{26}
\STRING{"Whisper"}%
\end{fricasmath}
\end{TeXOutput}
\formatResultType{Union(name: String, ...)}
\end{xtc}

All told, there are over forty different data structures in
\Language{}.
Using the domain constructors described in \chapref{ugDomains}, you
can add your own data structure or extend an existing one.
Choosing the right data structure for your application may be the key
to obtaining good performance.

% *********************************************************************
\head{section}{Expanding to Higher Dimensions}{ugIntroTwoDim}
% *********************************************************************
%

To get higher dimensional aggregates, you can create one-dimensional
aggregates with elements that are themselves
aggregates, for example, lists of lists, one-dimensional arrays of
lists of multisets, and so on.
For applications requiring two-dimensional homogeneous aggregates,
you will likely find {\it two-dimensional arrays}
\index{matrix}
and {\it matrices} most useful.
\index{array!two-dimensional}

The entries in \spadtype{TwoDimensionalArray} and
\spadtype{Matrix} objects
are all the same type, except that those for
\spadtype{Matrix} must belong to a \spadtype{Ring}.
You create and access elements in roughly the same way.
Since matrices have an understood algebraic structure, certain algebraic
operations are available for matrices but not for arrays.
Because of this, we limit our discussion here to
\spadtype{Matrix}, that can be regarded as an extension of
\spadtype{TwoDimensionalArray}.\footnote{See
\xmpref{TwoDimensionalArray} for more information about arrays.
For more information about \Language{}'s linear algebra
facilities, see \xmpref{Matrix}, \xmpref{Permanent},
\xmpref{SquareMatrix}, \xmpref{Vector},
\spadref{ugProblemEigen}%
\ (computation of eigenvalues and eigenvectors)
, and
\spadref{ugProblemLinPolEqn}%
\ (solution of linear and polynomial equations)
.}

\begin{xtc}
\begin{xtccomment}
You can create a matrix from a list of lists,
\index{matrix!creating}
where each of the inner lists represents a row of the matrix.
\end{xtccomment}
\begin{spadsrc}
m := matrix([[1,2], [3,4]]) 
\end{spadsrc}
\begin{TeXOutput}
\begin{fricasmath}{1}
\begin{MATRIX}{2}1&2\\3&4\end{MATRIX}%
\end{fricasmath}
\end{TeXOutput}
\formatResultType{Matrix(Integer)}
\end{xtc}
\begin{xtc}
\begin{xtccomment}
The ``collections'' construct (see \spadref{ugLangIts}) is
useful for creating matrices whose entries are given by formulas.
\index{matrix!Hilbert}
\end{xtccomment}
\begin{spadsrc}
matrix([[1/(i + j - x) for i in 1..4] for j in 1..4]) 
\end{spadsrc}
\begin{TeXOutput}
\begin{fricasmath}{2}
\begin{MATRIX}{4}-{\frac{1}{\SYMBOL{x}-{2}}}&-{\frac{1}{\SYMBOL{x}-{3}}}&-{%
\frac{1}{\SYMBOL{x}-{4}}}&-{\frac{1}{\SYMBOL{x}-{5}}}\\-{\frac{1}{\SYMBOL{x}-%
{3}}}&-{\frac{1}{\SYMBOL{x}-{4}}}&-{\frac{1}{\SYMBOL{x}-{5}}}&-{\frac{1}{%
\SYMBOL{x}-{6}}}\\-{\frac{1}{\SYMBOL{x}-{4}}}&-{\frac{1}{\SYMBOL{x}-{5}}}&-{%
\frac{1}{\SYMBOL{x}-{6}}}&-{\frac{1}{\SYMBOL{x}-{7}}}\\-{\frac{1}{\SYMBOL{x}-%
{5}}}&-{\frac{1}{\SYMBOL{x}-{6}}}&-{\frac{1}{\SYMBOL{x}-{7}}}&-{\frac{1}{%
\SYMBOL{x}-{8}}}\end{MATRIX}%
\end{fricasmath}
\end{TeXOutput}
\formatResultType{Matrix(Fraction(Polynomial(Integer)))}
\end{xtc}
\begin{xtc}
\begin{xtccomment}
Let \spad{vm} denote the three by three Vandermonde matrix.
\end{xtccomment}
\begin{spadsrc}
vm := matrix [[1,1,1], [x,y,z], [x*x,y*y,z*z]] 
\end{spadsrc}
\begin{TeXOutput}
\begin{fricasmath}{3}
\begin{MATRIX}{3}1&1&1\\\SYMBOL{x}&\SYMBOL{y}&\SYMBOL{z}\\\SUPER{\SYMBOL{x}}{%
2}&\SUPER{\SYMBOL{y}}{2}&\SUPER{\SYMBOL{z}}{2}\end{MATRIX}%
\end{fricasmath}
\end{TeXOutput}
\formatResultType{Matrix(Polynomial(Integer))}
\end{xtc}
\begin{xtc}
\begin{xtccomment}
Use this syntax to extract an entry in the matrix.
\end{xtccomment}
\begin{spadsrc}
vm(3,3) 
\end{spadsrc}
\begin{TeXOutput}
\begin{fricasmath}{4}
\SUPER{\SYMBOL{z}}{2}%
\end{fricasmath}
\end{TeXOutput}
\formatResultType{Polynomial(Integer)}
\end{xtc}
\begin{xtc}
\begin{xtccomment}
You can also pull out a \spadfun{row} or a \spad{column}.
\end{xtccomment}
\begin{spadsrc}
column(vm,2) 
\end{spadsrc}
\begin{TeXOutput}
\begin{fricasmath}{5}
\BRACKET{1\COMMA \SYMBOL{y}\COMMA \SUPER{\SYMBOL{y}}{2}}%
\end{fricasmath}
\end{TeXOutput}
\formatResultType{Vector(Polynomial(Integer))}
\end{xtc}
\begin{xtc}
\begin{xtccomment}
You can do arithmetic.
\end{xtccomment}
\begin{spadsrc}
vm * vm 
\end{spadsrc}
\begin{TeXOutput}
\begin{fricasmath}{6}
\begin{MATRIX}{3}\SUPER{\SYMBOL{x}}{2}+\SYMBOL{x}+1&\SUPER{\SYMBOL{y}}{2}+%
\SYMBOL{y}+1&\SUPER{\SYMBOL{z}}{2}+\SYMBOL{z}+1\\\SUPER{\SYMBOL{x}}{2}\TIMES %
\SYMBOL{z}+\SYMBOL{x}\TIMES \SYMBOL{y}+\SYMBOL{x}&\SUPER{\SYMBOL{y}}{2}%
\TIMES \SYMBOL{z}+\SUPER{\SYMBOL{y}}{2}+\SYMBOL{x}&\SUPER{\SYMBOL{z}}{3}+%
\SYMBOL{y}\TIMES \SYMBOL{z}+\SYMBOL{x}\\\SUPER{\SYMBOL{x}}{2}\TIMES \SUPER{%
\SYMBOL{z}}{2}+\SYMBOL{x}\TIMES \SUPER{\SYMBOL{y}}{2}+\SUPER{\SYMBOL{x}}{2}&%
\SUPER{\SYMBOL{y}}{2}\TIMES \SUPER{\SYMBOL{z}}{2}+\SUPER{\SYMBOL{y}}{3}+%
\SUPER{\SYMBOL{x}}{2}&\SUPER{\SYMBOL{z}}{4}+\SUPER{\SYMBOL{y}}{2}\TIMES %
\SYMBOL{z}+\SUPER{\SYMBOL{x}}{2}\end{MATRIX}%
\end{fricasmath}
\end{TeXOutput}
\formatResultType{Matrix(Polynomial(Integer))}
\end{xtc}
\begin{xtc}
\begin{xtccomment}
You can perform operations such as
\spadfun{transpose}, \spadfun{trace}, and \spadfun{determinant}.
\end{xtccomment}
\begin{spadsrc}
factor determinant vm 
\end{spadsrc}
\begin{TeXOutput}
\begin{fricasmath}{7}
\PAREN{\SYMBOL{y}-{\SYMBOL{x}}}\TIMES \PAREN{\SYMBOL{z}-{\SYMBOL{y}}}\TIMES %
\PAREN{\SYMBOL{z}-{\SYMBOL{x}}}%
\end{fricasmath}
\end{TeXOutput}
\formatResultType{Factored(Polynomial(Integer))}
\end{xtc}

% *********************************************************************
\head{section}{Writing Your Own Functions}{ugIntroYou}
% *********************************************************************
%

\Language{} provides you with a very large library of predefined
operations and objects to compute with.
You can use the \Language{} library of constructors to create new
objects dynamically of quite arbitrary complexity.
For example, you can make lists of matrices of fractions of
polynomials with complex floating point numbers as coefficients.
Moreover, the library provides a wealth of operations that allow
you to create and manipulate these objects.

For many applications,
you need to interact with the interpreter and write some
\Language{} programs to tackle your application.
\Language{} allows you to write functions interactively,
\index{function}
thereby effectively extending the system library.
Here we give a few simple examples, leaving the details to \chapref{ugUser}.

We begin by looking at several ways that you can define the
``factorial'' function in \Language{}.
The first way is to give a
\index{function!piece-wise definition}
piece-wise definition of the function.
\index{piece-wise function definition}
This method is best for a general recurrence
relation since the pieces are gathered together and compiled into
an efficient iterative function.
Furthermore, enough previously computed values are automatically
saved so that a subsequent call to the function can pick up from
where it left off.

\begin{xtc}
\begin{xtccomment}
Define the value of \userfun{fact} at \spad{0}.
\end{xtccomment}
\begin{spadsrc}
fact(0) == 1 
\end{spadsrc}
\end{xtc}
\begin{xtc}
\begin{xtccomment}
Define the value of \spad{fact(n)} for general \spad{n}.
\end{xtccomment}
\begin{spadsrc}
fact(n) == n*fact(n-1)
\end{spadsrc}
\end{xtc}
\begin{xtc}
\begin{xtccomment}
Ask for the value at \spad{50}.
The resulting function created by \Language{}
computes the value by iteration.
\end{xtccomment}
\begin{spadsrc}
fact(50) 
\end{spadsrc}
\begin{MessageOutput}
   Compiling function fact with type Integer -> Integer 
\end{MessageOutput}
\begin{MessageOutput}
   Compiling function fact as a recurrence relation.
\end{MessageOutput}
\begin{TeXOutput}
\begin{fricasmath}{3}
30414093201713378043 61260816606476884437 76415689605120000000 00000%
\end{fricasmath}
\end{TeXOutput}
\formatResultType{PositiveInteger}
\end{xtc}
\begin{xtc}
\begin{xtccomment}
A second definition uses an \spad{if-then-else} and recursion.
\end{xtccomment}
\begin{spadsrc}
fac(n) == if n < 3 then n else n * fac(n - 1) 
\end{spadsrc}
\end{xtc}
\begin{xtc}
\begin{xtccomment}
This function is less efficient than the previous version since
each iteration involves a recursive function call.
\end{xtccomment}
\begin{spadsrc}
fac(50) 
\end{spadsrc}
\begin{MessageOutput}
   Compiling function fac with type Integer -> Integer 
\end{MessageOutput}
\begin{TeXOutput}
\begin{fricasmath}{5}
30414093201713378043 61260816606476884437 76415689605120000000 00000%
\end{fricasmath}
\end{TeXOutput}
\formatResultType{PositiveInteger}
\end{xtc}
\begin{xtc}
\begin{xtccomment}
A third version directly uses iteration.
\end{xtccomment}
\begin{spadsrc}
fa(n) == (a := 1; for i in 2..n repeat a := a*i; a) 
\end{spadsrc}
\end{xtc}
\begin{xtc}
\begin{xtccomment}
This is the least space-consumptive version.
\end{xtccomment}
\begin{spadsrc}
fa(50) 
\end{spadsrc}
\begin{MessageOutput}
   Compiling function fa with type PositiveInteger -> PositiveInteger 
\end{MessageOutput}
\begin{TeXOutput}
\begin{fricasmath}{7}
30414093201713378043 61260816606476884437 76415689605120000000 00000%
\end{fricasmath}
\end{TeXOutput}
\formatResultType{PositiveInteger}
\end{xtc}
\begin{xtc}
\begin{xtccomment}
A final version appears to construct a large list and then reduces over
it with multiplication.
\end{xtccomment}
\begin{spadsrc}
f(n) == reduce(*,[i for i in 2..n]) 
\end{spadsrc}
\end{xtc}
\begin{xtc}
\begin{xtccomment}
In fact, the resulting computation is optimized into an efficient
iteration loop equivalent to that of the third version.
\end{xtccomment}
\begin{spadsrc}
f(50) 
\end{spadsrc}
\begin{MessageOutput}
   Compiling function f with type PositiveInteger -> PositiveInteger 
\end{MessageOutput}
\begin{TeXOutput}
\begin{fricasmath}{9}
30414093201713378043 61260816606476884437 76415689605120000000 00000%
\end{fricasmath}
\end{TeXOutput}
\formatResultType{PositiveInteger}
\end{xtc}
\begin{xtc}
\begin{xtccomment}
The library version uses an algorithm that is different from the four
above because it highly optimizes the recurrence relation definition of
\spadfun{factorial}.
\end{xtccomment}
\begin{spadsrc}
factorial(50)
\end{spadsrc}
\begin{TeXOutput}
\begin{fricasmath}{10}
30414093201713378043 61260816606476884437 76415689605120000000 00000%
\end{fricasmath}
\end{TeXOutput}
\formatResultType{PositiveInteger}
\end{xtc}

You are not limited to one-line functions in \Language{}.
If you place your function definitions in {\bf .input} files
\index{file!input}
(see \spadref{ugInOutIn}), you can have
multi-line functions that use indentation for grouping.

Given \spad{n} elements, \spadfun{diagonalMatrix} creates an
\spad{n} by \spad{n} matrix with those elements down the diagonal.
This function uses a permutation matrix
that interchanges the \spad{i}th and \spad{j}th rows of a matrix
by which it is right-multiplied.

\begin{xtc}
\begin{xtccomment}
This function definition shows a style of definition that can be used
in {\bf .input} files.
Indentation is used to create \spadglossSee{blocks}{block}:
sequences of expressions that are evaluated in sequence except as
modified by control statements such as \spad{if-then-else} and \spad{return}.
\end{xtccomment}
\begin{spadsrc}
permMat(n, i, j) ==
  m := diagonalMatrix
    [(if i = k or j = k then 0 else 1)
      for k in 1..n]
  m(i,j) := 1
  m(j,i) := 1
  m
\end{spadsrc}
\end{xtc}
\begin{xtc}
\begin{xtccomment}
This creates a four by four matrix that interchanges the second and third
rows.
\end{xtccomment}
\begin{spadsrc}
p := permMat(4,2,3) 
\end{spadsrc}
\begin{MessageOutput}
   Compiling function permMat with type (PositiveInteger,
      PositiveInteger,PositiveInteger) -> Matrix(NonNegativeInteger) 
\end{MessageOutput}
\begin{TeXOutput}
\begin{fricasmath}{12}
\begin{MATRIX}{4}1&0&0&0\\0&0&1&0\\0&1&0&0\\0&0&0&1\end{MATRIX}%
\end{fricasmath}
\end{TeXOutput}
\formatResultType{Matrix(NonNegativeInteger)}
\end{xtc}
\begin{xtc}
\begin{xtccomment}
Create an example matrix to permute.
\end{xtccomment}
\begin{spadsrc}
m := matrix [[4*i + j for j in 1..4] for i in 0..3]
\end{spadsrc}
\begin{TeXOutput}
\begin{fricasmath}{13}
\begin{MATRIX}{4}1&2&3&4\\5&6&7&8\\9&10&11&12\\13&14&15&16\end{MATRIX}%
\end{fricasmath}
\end{TeXOutput}
\formatResultType{Matrix(NonNegativeInteger)}
\end{xtc}
\begin{xtc}
\begin{xtccomment}
Interchange the second and third rows of m.
\end{xtccomment}
\begin{spadsrc}
permMat(4,2,3) * m 
\end{spadsrc}
\begin{TeXOutput}
\begin{fricasmath}{14}
\begin{MATRIX}{4}1&2&3&4\\9&10&11&12\\5&6&7&8\\13&14&15&16\end{MATRIX}%
\end{fricasmath}
\end{TeXOutput}
\formatResultType{Matrix(NonNegativeInteger)}
\end{xtc}

A function can also be passed as an argument to another function,
which then applies the function or passes it off to some other
function that does.
You often have to declare the type of a function that has
functional arguments.

\begin{xtc}
\begin{xtccomment}
This declares \userfun{t} to be a two-argument function that
returns a \spadtype{Float}.
The first argument is a function that takes one \spadtype{Float}
argument and returns a \spadtype{Float}.
\end{xtccomment}
\begin{spadsrc}
t : (Float -> Float, Float) -> Float 
\end{spadsrc}
\end{xtc}
\begin{xtc}
\begin{xtccomment}
This is the definition of \userfun{t}.
\end{xtccomment}
\begin{spadsrc}
t(fun, x) == fun(x)^2 + sin(x)^2 
\end{spadsrc}
\end{xtc}
\begin{xtc}
\begin{xtccomment}
We have not defined a \spadfun{cos} in the workspace. The one from the
\Language{} library will do.
\end{xtccomment}
\begin{spadsrc}
t(cos, 5.2058) 
\end{spadsrc}
\begin{MessageOutput}
   Compiling function t with type ((Float -> Float),Float) -> Float 
\end{MessageOutput}
\begin{TeXOutput}
\begin{fricasmath}{17}
\STRING{1.0}%
\end{fricasmath}
\end{TeXOutput}
\formatResultType{Float}
\end{xtc}
\begin{xtc}
\begin{xtccomment}
Here we define our own (user-defined) function.
\end{xtccomment}
\begin{spadsrc}
cosinv(y) == cos(1/y) 
\end{spadsrc}
\end{xtc}
\begin{xtc}
\begin{xtccomment}
Pass this function as an argument to \userfun{t}.
\end{xtccomment}
\begin{spadsrc}
t(cosinv, 5.2058) 
\end{spadsrc}
\begin{MessageOutput}
   Compiling function cosinv with type Float -> Float 
\end{MessageOutput}
\begin{TeXOutput}
\begin{fricasmath}{19}
\STRING{1.739223724180051649254147684772932520785}%
\end{fricasmath}
\end{TeXOutput}
\formatResultType{Float}
\end{xtc}

\Language{} also has pattern matching capabilities for
\index{simplification}
simplification
\index{pattern matching}
of expressions and for defining new functions by rules.
For example, suppose that you want to apply regularly a transformation
that groups together products of radicals:
\begin{displaymath}
\sqrt{a}\:\sqrt{b} \mapsto \sqrt{ab}, \quad
(\forall a)(\forall b)
\end{displaymath}
Note that such a transformation is not generally correct.
\Language{} never uses it automatically.

\begin{xtc}
\begin{xtccomment}
Give this rule the name \userfun{groupSqrt}.
\end{xtccomment}
\begin{spadsrc}
groupSqrt := rule(sqrt(a) * sqrt(b) == sqrt(a*b)) 
\end{spadsrc}
\begin{TeXOutput}
\begin{fricasmath}{20}
\SYMBOL{\%C}\TIMES \sqrt{\SYMBOL{a}}\TIMES \sqrt{\SYMBOL{b}}\SYMBOL{\ ==\ }%
\SYMBOL{\%C}\TIMES \sqrt{\SYMBOL{a}\TIMES \SYMBOL{b}}%
\end{fricasmath}
\end{TeXOutput}
\formatResultType{RewriteRule(Integer, Integer, Expression(Integer))}
\end{xtc}
\begin{xtc}
\begin{xtccomment}
Here is a test expression.
\end{xtccomment}
\begin{spadsrc}
a := (sqrt(x) + sqrt(y) + sqrt(z))^4 
\end{spadsrc}
\begin{TeXOutput}
\begin{fricasmath}{21}
\PAREN{\PAREN{4\TIMES \SYMBOL{z}+4\TIMES \SYMBOL{y}+12\TIMES \SYMBOL{x}}%
\TIMES \sqrt{\SYMBOL{y}}+\PAREN{4\TIMES \SYMBOL{z}+12\TIMES \SYMBOL{y}+4%
\TIMES \SYMBOL{x}}\TIMES \sqrt{\SYMBOL{x}}}\TIMES \sqrt{\SYMBOL{z}}+\PAREN{12%
\TIMES \SYMBOL{z}+4\TIMES \SYMBOL{y}+4\TIMES \SYMBOL{x}}\TIMES \sqrt{\SYMBOL{%
x}}\TIMES \sqrt{\SYMBOL{y}}+\SUPER{\SYMBOL{z}}{2}+\PAREN{6\TIMES \SYMBOL{y}+6%
\TIMES \SYMBOL{x}}\TIMES \SYMBOL{z}+\SUPER{\SYMBOL{y}}{2}+6\TIMES \SYMBOL{x}%
\TIMES \SYMBOL{y}+\SUPER{\SYMBOL{x}}{2}%
\end{fricasmath}
\end{TeXOutput}
\formatResultType{Expression(Integer)}
\end{xtc}
\begin{xtc}
\begin{xtccomment}
The rule
\userfun{groupSqrt} successfully simplifies the expression.
\end{xtccomment}
\begin{spadsrc}
groupSqrt a 
\end{spadsrc}
\begin{TeXOutput}
\begin{fricasmath}{22}
\PAREN{4\TIMES \SYMBOL{z}+4\TIMES \SYMBOL{y}+12\TIMES \SYMBOL{x}}\TIMES \sqrt%
{\SYMBOL{y}\TIMES \SYMBOL{z}}+\PAREN{4\TIMES \SYMBOL{z}+12\TIMES \SYMBOL{y}+4%
\TIMES \SYMBOL{x}}\TIMES \sqrt{\SYMBOL{x}\TIMES \SYMBOL{z}}+\PAREN{12\TIMES %
\SYMBOL{z}+4\TIMES \SYMBOL{y}+4\TIMES \SYMBOL{x}}\TIMES \sqrt{\SYMBOL{x}%
\TIMES \SYMBOL{y}}+\SUPER{\SYMBOL{z}}{2}+\PAREN{6\TIMES \SYMBOL{y}+6\TIMES %
\SYMBOL{x}}\TIMES \SYMBOL{z}+\SUPER{\SYMBOL{y}}{2}+6\TIMES \SYMBOL{x}\TIMES %
\SYMBOL{y}+\SUPER{\SYMBOL{x}}{2}%
\end{fricasmath}
\end{TeXOutput}
\formatResultType{Expression(Integer)}
\end{xtc}

% *********************************************************************
\head{section}{Polynomials}{ugIntroVariables}
% *********************************************************************
%

Polynomials are the commonly used algebraic types in symbolic
computation.
\index{polynomial}
Interactive users of \Language{} generally only see one type
of polynomial: \spadtype{Polynomial(R)}.
This type represents polynomials in any number of unspecified
variables over a particular coefficient domain \spad{R}.
This type represents its coefficients
\spadglossSee{sparsely}{sparse}: only terms with non-zero
coefficients are represented.
\exptypeindex{Polynomial}

In building applications, many other kinds of polynomial
representations are useful.
Polynomials may have one variable or multiple variables, the
variables can be named or unnamed, the coefficients can be stored
sparsely or densely.
So-called ``distributed multivariate polynomials'' store
polynomials as coefficients paired with vectors of exponents.
This type is particularly efficient for use in algorithms for
solving systems of non-linear polynomial equations.

\begin{xtc}
\begin{xtccomment}
The polynomial constructor most familiar to the interactive user
is \spadtype{Polynomial}.
\end{xtccomment}
\begin{spadsrc}
(x^2 - x*y^3 +3*y)^2
\end{spadsrc}
\begin{TeXOutput}
\begin{fricasmath}{1}
\SUPER{\SYMBOL{x}}{2}\TIMES \SUPER{\SYMBOL{y}}{6}-{6\TIMES \SYMBOL{x}\TIMES %
\SUPER{\SYMBOL{y}}{4}}-{2\TIMES \SUPER{\SYMBOL{x}}{3}\TIMES \SUPER{\SYMBOL{y}%
}{3}}+9\TIMES \SUPER{\SYMBOL{y}}{2}+6\TIMES \SUPER{\SYMBOL{x}}{2}\TIMES %
\SYMBOL{y}+\SUPER{\SYMBOL{x}}{4}%
\end{fricasmath}
\end{TeXOutput}
\formatResultType{Polynomial(Integer)}
\end{xtc}
\begin{xtc}
\begin{xtccomment}
If you wish to restrict the variables used,
\spadtype{UnivariatePolynomial}
provides polynomials in one variable.
\exptypeindex{UnivariatePolynomial}
\end{xtccomment}
\begin{spadsrc}
p: UP(x,INT) := (3*x-1)^2 * (2*x + 8)
\end{spadsrc}
\begin{TeXOutput}
\begin{fricasmath}{2}
18\TIMES \SUPER{\SYMBOL{x}}{3}+60\TIMES \SUPER{\SYMBOL{x}}{2}-{46\TIMES %
\SYMBOL{x}}+8%
\end{fricasmath}
\end{TeXOutput}
\formatResultType{UnivariatePolynomial(x, Integer)}
\end{xtc}
\begin{xtc}
\begin{xtccomment}
The constructor
\spadtype{MultivariatePolynomial} provides polynomials in one or more
specified variables.
\exptypeindex{MultivariatePolynomial}
\end{xtccomment}
\begin{spadsrc}
m: MPOLY([x,y],INT) := (x^2-x*y^3+3*y)^2 
\end{spadsrc}
\begin{TeXOutput}
\begin{fricasmath}{3}
\SUPER{\SYMBOL{x}}{4}-{2\TIMES \SUPER{\SYMBOL{y}}{3}\TIMES \SUPER{\SYMBOL{x}%
}{3}}+\PAREN{\SUPER{\SYMBOL{y}}{6}+6\TIMES \SYMBOL{y}}\TIMES \SUPER{\SYMBOL{x%
}}{2}-{6\TIMES \SUPER{\SYMBOL{y}}{4}\TIMES \SYMBOL{x}}+9\TIMES \SUPER{\SYMBOL%
{y}}{2}%
\end{fricasmath}
\end{TeXOutput}
\formatResultType{MultivariatePolynomial([x, y], Integer)}
\end{xtc}
\begin{xtc}
\begin{xtccomment}
You can change the way the polynomial appears by modifying the variable
ordering in the explicit list.
\end{xtccomment}
\begin{spadsrc}
m :: MPOLY([y,x],INT) 
\end{spadsrc}
\begin{TeXOutput}
\begin{fricasmath}{4}
\SUPER{\SYMBOL{x}}{2}\TIMES \SUPER{\SYMBOL{y}}{6}-{6\TIMES \SYMBOL{x}\TIMES %
\SUPER{\SYMBOL{y}}{4}}-{2\TIMES \SUPER{\SYMBOL{x}}{3}\TIMES \SUPER{\SYMBOL{y}%
}{3}}+9\TIMES \SUPER{\SYMBOL{y}}{2}+6\TIMES \SUPER{\SYMBOL{x}}{2}\TIMES %
\SYMBOL{y}+\SUPER{\SYMBOL{x}}{4}%
\end{fricasmath}
\end{TeXOutput}
\formatResultType{MultivariatePolynomial([y, x], Integer)}
\end{xtc}
\begin{xtc}
\begin{xtccomment}
The constructor
\spadtype{DistributedMultivariatePolynomial} provides
polynomials in one or more specified variables with the monomials
ordered lexicographically.
\exptypeindex{DistributedMultivariatePolynomial}
\end{xtccomment}
\begin{spadsrc}
m :: DMP([y,x],INT) 
\end{spadsrc}
\begin{TeXOutput}
\begin{fricasmath}{5}
\SUPER{\SYMBOL{y}}{6}\TIMES \SUPER{\SYMBOL{x}}{2}-{6\TIMES \SUPER{\SYMBOL{y}%
}{4}\TIMES \SYMBOL{x}}-{2\TIMES \SUPER{\SYMBOL{y}}{3}\TIMES \SUPER{\SYMBOL{x}%
}{3}}+9\TIMES \SUPER{\SYMBOL{y}}{2}+6\TIMES \SYMBOL{y}\TIMES \SUPER{\SYMBOL{x%
}}{2}+\SUPER{\SYMBOL{x}}{4}%
\end{fricasmath}
\end{TeXOutput}
\formatResultType{DistributedMultivariatePolynomial([y, x], Integer)}
\end{xtc}
\begin{xtc}
\begin{xtccomment}
The constructor
\spadtype{HomogeneousDistributedMultivariatePolynomial} is similar except that
the monomials are ordered by total order refined by reverse
lexicographic order.
\exptypeindex{HomogeneousDistributedMultivariatePolynomial}
\end{xtccomment}
\begin{spadsrc}
m :: HDMP([y,x],INT) 
\end{spadsrc}
\begin{TeXOutput}
\begin{fricasmath}{6}
\SUPER{\SYMBOL{y}}{6}\TIMES \SUPER{\SYMBOL{x}}{2}-{2\TIMES \SUPER{\SYMBOL{y}%
}{3}\TIMES \SUPER{\SYMBOL{x}}{3}}-{6\TIMES \SUPER{\SYMBOL{y}}{4}\TIMES %
\SYMBOL{x}}+\SUPER{\SYMBOL{x}}{4}+6\TIMES \SYMBOL{y}\TIMES \SUPER{\SYMBOL{x}%
}{2}+9\TIMES \SUPER{\SYMBOL{y}}{2}%
\end{fricasmath}
\end{TeXOutput}
\formatResultType{HomogeneousDistributedMultivariatePolynomial([y, x], Integer)}
\end{xtc}

More generally, the domain constructor
\spadtype{GeneralDistributedMultivariatePolynomial} allows the
user to provide an arbitrary predicate to define his own term ordering.
\exptypeindex{GeneralDistributedMultivariatePolynomial}
These last three constructors are typically used in
Gr\"{o}bner basis
\index{Groebner basis@{Gr\protect\"{o}bner basis}}
applications and when a flat (that is, non-recursive) display is
wanted and the term ordering is critical for controlling the computation.

% *********************************************************************
\head{section}{Limits}{ugIntroCalcLimits}
% *********************************************************************
%

\Language{}'s \spadfun{limit} function is usually used to
evaluate limits of quotients where the numerator and denominator
\index{limit}
both tend to zero or both tend to infinity.
To find the limit of an expression \spad{f} as a real variable
\spad{x} tends to a limit value \spad{a}, enter \spad{limit(f, x=a)}.
Use \spadfun{complexLimit} if the variable is complex.
Additional information and examples of limits are in
\spadref{ugProblemLimits}.

\begin{xtc}
\begin{xtccomment}
You can take limits of functions with parameters.
\index{limit!of function with parameters}
\end{xtccomment}
\begin{spadsrc}
g := csc(a*x) / csch(b*x) 
\end{spadsrc}
\begin{TeXOutput}
\begin{fricasmath}{1}
\frac{\csc{\PAREN{\SYMBOL{a}\TIMES \SYMBOL{x}}}}{\csch{\PAREN{\SYMBOL{b}%
\TIMES \SYMBOL{x}}}}%
\end{fricasmath}
\end{TeXOutput}
\formatResultType{Expression(Integer)}
\end{xtc}
\begin{xtc}
\begin{xtccomment}
As you can see, the limit is expressed in terms of the parameters.
\end{xtccomment}
\begin{spadsrc}
limit(g,x=0) 
\end{spadsrc}
\begin{TeXOutput}
\begin{fricasmath}{2}
\frac{\SYMBOL{b}}{\SYMBOL{a}}%
\end{fricasmath}
\end{TeXOutput}
\formatResultType{Union(OrderedCompletion(Expression(Integer)), ...)}
\end{xtc}
%
\begin{xtc}
\begin{xtccomment}
A variable may also approach plus or minus infinity:
\end{xtccomment}
\begin{spadsrc}
h := (1 + k/x)^x 
\end{spadsrc}
\begin{TeXOutput}
\begin{fricasmath}{3}
\SUPER{\PAREN{\frac{\SYMBOL{x}+\SYMBOL{k}}{\SYMBOL{x}}}}{\SYMBOL{x}}%
\end{fricasmath}
\end{TeXOutput}
\formatResultType{Expression(Integer)}
\end{xtc}
\begin{xtc}
\begin{xtccomment}
Use \spad{%plusInfinity} and \spad{%minusInfinity} to
denote $\infty$ and $-\infty$.
\end{xtccomment}
\begin{spadsrc}
limit(h,x=%plusInfinity) 
\end{spadsrc}
\begin{TeXOutput}
\begin{fricasmath}{4}
\SUPER{\EulerE }{\SYMBOL{k}}%
\end{fricasmath}
\end{TeXOutput}
\formatResultType{Union(OrderedCompletion(Expression(Integer)), ...)}
\end{xtc}
\begin{xtc}
\begin{xtccomment}
A function can be defined on both sides of a particular value, but
may tend to different limits as its variable approaches that value from the
left and from the right.
\end{xtccomment}
\begin{spadsrc}
limit(sqrt(y^2)/y,y = 0)
\end{spadsrc}
\begin{TeXOutput}
\begin{fricasmath}{5}
\BRACKET{\SYMBOL{leftHandLimit}=-{1}\COMMA \SYMBOL{rightHandLimit}=1}%
\end{fricasmath}
\end{TeXOutput}
\formatResultType{Union(Record(leftHandLimit: Union(OrderedCompletion(Expression(Integer)), "failed"), rightHandLimit: Union(OrderedCompletion(Expression(Integer)), "failed")), ...)}
\end{xtc}
\begin{xtc}
\begin{xtccomment}
As \spad{x} approaches \spad{0} along the real axis, \spad{exp(-1/x^2)}
tends to \spad{0}.
\end{xtccomment}
\begin{spadsrc}
limit(exp(-1/x^2),x = 0)
\end{spadsrc}
\begin{TeXOutput}
\begin{fricasmath}{6}
0%
\end{fricasmath}
\end{TeXOutput}
\formatResultType{Union(OrderedCompletion(Expression(Integer)), ...)}
\end{xtc}
\begin{xtc}
\begin{xtccomment}
However, if \spad{x} is allowed to approach \spad{0} along any path in the
complex plane, the limiting value of \spad{exp(-1/x^2)} depends on the
path taken because the function has an essential singularity at \spad{x=0}.
This is reflected in the error message returned by the function.
\end{xtccomment}
\begin{spadsrc}
complexLimit(exp(-1/x^2),x = 0)
\end{spadsrc}
\begin{TeXOutput}
\begin{fricasmath}{7}
\STRING{"failed"}%
\end{fricasmath}
\end{TeXOutput}
\formatResultType{Union("failed", ...)}
\end{xtc}

% *********************************************************************
\head{section}{Series}{ugIntroSeries}
% *********************************************************************
%

\Language{} also provides power series.
\index{series!power}
By default, \Language{} tries to compute and display the first ten elements
of a series.
Use \spadsys{)set streams calculate} to change the default value
to something else.
\syscmdindex{set streams calculate}
For the purposes of this book, we have used this system command to display
fewer than ten terms.
For more information about working with series, see
\spadref{ugProblemSeries}.

\begin{xtc}
\begin{xtccomment}
You can convert a functional expression to a power series by using the
operation \spadfun{series}.
In this example,
\spad{sin(a*x)} is expanded in powers of \spad{(x - 0)},
that is, in powers of \spad{x}.
\end{xtccomment}
\begin{spadsrc}
series(sin(a*x),x = 0)
\end{spadsrc}
\begin{TeXOutput}
\begin{fricasmath}{1}
\SYMBOL{a}\TIMES \SYMBOL{x}-{\frac{\SUPER{\SYMBOL{a}}{3}}{6}\TIMES \SUPER{%
\SYMBOL{x}}{3}}+\frac{\SUPER{\SYMBOL{a}}{5}}{120}\TIMES \SUPER{\SYMBOL{x}}{5}%
-{\frac{\SUPER{\SYMBOL{a}}{7}}{5040}\TIMES \SUPER{\SYMBOL{x}}{7}}+\FUN{O}%
\PAREN{\SUPER{\SYMBOL{x}}{9}}%
\end{fricasmath}
\end{TeXOutput}
\formatResultType{UnivariatePuiseuxSeries(Expression(Integer), x, 0)}
\end{xtc}
\begin{xtc}
\begin{xtccomment}
This expression expands
\spad{sin(a*x)} in powers of \spad{(x - %pi/4)}.
\end{xtccomment}
\begin{spadsrc}
series(sin(a*x),x = %pi/4)
\end{spadsrc}
\begin{TeXOutput}
\begin{fricasmath}{2}
\sin{\PAREN{\frac{\SYMBOL{a}\TIMES \pi }{4}}}+\SYMBOL{a}\TIMES \cos{\PAREN{%
\frac{\SYMBOL{a}\TIMES \pi }{4}}}\TIMES \PAREN{\SYMBOL{x}-{\frac{\pi }{4}}}-{%
\frac{\SUPER{\SYMBOL{a}}{2}\TIMES \sin{\PAREN{\frac{\SYMBOL{a}\TIMES \pi }{4}%
}}}{2}\TIMES \SUPER{\PAREN{\SYMBOL{x}-{\frac{\pi }{4}}}}{2}}-{\frac{\SUPER{%
\SYMBOL{a}}{3}\TIMES \cos{\PAREN{\frac{\SYMBOL{a}\TIMES \pi }{4}}}}{6}\TIMES %
\SUPER{\PAREN{\SYMBOL{x}-{\frac{\pi }{4}}}}{3}}+\frac{\SUPER{\SYMBOL{a}}{4}%
\TIMES \sin{\PAREN{\frac{\SYMBOL{a}\TIMES \pi }{4}}}}{24}\TIMES \SUPER{\PAREN%
{\SYMBOL{x}-{\frac{\pi }{4}}}}{4}+\frac{\SUPER{\SYMBOL{a}}{5}\TIMES \cos{%
\PAREN{\frac{\SYMBOL{a}\TIMES \pi }{4}}}}{120}\TIMES \SUPER{\PAREN{\SYMBOL{x}%
-{\frac{\pi }{4}}}}{5}-{\frac{\SUPER{\SYMBOL{a}}{6}\TIMES \sin{\PAREN{\frac{%
\SYMBOL{a}\TIMES \pi }{4}}}}{720}\TIMES \SUPER{\PAREN{\SYMBOL{x}-{\frac{\pi %
}{4}}}}{6}}-{\frac{\SUPER{\SYMBOL{a}}{7}\TIMES \cos{\PAREN{\frac{\SYMBOL{a}%
\TIMES \pi }{4}}}}{5040}\TIMES \SUPER{\PAREN{\SYMBOL{x}-{\frac{\pi }{4}}}}{7}%
}+\FUN{O}\PAREN{\SUPER{\PAREN{\SYMBOL{x}-{\frac{\pi }{4}}}}{8}}%
\end{fricasmath}
\end{TeXOutput}
\formatResultType{UnivariatePuiseuxSeries(Expression(Integer), x, \%pi/4)}
\end{xtc}
\begin{xtc}
\begin{xtccomment}
\Language{} provides
\index{series!Puiseux}
{\it Puiseux series:}
\index{Puiseux series}
series with rational number exponents.
The first argument to \spadfun{series} is an in-place function that
computes the \eth{n} coefficient.
(Recall that
the \spadSyntax{+->} is an infix operator meaning ``maps to.'')
\end{xtccomment}
\begin{spadsrc}
series(n +-> (-1)^((3*n - 4)/6)/factorial(n - 1/3),x = 0,4/3..,2)
\end{spadsrc}
\begin{TeXOutput}
\begin{fricasmath}{3}
\SUPER{\SYMBOL{x}}{\frac{4}{3}}-{\frac{1}{6}\TIMES \SUPER{\SYMBOL{x}}{\frac{%
10}{3}}}+\FUN{O}\PAREN{\SUPER{\SYMBOL{x}}{4}}%
\end{fricasmath}
\end{TeXOutput}
\formatResultType{UnivariatePuiseuxSeries(Expression(Integer), x, 0)}
\end{xtc}
\begin{xtc}
\begin{xtccomment}
Once you have created a power series, you can perform arithmetic operations
on that series.
We compute the Taylor expansion of \spad{1/(1-x)}.
\index{series!Taylor}
\end{xtccomment}
\begin{spadsrc}
f := series(1/(1-x),x = 0) 
\end{spadsrc}
\begin{TeXOutput}
\begin{fricasmath}{4}
1+\SYMBOL{x}+\SUPER{\SYMBOL{x}}{2}+\SUPER{\SYMBOL{x}}{3}+\SUPER{\SYMBOL{x}}{4%
}+\SUPER{\SYMBOL{x}}{5}+\SUPER{\SYMBOL{x}}{6}+\SUPER{\SYMBOL{x}}{7}+\FUN{O}%
\PAREN{\SUPER{\SYMBOL{x}}{8}}%
\end{fricasmath}
\end{TeXOutput}
\formatResultType{UnivariatePuiseuxSeries(Expression(Integer), x, 0)}
\end{xtc}
\begin{xtc}
\begin{xtccomment}
Compute the square of the series.
\end{xtccomment}
\begin{spadsrc}
f ^ 2 
\end{spadsrc}
\begin{TeXOutput}
\begin{fricasmath}{5}
1+2\TIMES \SYMBOL{x}+3\TIMES \SUPER{\SYMBOL{x}}{2}+4\TIMES \SUPER{\SYMBOL{x}%
}{3}+5\TIMES \SUPER{\SYMBOL{x}}{4}+6\TIMES \SUPER{\SYMBOL{x}}{5}+7\TIMES %
\SUPER{\SYMBOL{x}}{6}+8\TIMES \SUPER{\SYMBOL{x}}{7}+\FUN{O}\PAREN{\SUPER{%
\SYMBOL{x}}{8}}%
\end{fricasmath}
\end{TeXOutput}
\formatResultType{UnivariatePuiseuxSeries(Expression(Integer), x, 0)}
\end{xtc}
\begin{xtc}
\begin{xtccomment}
The usual elementary functions
(\spadfun{log}, \spadfun{exp}, trigonometric functions, and so on)
are defined for power series.
\end{xtccomment}
\begin{spadsrc}
f := series(1/(1-x),x = 0) 
\end{spadsrc}
\begin{TeXOutput}
\begin{fricasmath}{6}
1+\SYMBOL{x}+\SUPER{\SYMBOL{x}}{2}+\SUPER{\SYMBOL{x}}{3}+\SUPER{\SYMBOL{x}}{4%
}+\SUPER{\SYMBOL{x}}{5}+\SUPER{\SYMBOL{x}}{6}+\SUPER{\SYMBOL{x}}{7}+\FUN{O}%
\PAREN{\SUPER{\SYMBOL{x}}{8}}%
\end{fricasmath}
\end{TeXOutput}
\formatResultType{UnivariatePuiseuxSeries(Expression(Integer), x, 0)}
\end{xtc}
\begin{xtc}
\begin{xtccomment}
\end{xtccomment}
\begin{spadsrc}
g := log(f) 
\end{spadsrc}
\begin{TeXOutput}
\begin{fricasmath}{7}
\SYMBOL{x}+\frac{1}{2}\TIMES \SUPER{\SYMBOL{x}}{2}+\frac{1}{3}\TIMES \SUPER{%
\SYMBOL{x}}{3}+\frac{1}{4}\TIMES \SUPER{\SYMBOL{x}}{4}+\frac{1}{5}\TIMES %
\SUPER{\SYMBOL{x}}{5}+\frac{1}{6}\TIMES \SUPER{\SYMBOL{x}}{6}+\frac{1}{7}%
\TIMES \SUPER{\SYMBOL{x}}{7}+\frac{1}{8}\TIMES \SUPER{\SYMBOL{x}}{8}+\FUN{O}%
\PAREN{\SUPER{\SYMBOL{x}}{9}}%
\end{fricasmath}
\end{TeXOutput}
\formatResultType{UnivariatePuiseuxSeries(Expression(Integer), x, 0)}
\end{xtc}
\begin{xtc}
\begin{xtccomment}
\end{xtccomment}
\begin{spadsrc}
exp(g) 
\end{spadsrc}
\begin{TeXOutput}
\begin{fricasmath}{8}
1+\SYMBOL{x}+\SUPER{\SYMBOL{x}}{2}+\SUPER{\SYMBOL{x}}{3}+\SUPER{\SYMBOL{x}}{4%
}+\SUPER{\SYMBOL{x}}{5}+\SUPER{\SYMBOL{x}}{6}+\SUPER{\SYMBOL{x}}{7}+\FUN{O}%
\PAREN{\SUPER{\SYMBOL{x}}{8}}%
\end{fricasmath}
\end{TeXOutput}
\formatResultType{UnivariatePuiseuxSeries(Expression(Integer), x, 0)}
\end{xtc}
% Warning: currently there are (interpreter) problems with converting
% rational functions and polynomials to power series.
\begin{xtc}
\begin{xtccomment}
Here is a way to obtain numerical approximations of
\spad{e} from the Taylor series expansion of \spad{exp(x)}.
First create the desired Taylor expansion.
\end{xtccomment}
\begin{spadsrc}
f := taylor(exp(x)) 
\end{spadsrc}
\begin{TeXOutput}
\begin{fricasmath}{9}
1+\SYMBOL{x}+\frac{1}{2}\TIMES \SUPER{\SYMBOL{x}}{2}+\frac{1}{6}\TIMES \SUPER%
{\SYMBOL{x}}{3}+\frac{1}{24}\TIMES \SUPER{\SYMBOL{x}}{4}+\frac{1}{120}\TIMES %
\SUPER{\SYMBOL{x}}{5}+\frac{1}{720}\TIMES \SUPER{\SYMBOL{x}}{6}+\frac{1}{5040%
}\TIMES \SUPER{\SYMBOL{x}}{7}+\FUN{O}\PAREN{\SUPER{\SYMBOL{x}}{8}}%
\end{fricasmath}
\end{TeXOutput}
\formatResultType{UnivariateTaylorSeries(Expression(Integer), x, 0)}
\end{xtc}
\begin{xtc}
\begin{xtccomment}
Evaluate the series at the value \spad{1.0}.
% Warning: syntax for evaluating power series may change.
As you see, you get a sequence of partial sums.
\end{xtccomment}
\begin{spadsrc}
eval(f,1.0) 
\end{spadsrc}
\begin{TeXOutput}
\begin{fricasmath}{10}
\BRACKET{\STRING{1.0}\COMMA \STRING{2.0}\COMMA \STRING{2.5}\COMMA \STRING{%
2.666666666666666666666666666666666666667}\COMMA \STRING{%
2.708333333333333333333333333333333333333}\COMMA \STRING{%
2.716666666666666666666666666666666666667}\COMMA \STRING{%
2.718055555555555555555555555555555555556}\COMMA \STRING{...}}%
\end{fricasmath}
\end{TeXOutput}
\formatResultType{Stream(Expression(Float))}
\end{xtc}

% *********************************************************************
\head{section}{Derivatives}{ugIntroCalcDeriv}
% *********************************************************************
%
Use the \Language{} function \spadfun{D} to differentiate an
\index{derivative}
expression.
\index{differentiation}

\vskip 2pc
\begin{xtc}
\begin{xtccomment}
To find the derivative of an expression \spad{f} with respect to a
variable \spad{x}, enter \spad{D(f, x)}.
\end{xtccomment}
\begin{spadsrc}
f := exp exp x 
\end{spadsrc}
\begin{TeXOutput}
\begin{fricasmath}{1}
\SUPER{\EulerE }{\SUPER{\EulerE }{\SYMBOL{x}}}%
\end{fricasmath}
\end{TeXOutput}
\formatResultType{Expression(Integer)}
\end{xtc}
\begin{xtc}
\begin{xtccomment}
\end{xtccomment}
\begin{spadsrc}
D(f, x) 
\end{spadsrc}
\begin{TeXOutput}
\begin{fricasmath}{2}
\SUPER{\EulerE }{\SYMBOL{x}}\TIMES \SUPER{\EulerE }{\SUPER{\EulerE }{\SYMBOL{%
x}}}%
\end{fricasmath}
\end{TeXOutput}
\formatResultType{Expression(Integer)}
\end{xtc}
\begin{xtc}
\begin{xtccomment}
An optional third argument \spad{n} in \spadfun{D} asks
\Language{} for the \eth{n} derivative of \spad{f}.
This finds the fourth derivative of \spad{f} with respect to \spad{x}.
\end{xtccomment}
\begin{spadsrc}
D(f, x, 4) 
\end{spadsrc}
\begin{TeXOutput}
\begin{fricasmath}{3}
\PAREN{\SUPER{\PAREN{\SUPER{\EulerE }{\SYMBOL{x}}}}{4}+6\TIMES \SUPER{\PAREN{%
\SUPER{\EulerE }{\SYMBOL{x}}}}{3}+7\TIMES \SUPER{\PAREN{\SUPER{\EulerE }{%
\SYMBOL{x}}}}{2}+\SUPER{\EulerE }{\SYMBOL{x}}}\TIMES \SUPER{\EulerE }{\SUPER{%
\EulerE }{\SYMBOL{x}}}%
\end{fricasmath}
\end{TeXOutput}
\formatResultType{Expression(Integer)}
\end{xtc}
\begin{xtc}
\begin{xtccomment}
You can also compute partial derivatives by specifying the order of
\index{differentiation!partial}
differentiation.
\end{xtccomment}
\begin{spadsrc}
g := sin(x^2 + y) 
\end{spadsrc}
\begin{TeXOutput}
\begin{fricasmath}{4}
\sin{\PAREN{\SYMBOL{y}+\SUPER{\SYMBOL{x}}{2}}}%
\end{fricasmath}
\end{TeXOutput}
\formatResultType{Expression(Integer)}
\end{xtc}
\begin{xtc}
\begin{xtccomment}
\end{xtccomment}
\begin{spadsrc}
D(g, y) 
\end{spadsrc}
\begin{TeXOutput}
\begin{fricasmath}{5}
\cos{\PAREN{\SYMBOL{y}+\SUPER{\SYMBOL{x}}{2}}}%
\end{fricasmath}
\end{TeXOutput}
\formatResultType{Expression(Integer)}
\end{xtc}
\begin{xtc}
\begin{xtccomment}
\end{xtccomment}
\begin{spadsrc}
D(g, [y, y, x, x]) 
\end{spadsrc}
\begin{TeXOutput}
\begin{fricasmath}{6}
4\TIMES \SUPER{\SYMBOL{x}}{2}\TIMES \sin{\PAREN{\SYMBOL{y}+\SUPER{\SYMBOL{x}%
}{2}}}-{2\TIMES \cos{\PAREN{\SYMBOL{y}+\SUPER{\SYMBOL{x}}{2}}}}%
\end{fricasmath}
\end{TeXOutput}
\formatResultType{Expression(Integer)}
\end{xtc}

\Language{} can manipulate the derivatives (partial and iterated) of
\index{differentiation!formal}
expressions involving formal operators.
All the dependencies must be explicit.
\begin{xtc}
\begin{xtccomment}
This returns \spad{0} since \spad{F} (so far)
does not explicitly depend on \spad{x}.
\end{xtccomment}
\begin{spadsrc}
D(F,x)
\end{spadsrc}
\begin{TeXOutput}
\begin{fricasmath}{7}
0%
\end{fricasmath}
\end{TeXOutput}
\formatResultType{Polynomial(Integer)}
\end{xtc}
Suppose that we have \spad{F} a function of \spad{x},
\spad{y}, and \spad{z}, where \spad{x} and \spad{y} are themselves
functions of \spad{z}.
\begin{xtc}
\begin{xtccomment}
Start by declaring that \spad{F}, \spad{x}, and \spad{y}
are operators.
\index{operator}
\end{xtccomment}
\begin{spadsrc}
F := operator 'F; x := operator 'x; y := operator 'y
\end{spadsrc}
\begin{TeXOutput}
\begin{fricasmath}{8}
\SYMBOL{y}%
\end{fricasmath}
\end{TeXOutput}
\formatResultType{BasicOperator}
\end{xtc}
\begin{xtc}
\begin{xtccomment}
You can use \spad{F}, \spad{x}, and \spad{y} in expressions.
\end{xtccomment}
\begin{spadsrc}
a := F(x z, y z, z^2) + x y(z+1) 
\end{spadsrc}
\begin{TeXOutput}
\begin{fricasmath}{9}
\FUN{x}\PAREN{\FUN{y}\PAREN{\SYMBOL{z}+1}}+\FUN{F}\PAREN{\FUN{x}\PAREN{%
\SYMBOL{z}},\FUN{y}\PAREN{\SYMBOL{z}},\SUPER{\SYMBOL{z}}{2}}%
\end{fricasmath}
\end{TeXOutput}
\formatResultType{Expression(Integer)}
\end{xtc}
\begin{xtc}
\begin{xtccomment}
Differentiate formally with respect to \spad{z}.
The formal derivatives appearing in \spad{dadz} are not just formal symbols,
but do represent the derivatives of \spad{x}, \spad{y}, and \spad{F}.
\end{xtccomment}
\begin{spadsrc}
dadz := D(a, z)
\end{spadsrc}
\begin{TeXOutput}
\begin{fricasmath}{10}
2\TIMES \SYMBOL{z}\TIMES \SUB{\SYMBOL{F}}{\SYMBOL{,}3}\PAREN{\FUN{x}\PAREN{%
\SYMBOL{z}},\FUN{y}\PAREN{\SYMBOL{z}},\SUPER{\SYMBOL{z}}{2}}+\PRIME{\SYMBOL{y%
}}{\STRING{,}}\PAREN{\SYMBOL{z}}\TIMES \SUB{\SYMBOL{F}}{\SYMBOL{,}2}\PAREN{%
\FUN{x}\PAREN{\SYMBOL{z}},\FUN{y}\PAREN{\SYMBOL{z}},\SUPER{\SYMBOL{z}}{2}}+%
\PRIME{\SYMBOL{x}}{\STRING{,}}\PAREN{\SYMBOL{z}}\TIMES \SUB{\SYMBOL{F}}{%
\SYMBOL{,}1}\PAREN{\FUN{x}\PAREN{\SYMBOL{z}},\FUN{y}\PAREN{\SYMBOL{z}},\SUPER%
{\SYMBOL{z}}{2}}+\PRIME{\SYMBOL{x}}{\STRING{,}}\PAREN{\FUN{y}\PAREN{\SYMBOL{z%
}+1}}\TIMES \PRIME{\SYMBOL{y}}{\STRING{,}}\PAREN{\SYMBOL{z}+1}%
\end{fricasmath}
\end{TeXOutput}
\formatResultType{Expression(Integer)}
\end{xtc}
\begin{xtc}
\begin{xtccomment}
You can evaluate the above for particular functional
values of \spad{F}, \spad{x}, and \spad{y}.
If \spad{x(z)} is \spad{exp(z)} and \spad{y(z)} is \spad{log(z+1)}, then
this evaluates \spad{dadz}.
\end{xtccomment}
\begin{spadsrc}
eval(eval(dadz, 'x, z +-> exp z), 'y, z +-> log(z+1))
\end{spadsrc}
\begin{TeXOutput}
\begin{fricasmath}{11}
\frac{\PAREN{2\TIMES \SUPER{\SYMBOL{z}}{2}+2\TIMES \SYMBOL{z}}\TIMES \SUB{%
\SYMBOL{F}}{\SYMBOL{,}3}\PAREN{\SUPER{\EulerE }{\SYMBOL{z}},\log{\PAREN{%
\SYMBOL{z}+1}},\SUPER{\SYMBOL{z}}{2}}+\SUB{\SYMBOL{F}}{\SYMBOL{,}2}\PAREN{%
\SUPER{\EulerE }{\SYMBOL{z}},\log{\PAREN{\SYMBOL{z}+1}},\SUPER{\SYMBOL{z}}{2}%
}+\PAREN{\SYMBOL{z}+1}\TIMES \SUPER{\EulerE }{\SYMBOL{z}}\TIMES \SUB{\SYMBOL{%
F}}{\SYMBOL{,}1}\PAREN{\SUPER{\EulerE }{\SYMBOL{z}},\log{\PAREN{\SYMBOL{z}+1}%
},\SUPER{\SYMBOL{z}}{2}}+\SYMBOL{z}+1}{\SYMBOL{z}+1}%
\end{fricasmath}
\end{TeXOutput}
\formatResultType{Expression(Integer)}
\end{xtc}
\begin{xtc}
\begin{xtccomment}
You obtain the same result by first evaluating \spad{a} and
then differentiating.
\end{xtccomment}
\begin{spadsrc}
eval(eval(a, 'x, z +-> exp z), 'y, z +-> log(z+1)) 
\end{spadsrc}
\begin{TeXOutput}
\begin{fricasmath}{12}
\FUN{F}\PAREN{\SUPER{\EulerE }{\SYMBOL{z}},\log{\PAREN{\SYMBOL{z}+1}},\SUPER{%
\SYMBOL{z}}{2}}+\SYMBOL{z}+2%
\end{fricasmath}
\end{TeXOutput}
\formatResultType{Expression(Integer)}
\end{xtc}
\begin{xtc}
\begin{xtccomment}
\end{xtccomment}
\begin{spadsrc}
D(%, z)
\end{spadsrc}
\begin{TeXOutput}
\begin{fricasmath}{13}
\frac{\PAREN{2\TIMES \SUPER{\SYMBOL{z}}{2}+2\TIMES \SYMBOL{z}}\TIMES \SUB{%
\SYMBOL{F}}{\SYMBOL{,}3}\PAREN{\SUPER{\EulerE }{\SYMBOL{z}},\log{\PAREN{%
\SYMBOL{z}+1}},\SUPER{\SYMBOL{z}}{2}}+\SUB{\SYMBOL{F}}{\SYMBOL{,}2}\PAREN{%
\SUPER{\EulerE }{\SYMBOL{z}},\log{\PAREN{\SYMBOL{z}+1}},\SUPER{\SYMBOL{z}}{2}%
}+\PAREN{\SYMBOL{z}+1}\TIMES \SUPER{\EulerE }{\SYMBOL{z}}\TIMES \SUB{\SYMBOL{%
F}}{\SYMBOL{,}1}\PAREN{\SUPER{\EulerE }{\SYMBOL{z}},\log{\PAREN{\SYMBOL{z}+1}%
},\SUPER{\SYMBOL{z}}{2}}+\SYMBOL{z}+1}{\SYMBOL{z}+1}%
\end{fricasmath}
\end{TeXOutput}
\formatResultType{Expression(Integer)}
\end{xtc}

% *********************************************************************
\head{section}{Integration}{ugIntroIntegrate}
% *********************************************************************
%

\Language{} has extensive library facilities for integration.
\index{integration}

The first example is the integration of a fraction with
denominator that factors into a quadratic and a quartic
irreducible polynomial.
The usual partial fraction approach used by most other computer
algebra systems either fails or introduces expensive unneeded
algebraic numbers.

\begin{xtc}
\begin{xtccomment}
We use a factorization-free algorithm.
\end{xtccomment}
\begin{spadsrc}
integrate((x^2+2*x+1)/((x+1)^6+1),x)
\end{spadsrc}
\begin{TeXOutput}
\begin{fricasmath}{1}
\frac{\arctan{\PAREN{\SUPER{\SYMBOL{x}}{3}+3\TIMES \SUPER{\SYMBOL{x}}{2}+3%
\TIMES \SYMBOL{x}+1}}}{3}%
\end{fricasmath}
\end{TeXOutput}
\formatResultType{Union(Expression(Integer), ...)}
\end{xtc}

When real parameters are present, the form of the integral can depend on
the signs of some expressions.

\begin{xtc}
\begin{xtccomment}
Rather than query the user or make sign assumptions, \Language{} returns
all possible answers.
\end{xtccomment}
\begin{spadsrc}
integrate(1/(x^2 + a),x)
\end{spadsrc}
\begin{TeXOutput}
\begin{fricasmath}{2}
\BRACKET{\frac{\log{\PAREN{\frac{\PAREN{\SUPER{\SYMBOL{x}}{2}-{\SYMBOL{a}}}%
\TIMES \sqrt{-{\SYMBOL{a}}}+2\TIMES \SYMBOL{a}\TIMES \SYMBOL{x}}{\SUPER{%
\SYMBOL{x}}{2}+\SYMBOL{a}}}}}{2\TIMES \sqrt{-{\SYMBOL{a}}}}\COMMA \frac{%
\arctan{\PAREN{\frac{\SYMBOL{x}\TIMES \sqrt{\SYMBOL{a}}}{\SYMBOL{a}}}}}{\sqrt%
{\SYMBOL{a}}}}%
\end{fricasmath}
\end{TeXOutput}
\formatResultType{Union(List(Expression(Integer)), ...)}
\end{xtc}

The \spadfun{integrate} operation generally assumes that all
parameters are real.
The only exception is when the integrand has complex valued
quantities.

\begin{xtc}
\begin{xtccomment}
If the parameter is complex instead of real, then the notion of sign is
undefined and there is a unique answer.
You can request this answer by ``prepending'' the word ``complex'' to the
command name:
\end{xtccomment}
\begin{spadsrc}
complexIntegrate(1/(x^2 + a),x)
\end{spadsrc}
\begin{TeXOutput}
\begin{fricasmath}{3}
\frac{\sqrt{-{\frac{1}{\SYMBOL{a}}}}\TIMES \log{\PAREN{\SYMBOL{a}\TIMES \sqrt%
{-{\frac{1}{\SYMBOL{a}}}}+\SYMBOL{x}}}-{\sqrt{-{\frac{1}{\SYMBOL{a}}}}\TIMES %
\log{\PAREN{-{\SYMBOL{a}\TIMES \sqrt{-{\frac{1}{\SYMBOL{a}}}}}+\SYMBOL{x}}}}%
}{2}%
\end{fricasmath}
\end{TeXOutput}
\formatResultType{Expression(Integer)}
\end{xtc}

The following two examples illustrate the limitations of
table-based approaches.
The two integrands are very similar, but the answer to one of them
requires the addition of two new algebraic numbers.

\begin{xtc}
\begin{xtccomment}
This one is the easy one.
The next one looks very similar
but the answer is much more complicated.
\end{xtccomment}
\begin{spadsrc}
integrate(x^3 / (a+b*x)^(1/3),x)
\end{spadsrc}
\begin{TeXOutput}
\begin{fricasmath}{4}
\frac{\PAREN{120\TIMES \SUPER{\SYMBOL{b}}{3}\TIMES \SUPER{\SYMBOL{x}}{3}-{135%
\TIMES \SYMBOL{a}\TIMES \SUPER{\SYMBOL{b}}{2}\TIMES \SUPER{\SYMBOL{x}}{2}}+%
162\TIMES \SUPER{\SYMBOL{a}}{2}\TIMES \SYMBOL{b}\TIMES \SYMBOL{x}-{243\TIMES %
\SUPER{\SYMBOL{a}}{3}}}\TIMES \SUPER{\nthroot{\SYMBOL{b}\TIMES \SYMBOL{x}+%
\SYMBOL{a}}{3}}{2}}{440\TIMES \SUPER{\SYMBOL{b}}{4}}%
\end{fricasmath}
\end{TeXOutput}
\formatResultType{Union(Expression(Integer), ...)}
\end{xtc}
\begin{xtc}
\begin{xtccomment}
Only an algorithmic approach
is guaranteed to find what new constants must be added in order to
find a solution.
\end{xtccomment}
\begin{spadsrc}
integrate(1 / (x^3 * (a+b*x)^(1/3)),x)
\end{spadsrc}
\begin{TeXOutput}
\begin{fricasmath}{5}
\frac{-{2\TIMES \SUPER{\SYMBOL{b}}{2}\TIMES \SUPER{\SYMBOL{x}}{2}\TIMES \sqrt%
{3}\TIMES \log{\PAREN{\nthroot{\SYMBOL{a}}{3}\TIMES \SUPER{\nthroot{\SYMBOL{b%
}\TIMES \SYMBOL{x}+\SYMBOL{a}}{3}}{2}+\SUPER{\nthroot{\SYMBOL{a}}{3}}{2}%
\TIMES \nthroot{\SYMBOL{b}\TIMES \SYMBOL{x}+\SYMBOL{a}}{3}+\SYMBOL{a}}}}+4%
\TIMES \SUPER{\SYMBOL{b}}{2}\TIMES \SUPER{\SYMBOL{x}}{2}\TIMES \sqrt{3}%
\TIMES \log{\PAREN{\SUPER{\nthroot{\SYMBOL{a}}{3}}{2}\TIMES \nthroot{\SYMBOL{%
b}\TIMES \SYMBOL{x}+\SYMBOL{a}}{3}-{\SYMBOL{a}}}}+12\TIMES \SUPER{\SYMBOL{b}%
}{2}\TIMES \SUPER{\SYMBOL{x}}{2}\TIMES \arctan{\PAREN{\frac{2\TIMES \sqrt{3}%
\TIMES \SUPER{\nthroot{\SYMBOL{a}}{3}}{2}\TIMES \nthroot{\SYMBOL{b}\TIMES %
\SYMBOL{x}+\SYMBOL{a}}{3}+\SYMBOL{a}\TIMES \sqrt{3}}{3\TIMES \SYMBOL{a}}}}+%
\PAREN{12\TIMES \SYMBOL{b}\TIMES \SYMBOL{x}-{9\TIMES \SYMBOL{a}}}\TIMES \sqrt%
{3}\TIMES \nthroot{\SYMBOL{a}}{3}\TIMES \SUPER{\nthroot{\SYMBOL{b}\TIMES %
\SYMBOL{x}+\SYMBOL{a}}{3}}{2}}{18\TIMES \SUPER{\SYMBOL{a}}{2}\TIMES \SUPER{%
\SYMBOL{x}}{2}\TIMES \sqrt{3}\TIMES \nthroot{\SYMBOL{a}}{3}}%
\end{fricasmath}
\end{TeXOutput}
\formatResultType{Union(Expression(Integer), ...)}
\end{xtc}

Some computer algebra systems use heuristics or table-driven
approaches to integration.
When these systems cannot determine the answer to an integration
problem, they reply ``I don't know.'' \Language{} uses a
algorithm for integration.
that conclusively proves that an integral cannot be expressed in
terms of elementary functions.

\begin{xtc}
\begin{xtccomment}
When \Language{} returns an integral sign, it has proved
that no answer exists as an elementary function.
\end{xtccomment}
\begin{spadsrc}
integrate(log(1 + sqrt(a*x + b)) / x,x)
\end{spadsrc}
\begin{TeXOutput}
\begin{fricasmath}{6}
\int^{\SYMBOL{x}} \frac{\log{\PAREN{\sqrt{\SYMBOL{b}+\SYMBOL{\%D}\TIMES %
\SYMBOL{a}}+1}}}{\SYMBOL{\%D}}\TIMES \SYMBOL{d}\SYMBOL{\%D}%
\end{fricasmath}
\end{TeXOutput}
\formatResultType{Union(Expression(Integer), ...)}
\end{xtc}
\Language{} can handle complicated mixed functions much beyond what you
can find in tables.
\begin{xtc}
\begin{xtccomment}
Whenever possible, \Language{} tries to express the answer using the functions
present in the integrand.
\end{xtccomment}
\begin{spadsrc}
integrate((sinh(1+sqrt(x+b))+2*sqrt(x+b)) / (sqrt(x+b) * (x + cosh(1+sqrt(x + b)))), x)
\end{spadsrc}
\begin{TeXOutput}
\begin{fricasmath}{7}
2\TIMES \log{\PAREN{\frac{-{2\TIMES \cosh{\PAREN{\sqrt{\SYMBOL{x}+\SYMBOL{b}}%
+1}}}-{2\TIMES \SYMBOL{x}}}{\sinh{\PAREN{\sqrt{\SYMBOL{x}+\SYMBOL{b}}+1}}-{%
\cosh{\PAREN{\sqrt{\SYMBOL{x}+\SYMBOL{b}}+1}}}}}}-{2\TIMES \sqrt{\SYMBOL{x}+%
\SYMBOL{b}}}%
\end{fricasmath}
\end{TeXOutput}
\formatResultType{Union(Expression(Integer), ...)}
\end{xtc}
\begin{xtc}
\begin{xtccomment}
A strong structure-checking algorithm in \Language{} finds hidden algebraic
relationships between functions.
\end{xtccomment}
\begin{spadsrc}
integrate(tan(atan(x)/3),x)
\end{spadsrc}
\begin{TeXOutput}
\begin{fricasmath}{8}
\frac{8\TIMES \log{\PAREN{3\TIMES \SUPER{\PAREN{\tan{\PAREN{\frac{\arctan{%
\SYMBOL{x}}}{3}}}}}{2}-{1}}}-{3\TIMES \SUPER{\PAREN{\tan{\PAREN{\frac{\arctan%
{\SYMBOL{x}}}{3}}}}}{2}}+18\TIMES \SYMBOL{x}\TIMES \tan{\PAREN{\frac{\arctan{%
\SYMBOL{x}}}{3}}}}{18}%
\end{fricasmath}
\end{TeXOutput}
\formatResultType{Union(Expression(Integer), ...)}
\end{xtc}
\noindent
%%--> Bob---> please make these formulas in this section smaller.
The discovery of this algebraic relationship is necessary for correct
integration of this function.
Here are the details:
\begin{enumerate}
\item
If $x=\tan t$ and
$g=\tan (t/3)$ then the following
algebraic relation is true:
$${g^3-3xg^2-3g+x=0}$$
\item
Integrate \spad{g} using this algebraic relation; this produces:
\begin{displaymath}
\frac{(24g^2 - 8)\log(3g^2 - 1) + (81x^2 + 24)g^2 + 72xg - 27x^2 - 16}%
  {54g^2 - 18}
\end{displaymath}
\item
Rationalize the denominator, producing:
\begin{displaymath}
\frac{8\log(3g^2-1) - 3g^2 + 18xg + 16}{18}
\end{displaymath}
Replace \spad{g} by the initial definition
$g = \tan(\arctan(x)/3)$
to produce the final result.
\end{enumerate}

\begin{xtc}
\begin{xtccomment}
This is an example of a mixed function where
the algebraic layer is over the transcendental one.
\end{xtccomment}
\begin{spadsrc}
integrate((x + 1) / (x*(x + log x) ^ (3/2)), x)
\end{spadsrc}
\begin{TeXOutput}
\begin{fricasmath}{9}
-{\frac{2\TIMES \sqrt{\log{\SYMBOL{x}}+\SYMBOL{x}}}{\log{\SYMBOL{x}}+\SYMBOL{%
x}}}%
\end{fricasmath}
\end{TeXOutput}
\formatResultType{Union(Expression(Integer), ...)}
\end{xtc}
\begin{xtc}
\begin{xtccomment}
While incomplete for non-elementary functions, \Language{} can
handle some of them.
\end{xtccomment}
\begin{spadsrc}
integrate(exp(-x^2) * erf(x) / (erf(x)^3 - erf(x)^2 - erf(x) + 1),x)
\end{spadsrc}
\begin{TeXOutput}
\begin{fricasmath}{10}
\frac{\PAREN{\erf{\SYMBOL{x}}-{1}}\TIMES \sqrt{\pi }\TIMES \log{\PAREN{\frac{%
\erf{\SYMBOL{x}}-{1}}{\erf{\SYMBOL{x}}+1}}}-{2\TIMES \sqrt{\pi }}}{8\TIMES %
\erf{\SYMBOL{x}}-{8}}%
\end{fricasmath}
\end{TeXOutput}
\formatResultType{Union(Expression(Integer), ...)}
\end{xtc}

More examples of \Language{}'s integration capabilities are discussed in
\spadref{ugProblemIntegration}.

% *********************************************************************
\head{section}{Differential Equations}{ugIntroDiffEqns}
% *********************************************************************
%
The general approach used in integration also carries over to the
solution of linear differential equations.

\begin{xtc}
\begin{xtccomment}
Let's solve some differential equations.
Let \spad{y} be the unknown function in terms of \spad{x}.
\end{xtccomment}
\begin{spadsrc}
y := operator 'y 
\end{spadsrc}
\begin{TeXOutput}
\begin{fricasmath}{1}
\SYMBOL{y}%
\end{fricasmath}
\end{TeXOutput}
\formatResultType{BasicOperator}
\end{xtc}
\begin{xtc}
\begin{xtccomment}
Here we solve a third order equation with polynomial coefficients.
\end{xtccomment}
\begin{spadsrc}
deq := x^3 * D(y x, x, 3) + x^2 * D(y x, x, 2) - 2 * x * D(y x, x) + 2 * y x = 2 * x^4 
\end{spadsrc}
\begin{TeXOutput}
\begin{fricasmath}{2}
\SUPER{\SYMBOL{x}}{3}\TIMES \PRIME{\SYMBOL{y}}{\STRING{,,,}}\PAREN{\SYMBOL{x}%
}+\SUPER{\SYMBOL{x}}{2}\TIMES \PRIME{\SYMBOL{y}}{\STRING{,,}}\PAREN{\SYMBOL{x%
}}-{2\TIMES \SYMBOL{x}\TIMES \PRIME{\SYMBOL{y}}{\STRING{,}}\PAREN{\SYMBOL{x}}%
}+2\TIMES \FUN{y}\PAREN{\SYMBOL{x}}=2\TIMES \SUPER{\SYMBOL{x}}{4}%
\end{fricasmath}
\end{TeXOutput}
\formatResultType{Equation(Expression(Integer))}
\end{xtc}
\begin{xtc}
\begin{xtccomment}
\end{xtccomment}
\begin{spadsrc}
solve(deq, y, x) 
\end{spadsrc}
\begin{TeXOutput}
\begin{fricasmath}{3}
\BRACKET{\SYMBOL{particular}=\frac{\SUPER{\SYMBOL{x}}{5}-{10\TIMES \SUPER{%
\SYMBOL{x}}{3}}+20\TIMES \SUPER{\SYMBOL{x}}{2}+4}{15\TIMES \SYMBOL{x}}\COMMA %
\SYMBOL{basis}=\BRACKET{\frac{2\TIMES \SUPER{\SYMBOL{x}}{3}-{3\TIMES \SUPER{%
\SYMBOL{x}}{2}}+1}{\SYMBOL{x}}\COMMA \frac{\SUPER{\SYMBOL{x}}{3}-{1}}{\SYMBOL%
{x}}\COMMA \frac{\SUPER{\SYMBOL{x}}{3}-{3\TIMES \SUPER{\SYMBOL{x}}{2}}-{1}}{%
\SYMBOL{x}}}}%
\end{fricasmath}
\end{TeXOutput}
\formatResultType{Union(Record(particular: Expression(Integer), basis: List(Expression(Integer))), ...)}
\end{xtc}
\begin{xtc}
\begin{xtccomment}
Here we find all the algebraic function solutions of the equation.
\end{xtccomment}
\begin{spadsrc}
deq := (x^2 + 1) * D(y x, x, 2) + 3 * x * D(y x, x) + y x = 0 
\end{spadsrc}
\begin{TeXOutput}
\begin{fricasmath}{4}
\PAREN{\SUPER{\SYMBOL{x}}{2}+1}\TIMES \PRIME{\SYMBOL{y}}{\STRING{,,}}\PAREN{%
\SYMBOL{x}}+3\TIMES \SYMBOL{x}\TIMES \PRIME{\SYMBOL{y}}{\STRING{,}}\PAREN{%
\SYMBOL{x}}+\FUN{y}\PAREN{\SYMBOL{x}}=0%
\end{fricasmath}
\end{TeXOutput}
\formatResultType{Equation(Expression(Integer))}
\end{xtc}
\begin{xtc}
\begin{xtccomment}
\end{xtccomment}
\begin{spadsrc}
solve(deq, y, x) 
\end{spadsrc}
\begin{TeXOutput}
\begin{fricasmath}{5}
\BRACKET{\SYMBOL{particular}=0\COMMA \SYMBOL{basis}=\BRACKET{\frac{1}{\sqrt{%
\SUPER{\SYMBOL{x}}{2}+1}}\COMMA \frac{\log{\PAREN{\sqrt{\SUPER{\SYMBOL{x}}{2}%
+1}-{\SYMBOL{x}}}}}{\sqrt{\SUPER{\SYMBOL{x}}{2}+1}}}}%
\end{fricasmath}
\end{TeXOutput}
\formatResultType{Union(Record(particular: Expression(Integer), basis: List(Expression(Integer))), ...)}
\end{xtc}

Coefficients of differential equations can come from arbitrary
constant fields.
For example, coefficients can contain algebraic numbers.

\begin{xtc}
\begin{xtccomment}
This example has solutions
whose logarithmic derivative is an algebraic function of
degree two.
\end{xtccomment}
\begin{spadsrc}
eq := 2*x^3 * D(y x,x,2) + 3*x^2 * D(y x,x) - 2 * y x
\end{spadsrc}
\begin{TeXOutput}
\begin{fricasmath}{6}
2\TIMES \SUPER{\SYMBOL{x}}{3}\TIMES \PRIME{\SYMBOL{y}}{\STRING{,,}}\PAREN{%
\SYMBOL{x}}+3\TIMES \SUPER{\SYMBOL{x}}{2}\TIMES \PRIME{\SYMBOL{y}}{\STRING{,}%
}\PAREN{\SYMBOL{x}}-{2\TIMES \FUN{y}\PAREN{\SYMBOL{x}}}%
\end{fricasmath}
\end{TeXOutput}
\formatResultType{Expression(Integer)}
\end{xtc}
\begin{xtc}
\begin{xtccomment}
\end{xtccomment}
\begin{spadsrc}
solve(eq,y,x).basis
\end{spadsrc}
\begin{TeXOutput}
\begin{fricasmath}{7}
\BRACKET{\SUPER{\EulerE }{-{\frac{2}{\sqrt{\SYMBOL{x}}}}}\COMMA \SUPER{%
\EulerE }{\frac{2}{\sqrt{\SYMBOL{x}}}}}%
\end{fricasmath}
\end{TeXOutput}
\formatResultType{List(Expression(Integer))}
\end{xtc}

\begin{xtc}
\begin{xtccomment}
Here's another differential equation to solve.
\end{xtccomment}
\begin{spadsrc}
deq := D(y x, x) = y(x) / (x + y(x) * log y x) 
\end{spadsrc}
\begin{TeXOutput}
\begin{fricasmath}{8}
\PRIME{\SYMBOL{y}}{\STRING{,}}\PAREN{\SYMBOL{x}}=\frac{\FUN{y}\PAREN{\SYMBOL{%
x}}}{\FUN{y}\PAREN{\SYMBOL{x}}\TIMES \log{\FUN{y}\PAREN{\SYMBOL{x}}}+\SYMBOL{%
x}}%
\end{fricasmath}
\end{TeXOutput}
\formatResultType{Equation(Expression(Integer))}
\end{xtc}
\begin{xtc}
\begin{xtccomment}
\end{xtccomment}
\begin{spadsrc}
solve(deq, y, x) 
\end{spadsrc}
\begin{TeXOutput}
\begin{fricasmath}{9}
\frac{\FUN{y}\PAREN{\SYMBOL{x}}\TIMES \SUPER{\PAREN{\log{\FUN{y}\PAREN{%
\SYMBOL{x}}}}}{2}-{2\TIMES \SYMBOL{x}}}{2\TIMES \FUN{y}\PAREN{\SYMBOL{x}}}%
\end{fricasmath}
\end{TeXOutput}
\formatResultType{Union(Expression(Integer), ...)}
\end{xtc}

Rather than attempting to get a closed form solution of
a differential equation, you instead might want to find an
approximate solution in the form of a series.

\begin{xtc}
\begin{xtccomment}
Let's solve a system of nonlinear first order equations and get a
solution in power series.
Tell \Language{} that \spad{x} is also an operator.
\end{xtccomment}
\begin{spadsrc}
x := operator 'x
\end{spadsrc}
\begin{TeXOutput}
\begin{fricasmath}{10}
\SYMBOL{x}%
\end{fricasmath}
\end{TeXOutput}
\formatResultType{BasicOperator}
\end{xtc}
\begin{xtc}
\begin{xtccomment}
Here are the two equations forming our system.
\end{xtccomment}
\begin{spadsrc}
eq1 := D(x(t), t) = 1 + x(t)^2
\end{spadsrc}
\begin{TeXOutput}
\begin{fricasmath}{11}
\PRIME{\SYMBOL{x}}{\STRING{,}}\PAREN{\SYMBOL{t}}=\SUPER{\FUN{x}\PAREN{\SYMBOL%
{t}}}{2}+1%
\end{fricasmath}
\end{TeXOutput}
\formatResultType{Equation(Expression(Integer))}
\end{xtc}
\begin{xtc}
\begin{xtccomment}
\end{xtccomment}
\begin{spadsrc}
eq2 := D(y(t), t) = x(t) * y(t)
\end{spadsrc}
\begin{TeXOutput}
\begin{fricasmath}{12}
\PRIME{\SYMBOL{y}}{\STRING{,}}\PAREN{\SYMBOL{t}}=\FUN{x}\PAREN{\SYMBOL{t}}%
\TIMES \FUN{y}\PAREN{\SYMBOL{t}}%
\end{fricasmath}
\end{TeXOutput}
\formatResultType{Equation(Expression(Integer))}
\end{xtc}
\begin{xtc}
\begin{xtccomment}
We can solve the system around \spad{t = 0} with the initial conditions
\spad{x(0) = 0} and \spad{y(0) = 1}.
Notice that since we give the unknowns in the
order \spad{[x, y]}, the answer is a list of two series in the order
\spad{[series for x(t), series for y(t)]}.
\end{xtccomment}
\begin{spadsrc}
seriesSolve([eq2, eq1], [x, y], t = 0, [y(0) = 1, x(0) = 0])
\end{spadsrc}
\begin{MessageOutput}
   Compiling function %IX with type List(UnivariateTaylorSeries(
      Expression(Integer),t,0)) -> UnivariateTaylorSeries(Expression(
      Integer),t,0) 
\end{MessageOutput}
\begin{MessageOutput}
   Compiling function %IY with type List(UnivariateTaylorSeries(
      Expression(Integer),t,0)) -> UnivariateTaylorSeries(Expression(
      Integer),t,0) 
\end{MessageOutput}
\begin{TeXOutput}
\begin{fricasmath}{13}
\BRACKET{\SYMBOL{t}+\frac{1}{3}\TIMES \SUPER{\SYMBOL{t}}{3}+\frac{2}{15}%
\TIMES \SUPER{\SYMBOL{t}}{5}+\frac{17}{315}\TIMES \SUPER{\SYMBOL{t}}{7}+\FUN{%
O}\PAREN{\SUPER{\SYMBOL{t}}{8}}\COMMA 1+\frac{1}{2}\TIMES \SUPER{\SYMBOL{t}}{%
2}+\frac{5}{24}\TIMES \SUPER{\SYMBOL{t}}{4}+\frac{61}{720}\TIMES \SUPER{%
\SYMBOL{t}}{6}+\FUN{O}\PAREN{\SUPER{\SYMBOL{t}}{8}}}%
\end{fricasmath}
\end{TeXOutput}
\formatResultType{List(UnivariateTaylorSeries(Expression(Integer), t, 0))}
\end{xtc}

% *********************************************************************
\head{section}{Solution of Equations}{ugIntroSolution}
% *********************************************************************
%

\Language{} also has state-of-the-art algorithms for the solution
of systems of polynomial equations.
When the number of equations and unknowns is the same, and you
have no symbolic coefficients, you can use \spadfun{solve} for
real roots and \spadfun{complexSolve} for complex roots.
In each case, you tell \Language{} how accurate you want your
result to be.
All operations in the \spadfun{solve} family return answers in
the form of a list of solution sets, where each solution set is a
list of equations.

\begin{xtc}
\begin{xtccomment}
A system of two equations involving a symbolic
parameter \spad{t}.
\end{xtccomment}
\begin{spadsrc}
S(t) == [x^2-2*y^2 - t,x*y-y-5*x + 5]
\end{spadsrc}
\end{xtc}
\begin{xtc}
\begin{xtccomment}
Find the real roots of \spad{S(19)} with
rational arithmetic, correct to within \smath{1/10^{20}}.
\end{xtccomment}
\begin{spadsrc}
solve(S(19),1/10^20)
\end{spadsrc}
\begin{MessageOutput}
   Compiling function S with type PositiveInteger -> List(Polynomial(
      Integer)) 
\end{MessageOutput}
\begin{TeXOutput}
\begin{fricasmath}{2}
\BRACKET{\BRACKET{\SYMBOL{y}=5\COMMA \SYMBOL{x}=-{\frac{%
80336736493669365924 189585}{96714065569170333976 49408}}}\COMMA \BRACKET{%
\SYMBOL{y}=5\COMMA \SYMBOL{x}=\frac{80336736493669365924 189585}{%
96714065569170333976 49408}}}%
\end{fricasmath}
\end{TeXOutput}
\formatResultType{List(List(Equation(Polynomial(Fraction(Integer)))))}
\end{xtc}
\begin{xtc}
\begin{xtccomment}
Find the complex roots of \spad{S(19)} with floating
point coefficients to \spad{20} digits accuracy in the mantissa.
\end{xtccomment}
\begin{spadsrc}
complexSolve(S(19),10.e-20)
\end{spadsrc}
\begin{TeXOutput}
\begin{fricasmath}{3}
\BRACKET{\BRACKET{\SYMBOL{y}=\STRING{5.0}\COMMA \SYMBOL{x}=\STRING{%
8.306623862918074852584262744905695155698151691481840582865006639146088}}%
\COMMA \BRACKET{\SYMBOL{y}=\STRING{5.0}\COMMA \SYMBOL{x}=-{\STRING{%
8.306623862918074852584262744905695155698}}}\COMMA \BRACKET{\SYMBOL{y}=-{%
\STRING{3.0}\TIMES \ImaginaryI }\COMMA \SYMBOL{x}=\STRING{1.0}}\COMMA %
\BRACKET{\SYMBOL{y}=\STRING{3.0}\TIMES \ImaginaryI \COMMA \SYMBOL{x}=\STRING{%
1.0}}}%
\end{fricasmath}
\end{TeXOutput}
\formatResultType{List(List(Equation(Polynomial(Complex(Float)))))}
\end{xtc}
\begin{xtc}
\begin{xtccomment}
If a system of equations has symbolic coefficients and you want
a solution in radicals, try \spadfun{radicalSolve}.
\end{xtccomment}
\begin{spadsrc}
radicalSolve(S(a),[x,y])
\end{spadsrc}
\begin{MessageOutput}
   Compiling function S with type Variable(a) -> List(Polynomial(
      Integer)) 
\end{MessageOutput}
\begin{TeXOutput}
\begin{fricasmath}{4}
\BRACKET{\BRACKET{\SYMBOL{x}=-{\sqrt{\SYMBOL{a}+50}}\COMMA \SYMBOL{y}=5}%
\COMMA \BRACKET{\SYMBOL{x}=\sqrt{\SYMBOL{a}+50}\COMMA \SYMBOL{y}=5}\COMMA %
\BRACKET{\SYMBOL{x}=1\COMMA \SYMBOL{y}=\frac{\sqrt{-{\SYMBOL{a}}+1}}{\sqrt{2}%
}}\COMMA \BRACKET{\SYMBOL{x}=1\COMMA \SYMBOL{y}=-{\frac{\sqrt{-{\SYMBOL{a}}+1%
}}{\sqrt{2}}}}}%
\end{fricasmath}
\end{TeXOutput}
\formatResultType{List(List(Equation(Expression(Integer))))}
\end{xtc}
For systems of equations with symbolic coefficients, you can
apply \spadfun{solve}, listing the variables that you want
\Language{} to solve for.
For polynomial equations, a solution cannot usually be expressed
solely in terms of the other variables.
Instead, the solution is presented as a ``triangular'' system of
equations, where each polynomial has coefficients involving
only the succeeding variables. This is analogous to converting  a linear system
of equations to ``triangular form''.
\begin{xtc}
\begin{xtccomment}
A system of three equations in five variables.
\end{xtccomment}
\begin{spadsrc}
eqns := [x^2 - y + z,x^2*z + x^4 - b*y, y^2 *z - a - b*x]
\end{spadsrc}
\begin{TeXOutput}
\begin{fricasmath}{5}
\BRACKET{\SYMBOL{z}-{\SYMBOL{y}}+\SUPER{\SYMBOL{x}}{2}\COMMA \SUPER{\SYMBOL{x%
}}{2}\TIMES \SYMBOL{z}-{\SYMBOL{b}\TIMES \SYMBOL{y}}+\SUPER{\SYMBOL{x}}{4}%
\COMMA \SUPER{\SYMBOL{y}}{2}\TIMES \SYMBOL{z}-{\SYMBOL{b}\TIMES \SYMBOL{x}}-{%
\SYMBOL{a}}}%
\end{fricasmath}
\end{TeXOutput}
\formatResultType{List(Polynomial(Integer))}
\end{xtc}
\begin{xtc}
\begin{xtccomment}
Solve the system for unknowns \smath{[x,y,z]},
reducing the solution to triangular form.
\end{xtccomment}
\begin{spadsrc}
solve(eqns,[x,y,z])
\end{spadsrc}
\begin{TeXOutput}
\begin{fricasmath}{6}
\BRACKET{\BRACKET{\SYMBOL{x}=-{\frac{\SYMBOL{a}}{\SYMBOL{b}}}\COMMA \SYMBOL{y%
}=0\COMMA \SYMBOL{z}=-{\frac{\SUPER{\SYMBOL{a}}{2}}{\SUPER{\SYMBOL{b}}{2}}}}%
\COMMA \BRACKET{\SYMBOL{x}=\frac{\SUPER{\SYMBOL{z}}{3}+2\TIMES \SYMBOL{b}%
\TIMES \SUPER{\SYMBOL{z}}{2}+\SUPER{\SYMBOL{b}}{2}\TIMES \SYMBOL{z}-{\SYMBOL{%
a}}}{\SYMBOL{b}}\COMMA \SYMBOL{y}=\SYMBOL{z}+\SYMBOL{b}\COMMA \SUPER{\SYMBOL{%
z}}{6}+4\TIMES \SYMBOL{b}\TIMES \SUPER{\SYMBOL{z}}{5}+6\TIMES \SUPER{\SYMBOL{%
b}}{2}\TIMES \SUPER{\SYMBOL{z}}{4}+\PAREN{4\TIMES \SUPER{\SYMBOL{b}}{3}-{2%
\TIMES \SYMBOL{a}}}\TIMES \SUPER{\SYMBOL{z}}{3}+\PAREN{\SUPER{\SYMBOL{b}}{4}-%
{4\TIMES \SYMBOL{a}\TIMES \SYMBOL{b}}}\TIMES \SUPER{\SYMBOL{z}}{2}-{2\TIMES %
\SYMBOL{a}\TIMES \SUPER{\SYMBOL{b}}{2}\TIMES \SYMBOL{z}}-{\SUPER{\SYMBOL{b}}{%
3}}+\SUPER{\SYMBOL{a}}{2}=0}}%
\end{fricasmath}
\end{TeXOutput}
\formatResultType{List(List(Equation(Fraction(Polynomial(Integer)))))}
\end{xtc}
% *********************************************************************
\head{section}{System Commands}{ugIntroSysCmmands}
% *********************************************************************
%

We conclude our tour of \Language{} with a brief discussion
of \spadgloss{system commands}.
System commands are special statements that start with a
closing parenthesis (\spadSyntax{)}). They are used to control or
display your \Language{} environment, start the \HyperName{}
system, issue operating system commands and leave \Language{}.
For example, \spadsys{)system} is used
to issue commands to the operating system from \Language{}.
\syscmdindex{system}
Here is a brief description of some of these commands.
For more information on specific commands, see
\appxref{ugSysCmd}.

Perhaps the most important user command is the \spadsys{)clear all}
command that initializes your environment.
Every section and subsection in this book has an invisible
\spadsys{)clear all} that is read prior to the examples given in
the section.
\spadsys{)clear all} gives you a fresh, empty environment with no
user variables defined and the step number reset to \spad{1}.
The \spadsys{)clear} command can also be used to selectively clear
values and properties of system variables.

Another useful system command is \spadsys{)read}.
A preferred way to develop an application in \Language{} is to put
your interactive commands into a file, say {\bf my.input} file.
To get \Language{} to read this file, you use the system command
\spadsys{)read my.input}.
If you need to make changes to your approach or definitions, go
into your favorite editor, change {\bf my.input}, then
\spadsys{)read my.input} again.

Other system commands include: \spadsys{)history}, to display
previous input and/or output lines; \spadsys{)display}, to display
properties and values of workspace variables; and \spadsys{)what}.

\begin{xtc}
\begin{xtccomment}
Issue \spadsys{)what} to get a list of \Language{} objects that
contain a given substring in their name.
\end{xtccomment}
\begin{spadsrc}
)what operations integrate
\end{spadsrc}
\begin{SysCmdOutput}

Operations whose names satisfy the above pattern(s):

HermiteIntegrate      algintegrate          complexIntegrate      
expintegrate          fintegrate            infieldIntegrate      
integrate             integrateIfCan        integrate_sols        
internalIntegrate     internalIntegrate0    lambintegrate         
lazyGintegrate        lazyIntegrate         lfintegrate           
monomialIntegrate     palgintegrate         pmComplexintegrate    
pmintegrate           primintegrate         
   
      To get more information about an operation such as integrate , 
      issue the command )display op integrate 
\end{SysCmdOutput}
\end{xtc}

% *********************************************************************
%\head{subsection}{Undo}{ugIntroUndo}
% *********************************************************************

A useful system command is \spadsys{)undo}.
Sometimes while computing interactively with \Language{}, you make
a mistake and enter an incorrect definition or assignment.
Or perhaps you
need to try one of several alternative approaches, one after
another, to find the best way to approach an application.
For this, you will find the \spadgloss{undo} facility of
\Language{} helpful.

System command \spadsys{)undo n} means ``undo back to step \spad{n}''; it
restores the values of user variables to those that existed
immediately after input expression \spad{n} was evaluated.
Similarly, \spadsys{)undo -n} undoes changes caused by the last
\spad{n} input expressions.
Once you have done an \spadsys{)undo},
you can continue on from there, or make a change and
{\bf redo} all your input expressions from the point
of the \spadsys{)undo} forward.
The \spadsys{)undo} is completely general: it changes the environment
like any user expression.
Thus you can \spadsys{)undo} any previous undo.

Here is a sample dialogue between user and \Language{}.
\begin{xtc}
\begin{xtccomment}
``Let me define
two mutually dependent functions \spad{f} and \spad{g} piece-wise.''
\end{xtccomment}
\begin{spadsrc}
f(0) == 1; g(0) == 1
\end{spadsrc}
\end{xtc}
\begin{xtc}
\begin{xtccomment}
``Here is the general term for \spad{f}.''
\end{xtccomment}
\begin{spadsrc}
f(n) == e/2*f(n-1) - x*g(n-1)
\end{spadsrc}
\end{xtc}
\begin{xtc}
\begin{xtccomment}
``And here is the general term for \spad{g}.''
\end{xtccomment}
\begin{spadsrc}
g(n) == -x*f(n-1) + d/3*g(n-1)
\end{spadsrc}
\end{xtc}
\begin{xtc}
\begin{xtccomment}
``What is value of \spad{f(3)}?''
\end{xtccomment}
\begin{spadsrc}
f(3)
\end{spadsrc}
\begin{MessageOutput}
   Compiling function g with type Integer -> Polynomial(Fraction(
      Integer)) 
\end{MessageOutput}
\begin{MessageOutput}
   Compiling function g as a recurrence relation.
\end{MessageOutput}
\begin{MessageOutput}
   Compiling function g with type Integer -> Polynomial(Fraction(
      Integer)) 
\end{MessageOutput}
\begin{MessageOutput}
   Compiling function g as a recurrence relation.
\end{MessageOutput}
\begin{MessageOutput}
   Compiling function f with type Integer -> Polynomial(Fraction(
      Integer)) 
\end{MessageOutput}
\begin{MessageOutput}
   Compiling function f as a recurrence relation.
\end{MessageOutput}
\begin{TeXOutput}
\begin{fricasmath}{4}
-{\SUPER{\SYMBOL{x}}{3}}+\PAREN{\SYMBOL{e}+\frac{1}{3}\TIMES \SYMBOL{d}}%
\TIMES \SUPER{\SYMBOL{x}}{2}+\PAREN{-{\frac{1}{4}\TIMES \SUPER{\SYMBOL{e}}{2}%
}-{\frac{1}{6}\TIMES \SYMBOL{d}\TIMES \SYMBOL{e}}-{\frac{1}{9}\TIMES \SUPER{%
\SYMBOL{d}}{2}}}\TIMES \SYMBOL{x}+\frac{1}{8}\TIMES \SUPER{\SYMBOL{e}}{3}%
\end{fricasmath}
\end{TeXOutput}
\formatResultType{Polynomial(Fraction(Integer))}
\end{xtc}
\begin{noOutputXtc}
\begin{xtccomment}
``Hmm, I think I want to define \spad{f} differently.
Undo to the environment right after I defined \spad{f}.''
\end{xtccomment}
\begin{spadsrc}
)undo 2
\end{spadsrc}
\end{noOutputXtc}
\begin{xtc}
\begin{xtccomment}
``Here is how I think I want \spad{f} to be defined instead.''
\end{xtccomment}
\begin{spadsrc}
f(n) == d/3*f(n-1) - x*g(n-1)
\end{spadsrc}
\begin{MessageOutput}
   1 old definition(s) deleted for function or rule f 
\end{MessageOutput}
\end{xtc}
\begin{noOutputXtc}
\begin{xtccomment}
Redo the computation from expression \spad{3} forward.
\end{xtccomment}
\begin{spadsrc}
)undo )redo
\end{spadsrc}
\end{noOutputXtc}
\begin{noOutputXtc}
\begin{xtccomment}
``I want my old definition of
\spad{f} after all. Undo the undo and restore
the environment to that immediately after \spad{(4)}.''
\end{xtccomment}
\begin{spadsrc}
)undo 4
\end{spadsrc}
\end{noOutputXtc}
\begin{xtc}
\begin{xtccomment}
``Check that the value of \spad{f(3)} is restored.''
\end{xtccomment}
\begin{spadsrc}
f(3)
\end{spadsrc}
\begin{MessageOutput}
   Compiling function g with type Integer -> Polynomial(Fraction(
      Integer)) 
\end{MessageOutput}
\begin{MessageOutput}
   Compiling function g as a recurrence relation.
\end{MessageOutput}
\begin{MessageOutput}
   Compiling function g with type Integer -> Polynomial(Fraction(
      Integer)) 
\end{MessageOutput}
\begin{MessageOutput}
   Compiling function g as a recurrence relation.
\end{MessageOutput}
\begin{MessageOutput}
   Compiling function f with type Integer -> Polynomial(Fraction(
      Integer)) 
\end{MessageOutput}
\begin{MessageOutput}
   Compiling function f as a recurrence relation.
\end{MessageOutput}
\begin{TeXOutput}
\begin{fricasmath}{6}
-{\SUPER{\SYMBOL{x}}{3}}+\PAREN{\SYMBOL{e}+\frac{1}{3}\TIMES \SYMBOL{d}}%
\TIMES \SUPER{\SYMBOL{x}}{2}+\PAREN{-{\frac{1}{4}\TIMES \SUPER{\SYMBOL{e}}{2}%
}-{\frac{1}{6}\TIMES \SYMBOL{d}\TIMES \SYMBOL{e}}-{\frac{1}{9}\TIMES \SUPER{%
\SYMBOL{d}}{2}}}\TIMES \SYMBOL{x}+\frac{1}{8}\TIMES \SUPER{\SYMBOL{e}}{3}%
\end{fricasmath}
\end{TeXOutput}
\formatResultType{Polynomial(Fraction(Integer))}
\end{xtc}

After you have gone off on several tangents, then backtracked to
previous points in your conversation using \spadsys{)undo}, you
might want to save all the ``correct'' input commands you issued,
disregarding those undone.
The system command \spadsys{)history )write mynew.input} writes a
clean straight-line program onto the file {\bf mynew.input} on
your disk.

This concludes your tour of \Language{}.
To disembark, issue the system command \spadsys{)quit} to leave \Language{}
and return to the operating system.

% !! DO NOT MODIFY THIS FILE BY HAND !! Created by spool2tex.awk.

% Copyright (c) 1991-2002, The Numerical ALgorithms Group Ltd.
% All rights reserved.
%
% Redistribution and use in source and binary forms, with or without
% modification, are permitted provided that the following conditions are
% met:
%
%     - Redistributions of source code must retain the above copyright
%       notice, this list of conditions and the following disclaimer.
%
%     - Redistributions in binary form must reproduce the above copyright
%       notice, this list of conditions and the following disclaimer in
%       the documentation and/or other materials provided with the
%       distribution.
%
%     - Neither the name of The Numerical ALgorithms Group Ltd. nor the
%       names of its contributors may be used to endorse or promote products
%       derived from this software without specific prior written permission.
%
% THIS SOFTWARE IS PROVIDED BY THE COPYRIGHT HOLDERS AND CONTRIBUTORS "AS
% IS" AND ANY EXPRESS OR IMPLIED WARRANTIES, INCLUDING, BUT NOT LIMITED
% TO, THE IMPLIED WARRANTIES OF MERCHANTABILITY AND FITNESS FOR A
% PARTICULAR PURPOSE ARE DISCLAIMED. IN NO EVENT SHALL THE COPYRIGHT OWNER
% OR CONTRIBUTORS BE LIABLE FOR ANY DIRECT, INDIRECT, INCIDENTAL, SPECIAL,
% EXEMPLARY, OR CONSEQUENTIAL DAMAGES (INCLUDING, BUT NOT LIMITED TO,
% PROCUREMENT OF SUBSTITUTE GOODS OR SERVICES-- LOSS OF USE, DATA, OR
% PROFITS-- OR BUSINESS INTERRUPTION) HOWEVER CAUSED AND ON ANY THEORY OF
% LIABILITY, WHETHER IN CONTRACT, STRICT LIABILITY, OR TORT (INCLUDING
% NEGLIGENCE OR OTHERWISE) ARISING IN ANY WAY OUT OF THE USE OF THIS
% SOFTWARE, EVEN IF ADVISED OF THE POSSIBILITY OF SUCH DAMAGE.

% *********************************************************************
\head{chapter}{Using Types and Modes}{ugTypes}
% *********************************************************************

In this chapter we look at the key notion of \spadgloss{type} and its
generalization \spadgloss{mode}.
We show that every \Language{} object has a type that
determines what you can do with the object.
In particular, we explain how to use types to call specific functions
from particular parts of the library and how types and modes can be used
to create new objects from old.
We also look at \pspadtype{Record} and \pspadtype{Union} types and
the special type \spadtype{Any}.
Finally, we give you an idea of how \Language{} manipulates types and
modes internally to resolve ambiguities.

% *********************************************************************
\head{section}{The Basic Idea}{ugTypesBasic}
% *********************************************************************

The \Language{} world deals with many kinds of objects.
There are mathematical objects such as numbers and polynomials,
data structure objects such as lists and arrays, and graphics
objects such as points and graphic images.
Functions are objects too.

\Language{} organizes objects using the notion of \spadglossSee{domain of
computation}{domain}, or simply \spadgloss{domain}.
Each domain denotes a class of objects.
The class of objects it denotes is usually given by the name of the
domain: \spadtype{Integer} for the integers, \spadtype{Float} for
floating-point numbers, and so on.
The convention is that the first letter of a domain name is capitalized.
Similarly, the domain \spadtype{Polynomial(Integer)} denotes ``polynomials
with integer coefficients.''
Also, \spadtype{Matrix(Float)} denotes ``matrices with floating-point
entries.''

Every basic \Language{} object belongs to a unique domain.
The integer \spad{3} belongs to the domain \spadtype{Integer} and
the polynomial \spad{x + 3} belongs to the domain
\spadtype{Polynomial(Integer)}.
The domain of an object is also called its \spadgloss{type}.
Thus we speak of ``the type \spadtype{Integer}''
and ``the type \spadtype{Polynomial(Integer)}.''

\begin{xtc}
\begin{xtccomment}
After an \Language{} computation, the type is displayed toward the
right-hand side of the page (or screen).
\end{xtccomment}
\begin{spadsrc}
-3
\end{spadsrc}
\begin{TeXOutput}
\begin{fricasmath}{1}
-{3}%
\end{fricasmath}
\end{TeXOutput}
\formatResultType{Integer}
\end{xtc}
\begin{xtc}
\begin{xtccomment}
Here we create a rational number but it looks like the last result.
The type however tells you it is different.
You cannot identify the type of an object by how \Language{}
displays the object.
\end{xtccomment}
\begin{spadsrc}
-3/1
\end{spadsrc}
\begin{TeXOutput}
\begin{fricasmath}{2}
-{3}%
\end{fricasmath}
\end{TeXOutput}
\formatResultType{Fraction(Integer)}
\end{xtc}
\begin{xtc}
\begin{xtccomment}
When a computation produces a result of a simpler type, \Language{} leaves
the type unsimplified.
Thus no information is lost.
\end{xtccomment}
\begin{spadsrc}
x + 3 - x 
\end{spadsrc}
\begin{TeXOutput}
\begin{fricasmath}{3}
3%
\end{fricasmath}
\end{TeXOutput}
\formatResultType{Polynomial(Integer)}
\end{xtc}
\begin{xtc}
\begin{xtccomment}
This seldom matters since \Language{} retracts the answer to the
simpler type if it is necessary.
\end{xtccomment}
\begin{spadsrc}
factorial(%) 
\end{spadsrc}
\begin{TeXOutput}
\begin{fricasmath}{4}
6%
\end{fricasmath}
\end{TeXOutput}
\formatResultType{Expression(Integer)}
\end{xtc}
\begin{xtc}
\begin{xtccomment}
When you issue a positive number, the type \spadtype{PositiveInteger} is
printed.
Surely, \spad{3} also has type \spadtype{Integer}!
The curious reader may now have two questions.
First, is the type of an object not unique?
Second, how is \spadtype{PositiveInteger} related to \spadtype{Integer}?
Read on!
\end{xtccomment}
\begin{spadsrc}
3
\end{spadsrc}
\begin{TeXOutput}
\begin{fricasmath}{5}
3%
\end{fricasmath}
\end{TeXOutput}
\formatResultType{PositiveInteger}
\end{xtc}

Any domain can be refined to a \spadgloss{subdomain} by a membership
\spadgloss{predicate}.\footnote{A predicate is a function that,
when applied to an object of the domain, returns either
\spad{true} or \spad{false}.}
For example, the domain \spadtype{Integer} can be refined to the
subdomain \spadtype{PositiveInteger}, the set of integers
\spad{x} such that \spad{x > 0}, by giving the \Language{}
predicate \spad{x +-> x > 0}.
Similarly, \Language{} can define subdomains such as ``the
subdomain of diagonal matrices,'' ``the subdomain of lists of length
two,'' ``the subdomain of monic irreducible polynomials in
\spad{x},'' and so on.
Trivially, any domain is a subdomain of itself.

While an object belongs to a unique domain, it can belong to any
number of subdomains.
Any subdomain of the domain of an object can be used as the
{\it type} of that object.
The type of \spad{3} is indeed both \spadtype{Integer} and
\spadtype{PositiveInteger} as well as any other subdomain of
integer whose predicate is satisfied, such as ``the prime
integers,'' ``the odd positive integers between 3 and 17,'' and so
on.

% *********************************************************************
\head{subsection}{Domain Constructors}{ugTypesBasicDomainCons}
% *********************************************************************

In \Language{}, domains are objects.
You can create them, pass them to functions, and, as we'll see later, test
them for certain properties.

In \Language{}, you ask for a value of a function by applying its name
to a set of arguments.

\begin{xtc}
\begin{xtccomment}
To ask for ``the factorial of 7'' you enter this expression to
\Language{}.
This applies the function \spad{factorial} to the value \spad{7}
to compute the result.
\end{xtccomment}
\begin{spadsrc}
factorial(7)
\end{spadsrc}
\begin{TeXOutput}
\begin{fricasmath}{1}
5040%
\end{fricasmath}
\end{TeXOutput}
\formatResultType{PositiveInteger}
\end{xtc}
\begin{xtc}
\begin{xtccomment}
Enter the type \spadtype{Polynomial (Integer)} as an expression to
\Language{}.
This looks much like a function call as well.
It is!
The result is appropriately stated to be of type
\spadtype{Domain}, which
according to our usual convention, denotes the class of all domains.
\end{xtccomment}
\begin{spadsrc}
Polynomial(Integer)
\end{spadsrc}
\begin{TeXOutput}
\begin{fricasmath}{2}
\STRING{Polynomial(Integer)}%
\end{fricasmath}
\end{TeXOutput}
\formatResultType{Type}
\end{xtc}

The most basic operation involving domains is that of building a new
domain from a given one.
To create the domain of ``polynomials over the integers,'' \Language{}
applies the function \spadtype{Polynomial} to the domain
\spadtype{Integer}.
A function like \spadtype{Polynomial} is called a \spadgloss{domain
constructor} or,
\index{constructor!domain}
more simply, a
\spadgloss{constructor}.
A domain constructor is a function that creates a domain.
An argument to a domain constructor can be another domain or, in general,
an arbitrary kind of object.
\spadtype{Polynomial} takes a single domain argument while
\spadtype{SquareMatrix} takes a positive integer as an argument
to give its dimension and
a domain argument to give the type of its components.

What kinds of domains can you use as the argument to
\spadtype{Polynomial} or \spadtype{SquareMatrix} or
\spadtype{List}?
Well, the first two are mathematical in nature.
You want to be able to perform algebraic operations like
\spadop{+} and \spadop{*} on polynomials and square matrices,
and operations such as \spadfun{determinant} on square matrices.
So you want to allow polynomials of integers {\it and} polynomials
of square matrices with complex number coefficients and, in
general, anything that ``makes sense.'' At the same time, you
don't want \Language{} to be able to build nonsense domains such
as ``polynomials of strings!''

In contrast to algebraic structures, data structures can hold any
kind of object.
Operations on lists such as \spadfunFrom{insert}{List},
\spadfunFrom{delete}{List}, and \spadfunFrom{concat}{List} just
manipulate the list itself without changing or operating on its
elements.
Thus you can build \spadtype{List} over almost any datatype,
including itself.
\begin{xtc}
\begin{xtccomment}
Create a complicated algebraic domain.
\end{xtccomment}
\begin{spadsrc}
List (List (Matrix (Polynomial (Complex (Fraction (Integer))))))
\end{spadsrc}
\begin{TeXOutput}
\begin{fricasmath}{3}
\STRING{List(List(Matrix(Polynomial(Complex(Fraction(Integer))))))}%
\end{fricasmath}
\end{TeXOutput}
\formatResultType{Type}
\end{xtc}
\begin{xtc}
\begin{xtccomment}
Try to create a meaningless domain.
\end{xtccomment}
\begin{spadsrc}
Polynomial(String)
\end{spadsrc}
\begin{MessageOutput}
   Polynomial(String) is not a valid type.
\end{MessageOutput}
\end{xtc}
Evidently from our last example, \Language{} has some mechanism
that tells what a constructor can use as an argument.
This brings us to the notion of \spadgloss{category}.
As domains are objects, they too have a domain.
The domain of a domain is a category.
A category is simply a type whose members are domains.

A common algebraic category is \spadtype{Ring}, the class of all domains
that are ``rings.''
A ring is an algebraic structure with constants \spad{0} and \spad{1} and
operations \spadopFrom{+}{Ring}, \spadopFrom{-}{Ring}, and
\spadopFrom{*}{Ring}.
These operations are assumed ``closed'' with respect to the domain,
meaning that they take two objects of the domain and produce a result
object also in the domain.
The operations are understood to satisfy certain ``axioms,'' certain
mathematical principles providing the algebraic foundation for rings.
For example, the {\it additive inverse axiom} for rings states:
\begin{center}
Every element \spad{x} has an additive inverse \spad{y} such
that \spad{x + y = 0}.
\end{center}
The prototypical example of a domain that is a ring is the integers.
Keep them in mind whenever we mention \spadtype{Ring}.

Many algebraic domain constructors such as \spadtype{Complex},
\spadtype{Polynomial}, \spadtype{Fraction}, take rings as
arguments and return rings as values.
You can use the infix operator ``\spad{has}''
\spadkey{has}
to ask a domain if it belongs to a particular category.

\begin{xtc}
\begin{xtccomment}
All numerical types are rings.
Domain constructor \spadtype{Polynomial} builds ``the ring of polynomials
over any other ring.''
\end{xtccomment}
\begin{spadsrc}
Polynomial(Integer) has Ring
\end{spadsrc}
\begin{TeXOutput}
\begin{fricasmath}{4}
\STRING{true}%
\end{fricasmath}
\end{TeXOutput}
\formatResultType{Boolean}
\end{xtc}
\begin{xtc}
\begin{xtccomment}
Constructor \spadtype{List} never produces a ring.
\end{xtccomment}
\begin{spadsrc}
List(Integer) has Ring
\end{spadsrc}
\begin{TeXOutput}
\begin{fricasmath}{5}
\STRING{false}%
\end{fricasmath}
\end{TeXOutput}
\formatResultType{Boolean}
\end{xtc}
\begin{xtc}
\begin{xtccomment}
The constructor \spadtype{Matrix(R)} builds ``the domain of all matrices
over the ring \spad{R}.'' This domain is never a ring since the operations
\spadSyntax{+}, \spadSyntax{-}, and \spadSyntax{*} on matrices of arbitrary
shapes are undefined.
\end{xtccomment}
\begin{spadsrc}
Matrix(Integer) has Ring
\end{spadsrc}
\begin{TeXOutput}
\begin{fricasmath}{6}
\STRING{false}%
\end{fricasmath}
\end{TeXOutput}
\formatResultType{Boolean}
\end{xtc}
\begin{xtc}
\begin{xtccomment}
Thus you can never build polynomials over matrices.
\end{xtccomment}
\begin{spadsrc}
Polynomial(Matrix(Integer))
\end{spadsrc}
\begin{MessageOutput}
   Polynomial(Matrix(Integer)) is not a valid type.
\end{MessageOutput}
\end{xtc}
\begin{xtc}
\begin{xtccomment}
Use \spadtype{SquareMatrix(n,R)} instead.
For any positive integer \spad{n}, it builds ``the ring of \spad{n} by
\spad{n} matrices over \spad{R}.''
\end{xtccomment}
\begin{spadsrc}
Polynomial(SquareMatrix(7,Complex(Integer)))
\end{spadsrc}
\begin{TeXOutput}
\begin{fricasmath}{7}
\STRING{Polynomial(SquareMatrix(7,Complex(Integer)))}%
\end{fricasmath}
\end{TeXOutput}
\formatResultType{Type}
\end{xtc}

Another common category is \spadtype{Field}, the class of all fields.
\index{field}
A field is a ring with additional operations.
For example, a field has commutative multiplication and
a closed operation \spadopFrom{/}{Field} for the
division of two elements.
\spadtype{Integer} is not a field since, for example, \spad{3/2} does not
have an integer result.
The prototypical example of a field is the rational numbers, that is, the
domain \spadtype{Fraction(Integer)}.
In general, the constructor \spadtype{Fraction} takes a ring as an
argument and returns a field.\footnote{Actually,
the argument domain must have some additional
properties so as to belong to category \spadtype{IntegralDomain}.}
Other domain constructors, such as \spadtype{Complex}, build fields only if
their argument domain is a field.

\begin{xtc}
\begin{xtccomment}
The complex integers (often called the ``Gaussian integers'') do not form
a field.
\end{xtccomment}
\begin{spadsrc}
Complex(Integer) has Field
\end{spadsrc}
\begin{TeXOutput}
\begin{fricasmath}{8}
\STRING{false}%
\end{fricasmath}
\end{TeXOutput}
\formatResultType{Boolean}
\end{xtc}
\begin{xtc}
\begin{xtccomment}
But fractions of complex integers do.
\end{xtccomment}
\begin{spadsrc}
Fraction(Complex(Integer)) has Field
\end{spadsrc}
\begin{TeXOutput}
\begin{fricasmath}{9}
\STRING{true}%
\end{fricasmath}
\end{TeXOutput}
\formatResultType{Boolean}
\end{xtc}
\begin{xtc}
\begin{xtccomment}
The algebraically equivalent domain of complex rational numbers is a field
since domain constructor \spadtype{Complex} produces a field whenever its
argument is a field.
\end{xtccomment}
\begin{spadsrc}
Complex(Fraction(Integer)) has Field
\end{spadsrc}
\begin{TeXOutput}
\begin{fricasmath}{10}
\STRING{true}%
\end{fricasmath}
\end{TeXOutput}
\formatResultType{Boolean}
\end{xtc}

The most basic category is \spadtype{Type}.
\exptypeindex{Type}
It denotes the class of all domains and
subdomains.\footnote{\spadtype{Type} does not denote the class of
all types.
The type of all categories is \spadtype{Category}.
The type of \spadtype{Type} itself is undefined.}
Domain constructor \spadtype{List} is able to build ``lists of
elements from domain \spad{D}'' for arbitrary \spad{D} simply by
requiring that \spad{D} belong to category \spadtype{Type}.

Now, you may ask, what exactly is a category?
\index{category}
Like domains, categories can be defined in the \Language{} language.
A category is defined by three components:
%
\begin{enumerate}
\item a name (for example, \spadtype{Ring}),
used to refer to the class of domains that the category represents;
\item a set of operations, used to refer to the operations that
the domains of this class support
(for example, \spadop{+}, \spadop{-}, and \spadop{*} for rings); and
\item an optional list of other categories that this category extends.
\end{enumerate}
%
This last component is a new idea.
And it is key to the design of \Language{}!
Because categories can extend one another, they form hierarchies.
Detailed charts showing the category hierarchies in \Language{} are
displayed in the endpages of this book.
There you see that all categories are extensions of \spadtype{Type} and that
\spadtype{Field} is an extension of \spadtype{Ring}.

The operations supported by the domains of a category are called the
\spadglossSee{exports}{export} of that category because these are the
operations made available for system-wide use.
The exports of a domain of a given category are not only the ones
explicitly mentioned by the category.
Since a category extends other categories, the operations of these other
categories---and all categories these other categories extend---are also
exported by the domains.

For example, polynomial domains belong to \spadtype{PolynomialCategory}.
This category explicitly mentions some twenty-nine
operations on polynomials, but it
extends eleven other categories (including \spadtype{Ring}).
As a result, the current system has over one hundred operations on polynomials.

If a domain belongs to a category that extends, say, \spadtype{Ring}, it
is convenient to say that the domain exports \spadtype{Ring}.
The name of the category thus provides a convenient shorthand for the list
of operations exported by the category.
Rather than listing operations such as \spadopFrom{+}{Ring} and
\spadopFrom{*}{Ring} of \spadtype{Ring} each time they are needed, the
definition of a type simply asserts that it exports category
\spadtype{Ring}.

The category name, however, is more than a shorthand.
The name \spadtype{Ring}, in fact, implies that the operations exported by
rings are required to satisfy a set of ``axioms'' associated with the name
\spadtype{Ring}.\footnote{This subtle
but important feature distinguishes \Language{} from
other abstract datatype designs.}

Why is it not correct to assume that some type is a ring if it exports all
of the operations of \spadtype{Ring}?
Here is why.
Some languages such as {\bf APL}
\index{APL}
denote the \spadtype{Boolean} constants \spad{true} and
\spad{false} by the integers \spad{1} and \spad{0}
respectively, then use \spadop{+} and \spadop{*} to denote the
logical operators \spadfun{or} and \spadfun{and}.
But with these definitions
\spadtype{Boolean} is not a ring since the additive inverse
axiom is violated.\footnote{There is no inverse element \spad{a}
such that \spad{1 + a = 0}, or, in the usual terms:
\spad{true or a = false}.}
This alternative definition of \spadtype{Boolean} can be easily
and correctly implemented in \Language{}, since
\spadtype{Boolean} simply does not assert that it is of category
\spadtype{Ring}.
This prevents the system from building meaningless domains such as
\spadtype{Polynomial(Boolean)} and then wrongfully applying
algorithms that presume that the ring axioms hold.


Enough on categories. To learn more about them, see
\chapref{ugCategories}.
We now return to our discussion of domains.

Domains \spadgloss{export} a set of operations to make them
available for system-wide use.
\spadtype{Integer}, for example, exports the operations
\spadopFrom{+}{Integer} and \spadopFrom{=}{Integer} given by
the \spadglossSee{signatures}{signature}
\spadopFrom{+}{Integer}: \spad{(Integer,Integer)->Integer} and
\spadopFrom{=}{Integer}: \spad{(Integer,Integer)->Boolean},
respectively.
Each of these operations takes two \spadtype{Integer} arguments.
The \spadopFrom{+}{Integer} operation also returns an \spadtype{Integer} but
\spadopFrom{=}{Integer} returns a \spadtype{Boolean}: \spad{true} or
\spad{false}.
The operations exported by a domain usually manipulate objects of
the domain---but not always.

The operations of a domain may actually take as arguments, and return as
values, objects from any domain.
For example, \spadtype{Fraction (Integer)} exports the operations
\spadopFrom{/}{Fraction}: \spad{(Integer,Integer)->Fraction(Integer)}
and \spadfunFrom{characteristic}{Fraction}:
\spad{->NonNegativeInteger}.

Suppose all operations of a domain take as arguments and return
as values, only objects from {\it other} domains.
\index{package}
This kind of domain
\index{constructor!package}
is what \Language{} calls a \spadgloss{package}.

A package does not designate a class of objects at all.
Rather, a package is just a collection of operations.
Actually the bulk of the \Language{} library of algorithms consists
of packages.
The facilities for factorization; integration; solution of linear,
polynomial, and differential equations; computation of limits; and so
on, are all defined in packages.
Domains needed by algorithms can be passed to a package as arguments or
used by name if they are not ``variable.''
Packages are useful for defining operations that convert objects of one
type to another, particularly when these types have different
parameterizations.
As an example, the package \spadtype{PolynomialFunction2(R,S)} defines
operations that convert polynomials over a domain \spad{R} to polynomials
over \spad{S}.
To convert an object from \spadtype{Polynomial(Integer)} to
\spadtype{Polynomial(Float)}, \Language{} builds the package
\spadtype{PolynomialFunctions2(Integer,Float)} in order to create the
required conversion function.
(This happens ``behind the scenes'' for you: see \spadref{ugTypesConvert}
for details on how to convert objects.)

\Language{} categories, domains and packages and all their contained
functions are written in the \Language{} programming language and have
been compiled into machine code.
This is what comprises the \Language{} \spadgloss{library}.
In the rest of this book we show you how to use these domains and
their functions and how to write your own functions.

% *********************************************************************
\head{section}{Writing Types and Modes}{ugTypesWriting}
% *********************************************************************
%

We have already seen in
the last section
several examples of types.
Most of these examples had either no arguments (for example,
\spadtype{Integer}) or one argument (for example,
\spadtype{Polynomial (Integer)}).
In this section we give details about writing arbitrary types.
We then define \spadglossSee{modes}{mode} and discuss how to write
them.
We conclude the section with a discussion on constructor
abbreviations.

\begin{xtc}
\begin{xtccomment}
When might you need to write a type or mode?
You need to do so when you declare variables.
\end{xtccomment}
\begin{spadsrc}
a : PositiveInteger
\end{spadsrc}
\end{xtc}
\begin{xtc}
\begin{xtccomment}
You need to do so when you declare functions
(\spadref{ugTypesDeclare}),
\end{xtccomment}
\begin{spadsrc}
f : Integer -> String
\end{spadsrc}
\end{xtc}
\begin{xtc}
\begin{xtccomment}
You need to do so when you convert an object from one type to another
(\spadref{ugTypesConvert}).
\end{xtccomment}
\begin{spadsrc}
factor(2 :: Complex(Integer))
\end{spadsrc}
\begin{TeXOutput}
\begin{fricasmath}{3}
-{\ImaginaryI \TIMES \SUPER{\PAREN{1+\ImaginaryI }}{2}}%
\end{fricasmath}
\end{TeXOutput}
\formatResultType{Factored(Complex(Integer))}
\end{xtc}
\begin{xtc}
\begin{xtccomment}
\end{xtccomment}
\begin{spadsrc}
(2 = 3)$Integer
\end{spadsrc}
\begin{TeXOutput}
\begin{fricasmath}{4}
\STRING{false}%
\end{fricasmath}
\end{TeXOutput}
\formatResultType{Boolean}
\end{xtc}
\begin{xtc}
\begin{xtccomment}
You need to do so when you give computation target type information
(\spadref{ugTypesPkgCall}).
\end{xtccomment}
\begin{spadsrc}
(2 = 3)@Boolean
\end{spadsrc}
\begin{TeXOutput}
\begin{fricasmath}{5}
\STRING{false}%
\end{fricasmath}
\end{TeXOutput}
\formatResultType{Boolean}
\end{xtc}

% *********************************************************************
\head{subsection}{Types with No Arguments}{ugTypesWritingZero}
% *********************************************************************

A constructor with no arguments can be written either
\index{type!using parentheses}
with or without
\index{parentheses!using with types}
trailing opening and closing parentheses (\spadSyntax{()}).
\begin{center}
\begin{tabular}{ccc}
\spadtype{Boolean()} is the same as \spadtype{Boolean} & \quad &
\spadtype{Integer()} is the same as \spadtype{Integer} \\
\spadtype{String()} is the same as \spadtype{String} & \quad &
\spadtype{Void()} is the same as \spadtype{Void} \\
\end{tabular}
\end{center}
and so on.
It is customary to omit the parentheses.

% *********************************************************************
\head{subsection}{Types with One Argument}{ugTypesWritingOne}
% *********************************************************************

A constructor with one argument can frequently be
\index{type!using parentheses}
written with no
\index{parentheses!using with types}
parentheses.
Types nest from right to left so that
\spadtype{Complex Fraction Polynomial Integer} is the same as
\spadtype{Complex (Fraction (Polynomial (Integer)))}.
You need to use parentheses to force the application of a constructor
to the correct argument, but you need not use any more than is
necessary to remove ambiguities.

Here are some guidelines for using parentheses (they are possibly slightly
more restrictive than they need to be).
\begin{xtc}
\begin{xtccomment}
If the argument is an expression like \spad{2 + 3}
then you must enclose the argument in parentheses.
\end{xtccomment}
\begin{spadsrc}
e : PrimeField(2 + 3)
\end{spadsrc}
\end{xtc}
%
\begin{xtc}
\begin{xtccomment}
If the type is to be used with package calling
then you must enclose the argument in parentheses.
\end{xtccomment}
\begin{spadsrc}
content(2)$Polynomial(Integer)
\end{spadsrc}
\begin{TeXOutput}
\begin{fricasmath}{2}
2%
\end{fricasmath}
\end{TeXOutput}
\formatResultType{Integer}
\end{xtc}
\begin{xtc}
\begin{xtccomment}
Alternatively, you can write the type without parentheses
then enclose the whole type expression with parentheses.
\end{xtccomment}
\begin{spadsrc}
content(2)$(Polynomial Complex Fraction Integer)
\end{spadsrc}
\begin{TeXOutput}
\begin{fricasmath}{3}
2%
\end{fricasmath}
\end{TeXOutput}
\formatResultType{Complex(Fraction(Integer))}
\end{xtc}
\begin{xtc}
\begin{xtccomment}
If you supply computation target type information
(\spadref{ugTypesPkgCall})
then you should enclose the argument in parentheses.
\end{xtccomment}
\begin{spadsrc}
(2/3)@Fraction(Polynomial(Integer))
\end{spadsrc}
\begin{TeXOutput}
\begin{fricasmath}{4}
\frac{2}{3}%
\end{fricasmath}
\end{TeXOutput}
\formatResultType{Fraction(Polynomial(Integer))}
\end{xtc}
%
\begin{xtc}
\begin{xtccomment}
If the type itself has parentheses around it and we are
not in the case of the first example above,
then the parentheses can usually be omitted.
\end{xtccomment}
\begin{spadsrc}
(2/3)@Fraction(Polynomial Integer)
\end{spadsrc}
\begin{TeXOutput}
\begin{fricasmath}{5}
\frac{2}{3}%
\end{fricasmath}
\end{TeXOutput}
\formatResultType{Fraction(Polynomial(Integer))}
\end{xtc}
%
\begin{xtc}
\begin{xtccomment}
If the type is used in a declaration and the argument is a single-word
type, integer or symbol,
then the parentheses can usually be omitted.
\end{xtccomment}
\begin{spadsrc}
(d,f,g) : Complex Polynomial Integer
\end{spadsrc}
\end{xtc}

% *********************************************************************
\head{subsection}{Types with More Than One Argument}{ugTypesWritingMore}
% *********************************************************************

If a constructor
\index{type!using parentheses}
has more than
\index{parentheses!using with types}
one argument, you must use parentheses.
Some examples are
\begin{center}
\spadtype{UnivariatePolynomial(x, Float)} \\
\spadtype{MultivariatePolynomial([z,w,r], Complex Float)} \\
\spadtype{SquareMatrix(3, Integer)} \\
\spadtype{FactoredFunctions2(Integer,Fraction Integer)}
\end{center}

% *********************************************************************
\head{subsection}{Modes}{ugTypesWritingModes}
% *********************************************************************

A \spadgloss{mode} is a type that possibly is a
question mark (\spadSyntax{?}) or contains one in an argument
position.
For example, the following are all modes.
\begin{center}
\begin{tabular}{ccc}
\spadtype{?} & \quad &
\spadtype{Polynomial ?} \\
\spadtype{Matrix Polynomial ?} & \quad &
\spadtype{SquareMatrix(3,?)} \\
\spadtype{Integer} & \quad &
\spadtype{OneDimensionalArray(Float)}
\end{tabular}
\end{center}

As is evident from these examples, a mode is a type with a
part that is not specified (indicated by a question mark).
Only one \spadSyntax{?} is allowed per mode and it must appear in the
most deeply nested argument that is a type. Thus
\nonLibAxiomType{?(Integer)},
\nonLibAxiomType{Matrix(? (Polynomial))},
\nonLibAxiomType{SquareMatrix(?, Integer)} and
\nonLibAxiomType{SquareMatrix(?, ?)} are all invalid.
The question mark must take the place of a domain, not data (for example,
the integer that is the dimension of a square matrix).
This rules out, for example, the two \spadtype{SquareMatrix}
expressions.

Modes can be used for declarations
(\spadref{ugTypesDeclare})
and conversions
(\spadref{ugTypesConvert}).
However, you cannot use a mode for package calling or giving target
type information.

% *********************************************************************
\head{subsection}{Abbreviations}{ugTypesWritingAbbr}
% *********************************************************************

Every constructor has an abbreviation that
\index{abbreviation!constructor}
you can freely
\index{constructor!abbreviation}
substitute for the constructor name.
In some cases, the abbreviation is nothing more than the
capitalized version of the constructor name.

\beginImportant
Aside from allowing types to be written more concisely,
abbreviations are used by \Language{} to name various system
files for constructors (such as library filenames, test input
files and example files).
Here are some common abbreviations.
\begin{center}
\begin{tabular}{ll}
\small\spadtype{COMPLEX}   abbreviates \spadtype{Complex}             &
\small\spadtype{DFLOAT}    abbreviates \spadtype{DoubleFloat}         \\
\small\spadtype{EXPR}      abbreviates \spadtype{Expression}          &
\small\spadtype{FLOAT}     abbreviates \spadtype{Float}               \\
\small\spadtype{FRAC}      abbreviates \spadtype{Fraction}            &
\small\spadtype{INT}       abbreviates \spadtype{Integer}             \\
\small\spadtype{MATRIX}    abbreviates \spadtype{Matrix}              &
\small\spadtype{NNI}       abbreviates \spadtype{NonNegativeInteger}  \\
\small\spadtype{PI}        abbreviates \spadtype{PositiveInteger}     &
\small\spadtype{POLY}      abbreviates \spadtype{Polynomial}          \\
\small\spadtype{STRING}    abbreviates \spadtype{String}              &
\small\spadtype{UP}        abbreviates \spadtype{UnivariatePolynomial}\\
\end{tabular}
\end{center}
\endImportant

You can combine both full constructor names and abbreviations
in a type expression.
Here are some types using abbreviations.
\begin{center}
\spadtype{POLY INT} is the same as \spadtype{Polynomial(INT)} \\
\spadtype{POLY(Integer)} is the same as \spadtype{Polynomial(Integer)} \\
\spadtype{POLY(Integer)} is the same as \spadtype{Polynomial(INT)} \\
\spadtype{FRAC(COMPLEX(INT))} is the same as \spadtype{Fraction Complex Integer} \\
\spadtype{FRAC(COMPLEX(INT))} is the same as \spadtype{FRAC(Complex Integer)} \\
\end{center}

There are several ways of finding the names of constructors and
their abbreviations.
For a specific constructor, use \spadsys{)abbreviation query}.
\syscmdindex{abbreviation}
You can also use the \spadsys{)what} system command to see the names
and abbreviations of constructors.
\syscmdindex{what}
For more information about \spadsys{)what}, see
\spadref{ugSysCmdwhat}.
\begin{xtc}
\begin{xtccomment}
\spadsys{)abbreviation query} can be
abbreviated (no pun intended) to \spadsys{)abb q}.
\end{xtccomment}
\begin{spadsrc}
)abb q Integer
\end{spadsrc}
\begin{SysCmdOutput}
   INT abbreviates domain Integer 
\end{SysCmdOutput}
\end{xtc}
\begin{xtc}
\begin{xtccomment}
The \spadsys{)abbreviation query} command lists
the constructor name if you give the abbreviation.
Issue \spadsys{)abb q} if you want to see the names and abbreviations
of all \Language{} constructors.
\end{xtccomment}
\begin{spadsrc}
)abb q DMP
\end{spadsrc}
\begin{SysCmdOutput}
   DMP abbreviates domain DistributedMultivariatePolynomial 
\end{SysCmdOutput}
\end{xtc}
\begin{xtc}
\begin{xtccomment}
Issue this to see all packages whose names contain the string ``ode''.
\syscmdindex{what packages}
\end{xtccomment}
\begin{spadsrc}
)what packages ode
\end{spadsrc}
\begin{SysCmdOutput}
-------------------------------- Packages ---------------------------------

Packages with names matching patterns:
     ode 

 COMPCODE compCode                     EXPRODE  ExpressionSpaceODESolver
 FCPAK1   FortranCodePackage1          FCTOOL   FortranCodeTools
 GRAY     GrayCode                     LODEEF   ElementaryFunctionLODESolver
 NODE1    NonLinearFirstOrderODESolver  ODECONST ConstantLODE
 ODEEF    ElementaryFunctionODESolver  ODEINT   ODEIntegration
 ODEPAL   PureAlgebraicLODE            ODERAT   RationalLODE
 ODERED   ReduceLODE                   ODESYS   SystemODESolver
 ODETOOLS ODETools
 UTSODE   UnivariateTaylorSeriesODESolver
 UTSODETL UTSodetools
\end{SysCmdOutput}
\end{xtc}

% *********************************************************************
\head{section}{Declarations}{ugTypesDeclare}
% *********************************************************************
%
A \spadgloss{declaration} is an expression used
to restrict the type of values that can be assigned to variables.
A colon (\spadSyntax{:}) is always used after a variable or
list of variables to be declared.

\beginImportant
For a single variable, the syntax for declaration is
\begin{center}
{\it variableName \spad{:} typeOrMode}
\end{center}
For multiple variables, the syntax is
\begin{center}
{\tt (\subscriptIt{variableName}{1}, \subscriptIt{variableName}{2}, \ldots \subscriptIt{variableName}{N}): {\it typeOrMode}}
\end{center}
\endImportant

You can always combine a declaration with an assignment.
When you do, it is equivalent to first giving a declaration statement,
then giving an assignment.
For more information on assignment, see
\spadref{ugIntroAssign} and
\spadref{ugLangAssign}.
To see how to declare your own functions, see
\spadref{ugUserDeclare}.

\begin{xtc}
\begin{xtccomment}
This declares one variable to have a type.
\end{xtccomment}
\begin{spadsrc}
a : Integer 
\end{spadsrc}
\end{xtc}
\begin{xtc}
\begin{xtccomment}
This declares several variables to have a type.
\end{xtccomment}
\begin{spadsrc}
(b,c) : Integer 
\end{spadsrc}
\end{xtc}
\begin{xtc}
\begin{xtccomment}
\spad{a, b} and \spad{c} can only hold integer values.
\end{xtccomment}
\begin{spadsrc}
a := 45 
\end{spadsrc}
\begin{TeXOutput}
\begin{fricasmath}{3}
45%
\end{fricasmath}
\end{TeXOutput}
\formatResultType{Integer}
\end{xtc}
\begin{xtc}
\begin{xtccomment}
If a value cannot be converted to a declared type,
an error message is displayed.
\end{xtccomment}
\begin{spadsrc}
b := 4/5 
\end{spadsrc}
\begin{MessageOutput}
   Cannot convert right-hand side of assignment
   4
   -
   5

      to an object of the type Integer of the left-hand side.
\end{MessageOutput}
\end{xtc}
\begin{xtc}
\begin{xtccomment}
This declares a variable with a mode.
\end{xtccomment}
\begin{spadsrc}
n : Complex ? 
\end{spadsrc}
\end{xtc}
\begin{xtc}
\begin{xtccomment}
This declares several variables with a mode.
\end{xtccomment}
\begin{spadsrc}
(p,q,r) : Matrix Polynomial ? 
\end{spadsrc}
\end{xtc}
\begin{xtc}
\begin{xtccomment}
This complex object has integer real and imaginary parts.
\end{xtccomment}
\begin{spadsrc}
n := -36 + 9 * %i 
\end{spadsrc}
\begin{TeXOutput}
\begin{fricasmath}{6}
-{36}+9\TIMES \ImaginaryI %
\end{fricasmath}
\end{TeXOutput}
\formatResultType{Complex(Integer)}
\end{xtc}
\begin{xtc}
\begin{xtccomment}
This complex object has fractional symbolic real and imaginary parts.
\end{xtccomment}
\begin{spadsrc}
n := complex(4/(x + y),y/x) 
\end{spadsrc}
\begin{TeXOutput}
\begin{fricasmath}{7}
\frac{4}{\SYMBOL{y}+\SYMBOL{x}}+\frac{\SYMBOL{y}}{\SYMBOL{x}}\TIMES %
\ImaginaryI %
\end{fricasmath}
\end{TeXOutput}
\formatResultType{Complex(Fraction(Polynomial(Integer)))}
\end{xtc}
\begin{xtc}
\begin{xtccomment}
This matrix has entries that are polynomials with integer
coefficients.
\end{xtccomment}
\begin{spadsrc}
p := [[1,2],[3,4],[5,6]] 
\end{spadsrc}
\begin{TeXOutput}
\begin{fricasmath}{8}
\begin{MATRIX}{2}1&2\\3&4\\5&6\end{MATRIX}%
\end{fricasmath}
\end{TeXOutput}
\formatResultType{Matrix(Polynomial(Integer))}
\end{xtc}
\begin{xtc}
\begin{xtccomment}
This matrix has a single entry that is a polynomial with
rational number coefficients.
\end{xtccomment}
\begin{spadsrc}
q := [[x - 2/3]] 
\end{spadsrc}
\begin{TeXOutput}
\begin{fricasmath}{9}
\begin{MATRIX}{1}\SYMBOL{x}-{\frac{2}{3}}\end{MATRIX}%
\end{fricasmath}
\end{TeXOutput}
\formatResultType{Matrix(Polynomial(Fraction(Integer)))}
\end{xtc}
\begin{xtc}
\begin{xtccomment}
This matrix has entries that are polynomials with complex integer
coefficients.
\end{xtccomment}
\begin{spadsrc}
r := [[1-%i*x,7*y+4*%i]] 
\end{spadsrc}
\begin{TeXOutput}
\begin{fricasmath}{10}
\begin{MATRIX}{2}-{\ImaginaryI \TIMES \SYMBOL{x}}+1&7\TIMES \SYMBOL{y}+4%
\TIMES \ImaginaryI \end{MATRIX}%
\end{fricasmath}
\end{TeXOutput}
\formatResultType{Matrix(Polynomial(Complex(Integer)))}
\end{xtc}
%
\begin{xtc}
\begin{xtccomment}
Note the difference between this and the next example.
This is a complex object with polynomial real and imaginary parts.
\end{xtccomment}
\begin{spadsrc}
f : COMPLEX POLY ? := (x + y*%i)^2
\end{spadsrc}
\begin{TeXOutput}
\begin{fricasmath}{11}
-{\SUPER{\SYMBOL{y}}{2}}+\SUPER{\SYMBOL{x}}{2}+2\TIMES \SYMBOL{x}\TIMES %
\SYMBOL{y}\TIMES \ImaginaryI %
\end{fricasmath}
\end{TeXOutput}
\formatResultType{Complex(Polynomial(Integer))}
\end{xtc}
\begin{xtc}
\begin{xtccomment}
This is a polynomial with complex integer coefficients.
The objects are convertible from one to the other.
See \spadref{ugTypesConvert} for more information.
\end{xtccomment}
\begin{spadsrc}
g : POLY COMPLEX ? := (x + y*%i)^2
\end{spadsrc}
\begin{TeXOutput}
\begin{fricasmath}{12}
-{\SUPER{\SYMBOL{y}}{2}}+2\TIMES \ImaginaryI \TIMES \SYMBOL{x}\TIMES \SYMBOL{%
y}+\SUPER{\SYMBOL{x}}{2}%
\end{fricasmath}
\end{TeXOutput}
\formatResultType{Polynomial(Complex(Integer))}
\end{xtc}

% *********************************************************************
\head{section}{Records}{ugTypesRecords}
% *********************************************************************
%
A \pspadtype{Record} is an object composed of one or more other objects,
\index{Record@\protect\nonLibAxiomType{Record}}
each of which is referenced
\index{selector!record}
with
\index{record!selector}
a \spadgloss{selector}.
Components can all belong to the same type or each can have a different type.

% ----------------------------------------------------------------------
\beginImportant
The syntax for writing a \pspadtype{Record} type is
\begin{center}
{\tt Record(\subscriptIt{selector}{1}:\subscriptIt{type}{1}, \subscriptIt{selector}{2}:\subscriptIt{type}{2}, \ldots, \subscriptIt{selector}{N}:\subscriptIt{type}{N})}
\end{center}
You must be careful if a selector has the same name as a variable in the
workspace.
If this occurs, precede the selector name by a single
\index{quote}
quote.
\endImportant
% ----------------------------------------------------------------------

Record components are implicitly ordered.
All the components of a record can
be set at once by assigning the record a
bracketed \spadgloss{tuple} of values of the proper length
(for example, \spad{r : Record(a: Integer, b: String) := [1, "two"]}).
To access a component of a record \spad{r},
write the name \spad{r}, followed by a period, followed by a selector.

%
\begin{xtc}
\begin{xtccomment}
The object returned by this computation is a record with two components: a
\spad{quotient} part and a \spad{remainder} part.
\end{xtccomment}
\begin{spadsrc}
u := divide(5,2) 
\end{spadsrc}
\begin{TeXOutput}
\begin{fricasmath}{1}
\BRACKET{\SYMBOL{quotient}=2\COMMA \SYMBOL{remainder}=1}%
\end{fricasmath}
\end{TeXOutput}
\formatResultType{Record(quotient: Integer, remainder: Integer)}
\end{xtc}
%
\begin{xtc}
\begin{xtccomment}
This is the quotient part.
\end{xtccomment}
\begin{spadsrc}
u.quotient 
\end{spadsrc}
\begin{TeXOutput}
\begin{fricasmath}{2}
2%
\end{fricasmath}
\end{TeXOutput}
\formatResultType{PositiveInteger}
\end{xtc}
\begin{xtc}
\begin{xtccomment}
This is the remainder part.
\end{xtccomment}
\begin{spadsrc}
u.remainder 
\end{spadsrc}
\begin{TeXOutput}
\begin{fricasmath}{3}
1%
\end{fricasmath}
\end{TeXOutput}
\formatResultType{PositiveInteger}
\end{xtc}
%
\begin{xtc}
\begin{xtccomment}
You can use selector expressions on the left-hand side of an assignment
to change destructively the components of a record.
\end{xtccomment}
\begin{spadsrc}
u.quotient := 8978 
\end{spadsrc}
\begin{TeXOutput}
\begin{fricasmath}{4}
8978%
\end{fricasmath}
\end{TeXOutput}
\formatResultType{PositiveInteger}
\end{xtc}
\begin{xtc}
\begin{xtccomment}
The selected component \spad{quotient} has the value \spad{8978},
which is what is returned by the assignment.
Check that the value of \spad{u} was modified.
\end{xtccomment}
\begin{spadsrc}
u 
\end{spadsrc}
\begin{TeXOutput}
\begin{fricasmath}{5}
\BRACKET{\SYMBOL{quotient}=8978\COMMA \SYMBOL{remainder}=1}%
\end{fricasmath}
\end{TeXOutput}
\formatResultType{Record(quotient: Integer, remainder: Integer)}
\end{xtc}
\begin{xtc}
\begin{xtccomment}
Selectors are evaluated.
Thus you can use variables that evaluate to selectors instead of the
selectors themselves.
\end{xtccomment}
\begin{spadsrc}
s := 'quotient 
\end{spadsrc}
\begin{TeXOutput}
\begin{fricasmath}{6}
\SYMBOL{quotient}%
\end{fricasmath}
\end{TeXOutput}
\formatResultType{Variable(quotient)}
\end{xtc}
\begin{xtc}
\begin{xtccomment}
Be careful!
A selector could have the same name as a variable in the workspace.
If this occurs, precede the selector name by a single quote, as in
\spad{u.'quotient}.
\index{selector!quoting}
\end{xtccomment}
\begin{spadsrc}
divide(5,2).s 
\end{spadsrc}
\begin{TeXOutput}
\begin{fricasmath}{7}
2%
\end{fricasmath}
\end{TeXOutput}
\formatResultType{PositiveInteger}
\end{xtc}
\begin{xtc}
\begin{xtccomment}
Here we declare that the value of \spad{bd}
has two components: a string,
to be accessed via \spad{name}, and an integer,
to be accessed via \spad{birthdayMonth}.
\end{xtccomment}
\begin{spadsrc}
bd : Record(name : String, birthdayMonth : Integer) 
\end{spadsrc}
\end{xtc}
\begin{xtc}
\begin{xtccomment}
You must initially set the value of the entire \pspadtype{Record}
at once.
\end{xtccomment}
\begin{spadsrc}
bd := ["Judith", 3] 
\end{spadsrc}
\begin{TeXOutput}
\begin{fricasmath}{9}
\BRACKET{\SYMBOL{name}=\STRING{"Judith"}\COMMA \SYMBOL{birthdayMonth}=3}%
\end{fricasmath}
\end{TeXOutput}
\formatResultType{Record(name: String, birthdayMonth: Integer)}
\end{xtc}
\begin{xtc}
\begin{xtccomment}
Once set, you can change any of the individual components.
\end{xtccomment}
\begin{spadsrc}
bd.name := "Katie" 
\end{spadsrc}
\begin{TeXOutput}
\begin{fricasmath}{10}
\STRING{"Katie"}%
\end{fricasmath}
\end{TeXOutput}
\formatResultType{String}
\end{xtc}
\begin{xtc}
\begin{xtccomment}
Records may be nested and the selector names can be shared at
different levels.
\end{xtccomment}
\begin{spadsrc}
r : Record(a : Record(b: Integer, c: Integer), b: Integer) 
\end{spadsrc}
\end{xtc}
\begin{xtc}
\begin{xtccomment}
The record \spad{r} has a \spad{b} selector at two different levels.
Here is an initial value for \spad{r}.
\end{xtccomment}
\begin{spadsrc}
r := [[1,2],3] 
\end{spadsrc}
\begin{TeXOutput}
\begin{fricasmath}{12}
\BRACKET{\SYMBOL{a}=\BRACKET{\SYMBOL{b}=1\COMMA \SYMBOL{c}=2}\COMMA \SYMBOL{b%
}=3}%
\end{fricasmath}
\end{TeXOutput}
\formatResultType{Record(a: Record(b: Integer, c: Integer), b: Integer)}
\end{xtc}
\begin{xtc}
\begin{xtccomment}
This extracts the \spad{b} component from the \spad{a} component of \spad{r}.
\end{xtccomment}
\begin{spadsrc}
r.a.b 
\end{spadsrc}
\begin{TeXOutput}
\begin{fricasmath}{13}
1%
\end{fricasmath}
\end{TeXOutput}
\formatResultType{PositiveInteger}
\end{xtc}
\begin{xtc}
\begin{xtccomment}
This extracts the \spad{b} component from \spad{r}.
\end{xtccomment}
\begin{spadsrc}
r.b 
\end{spadsrc}
\begin{TeXOutput}
\begin{fricasmath}{14}
3%
\end{fricasmath}
\end{TeXOutput}
\formatResultType{PositiveInteger}
\end{xtc}
%
\begin{xtc}
\begin{xtccomment}
You can also use spaces or parentheses to refer to \pspadtype{Record}
components.
This is the same as \spad{r.a}.
\end{xtccomment}
\begin{spadsrc}
r(a) 
\end{spadsrc}
\begin{TeXOutput}
\begin{fricasmath}{15}
\BRACKET{\SYMBOL{b}=1\COMMA \SYMBOL{c}=2}%
\end{fricasmath}
\end{TeXOutput}
\formatResultType{Record(b: Integer, c: Integer)}
\end{xtc}
\begin{xtc}
\begin{xtccomment}
This is the same as \spad{r.b}.
\end{xtccomment}
\begin{spadsrc}
r b 
\end{spadsrc}
\begin{TeXOutput}
\begin{fricasmath}{16}
3%
\end{fricasmath}
\end{TeXOutput}
\formatResultType{PositiveInteger}
\end{xtc}
\begin{xtc}
\begin{xtccomment}
This is the same as \spad{r.b := 10}.
\end{xtccomment}
\begin{spadsrc}
r(b) := 10 
\end{spadsrc}
\begin{TeXOutput}
\begin{fricasmath}{17}
10%
\end{fricasmath}
\end{TeXOutput}
\formatResultType{PositiveInteger}
\end{xtc}
\begin{xtc}
\begin{xtccomment}
Look at \spad{r} to make sure it was modified.
\end{xtccomment}
\begin{spadsrc}
r 
\end{spadsrc}
\begin{TeXOutput}
\begin{fricasmath}{18}
\BRACKET{\SYMBOL{a}=\BRACKET{\SYMBOL{b}=1\COMMA \SYMBOL{c}=2}\COMMA \SYMBOL{b%
}=10}%
\end{fricasmath}
\end{TeXOutput}
\formatResultType{Record(a: Record(b: Integer, c: Integer), b: Integer)}
\end{xtc}

% *********************************************************************
\head{section}{Unions}{ugTypesUnions}
% *********************************************************************
%
Type \pspadtype{Union} is used for objects that
can be of any of a specific finite set of types.
\index{Union@\protect\nonLibAxiomType{Union}}
Two versions of unions are available,
one with selectors (like records) and one without.
\index{union}

% *********************************************************************
\head{subsection}{Unions Without Selectors}{ugTypesUnionsWOSel}
% *********************************************************************

The declaration \spad{x : Union(Integer, String, Float)}
states that \spad{x} can have values that are integers,
strings or ``big'' floats.
If, for example, the \pspadtype{Union} object is an integer, the object is
said to belong to the \spadtype{Integer} {\it branch}
of the \pspadtype{Union}.\footnote{
Note that we are being a bit careless with the language here.
Technically, the type of \spad{x} is always
\pspadtype{Union(Integer, String, Float)}.
If it belongs to the \spadtype{Integer} branch, \spad{x}
may be converted to an object of type \spadtype{Integer}.}

% ----------------------------------------------------------------------
\beginImportant
The syntax for writing a \pspadtype{Union} type without selectors is
\begin{center}
{\tt Union(\subscriptIt{type}{1}, \subscriptIt{type}{2}, \ldots, \subscriptIt{type}{N})}
\end{center}
The types in a union without selectors must be distinct.
\endImportant
% ----------------------------------------------------------------------

It is possible to create unions like
\pspadtype{Union(Integer, PositiveInteger)} but they are
difficult to work with because of the overlap in the branch
types.
See below for the rules \Language{} uses for converting something
into a union object.

The \spad{case} infix
\spadkey{case}
operator returns a \spadtype{Boolean}
and can be used to determine the branch in which an object lies.

\begin{xtc}
\begin{xtccomment}
This function displays a message stating in which
branch of the \pspadtype{Union} the object (defined as \spad{x}
above) lies.
\end{xtccomment}
\begin{spadsrc}
sayBranch(x : Union(Integer,String,Float)) : Void  ==
  output
    x case Integer => "Integer branch"
    x case String  => "String branch"
    "Float branch"
\end{spadsrc}
\begin{MessageOutput}
   Function declaration sayBranch : Union(Integer,String,Float) -> Void
      has been added to workspace.
\end{MessageOutput}
\end{xtc}
%
\begin{xtc}
\begin{xtccomment}
This tries \userfun{sayBranch} with an integer.
\end{xtccomment}
\begin{spadsrc}
sayBranch 1 
\end{spadsrc}
\begin{MessageOutput}
   Compiling function sayBranch with type Union(Integer,String,Float)
       -> Void 
\end{MessageOutput}
\end{xtc}
\begin{xtc}
\begin{xtccomment}
This tries \userfun{sayBranch} with a string.
\end{xtccomment}
\begin{spadsrc}
sayBranch "hello" 
\end{spadsrc}
\end{xtc}
\begin{xtc}
\begin{xtccomment}
This tries \userfun{sayBranch} with a floating-point number.
\end{xtccomment}
\begin{spadsrc}
sayBranch 2.718281828 
\end{spadsrc}
\end{xtc}
%

There are two things of interest about this particular
example to which we would like to draw your attention.
\begin{enumerate}
%
\item \Language{} normally converts a result to the target value
before passing it to the function.
If we left the declaration information out of this function definition
then the \spad{sayBranch} call would have been attempted with an
\spadtype{Integer} rather than a \pspadtype{Union}, and an error would have
resulted.
%
\item The types in a \pspadtype{Union} are searched in the order given.
So if the type were given as

\noindent
{\small\spad{sayBranch(x: Union(String,Integer,Float,Any)): Void}}

\noindent
then the result would have been ``String branch'' because there
is a conversion from \spadtype{Integer} to \spadtype{String}.
\end{enumerate}

Sometimes \pspadtype{Union} types can have extremely
long names.
\Language{} therefore abbreviates the names of unions by printing
the type of the branch first within the \pspadtype{Union} and then
eliding the remaining types with an ellipsis (\spadSyntax{...}).

\begin{xtc}
\begin{xtccomment}
Here the \spadtype{Integer} branch is displayed first.
Use \spadSyntax{::} to create a \pspadtype{Union} object from an object.
\end{xtccomment}
\begin{spadsrc}
78 :: Union(Integer,String)
\end{spadsrc}
\begin{TeXOutput}
\begin{fricasmath}{5}
78%
\end{fricasmath}
\end{TeXOutput}
\formatResultType{Union(Integer, ...)}
\end{xtc}
\begin{xtc}
\begin{xtccomment}
Here the \spadtype{String} branch is displayed first.
\end{xtccomment}
\begin{spadsrc}
s := "string" :: Union(Integer,String) 
\end{spadsrc}
\begin{TeXOutput}
\begin{fricasmath}{6}
\STRING{"string"}%
\end{fricasmath}
\end{TeXOutput}
\formatResultType{Union(String, ...)}
\end{xtc}
\begin{xtc}
\begin{xtccomment}
Use \spad{typeOf} to see the full and actual \pspadtype{Union} type.
\spadkey{typeOf}
\end{xtccomment}
\begin{spadsrc}
typeOf s
\end{spadsrc}
\begin{TeXOutput}
\begin{fricasmath}{7}
\STRING{Union(Integer,String)}%
\end{fricasmath}
\end{TeXOutput}
\formatResultType{Type}
\end{xtc}
\begin{xtc}
\begin{xtccomment}
A common operation that returns a union is \spadfunFrom{exquo}{Integer}
which returns the ``exact quotient'' if the quotient is exact,...
\end{xtccomment}
\begin{spadsrc}
three := exquo(6,2) 
\end{spadsrc}
\begin{TeXOutput}
\begin{fricasmath}{8}
3%
\end{fricasmath}
\end{TeXOutput}
\formatResultType{Union(Integer, ...)}
\end{xtc}
\begin{xtc}
\begin{xtccomment}
and \spad{"failed"} if the quotient is not exact.
\end{xtccomment}
\begin{spadsrc}
exquo(5,2)
\end{spadsrc}
\begin{TeXOutput}
\begin{fricasmath}{9}
\STRING{"failed"}%
\end{fricasmath}
\end{TeXOutput}
\formatResultType{Union("failed", ...)}
\end{xtc}
\begin{xtc}
\begin{xtccomment}
A union with a \spad{"failed"} is frequently used to indicate the failure
or lack of applicability of an object.
As another example, assign an integer a variable \spad{r} declared to be a
rational number.
\end{xtccomment}
\begin{spadsrc}
r: FRAC INT := 3 
\end{spadsrc}
\begin{TeXOutput}
\begin{fricasmath}{10}
3%
\end{fricasmath}
\end{TeXOutput}
\formatResultType{Fraction(Integer)}
\end{xtc}
\begin{xtc}
\begin{xtccomment}
The operation \spadfunFrom{retractIfCan}{Fraction} tries to retract the
fraction to the underlying domain \spadtype{Integer}.
It produces a union object.
Here it succeeds.
\end{xtccomment}
\begin{spadsrc}
retractIfCan(r) 
\end{spadsrc}
\begin{TeXOutput}
\begin{fricasmath}{11}
3%
\end{fricasmath}
\end{TeXOutput}
\formatResultType{Union(Integer, ...)}
\end{xtc}
\begin{xtc}
\begin{xtccomment}
Assign it a rational number.
\end{xtccomment}
\begin{spadsrc}
r := 3/2 
\end{spadsrc}
\begin{TeXOutput}
\begin{fricasmath}{12}
\frac{3}{2}%
\end{fricasmath}
\end{TeXOutput}
\formatResultType{Fraction(Integer)}
\end{xtc}
\begin{xtc}
\begin{xtccomment}
Here the retraction fails.
\end{xtccomment}
\begin{spadsrc}
retractIfCan(r) 
\end{spadsrc}
\begin{TeXOutput}
\begin{fricasmath}{13}
\STRING{"failed"}%
\end{fricasmath}
\end{TeXOutput}
\formatResultType{Union("failed", ...)}
\end{xtc}

% *********************************************************************
\head{subsection}{Unions With Selectors}{ugTypesUnionsWSel}
% *********************************************************************

Like records (\spadref{ugTypesRecords}),
you can write \pspadtype{Union} types
\index{selector!union}
with selectors.
\index{union!selector}

% ----------------------------------------------------------------------
\beginImportant
The syntax for writing a \pspadtype{Union} type with selectors is
\begin{center}
{\tt Union(\subscriptIt{selector}{1}:\subscriptIt{type}{1}, \subscriptIt{selector}{2}:\subscriptIt{type}{2}, \ldots, \subscriptIt{selector}{N}:\subscriptIt{type}{N})}
\end{center}
You must be careful if a selector has the same name as a variable in the
workspace.
If this occurs, precede the selector name by a single
\index{quote}
quote.
\index{selector!quoting}
It is an error to use a selector that does not correspond to the branch of
the \pspadtype{Union} in which the element actually lies.
\endImportant
% ----------------------------------------------------------------------

Be sure to understand the difference between records and unions
with selectors.
\index{union!difference from record}
Records can have more than one component and the selectors are
used to refer to the components.
\index{record!difference from union}
Unions always have one component but the type of that one
component can vary.
An object of type \pspadtype{Record(a: Integer, b: Float, c: String)}
contains an integer {\it and} a float  {\it and} a
string.
An object of type \pspadtype{Union(a: Integer, b: Float, c: String)}
contains an integer {\it or} a float  {\it or} a
string.

Here is a version of the \userfun{sayBranch} function (cf.
\spadref{ugTypesUnionsWOSel}) that works with a union with selectors.
It displays a message stating in which branch of the \pspadtype{Union} the
object lies.
\begin{verbatim}
sayBranch(x:Union(i:Integer,s:String,f:Float)):Void==
  output
    x case i => "Integer branch"
    x case s  => "String branch"
    "Float branch"
\end{verbatim}
Note that \spad{case} uses the selector name as its right-hand argument.
\spadkey{case}
If you accidentally use the branch type on the right-hand side of
\spad{case}, \spad{false} will be returned.

\begin{xtc}
\begin{xtccomment}
Declare variable \spad{u} to have a union type with selectors.
\end{xtccomment}
\begin{spadsrc}
u : Union(i : Integer, s : String) 
\end{spadsrc}
\end{xtc}
\begin{xtc}
\begin{xtccomment}
Give an initial value to \spad{u}.
\end{xtccomment}
\begin{spadsrc}
u := "good morning" 
\end{spadsrc}
\begin{TeXOutput}
\begin{fricasmath}{2}
\STRING{"good\ morning"}%
\end{fricasmath}
\end{TeXOutput}
\formatResultType{Union(s: String, ...)}
\end{xtc}
\begin{xtc}
\begin{xtccomment}
Use \spad{case} to determine in which
branch of a \pspadtype{Union} an object lies.
\end{xtccomment}
\begin{spadsrc}
u case i 
\end{spadsrc}
\begin{TeXOutput}
\begin{fricasmath}{3}
\STRING{false}%
\end{fricasmath}
\end{TeXOutput}
\formatResultType{Boolean}
\end{xtc}
\begin{xtc}
\begin{xtccomment}
\end{xtccomment}
\begin{spadsrc}
u case s 
\end{spadsrc}
\begin{TeXOutput}
\begin{fricasmath}{4}
\STRING{true}%
\end{fricasmath}
\end{TeXOutput}
\formatResultType{Boolean}
\end{xtc}
\begin{xtc}
\begin{xtccomment}
To access the element in a particular branch, use the selector.
\end{xtccomment}
\begin{spadsrc}
u.s 
\end{spadsrc}
\begin{TeXOutput}
\begin{fricasmath}{5}
\STRING{"good\ morning"}%
\end{fricasmath}
\end{TeXOutput}
\formatResultType{String}
\end{xtc}

% *********************************************************************
\head{section}{The ``Any'' Domain}{ugTypesAnyNone}
% *********************************************************************

With the exception of objects of type \pspadtype{Record}, all \Language{}
data structures are homogenous, that is, they hold objects all of the same
type.
\exptypeindex{Any}
If you need to get around this, you can use type \spadtype{Any}.
Using \spadtype{Any}, for example, you can create lists whose
elements are integers, rational numbers, strings, and even other lists.

\begin{xtc}
\begin{xtccomment}
Declare \spad{u} to have type \spadtype{Any}.
\end{xtccomment}
\begin{spadsrc}
u: Any
\end{spadsrc}
\end{xtc}
\begin{xtc}
\begin{xtccomment}
Assign a list of mixed type values to \spad{u}
\end{xtccomment}
\begin{spadsrc}
u := [1, 7.2, 3/2, x^2, "wally"]
\end{spadsrc}
\begin{TeXOutput}
\begin{fricasmath}{2}
\BRACKET{1\COMMA \STRING{7.2}\COMMA \frac{3}{2}\COMMA \SUPER{\SYMBOL{x}}{2}%
\COMMA \STRING{"wally"}}%
\end{fricasmath}
\end{TeXOutput}
\formatResultType{List(Any)}
\end{xtc}
\begin{xtc}
\begin{xtccomment}
When we ask for the elements, \Language{} displays these types.
\end{xtccomment}
\begin{spadsrc}
u.1 
\end{spadsrc}
\begin{TeXOutput}
\begin{fricasmath}{3}
1%
\end{fricasmath}
\end{TeXOutput}
\formatResultType{PositiveInteger}
\end{xtc}
\begin{xtc}
\begin{xtccomment}
Actually, these objects belong to \spadtype{Any} but \Language{}
automatically converts them to their natural types for you.
\end{xtccomment}
\begin{spadsrc}
u.3 
\end{spadsrc}
\begin{TeXOutput}
\begin{fricasmath}{4}
\frac{3}{2}%
\end{fricasmath}
\end{TeXOutput}
\formatResultType{Fraction(Integer)}
\end{xtc}
\begin{xtc}
\begin{xtccomment}
Since type \spadtype{Any} can be anything,
it can only belong to type \spadtype{Type}.
Therefore it cannot be used in algebraic domains.
\end{xtccomment}
\begin{spadsrc}
v : Matrix(Any)
\end{spadsrc}
\begin{MessageOutput}
   Matrix(Any) is not a valid type.
\end{MessageOutput}
\end{xtc}

Perhaps you are wondering how \Language{} internally represents
objects of type \spadtype{Any}.
An object of type \spadtype{Any} consists not only a data part
representing its normal value, but also a type part (a {\it badge}) giving
\index{badge}
its type.
For example, the value \spad{1} of type \spadtype{PositiveInteger} as an
object of type \spadtype{Any} internally looks like
\spad{[1,PositiveInteger()]}.

%When should you use \spadtype{Any} instead of a \pspadtype{Union} type?
%Can you plan ahead?
%For a \pspadtype{Union}, you must know in advance exactly which types you
%are
%\index{union!vs. Any@{vs. \protect\nonLibAxiomType{Any}}}
%going to allow.
%For \spadtype{Any}, anything that comes along can be accommodated.

% *********************************************************************
\head{section}{Conversion}{ugTypesConvert}
% *********************************************************************
%
\beginImportant
\spadglossSee{Conversion}{conversion}
is the process of changing an object of one type
into an object of another type.
The syntax for conversion is:
\begin{center}
{\it object} {\tt ::} {\it newType}
\end{center}
\endImportant

\begin{xtc}
\begin{xtccomment}
By default, \spad{3} has the type \spadtype{PositiveInteger}.
\end{xtccomment}
\begin{spadsrc}
3
\end{spadsrc}
\begin{TeXOutput}
\begin{fricasmath}{1}
3%
\end{fricasmath}
\end{TeXOutput}
\formatResultType{PositiveInteger}
\end{xtc}
\begin{xtc}
\begin{xtccomment}
We can change this into an object of type \spadtype{Fraction Integer}
by using \spadSyntax{::}.
\end{xtccomment}
\begin{spadsrc}
3 :: Fraction Integer
\end{spadsrc}
\begin{TeXOutput}
\begin{fricasmath}{2}
3%
\end{fricasmath}
\end{TeXOutput}
\formatResultType{Fraction(Integer)}
\end{xtc}

A \spadgloss{coercion} is a special kind of conversion that \Language{} is
allowed to do automatically when you enter an expression.
Coercions are usually somewhat safer than more general conversions.
The \Language{} library contains operations called
\spadfun{coerce} and \spadfun{convert}.
Only the \spadfun{coerce} operations can be used by the
interpreter to change an object into an object of another type unless
you explicitly use a \spadSyntax{::}.

By now you will be quite familiar with what types and modes look like.
It is useful to think of a type or mode as a pattern
for what you want the result to be.
\begin{xtc}
\begin{xtccomment}
Let's start with a square matrix of polynomials with complex rational number
coefficients.
\exptypeindex{SquareMatrix}
\end{xtccomment}
\begin{spadsrc}
m : SquareMatrix(2,POLY COMPLEX FRAC INT) 
\end{spadsrc}
\end{xtc}
\begin{xtc}
\begin{xtccomment}
\end{xtccomment}
\begin{spadsrc}
m := matrix [[x-3/4*%i,z*y^2+1/2],[3/7*%i*y^4 - x,12-%i*9/5]] 
\end{spadsrc}
\begin{TeXOutput}
\begin{fricasmath}{4}
\begin{MATRIX}{2}\SYMBOL{x}-{\frac{3}{4}\TIMES \ImaginaryI }&\SUPER{\SYMBOL{y%
}}{2}\TIMES \SYMBOL{z}+\frac{1}{2}\\\frac{3}{7}\TIMES \ImaginaryI \TIMES %
\SUPER{\SYMBOL{y}}{4}-{\SYMBOL{x}}&12-{\frac{9}{5}\TIMES \ImaginaryI }%
\end{MATRIX}%
\end{fricasmath}
\end{TeXOutput}
\formatResultType{SquareMatrix(2, Polynomial(Complex(Fraction(Integer))))}
\end{xtc}
\begin{xtc}
\begin{xtccomment}
We first want to interchange the \spadtype{Complex} and
\spadtype{Fraction} layers.
We do the conversion by doing the interchange in the type expression.
\end{xtccomment}
\begin{spadsrc}
m1 := m :: SquareMatrix(2,POLY FRAC COMPLEX INT) 
\end{spadsrc}
\begin{TeXOutput}
\begin{fricasmath}{5}
\begin{MATRIX}{2}\SYMBOL{x}-{\frac{3\TIMES \ImaginaryI }{4}}&\SUPER{\SYMBOL{y%
}}{2}\TIMES \SYMBOL{z}+\frac{1}{2}\\\frac{3\TIMES \ImaginaryI }{7}\TIMES %
\SUPER{\SYMBOL{y}}{4}-{\SYMBOL{x}}&\frac{60-{9\TIMES \ImaginaryI }}{5}%
\end{MATRIX}%
\end{fricasmath}
\end{TeXOutput}
\formatResultType{SquareMatrix(2, Polynomial(Fraction(Complex(Integer))))}
\end{xtc}
\begin{xtc}
\begin{xtccomment}
Interchange the \spadtype{Polynomial} and the
\spadtype{Fraction} levels.
\end{xtccomment}
\begin{spadsrc}
m2 := m1 :: SquareMatrix(2,FRAC POLY COMPLEX INT) 
\end{spadsrc}
\begin{TeXOutput}
\begin{fricasmath}{6}
\begin{MATRIX}{2}\frac{4\TIMES \SYMBOL{x}-{3\TIMES \ImaginaryI }}{4}&\frac{2%
\TIMES \SUPER{\SYMBOL{y}}{2}\TIMES \SYMBOL{z}+1}{2}\\\frac{3\TIMES %
\ImaginaryI \TIMES \SUPER{\SYMBOL{y}}{4}-{7\TIMES \SYMBOL{x}}}{7}&\frac{60-{9%
\TIMES \ImaginaryI }}{5}\end{MATRIX}%
\end{fricasmath}
\end{TeXOutput}
\formatResultType{SquareMatrix(2, Fraction(Polynomial(Complex(Integer))))}
\end{xtc}
\begin{xtc}
\begin{xtccomment}
Interchange the \spadtype{Polynomial} and the
\spadtype{Complex} levels.
\end{xtccomment}
\begin{spadsrc}
m3 := m2 :: SquareMatrix(2,FRAC COMPLEX POLY INT) 
\end{spadsrc}
\begin{TeXOutput}
\begin{fricasmath}{7}
\begin{MATRIX}{2}\frac{4\TIMES \SYMBOL{x}-{3\TIMES \ImaginaryI }}{4}&\frac{2%
\TIMES \SUPER{\SYMBOL{y}}{2}\TIMES \SYMBOL{z}+1}{2}\\\frac{-{7\TIMES \SYMBOL{%
x}}+3\TIMES \SUPER{\SYMBOL{y}}{4}\TIMES \ImaginaryI }{7}&\frac{60-{9\TIMES %
\ImaginaryI }}{5}\end{MATRIX}%
\end{fricasmath}
\end{TeXOutput}
\formatResultType{SquareMatrix(2, Fraction(Complex(Polynomial(Integer))))}
\end{xtc}

All the entries have changed types, although in comparing the
last two results only the entry in the lower left corner looks different.
We did all the intermediate steps to show you what \Language{} can do.

\begin{xtc}
\begin{xtccomment}
In fact, we could have combined all these into one conversion.
\end{xtccomment}
\begin{spadsrc}
m :: SquareMatrix(2,FRAC COMPLEX POLY INT) 
\end{spadsrc}
\begin{TeXOutput}
\begin{fricasmath}{8}
\begin{MATRIX}{2}\frac{4\TIMES \SYMBOL{x}-{3\TIMES \ImaginaryI }}{4}&\frac{2%
\TIMES \SUPER{\SYMBOL{y}}{2}\TIMES \SYMBOL{z}+1}{2}\\\frac{-{7\TIMES \SYMBOL{%
x}}+3\TIMES \SUPER{\SYMBOL{y}}{4}\TIMES \ImaginaryI }{7}&\frac{60-{9\TIMES %
\ImaginaryI }}{5}\end{MATRIX}%
\end{fricasmath}
\end{TeXOutput}
\formatResultType{SquareMatrix(2, Fraction(Complex(Polynomial(Integer))))}
\end{xtc}

There are times when \Language{} is not be able to do the conversion
in one step.
You may need to break up the transformation into several conversions
in order to get an object of the desired type.

We cannot move either \spadtype{Fraction} or \spadtype{Complex}
above (or to the left of, depending on how you look at it)
\spadtype{SquareMatrix} because each of these levels requires that its
argument type have commutative multiplication, whereas
\spadtype{SquareMatrix} does not.\footnote{\spadtype{Fraction} requires
that its argument belong to the category \spadtype{IntegralDomain} and
\index{category}
\spadtype{Complex} requires that its argument belong to
\spadtype{CommutativeRing}. See
\spadref{ugTypesBasic}
for a brief discussion of categories.}
The \spadtype{Integer} level did not move anywhere
because it does not allow any arguments.
We also did not move the \spadtype{SquareMatrix} part anywhere, but
we could have.
\begin{xtc}
\begin{xtccomment}
Recall that \spad{m} looks like this.
\end{xtccomment}
\begin{spadsrc}
m 
\end{spadsrc}
\begin{TeXOutput}
\begin{fricasmath}{9}
\begin{MATRIX}{2}\SYMBOL{x}-{\frac{3}{4}\TIMES \ImaginaryI }&\SUPER{\SYMBOL{y%
}}{2}\TIMES \SYMBOL{z}+\frac{1}{2}\\\frac{3}{7}\TIMES \ImaginaryI \TIMES %
\SUPER{\SYMBOL{y}}{4}-{\SYMBOL{x}}&12-{\frac{9}{5}\TIMES \ImaginaryI }%
\end{MATRIX}%
\end{fricasmath}
\end{TeXOutput}
\formatResultType{SquareMatrix(2, Polynomial(Complex(Fraction(Integer))))}
\end{xtc}
\begin{xtc}
\begin{xtccomment}
If we want a polynomial with matrix coefficients rather than a matrix
with polynomial entries, we can just do the conversion.
\end{xtccomment}
\begin{spadsrc}
m :: POLY SquareMatrix(2,COMPLEX FRAC INT) 
\end{spadsrc}
\begin{TeXOutput}
\begin{fricasmath}{10}
\begin{MATRIX}{2}0&1\\0&0\end{MATRIX}\TIMES \SUPER{\SYMBOL{y}}{2}\TIMES %
\SYMBOL{z}+\begin{MATRIX}{2}0&0\\\frac{3}{7}\TIMES \ImaginaryI &0\end{MATRIX}%
\TIMES \SUPER{\SYMBOL{y}}{4}+\begin{MATRIX}{2}1&0\\-{1}&0\end{MATRIX}\TIMES %
\SYMBOL{x}+\begin{MATRIX}{2}-{\frac{3}{4}\TIMES \ImaginaryI }&\frac{1}{2}\\0&%
12-{\frac{9}{5}\TIMES \ImaginaryI }\end{MATRIX}%
\end{fricasmath}
\end{TeXOutput}
\formatResultType{Polynomial(SquareMatrix(2, Complex(Fraction(Integer))))}
\end{xtc}
\begin{xtc}
\begin{xtccomment}
We have not yet used modes for any conversions.
Modes are a great shorthand for indicating the type of the
object you want.
Instead of using the long type expression in the
last example, we could have simply said this.
\end{xtccomment}
\begin{spadsrc}
m :: POLY ? 
\end{spadsrc}
\begin{TeXOutput}
\begin{fricasmath}{11}
\begin{MATRIX}{2}0&1\\0&0\end{MATRIX}\TIMES \SUPER{\SYMBOL{y}}{2}\TIMES %
\SYMBOL{z}+\begin{MATRIX}{2}0&0\\\frac{3}{7}\TIMES \ImaginaryI &0\end{MATRIX}%
\TIMES \SUPER{\SYMBOL{y}}{4}+\begin{MATRIX}{2}1&0\\-{1}&0\end{MATRIX}\TIMES %
\SYMBOL{x}+\begin{MATRIX}{2}-{\frac{3}{4}\TIMES \ImaginaryI }&\frac{1}{2}\\0&%
12-{\frac{9}{5}\TIMES \ImaginaryI }\end{MATRIX}%
\end{fricasmath}
\end{TeXOutput}
\formatResultType{Polynomial(SquareMatrix(2, Complex(Fraction(Integer))))}
\end{xtc}
\begin{xtc}
\begin{xtccomment}
We can also indicate more structure if we want the entries
of the matrices to be fractions.
\end{xtccomment}
\begin{spadsrc}
m :: POLY SquareMatrix(2,FRAC ?) 
\end{spadsrc}
\begin{TeXOutput}
\begin{fricasmath}{12}
\begin{MATRIX}{2}0&1\\0&0\end{MATRIX}\TIMES \SUPER{\SYMBOL{y}}{2}\TIMES %
\SYMBOL{z}+\begin{MATRIX}{2}0&0\\\frac{3\TIMES \ImaginaryI }{7}&0\end{MATRIX}%
\TIMES \SUPER{\SYMBOL{y}}{4}+\begin{MATRIX}{2}1&0\\-{1}&0\end{MATRIX}\TIMES %
\SYMBOL{x}+\begin{MATRIX}{2}-{\frac{3\TIMES \ImaginaryI }{4}}&\frac{1}{2}\\0&%
\frac{60-{9\TIMES \ImaginaryI }}{5}\end{MATRIX}%
\end{fricasmath}
\end{TeXOutput}
\formatResultType{Polynomial(SquareMatrix(2, Fraction(Complex(Integer))))}
\end{xtc}

% *********************************************************************
\head{section}{Subdomains Again}{ugTypesSubdomains}
% *********************************************************************

A \spadgloss{subdomain} \spad{S} of a domain \spad{D} is a domain
consisting of
\begin{enumerate}
\item those elements of \spad{D} that satisfy some
\spadgloss{predicate} (that is, a test that returns \spad{true} or
\spad{false}) and
\item a subset of the operations of \spad{D}.
\end{enumerate}
Every domain is a subdomain of itself, trivially satisfying the
membership test: \spad{true}.

Currently, there are only two system-defined subdomains in \Language{} that receive
substantial use.
\spadtype{PositiveInteger} and
\spadtype{NonNegativeInteger} are subdomains of \spadtype{Integer}.
An element \spad{x} of \spadtype{NonNegativeInteger} is an integer
that is greater than or equal to zero, that is, satisfies
\spad{x >= 0.}
An element \spad{x} of \spadtype{PositiveInteger} is a nonnegative integer
that is, in fact, greater than zero, that is, satisfies \spad{x > 0.}
Not all operations from \spadtype{Integer} are available for these
subdomains.
For example, negation and subtraction are not provided since the subdomains
are not closed under those operations.
When you use an integer in an expression, \Language{} assigns to it the
type that is the most specific subdomain whose predicate is satisfied.
\begin{xtc}
\begin{xtccomment}
This is a positive integer.
\end{xtccomment}
\begin{spadsrc}
5
\end{spadsrc}
\begin{TeXOutput}
\begin{fricasmath}{1}
5%
\end{fricasmath}
\end{TeXOutput}
\formatResultType{PositiveInteger}
\end{xtc}
\begin{xtc}
\begin{xtccomment}
This is a nonnegative integer.
\end{xtccomment}
\begin{spadsrc}
0
\end{spadsrc}
\begin{TeXOutput}
\begin{fricasmath}{2}
0%
\end{fricasmath}
\end{TeXOutput}
\formatResultType{NonNegativeInteger}
\end{xtc}
\begin{xtc}
\begin{xtccomment}
This is neither of the above.
\end{xtccomment}
\begin{spadsrc}
-5
\end{spadsrc}
\begin{TeXOutput}
\begin{fricasmath}{3}
-{5}%
\end{fricasmath}
\end{TeXOutput}
\formatResultType{Integer}
\end{xtc}
\begin{xtc}
\begin{xtccomment}
Furthermore, unless you are assigning an integer to a declared variable
or using a conversion, any integer result has as type the most
specific subdomain.
\end{xtccomment}
\begin{spadsrc}
(-2) - (-3)
\end{spadsrc}
\begin{TeXOutput}
\begin{fricasmath}{4}
1%
\end{fricasmath}
\end{TeXOutput}
\formatResultType{PositiveInteger}
\end{xtc}
\begin{xtc}
\begin{xtccomment}
\end{xtccomment}
\begin{spadsrc}
0 :: Integer
\end{spadsrc}
\begin{TeXOutput}
\begin{fricasmath}{5}
0%
\end{fricasmath}
\end{TeXOutput}
\formatResultType{Integer}
\end{xtc}
\begin{xtc}
\begin{xtccomment}
\end{xtccomment}
\begin{spadsrc}
x : NonNegativeInteger := 5
\end{spadsrc}
\begin{TeXOutput}
\begin{fricasmath}{6}
5%
\end{fricasmath}
\end{TeXOutput}
\formatResultType{NonNegativeInteger}
\end{xtc}

When necessary, \Language{} converts an integer object into one belonging
to a less specific subdomain.
For example, in \spad{3-2}, the arguments to \spadopFrom{-}{Integer} are both
elements of \spadtype{PositiveInteger}, but this type does not provide
a subtraction operation.
Neither does \spadtype{NonNegativeInteger}, so \spad{3} and \spad{2}
are viewed as elements of \spadtype{Integer}, where their difference
can be calculated.
The result is \spad{1}, which \Language{} then automatically assigns
the type \spadtype{PositiveInteger}.

\begin{xtc}
\begin{xtccomment}
Certain operations are very sensitive to the subdomains to which their
arguments belong.
This is an element of \spadtype{PositiveInteger}.
\end{xtccomment}
\begin{spadsrc}
2 ^ 2
\end{spadsrc}
\begin{TeXOutput}
\begin{fricasmath}{7}
4%
\end{fricasmath}
\end{TeXOutput}
\formatResultType{PositiveInteger}
\end{xtc}
\begin{xtc}
\begin{xtccomment}
This is an element of \spadtype{Fraction Integer}.
\end{xtccomment}
\begin{spadsrc}
2 ^ (-2)
\end{spadsrc}
\begin{TeXOutput}
\begin{fricasmath}{8}
\frac{1}{4}%
\end{fricasmath}
\end{TeXOutput}
\formatResultType{Fraction(Integer)}
\end{xtc}
\begin{xtc}
\begin{xtccomment}
It makes sense then that this
is a list of elements of \spadtype{PositiveInteger}.
\end{xtccomment}
\begin{spadsrc}
[10^i for i in 2..5]
\end{spadsrc}
\begin{TeXOutput}
\begin{fricasmath}{9}
\BRACKET{100\COMMA 1000\COMMA 10000\COMMA 100000}%
\end{fricasmath}
\end{TeXOutput}
\formatResultType{List(PositiveInteger)}
\end{xtc}
What should the type of \spad{[10^(i-1) for i in 2..5]} be?
On one hand, \spad{i-1} is always an integer greater than zero
as \spad{i} ranges from \spad{2} to \spad{5} and so \spad{10^i}
is also always a positive integer.
On the other, \spad{i-1} is a very simple function of \spad{i}.
\Language{} does not try to analyze every such function over the
index's range of values to determine whether it is always positive
or nowhere negative.
For an arbitrary \Language{} function, this analysis is not possible.

\begin{xtc}
\begin{xtccomment}
So, to be consistent no such analysis is done and we get this.
\end{xtccomment}
\begin{spadsrc}
[10^(i-1) for i in 2..5]
\end{spadsrc}
\begin{TeXOutput}
\begin{fricasmath}{10}
\BRACKET{10\COMMA 100\COMMA 1000\COMMA 10000}%
\end{fricasmath}
\end{TeXOutput}
\formatResultType{List(Fraction(Integer))}
\end{xtc}
\begin{xtc}
\begin{xtccomment}
To get a list of elements of \spadtype{PositiveInteger} instead, you
have two choices.
You can use a conversion.
\end{xtccomment}
\begin{spadsrc}
[10^((i-1) :: PI) for i in 2..5]
\end{spadsrc}
\begin{MessageOutput}
   Compiling function G755 with type Integer -> Boolean 
\end{MessageOutput}
\begin{MessageOutput}
   Compiling function G757 with type NonNegativeInteger -> Boolean 
\end{MessageOutput}
\begin{TeXOutput}
\begin{fricasmath}{11}
\BRACKET{10\COMMA 100\COMMA 1000\COMMA 10000}%
\end{fricasmath}
\end{TeXOutput}
\formatResultType{List(PositiveInteger)}
\end{xtc}
\begin{xtc}
\begin{xtccomment}
Or you can use \spad{pretend}.
\spadkey{pretend}
\end{xtccomment}
\begin{spadsrc}
[10^((i-1) pretend PI) for i in 2..5]
\end{spadsrc}
\begin{TeXOutput}
\begin{fricasmath}{12}
\BRACKET{10\COMMA 100\COMMA 1000\COMMA 10000}%
\end{fricasmath}
\end{TeXOutput}
\formatResultType{List(PositiveInteger)}
\end{xtc}

The operation \spad{pretend} is used to defeat the \Language{}
type system.
The expression \spad{object pretend D} means ``make a new object
(without copying) of type \spad{D} from \spad{object}.''
If \spad{object} were an integer and you told \Language{}
to pretend it was a list, you would probably see a message about a
fatal error being caught and memory possibly being damaged.
Lists do not have the same internal representation as integers!

You use \spad{pretend} at your peril.
\index{peril}

\begin{xtc}
\begin{xtccomment}
Use \spad{pretend} with great care!
\Language{} trusts you that the value is of the specified type.
\end{xtccomment}
\begin{spadsrc}
(2/3) pretend Complex Integer
\end{spadsrc}
\begin{TeXOutput}
\begin{fricasmath}{13}
2+3\TIMES \ImaginaryI %
\end{fricasmath}
\end{TeXOutput}
\formatResultType{Complex(Integer)}
\end{xtc}

% *********************************************************************
\head{section}{Package Calling and Target Types}{ugTypesPkgCall}
% *********************************************************************

\Language{} works hard to figure out what you mean by an
expression without your having to qualify it with type
information.
Nevertheless, there are times when you need to help it along by
providing hints (or even orders!) to get \Language{} to do what
you want.

We saw in \spadref{ugTypesDeclare} that declarations using types
and modes control the type of the results produced.
For example, we can either produce a complex object with
polynomial real and imaginary parts or a polynomial with complex
integer coefficients, depending on the declaration.

\spadglossSee{Package calling}{package call} is how you tell
\Language{} to use a particular function from a particular part of
the library.

\begin{xtc}
\begin{xtccomment}
Use the \spadopFrom{/}{Fraction} from \spadtype{Fraction Integer}
to create a fraction of two integers.
\end{xtccomment}
\begin{spadsrc}
2/3
\end{spadsrc}
\begin{TeXOutput}
\begin{fricasmath}{1}
\frac{2}{3}%
\end{fricasmath}
\end{TeXOutput}
\formatResultType{Fraction(Integer)}
\end{xtc}
\begin{xtc}
\begin{xtccomment}
If we wanted a floating point number, we can say ``use the
\spadopFrom{/}{Float} in \spadtype{Float}.''
\end{xtccomment}
\begin{spadsrc}
(2/3)$Float
\end{spadsrc}
\begin{TeXOutput}
\begin{fricasmath}{2}
\STRING{0.66666666666666666667}%
\end{fricasmath}
\end{TeXOutput}
\formatResultType{Float}
\end{xtc}
\begin{xtc}
\begin{xtccomment}
Perhaps we actually wanted a fraction of complex integers.
\end{xtccomment}
\begin{spadsrc}
(2/3)$Fraction(Complex Integer)
\end{spadsrc}
\begin{TeXOutput}
\begin{fricasmath}{3}
\frac{2}{3}%
\end{fricasmath}
\end{TeXOutput}
\formatResultType{Fraction(Complex(Integer))}
\end{xtc}

In each case, \Language{} used the indicated operations, sometimes
first needing to convert the two integers into objects of an
appropriate type.
In these examples, \spadopFrom{/}{Fraction} is written as an
infix operator.

\beginImportant
To use package calling with an infix operator, use the
following syntax:
\begin{center}
{\tt ( \subscriptIt{arg}{1} {\it op} \subscriptIt{arg}{1} )\${\it type} }
\end{center}
\endImportant

We used, for example, \spad{(2/3)$Float}.
The expression \spad{2 + 3 + 4} is equivalent to \spad{(2+3) + 4.}
Therefore in the expression
\spad{(2 + 3 + 4)$Float} the second
\spadop{+} comes from the \spadtype{Float} domain.
Can you guess whether the first \spadop{+} comes from
\spadtype{Integer} or \spadtype{Float}?\footnote{\spadtype{Float},
because the package call causes \Language{} to convert
\spad{(2 + 3)} and \spad{4} to type \spadtype{Float}.
Before the sum is converted, it is given a target type (see below) of
\spadtype{Float} by \Language{} and then evaluated.
The target type causes the \spadop{+} from \spadtype{Float} to be used.}

\beginImportant
For an operator written before its arguments, you must use
parentheses around the arguments (even if there is only one),
and follow the closing parenthesis by a \spadSyntax{$}
and then the type.
\begin{center}
{\tt {\it fun} ( \subscriptIt{arg}{1}, \subscriptIt{arg}{1}, \ldots, \subscriptIt{arg}{N} )\${\it type}}
\end{center}
\endImportant

For example, to call the ``minimum'' function from \spadtype{DoubleFloat}
on two integers, you could write \spad{min(4,89)$DoubleFloat}.
Another use of package calling is to tell \Language{} to use a library
function rather than a function you defined.
We discuss this in \spadref{ugUserUse}.

Sometimes rather than specifying where an operation comes from, you just
want to say what type the result should be.
We say that you provide
\index{type!target}
a
\spadglossSee{target type}{target} for the expression.
\index{target type}
Instead of using a \spadSyntax{$}, use a \spadSyntax{@} to specify
the requested target type.
Otherwise, the syntax is the same.
Note that giving a target type is not the same as explicitly doing a
conversion.
The first says ``try to pick operations so that the result has
such-and-such a type.''
The second says ``compute the result and then convert to an object of
such-and-such a type.''

\begin{xtc}
\begin{xtccomment}
Sometimes it makes sense, as in this expression,
to say ``choose the operations in this expression so that
the final result is a \spadtype{Float}.''
\end{xtccomment}
\begin{spadsrc}
(2/3)@Float
\end{spadsrc}
\begin{TeXOutput}
\begin{fricasmath}{4}
\STRING{0.66666666666666666667}%
\end{fricasmath}
\end{TeXOutput}
\formatResultType{Float}
\end{xtc}

Here we used \spadSyntax{@} to say that the target type of the
left-hand side was \spadtype{Float}.
In this simple case, there was no real difference
between using \spadSyntax{$} and \spadSyntax{@}.
You can see the difference if you try the following.
\begin{xtc}
\begin{xtccomment}
This says to try to choose \spadop{+} so that the result is
a string.
\Language{} cannot do this.
\end{xtccomment}
\begin{spadsrc}
(2 + 3)@String
\end{spadsrc}
\begin{MessageOutput}
   An expression involving @ String actually evaluated to one of type 
      PositiveInteger . Perhaps you should use :: String .
\end{MessageOutput}
\end{xtc}
\begin{xtc}
\begin{xtccomment}
This says to get the \spadop{+} from \spadtype{String} and apply
it to the two integers.
\Language{} also cannot do this because there is no \spadop{+}
exported by \spadtype{String}.
\end{xtccomment}
\begin{spadsrc}
(2 + 3)$String
\end{spadsrc}
\begin{MessageOutput}
   The function + is not implemented in String .
\end{MessageOutput}
\end{xtc}
(By the way, the operation \spadfunFrom{concat}{String} or juxtaposition
is used to concatenate two strings.)
\exptypeindex{String}

When we have more than one operation in an expression, the
difference is even more evident.
The following two expressions show that \Language{} uses the
target type to create different objects.
The \spadop{+}, \spadop{*} and \spadop{^} operations are all
chosen so that an object of the correct final type is created.

\begin{xtc}
\begin{xtccomment}
This says that the operations should be chosen so
that the result is a \spadtype{Complex} object.
\end{xtccomment}
\begin{spadsrc}
((x + y * %i)^2)@(Complex Polynomial Integer)
\end{spadsrc}
\begin{TeXOutput}
\begin{fricasmath}{5}
-{\SUPER{\SYMBOL{y}}{2}}+\SUPER{\SYMBOL{x}}{2}+2\TIMES \SYMBOL{x}\TIMES %
\SYMBOL{y}\TIMES \ImaginaryI %
\end{fricasmath}
\end{TeXOutput}
\formatResultType{Complex(Polynomial(Integer))}
\end{xtc}
\begin{xtc}
\begin{xtccomment}
This says that the operations should be chosen so
that the result is a \spadtype{Polynomial} object.
\end{xtccomment}
\begin{spadsrc}
((x + y * %i)^2)@(Polynomial Complex Integer)
\end{spadsrc}
\begin{TeXOutput}
\begin{fricasmath}{6}
-{\SUPER{\SYMBOL{y}}{2}}+2\TIMES \ImaginaryI \TIMES \SYMBOL{x}\TIMES \SYMBOL{%
y}+\SUPER{\SYMBOL{x}}{2}%
\end{fricasmath}
\end{TeXOutput}
\formatResultType{Polynomial(Complex(Integer))}
\end{xtc}
\begin{xtc}
\begin{xtccomment}
What do you think might happen if we left off all
target type and package call information in this last example?
\end{xtccomment}
\begin{spadsrc}
(x + y * %i)^2 
\end{spadsrc}
\begin{TeXOutput}
\begin{fricasmath}{7}
-{\SUPER{\SYMBOL{y}}{2}}+2\TIMES \ImaginaryI \TIMES \SYMBOL{x}\TIMES \SYMBOL{%
y}+\SUPER{\SYMBOL{x}}{2}%
\end{fricasmath}
\end{TeXOutput}
\formatResultType{Polynomial(Complex(Integer))}
\end{xtc}
\begin{xtc}
\begin{xtccomment}
We can convert it to \spadtype{Complex} as an afterthought.
But this is more work than just saying making what we want in the first
place.
\end{xtccomment}
\begin{spadsrc}
% :: Complex ? 
\end{spadsrc}
\begin{TeXOutput}
\begin{fricasmath}{8}
-{\SUPER{\SYMBOL{y}}{2}}+\SUPER{\SYMBOL{x}}{2}+2\TIMES \SYMBOL{x}\TIMES %
\SYMBOL{y}\TIMES \ImaginaryI %
\end{fricasmath}
\end{TeXOutput}
\formatResultType{Complex(Polynomial(Integer))}
\end{xtc}

Finally, another use of package calling is to qualify fully an
operation that is passed as an argument to a function.

\begin{xtc}
\begin{xtccomment}
Start with a small matrix of integers.
\end{xtccomment}
\begin{spadsrc}
h := matrix [[8,6],[-4,9]] 
\end{spadsrc}
\begin{TeXOutput}
\begin{fricasmath}{9}
\begin{MATRIX}{2}8&6\\-{4}&9\end{MATRIX}%
\end{fricasmath}
\end{TeXOutput}
\formatResultType{Matrix(Integer)}
\end{xtc}
%
\begin{xtc}
\begin{xtccomment}
We want to produce a new matrix that has for entries the multiplicative
inverses of the entries of \spad{h}.
One way to do this is by calling
\spadfunFrom{map}{MatrixCategoryFunctions2} with the
\spadfunFrom{inv}{Fraction} function from \spadtype{Fraction (Integer)}.
\end{xtccomment}
\begin{spadsrc}
map(inv$Fraction(Integer),h) 
\end{spadsrc}
\begin{TeXOutput}
\begin{fricasmath}{10}
\begin{MATRIX}{2}\frac{1}{8}&\frac{1}{6}\\-{\frac{1}{4}}&\frac{1}{9}%
\end{MATRIX}%
\end{fricasmath}
\end{TeXOutput}
\formatResultType{Matrix(Fraction(Integer))}
\end{xtc}
\begin{xtc}
\begin{xtccomment}
We could have been a bit less verbose and used abbreviations.
\end{xtccomment}
\begin{spadsrc}
map(inv$FRAC(INT),h) 
\end{spadsrc}
\begin{TeXOutput}
\begin{fricasmath}{11}
\begin{MATRIX}{2}\frac{1}{8}&\frac{1}{6}\\-{\frac{1}{4}}&\frac{1}{9}%
\end{MATRIX}%
\end{fricasmath}
\end{TeXOutput}
\formatResultType{Matrix(Fraction(Integer))}
\end{xtc}
%
\begin{xtc}
\begin{xtccomment}
As it turns out, \Language{} is smart enough to know what we mean
anyway.
We can just say this.
\end{xtccomment}
\begin{spadsrc}
map(inv,h) 
\end{spadsrc}
\begin{TeXOutput}
\begin{fricasmath}{12}
\begin{MATRIX}{2}\frac{1}{8}&\frac{1}{6}\\-{\frac{1}{4}}&\frac{1}{9}%
\end{MATRIX}%
\end{fricasmath}
\end{TeXOutput}
\formatResultType{Matrix(Fraction(Integer))}
\end{xtc}

% *********************************************************************
\head{section}{Resolving Types}{ugTypesResolve}
% *********************************************************************

In this section we briefly describe an internal process by which
\index{resolve}
\Language{} determines a type to which two objects of possibly
different types can be converted.
We do this to give you further insight into how \Language{} takes
your input, analyzes it, and produces a result.

What happens when you enter \spad{x + 1} to \Language{}?
Let's look at what you get from the two terms of this expression.

\begin{xtc}
\begin{xtccomment}
This is a symbolic object whose type indicates the name.
\end{xtccomment}
\begin{spadsrc}
x
\end{spadsrc}
\begin{TeXOutput}
\begin{fricasmath}{1}
\SYMBOL{x}%
\end{fricasmath}
\end{TeXOutput}
\formatResultType{Variable(x)}
\end{xtc}
\begin{xtc}
\begin{xtccomment}
This is a positive integer.
\end{xtccomment}
\begin{spadsrc}
1
\end{spadsrc}
\begin{TeXOutput}
\begin{fricasmath}{2}
1%
\end{fricasmath}
\end{TeXOutput}
\formatResultType{PositiveInteger}
\end{xtc}

There are no operations in \spadtype{PositiveInteger} that add
positive integers to objects of type \spadtype{Variable(x)} nor
are there any in \spadtype{Variable(x)}.
Before it can add the two parts, \Language{} must come up with
a common type to which both \spad{x} and \spad{1} can be
converted.
We say that \Language{} must {\it resolve} the two types
into a common type.
In this example, the common type is \spadtype{Polynomial(Integer)}.

\begin{xtc}
\begin{xtccomment}
Once this is determined, both parts are converted into polynomials,
and the addition operation from \spadtype{Polynomial(Integer)} is used to
get the answer.
\end{xtccomment}
\begin{spadsrc}
x + 1
\end{spadsrc}
\begin{TeXOutput}
\begin{fricasmath}{3}
\SYMBOL{x}+1%
\end{fricasmath}
\end{TeXOutput}
\formatResultType{Polynomial(Integer)}
\end{xtc}
\begin{xtc}
\begin{xtccomment}
\Language{} can always resolve two types: if nothing resembling
the original types can be found, then \spadtype{Any} is be used.
\exptypeindex{Any}
This is fine and useful in some cases.
\end{xtccomment}
\begin{spadsrc}
["string",3.14159]
\end{spadsrc}
\begin{TeXOutput}
\begin{fricasmath}{4}
\BRACKET{\STRING{"string"}\COMMA \STRING{3.14159}}%
\end{fricasmath}
\end{TeXOutput}
\formatResultType{List(Any)}
\end{xtc}
\begin{xtc}
\begin{xtccomment}
In other cases objects of type \spadtype{Any} can't be used
by the operations you specified.
\end{xtccomment}
\begin{spadsrc}
"string" + 3.14159
\end{spadsrc}
\begin{MessageOutput}
   There are 14 exposed and 9 unexposed library operations named + 
      having 2 argument(s) but none was determined to be applicable. 
      Use HyperDoc Browse, or issue
                                )display op +
      to learn more about the available operations. Perhaps 
      package-calling the operation or using coercions on the arguments
      will allow you to apply the operation.
\end{MessageOutput}
\begin{MessageOutput}
   Cannot find a definition or applicable library operation named + 
      with argument type(s) 
                                   String
                                    Float
      
      Perhaps you should use "@" to indicate the required return type, 
      or "$" to specify which version of the function you need.
\end{MessageOutput}
\end{xtc}
Although this example was contrived, your expressions may need
to be qualified slightly to help \Language{} resolve the
types involved.
You may need to declare a few variables, do some package calling,
provide some target type information or do some explicit
conversions.

We suggest that you just enter the expression you want evaluated and
see what \Language{} does.
We think you will be impressed with its ability to ``do what I
mean.''
If \Language{} is still being obtuse, give it some hints.
As you work with \Language{}, you will learn where it needs a
little help to analyze quickly and perform your computations.

% *********************************************************************
\head{section}{Exposing Domains and Packages}{ugTypesExpose}
% *********************************************************************

In this section we discuss how \Language{} makes some operations
available to you while hiding others that are meant to be used by
developers or only in rare cases.
If you are a new user of \Language{}, it is likely that everything
you need is available by default and you may want
to skip over this section on first reading.

Every
\index{constructor!exposed}
domain and package in the \Language{} library
\index{constructor!hidden}
is
\index{exposed!constructor}
either
\spadglossSee{exposed}{expose} (meaning that you can use its operations without doing
anything special) or it is {\it hidden} (meaning you have to either
package call
(see \spadref{ugTypesPkgCall})
the operations it contains or explicitly expose it to use the
operations).
The initial exposure status for a constructor is set in the
file {\bf exposed.lsp} (see the {\it Installer's Note}
\index{exposed.lsp @{\bf exposed.lsp}}
for \Language{}
\index{file!exposed.lsp @{\bf exposed.lsp}}
if you need to know the location of this file).
Constructors are collected together in
\index{group!exposure}
{\it exposure groups}.
\index{exposure!group}
Categories are all in the exposure group ``categories'' and the
bulk of the basic set of packages and domains that are exposed
are in the exposure group ``basic.''
Here is an abbreviated sample of the file (without the Lisp parentheses):
\begin{verbatim}
basic
        AlgebraicNumber                          AN
        AlgebraGivenByStructuralConstants        ALGSC
        Any                                      ANY
        AnyFunctions1                            ANY1
        BinaryExpansion                          BINARY
        Boolean                                  BOOLEAN
        CardinalNumber                           CARD
        CartesianTensor                          CARTEN
        Character                                CHAR
        CharacterClass                           CCLASS
        CliffordAlgebra                          CLIF
        Color                                    COLOR
        Complex                                  COMPLEX
        ContinuedFraction                        CONTFRAC
        DecimalExpansion                         DECIMAL
        ...
\end{verbatim}
\begin{verbatim}
categories
        AbelianGroup                             ABELGRP
        AbelianMonoid                            ABELMON
        AbelianMonoidRing                        AMR
        AbelianSemiGroup                         ABELSG
        Aggregate                                AGG
        Algebra                                  ALGEBRA
        AlgebraicallyClosedField                 ACF
        AlgebraicallyClosedFunctionSpace         ACFS
        ArcHyperbolicFunctionCategory            AHYP
        ...
\end{verbatim}
For each constructor in a group, the full name and the abbreviation
is given.
There are other groups in {\bf exposed.lsp} but initially only the
constructors in exposure groups ``basic'' and ``categories'' are exposed.

As an interactive user of \Language{}, you do not need to modify
this file.
Instead, use \spadsys{)set expose} to expose, hide or query the exposure
status of an individual constructor or exposure group.
\syscmdindex{set expose}
The reason for having exposure groups is to be able to expose or hide
multiple constructors with a single command.
For example, you might group together into exposure group ``quantum'' a
number of domains and packages useful for quantum mechanical computations.
These probably should not be available to every user, but you want an easy
way to make the whole collection visible to \Language{} when it is looking
for operations to apply.

If you wanted to hide all the basic constructors available by default, you
would issue \spadsys{)set expose drop group basic}.
\syscmdindex{set expose drop group} We do not recommend that you do this.
If, however, you discover that you have hidden all the basic constructors,
you should issue \spadsys{)set expose add group basic} to restore your
default environment.
\syscmdindex{set expose add group}

It is more likely that you would want to expose or hide individual
constructors.
In \spadref{ugUserTriangle} we use several operations from
\spadtype{OutputForm}, a domain usually hidden.
To avoid package calling every operation from \spadtype{OutputForm}, we
expose the domain and let \Language{} conclude that those operations should
be used.
Use \spadsys{)set expose add constructor} and \spadsys{)set expose drop
constructor} to expose and hide a constructor, respectively.
\syscmdindex{set expose drop constructor}
You should use the constructor name, not the abbreviation.
The \spadsys{)set expose} command guides you through these options.
\syscmdindex{set expose add constructor}

If you expose a previously hidden constructor, \Language{}
exhibits new behavior (that was your intention) though you might not
expect the results that you get.
\spadtype{OutputForm} is, in fact, one of the worst offenders in this
regard.
\exptypeindex{OutputForm}
This domain is meant to be used by other domains for creating a
structure that \Language{} knows how to display.
It has functions like \spadopFrom{+}{OutputForm} that form output
representations rather than do mathematical calculations.
Because of the order in which \Language{} looks at constructors
when it is deciding what operation to apply, \spadtype{OutputForm}
might be used instead of what you expect.
\begin{xtc}
\begin{xtccomment}
This is a polynomial.
\end{xtccomment}
\begin{spadsrc}
x + x
\end{spadsrc}
\begin{TeXOutput}
\begin{fricasmath}{1}
2\TIMES \SYMBOL{x}%
\end{fricasmath}
\end{TeXOutput}
\formatResultType{Polynomial(Integer)}
\end{xtc}
\begin{xtc}
\begin{xtccomment}
Expose \spadtype{OutputForm}.
\end{xtccomment}
\begin{spadsrc}
)set expose add constructor OutputForm 
\end{spadsrc}
\begin{SysCmdOutput}
   OutputForm is now explicitly exposed in frame initial 
\end{SysCmdOutput}
\end{xtc}
\begin{xtc}
\begin{xtccomment}
This is what we get when \spadtype{OutputForm} is automatically
available.
\end{xtccomment}
\begin{spadsrc}
x + x 
\end{spadsrc}
\begin{TeXOutput}
\begin{fricasmath}{2}
\SYMBOL{x}+\SYMBOL{x}%
\end{fricasmath}
\end{TeXOutput}
\formatResultType{OutputForm}
\end{xtc}
\begin{xtc}
\begin{xtccomment}
Hide \spadtype{OutputForm} so we don't run into problems
with any later examples!
\end{xtccomment}
\begin{spadsrc}
)set expose drop constructor OutputForm 
\end{spadsrc}
\begin{SysCmdOutput}
   OutputForm is now explicitly hidden in frame initial 
\end{SysCmdOutput}
\end{xtc}

Finally, exposure is done on a frame-by-frame basis.
A \spadgloss{frame} (see \spadref{ugSysCmdframe})
\index{frame!exposure and}
is one of possibly several
logical \Language{} workspaces within a physical one, each having
its own environment (for example, variables and function definitions).
If you have several \Language{} workspace windows on your screen, they
are all different frames, automatically created for you by \HyperName{}.
Frames can be manually created, made active and destroyed by the
\spadsys{)frame} system command.
\syscmdindex{frame}
They do not share exposure information, so you need to use
\spadsys{)set expose} in each one to add or drop constructors from view.

% *********************************************************************
\head{section}{Commands for Snooping}{ugAvailSnoop}
% *********************************************************************

To conclude this chapter, we introduce you to some system commands
that you can use for getting more information about domains,
packages, categories, and operations.
The most powerful \Language{} facility for getting information about
constructors and operations is the \Browse{} component of \HyperName{}.
This is discussed in \chapref{ugBrowse}.

Use the \spadsys{)what} system command to see lists of system objects
whose name contain a particular substring (uppercase or lowercase is
not significant).
\syscmdindex{what}

\begin{xtc}
\begin{xtccomment}
Issue this to see a list of all operations with
``{\tt complex}'' in their names.
\syscmdindex{what operation}
\end{xtccomment}
\begin{spadsrc}
)what operation complex
\end{spadsrc}
\begin{SysCmdOutput}

Operations whose names satisfy the above pattern(s):

chainComplex                      coChainComplex                    
complex                           complex?                          
complexEigenvalues                complexEigenvectors               
complexElementary                 complexExpand                     
complexForm                       complexIntegrate                  
complexLimit                      complexNormalize                  
complexNumeric                    complexNumericIfCan               
complexRoots                      complexSolve                      
complexZeros                      createLowComplexityNormalBasis    
createLowComplexityTable          cubicalComplex                    
deltaComplex                      doubleComplex?                    
drawComplex                       drawComplexVectorField            
fortranComplex                    fortranDoubleComplex              
pmComplexintegrate                simplicialComplex                 
simplicialComplexIfCan            testComplexEquals                 
testComplexEqualsAux              xftestComplexEquals               
xftestComplexEqualsAux            
   
      To get more information about an operation such as complexForm , 
      issue the command )display op complexForm 
\end{SysCmdOutput}
\end{xtc}
\begin{xtc}
\begin{xtccomment}
If you want to see all domains with ``{\tt matrix}'' in their names, issue
this.
\syscmdindex{what domain}
\end{xtccomment}
\begin{spadsrc}
)what domain matrix
\end{spadsrc}
\begin{SysCmdOutput}
--------------------------------- Domains ---------------------------------

Domains with names matching patterns:
     matrix 

 CDFMAT   ComplexDoubleFloatMatrix     DFMAT    DoubleFloatMatrix
 DHMATRIX DenavitHartenbergMatrix      DPMM     DirectProductMatrixModule
 IMATRIX  IndexedMatrix                LSQM     LieSquareMatrix
 M3D      ThreeDimensionalMatrix       MATCAT-  MatrixCategory&
 MATRIX   Matrix                       RMATCAT- RectangularMatrixCategory&
 RMATRIX  RectangularMatrix            SEM      SparseEchelonMatrix
 SMATCAT- SquareMatrixCategory&        SQMATRIX SquareMatrix
 U16MAT   U16Matrix                    U32MAT   U32Matrix
 U8MAT    U8Matrix
\end{SysCmdOutput}
\end{xtc}
\begin{xtc}
\begin{xtccomment}
Similarly, if you wish to see all packages whose names contain
``{\tt gauss}'', enter this.
\syscmdindex{what packages}
\end{xtccomment}
\begin{spadsrc}
)what package gauss
\end{spadsrc}
\begin{SysCmdOutput}
-------------------------------- Packages ---------------------------------

Packages with names matching patterns:
     gauss 

 FFFG     FractionFreeFastGaussian
 FFFGF    FractionFreeFastGaussianFractions
 GAUSSFAC GaussianFactorizationPackage  UGAUSS   UnitGaussianElimination
\end{SysCmdOutput}
\end{xtc}
\begin{xtc}
\begin{xtccomment}
This command shows all
the operations that \spadtype{Any} provides.
Wherever \spadSyntax{$} appears, it means ``\spadtype{Any}''.
\syscmdindex{show}
\end{xtccomment}
\begin{spadsrc}
)show Any
\end{spadsrc}
\begin{SysCmdOutput}
 Any is a domain constructor
 Abbreviation for Any is ANY 
 This constructor is exposed in this frame.
------------------------------- Operations --------------------------------
 ?=? : (%,%) -> Boolean                any : (SExpression,None) -> %
 coerce : % -> OutputForm              dom : % -> SExpression
 domainOf : % -> OutputForm            hash : % -> SingleInteger
 latex : % -> String                   obj : % -> None
 objectOf : % -> OutputForm            ?~=? : (%,%) -> Boolean
 hashUpdate! : (HashState,%) -> HashState
 showTypeInOutput : Boolean -> String

\end{SysCmdOutput}
\end{xtc}
\begin{xtc}
\begin{xtccomment}
This displays all operations with the name \spadfun{complex}.
\syscmdindex{display operation}
\end{xtccomment}
\begin{spadsrc}
)display operation complex
\end{spadsrc}
\begin{SysCmdOutput}

There is one exposed function called complex :
   [1] (D1,D1) -> D from D if D has COMPCAT(D1) and D1 has COMRING
\end{SysCmdOutput}
\end{xtc}
Let's analyze this output.
\begin{xtc}
\begin{xtccomment}
First we find out what some of the abbreviations mean.
\end{xtccomment}
\begin{spadsrc}
)abbreviation query COMPCAT
\end{spadsrc}
\begin{SysCmdOutput}
   COMPCAT abbreviates category ComplexCategory 
\end{SysCmdOutput}
\end{xtc}
\begin{xtc}
\begin{xtccomment}
\end{xtccomment}
\begin{spadsrc}
)abbreviation query COMRING
\end{spadsrc}
\begin{SysCmdOutput}
   COMRING abbreviates category CommutativeRing 
\end{SysCmdOutput}
\end{xtc}

So if \spad{D1} is a commutative ring (such as the integers or
floats) and \spad{D} belongs to
\spadtype{ComplexCategory D1},
then there is an operation called \spadfun{complex} that
takes two elements of \spad{D1} and creates an element of
\spad{D}.
The primary example of a constructor implementing domains
belonging to \spadtype{ComplexCategory} is \spadtype{Complex}.
See \xmpref{Complex} for more information on that and see
\spadref{ugUserDeclare} for more information on function types.

% !! DO NOT MODIFY THIS FILE BY HAND !! Created by spool2tex.awk.

% Copyright (c) 1991-2002, The Numerical ALgorithms Group Ltd.
% All rights reserved.
%
% Redistribution and use in source and binary forms, with or without
% modification, are permitted provided that the following conditions are
% met:
%
%     - Redistributions of source code must retain the above copyright
%       notice, this list of conditions and the following disclaimer.
%
%     - Redistributions in binary form must reproduce the above copyright
%       notice, this list of conditions and the following disclaimer in
%       the documentation and/or other materials provided with the
%       distribution.
%
%     - Neither the name of The Numerical ALgorithms Group Ltd. nor the
%       names of its contributors may be used to endorse or promote products
%       derived from this software without specific prior written permission.
%
% THIS SOFTWARE IS PROVIDED BY THE COPYRIGHT HOLDERS AND CONTRIBUTORS "AS
% IS" AND ANY EXPRESS OR IMPLIED WARRANTIES, INCLUDING, BUT NOT LIMITED
% TO, THE IMPLIED WARRANTIES OF MERCHANTABILITY AND FITNESS FOR A
% PARTICULAR PURPOSE ARE DISCLAIMED. IN NO EVENT SHALL THE COPYRIGHT OWNER
% OR CONTRIBUTORS BE LIABLE FOR ANY DIRECT, INDIRECT, INCIDENTAL, SPECIAL,
% EXEMPLARY, OR CONSEQUENTIAL DAMAGES (INCLUDING, BUT NOT LIMITED TO,
% PROCUREMENT OF SUBSTITUTE GOODS OR SERVICES-- LOSS OF USE, DATA, OR
% PROFITS-- OR BUSINESS INTERRUPTION) HOWEVER CAUSED AND ON ANY THEORY OF
% LIABILITY, WHETHER IN CONTRACT, STRICT LIABILITY, OR TORT (INCLUDING
% NEGLIGENCE OR OTHERWISE) ARISING IN ANY WAY OUT OF THE USE OF THIS
% SOFTWARE, EVEN IF ADVISED OF THE POSSIBILITY OF SUCH DAMAGE.


% *********************************************************************
\head{chapter}{Using \HyperName{}}{ugHyper}
% *********************************************************************

\begin{figure}[htbp]
\begin{picture}(324,260)%(-54,0)
\special{psfile=h-root.ps}
\end{picture}
\caption{The \HyperName{} root window page.}
\end{figure}

\HyperName{} is the gateway to \Language{}.
\index{HyperDoc @{\protect\HyperName{}}}
It's both an on-line tutorial and an on-line reference manual.
It also enables you to use \Language{} simply by using the mouse and
filling in templates.
\HyperName{} is available to you if you are running \Language{} under the
X Window System.

Pages usually have active areas, marked in
{\bf this font} (bold face).
As you move the mouse pointer to an active area, the pointer changes from
a filled dot to an open circle.
The active areas are usually linked to other pages.
When you click on an active area, you move to the linked page.



% *********************************************************************
\head{section}{Headings}{ugHyperHeadings}
% *********************************************************************
%
Most pages have a standard set of buttons at the top of the page.
This is what they mean:

\begin{description}
\item[\StdHelpButton{}] Click on this to get help.
The button only appears if there is specific help for the page you are
viewing.
You can get {\it general} help for \HyperName{} by clicking the help
button on the home page.

\item[\UpButton{}] Click here to go back one page.
By clicking on this button repeatedly, you can go back several pages and
then take off in a new direction.

\item[\ReturnButton{}] Go back to the home page, that is,
the page on which you started.
Use \HyperName{} to explore, to make forays into new topics.
Don't worry about how to get back.
\HyperName{} remembers where you came from.
Just click on this button to return.

\item[\StdExitButton{}] From the root window (the one that is displayed when
you start the system) this button leaves the \HyperName{} program, and it
must be restarted if you want to use it again.
From any other \HyperName{} window, it just makes that one window go away.
You {\it must} use this button to get rid of a window.
If you use the window manager ``Close'' button, then all of \HyperName{}
goes away.
\end{description}
%
The buttons are not displayed if they are not applicable to the page
you are viewing.
For example, there is no \ReturnButton{} button on the top-level menu.

% *********************************************************************
\head{section}{Key Definitions}{ugHyperKeys}
% *********************************************************************

The following keyboard definitions are in effect throughout
\HyperName{}.
See \spadref{ugHyperScroll} and
\spadref{ugHyperInput}
for some contextual key definitions.
%
\begin{description}
\item[F1] Display the main help page.
\item[F3] Same as \StdExitButton{}, makes the window go away if you are not at the top-level window or quits the \HyperName{} facility if you are at the top-level.
\item[F5] Rereads the \HyperName{} database, if necessary (for system developers).
\item[F9] Displays this information about key definitions.
\item[F12] Same as {\bf F3}.
\item[Up Arrow] Scroll up one line.
\item[Down Arrow] Scroll down one line.
\item[Page Up] Scroll up one page.
\item[Page Down] Scroll down one page.
\end{description}

% *********************************************************************
\head{section}{Scroll Bars}{ugHyperScroll}
% *********************************************************************
%

Whenever there is too much text to fit on a page, a {\it scroll
\index{scroll bar}
bar} automatically appears along the right side.

With a scroll bar, your page becomes an aperture, that is, a
window into a larger amount of text than can be displayed at one
time.
The scroll bar lets you move up and down in the text to see
different parts.
It also shows where the aperture is relative to the whole text.
The aperture is indicated by a strip on the scroll bar.

Move the cursor with the mouse to the ``down-arrow'' at the bottom
of the scroll bar and click.
See that the aperture moves down one line.
Do it several times.
Each time you click, the aperture moves down one line.
Move the mouse to the ``up-arrow'' at the top of the scroll bar and
click.
The aperture moves up one line each time you click.

Next move the mouse to any position along the middle of the scroll bar and
click.
\HyperName{} attempts to move the top of the aperture to this point in
the text.

You cannot make the aperture go off the bottom edge.
When the aperture is about half the size of text, the lowest you can move
the aperture is halfway down.

To move up or down one screen at a time, use the
\fbox{\bf PageUp} and
\fbox{\bf PageDown} keys on your keyboard.
They move the visible part of the region up and down one page each time you
press them.

If the \HyperName{} page does not contain an input area
(see \spadref{ugHyperInput}), you can also use the
\fbox{\bf Home} and
\fbox{$\uparrow$} and
\fbox{$\downarrow$} arrow keys to navigate.
When you press the \fbox{\bf Home} key,
the screen is positioned at the very top of the page.
Use the \fbox{$\uparrow$} and
\fbox{$\downarrow$} arrow keys to move the screen up
and down one line at a time, respectively.

% *********************************************************************
\head{section}{Input Areas}{ugHyperInput}
% *********************************************************************
%
Input areas are boxes where you can put data.

To enter characters, first
move your mouse cursor to somewhere within the \HyperName{} page.
Characters that you type are inserted in front of the underscore.
This means that when you type characters at your keyboard, they
go into this first input area.

The input area grows to accommodate as many characters as you type.
Use the \fbox{\bf Backspace}
key to erase characters to the left.
To modify what you type, use the right-arrow \fbox{$\rightarrow$}
and left-arrow keys \fbox{$\leftarrow$} and the
keys \fbox{\bf Insert},
\fbox{\bf Delete},
\fbox{\bf Home} and
\fbox{\bf End}.
These keys are found immediately on the right of the standard IBM keyboard.

If you press the
\fbox{\bf Home}
key, the cursor moves to the beginning of the line and if you press the
\fbox{\bf End}
key, the cursor moves to the end of the line.
Pressing
\fbox{\bf Ctrl}--\fbox{\bf End}
deletes all the text from the cursor to the end of the line.

A page may have more than one input area.
Only one input area has an underscore cursor.
When you first see apage, the top-most input area contains the
cursor.
To type information into another input area,
use the \fbox{\bf Enter} or
\fbox{\bf Tab} key to move
from one input area to another.
To move in the reverse order, use
\fbox{\bf Shift}--\fbox{\bf Tab}.

You can also move from one input area to another using your mouse.
Notice that each input area is active. Click on one of the areas.
As you can see, the underscore cursor moves to that window.

% *********************************************************************
\head{section}{Radio Buttons and Toggles}{ugHyperButtons}
% *********************************************************************
%
Some pages have {\it radio buttons} and {\it toggles}.
Radio buttons are a group of buttons like those on car radios: you can
select only one at a time.
Once you have selected a button, it appears to be inverted and
contains a checkmark.
To change the selection, move the cursor with the mouse to a different radio
button and click.


A toggle is an independent button that displays some on/off
state.
When ``on'', the button appears to be inverted and
contains a checkmark.
When ``off'', the button is raised.
%
Unlike radio buttons, you can set a group of them any way you like.
To change toggle the selection, move the cursor with the mouse
to the button and click.

% *********************************************************************
\head{section}{Search Strings}{ugHyperSearch}
% *********************************************************************
%
A {\it search string} is used for searching some database.
To learn about search strings, we suggest that
you bring up the \HyperName{} glossary.
To do this from the top-level page of \HyperName{}:
\begin{enumerate}
\item Click on \windowlink{Reference}{TopReferencePage},
bringing up the \Language{} Reference page.
\item Click on \windowlink{Glossary}{GlossaryPage}, bringing up the glossary.
\end{enumerate}

The glossary has an input area at its bottom.
We review the various kinds of search strings
you can enter to search the glossary.

The simplest search string is a word, for example, {\tt operation}.
A word only matches an entry having exactly that spelling.
Enter the word {\tt operation} into the input area above then click on
{\bf Search}.
As you can see, {\tt operation} matches only one entry, namely with {\tt
operation} itself.

Normally matching is insensitive to whether the alphabetic characters of your
search string are in uppercase or lowercase.
Thus {\tt operation} and {\tt OperAtion} both have the same effect.
%If you prefer that matching be case-sensitive, issue the command
%\spadsys{set HHyperName mixedCase} command to the interpreter.

You will very often want to use the wildcard \spadSyntax{*} in your search
string so as to match multiple entries in the list.
The search key \spadSyntax{*}  matches every entry in the list.
You can also use \spadSyntax{*} anywhere within a search string to match an
arbitrary substring.
Try {\tt cat*} for example:
enter {\tt cat*} into the input area and click on {\bf Search}.
This matches several entries.

You use any number of wildcards in a search string as long as they are
not adjacent.
Try search strings such as {\tt *dom*}.
As you see, this search string  matches {\tt domain}, {\tt domain
constructor}, {\tt subdomain}, and so on.

% *********************************************************************
\head{subsection}{Logical Searches}{ugLogicalSearches}
% *********************************************************************

For more complicated searches, you can use
\spadSyntax{and}, \spadSyntax{or}, and \spadSyntax{not}
with basic search strings;
write logical expressions using these three operators just as
in the \Language{} language.
For example, {\tt domain or package}  matches the two
entries {\tt domain} and {\tt package}.
Similarly, {\tt dom* and *con*} matches {\tt domain constructor}
and others.
Also {\tt not *a*} matches every entry that does not contain
the letter {\tt a} somewhere.

Use parentheses for grouping.
For example, {\tt dom* and (not *con*)}
matches {\tt domain} but not {\tt domain constructor}.

There is no limit to how complex your logical expression can be.
For example,
\begin{center}
{\tt a* or b* or c* or d* or e* and (not *a*)}
\end{center}
is a valid expression.

% *********************************************************************
\head{section}{Example Pages}{ugHyperExample}
% *********************************************************************
%
Many pages have \Language{} example commands.
%
%
Each command has an active ``button'' along the left margin.
When you click on this button, the output for the command is
``pasted-in.''
Click again on the button and you see that the pasted-in output
disappears.

Maybe you would like to run an example?
To do so, just click on any part of its text!
When you do, the example line is copied into a new interactive
\Language{} buffer for this \HyperName{} page.

Sometimes one example line cannot be run before you run an earlier one.
Don't worry---\HyperName{} automatically runs all the necessary
lines in the right order!

The new interactive \Language{} buffer disappears when you leave
\HyperName{}.
If you want to get rid of it beforehand,
use the {\bf Cancel} button of the X Window manager
or issue the \Language{} system command \spadsys{)close}.
\syscmdindex{close}

% *********************************************************************
\head{section}{X Window Resources for \HyperName{}}{ugHyperResources}
% *********************************************************************
%
You can control the appearance of \HyperName{} while running under Version 11
\index{HyperDoc @{\protect\HyperName{}}!X Window System defaults}
of the X Window System by placing the following resources
\index{X Window System}
in the file {\bf .Xdefaults} in your home directory.
\index{file!.Xdefaults @{\bf .Xdefaults}}
In what follows, {\it font} is any valid X11 font name
\index{font}
(for example, {\tt Rom14}) and {\it color} is any valid X11 color
\index{color}
specification (for example, {\tt NavyBlue}).
For more information about fonts and colors, refer to the
X Window documentation for your system.
\begin{description}
\item[{\tt FriCAS.hyperdoc.RmFont:} {\it font}] \ \newline
This is the standard text font.  \xdefault{Rom14}
\item[{\tt FriCAS.hyperdoc.RmColor:} {\it color}] \ \newline
This is the standard text color.  \xdefault{black}
\item[{\tt FriCAS.hyperdoc.ActiveFont:} {\it font}] \ \newline
This is the font used for \HyperName{} link buttons.  \xdefault{Bld14}
\item[{\tt FriCAS.hyperdoc.ActiveColor:} {\it color}] \ \newline
This is the color used for \HyperName{} link buttons.  \xdefault{black}
\item[{\tt FriCAS.hyperdoc.FriCASFont:} {\it font}] \ \newline
This is the font used for active \Language{} commands.
\xdefault{Bld14}
\item[{\tt FriCAS.hyperdoc.FriCASColor:} {\it color}] \ \newline
This is the color used for active \Language{} commands.
\xdefault{black}
\item[{\tt FriCAS.hyperdoc.BoldFont:} {\it font}] \ \newline
This is the font used for bold face.  \xdefault{Bld14}
\item[{\tt FriCAS.hyperdoc.BoldColor:} {\it color}] \ \newline
This is the color used for bold face.  \xdefault{black}
\item[{\tt FriCAS.hyperdoc.TtFont:} {\it font}] \ \newline
This is the font used for \Language{} output in \HyperName{}.
This font must be fixed-width.  \xdefault{Rom14}
\item[{\tt FriCAS.hyperdoc.TtColor:} {\it color}] \ \newline
This is the color used for \Language{} output in \HyperName{}.
\xdefault{black}
\item[{\tt FriCAS.hyperdoc.EmphasizeFont:} {\it font}] \ \newline
This is the font used for italics.  \xdefault{Itl14}
\item[{\tt FriCAS.hyperdoc.EmphasizeColor:} {\it color}] \ \newline
This is the color used for italics.  \xdefault{black}
\item[{\tt FriCAS.hyperdoc.InputBackground:} {\it color}] \ \newline
This is the color used as the background for input areas.
\xdefault{black}
\item[{\tt FriCAS.hyperdoc.InputForeground:} {\it color}] \ \newline
This is the color used as the foreground for input areas.
\xdefault{white}
\item[{\tt FriCAS.hyperdoc.BorderColor:} {\it color}] \ \newline
This is the color used for drawing border lines.
\xdefault{black}
\item[{\tt FriCAS.hyperdoc.Background:} {\it color}] \ \newline
This is the color used for the background of all windows.
\xdefault{white}

Note: In the past resource names used word Axiom instead of FriCAS.

\end{description}
\begin{SysCmdOutput}
\end{SysCmdOutput}

% !! DO NOT MODIFY THIS FILE BY HAND !! Created by spool2tex.awk.

% Copyright (c) 1991-2002, The Numerical ALgorithms Group Ltd.
% All rights reserved.
%
% Redistribution and use in source and binary forms, with or without
% modification, are permitted provided that the following conditions are
% met:
%
%     - Redistributions of source code must retain the above copyright
%       notice, this list of conditions and the following disclaimer.
%
%     - Redistributions in binary form must reproduce the above copyright
%       notice, this list of conditions and the following disclaimer in
%       the documentation and/or other materials provided with the
%       distribution.
%
%     - Neither the name of The Numerical ALgorithms Group Ltd. nor the
%       names of its contributors may be used to endorse or promote products
%       derived from this software without specific prior written permission.
%
% THIS SOFTWARE IS PROVIDED BY THE COPYRIGHT HOLDERS AND CONTRIBUTORS "AS
% IS" AND ANY EXPRESS OR IMPLIED WARRANTIES, INCLUDING, BUT NOT LIMITED
% TO, THE IMPLIED WARRANTIES OF MERCHANTABILITY AND FITNESS FOR A
% PARTICULAR PURPOSE ARE DISCLAIMED. IN NO EVENT SHALL THE COPYRIGHT OWNER
% OR CONTRIBUTORS BE LIABLE FOR ANY DIRECT, INDIRECT, INCIDENTAL, SPECIAL,
% EXEMPLARY, OR CONSEQUENTIAL DAMAGES (INCLUDING, BUT NOT LIMITED TO,
% PROCUREMENT OF SUBSTITUTE GOODS OR SERVICES-- LOSS OF USE, DATA, OR
% PROFITS-- OR BUSINESS INTERRUPTION) HOWEVER CAUSED AND ON ANY THEORY OF
% LIABILITY, WHETHER IN CONTRACT, STRICT LIABILITY, OR TORT (INCLUDING
% NEGLIGENCE OR OTHERWISE) ARISING IN ANY WAY OUT OF THE USE OF THIS
% SOFTWARE, EVEN IF ADVISED OF THE POSSIBILITY OF SUCH DAMAGE.

% *********************************************************************
\head{chapter}{Input Files and Output Styles}{ugInOut}
% *********************************************************************

In this chapter we discuss how to collect \Language{} statements
and commands into files and then read the contents into the
workspace.
We also show how to display the results of your computations in
several different styles including \TeX{}, FORTRAN and
monospace two-dimensional format.\footnote{\TeX{} is a
trademark of the American Mathematical Society.}

The printed version of this book uses the \Language{}
\TeX{} output formatter.
When we demonstrate a particular output style, we will need to
turn \TeX{} formatting off and the output style on so
that the correct output is shown in the text.

% *********************************************************************
\head{section}{Input Files}{ugInOutIn}
% *********************************************************************
%
In this section we explain what an {\it input file} is and
\index{file!input}
why you would want to know about it.
We discuss where \Language{} looks for input files and how you can
direct it to look elsewhere.
We also show how to read the contents of an input file into the
\spadgloss{workspace} and how to use the \spadgloss{history}
facility to generate an input file from the statements you have
entered directly into the workspace.

An {\it input} file contains \Language{} expressions and system
commands.
Anything that you can enter directly to \Language{} can be put
into an input file.
This is how you save input functions and expressions that you wish
to read into \Language{} more than one time.

To read an input file into \Language{}, use the \spadsys{)read}
system command.
\syscmdindex{read}
For example, you can read a file in a particular directory by issuing
\begin{verbatim}
)read /spad/src/input/matrix.input
\end{verbatim}
The ``{\bf .input}'' is optional; this also works:
\begin{verbatim}
)read /spad/src/input/matrix
\end{verbatim}
What happens if you just enter
\spadsys{)read matrix.input} or even \spadsys{)read matrix}?
\Language{} looks in your current working directory for input files
that are not qualified by a directory name.
Typically, this directory is the directory from which you invoked
\Language{}.
To change the current working directory, use the \spadsys{)cd} system command.
The command \spadsys{)cd} by itself shows the current
working
\index{directory!default for searching}
directory.
\syscmdindex{cd}
To change it to
\index{file!input!where found}
the \spadsys{src/input} subdirectory for user ``babar'',
issue
\begin{verbatim}
)cd /u/babar/src/input
\end{verbatim}
\Language{} looks first in this directory for an input file.
If it is not found, it looks in the system's directories, assuming
you meant some input file that was provided with \Language{}.

\beginImportant
If you have the \Language{} history facility turned on (which it is
by default), you can save all the lines you have entered into the
workspace by entering
\begin{verbatim}
)history )write
\end{verbatim}
\syscmdindex{history )write}

\Language{} tells you what input file to edit to see your
statements.
The file is in your home directory or in the directory you
specified with \spadsys{)cd}.
\syscmdindex{cd}
\endImportant

In \spadref{ugLangBlocks}
we discuss using indentation in input files to group statements
into {\it blocks.}

% *********************************************************************
\head{section}{The fricas.input File}{ugInOutSpadprof}
% *********************************************************************

When \Language{} starts up, it tries to read the input file
{\bf fricas.input} from your home
\index{start-up profile file}
directory.
\index{file!start-up profile}
It there is no {\bf fricas.input} in your home directory, it reads the copy
located in its own {\bf src/input} directory.
\index{file!fricas.input @{\bf fricas.input}}
The file usually contains
system commands to personalize your \Language{} environment.
In the remainder of this section we mention a few things
that users frequently place in their
{\bf fricas.input} files.

In order to have FORTRAN output always produced from your
computations, place the system command
\spadsys{)set output fortran on}
in {\bf fricas.input}.
\syscmdindex{quit}
If you do want to be prompted for confirmation when you issue
the \spadsys{)quit} system command, place
\spadsys{)set quit protected}
in {\bf fricas.input}.
\syscmdindex{set quit protected}
If you then decide that you do not want to be prompted, issue
\spadsys{)set quit unprotected}.
\syscmdindex{set quit unprotected}
This is the default setting
% so that new users do not leave \Language{}
% inadvertently.
\footnote{The
system command \spadsys{)pquit} always prompts you for
confirmation.}

To see the other system variables you can set, issue \spadsys{)set}
or use the \HyperName{} {\bf Settings} facility to view and change
\Language{} system variables.

% *********************************************************************
\head{section}{Common Features of Using Output Formats}{ugInOutOut}
% *********************************************************************

In this section we discuss how to start and stop the display
\index{output formats!common features}
of the different output formats and how to send the output to the
screen or to a file.
\index{file!sending output to}
To fix ideas, we use FORTRAN output format for most of the
examples.

You can use the \spadsys{)set output}
system
\index{output formats!starting}
command to
\index{output formats!stopping}
toggle or redirect the different kinds of output.
\syscmdindex{set output}
The name of the kind of output follows ``output'' in the command.
The names are

\begin{tabular}{@{}ll}
\bf fortran & for FORTRAN output. \\
\bf algebra & for monospace two-dimensional mathematical output. \\
\bf tex     & for \TeX{} output. \\
\bf script  & for IBM Script Formula Format output.
\end{tabular}

For example, issue \spadsys{)set output fortran on} to turn on
FORTRAN format and
issue \spadsys{)set output fortran off} to turn it off.
By default, {\tt algebra} is {\tt on} and all others are {\tt off}.
\syscmdindex{set output fortran}
When output is started, it is sent to the screen.
To send the output to a file, give the file name without
\index{output formats!sending to file}
directory or extension.
\Language{} appends a file extension depending on the kind of
output being produced.
\begin{xtc}
\begin{xtccomment}
Issue this to redirect FORTRAN output to, for example, the file
{\bf linalg.sfort}.
\end{xtccomment}
\begin{spadsrc}
)set output fortran linalg
\end{spadsrc}
\begin{SysCmdOutput}
   FORTRAN output will be written to file 
      /home/kfp/Devel/rh_fricas/src/doc/linalg.sfort .
\end{SysCmdOutput}
\end{xtc}
\begin{nullXtc}
\begin{xtccomment}
You must {\it also} turn on the creation of FORTRAN output.
The above just says where it goes if it is created.
\end{xtccomment}
\begin{spadsrc}
)set output fortran on
\end{spadsrc}
\end{nullXtc}
In what directory is this output placed?
It goes into the directory from which you started \Language{},
or if you have used the \spadsys{)cd} system command, the one
that you specified with \spadsys{)cd}.
\syscmdindex{cd}
You should use \spadsys{)cd} before you send the output to the file.

\begin{nullXtc}
\begin{xtccomment}
You can always direct output back to the screen by issuing this.
\index{output formats!sending to screen}
\end{xtccomment}
\begin{spadsrc}
)set output fortran console
\end{spadsrc}
\end{nullXtc}
\begin{nullXtc}
\begin{xtccomment}
Let's make sure FORTRAN formatting is off so that nothing we
do from now on produces FORTRAN output.
\end{xtccomment}
\begin{spadsrc}
)set output fortran off
\end{spadsrc}
\end{nullXtc}
\begin{nullXtc}
\begin{xtccomment}
We also delete the demonstrated output file we created.
\end{xtccomment}
\begin{spadsrc}
)system rm linalg.sfort
\end{spadsrc}
\end{nullXtc}

You can abbreviate the words ``\spad{on},'' ``\spad{off}'' and
``\spad{console}'' to the minimal number
of characters needed to distinguish them.
Because of this, you cannot send output to files called
{\bf on.sfort, off.sfort, of.sfort,
console.sfort, consol.sfort} and so on.

The width of the output on the page is set by
\index{output formats!line length}
\spadsys{)set output length}
for all formats except FORTRAN.
\syscmdindex{set output length}
Use \spadsys{)set fortran fortlength} to
change the FORTRAN line length from its default value of \spad{72}.

% *********************************************************************
\head{section}{Monospace Two-Dimensional Mathematical Format}{ugInOutAlgebra}
% *********************************************************************

This is the default output format for \Language{}.
\syscmdindex{set output algebra}
It is usually on when you start the system.
\index{output formats!monospace 2D}
\index{monospace 2D output format}

\begin{nullXtc}
\begin{xtccomment}
If it is not, issue this.
\end{xtccomment}
\begin{spadsrc}
)set output algebra on 
\end{spadsrc}
\end{nullXtc}
\begin{nullXtc}
\begin{xtccomment}
Since the printed version of this book
(as opposed to the \HyperName{} version)
shows output produced by the
\TeX{} output formatter,
let us temporarily turn off
\TeX{} output.
\end{xtccomment}
\begin{spadsrc}
)set output tex off 
\end{spadsrc}
\end{nullXtc}
\begin{nullXtc}
\begin{xtccomment}
Here is an example of what it looks like.
\end{xtccomment}
\begin{spadsrc}
matrix [[i*x^i + j*%i*y^j for i in 1..2] for j in 3..4] 
\end{spadsrc}
\end{nullXtc}
\begin{verbatim}
+     3           3     2+
|3%i y  + x  3%i y  + 2x |
|                        |
|     4           4     2|
+4%i y  + x  4%i y  + 2x +
\end{verbatim}

\begin{nullXtc}
\begin{xtccomment}
Issue this to turn off this kind of formatting.
\end{xtccomment}
\begin{spadsrc}
)set output algebra off
\end{spadsrc}
\end{nullXtc}
\begin{nullXtc}
\begin{xtccomment}
Turn \TeX{} output on again.
\end{xtccomment}
\begin{spadsrc}
)set output tex on
\end{spadsrc}
\end{nullXtc}

The characters used for the matrix brackets above are rather ugly.
You get this character set when you issue
\index{character set}
\spadsys{)set output characters plain}.
\syscmdindex{set output characters}
This character set should be used when your machine or your
version of \Language{}
does not support Unicode character set.
If your machine and your version of \Language{} support Unicode, issue
\spadsys{)set output characters default}
to get better looking output.

% *********************************************************************
\head{section}{TeX Format}{ugInOutTeX}
% *********************************************************************

\Language{} can produce \TeX{} output for your
\index{output formats!TeX @{\TeX}}
expressions.
\index{TeX output format @{\TeX} output format}
The output is produced using macros from the
\LaTeX{} document preparation system by
Leslie Lamport.\footnote{See Leslie Lamport, {\it LaTeX: A Document
Preparation System,} Reading, Massachusetts: Addison-Wesley
Publishing Company, Inc., 1986.}
The printed version of this book was produced using this formatter.

\begin{noOutputXtc}
\begin{xtccomment}
To turn on \TeX{} output formatting, issue this.
\syscmdindex{set output tex}
\end{xtccomment}
\begin{spadsrc}
)set output tex on 
\end{spadsrc}
\end{noOutputXtc}
Here is an example of its output.
\begin{verbatim}
matrix [[i*x^i + j*\%i*y^j for i in 1..2] for j in 3..4]

\begin{fricasmath}{1}
\begin{MATRIX}{2}3\TIMES \ImaginaryI \TIMES \SUPER{\SYMBOL{y}}{3}+\SYMBOL{x}&%
3\TIMES \ImaginaryI \TIMES \SUPER{\SYMBOL{y}}{3}+2\TIMES \SUPER{\SYMBOL{x}}{2%
}\\4\TIMES \ImaginaryI \TIMES \SUPER{\SYMBOL{y}}{4}+\SYMBOL{x}&4\TIMES %
\ImaginaryI \TIMES \SUPER{\SYMBOL{y}}{4}+2\TIMES \SUPER{\SYMBOL{x}}{2}%
\end{MATRIX}%
\end{fricasmath}
\end{verbatim}
With the definition of the fricasmath environment as defined in
fricasmath.sty this formats as
\begin{fricasmath}{1}
\begin{MATRIX}{2}
3\TIMES \ImaginaryI \TIMES \SUPER{\SYMBOL{y}}{3}+\SYMBOL{x}&%
3\TIMES \ImaginaryI \TIMES \SUPER{\SYMBOL{y}}{3}+2\TIMES \SUPER{\SYMBOL{x}}{2%
}\\4\TIMES \ImaginaryI \TIMES \SUPER{\SYMBOL{y}}{4}+\SYMBOL{x}&4\TIMES %
\ImaginaryI \TIMES \SUPER{\SYMBOL{y}}{4}+2\TIMES \SUPER{\SYMBOL{x}}{2}%
\end{MATRIX}%
\end{fricasmath}
To turn \TeX{} output formatting off, issue
\spadsys{)set output tex off}.
The \LaTeX{} macros in the output generated by \Language{}
are generic. See the source file of \spadtype{TexFormat} for
appropriate definitions of these commands.

% *********************************************************************
\head{section}{Math ML Format}{ugInOutMathML}
% *********************************************************************

\Language{} can
\index{output formats!Math ML Format}
produce Math ML format output for your
\index{Math ML Format}
expressions.

\begin{noOutputXtc}
\begin{xtccomment}
To turn Math ML Format on, issue this.
\syscmdindex{set output mathml}
\end{xtccomment}
\begin{spadsrc}
)set output mathml on
\end{spadsrc}
\end{noOutputXtc}
Here is an example of its output.
\begin{verbatim}
x+sqrt(2)

<math xmlns="http://www.w3.org/1998/Math/MathML" mathsize="big" display="block">
<mrow><mi>x</mi><mo>+</mo><msqrt><mrow><mn>2</mn></mrow></msqrt></mrow>
</math>
\end{verbatim}
\begin{noOutputXtc}
\begin{xtccomment}
To turn Math ML Format output formatting off, issue this.
\end{xtccomment}
\begin{spadsrc}
)set output mathml off
\end{spadsrc}
\end{noOutputXtc}
%
% *********************************************************************
\head{section}{Texmacs Format}{ugInOutTexmacs}
% *********************************************************************

\Language{} can
\index{output formats!Texmacs Format}
produce Texmacs Scheme format output for your
\index{Texmacs Format}
expressions.  This is mostly useful for interfacing with Texmacs.

\begin{noOutputXtc}
\begin{xtccomment}
To turn Texmacs Format on, issue this.
\syscmdindex{set output texmacs}
\end{xtccomment}
\begin{spadsrc}
)set output texmacs on
\end{spadsrc}
\end{noOutputXtc}
Here is an example of its output.
\begin{verbatim}
x+sqrt(2)

scheme: (with "mode" "math"
(concat (concat  "x" ) "+" (sqrt (concat  "2" )))
)
\end{verbatim}
\begin{noOutputXtc}
\begin{xtccomment}
To turn Texmacs Format output formatting off, issue this.
\end{xtccomment}
\begin{spadsrc}
)set output texmacs off
\end{spadsrc}
\end{noOutputXtc}
%
% *********************************************************************
\head{section}{IBM Script Formula Format}{ugInOutScript}
% *********************************************************************

\Language{} can
\index{output formats!IBM Script Formula Format}
produce IBM Script Formula Format output for your
\index{IBM Script Formula Format}
expressions.

\begin{noOutputXtc}
\begin{xtccomment}
To turn IBM Script Formula Format on, issue this.
\syscmdindex{set output script}
\end{xtccomment}
\begin{spadsrc}
)set output script on
\end{spadsrc}
\end{noOutputXtc}
Here is an example of its output.
\begin{verbatim}
matrix [[i*x^i + j*%i*y^j for i in 1..2] for j in 3..4]

.eq set blank @
:df.
<left lb <<<<3 @@ %i @@ <y sup 3>>+x> here <<3 @@ %i @@
<y sup 3>>+<2 @@ <x sup 2>>>> habove <<<4 @@ %i @@
<y sup 4>>+x> here <<4 @@ %i @@ <y sup 4>>+<2 @@
<x up 2>>>>> right rb>
:edf.
\end{verbatim}
\begin{noOutputXtc}
\begin{xtccomment}
To turn IBM Script Formula Format output formatting off, issue this.
\end{xtccomment}
\begin{spadsrc}
)set output script off
\end{spadsrc}
\end{noOutputXtc}

% *********************************************************************
\head{section}{FORTRAN Format}{ugInOutFortran}
% *********************************************************************

In addition to turning FORTRAN output on and off and stating where the
\index{output formats!FORTRAN}
output should be placed, there are many options that control the
\index{FORTRAN output format}
appearance of the generated code.
In this section we describe some of the basic options.
Issue \spadsys{)set fortran} to see a full list with their current
settings.

The output FORTRAN expression usually begins in column 7.
If the expression needs more than one line, the ampersand character
\spadSyntax{&} is used in column 6.
Since some versions of FORTRAN have restrictions on the number of lines
per statement, \Language{} breaks long expressions into segments with
a maximum of 1320 characters (20 lines of 66 characters) per segment.
\syscmdindex{set fortran}
If you want to change this, say, to 660 characters,
issue the system command
\syscmdindex{set fortran explength}
\spadsys{)set fortran explength 660}.
\index{FORTRAN output format!breaking into multiple statements}
You can turn off the line breaking by issuing
\spadsys{)set fortran segment off}.
\syscmdindex{set fortran segment}
Various code optimization levels are available.
%
\begin{noOutputXtc}
\begin{xtccomment}
FORTRAN output is produced after you issue this.
\syscmdindex{set output fortran}
\end{xtccomment}
\begin{spadsrc}
)set output fortran on 
\end{spadsrc}
\end{noOutputXtc}
\begin{noOutputXtc}
\begin{xtccomment}
For the initial examples, we set the optimization level to 0, which is the
lowest level.
\syscmdindex{set fortran optlevel}
\end{xtccomment}
\begin{spadsrc}
)set fortran optlevel 0 
\end{spadsrc}
\end{noOutputXtc}
\begin{noOutputXtc}
\begin{xtccomment}
The output is usually in columns 7 through 72, although fewer columns
are used in the following examples so that the output
\index{FORTRAN output format!line length}
fits nicely on the page.
\end{xtccomment}
\begin{spadsrc}
)set fortran fortlength 60
\end{spadsrc}
\end{noOutputXtc}
\begin{xtc}
\begin{xtccomment}
By default, the output goes to the screen and is displayed
before the standard \Language{} two-dimensional output.
In this example, an
assignment to the variable \spad{R1} was generated because this is
the result of step 1.
\end{xtccomment}
\begin{spadsrc}
(x+y)^3 
\end{spadsrc}
\begin{TeXOutput}
\begin{fricasmath}{1}
\SUPER{\SYMBOL{y}}{3}+3\TIMES \SYMBOL{x}\TIMES \SUPER{\SYMBOL{y}}{2}+3\TIMES %
\SUPER{\SYMBOL{x}}{2}\TIMES \SYMBOL{y}+\SUPER{\SYMBOL{x}}{3}%
\end{fricasmath}
\end{TeXOutput}
\formatResultType{Polynomial(Integer)}
\end{xtc}
\begin{xtc}
\begin{xtccomment}
Here is an example that illustrates the line breaking.
\end{xtccomment}
\begin{spadsrc}
(x+y+z)^3 
\end{spadsrc}
\begin{TeXOutput}
\begin{fricasmath}{2}
\SUPER{\SYMBOL{z}}{3}+\PAREN{3\TIMES \SYMBOL{y}+3\TIMES \SYMBOL{x}}\TIMES %
\SUPER{\SYMBOL{z}}{2}+\PAREN{3\TIMES \SUPER{\SYMBOL{y}}{2}+6\TIMES \SYMBOL{x}%
\TIMES \SYMBOL{y}+3\TIMES \SUPER{\SYMBOL{x}}{2}}\TIMES \SYMBOL{z}+\SUPER{%
\SYMBOL{y}}{3}+3\TIMES \SYMBOL{x}\TIMES \SUPER{\SYMBOL{y}}{2}+3\TIMES \SUPER{%
\SYMBOL{x}}{2}\TIMES \SYMBOL{y}+\SUPER{\SYMBOL{x}}{3}%
\end{fricasmath}
\end{TeXOutput}
\formatResultType{Polynomial(Integer)}
\end{xtc}

Note in the above examples that integers are generally converted to
\index{FORTRAN output format!integers vs. floats}
floating point numbers, except in exponents.
This is the default behavior but can be turned off by issuing
\spadsys{)set fortran ints2floats off}.
\syscmdindex{set fortran ints2floats}
The rules governing when the conversion is done are:
\begin{enumerate}
\item If an integer is an exponent, convert it to a floating point
number if it is greater than 32767 in absolute value, otherwise leave it
as an integer.
\item Convert all other integers in an expression to floating
point numbers.
\end{enumerate}
These rules only govern integers in expressions.
Numbers generated by \Language{} for \spad{DIMENSION} statements are also
integers.

To set the type of generated FORTRAN data,
\index{FORTRAN output format!data types}
use one of the following:
\begin{verbatim}
)set fortran defaulttype REAL
)set fortran defaulttype INTEGER
)set fortran defaulttype COMPLEX
)set fortran defaulttype LOGICAL
)set fortran defaulttype CHARACTER
\end{verbatim}

\begin{xtc}
\begin{xtccomment}
When temporaries are created, they are given a default type of
{\tt REAL.}
Also, the {\tt REAL} versions of functions are used by default.
\end{xtccomment}
\begin{spadsrc}
sin(x) 
\end{spadsrc}
\begin{TeXOutput}
\begin{fricasmath}{3}
\sin{\SYMBOL{x}}%
\end{fricasmath}
\end{TeXOutput}
\formatResultType{Expression(Integer)}
\end{xtc}
\begin{noOutputXtc}
\begin{xtccomment}
At optimization level 1, \Language{} removes common subexpressions.
\index{FORTRAN output format!optimization level}
\syscmdindex{set fortran optlevel}
\end{xtccomment}
\begin{spadsrc}
)set fortran optlevel 1 
\end{spadsrc}
\end{noOutputXtc}
\begin{xtc}
\begin{xtccomment}
\end{xtccomment}
\begin{spadsrc}
(x+y+z)^3 
\end{spadsrc}
\begin{TeXOutput}
\begin{fricasmath}{4}
\SUPER{\SYMBOL{z}}{3}+\PAREN{3\TIMES \SYMBOL{y}+3\TIMES \SYMBOL{x}}\TIMES %
\SUPER{\SYMBOL{z}}{2}+\PAREN{3\TIMES \SUPER{\SYMBOL{y}}{2}+6\TIMES \SYMBOL{x}%
\TIMES \SYMBOL{y}+3\TIMES \SUPER{\SYMBOL{x}}{2}}\TIMES \SYMBOL{z}+\SUPER{%
\SYMBOL{y}}{3}+3\TIMES \SYMBOL{x}\TIMES \SUPER{\SYMBOL{y}}{2}+3\TIMES \SUPER{%
\SYMBOL{x}}{2}\TIMES \SYMBOL{y}+\SUPER{\SYMBOL{x}}{3}%
\end{fricasmath}
\end{TeXOutput}
\formatResultType{Polynomial(Integer)}
\end{xtc}
\begin{noOutputXtc}
\begin{xtccomment}
This changes the precision to {\tt DOUBLE}.
\syscmdindex{set fortran precision double}
Substitute \spad{single} for \spad{double}
\index{FORTRAN output format!precision}
to return to single precision.
\syscmdindex{set fortran precision single}
\end{xtccomment}
\begin{spadsrc}
)set fortran precision double 
\end{spadsrc}
\end{noOutputXtc}
\begin{xtc}
\begin{xtccomment}
Complex constants display the precision.
\end{xtccomment}
\begin{spadsrc}
2.3 + 5.6*%i  
\end{spadsrc}
\begin{TeXOutput}
\begin{fricasmath}{5}
\STRING{2.3}+\STRING{5.6}\TIMES \ImaginaryI %
\end{fricasmath}
\end{TeXOutput}
\formatResultType{Complex(Float)}
\end{xtc}
\begin{xtc}
\begin{xtccomment}
The function names that \Language{} generates depend on the chosen
precision.
\end{xtccomment}
\begin{spadsrc}
sin %e  
\end{spadsrc}
\begin{TeXOutput}
\begin{fricasmath}{6}
\sin{\EulerE }%
\end{fricasmath}
\end{TeXOutput}
\formatResultType{Expression(Integer)}
\end{xtc}
\begin{noOutputXtc}
\begin{xtccomment}
Reset the precision to \spad{single} and look at these two
examples again.
\end{xtccomment}
\begin{spadsrc}
)set fortran precision single 
\end{spadsrc}
\end{noOutputXtc}
\begin{xtc}
\begin{xtccomment}
\end{xtccomment}
\begin{spadsrc}
2.3 + 5.6*%i  
\end{spadsrc}
\begin{TeXOutput}
\begin{fricasmath}{7}
\STRING{2.3}+\STRING{5.6}\TIMES \ImaginaryI %
\end{fricasmath}
\end{TeXOutput}
\formatResultType{Complex(Float)}
\end{xtc}
\begin{xtc}
\begin{xtccomment}
\end{xtccomment}
\begin{spadsrc}
sin %e  
\end{spadsrc}
\begin{TeXOutput}
\begin{fricasmath}{8}
\sin{\EulerE }%
\end{fricasmath}
\end{TeXOutput}
\formatResultType{Expression(Integer)}
\end{xtc}
\begin{xtc}
\begin{xtccomment}
Expressions that look like lists, streams, sets or matrices cause
array code to be generated.
\end{xtccomment}
\begin{spadsrc}
[x+1,y+1,z+1] 
\end{spadsrc}
\begin{TeXOutput}
\begin{fricasmath}{9}
\BRACKET{\SYMBOL{x}+1\COMMA \SYMBOL{y}+1\COMMA \SYMBOL{z}+1}%
\end{fricasmath}
\end{TeXOutput}
\formatResultType{List(Polynomial(Integer))}
\end{xtc}
\begin{xtc}
\begin{xtccomment}
A temporary variable is generated to be the name of the array.
\index{FORTRAN output format!arrays}
This may have to be changed in your particular application.
\end{xtccomment}
\begin{spadsrc}
set[2,3,4,3,5] 
\end{spadsrc}
\begin{TeXOutput}
\begin{fricasmath}{10}
\BRACE{2\COMMA 3\COMMA 4\COMMA 5}%
\end{fricasmath}
\end{TeXOutput}
\formatResultType{Set(PositiveInteger)}
\end{xtc}
\begin{xtc}
\begin{xtccomment}
By default, the starting index for generated FORTRAN arrays is \spad{0}.
\end{xtccomment}
\begin{spadsrc}
matrix [[2.3,9.7],[0.0,18.778]] 
\end{spadsrc}
\begin{TeXOutput}
\begin{fricasmath}{11}
\begin{MATRIX}{2}\STRING{2.3}&\STRING{9.7}\\\STRING{0.0}&\STRING{18.778}%
\end{MATRIX}%
\end{fricasmath}
\end{TeXOutput}
\formatResultType{Matrix(Float)}
\end{xtc}
\begin{noOutputXtc}
\begin{xtccomment}
To change the starting index for generated FORTRAN arrays to be \spad{1},
\syscmdindex{set fortran startindex}
issue this.
This value can only be \spad{0} or \spad{1}.
\end{xtccomment}
\begin{spadsrc}
)set fortran startindex 1 
\end{spadsrc}
\end{noOutputXtc}
\begin{xtc}
\begin{xtccomment}
Look at the code generated for the matrix again.
\end{xtccomment}
\begin{spadsrc}
matrix [[2.3,9.7],[0.0,18.778]] 
\end{spadsrc}
\begin{TeXOutput}
\begin{fricasmath}{12}
\begin{MATRIX}{2}\STRING{2.3}&\STRING{9.7}\\\STRING{0.0}&\STRING{18.778}%
\end{MATRIX}%
\end{fricasmath}
\end{TeXOutput}
\formatResultType{Matrix(Float)}
\end{xtc}

% *********************************************************************
\head{section}{General Fortran-generation utilities in \Language{}}{ugGeneralFortran}
% *********************************************************************

This section describes more advanced facilities which are available to users
who wish to generate Fortran code from within \Language{}.  There are
facilities to manipulate templates, store type information, and generate
code fragments or complete programs.

% ----------------------------------------------------------------------
\head{subsection}{Template Manipulation}{ugGenForTemplate}
% ----------------------------------------------------------------------

A template is a skeletal program which is ``fleshed out'' with data when
it is processed.  It is a sequence of {\em active} and {\em passive} parts:
active parts are sequences of \Language{} commands which are processed as if they
had been typed into the interpreter; passive parts are simply echoed
verbatim on the Fortran output stream.

Suppose, for example, that we have the following template, stored in
the file ``test.tem'':
\begin{verbatim}
-- A simple template
beginVerbatim
      DOUBLE PRECISION FUNCTION F(X)
      DOUBLE PRECISION X
endVerbatim
outputAsFortran("F",f)
beginVerbatim
      RETURN
      END
endVerbatim
\end{verbatim}
The passive parts lie between the two
tokens {\tt beginVerbatim} and \linebreak {\tt endVerbatim}.  There
are two active statements: one which is simply an \Language{} (
{\tt --})
comment, and one which produces an assignment to the current value
of {\tt f}.  We could use it as follows:
\begin{verbatim}
(4) ->f := 4.0/(1+X^2)

           4
   (4)   ------
          2
         X  + 1

(5) ->processTemplate "test.tem"
      DOUBLE PRECISION FUNCTION F(X)
      DOUBLE PRECISION X
      F=4.0D0/(X*X+1.0D0)
      RETURN
      END

   (5)  "CONSOLE"
\end{verbatim}

(A more reliable method of specifying the filename will be introduced
below.)  Note that the Fortran assignment {\tt F=4.0D0/(X*X+1.0D0)}
automatically converted 4.0 and 1 into DOUBLE PRECISION numbers; in
general, the \Language{} Fortran generation facility will convert
anything which should be a floating point object into either
a Fortran REAL or DOUBLE PRECISION object.
\begin{noOutputXtc}
\begin{xtccomment}
Which alternative is used is determined by the command
\end{xtccomment}
\begin{spadsrc}
)set fortran precision
\end{spadsrc}
\begin{SysCmdOutput}
-------------------------- The precision Option ---------------------------

 Description: precision of generated FORTRAN objects

 The precision option may be followed by any one of the following:

 -> single 
    double

 The current setting is indicated within the list.

\end{SysCmdOutput}
\end{noOutputXtc}

It is sometimes useful to end a template before the file itself ends (e.g. to
allow the template to be tested incrementally or so that a piece of text
describing how the template works can be included).  It is of course possible
to ``comment-out'' the remainder of the file.  Alternatively, the single token
{\tt endInput} as part of an active portion of the template will cause
processing to be ended prematurely at that point.

The \spadfun{processTemplate} command comes in two flavours.  In the first case,
illustrated above, it takes one argument of domain \spadtype{FileName},
the name of the template to be processed, and writes its output on the
current Fortran output stream.  In general, a filename can be generated
from {\em directory}, {\em name} and {\em extension} components, using
the operation \spadfun{filename}, as in
\begin{verbatim}
processTemplate filename("","test","tem")
\end{verbatim}
There is an alternative version of \spadfun{processTemplate}, which
takes two arguments (both of domain \spadtype{FileName}).  In this case the
first argument is the name of the template to be processed, and the
second is the file in which to write the results.  Both versions return
the location of the generated Fortran code as their result
({\tt "CONSOLE"} in the above example).

It is sometimes useful to be able to mix active and passive parts of a
line or statement.  For example you might want to generate a Fortran
Comment describing your data set.  For this kind of application we
provide three functions as follows:
\begin{center}
\begin{tabular}{p{1.8in}p{2.6in}}
\spadfun{fortranLiteral} & writes a string on the Fortran output stream \\
 & \\
\spadfun{fortranCarriageReturn} & writes a carriage return on the Fortran output stream \\
& \\
\spadfun{fortranLiteralLine} & writes a string followed by a return
on the Fortran output stream \\
\end{tabular}
\end{center}
\begin{xtc}
\begin{xtccomment}
So we could create our comment as follows:
\end{xtccomment}
\begin{spadsrc}
m := matrix [[1,2,3],[4,5,6]]
\end{spadsrc}
\begin{TeXOutput}
\begin{fricasmath}{1}
\begin{MATRIX}{3}1&2&3\\4&5&6\end{MATRIX}%
\end{fricasmath}
\end{TeXOutput}
\formatResultType{Matrix(Integer)}
\end{xtc}
\begin{xtc}
\begin{xtccomment}
\end{xtccomment}
\begin{spadsrc}
fortranLiteralLine(concat ["C\ \ \ \ \ \ The\ Matrix\ has\ ", nrows(m)::String, "\ rows\ and\ ", ncols(m)::String, "\ columns"])$FortranTemplate 
\end{spadsrc}
\end{xtc}
\begin{xtc}
\begin{xtccomment}
or, alternatively:
\end{xtccomment}
\begin{spadsrc}
fortranLiteral("C\ \ \ \ \ \ The\ Matrix\ has\ ")$FortranTemplate
\end{spadsrc}
\end{xtc}
\begin{xtc}
\begin{xtccomment}
\end{xtccomment}
\begin{spadsrc}
fortranLiteral(nrows(m)::String)$FortranTemplate
\end{spadsrc}
\end{xtc}
\begin{xtc}
\begin{xtccomment}
\end{xtccomment}
\begin{spadsrc}
fortranLiteral("\ rows\ and\ ")$FortranTemplate
\end{spadsrc}
\end{xtc}
\begin{xtc}
\begin{xtccomment}
\end{xtccomment}
\begin{spadsrc}
fortranLiteral(ncols(m)::String)$FortranTemplate
\end{spadsrc}
\end{xtc}
\begin{xtc}
\begin{xtccomment}
\end{xtccomment}
\begin{spadsrc}
fortranLiteral("\ columns")$FortranTemplate
\end{spadsrc}
\end{xtc}
\begin{noOutputXtc}
\begin{xtccomment}
\end{xtccomment}
\begin{spadsrc}
fortranCarriageReturn()$FortranTemplate
\end{spadsrc}
\end{noOutputXtc}

We should stress that these functions, together with the \spadfun{outputAsFortran}
function are the {\em only} sure ways
of getting output to appear on the Fortran output stream.  Attempts to use
\Language{} commands such as \spadfun{output} or \spadfunX{writeline} may appear to give
the required result when displayed on the console, but will give the wrong
result when Fortran and algebraic output are sent to differing locations.  On
the other hand, these functions can be used to send helpful messages to the
user, without interfering with the generated Fortran.

% ----------------------------------------------------------------------
\head{subsection}{Manipulating the Fortran Output Stream}{ugGneForManipulating}
% ----------------------------------------------------------------------
\exptypeindex{FortranOutputStackPackage}

Sometimes it is useful to manipulate the Fortran output stream in a program,
possibly without being aware of its current value.  The main use of this is
for gathering type declarations (see ``Fortran Types'' below) but it can be useful
in other contexts as well.  Thus we provide a set of commands to manipulate
a stack of (open) output streams.  Only one stream can be written to at
any given time.  The stack is never empty---its initial value is the
console or the current value of the Fortran output stream, and can be
determined using
\begin{xtc}
\begin{xtccomment}
\end{xtccomment}
\begin{spadsrc}
topFortranOutputStack()$FortranOutputStackPackage
\end{spadsrc}
\begin{TeXOutput}
\begin{fricasmath}{1}
\STRING{"/home/kfp/Devel/rh\_fricas/src/doc/linalg.sfort"}%
\end{fricasmath}
\end{TeXOutput}
\formatResultType{String}
\end{xtc}
(see below).
The commands available to manipulate the stack are:
\begin{center}
\begin{tabular}{ll}
\spadfun{clearFortranOutputStack} & resets the stack to the console \\
 & \\
\spadfun{pushFortranOutputStack} & pushes a \spadtype{FileName} onto the stack \\
 & \\
\spadfun{popFortranOutputStack} & pops the stack \\
 & \\
\spadfun{showFortranOutputStack} & returns the current stack \\
 & \\
\spadfun{topFortranOutputStack} & returns the top element of the stack \\
\end{tabular}
\end{center}
These commands are all part of \spadtype{FortranOutputStackPackage}.

% ----------------------------------------------------------------------
\head{subsection}{Fortran Types}{ugGenForTypes}
% ----------------------------------------------------------------------

When generating code it is important to keep track of the Fortran types of
the objects which we are generating.  This is useful for a number of reasons,
not least to ensure that we are actually generating legal Fortran code.  The
current type system is built up in several layers, and we shall describe each
in turn.

% ----------------------------------------------------------------------
\head{subsection}{FortranScalarType}{ugGenForScalarType}
% ----------------------------------------------------------------------
\exptypeindex{FortranScalarType}

This domain represents the simple Fortran datatypes: REAL, DOUBLE PRECISION,
COMPLEX, LOGICAL, INTEGER, and CHARACTER.
It is possible to \spadfun{coerce} a \spadtype{String} or \spadtype{Symbol}
into the domain, test whether two objects are equal, and also apply
the predicate functions \spadfunFrom{real?}{FortranScalarType} etc.

% ----------------------------------------------------------------------
\head{subsection}{FortranType}{ugGenForType}
% ----------------------------------------------------------------------
\exptypeindex{FortranType}

This domain represents ``full'' types: i.e., datatype plus array dimensions
(where appropriate) plus whether or not the parameter is an external
subprogram.  It is possible to \spadfun{coerce} an object of
\spadtype{FortranScalarType} into the domain or \spadfun{construct} one
from an element of \spadtype{FortranScalarType}, a list of
\spadtype{Polynomial Integer}s (which can of course be simple integers or
symbols) representing its dimensions, and
a \spadtype{Boolean} declaring whether it is external or not.  The list
of dimensions must be empty if the \spadtype{Boolean} is {\tt true}.
The functions \spadfun{scalarTypeOf}, \spadfun{dimensionsOf} and
\spadfun{external?} return the appropriate
parts, and it is possible to get the various basic Fortran Types via
functions like \spadfun{fortranReal}.
\begin{xtc}
\begin{xtccomment}
For example:
\end{xtccomment}
\begin{spadsrc}
type:=construct(real,[i,10],false)$FortranType
\end{spadsrc}
\end{xtc}
\begin{xtc}
\begin{xtccomment}
or
\end{xtccomment}
\begin{spadsrc}
type:=[real,[i,10],false]$FortranType
\end{spadsrc}
\end{xtc}
\begin{xtc}
\begin{xtccomment}
\end{xtccomment}
\begin{spadsrc}
scalarTypeOf type
\end{spadsrc}
\begin{TeXOutput}
\begin{fricasmath}{1}
\SYMBOL{REAL}%
\end{fricasmath}
\end{TeXOutput}
\formatResultType{Union(fst: FortranScalarType, ...)}
\end{xtc}
\begin{xtc}
\begin{xtccomment}
\end{xtccomment}
\begin{spadsrc}
dimensionsOf type
\end{spadsrc}
\begin{TeXOutput}
\begin{fricasmath}{2}
\BRACKET{\SYMBOL{i}\COMMA 10}%
\end{fricasmath}
\end{TeXOutput}
\formatResultType{List(Polynomial(Integer))}
\end{xtc}
\begin{xtc}
\begin{xtccomment}
\end{xtccomment}
\begin{spadsrc}
external?  type
\end{spadsrc}
\begin{TeXOutput}
\begin{fricasmath}{3}
\STRING{false}%
\end{fricasmath}
\end{TeXOutput}
\formatResultType{Boolean}
\end{xtc}
\begin{xtc}
\begin{xtccomment}
\end{xtccomment}
\begin{spadsrc}
fortranLogical()$FortranType
\end{spadsrc}
\begin{TeXOutput}
\begin{fricasmath}{4}
\SYMBOL{LOGICAL}%
\end{fricasmath}
\end{TeXOutput}
\formatResultType{FortranType}
\end{xtc}
\begin{xtc}
\begin{xtccomment}
\end{xtccomment}
\begin{spadsrc}
construct(integer,[],true)$FortranType
\end{spadsrc}
\end{xtc}

% ----------------------------------------------------------------------
\head{subsection}{SymbolTable}{ugGenForSymbolTable}
% ----------------------------------------------------------------------
\exptypeindex{SymbolTable}

This domain creates and manipulates a symbol table for generated Fortran code.
This is used by \spadtype{FortranProgram} to represent the types of objects in
a subprogram.  The commands available are:\newline
\begin{center}
\begin{tabular}{ll}
\spadfun{empty} & creates a new \spadtype{SymbolTable} \\
 & \\
\spadfunX{declare} & creates a new entry in a table \\
 & \\
\spadfun{fortranTypeOf} & returns the type of an object in a table \\
 & \\
\spadfun{parametersOf} & returns a list of all the symbols in the table \\
 & \\
\spadfun{typeList} & returns a list of all objects of a given type \\
 & \\
\spadfun{typeLists} & returns a list of lists of all objects sorted by type \\
 & \\
\spadfun{externalList} & returns a list of all {\tt EXTERNAL} objects \\
 & \\
\spadfun{printTypes} & produces Fortran type declarations from a table\\
\end{tabular}
\end{center}
\begin{xtc}
\begin{xtccomment}
\end{xtccomment}
\begin{spadsrc}
symbols := empty()$SymbolTable
\end{spadsrc}
\begin{TeXOutput}
\begin{fricasmath}{1}
\STRING{table}\PAREN{}%
\end{fricasmath}
\end{TeXOutput}
\formatResultType{SymbolTable}
\end{xtc}
\begin{xtc}
\begin{xtccomment}
\end{xtccomment}
\begin{spadsrc}
declare!(X, fortranReal()$FortranType, symbols)
\end{spadsrc}
\begin{TeXOutput}
\begin{fricasmath}{2}
\SYMBOL{REAL}%
\end{fricasmath}
\end{TeXOutput}
\formatResultType{FortranType}
\end{xtc}
\begin{xtc}
\begin{xtccomment}
\end{xtccomment}
\begin{spadsrc}
declare!(M,construct(real,[i,j],false)$FortranType,symbols)
\end{spadsrc}
\end{xtc}
\begin{xtc}
\begin{xtccomment}
\end{xtccomment}
\begin{spadsrc}
declare!([i,j], fortranInteger()$FortranType, symbols)
\end{spadsrc}
\begin{TeXOutput}
\begin{fricasmath}{3}
\SYMBOL{INTEGER}%
\end{fricasmath}
\end{TeXOutput}
\formatResultType{FortranType}
\end{xtc}
\begin{xtc}
\begin{xtccomment}
\end{xtccomment}
\begin{spadsrc}
symbols
\end{spadsrc}
\end{xtc}
\begin{xtc}
\begin{xtccomment}
\end{xtccomment}
\begin{spadsrc}
fortranTypeOf(i,symbols)
\end{spadsrc}
\begin{TeXOutput}
\begin{fricasmath}{4}
\SYMBOL{INTEGER}%
\end{fricasmath}
\end{TeXOutput}
\formatResultType{FortranType}
\end{xtc}
\begin{xtc}
\begin{xtccomment}
\end{xtccomment}
\begin{spadsrc}
typeList(real,symbols)
\end{spadsrc}
\begin{TeXOutput}
\begin{fricasmath}{5}
\BRACKET{\SYMBOL{X}\COMMA \BRACKET{\SYMBOL{M}\COMMA \SYMBOL{i}\COMMA \SYMBOL{%
j}}}%
\end{fricasmath}
\end{TeXOutput}
\formatResultType{List(Union(name: Symbol, bounds: List(Union(S: Symbol, P: Polynomial(Integer)))))}
\end{xtc}
\begin{xtc}
\begin{xtccomment}
\end{xtccomment}
\begin{spadsrc}
printTypes symbols
\end{spadsrc}
\end{xtc}

% ----------------------------------------------------------------------
\head{subsection}{TheSymbolTable}{ugGenForTheSymbolTable}
% ----------------------------------------------------------------------
\exptypeindex{TheSymbolTable}

This domain creates and manipulates one global symbol table to be used, for
example, during template processing. It is
also used when
linking to external Fortran routines. The
information stored for each subprogram (and the main program segment, where
relevant) is:
\begin{itemize}
\item its name;
\item its return type;
\item its argument list;
\item and its argument types.
\end{itemize}
Initially, any information provided is deemed to be for the main program
segment.
\begin{xtc}
\begin{xtccomment}
Issuing the following command indicates that from now on all information
refers to the subprogram \spad{F}.
\end{xtccomment}
\begin{spadsrc}
newSubProgram(F)$TheSymbolTable
\end{spadsrc}
\end{xtc}
\begin{xtc}
\begin{xtccomment}
It is possible to return to processing the main program segment by issuing
the command:
\end{xtccomment}
\begin{spadsrc}
endSubProgram()$TheSymbolTable
\end{spadsrc}
\begin{TeXOutput}
\begin{fricasmath}{2}
\SYMBOL{MAIN}%
\end{fricasmath}
\end{TeXOutput}
\formatResultType{Symbol}
\end{xtc}
The following commands exist:
\begin{center}
\begin{tabular}{p{1.6in}p{2.8in}}
\spadfunX{returnType} & declares the return type of the current subprogram \\
 & \\
\spadfun{returnTypeOf} & returns the return type of a subprogram \\
 & \\
\spadfunX{argumentList} &  declares the argument list of the current subprogram \\
 & \\
\spadfun{argumentListOf} &  returns the argument list of a subprogram \\
 & \\
\spadfunX{declare} & provides type declarations for parameters of the current subprogram \\
 & \\
\spadfun{symbolTableOf} & returns the symbol table  of a subprogram \\
 & \\
\spadfun{printHeader} & produces the Fortran header for the current subprogram \\
\end{tabular}
\end{center}
In addition there are versions of these commands which are parameterised by
the name of a subprogram, and others parameterised by both the name of a
subprogram and by an instance of \spadtype{TheSymbolTable}.
\begin{xtc}
\begin{xtccomment}
\end{xtccomment}
\begin{spadsrc}
newSubProgram(F)$TheSymbolTable 
\end{spadsrc}
\end{xtc}
\begin{xtc}
\begin{xtccomment}
\end{xtccomment}
\begin{spadsrc}
argumentList!(F, [X])$TheSymbolTable
\end{spadsrc}
\end{xtc}
\begin{xtc}
\begin{xtccomment}
\end{xtccomment}
\begin{spadsrc}
returnType!(F,real)$TheSymbolTable 
\end{spadsrc}
\end{xtc}
\begin{xtc}
\begin{xtccomment}
\end{xtccomment}
\begin{spadsrc}
declare!(X,fortranReal(),F)$TheSymbolTable 
\end{spadsrc}
\begin{TeXOutput}
\begin{fricasmath}{6}
\SYMBOL{REAL}%
\end{fricasmath}
\end{TeXOutput}
\formatResultType{FortranType}
\end{xtc}
\begin{xtc}
\begin{xtccomment}
\end{xtccomment}
\begin{spadsrc}
printHeader(F)$TheSymbolTable
\end{spadsrc}
\end{xtc}

% ----------------------------------------------------------------------
\head{subsection}{Advanced Fortran Code Generation}{ugGenForAdvanced}
% ----------------------------------------------------------------------

This section describes facilities for representing Fortran statements, and
building up complete subprograms from them.

% ----------------------------------------------------------------------
\head{subsection}{Switch}{ugGenForSwitch}
% ----------------------------------------------------------------------
\exptypeindex{Switch}

This domain is used to represent statements like {\tt x < y}.  Although
these can be represented directly in \Language{}, it is a little cumbersome.
% since currently \Language{} performs several transformations on
% conditional operators, for example {\tt x >= y } is transformed to
% {\tt not(x < y)}, which while logically equivalent may be different
% from desired result.

Instead we have a set of operations, such as \spadfun{LT} to represent
\spad{<},
to let us build such statements.  The available constructors are:
\begin{center}
\begin{tabular}{ll}
\spadfun{LT} & $<$ \\
\spadfun{GT} & $>$ \\
\spadfun{LE} & $\leq$ \\
\spadfun{GE} & $\geq$ \\
\spadfun{EQ} & $=$ \\
\spadfun{AND} & $and$\\
\spadfun{OR} & $or$ \\
\spadfun{NOT} & $not$ \\
\end{tabular}
\end{center}
\begin{xtc}
\begin{xtccomment}
So for example:
\end{xtccomment}
\begin{spadsrc}
LT(x,y)$Switch
\end{spadsrc}
\begin{TeXOutput}
\begin{fricasmath}{1}
\SYMBOL{x}<\SYMBOL{y}%
\end{fricasmath}
\end{TeXOutput}
\formatResultType{Switch}
\end{xtc}

% ----------------------------------------------------------------------
\head{subsection}{FortranCode}{ugGenForCode}
% ----------------------------------------------------------------------

This domain represents code segments or operations: currently assignments,
conditionals, blocks, comments, gotos, continues, various kinds of loops,
and return statements.
\begin{xtc}
\begin{xtccomment}
For example we can create quite a complicated conditional statement using
assignments, and then turn it into Fortran code:
\end{xtccomment}
\begin{spadsrc}
c := cond(LT(X,Y),assign(F,X),cond(GT(Y,Z),assign(F,Y),assign(F,Z))$FortranCode)$FortranCode 
\end{spadsrc}
\begin{TeXOutput}
\begin{fricasmath}{1}
\STRING{"conditional"}%
\end{fricasmath}
\end{TeXOutput}
\formatResultType{FortranCode}
\end{xtc}
\begin{xtc}
\begin{xtccomment}
\end{xtccomment}
\begin{spadsrc}
printCode c
\end{spadsrc}
\end{xtc}

The Fortran code is printed
on the current Fortran output stream.

% ----------------------------------------------------------------------
\head{subsection}{FortranProgram}{ugGenForProgram}
% ----------------------------------------------------------------------
\exptypeindex{FortranProgram}

This domain is used to construct complete Fortran subprograms out of
elements of \spadtype{FortranCode}.  It is parameterised by the name of the
target subprogram (a \spadtype{Symbol}), its return type (from
\spadtype{Union}(\spadtype{FortranScalarType},``void'')),
its arguments (from \spadtype{List Symbol}), and
its symbol table (from \spadtype{SymbolTable}).  One can
\spadfun{coerce} elements of either \spadtype{FortranCode}
or \spadtype{Expression} into it.

\begin{xtc}
\begin{xtccomment}
First of all we create a symbol table:
\end{xtccomment}
\begin{spadsrc}
symbols := empty()$SymbolTable
\end{spadsrc}
\begin{TeXOutput}
\begin{fricasmath}{1}
\STRING{table}\PAREN{}%
\end{fricasmath}
\end{TeXOutput}
\formatResultType{SymbolTable}
\end{xtc}
\begin{xtc}
\begin{xtccomment}
Now put some type declarations into it:
\end{xtccomment}
\begin{spadsrc}
declare!([X,Y],fortranReal()$FortranType,symbols)
\end{spadsrc}
\begin{TeXOutput}
\begin{fricasmath}{2}
\SYMBOL{REAL}%
\end{fricasmath}
\end{TeXOutput}
\formatResultType{FortranType}
\end{xtc}
\begin{xtc}
\begin{xtccomment}
Then (for convenience)
we set up the particular instantiation of \spadtype{FortranProgram}
\end{xtccomment}
\begin{spadsrc}
FP := FortranProgram(F,real,[X,Y],symbols)
\end{spadsrc}
\begin{TeXOutput}
\begin{fricasmath}{3}
\STRING{FortranProgram(F,REAL,[X,Y],table(Y=REAL,X=REAL))}%
\end{fricasmath}
\end{TeXOutput}
\formatResultType{Type}
\end{xtc}
\begin{xtc}
\begin{xtccomment}
Create an object of type \spadtype{Expression(Integer)}:
\end{xtccomment}
\begin{spadsrc}
asp := X*sin(Y)
\end{spadsrc}
\begin{TeXOutput}
\begin{fricasmath}{4}
\SYMBOL{X}\TIMES \sin{\SYMBOL{Y}}%
\end{fricasmath}
\end{TeXOutput}
\formatResultType{Expression(Integer)}
\end{xtc}
\begin{xtc}
\begin{xtccomment}
Now \spadfun{coerce} it into \spadtype{FP}, and print its Fortran form:
\end{xtccomment}
\begin{spadsrc}
outputAsFortran(asp::FP)
\end{spadsrc}
\end{xtc}

We can generate a \spadtype{FortranProgram} using \spad{FortranCode}.
For example:
\begin{xtc}
\begin{xtccomment}
Augment our symbol table:
\end{xtccomment}
\begin{spadsrc}
declare!(Z,fortranReal()$FortranType,symbols)
\end{spadsrc}
\begin{TeXOutput}
\begin{fricasmath}{6}
\SYMBOL{REAL}%
\end{fricasmath}
\end{TeXOutput}
\formatResultType{FortranType}
\end{xtc}
\begin{xtc}
\begin{xtccomment}
and transform the conditional expression we prepared earlier:
\end{xtccomment}
\begin{spadsrc}
outputAsFortran([c,returns()$FortranCode]::FP) 
\end{spadsrc}
\begin{MessageOutput}
   Cannot convert the value from type List(Any) to FortranProgram(F,
      REAL,[X,Y],table(Z=REAL,Y=REAL,X=REAL)) .
\end{MessageOutput}
\end{xtc}

% !! DO NOT MODIFY THIS FILE BY HAND !! Created by spool2tex.awk.

% Copyright (c) 1991-2002, The Numerical ALgorithms Group Ltd.
% All rights reserved.
%
% Redistribution and use in source and binary forms, with or without
% modification, are permitted provided that the following conditions are
% met:
%
%     - Redistributions of source code must retain the above copyright
%       notice, this list of conditions and the following disclaimer.
%
%     - Redistributions in binary form must reproduce the above copyright
%       notice, this list of conditions and the following disclaimer in
%       the documentation and/or other materials provided with the
%       distribution.
%
%     - Neither the name of The Numerical ALgorithms Group Ltd. nor the
%       names of its contributors may be used to endorse or promote products
%       derived from this software without specific prior written permission.
%
% THIS SOFTWARE IS PROVIDED BY THE COPYRIGHT HOLDERS AND CONTRIBUTORS "AS
% IS" AND ANY EXPRESS OR IMPLIED WARRANTIES, INCLUDING, BUT NOT LIMITED
% TO, THE IMPLIED WARRANTIES OF MERCHANTABILITY AND FITNESS FOR A
% PARTICULAR PURPOSE ARE DISCLAIMED. IN NO EVENT SHALL THE COPYRIGHT OWNER
% OR CONTRIBUTORS BE LIABLE FOR ANY DIRECT, INDIRECT, INCIDENTAL, SPECIAL,
% EXEMPLARY, OR CONSEQUENTIAL DAMAGES (INCLUDING, BUT NOT LIMITED TO,
% PROCUREMENT OF SUBSTITUTE GOODS OR SERVICES-- LOSS OF USE, DATA, OR
% PROFITS-- OR BUSINESS INTERRUPTION) HOWEVER CAUSED AND ON ANY THEORY OF
% LIABILITY, WHETHER IN CONTRACT, STRICT LIABILITY, OR TORT (INCLUDING
% NEGLIGENCE OR OTHERWISE) ARISING IN ANY WAY OUT OF THE USE OF THIS
% SOFTWARE, EVEN IF ADVISED OF THE POSSIBILITY OF SUCH DAMAGE.


% *********************************************************************
\head{chapter}{Introduction to the \Language{} Interactive Language}{ugLang}
% *********************************************************************

In this chapter we look at some of the basic components of the
\Language{} language that you can use interactively.
We show how to create a \spadgloss{block} of expressions,
how to form loops and list iterations, how to modify the sequential
evaluation of a block and how to use {\tt if-then-else} to
evaluate parts of your program conditionally.
We suggest you first read the boxed material in each section and then
proceed to a more thorough reading of the chapter.

% *********************************************************************
\head{section}{Immediate and Delayed Assignments}{ugLangAssign}
% *********************************************************************

A \spadgloss{variable} in \Language{} refers to a value.
A variable has a name beginning with an uppercase or lowercase alphabetic
character, \spadSyntax{%}, or \spadSyntax{!}.
Successive characters (if any) can be any of the above, digits, or
\spadSyntax{?}.
Case is distinguished.
The following are all examples of valid, distinct variable names:
\begin{verbatim}
a             tooBig?    a1B2c3%!?
A             %j         numberOfPoints
beta6         %J         numberofpoints
\end{verbatim}

The \spadSyntax{:=} operator is the immediate \spadgloss{assignment}
operator.
\index{assignment!immediate}
Use it to associate a value with a variable.
\index{immediate assignment}

\beginImportant
The syntax for immediate assignment for a single variable is
\begin{center}
{\it variable} \spad{:=} {\it expression}
\end{center}
The value returned by an immediate assignment is the value of {\it expression}.
\endImportant

\begin{xtc}
\begin{xtccomment}
The right-hand side of the expression is evaluated,
yielding \spad{1}.  This value is then assigned to \spad{a}.
\end{xtccomment}
\begin{spadsrc}
a := 1 
\end{spadsrc}
\begin{TeXOutput}
\begin{fricasmath}{1}
1%
\end{fricasmath}
\end{TeXOutput}
\formatResultType{PositiveInteger}
\end{xtc}
\begin{xtc}
\begin{xtccomment}
The right-hand side of the expression is evaluated,
yielding \spad{1}.  This value is then assigned to \spad{b}.
Thus \spad{a} and \spad{b} both have the value \spad{1} after the sequence
of assignments.
\end{xtccomment}
\begin{spadsrc}
b := a 
\end{spadsrc}
\begin{TeXOutput}
\begin{fricasmath}{2}
1%
\end{fricasmath}
\end{TeXOutput}
\formatResultType{PositiveInteger}
\end{xtc}
\begin{xtc}
\begin{xtccomment}
What is the value of \spad{b} if \spad{a} is
assigned the value \spad{2}?
\end{xtccomment}
\begin{spadsrc}
a := 2 
\end{spadsrc}
\begin{TeXOutput}
\begin{fricasmath}{3}
2%
\end{fricasmath}
\end{TeXOutput}
\formatResultType{PositiveInteger}
\end{xtc}
\begin{xtc}
\begin{xtccomment}
As you see, the value of \spad{b} is left unchanged.
\end{xtccomment}
\begin{spadsrc}
b 
\end{spadsrc}
\begin{TeXOutput}
\begin{fricasmath}{4}
1%
\end{fricasmath}
\end{TeXOutput}
\formatResultType{PositiveInteger}
\end{xtc}
This is what we mean when we say this kind of assignment is
{\it immediate};
\spad{b} has no dependency on \spad{a} after the initial assignment.
This is the usual notion of assignment found in programming
languages such as C,
\index{C language!assignment}
PASCAL
\index{PASCAL!assignment}
and FORTRAN.
\index{FORTRAN!assignment}

\Language{} provides delayed assignment with \spadSyntax{==}.
\index{assignment!delayed}
This implements a
\index{delayed assignment}
delayed evaluation of the right-hand side and dependency
checking.

\beginImportant
The syntax for delayed assignment is
\begin{center}
{\it variable} \spad{==} {\it expression}
\end{center}
The value returned by a delayed assignment is \void{}.
\endImportant

\begin{xtc}
\begin{xtccomment}
Using \spad{a} and \spad{b} as above, these are the corresponding delayed
assignments.
\end{xtccomment}
\begin{spadsrc}
a == 1 
\end{spadsrc}
\end{xtc}
\begin{xtc}
\begin{xtccomment}
\end{xtccomment}
\begin{spadsrc}
b == a 
\end{spadsrc}
\end{xtc}
\begin{xtc}
\begin{xtccomment}
The right-hand side of each delayed assignment
is left unevaluated until the
variables on the left-hand sides are evaluated.
Therefore this evaluation and \ldots
\end{xtccomment}
\begin{spadsrc}
a 
\end{spadsrc}
\begin{MessageOutput}
   Compiling body of rule a to compute value of type PositiveInteger 
\end{MessageOutput}
\begin{TeXOutput}
\begin{fricasmath}{7}
1%
\end{fricasmath}
\end{TeXOutput}
\formatResultType{PositiveInteger}
\end{xtc}
\begin{xtc}
\begin{xtccomment}
this evaluation seem the same as before.
\end{xtccomment}
\begin{spadsrc}
b 
\end{spadsrc}
\begin{MessageOutput}
   Compiling body of rule b to compute value of type PositiveInteger 
\end{MessageOutput}
\begin{TeXOutput}
\begin{fricasmath}{8}
1%
\end{fricasmath}
\end{TeXOutput}
\formatResultType{PositiveInteger}
\end{xtc}
\begin{xtc}
\begin{xtccomment}
If we change \spad{a} to \spad{2}
\end{xtccomment}
\begin{spadsrc}
a == 2 
\end{spadsrc}
\begin{MessageOutput}
   Compiled code for a has been cleared.
\end{MessageOutput}
\begin{MessageOutput}
   Compiled code for b has been cleared.
\end{MessageOutput}
\begin{MessageOutput}
   1 old definition(s) deleted for function or rule a 
\end{MessageOutput}
\end{xtc}
\begin{xtc}
\begin{xtccomment}
then
\spad{a} evaluates to \spad{2}, as expected, but
\end{xtccomment}
\begin{spadsrc}
a 
\end{spadsrc}
\begin{MessageOutput}
   Compiling body of rule a to compute value of type PositiveInteger 
\end{MessageOutput}
\begin{TeXOutput}
\begin{fricasmath}{10}
2%
\end{fricasmath}
\end{TeXOutput}
\formatResultType{PositiveInteger}
\end{xtc}
\begin{xtc}
\begin{xtccomment}
the value of \spad{b} reflects the change to \spad{a}.
\end{xtccomment}
\begin{spadsrc}
b 
\end{spadsrc}
\begin{MessageOutput}
   Compiling body of rule b to compute value of type PositiveInteger 
\end{MessageOutput}
\begin{TeXOutput}
\begin{fricasmath}{11}
2%
\end{fricasmath}
\end{TeXOutput}
\formatResultType{PositiveInteger}
\end{xtc}

It is possible to set several variables at the same time
\index{assignment!multiple immediate}
by using
\index{multiple immediate assignment}
a \spadgloss{tuple} of variables and a tuple of expressions.\footnote{A
\spadgloss{tuple} is a collection of things separated by commas, often
surrounded by parentheses.}

% ----------------------------------------------------------------------
\beginImportant
The syntax for multiple immediate assignments is
\begin{center}
{\tt ( \subscriptIt{var}{1}, \subscriptIt{var}{2}, \ldots, \subscriptIt{var}{N} ) := ( \subscriptIt{expr}{1}, \subscriptIt{expr}{2}, \ldots, \subscriptIt{expr}{N} ) }
\end{center}
The value returned by an immediate assignment is the value of
\subscriptIt{expr}{N}.
\endImportant
% ----------------------------------------------------------------------

\begin{xtc}
\begin{xtccomment}
This sets \spad{x} to \spad{1} and \spad{y} to \spad{2}.
\end{xtccomment}
\begin{spadsrc}
(x,y) := (1,2) 
\end{spadsrc}
\begin{TeXOutput}
\begin{fricasmath}{12}
2%
\end{fricasmath}
\end{TeXOutput}
\formatResultType{PositiveInteger}
\end{xtc}
Multiple immediate assigments are parallel in the sense that the
expressions on the right are all evaluated before any assignments
on the left are made.
However, the order of evaluation of these expressions is undefined.
\begin{xtc}
\begin{xtccomment}
You can use multiple immediate assignment to swap the
values held by variables.
\end{xtccomment}
\begin{spadsrc}
(x,y) := (y,x) 
\end{spadsrc}
\begin{TeXOutput}
\begin{fricasmath}{13}
1%
\end{fricasmath}
\end{TeXOutput}
\formatResultType{PositiveInteger}
\end{xtc}
\begin{xtc}
\begin{xtccomment}
\spad{x} has the previous value of \spad{y}.
\end{xtccomment}
\begin{spadsrc}
x 
\end{spadsrc}
\begin{TeXOutput}
\begin{fricasmath}{14}
2%
\end{fricasmath}
\end{TeXOutput}
\formatResultType{PositiveInteger}
\end{xtc}
\begin{xtc}
\begin{xtccomment}
\spad{y} has the previous value of \spad{x}.
\end{xtccomment}
\begin{spadsrc}
y 
\end{spadsrc}
\begin{TeXOutput}
\begin{fricasmath}{15}
1%
\end{fricasmath}
\end{TeXOutput}
\formatResultType{PositiveInteger}
\end{xtc}

There is no syntactic form for multiple delayed assignments.
See the discussion in
\spadref{ugUserDelay}
about how \Language{} differentiates between delayed assignments and
user functions of no arguments.

% *********************************************************************
\head{section}{Blocks}{ugLangBlocks}
% *********************************************************************

%%
%% We should handle tabs in pile correctly but so far we do not.
%%

A \spadgloss{block} is a sequence of expressions evaluated
in the order that they appear, except as modified by control expressions
such as \spad{break},
\spadkey{break}
\spad{return},
\spadkey{return}
\spad{iterate} and
\spadkey{iterate}
\spad{if-then-else} constructions.
The value of a block is the value of the expression last evaluated
in the block.

To leave a block early, use \spadSyntax{=>}.
For example, \spad{i < 0 => x}.
The expression before the \spadSyntax{=>} must evaluate to
\spad{true} or \spad{false}.
The expression following the \spadSyntax{=>} is the return value
for the block.

A block can be constructed in two ways:
\begin{enumerate}
\item the expressions can be separated by semicolons
and the resulting expression surrounded by parentheses, and
\item the expressions can be written on succeeding lines with each line
indented the same number of spaces (which must be greater than zero).
\index{indentation}
A block entered in this form is
called a \spadgloss{pile}.
\end{enumerate}
Only the first form is available if you are entering expressions
directly to \Language{}.
Both forms are available in {\bf .input} files.

\beginImportant
The syntax for a simple block of expressions entered interactively is
\begin{center}
{\tt ( \subscriptIt{expression}{1}; \subscriptIt{expression}{2}; \ldots; \subscriptIt{expression}{N} )}
\end{center}
The value returned by a block is the value of an
\spadSyntax{=>} expression, or \subscriptIt{expression}{N}
if no \spadSyntax{=>} is encountered.
\endImportant

In {\bf .input} files, blocks can also be written using
\spadglossSee{piles}{pile}.
The examples throughout this book are assumed to come from {\bf .input} files.

\begin{xtc}
\begin{xtccomment}
In this example, we assign a rational number to \spad{a} using a block
consisting of three expressions.
This block is written as a pile.
Each expression in the pile has the same indentation, in this case two
spaces to the right of the first line.
\end{xtccomment}
\begin{spadsrc}
a :=
  i := gcd(234,672)
  i := 3*i^5 - i + 1
  1 / i
\end{spadsrc}
\begin{TeXOutput}
\begin{fricasmath}{1}
\frac{1}{23323}%
\end{fricasmath}
\end{TeXOutput}
\formatResultType{Fraction(Integer)}
\end{xtc}
\begin{xtc}
\begin{xtccomment}
Here is the same block written on one line.
This is how you are required to enter it at the input prompt.
\end{xtccomment}
\begin{spadsrc}
a := (i := gcd(234,672); i := 3*i^5 - i + 1; 1 / i)
\end{spadsrc}
\begin{TeXOutput}
\begin{fricasmath}{2}
\frac{1}{23323}%
\end{fricasmath}
\end{TeXOutput}
\formatResultType{Fraction(Integer)}
\end{xtc}
\begin{xtc}
\begin{xtccomment}
Blocks can be used to put several expressions on one line.
The value returned is that of the last expression.
\end{xtccomment}
\begin{spadsrc}
(a := 1; b := 2; c := 3; [a,b,c]) 
\end{spadsrc}
\begin{TeXOutput}
\begin{fricasmath}{3}
\BRACKET{1\COMMA 2\COMMA 3}%
\end{fricasmath}
\end{TeXOutput}
\formatResultType{List(PositiveInteger)}
\end{xtc}

\Language{} gives you two ways of writing a block and the
preferred way in an {\bf .input} file is to use a pile.
\index{file!input}
Roughly speaking, a pile is
a block whose constituent expressions are indented the same amount.
You begin a pile by starting a new line for the first expression,
indenting it to the right of the previous line.
You then enter the second expression on a new line, vertically aligning
it with the first line. And so on.
If you need to enter an inner pile, further indent its lines to the right
of the outer pile.
\Language{} knows where a pile ends.
It ends when a subsequent line is indented to the left of the pile or
the end of the file.

\begin{xtc}
\begin{xtccomment}
Blocks can be used to perform several steps before an assignment
(immediate or delayed) is made.
\end{xtccomment}
\begin{spadsrc}
d :=
   c := a^2 + b^2
   sqrt(c * 1.3)
\end{spadsrc}
\begin{TeXOutput}
\begin{fricasmath}{4}
\STRING{2.549509756796392415}%
\end{fricasmath}
\end{TeXOutput}
\formatResultType{Float}
\end{xtc}
\begin{xtc}
\begin{xtccomment}
Blocks can be used in the arguments to functions.
(Here \spad{h} is assigned \spad{2.1 + 3.5}.)
\end{xtccomment}
\begin{spadsrc}
h := 2.1 +
   1.0
   3.5
\end{spadsrc}
\begin{TeXOutput}
\begin{fricasmath}{5}
\STRING{5.6}%
\end{fricasmath}
\end{TeXOutput}
\formatResultType{Float}
\end{xtc}
\begin{xtc}
\begin{xtccomment}
Here the second argument to \spadfun{eval} is \spad{x = z}, where
the value of \spad{z} is computed in the first line of the block
starting on the second line.
\end{xtccomment}
\begin{spadsrc}
eval(x^2 - x*y^2,
     z := %pi/2.0 - exp(4.1)
     x = z
   )
\end{spadsrc}
\begin{TeXOutput}
\begin{fricasmath}{6}
\STRING{58.769491270567072878}\TIMES \SUPER{\SYMBOL{y}}{2}+\STRING{%
3453.853104201259382}%
\end{fricasmath}
\end{TeXOutput}
\formatResultType{Polynomial(Float)}
\end{xtc}
\begin{xtc}
\begin{xtccomment}
Blocks can be used in the clauses of \spad{if-then-else}
expressions (see \spadref{ugLangIf}).
\end{xtccomment}
\begin{spadsrc}
if h > 3.1 then 1.0 else (z := cos(h); max(z,0.5)) 
\end{spadsrc}
\begin{TeXOutput}
\begin{fricasmath}{7}
\STRING{1.0}%
\end{fricasmath}
\end{TeXOutput}
\formatResultType{Float}
\end{xtc}
\begin{xtc}
\begin{xtccomment}
This is the pile version of the last block.
\end{xtccomment}
\begin{spadsrc}
if h > 3.1 then
    1.0
  else
    z := cos(h)
    max(z,0.5)
\end{spadsrc}
\begin{TeXOutput}
\begin{fricasmath}{8}
\STRING{1.0}%
\end{fricasmath}
\end{TeXOutput}
\formatResultType{Float}
\end{xtc}
\begin{xtc}
\begin{xtccomment}
Blocks can be nested.
\end{xtccomment}
\begin{spadsrc}
a := (b := factorial(12); c := (d := eulerPhi(22); factorial(d));b+c)
\end{spadsrc}
\begin{TeXOutput}
\begin{fricasmath}{9}
482630400%
\end{fricasmath}
\end{TeXOutput}
\formatResultType{PositiveInteger}
\end{xtc}
\begin{xtc}
\begin{xtccomment}
This is the pile version of the last block.
\end{xtccomment}
\begin{spadsrc}
a :=
  b := factorial(12)
  c :=
    d := eulerPhi(22)
    factorial(d)
  b+c
\end{spadsrc}
\begin{TeXOutput}
\begin{fricasmath}{10}
482630400%
\end{fricasmath}
\end{TeXOutput}
\formatResultType{PositiveInteger}
\end{xtc}

\begin{xtc}
\begin{xtccomment}
Since \spad{c + d} does equal \spad{3628855}, \spad{a} has the value
of \spad{c} and the last line is never evaluated.
\end{xtccomment}
\begin{spadsrc}
a :=
  c := factorial 10
  d := fibonacci 10
  c + d = 3628855 => c
  d
\end{spadsrc}
\begin{TeXOutput}
\begin{fricasmath}{11}
3628800%
\end{fricasmath}
\end{TeXOutput}
\formatResultType{PositiveInteger}
\end{xtc}

% *********************************************************************
\head{section}{if-then-else}{ugLangIf}
% *********************************************************************

Like many other programming languages, \Language{} uses the three
% Remark: following strangeness with \spadkey is because \spadkey
% only creates an index term. Multiple sequential index macros without
% intervening text cause starnge spaces in the text.
keywords \spadkey{if} \spad{if, then} \spadkey{then} and \spad{else}
\spadkey{else} to form
\index{conditional}
conditional expressions.
The \spad{else} part of the conditional is optional.
The expression between the \spad{if} and \spad{then} keywords
is a
\spadgloss{predicate}: an expression that evaluates to or is convertible to
either {\tt true} or {\tt false}, that is,
a \spadtype{Boolean}.
\exptypeindex{Boolean}

% ----------------------------------------------------------------------
\beginImportant
The syntax for conditional expressions is
\begin{center}
{\tt if {\it predicate} then \subscriptIt{expression}{1} else \subscriptIt{expression}{2}}
\end{center}
where the \spad{else} \subscriptIt{\it expression}{2} part is optional.
The value returned from a conditional expression is
\subscriptIt{\it expression}{1} if the predicate evaluates to \spad{true}
and \subscriptIt{\it expression}{2} otherwise.
If no \spad{else} clause is given, the value is always \void{}.
\endImportant
% ----------------------------------------------------------------------

An \spad{if-then-else} expression always returns a value.
If the
\spad{else} clause is missing then the entire expression returns
\void{}.
If both clauses are present, the type of the value returned by \spad{if}
is obtained by resolving the types of the values of the two clauses.
See \spadref{ugTypesResolve}
for more information.

The predicate must evaluate to, or be convertible to, an object of type
\spadtype{Boolean}: {\tt true} or {\tt false}.
By default, the equal sign \spadopFrom{=}{Equation} creates
\index{equation}
an equation.

\begin{xtc}
\begin{xtccomment}
This is an equation.
\exptypeindex{Equation}
In particular, it is an object of type \spadtype{Equation Polynomial Integer}.
\end{xtccomment}
\begin{spadsrc}
x + 1 = y
\end{spadsrc}
\begin{TeXOutput}
\begin{fricasmath}{1}
\SYMBOL{x}+1=\SYMBOL{y}%
\end{fricasmath}
\end{TeXOutput}
\formatResultType{Equation(Polynomial(Integer))}
\end{xtc}
However, for predicates in \spad{if} expressions, \Language{}
\index{equality testing}
places a default target type of \spadtype{Boolean} on the
predicate and equality testing is performed.
Thus you need not qualify the \spadSyntax{=} in any way.
In other contexts you may need to tell \Language{} that you want
to test for equality rather than create an equation.
In those cases, use \spadSyntax{@} and a target type of
\spadtype{Boolean}.
See \spadref{ugTypesPkgCall} for more information.

The compound symbol meaning ``not equal'' in \Language{} is
\index{inequality testing}
\spadop{~=}.
This can be used directly without a package call or a target specification.
The expression
\spad{a ~= b} is directly translated into
\spad{not (a = b)}.

Many other functions have return values of type \spadtype{Boolean}.
These include \spadop{<}, \spadop{<=}, \spadop{>},
\spadop{>=}, \spadop{~=} and \spad{member?}.
By convention, operations with names ending in \spadSyntax{?}
return \spadtype{Boolean} values.

The usual rules for piles are suspended for conditional expressions.
In {\bf .input} files, the \spad{then} and
\spad{else} keywords can begin in the same column as the corresponding
\spad{if} but may also appear to the right.
Each of the following styles of writing \spad{if-then-else}
expressions is acceptable:
\begin{verbatim}
if i>0 then output("positive") else output("nonpositive")

if i > 0 then output("positive")
  else output("nonpositive")

if i > 0 then output("positive")
else output("nonpositive")

if i > 0
then output("positive")
else output("nonpositive")

if i > 0
  then output("positive")
  else output("nonpositive")
\end{verbatim}

A block can follow the \spad{then} or \spad{else} keywords.
In the following two assignments to \spad{a}, the \spad{then} and \spad{else}
clauses each are followed by two-line piles.
The value returned in each is the value of the second line.

\begin{verbatim}
a :=
  if i > 0 then
    j := sin(i * pi())
    exp(j + 1/j)
  else
    j := cos(i * 0.5 * pi())
    log(abs(j)^5 + 1)

a :=
  if i > 0
    then
      j := sin(i * pi())
      exp(j + 1/j)
    else
      j := cos(i * 0.5 * pi())
      log(abs(j)^5 + 1)
\end{verbatim}
These are both equivalent to the following:
\begin{verbatim}
a :=
  if i > 0 then (j := sin(i * pi()); exp(j + 1/j))
  else (j := cos(i * 0.5 * pi()); log(abs(j)^5 + 1))
\end{verbatim}

% *********************************************************************
\head{section}{Loops}{ugLangLoops}
% *********************************************************************

A \spadgloss{loop} is an expression that contains another expression,
\index{loop}
called the {\it loop body}, which is to be evaluated zero or more
\index{loop!body}
times.
All loops contain the \spad{repeat} keyword and return \void{}.
Loops can contain inner loops to any depth.

\beginImportant
The most basic loop is of the form
\begin{center}
\spad{repeat} {\it loopBody}
\end{center}
Unless {\it loopBody} contains a \spad{break} or \spad{return} expression,
the loop repeats forever.
The value returned by the loop is \void{}.
\endImportant

% *********************************************************************
\head{subsection}{Compiling vs. Interpreting Loops}{ugLangLoopsCompInt}
% *********************************************************************

\Language{} tries to determine completely the type of every
object in a loop and then to translate the loop body to LISP or even to
machine code.
This translation is called \spadglossSee{compilation}{compiler}.

If \Language{} decides that it cannot compile the loop, it issues a
\index{loop!compilation}
message stating the problem and then the following message:
%
\begin{center}
{\bf We will attempt to step through and interpret the code.}
\end{center}
%
It is still possible that \Language{} can evaluate the loop but in
\spadgloss{interpret-code mode}.
See \spadref{ugUserCompInt} where this is discussed in terms
\index{panic!avoiding}
of compiling versus interpreting functions.

% *********************************************************************
\head{subsection}{return in Loops}{ugLangLoopsReturn}
% *********************************************************************

A \spad{return} expression is used to exit a function with
\index{loop!leaving via return}
a particular value.
In particular, if a \spad{return} is in a loop within the
\spadkey{return}
function, the loop is terminated whenever the \spad{return}
is evaluated.
%> This is a bug! The compiler should never accept allow
%> Void to be the return type of a function when it has to use
%> resolve to determine it.
\begin{xtc}
\begin{xtccomment}
Suppose we start with this.
\end{xtccomment}
\begin{spadsrc}
f() ==
  i := 1
  repeat
    if factorial(i) > 1000 then return i
    i := i + 1
\end{spadsrc}
\end{xtc}
\begin{xtc}
\begin{xtccomment}
When \spad{factorial(i)} is big enough, control passes from
inside the loop all the way outside the function, returning the
value of \spad{i} (or so we think).
\end{xtccomment}
\begin{spadsrc}
f() 
\end{spadsrc}
\begin{MessageOutput}
   Compiling function f with type () -> Void 
\end{MessageOutput}
\end{xtc}

What went wrong?
Isn't it obvious that this function should return an integer?
Well, \Language{} makes no attempt to analyze the structure of a
loop to determine if it always returns a value because, in
general, this is impossible.
So \Language{} has this simple rule: the type of the function is
determined by the type of its body, in this case a block.
The normal value of a block is the value of its last expression,
in this case, a loop.
And the value of every loop is \void{}!
So the return type of \userfun{f} is \spadtype{Void}.

There are two ways to fix this.
The best way is for you to tell \Language{} what the return type
of \spad{f} is.
You do this by giving \spad{f} a declaration \spad{f: () -> Integer}
prior to calling for its value.
This tells \Language{}: ``trust me---an integer is returned.''
We'll explain more about this in the next chapter.
Another clumsy way is to add a dummy expression as follows.

\begin{xtc}
\begin{xtccomment}
Since we want an integer, let's stick in a dummy final expression that is
an integer and will never be evaluated.
\end{xtccomment}
\begin{spadsrc}
f() ==
  i := 1
  repeat
    if factorial(i) > 1000 then return i
    i := i + 1
  0
\end{spadsrc}
\begin{MessageOutput}
   Compiled code for f has been cleared.
\end{MessageOutput}
\begin{MessageOutput}
   1 old definition(s) deleted for function or rule f 
\end{MessageOutput}
\end{xtc}
\begin{xtc}
\begin{xtccomment}
When we try \userfun{f} again we get what we wanted.
See
\spadref{ugUserBlocks}
for more information.
\end{xtccomment}
\begin{spadsrc}
f() 
\end{spadsrc}
\begin{MessageOutput}
   Compiling function f with type () -> NonNegativeInteger 
\end{MessageOutput}
\begin{TeXOutput}
\begin{fricasmath}{4}
7%
\end{fricasmath}
\end{TeXOutput}
\formatResultType{PositiveInteger}
\end{xtc}

% *********************************************************************
\head{subsection}{break in Loops}{ugLangLoopsBreak}
% *********************************************************************

The \spad{break} keyword is often more useful
\spadkey{break}
in terminating
\index{loop!leaving via break}
a loop.
%>  and more in keeping with the ideas of structured programming.
A \spad{break} causes control to transfer to the expression
immediately following the loop.
As loops always return \void{},
you cannot return a value with \spad{break}.
That is, \spad{break} takes no argument.

\begin{xtc}
\begin{xtccomment}
This example is a modification of the last example in
the previous section.
Instead of using \spad{return}, we'll use \spad{break}.
\end{xtccomment}
\begin{spadsrc}
f() ==
  i := 1
  repeat
    if factorial(i) > 1000 then break
    i := i + 1
  i
\end{spadsrc}
\end{xtc}
\begin{xtc}
\begin{xtccomment}
The loop terminates when \spad{factorial(i)} gets big enough,
the last line of the function evaluates to the corresponding ``good''
value of \spad{i}, and the function terminates, returning that value.
\end{xtccomment}
\begin{spadsrc}
f() 
\end{spadsrc}
\begin{MessageOutput}
   Compiling function f with type () -> PositiveInteger 
\end{MessageOutput}
\begin{TeXOutput}
\begin{fricasmath}{2}
7%
\end{fricasmath}
\end{TeXOutput}
\formatResultType{PositiveInteger}
\end{xtc}
\begin{xtc}
\begin{xtccomment}
You can only use \spad{break} to terminate the evaluation of one loop.
Let's consider a loop within a loop, that is, a loop with a nested loop.
First, we initialize two counter variables.
\end{xtccomment}
\begin{spadsrc}
(i,j) := (1, 1) 
\end{spadsrc}
\begin{TeXOutput}
\begin{fricasmath}{3}
1%
\end{fricasmath}
\end{TeXOutput}
\formatResultType{PositiveInteger}
\end{xtc}
\begin{xtc}
\begin{xtccomment}
Nested loops must have multiple \spad{break}
\index{loop!nested}
expressions at the appropriate nesting level.
How would you rewrite this so \spad{(i + j) > 10} is only evaluated once?
\end{xtccomment}
\begin{spadsrc}
repeat
  repeat
    if (i + j) > 10 then break
    j := j + 1
  if (i + j) > 10 then break
  i := i + 1
\end{spadsrc}
\end{xtc}

% *********************************************************************
\head{subsection}{break vs. {\tt =>} in Loop Bodies}{ugLangLoopsBreakVs}
% *********************************************************************

Compare the following two loops:

\begin{verbatim}
i := 1                            i := 1
repeat                            repeat
  i := i + 1                        i := i + 1
  i > 3 => i                        if i > 3 then break
  output(i)                         output(i)
\end{verbatim}

In the example on the left, the values
\mathOrSpad{2} and \mathOrSpad{3} for \spad{i} are displayed
but then the \spadSyntax{=>} does not allow control to reach the call to
\spadfunFrom{output}{OutputForm} again.
The loop will not terminate
until you run out of space or interrupt the execution.
The variable \spad{i} will continue to be incremented because
the \spadSyntax{=>} only means to leave the {\it block,} not the loop.

In the example on the right,
upon reaching \mathOrSpad{4}, the \spad{break} will be
executed, and both the block and the loop will terminate.
This is one of the reasons why both \spadSyntax{=>} and \spad{break} are
provided.
Using a \spad{while} clause (see below) with the \spadSyntax{=>}
\spadkey{while}
lets you simulate the action of \spad{break}.

% *********************************************************************
\head{subsection}{More Examples of break}{ugLangLoopsBreakMore}
% *********************************************************************

Here we give four examples of \spad{repeat} loops that
terminate when a value exceeds a given bound.

\vskip 1pc
\begin{xtc}
\begin{xtccomment}
First, initialize \spad{i} as the loop counter.
\end{xtccomment}
\begin{spadsrc}
i := 0 
\end{spadsrc}
\begin{TeXOutput}
\begin{fricasmath}{1}
0%
\end{fricasmath}
\end{TeXOutput}
\formatResultType{NonNegativeInteger}
\end{xtc}
\begin{xtc}
\begin{xtccomment}
Here is the first loop.
When the square of \spad{i} exceeds \spad{100}, the loop terminates.
\end{xtccomment}
\begin{spadsrc}
repeat
  i := i + 1
  if i^2 > 100 then break
\end{spadsrc}
\end{xtc}
\begin{xtc}
\begin{xtccomment}
Upon completion, \spad{i} should have the value \spad{11}.
\end{xtccomment}
\begin{spadsrc}
i 
\end{spadsrc}
\begin{TeXOutput}
\begin{fricasmath}{3}
11%
\end{fricasmath}
\end{TeXOutput}
\formatResultType{NonNegativeInteger}
\end{xtc}
%
%
\begin{xtc}
\begin{xtccomment}
Do the same thing except use \spadSyntax{=>} instead
an \spad{if-then} expression.
\end{xtccomment}
\begin{spadsrc}
i := 0 
\end{spadsrc}
\begin{TeXOutput}
\begin{fricasmath}{4}
0%
\end{fricasmath}
\end{TeXOutput}
\formatResultType{NonNegativeInteger}
\end{xtc}
\begin{xtc}
\begin{xtccomment}
\end{xtccomment}
\begin{spadsrc}
repeat
  i := i + 1
  i^2 > 100 => break
\end{spadsrc}
\end{xtc}
\begin{xtc}
\begin{xtccomment}
\end{xtccomment}
\begin{spadsrc}
i 
\end{spadsrc}
\begin{TeXOutput}
\begin{fricasmath}{6}
11%
\end{fricasmath}
\end{TeXOutput}
\formatResultType{NonNegativeInteger}
\end{xtc}
%
%
\begin{xtc}
\begin{xtccomment}
As a third example, we use a simple loop to compute \spad{n!}.
\end{xtccomment}
\begin{spadsrc}
(n, i, f) := (100, 1, 1) 
\end{spadsrc}
\begin{TeXOutput}
\begin{fricasmath}{7}
1%
\end{fricasmath}
\end{TeXOutput}
\formatResultType{PositiveInteger}
\end{xtc}
\begin{xtc}
\begin{xtccomment}
Use \spad{i} as the iteration variable and \spad{f}
to compute the factorial.
\end{xtccomment}
\begin{spadsrc}
repeat
  if i > n then break
  f := f * i
  i := i + 1
\end{spadsrc}
\end{xtc}
\begin{xtc}
\begin{xtccomment}
Look at the value of \spad{f}.
\end{xtccomment}
\begin{spadsrc}
f 
\end{spadsrc}
\begin{TeXOutput}
\begin{fricasmath}{9}
93326215443944152681 69923885626670049071 59682643816214685929 63895217599993229915 60894146397615651828 62536979208272237582 51185210916864000000 000000000000000000%
\end{fricasmath}
\end{TeXOutput}
\formatResultType{PositiveInteger}
\end{xtc}
%
%
\begin{xtc}
\begin{xtccomment}
Finally, we show an example of nested loops.
First define a four by four matrix.
\end{xtccomment}
\begin{spadsrc}
m := matrix [[21,37,53,14], [8,-24,22,-16], [2,10,15,14], [26,33,55,-13]] 
\end{spadsrc}
\begin{TeXOutput}
\begin{fricasmath}{10}
\begin{MATRIX}{4}21&37&53&14\\8&-{24}&22&-{16}\\2&10&15&14\\26&33&55&-{13}%
\end{MATRIX}%
\end{fricasmath}
\end{TeXOutput}
\formatResultType{Matrix(Integer)}
\end{xtc}
\begin{xtc}
\begin{xtccomment}
Next, set row counter \spad{r} and column counter \spad{c} to
\mathOrSpad{1}.
Note: if we were writing a function, these would all be local
variables rather than global workspace variables.
\end{xtccomment}
\begin{spadsrc}
(r, c) := (1, 1) 
\end{spadsrc}
\begin{TeXOutput}
\begin{fricasmath}{11}
1%
\end{fricasmath}
\end{TeXOutput}
\formatResultType{PositiveInteger}
\end{xtc}
\begin{xtc}
\begin{xtccomment}
Also, let \spad{lastrow} and
\spad{lastcol} be the final row and column index.
\end{xtccomment}
\begin{spadsrc}
(lastrow, lastcol) := (nrows(m), ncols(m)) 
\end{spadsrc}
\begin{TeXOutput}
\begin{fricasmath}{12}
4%
\end{fricasmath}
\end{TeXOutput}
\formatResultType{PositiveInteger}
\end{xtc}
%
\begin{xtc}
\begin{xtccomment}
Scan the rows looking for the first negative element.
We remark that you can reformulate this example in a better, more
concise form by using a \spad{for} clause with \spad{repeat}.
See
\spadref{ugLangLoopsForIn}
for more information.
\end{xtccomment}
\begin{spadsrc}
repeat
  if r > lastrow then break
  c := 1
  repeat
    if c > lastcol then break
    if elt(m,r,c) < 0 then
      output [r, c, elt(m,r,c)]
      r := lastrow
      break     -- don't look any further
    c := c + 1
  r := r + 1
\end{spadsrc}
\end{xtc}

% *********************************************************************
\head{subsection}{iterate in Loops}{ugLangLoopsIterate}
% *********************************************************************

\Language{} provides an \spad{iterate} expression that
\spadkey{iterate}
skips over the remainder of a loop body and starts the next loop iteration.
\begin{xtc}
\begin{xtccomment}
We first initialize a counter.
\end{xtccomment}
\begin{spadsrc}
i := 0 
\end{spadsrc}
\begin{TeXOutput}
\begin{fricasmath}{1}
0%
\end{fricasmath}
\end{TeXOutput}
\formatResultType{NonNegativeInteger}
\end{xtc}
\begin{xtc}
\begin{xtccomment}
Display the even integers from \spad{2} to \spad{5}.
\end{xtccomment}
\begin{spadsrc}
repeat
  i := i + 1
  if i > 5 then break
  if odd?(i) then iterate
  output(i)
\end{spadsrc}
\end{xtc}

% *********************************************************************
\head{subsection}{while Loops}{ugLangLoopsWhile}
% *********************************************************************

The \spad{repeat} in a loop can be modified by adding one or
more \spad{while} clauses.
\spadkey{while}
Each clause contains a \spadgloss{predicate}
immediately following the \spad{while} keyword.
The predicate is tested {\it before}
the evaluation of the body of the loop.
The loop body is evaluated whenever the predicates in a \spad{while}
clause are all \spad{true}.

\beginImportant
The syntax for a simple loop using \spad{while} is
\begin{center}
\spad{while} {\it predicate} \spad{repeat} {\it loopBody}
\end{center}
The {\it predicate} is evaluated before {\it loopBody} is evaluated.
A \spad{while} loop terminates immediately when {\it predicate}
evaluates to \spad{false} or when a \spad{break} or \spad{return}
expression is evaluated in {\it loopBody}.
The value returned by the loop is \void{}.
\endImportant

\begin{xtc}
\begin{xtccomment}
Here is a simple example of using \spad{while} in a loop.
We first initialize the counter.
\end{xtccomment}
\begin{spadsrc}
i := 1 
\end{spadsrc}
\begin{TeXOutput}
\begin{fricasmath}{1}
1%
\end{fricasmath}
\end{TeXOutput}
\formatResultType{PositiveInteger}
\end{xtc}
\begin{xtc}
\begin{xtccomment}
The steps involved in computing this example are
(1) set \spad{i} to \spad{1}, (2) test the condition \spad{i < 1} and
determine that it is not true, and (3) do not evaluate the
loop body and therefore do not display \spad{"hello"}.
\end{xtccomment}
\begin{spadsrc}
while i < 1 repeat
  output "hello"
  i := i + 1
\end{spadsrc}
\end{xtc}
\begin{xtc}
\begin{xtccomment}
If you have multiple predicates to be tested use the
logical \spad{and} operation to separate them.
\Language{} evaluates these predicates from left to right.
\end{xtccomment}
\begin{spadsrc}
(x, y) := (1, 1) 
\end{spadsrc}
\begin{TeXOutput}
\begin{fricasmath}{3}
1%
\end{fricasmath}
\end{TeXOutput}
\formatResultType{PositiveInteger}
\end{xtc}
\begin{xtc}
\begin{xtccomment}
\end{xtccomment}
\begin{spadsrc}
while x < 4 and y < 10 repeat
  output [x,y]
  x := x + 1
  y := y + 2
\end{spadsrc}
\end{xtc}
\begin{xtc}
\begin{xtccomment}
A \spad{break} expression can be included in a loop body to terminate a
loop even if the predicate in any \spad{while} clauses are not \spad{false}.
\end{xtccomment}
\begin{spadsrc}
(x, y) := (1, 1) 
\end{spadsrc}
\begin{TeXOutput}
\begin{fricasmath}{5}
1%
\end{fricasmath}
\end{TeXOutput}
\formatResultType{PositiveInteger}
\end{xtc}
\begin{xtc}
\begin{xtccomment}
This loop has multiple \spad{while} clauses and the loop terminates
before any one of their conditions evaluates to \spad{false}.
\end{xtccomment}
\begin{spadsrc}
while x < 4 while y < 10 repeat
  if x + y > 7 then break
  output [x,y]
  x := x + 1
  y := y + 2
\end{spadsrc}
\end{xtc}
\begin{xtc}
\begin{xtccomment}
Here's a different version of the nested loops that looked
for the first negative element in a matrix.
\end{xtccomment}
\begin{spadsrc}
m := matrix [[21,37,53,14], [8,-24,22,-16], [2,10,15,14], [26,33,55,-13]] 
\end{spadsrc}
\begin{TeXOutput}
\begin{fricasmath}{7}
\begin{MATRIX}{4}21&37&53&14\\8&-{24}&22&-{16}\\2&10&15&14\\26&33&55&-{13}%
\end{MATRIX}%
\end{fricasmath}
\end{TeXOutput}
\formatResultType{Matrix(Integer)}
\end{xtc}
\begin{xtc}
\begin{xtccomment}
Initialized the row index to \spad{1} and
get the number of rows and columns.
If we were writing a function, these would all be
local variables.
\end{xtccomment}
\begin{spadsrc}
r := 1 
\end{spadsrc}
\begin{TeXOutput}
\begin{fricasmath}{8}
1%
\end{fricasmath}
\end{TeXOutput}
\formatResultType{PositiveInteger}
\end{xtc}
\begin{xtc}
\begin{xtccomment}
\end{xtccomment}
\begin{spadsrc}
(lastrow, lastcol) := (nrows(m), ncols(m)) 
\end{spadsrc}
\begin{TeXOutput}
\begin{fricasmath}{9}
4%
\end{fricasmath}
\end{TeXOutput}
\formatResultType{PositiveInteger}
\end{xtc}
%
\begin{xtc}
\begin{xtccomment}
Scan the rows looking for the first negative element.
\end{xtccomment}
\begin{spadsrc}
while r <= lastrow repeat
  c := 1  -- index of first column
  while c <= lastcol repeat
    if elt(m,r,c) < 0 then
      output [r, c, elt(m,r,c)]
      r := lastrow
      break     -- don't look any further
    c := c + 1
  r := r + 1
\end{spadsrc}
\end{xtc}

% *********************************************************************
\head{subsection}{for Loops}{ugLangLoopsForIn}
% *********************************************************************

\Language{} provides the \spad{for}
\spadkey{for}
and \spad{in}
\spadkey{in}
keywords in \spad{repeat} loops,
allowing you to iterate across all
\index{iteration}
elements of a list, or to have a variable take on integral values
from a lower bound to an upper bound.
We shall refer to these modifying clauses of \spad{repeat} loops as
\spad{for} clauses.
These clauses can be present in addition to \spad{while} clauses.
As with all other types of \spad{repeat} loops, \spad{break} can
\spadkey{break}
be used to prematurely terminate the evaluation of the loop.

\beginImportant
The syntax for a simple loop using \spad{for} is
\begin{center}
\spad{for} {\it iterator} \spad{repeat} {\it loopBody}
\end{center}
The {\it iterator} has several forms.
Each form has an end test which is evaluated
before {\it loopBody} is evaluated.
A \spad{for} loop terminates immediately when the end test
succeeds (evaluates to \spad{true}) or when a \spad{break} or \spad{return}
expression is evaluated in {\it loopBody}.
The value returned by the loop is \void{}.
\endImportant

% *********************************************************************
\head{subsection}{for i in n..m repeat}{ugLangLoopsForInNM}
% *********************************************************************

If \spad{for}
\spadkey{for}
is followed by a variable name, the \spad{in}
\spadkey{in}
keyword and then an integer segment of the form \spad{n..m},
\index{segment}
the end test for this loop is the predicate \spad{i > m}.
The body of the loop is evaluated \spad{m-n+1} times if this
number is greater than 0.
If this number is less than or equal to 0, the loop body is not evaluated
at all.

The variable \spad{i} has the value
\spad{n, n+1, ..., m} for successive iterations
of the loop body.
The loop variable is a \spadgloss{local variable}
within the loop body: its value is not available outside the loop body
and its value and type within the loop body completely mask any outer
definition of a variable with the same name.

%
\begin{xtc}
\begin{xtccomment}
This loop prints the values of
${10}^3$, ${11}^3$, and $12^3$:
\end{xtccomment}
\begin{spadsrc}
for i in 10..12 repeat output(i^3)
\end{spadsrc}
\end{xtc}
%
\begin{xtc}
\begin{xtccomment}
Here is a sample list.
\end{xtccomment}
\begin{spadsrc}
a := [1,2,3] 
\end{spadsrc}
\begin{TeXOutput}
\begin{fricasmath}{2}
\BRACKET{1\COMMA 2\COMMA 3}%
\end{fricasmath}
\end{TeXOutput}
\formatResultType{List(PositiveInteger)}
\end{xtc}
\begin{xtc}
\begin{xtccomment}
Iterate across this list, using \spadSyntax{.} to access the elements of a list and
the \spadop{#} operation to count its elements.
\end{xtccomment}
\begin{spadsrc}
for i in 1..#a repeat output(a.i) 
\end{spadsrc}
\end{xtc}
%
This type of iteration is applicable to anything that uses \spadSyntax{.}.
You can also use it with functions that use indices to extract elements.
%
\begin{xtc}
\begin{xtccomment}
Define \spad{m} to be a matrix.
\end{xtccomment}
\begin{spadsrc}
m := matrix [[1,2],[4,3],[9,0]] 
\end{spadsrc}
\begin{TeXOutput}
\begin{fricasmath}{4}
\begin{MATRIX}{2}1&2\\4&3\\9&0\end{MATRIX}%
\end{fricasmath}
\end{TeXOutput}
\formatResultType{Matrix(NonNegativeInteger)}
\end{xtc}
\begin{xtc}
\begin{xtccomment}
Display the rows of \spad{m}.
\end{xtccomment}
\begin{spadsrc}
for i in 1..nrows(m) repeat output row(m,i) 
\end{spadsrc}
\end{xtc}
%
You can use \spad{iterate} with \spad{for}-loops.
\spadkey{iterate}
\begin{xtc}
\begin{xtccomment}
Display the even integers in a segment.
\end{xtccomment}
\begin{spadsrc}
for i in 1..5 repeat
  if odd?(i) then iterate
  output(i)
\end{spadsrc}
\end{xtc}

See \xmpref{Segment} for more information about segments.

% *********************************************************************
\head{subsection}{for i in n..m by s repeat}{ugLangLoopsForInNMS}
% *********************************************************************

By default, the difference between values taken on by a variable in loops
such as \spad{for i in n..m repeat ...} is \mathOrSpad{1}.
It is possible to supply another, possibly negative, step value by using
the \spad{by}
\spadkey{by}
keyword along with \spad{for} and \spad{in}.
Like the upper and lower bounds, the step value following the
\spad{by} keyword must be an integer.
Note that the loop
\spad{for i in 1..2 by 0 repeat output(i)}
will not terminate by itself, as the step value does not change the index
from its initial value of \mathOrSpad{1}.

\begin{xtc}
\begin{xtccomment}
This expression displays the odd integers between two bounds.
\end{xtccomment}
\begin{spadsrc}
for i in 1..5 by 2 repeat output(i)
\end{spadsrc}
\end{xtc}
\begin{xtc}
\begin{xtccomment}
Use this to display the numbers in reverse order.
\end{xtccomment}
\begin{spadsrc}
for i in 5..1 by -2 repeat output(i)
\end{spadsrc}
\end{xtc}

% *********************************************************************
\head{subsection}{for i in n.. repeat}{ugLangLoopsForInN}
% *********************************************************************

If the value after the \spadSyntax{..}
is omitted, the loop has no end test.
A potentially infinite loop is thus created.
The variable is given the successive values \spad{n, n+1, n+2, ...}
and the loop is terminated only if a \spad{break} or \spad{return}
expression is evaluated in the loop body.
However you may also add some other modifying clause on the
\spad{repeat} (for example, a \spad{while} clause) to stop the loop.

\begin{xtc}
\begin{xtccomment}
This loop displays the integers greater than or equal to \spad{15}
and less than the first prime greater than \spad{15}.
\end{xtccomment}
\begin{spadsrc}
for i in 15.. while not prime?(i) repeat output(i)
\end{spadsrc}
\end{xtc}

% *********************************************************************
\head{subsection}{for x in l repeat}{ugLangLoopsForInXL}
% *********************************************************************

Another variant of the \spad{for} loop has the form:
\begin{center}
{\it \spad{for} x \spad{in} list \spad{repeat} loopBody}
\end{center}
This form is used when you want to iterate directly over the
elements of a list.
In this form of the \spad{for} loop, the variable
\spad{x} takes on the value of each successive element in \spad{l}.
The end test is most simply stated in English: ``are there no more
\spad{x} in \spad{l}?''

\begin{xtc}
\begin{xtccomment}
If \spad{l} is this list,
\end{xtccomment}
\begin{spadsrc}
l := [0,-5,3] 
\end{spadsrc}
\begin{TeXOutput}
\begin{fricasmath}{1}
\BRACKET{0\COMMA -{5}\COMMA 3}%
\end{fricasmath}
\end{TeXOutput}
\formatResultType{List(Integer)}
\end{xtc}
\begin{xtc}
\begin{xtccomment}
display all elements of \spad{l}, one per line.
\end{xtccomment}
\begin{spadsrc}
for x in l repeat output(x) 
\end{spadsrc}
\end{xtc}

Since the list constructing expression \spad{expand [n..m]}
creates the list
\spad{[n, n+1, ..., m]}\footnote{This list is empty if \spad{n > m}.},
 you might be tempted to think that the loops
\begin{verbatim}
for i in n..m repeat output(i)
\end{verbatim}
and
\begin{verbatim}
for x in expand [n..m] repeat output(x)
\end{verbatim}
are equivalent.
The second form first creates the list
\spad{expand [n..m]} (no matter how large it might be) and
then does the iteration.
The first form potentially runs in much less space, as the index variable
\spad{i} is simply incremented once per loop and the list is not actually
created.
Using the first form is much more efficient.
%
\begin{xtc}
\begin{xtccomment}
Of course, sometimes you really want to iterate across a specific list.
This displays each of the factors of \spad{2400000}.
\end{xtccomment}
\begin{spadsrc}
for f in factors(factor(2400000)) repeat output(f)
\end{spadsrc}
\end{xtc}

% *********************************************************************
\head{subsection}{``Such that'' Predicates}{ugLangLoopsForInPred}
% *********************************************************************

A \spad{for} loop can be followed by a \spadSyntax{|} and then a
predicate.
The predicate qualifies the use of the values from the iterator following
the \spad{for}.
Think of the vertical bar
\spadSyntax{|} as the phrase ``such that.''
\begin{xtc}
\begin{xtccomment}
This loop expression
prints out the integers \spad{n} in the given segment
such that \spad{n} is odd.
\end{xtccomment}
\begin{spadsrc}
for n in 0..4 | odd? n repeat output n
\end{spadsrc}
\end{xtc}

\beginImportant
A \spad{for} loop can also be written
\begin{center}
\spad{for} {\it iterator} \spad{|} {\it predicate}  \spad{repeat} {\it loopBody}
\end{center}
which is equivalent to:
\begin{center}
\spad{for} {\it iterator} \spad{repeat if}
{\it predicate} \spad{then} {\it loopBody} \spad{else} \spad{iterate}
\end{center}
\endImportant

The predicate need not refer only to the variable in the \spad{for} clause:
any variable in an outer scope can be part of the predicate.
\begin{xtc}
\begin{xtccomment}
In this example, the predicate on the inner \spad{for} loop uses
\spad{i} from the outer loop and the \spad{j} from the \spad{for}
\index{iteration!nested}
clause that it directly modifies.
\end{xtccomment}
\begin{spadsrc}
for i in 1..50 repeat
  for j in 1..50 | factorial(i+j) < 25 repeat
    output [i,j]
\end{spadsrc}
\end{xtc}

% *********************************************************************
\head{subsection}{Parallel Iteration}{ugLangLoopsPar}
% *********************************************************************

The last example of
the previous section
gives an example of
\spadgloss{nested iteration}: a loop is contained
\index{iteration!nested}
in another loop.
\index{iteration!parallel}
Sometimes you want to iterate across two lists in parallel, or perhaps
you want to traverse a list while incrementing a variable.

\beginImportant
The general syntax of a repeat loop is
\begin{center}
{\tt \subscriptIt{iterator}{1} \subscriptIt{iterator}{2} \ldots \subscriptIt{iterator}{N} repeat {\it loopBody}}
\end{center}
where each {\it iterator} is either a \spad{for} or a \spad{while} clause.
The loop terminates immediately when the end test of any {\it iterator}
succeeds or when a \spad{break} or \spad{return} expression is evaluated
in {\it loopBody}.
The value returned by the loop is \void{}.
\endImportant

\begin{xtc}
\begin{xtccomment}
Here we write a loop to iterate across
two lists, computing the sum of the pairwise product
of elements. Here is the first list.
\end{xtccomment}
\begin{spadsrc}
l := [1,3,5,7] 
\end{spadsrc}
\begin{TeXOutput}
\begin{fricasmath}{1}
\BRACKET{1\COMMA 3\COMMA 5\COMMA 7}%
\end{fricasmath}
\end{TeXOutput}
\formatResultType{List(PositiveInteger)}
\end{xtc}
\begin{xtc}
\begin{xtccomment}
And the second.
\end{xtccomment}
\begin{spadsrc}
m := [100,200] 
\end{spadsrc}
\begin{TeXOutput}
\begin{fricasmath}{2}
\BRACKET{100\COMMA 200}%
\end{fricasmath}
\end{TeXOutput}
\formatResultType{List(PositiveInteger)}
\end{xtc}
\begin{xtc}
\begin{xtccomment}
The initial value of the sum counter.
\end{xtccomment}
\begin{spadsrc}
sum := 0 
\end{spadsrc}
\begin{TeXOutput}
\begin{fricasmath}{3}
0%
\end{fricasmath}
\end{TeXOutput}
\formatResultType{NonNegativeInteger}
\end{xtc}
\begin{xtc}
\begin{xtccomment}
The last two elements of \spad{l} are not used in the calculation
because \spad{m} has two fewer elements than \spad{l}.
\end{xtccomment}
\begin{spadsrc}
for x in l for y in m repeat
    sum := sum + x*y
\end{spadsrc}
\end{xtc}
\begin{xtc}
\begin{xtccomment}
Display the ``dot product.''
\end{xtccomment}
\begin{spadsrc}
sum 
\end{spadsrc}
\begin{TeXOutput}
\begin{fricasmath}{5}
700%
\end{fricasmath}
\end{TeXOutput}
\formatResultType{NonNegativeInteger}
\end{xtc}

\begin{xtc}
\begin{xtccomment}
Next, we write a loop to compute the sum of the products of the loop elements with
their positions in the loop.
\end{xtccomment}
\begin{spadsrc}
l := [2,3,5,7,11,13,17,19,23,29,31,37] 
\end{spadsrc}
\begin{TeXOutput}
\begin{fricasmath}{6}
\BRACKET{2\COMMA 3\COMMA 5\COMMA 7\COMMA 11\COMMA 13\COMMA 17\COMMA 19\COMMA %
23\COMMA 29\COMMA 31\COMMA 37}%
\end{fricasmath}
\end{TeXOutput}
\formatResultType{List(PositiveInteger)}
\end{xtc}
\begin{xtc}
\begin{xtccomment}
The initial sum.
\end{xtccomment}
\begin{spadsrc}
sum := 0 
\end{spadsrc}
\begin{TeXOutput}
\begin{fricasmath}{7}
0%
\end{fricasmath}
\end{TeXOutput}
\formatResultType{NonNegativeInteger}
\end{xtc}
\begin{xtc}
\begin{xtccomment}
Here looping stops when the list \spad{l} is exhausted, even though
the \spad{for i in 0..} specifies no terminating condition.
\end{xtccomment}
\begin{spadsrc}
for i in 0.. for x in l repeat sum := i * x 
\end{spadsrc}
\end{xtc}
\begin{xtc}
\begin{xtccomment}
Display this weighted sum.
\end{xtccomment}
\begin{spadsrc}
sum 
\end{spadsrc}
\begin{TeXOutput}
\begin{fricasmath}{9}
407%
\end{fricasmath}
\end{TeXOutput}
\formatResultType{NonNegativeInteger}
\end{xtc}

When \spadSyntax{|} is used to qualify any of the \spad{for} clauses in a
parallel iteration, the variables in the predicates can be from an outer
scope or from a \spad{for} clause in or to the left of a modified clause.

This is correct:
% output from following is too long to show
\begin{verbatim}
for i in 1..10 repeat
  for j in 200..300 | odd? (i+j) repeat
    output [i,j]
\end{verbatim}
This is not correct since the variable \spad{j} has not been
defined outside the inner loop.
\begin{verbatim}
for i in 1..10 | odd? (i+j) repeat  -- wrong, j not defined
  for j in 200..300 repeat
    output [i,j]
\end{verbatim}

%>% *********************************************************************
%>\head{subsection}{Mixing Loop Modifiers}{ugLangLoopsMix}
%>% *********************************************************************

\begin{xtc}
\begin{xtccomment}
This example shows that it is possible to mix several of the
\index{loop!mixing modifiers}
forms of \spad{repeat} modifying clauses on a loop.
\end{xtccomment}
\begin{spadsrc}
for i in 1..10
    for j in 151..160 | odd? j
      while i + j < 160 repeat
        output [i,j]
\end{spadsrc}
\end{xtc}
%
Here are useful rules for composing loop expressions:
\begin{enumerate}
\item \spad{while} predicates can only refer to variables that
are global (or in an outer scope)
or that are defined in \spad{for} clauses to the left of the
predicate.
\item A ``such that'' predicate (something following \spadSyntax{|})
must directly follow a \spad{for} clause and can only refer to
variables that are global (or in an outer scope)
or defined in the modified \spad{for} clause
or any \spad{for} clause to the left.
\end{enumerate}

% *********************************************************************
\head{section}{Creating Lists and Streams with Iterators}{ugLangIts}
% *********************************************************************

All of what we did for loops in \spadref{ugLangLoops}
\index{iteration}
can be transformed into expressions that create lists
\index{list!created by iterator}
and streams.
\index{stream!created by iterator}
The \spad{repeat,} \spad{break} or \spad{iterate} words are not used but
all the other ideas carry over.
Before we give you the general rule, here are some examples which
give you the idea.

\begin{xtc}
\begin{xtccomment}
This creates a simple list of the integers from \spad{1} to \spad{10}.
\end{xtccomment}
\begin{spadsrc}
mylist := [i for i in 1..10] 
\end{spadsrc}
\begin{TeXOutput}
\begin{fricasmath}{1}
\BRACKET{1\COMMA 2\COMMA 3\COMMA 4\COMMA 5\COMMA 6\COMMA 7\COMMA 8\COMMA 9%
\COMMA 10}%
\end{fricasmath}
\end{TeXOutput}
\formatResultType{List(PositiveInteger)}
\end{xtc}
\begin{xtc}
\begin{xtccomment}
Create a stream of the integers greater than or equal to \spad{1}.
\end{xtccomment}
\begin{spadsrc}
mystream := [i for i in 1..] 
\end{spadsrc}
\begin{TeXOutput}
\begin{fricasmath}{2}
\BRACKET{1\COMMA 2\COMMA 3\COMMA 4\COMMA 5\COMMA 6\COMMA 7\COMMA \STRING{...}%
}%
\end{fricasmath}
\end{TeXOutput}
\formatResultType{Stream(PositiveInteger)}
\end{xtc}
\begin{xtc}
\begin{xtccomment}
This is a list of the prime integers between \spad{1} and \spad{10},
inclusive.
\end{xtccomment}
\begin{spadsrc}
[i for i in 1..10 | prime? i]
\end{spadsrc}
\begin{TeXOutput}
\begin{fricasmath}{3}
\BRACKET{2\COMMA 3\COMMA 5\COMMA 7}%
\end{fricasmath}
\end{TeXOutput}
\formatResultType{List(PositiveInteger)}
\end{xtc}
\begin{xtc}
\begin{xtccomment}
This is a stream of the prime integers greater than or equal to \spad{1}.
\end{xtccomment}
\begin{spadsrc}
[i for i in 1..   | prime? i]
\end{spadsrc}
\begin{TeXOutput}
\begin{fricasmath}{4}
\BRACKET{2\COMMA 3\COMMA 5\COMMA 7\COMMA 11\COMMA 13\COMMA 17\COMMA \STRING{%
...}}%
\end{fricasmath}
\end{TeXOutput}
\formatResultType{Stream(PositiveInteger)}
\end{xtc}
\begin{xtc}
\begin{xtccomment}
This is a list of the integers between \spad{1} and \spad{10},
inclusive, whose squares are less than \spad{700}.
\end{xtccomment}
\begin{spadsrc}
[i for i in 1..10 while i*i < 700]
\end{spadsrc}
\begin{TeXOutput}
\begin{fricasmath}{5}
\BRACKET{1\COMMA 2\COMMA 3\COMMA 4\COMMA 5\COMMA 6\COMMA 7\COMMA 8\COMMA 9%
\COMMA 10}%
\end{fricasmath}
\end{TeXOutput}
\formatResultType{List(PositiveInteger)}
\end{xtc}
\begin{xtc}
\begin{xtccomment}
This is a stream of the integers greater than or equal to \spad{1}
whose squares are less than \spad{700}.
\end{xtccomment}
\begin{spadsrc}
[i for i in 1..   while i*i < 700]
\end{spadsrc}
\begin{TeXOutput}
\begin{fricasmath}{6}
\BRACKET{1\COMMA 2\COMMA 3\COMMA 4\COMMA 5\COMMA 6\COMMA 7\COMMA \STRING{...}%
}%
\end{fricasmath}
\end{TeXOutput}
\formatResultType{Stream(PositiveInteger)}
\end{xtc}

Got the idea?
Here is the general rule.
\index{collection}

\beginImportant
The general syntax of a collection is
\begin{center}
{\tt [ {\it collectExpression} \subscriptIt{iterator}{1}  \subscriptIt{iterator}{2}  \ldots  \subscriptIt{iterator}{N} ]}
\end{center}
where each \subscriptIt{iterator}{i} is either a \spad{for} or a
\spad{while} clause.
The loop terminates immediately when the end test of any
\subscriptIt{iterator}{i} succeeds or when a \spad{return} expression is
evaluated in {\it collectExpression}.
The value returned by the collection is either a list or a stream of
elements, one for each iteration of the {\it collectExpression}.
\endImportant

Be careful when you use \spad{while}
\index{stream!using while @{using {\tt while}}}
to create a stream.
By default, \Language{} tries to compute and display the first ten elements
of a stream.
If the \spad{while} condition is not satisfied quickly, \Language{}
can spend a long (possibly infinite) time trying to compute
\index{stream!number of elements computed}
the elements.
Use \spadsys{)set streams calculate} to change the default
to something else.
\syscmdindex{set streams calculate}
This also affects the number of terms computed and displayed for power
series.
For the purposes of this book, we have used this system
command to display fewer than ten terms.
\begin{xtc}
\begin{xtccomment}
Use nested iterators to create lists of
\index{iteration!nested}
lists which can then be given as an argument to \spadfun{matrix}.
\end{xtccomment}
\begin{spadsrc}
matrix [[x^i+j for i in 1..3] for j in 10..12]
\end{spadsrc}
\begin{TeXOutput}
\begin{fricasmath}{7}
\begin{MATRIX}{3}\SYMBOL{x}+10&\SUPER{\SYMBOL{x}}{2}+10&\SUPER{\SYMBOL{x}}{3}%
+10\\\SYMBOL{x}+11&\SUPER{\SYMBOL{x}}{2}+11&\SUPER{\SYMBOL{x}}{3}+11\\\SYMBOL%
{x}+12&\SUPER{\SYMBOL{x}}{2}+12&\SUPER{\SYMBOL{x}}{3}+12\end{MATRIX}%
\end{fricasmath}
\end{TeXOutput}
\formatResultType{Matrix(Polynomial(Integer))}
\end{xtc}
\begin{xtc}
\begin{xtccomment}
You can also create lists of streams, streams of lists and
streams of streams.
Here is a stream of streams.
\end{xtccomment}
\begin{spadsrc}
[[i/j for i in j+1..] for j in 1..]
\end{spadsrc}
\begin{TeXOutput}
\begin{fricasmath}{8}
\BRACKET{\BRACKET{2\COMMA 3\COMMA 4\COMMA 5\COMMA 6\COMMA 7\COMMA 8\COMMA %
\STRING{...}}\COMMA \BRACKET{\frac{3}{2}\COMMA 2\COMMA \frac{5}{2}\COMMA 3%
\COMMA \frac{7}{2}\COMMA 4\COMMA \frac{9}{2}\COMMA \STRING{...}}\COMMA %
\BRACKET{\frac{4}{3}\COMMA \frac{5}{3}\COMMA 2\COMMA \frac{7}{3}\COMMA \frac{%
8}{3}\COMMA 3\COMMA \frac{10}{3}\COMMA \STRING{...}}\COMMA \BRACKET{\frac{5}{%
4}\COMMA \frac{3}{2}\COMMA \frac{7}{4}\COMMA 2\COMMA \frac{9}{4}\COMMA \frac{%
5}{2}\COMMA \frac{11}{4}\COMMA \STRING{...}}\COMMA \BRACKET{\frac{6}{5}%
\COMMA \frac{7}{5}\COMMA \frac{8}{5}\COMMA \frac{9}{5}\COMMA 2\COMMA \frac{11%
}{5}\COMMA \frac{12}{5}\COMMA \STRING{...}}\COMMA \BRACKET{\frac{7}{6}\COMMA %
\frac{4}{3}\COMMA \frac{3}{2}\COMMA \frac{5}{3}\COMMA \frac{11}{6}\COMMA 2%
\COMMA \frac{13}{6}\COMMA \STRING{...}}\COMMA \BRACKET{\frac{8}{7}\COMMA %
\frac{9}{7}\COMMA \frac{10}{7}\COMMA \frac{11}{7}\COMMA \frac{12}{7}\COMMA %
\frac{13}{7}\COMMA 2\COMMA \STRING{...}}\COMMA \STRING{...}}%
\end{fricasmath}
\end{TeXOutput}
\formatResultType{Stream(Stream(Fraction(Integer)))}
\end{xtc}
\begin{xtc}
\begin{xtccomment}
You can use parallel iteration across lists and streams to create
\index{iteration!parallel}
new lists.
\end{xtccomment}
\begin{spadsrc}
[i/j for i in 3.. by 10 for j in 2..]
\end{spadsrc}
\begin{TeXOutput}
\begin{fricasmath}{9}
\BRACKET{\frac{3}{2}\COMMA \frac{13}{3}\COMMA \frac{23}{4}\COMMA \frac{33}{5}%
\COMMA \frac{43}{6}\COMMA \frac{53}{7}\COMMA \frac{63}{8}\COMMA \STRING{...}}%
\end{fricasmath}
\end{TeXOutput}
\formatResultType{Stream(Fraction(Integer))}
\end{xtc}
\begin{xtc}
\begin{xtccomment}
Iteration stops if the end of a list or stream is reached.
\end{xtccomment}
\begin{spadsrc}
[i^j for i in 1..7 for j in 2.. ]
\end{spadsrc}
\begin{TeXOutput}
\begin{fricasmath}{10}
\BRACKET{1\COMMA 8\COMMA 81\COMMA 1024\COMMA 15625\COMMA 279936\COMMA 5764801%
}%
\end{fricasmath}
\end{TeXOutput}
\formatResultType{Stream(Integer)}
\end{xtc}
%\xtc{
%or a while condition fails.
%}{
%\spadcommand{[i^j for i in 1..  for j in 2.. while i + j < 5 ]}
%}
\begin{xtc}
\begin{xtccomment}
As with loops, you can combine these modifiers to make very
complicated conditions.
\end{xtccomment}
\begin{spadsrc}
[[[i,j] for i in 10..15 | prime? i] for j in 17..22 | j = squareFreePart j]
\end{spadsrc}
\begin{TeXOutput}
\begin{fricasmath}{11}
\BRACKET{\BRACKET{\BRACKET{11\COMMA 17}\COMMA \BRACKET{13\COMMA 17}}\COMMA %
\BRACKET{\BRACKET{11\COMMA 19}\COMMA \BRACKET{13\COMMA 19}}\COMMA \BRACKET{%
\BRACKET{11\COMMA 21}\COMMA \BRACKET{13\COMMA 21}}\COMMA \BRACKET{\BRACKET{11%
\COMMA 22}\COMMA \BRACKET{13\COMMA 22}}}%
\end{fricasmath}
\end{TeXOutput}
\formatResultType{List(List(List(PositiveInteger)))}
\end{xtc}

See \xmpref{List} and \xmpref{Stream} for more information on creating and
manipulating lists and streams, respectively.

% *********************************************************************
\head{section}{An Example: Streams of Primes}{ugLangStreamsPrimes}
% *********************************************************************

We conclude this chapter with an example of the creation and manipulation
of infinite streams of prime integers.
This might be useful for experiments with numbers or other applications
where you are using sequences of primes over and over again.
As for all streams, the stream of primes is only computed as far out as you
need.
Once computed, however, all the primes up to that point are saved for
future reference.

Two useful operations provided by the \Language{} library are
\spadfunFrom{prime?}{IntegerPrimesPackage} and
\spadfunFrom{nextPrime}{IntegerPrimesPackage}.
A straight-forward way to create a stream of
prime numbers is to start with the stream of positive integers \spad{[2,..]} and
filter out those that are prime.
\begin{xtc}
\begin{xtccomment}
Create a stream of primes.
\end{xtccomment}
\begin{spadsrc}
primes : Stream Integer := [i for i in 2.. | prime? i]
\end{spadsrc}
\begin{TeXOutput}
\begin{fricasmath}{1}
\BRACKET{2\COMMA 3\COMMA 5\COMMA 7\COMMA 11\COMMA 13\COMMA 17\COMMA \STRING{%
...}}%
\end{fricasmath}
\end{TeXOutput}
\formatResultType{Stream(Integer)}
\end{xtc}
A more elegant way, however, is to use the \spadfunFrom{stream}{Stream}
operation from \spadtype{Stream}.
Given an initial value \spad{a} and a function \spad{f},
\spadfunFrom{stream}{Stream}
constructs the stream \spad{[a, f(a), f(f(a)), ...]}.
This function gives you the quickest method of getting the stream of primes.
\begin{xtc}
\begin{xtccomment}
This is how you use
\spadfunFrom{stream}{Stream} to
generate an infinite stream of primes.
\end{xtccomment}
\begin{spadsrc}
primes := stream(nextPrime,2)
\end{spadsrc}
\begin{TeXOutput}
\begin{fricasmath}{2}
\BRACKET{2\COMMA 3\COMMA 5\COMMA 7\COMMA 11\COMMA 13\COMMA 17\COMMA \STRING{%
...}}%
\end{fricasmath}
\end{TeXOutput}
\formatResultType{Stream(Integer)}
\end{xtc}
% decl seems necessary
\begin{xtc}
\begin{xtccomment}
Once the stream is generated, you might only be interested in
primes starting at a particular value.
\end{xtccomment}
\begin{spadsrc}
smallPrimes := [p for p in primes | p > 1000] 
\end{spadsrc}
\begin{TeXOutput}
\begin{fricasmath}{3}
\BRACKET{1009\COMMA 1013\COMMA 1019\COMMA 1021\COMMA 1031\COMMA 1033\COMMA %
1039\COMMA \STRING{...}}%
\end{fricasmath}
\end{TeXOutput}
\formatResultType{Stream(Integer)}
\end{xtc}
\begin{xtc}
\begin{xtccomment}
Here are the first 11 primes greater than 1000.
\end{xtccomment}
\begin{spadsrc}
[p for p in smallPrimes for i in 1..11] 
\end{spadsrc}
\begin{TeXOutput}
\begin{fricasmath}{4}
\BRACKET{1009\COMMA 1013\COMMA 1019\COMMA 1021\COMMA 1031\COMMA 1033\COMMA %
1039\COMMA \STRING{...}}%
\end{fricasmath}
\end{TeXOutput}
\formatResultType{Stream(Integer)}
\end{xtc}
\begin{xtc}
\begin{xtccomment}
Here is a stream of primes between 1000 and 1200.
\end{xtccomment}
\begin{spadsrc}
[p for p in smallPrimes while p < 1200] 
\end{spadsrc}
\begin{TeXOutput}
\begin{fricasmath}{5}
\BRACKET{1009\COMMA 1013\COMMA 1019\COMMA 1021\COMMA 1031\COMMA 1033\COMMA %
1039\COMMA \STRING{...}}%
\end{fricasmath}
\end{TeXOutput}
\formatResultType{Stream(Integer)}
\end{xtc}
\begin{xtc}
\begin{xtccomment}
To get these expanded into a finite stream,
you call \spadfunFrom{complete}{Stream} on the stream.
\end{xtccomment}
\begin{spadsrc}
complete %
\end{spadsrc}
\begin{TeXOutput}
\begin{fricasmath}{6}
\BRACKET{1009\COMMA 1013\COMMA 1019\COMMA 1021\COMMA 1031\COMMA 1033\COMMA %
1039\COMMA \STRING{...}}%
\end{fricasmath}
\end{TeXOutput}
\formatResultType{Stream(Integer)}
\end{xtc}
\begin{xtc}
\begin{xtccomment}
Twin primes are consecutive odd number pairs which are prime.
Here is the stream of twin primes.
\end{xtccomment}
\begin{spadsrc}
twinPrimes := [[p,p+2] for p in primes | prime?(p + 2)]
\end{spadsrc}
\begin{TeXOutput}
\begin{fricasmath}{7}
\BRACKET{\BRACKET{3\COMMA 5}\COMMA \BRACKET{5\COMMA 7}\COMMA \BRACKET{11%
\COMMA 13}\COMMA \BRACKET{17\COMMA 19}\COMMA \BRACKET{29\COMMA 31}\COMMA %
\BRACKET{41\COMMA 43}\COMMA \BRACKET{59\COMMA 61}\COMMA \STRING{...}}%
\end{fricasmath}
\end{TeXOutput}
\formatResultType{Stream(List(Integer))}
\end{xtc}
\begin{xtc}
\begin{xtccomment}
Since we already have the primes computed we can
avoid the call to \spadfunFrom{prime?}{IntegerPrimesPackage}
by using a double iteration.
This time we'll just generate a stream of the first of the twin primes.
\end{xtccomment}
\begin{spadsrc}
firstOfTwins:= [p for p in primes for q in rest primes | q=p+2]
\end{spadsrc}
\begin{TeXOutput}
\begin{fricasmath}{8}
\BRACKET{3\COMMA 5\COMMA 11\COMMA 17\COMMA 29\COMMA 41\COMMA 59\COMMA \STRING%
{...}}%
\end{fricasmath}
\end{TeXOutput}
\formatResultType{Stream(Integer)}
\end{xtc}

Let's try to compute the infinite stream of triplet primes,
the set of primes \spad{p} such that \spad{[p,p+2,p+4]}
are primes. For example, \spad{[3,5,7]} is a triple prime.
We could do this by a triple \spad{for} iteration.
A more economical way is to use \userfun{firstOfTwins}.
This time however, put a semicolon at the end of the line.

\begin{xtc}
\begin{xtccomment}
Put a semicolon at the end so that no
elements are computed.
\end{xtccomment}
\begin{spadsrc}
firstTriplets := [p for p in firstOfTwins for q in rest firstOfTwins | q = p+2];
\end{spadsrc}
\formatResultType{Stream(Integer)}
\end{xtc}

What happened?
As you know, by default
\Language{} displays the first ten
elements of a stream when you first display it.
And, therefore, it needs to compute them!
If you want {\it no} elements computed, just terminate the expression by a
semicolon (\spadSyntax{;}).\footnote{
Why does this happen? The semi-colon prevents the display of the
result of evaluating the expression.
Since no stream elements are needed for display (or anything else, so far),
none are computed.
}

\begin{xtc}
\begin{xtccomment}
Compute the first triplet prime.
\end{xtccomment}
\begin{spadsrc}
firstTriplets.1
\end{spadsrc}
\begin{TeXOutput}
\begin{fricasmath}{10}
3%
\end{fricasmath}
\end{TeXOutput}
\formatResultType{PositiveInteger}
\end{xtc}

If you want to compute another, just ask for it.
But wait a second!
Given three consecutive odd integers, one of them must be divisible
by 3. Thus there is only one triplet prime.
But suppose that you did not know this and wanted to know what was the
tenth triplet prime.
\begin{verbatim}
firstTriples.10
\end{verbatim}
To compute the tenth triplet prime, \Language{} first must compute the second,
the third, and so on.
But since there isn't even a second triplet prime, \Language{} will
compute forever.
Nonetheless, this effort can produce a useful result.
After waiting a bit, hit
\fbox{\bf Ctrl}--\fbox{\bf c}.
The system responds as follows.
\begin{verbatim}
   >> System error:
   Console interrupt.
   You are being returned to the top level of
   the interpreter.
\end{verbatim}
Let's say that you want to know how many primes have been computed.
Issue
\begin{verbatim}
numberOfComputedEntries primes
\end{verbatim}
and, for this discussion, let's say that the result is \spad{2045.}
\begin{xtc}
\begin{xtccomment}
How big is the \eth{2045} prime?
\end{xtccomment}
\begin{spadsrc}
primes.2045
\end{spadsrc}
\begin{TeXOutput}
\begin{fricasmath}{11}
17837%
\end{fricasmath}
\end{TeXOutput}
\formatResultType{PositiveInteger}
\end{xtc}

What you have learned is that there are no triplet primes between 5 and
17837.
Although this result is well known (some might even say trivial), there
are many experiments you could make where the result is not known.
What you see here is a paradigm for testing of hypotheses.
Here our hypothesis could have been: ``there is more than one triplet
prime.''
We have tested this hypothesis for 17837 cases.
With streams, you can let your machine run, interrupt it to see how far
it has progressed,
then start it up and let it continue from where it left off.

%> RDJ note to RSS:
%> Expressions not statements or lines--
%>   By an expression I mean any syntactically correct program fragment.
%>   Everything in AXIOM is an expression since every fragment has a value and a type.
%>   In most languages including LISP, a "statement" is different from an expression:
%>   it is executed for side-effect only and an error is incurred if you assign it a value.
%>   This "gimmick" takes care of incomplete expressions such as "if x > 0 then y" in blocks.
%>   In LISP, "u := (if x > 0 then y)" is illegal but in AXIOM it is legal.
%>   Also, in AXIOM the value of a repeat loop is void even though you might be
%>   be able to prove that it always returns a valid value (you have an example of this)!
%>   This will be considered a bug not a feature. But it is how things stand.
%>   In any case---this point should be in a box somewhere since it is key
%>   to a user's understanding to the language. I am not sure where. You only
%>   gain an appreciation for it after are awhile in chapter 5.

% !! DO NOT MODIFY THIS FILE BY HAND !! Created by spool2tex.awk.

% Copyright (c) 1991-2002, The Numerical ALgorithms Group Ltd.
% All rights reserved.
%
% Redistribution and use in source and binary forms, with or without
% modification, are permitted provided that the following conditions are
% met:
%
%     - Redistributions of source code must retain the above copyright
%       notice, this list of conditions and the following disclaimer.
%
%     - Redistributions in binary form must reproduce the above copyright
%       notice, this list of conditions and the following disclaimer in
%       the documentation and/or other materials provided with the
%       distribution.
%
%     - Neither the name of The Numerical ALgorithms Group Ltd. nor the
%       names of its contributors may be used to endorse or promote products
%       derived from this software without specific prior written permission.
%
% THIS SOFTWARE IS PROVIDED BY THE COPYRIGHT HOLDERS AND CONTRIBUTORS "AS
% IS" AND ANY EXPRESS OR IMPLIED WARRANTIES, INCLUDING, BUT NOT LIMITED
% TO, THE IMPLIED WARRANTIES OF MERCHANTABILITY AND FITNESS FOR A
% PARTICULAR PURPOSE ARE DISCLAIMED. IN NO EVENT SHALL THE COPYRIGHT OWNER
% OR CONTRIBUTORS BE LIABLE FOR ANY DIRECT, INDIRECT, INCIDENTAL, SPECIAL,
% EXEMPLARY, OR CONSEQUENTIAL DAMAGES (INCLUDING, BUT NOT LIMITED TO,
% PROCUREMENT OF SUBSTITUTE GOODS OR SERVICES-- LOSS OF USE, DATA, OR
% PROFITS-- OR BUSINESS INTERRUPTION) HOWEVER CAUSED AND ON ANY THEORY OF
% LIABILITY, WHETHER IN CONTRACT, STRICT LIABILITY, OR TORT (INCLUDING
% NEGLIGENCE OR OTHERWISE) ARISING IN ANY WAY OUT OF THE USE OF THIS
% SOFTWARE, EVEN IF ADVISED OF THE POSSIBILITY OF SUCH DAMAGE.


% *********************************************************************
\head{chapter}{User-Defined Functions, Macros and Rules}{ugUser}
% *********************************************************************

In this chapter we show you how to write functions and macros,
and we explain how \Language{} looks for and applies them.
We show some simple one-line examples of functions, together
with larger ones that are defined piece-by-piece or through the use of
piles.

% *********************************************************************
\head{section}{Functions vs. Macros}{ugUserFunMac}
% *********************************************************************

A function is a program to perform some
\index{function!vs. macro}
computation.
\index{macro!vs. function}
Most functions have names so that it is easy to refer to them.
A simple example of a function is one named
\spadfunFrom{abs}{Integer} which
computes the absolute value of an integer.
%
\begin{xtc}
\begin{xtccomment}
This is a use of the ``absolute value'' library function for integers.
\end{xtccomment}
\begin{spadsrc}
abs(-8)
\end{spadsrc}
\begin{TeXOutput}
\begin{fricasmath}{1}
8%
\end{fricasmath}
\end{TeXOutput}
\formatResultType{PositiveInteger}
\end{xtc}
\begin{xtc}
\begin{xtccomment}
This is an unnamed function that does the same thing, using the
``maps-to'' syntax \spadSyntax{+->} that we discuss in
\spadref{ugUserAnon}.
\end{xtccomment}
\begin{spadsrc}
(x +-> if x < 0 then -x else x)(-8)
\end{spadsrc}
\begin{TeXOutput}
\begin{fricasmath}{2}
8%
\end{fricasmath}
\end{TeXOutput}
\formatResultType{PositiveInteger}
\end{xtc}
%
Functions can be used alone or serve as the building blocks for larger
programs.
Usually they return a value that you might want to use in the next stage
of a computation, but not always (for example, see
\xmpref{Exit} and \xmpref{Void}).
They may also read data from your keyboard, move information from one
place to another, or format and display results on your screen.

In \Language{}, as in mathematics, functions
\index{function!parameters}
are usually \spadglossSee{parameterized}{parameterized form}.
Each time you {\it call} (some people say \spadgloss{apply} or
\spadglossSee{invoke}{invocation}) a function, you give
\index{parameters to a function}
values to the parameters (variables).
Such a value is called an \spadgloss{argument} of
\index{function!arguments}
the function.
\Language{} uses the arguments for the computation.
In this way you get different results depending on what you ``feed'' the
function.

Functions can have local variables or refer to global variables in the
workspace.
\Language{} can often \spadglossSee{compile}{compiler} functions so that
they execute very efficiently.
Functions can be passed as arguments to other functions.

Macros are textual substitutions.
They are used to clarify the meaning of constants or expressions and to be
templates for frequently used expressions.
Macros can be parameterized but they are not objects that can be passed as
arguments to functions.
In effect, macros are extensions to the \Language{} expression parser.

% *********************************************************************
\head{section}{Macros}{ugUserMacros}
% *********************************************************************

A \spadgloss{macro} provides general textual substitution of
\index{macro}
an \Language{} expression for a name.
You can think of a macro as being a generalized abbreviation.
You can only have one macro in your workspace with
a given name, no matter how many arguments it has.

% ----------------------------------------------------------------------
\beginImportant
The two general forms for macros are
\begin{center}
{\tt macro} {\it name} {\tt ==} {\it body} \\
{\tt macro} {\it name(arg1,...)} {\tt ==} {\it body}
\end{center}
where the body of the macro can be any \Language{} expression.
\endImportant
% ----------------------------------------------------------------------

%
\begin{xtc}
\begin{xtccomment}
For example, suppose you decided that you
like to use \spad{df} for \spadfun{D}.
You define the macro \spad{df} like this.
\end{xtccomment}
\begin{spadsrc}
macro df == D 
\end{spadsrc}
\end{xtc}
\begin{xtc}
\begin{xtccomment}
Whenever you type \spad{df}, the system expands it to
\spadfun{D}.
\end{xtccomment}
\begin{spadsrc}
df(x^2 + x + 1,x) 
\end{spadsrc}
\begin{TeXOutput}
\begin{fricasmath}{2}
2\TIMES \SYMBOL{x}+1%
\end{fricasmath}
\end{TeXOutput}
\formatResultType{Polynomial(Integer)}
\end{xtc}
\begin{xtc}
\begin{xtccomment}
Macros can be parameterized and so can be used for many different
kinds of objects.
\end{xtccomment}
\begin{spadsrc}
macro ff(x) == x^2 + 1 
\end{spadsrc}
\end{xtc}
\begin{xtc}
\begin{xtccomment}
Apply it to a number, a symbol, or an expression.
\end{xtccomment}
\begin{spadsrc}
ff z 
\end{spadsrc}
\begin{TeXOutput}
\begin{fricasmath}{4}
\SUPER{\SYMBOL{z}}{2}+1%
\end{fricasmath}
\end{TeXOutput}
\formatResultType{Polynomial(Integer)}
\end{xtc}
\begin{xtc}
\begin{xtccomment}
Macros can also be nested, but you get an error message if you
run out of space because of an infinite nesting loop.
\end{xtccomment}
\begin{spadsrc}
macro gg(x) == ff(2*x - 2/3) 
\end{spadsrc}
\end{xtc}
\begin{xtc}
\begin{xtccomment}
This new macro is fine as it does not produce a loop.
\end{xtccomment}
\begin{spadsrc}
gg(1/w) 
\end{spadsrc}
\begin{TeXOutput}
\begin{fricasmath}{6}
\frac{13\TIMES \SUPER{\SYMBOL{w}}{2}-{24\TIMES \SYMBOL{w}}+36}{9\TIMES \SUPER%
{\SYMBOL{w}}{2}}%
\end{fricasmath}
\end{TeXOutput}
\formatResultType{Fraction(Polynomial(Integer))}
\end{xtc}
%
\begin{xtc}
\begin{xtccomment}
This, however, loops since \spad{gg} is
defined in terms of \spad{ff}.
\end{xtccomment}
\begin{spadsrc}
macro ff(x) == gg(-x) 
\end{spadsrc}
\end{xtc}
\begin{xtc}
\begin{xtccomment}
The body of a macro can be a block.
\end{xtccomment}
\begin{spadsrc}
macro next == (past := present; present := future; future := past + present) 
\end{spadsrc}
\end{xtc}
\begin{xtc}
\begin{xtccomment}
Before entering \spad{next}, we need
values for \spad{present} and \spad{future}.
\end{xtccomment}
\begin{spadsrc}
present : Integer := 0 
\end{spadsrc}
\begin{TeXOutput}
\begin{fricasmath}{9}
0%
\end{fricasmath}
\end{TeXOutput}
\formatResultType{Integer}
\end{xtc}
\begin{xtc}
\begin{xtccomment}
\end{xtccomment}
\begin{spadsrc}
future : Integer := 1 
\end{spadsrc}
\begin{TeXOutput}
\begin{fricasmath}{10}
1%
\end{fricasmath}
\end{TeXOutput}
\formatResultType{Integer}
\end{xtc}
\begin{xtc}
\begin{xtccomment}
Repeatedly evaluating \spad{next} produces the next Fibonacci number.
\end{xtccomment}
\begin{spadsrc}
next 
\end{spadsrc}
\begin{TeXOutput}
\begin{fricasmath}{11}
1%
\end{fricasmath}
\end{TeXOutput}
\formatResultType{Integer}
\end{xtc}
\begin{xtc}
\begin{xtccomment}
And the next one.
\end{xtccomment}
\begin{spadsrc}
next 
\end{spadsrc}
\begin{TeXOutput}
\begin{fricasmath}{12}
2%
\end{fricasmath}
\end{TeXOutput}
\formatResultType{Integer}
\end{xtc}
\begin{xtc}
\begin{xtccomment}
Here is the infinite stream of the rest of the Fibonacci numbers.
\end{xtccomment}
\begin{spadsrc}
[next for i in 1..] 
\end{spadsrc}
\begin{TeXOutput}
\begin{fricasmath}{13}
\BRACKET{3\COMMA 5\COMMA 8\COMMA 13\COMMA 21\COMMA 34\COMMA 55\COMMA \STRING{%
...}}%
\end{fricasmath}
\end{TeXOutput}
\formatResultType{Stream(Integer)}
\end{xtc}
\begin{xtc}
\begin{xtccomment}
Bundle all the above lines into a single macro.
\end{xtccomment}
\begin{spadsrc}
macro fibStream ==
  present : Integer := 1
  future : Integer := 1
  [next for i in 1..] where
    macro next ==
      past := present
      present := future
      future := past + present
\end{spadsrc}
\end{xtc}
\begin{xtc}
\begin{xtccomment}
Use \spadfunFrom{concat}{Stream} to start with the first two
\index{Fibonacci numbers}
Fibonacci numbers.
\end{xtccomment}
\begin{spadsrc}
concat([0,1],fibStream) 
\end{spadsrc}
\begin{TeXOutput}
\begin{fricasmath}{15}
\BRACKET{0\COMMA 1\COMMA 2\COMMA 3\COMMA 5\COMMA 8\COMMA 13\COMMA \STRING{...%
}}%
\end{fricasmath}
\end{TeXOutput}
\formatResultType{Stream(Integer)}
\end{xtc}
\begin{xtc}
\begin{xtccomment}
An easier way to compute these numbers is to
use the library operation \spadfun{fibonacci}.
\end{xtccomment}
\begin{spadsrc}
[fibonacci i for i in 1..]
\end{spadsrc}
\begin{TeXOutput}
\begin{fricasmath}{16}
\BRACKET{1\COMMA 1\COMMA 2\COMMA 3\COMMA 5\COMMA 8\COMMA 13\COMMA \STRING{...%
}}%
\end{fricasmath}
\end{TeXOutput}
\formatResultType{Stream(Integer)}
\end{xtc}

% *********************************************************************
\head{section}{Introduction to Functions}{ugUserIntro}
% *********************************************************************

Each name in your workspace can refer to a single object.
This may be any kind of object including a function.
You can use interactively any function from the library or any that you
define in the workspace.
In the library the same name can have very many functions, but you
can have only one function with a given name, although it can have any
number of arguments that you choose.

If you define a function in the workspace that has the same name and number
of arguments as one in the library, then your definition takes precedence.
In fact, to get the library function you must
\spadglossSee{package-call}{package call} it (see \spadref{ugTypesPkgCall}).

To use a function in \Language{}, you apply it to its arguments.
Most functions are applied by entering the name of the function followed by
its argument or arguments.
\begin{xtc}
\begin{xtccomment}
\end{xtccomment}
\begin{spadsrc}
factor(12)
\end{spadsrc}
\begin{TeXOutput}
\begin{fricasmath}{1}
\SUPER{2}{2}\TIMES 3%
\end{fricasmath}
\end{TeXOutput}
\formatResultType{Factored(Integer)}
\end{xtc}
%
\begin{xtc}
\begin{xtccomment}
Some functions like \spadop{+} have {\it infix} \spadgloss{operators} as names.
\end{xtccomment}
\begin{spadsrc}
3 + 4
\end{spadsrc}
\begin{TeXOutput}
\begin{fricasmath}{2}
7%
\end{fricasmath}
\end{TeXOutput}
\formatResultType{PositiveInteger}
\end{xtc}
\begin{xtc}
\begin{xtccomment}
The function \spadop{+} has two arguments.
When you give it more than two arguments,
\Language{} groups the arguments to the left.
This expression is equivalent to \spad{(1 + 2) + 7}.
\end{xtccomment}
\begin{spadsrc}
1 + 2 + 7
\end{spadsrc}
\begin{TeXOutput}
\begin{fricasmath}{3}
10%
\end{fricasmath}
\end{TeXOutput}
\formatResultType{PositiveInteger}
\end{xtc}

All operations, including infix operators, can be written in prefix form,
that is, with the operation name followed by the arguments
in parentheses.
For example, \spad{2 + 3} can alternatively be written as \spad{+(2,3)}.
But \spad{+(2,3,4)} is an error since \spadop{+}
takes only two arguments.

Prefix operations are generally applied before the infix operation.
Thus \spad{factorial 3 + 1} means \spad{factorial(3) + 1} producing
\spad{7}, and
\spad{- 2 + 5} means \spad{(-2) + 5} producing \spad{3}.
An example of a prefix operator is prefix \spadop{-}.
For example, \spad{- 2 + 5} converts to \spad{(- 2) + 5} producing
the value \spad{3}.
Any prefix function taking two arguments can be written in
an infix manner by putting an
ampersand (\spadSyntax{&}) before the name.
Thus \spad{D(2*x,x)} can be written as
\spad{2*x &D x} returning \spad{2}.

Every function in \Language{} is identified by
a \spadgloss{name} and \spadgloss{type}.\footnote{An exception is
an ``anonymous function''
discussed in
\spadref{ugUserAnon}.}
The type of a function is always a mapping of the form
\spad{Source->Target}
where \spad{Source} and \spad{Target} are types.
To enter a type from the keyboard, enter the arrow by using
a hyphen \spadSyntax{-} followed by a greater-than sign
\spadSyntax{>}, e.g. {\tt Integer -> Integer}.

Let's go back to \spadop{+}.
There are many \spadop{+} functions in the
\Language{} library: one for integers, one for floats, another for
rational numbers, and so on.
These \spadop{+} functions have different types and thus are
different functions.
You've seen examples of this \spadgloss{overloading}
before---using the same name for different functions.
Overloading is the rule rather than the exception.
You can add two integers, two polynomials, two matrices or
two power series.
These are all done with the same function name
but with different functions.

% *********************************************************************
\head{section}{Declaring the Type of Functions}{ugUserDeclare}
% *********************************************************************

In \spadref{ugTypesDeclare} we discussed how to declare a variable
to restrict the kind of values that can be assigned to it.
In this section we show how to declare a variable that refers to
function objects.

% ----------------------------------------------------------------------
\beginImportant
A function is an object of type
\begin{center}
\spad{Source->Type}
\end{center}
where \spad{Source} and \spad{Target} can be any type.
A common type for \spad{Source} is
\spadtype{Tuple}(\subscriptIt{T}{1}, \ldots, \subscriptIt{T}{n}),
usually written
(\subscriptIt{T}{1}, \ldots, \subscriptIt{T}{n}),
to indicate a function of \spad{n} arguments.
\endImportant
% ----------------------------------------------------------------------

\begin{xtc}
\begin{xtccomment}
If \spad{g} takes an \spadtype{Integer}, a \spadtype{Float} and
another \spadtype{Integer}, and returns a
\spadtype{String}, the declaration is written this way.
\end{xtccomment}
\begin{spadsrc}
g: (Integer,Float,Integer) -> String
\end{spadsrc}
\end{xtc}
\begin{xtc}
\begin{xtccomment}
The types need not be written fully; using abbreviations, the above
declaration is:
\end{xtccomment}
\begin{spadsrc}
g: (INT,FLOAT,INT) -> STRING
\end{spadsrc}
\end{xtc}
\begin{xtc}
\begin{xtccomment}
It is possible for a function to take no arguments.
If \spad{ h} takes no arguments
but returns a \spadtype{Polynomial} \spadtype{Integer}, any
of the following declarations is acceptable.
\end{xtccomment}
\begin{spadsrc}
h: () -> POLY INT
\end{spadsrc}
\end{xtc}
\begin{xtc}
\begin{xtccomment}
\end{xtccomment}
\begin{spadsrc}
h: () -> Polynomial INT
\end{spadsrc}
\end{xtc}
\begin{xtc}
\begin{xtccomment}
\end{xtccomment}
\begin{spadsrc}
h: () -> POLY Integer
\end{spadsrc}
\end{xtc}


\beginImportant
Functions can also be declared when they are being defined.
The syntax for combined declaration/definition is:
\begin{center}
\frenchspacing{\tt {\it functionName}(\subscriptIt{parm}{1}: \subscriptIt{parmType}{1}, \ldots, \subscriptIt{parm}{N}: \subscriptIt{parmType}{N}): {\it functionReturnType}}
\end{center}
\endImportant

The following definition fragments show how this can be done for
the functions \spad{g} and \spad{h} above.
\begin{verbatim}
g(arg1: INT, arg2: FLOAT, arg3: INT): STRING == ...

h(): POLY INT == ...
\end{verbatim}

A current restriction on function declarations is that they must
involve fully specified types (that is, cannot include modes involving
explicit or implicit \spadSyntax{?}).
For more information on declaring things in general, see
\spadref{ugTypesDeclare}.

% *********************************************************************
\head{section}{One-Line Functions}{ugUserOne}
% *********************************************************************

As you use \Language{}, you will find that you will write many short functions
\index{function!one-line definition}
to codify sequences of operations that you often perform.
In this section we write some simple one-line functions.

\begin{xtc}
\begin{xtccomment}
This is a simple recursive factorial function for positive integers.
\end{xtccomment}
\begin{spadsrc}
fac n == if n < 3 then n else n * fac(n-1) 
\end{spadsrc}
\end{xtc}
\begin{xtc}
\begin{xtccomment}
\end{xtccomment}
\begin{spadsrc}
fac 10 
\end{spadsrc}
\begin{MessageOutput}
   Compiling function fac with type Integer -> Integer 
\end{MessageOutput}
\begin{TeXOutput}
\begin{fricasmath}{2}
3628800%
\end{fricasmath}
\end{TeXOutput}
\formatResultType{PositiveInteger}
\end{xtc}
%>> Thankfully, the $ is no longer needed in the next example.
\begin{xtc}
\begin{xtccomment}
This function computes \spad{1 + 1/2 + 1/3 + ... + 1/n}.
\end{xtccomment}
\begin{spadsrc}
s n == reduce(+,[1/i for i in 1..n]) 
\end{spadsrc}
\end{xtc}
\begin{xtc}
\begin{xtccomment}
\end{xtccomment}
\begin{spadsrc}
s 50 
\end{spadsrc}
\begin{MessageOutput}
   Compiling function s with type PositiveInteger -> Fraction(Integer) 
\end{MessageOutput}
\begin{TeXOutput}
\begin{fricasmath}{4}
\frac{13943237577224054960 759}{30990445042459967064 00}%
\end{fricasmath}
\end{TeXOutput}
\formatResultType{Fraction(Integer)}
\end{xtc}
\begin{xtc}
\begin{xtccomment}
This function computes a Mersenne number, several of which are prime.
\index{Mersenne number}
\end{xtccomment}
\begin{spadsrc}
mersenne i == 2^i - 1 
\end{spadsrc}
\end{xtc}
\begin{xtc}
\begin{xtccomment}
If you type \spad{mersenne}, \Language{} shows you
the function definition.
\end{xtccomment}
\begin{spadsrc}
mersenne 
\end{spadsrc}
\begin{TeXOutput}
\begin{fricasmath}{6}
\SYMBOL{mersenne}\ \SYMBOL{i}\ \SYMBOL{==}\ \SUPER{2}{\SYMBOL{i}}-{1}%
\end{fricasmath}
\end{TeXOutput}
\formatResultType{FunctionCalled(mersenne)}
\end{xtc}
\begin{xtc}
\begin{xtccomment}
Generate a stream of Mersenne numbers.
\end{xtccomment}
\begin{spadsrc}
[mersenne i for i in 1..] 
\end{spadsrc}
\begin{MessageOutput}
   Compiling function mersenne with type PositiveInteger -> Integer 
\end{MessageOutput}
\begin{TeXOutput}
\begin{fricasmath}{7}
\BRACKET{1\COMMA 3\COMMA 7\COMMA 15\COMMA 31\COMMA 63\COMMA 127\COMMA \STRING%
{...}}%
\end{fricasmath}
\end{TeXOutput}
\formatResultType{Stream(Integer)}
\end{xtc}
\begin{xtc}
\begin{xtccomment}
Create a stream of those values of \spad{i} such that
\spad{mersenne(i)} is prime.
\end{xtccomment}
\begin{spadsrc}
mersenneIndex := [n for n in 1.. | prime?(mersenne(n))] 
\end{spadsrc}
\begin{TeXOutput}
\begin{fricasmath}{8}
\BRACKET{2\COMMA 3\COMMA 5\COMMA 7\COMMA 13\COMMA 17\COMMA 19\COMMA \STRING{%
...}}%
\end{fricasmath}
\end{TeXOutput}
\formatResultType{Stream(PositiveInteger)}
\end{xtc}
\begin{xtc}
\begin{xtccomment}
Finally, write a function that returns the \eth{n} Mersenne
prime.
\end{xtccomment}
\begin{spadsrc}
mersennePrime n == mersenne mersenneIndex(n) 
\end{spadsrc}
\end{xtc}
\begin{xtc}
\begin{xtccomment}
\end{xtccomment}
\begin{spadsrc}
mersennePrime 5 
\end{spadsrc}
\begin{MessageOutput}
   Compiling function mersennePrime with type PositiveInteger -> 
      Integer 
\end{MessageOutput}
\begin{TeXOutput}
\begin{fricasmath}{10}
8191%
\end{fricasmath}
\end{TeXOutput}
\formatResultType{PositiveInteger}
\end{xtc}

% *********************************************************************
\head{section}{Declared vs. Undeclared Functions}{ugUserDecUndec}
% *********************************************************************

If you declare the type of a function, you can apply
it to any data that can be converted to the source type
of the function.

\begin{xtc}
\begin{xtccomment}
Define \userfun{f} with type \spad{Integer->Integer}.
\end{xtccomment}
\begin{spadsrc}
f(x: Integer): Integer == x + 1 
\end{spadsrc}
\begin{MessageOutput}
   Function declaration f : Integer -> Integer has been added to 
      workspace.
\end{MessageOutput}
\end{xtc}
\begin{xtc}
\begin{xtccomment}
The function
\userfun{f} can be applied to integers, \ldots
\end{xtccomment}
\begin{spadsrc}
f 9 
\end{spadsrc}
\begin{MessageOutput}
   Compiling function f with type Integer -> Integer 
\end{MessageOutput}
\begin{TeXOutput}
\begin{fricasmath}{2}
10%
\end{fricasmath}
\end{TeXOutput}
\formatResultType{PositiveInteger}
\end{xtc}
\begin{xtc}
\begin{xtccomment}
and to values that convert to integers, \ldots
\end{xtccomment}
\begin{spadsrc}
f(-2.0) 
\end{spadsrc}
\begin{TeXOutput}
\begin{fricasmath}{3}
-{1}%
\end{fricasmath}
\end{TeXOutput}
\formatResultType{Integer}
\end{xtc}
\begin{xtc}
\begin{xtccomment}
but not to values that cannot be converted to integers.
\end{xtccomment}
\begin{spadsrc}
f(2/3) 
\end{spadsrc}
\begin{MessageOutput}
   Conversion failed in the compiled user function f .
\end{MessageOutput}
\begin{MessageOutput}
   Cannot convert the value from type Fraction(Integer) to Integer .
\end{MessageOutput}
\end{xtc}

To make the function over a wide range of types, do not
declare its type.
\begin{xtc}
\begin{xtccomment}
Give the same definition with no declaration.
\end{xtccomment}
\begin{spadsrc}
g x == x + 1 
\end{spadsrc}
\end{xtc}
\begin{xtc}
\begin{xtccomment}
If \spad{x + 1} makes sense, you can apply \userfun{g} to \spad{x}.
\end{xtccomment}
\begin{spadsrc}
g 9 
\end{spadsrc}
\begin{MessageOutput}
   Compiling function g with type PositiveInteger -> PositiveInteger 
\end{MessageOutput}
\begin{TeXOutput}
\begin{fricasmath}{5}
10%
\end{fricasmath}
\end{TeXOutput}
\formatResultType{PositiveInteger}
\end{xtc}
\begin{xtc}
\begin{xtccomment}
A version of \userfun{g} with different argument types
get compiled for each new kind of argument used.
\end{xtccomment}
\begin{spadsrc}
g(2/3)  
\end{spadsrc}
\begin{MessageOutput}
   Compiling function g with type Fraction(Integer) -> Fraction(Integer
      ) 
\end{MessageOutput}
\begin{TeXOutput}
\begin{fricasmath}{6}
\frac{5}{3}%
\end{fricasmath}
\end{TeXOutput}
\formatResultType{Fraction(Integer)}
\end{xtc}
\begin{xtc}
\begin{xtccomment}
Here \spad{x+1} for \spad{x = "axiom"} makes no sense.
\end{xtccomment}
\begin{spadsrc}
g("axiom")
\end{spadsrc}
\begin{MessageOutput}
   There are 14 exposed and 9 unexposed library operations named + 
      having 2 argument(s) but none was determined to be applicable. 
      Use HyperDoc Browse, or issue
                                )display op +
      to learn more about the available operations. Perhaps 
      package-calling the operation or using coercions on the arguments
      will allow you to apply the operation.
\end{MessageOutput}
\begin{MessageOutput}
   Cannot find a definition or applicable library operation named + 
      with argument type(s) 
                                   String
                               PositiveInteger
      
      Perhaps you should use "@" to indicate the required return type, 
      or "$" to specify which version of the function you need.
\end{MessageOutput}
\begin{MessageOutput}
   FriCAS will attempt to step through and interpret the code.
\end{MessageOutput}
\begin{MessageOutput}
   There are 14 exposed and 9 unexposed library operations named + 
      having 2 argument(s) but none was determined to be applicable. 
      Use HyperDoc Browse, or issue
                                )display op +
      to learn more about the available operations. Perhaps 
      package-calling the operation or using coercions on the arguments
      will allow you to apply the operation.
\end{MessageOutput}
\begin{MessageOutput}
   Cannot find a definition or applicable library operation named + 
      with argument type(s) 
                                   String
                               PositiveInteger
      
      Perhaps you should use "@" to indicate the required return type, 
      or "$" to specify which version of the function you need.
\end{MessageOutput}
\end{xtc}

As you will see in \chapref{ugCategories},
\Language{} has a formal idea of categories for what ``makes sense.''

% *********************************************************************
\head{section}{Functions vs. Operations}{ugUserDecOpers}
% *********************************************************************

A function is an object that you can create, manipulate, pass to,
and return from functions (for some interesting examples of
library functions that manipulate functions, see
\xmpref{MappingPackage1}).
Yet, we often seem to use the term \spadgloss{operation} and
function interchangeably in \Language{}.
What is the distinction?

First consider values and types associated with some variable \spad{n} in
your workspace.
You can make the declaration \spad{n : Integer}, then assign \spad{n} an
integer value.
You then speak of the integer \spad{n}.
However, note that the integer is not the name \spad{n} itself, but
the value that you assign to \spad{n}.

Similarly, you can declare a variable \spad{f} in your workspace to have
type \spad{Integer->Integer}, then assign \spad{f}, through a definition
or an assignment of an anonymous function.
You then speak of the function \spad{f}.
However, the function is not \spad{f}, but the value that you
assign to \spad{f}.

A function is a value, in fact, some machine code for doing something.
Doing what?
Well, performing some \spadgloss{operation}.
Formally, an operation consists of the constituent parts of \spad{f} in your
workspace, excluding the value; thus an operation has a name and a type.
An operation is what domains and packages export.
Thus \spadtype{Ring} exports one operation \spadop{+}.
Every ring also exports this operation.
Also, the author of every ring in the system is obliged under contract
(see \spadref{ugPackagesAbstract})
to provide an implementation for this operation.

This chapter is all about functions---how you create them interactively and
how you apply them to meet your needs.
In \chapref{ugPackages} you will learn how to create them for the
\Language{} library.
Then in \chapref{ugCategories}, you will learn about categories and
exported operations.

% *********************************************************************
\head{section}{Delayed Assignments vs. Functions with No Arguments}{ugUserDelay}
% *********************************************************************

In \spadref{ugLangAssign} we discussed the difference between immediate and
\index{function!with no arguments}
delayed assignments.
In this section we show the difference between delayed
assignments and functions of no arguments.

\begin{xtc}
\begin{xtccomment}
A function of no arguments is sometimes called a {\it nullary function.}
\end{xtccomment}
\begin{spadsrc}
sin24() == sin(24.0) 
\end{spadsrc}
\end{xtc}
\begin{xtc}
\begin{xtccomment}
You must use the parentheses (\spadSyntax{()}) to evaluate it.
Like a delayed assignment, the right-hand-side of a function evaluation
is not evaluated until the left-hand-side is used.
\end{xtccomment}
\begin{spadsrc}
sin24() 
\end{spadsrc}
\begin{MessageOutput}
   Compiling function sin24 with type () -> Float 
\end{MessageOutput}
\begin{TeXOutput}
\begin{fricasmath}{2}
-{\STRING{0.90557836200662384514}}%
\end{fricasmath}
\end{TeXOutput}
\formatResultType{Float}
\end{xtc}
\begin{xtc}
\begin{xtccomment}
If you omit the parentheses, you just get the function definition.
%(Note how the explicit floating point number in the definition
%has been translated into a function call involving a mantissa,
%exponent and radix.)
\end{xtccomment}
\begin{spadsrc}
sin24 
\end{spadsrc}
\begin{TeXOutput}
\begin{fricasmath}{3}
\SYMBOL{sin24}\ \PAREN{}\ \SYMBOL{==}\ \sin{\STRING{24.0}}%
\end{fricasmath}
\end{TeXOutput}
\formatResultType{FunctionCalled(sin24)}
\end{xtc}
\begin{xtc}
\begin{xtccomment}
You do not use the parentheses \spadSyntax{()} in a delayed assignment\ldots
\end{xtccomment}
\begin{spadsrc}
cos24 == cos(24.0) 
\end{spadsrc}
\end{xtc}
\begin{xtc}
\begin{xtccomment}
nor in the evaluation.
\end{xtccomment}
\begin{spadsrc}
cos24 
\end{spadsrc}
\begin{MessageOutput}
   Compiling body of rule cos24 to compute value of type Float 
\end{MessageOutput}
\begin{TeXOutput}
\begin{fricasmath}{5}
\STRING{0.42417900733699697594}%
\end{fricasmath}
\end{TeXOutput}
\formatResultType{Float}
\end{xtc}
The only syntactic difference between delayed assignments
and nullary functions is that you use \spadSyntax{()} in the latter case.

% *********************************************************************
\head{section}{How \Language{} Determines What Function to Use}{ugUserUse}
% *********************************************************************

What happens if you define a function that has the same name as a library
function?
Well, if your function has the same name and number of arguments (we
sometimes say \spadgloss{arity}) as another function
in the library, then your function covers up the library function.
If you want then to call the library function, you will have to package-call
it.
\Language{} can use both the functions you write and those that come
from the library.
Let's do a simple example to illustrate this.
\begin{xtc}
\begin{xtccomment}
Suppose you (wrongly!) define \userfun{sin} in this way.
\end{xtccomment}
\begin{spadsrc}
sin x == 1.0 
\end{spadsrc}
\end{xtc}
\begin{xtc}
\begin{xtccomment}
The value \spad{1.0} is returned for any argument.
\end{xtccomment}
\begin{spadsrc}
sin 4.3 
\end{spadsrc}
\begin{MessageOutput}
   Compiling function sin with type Float -> Float 
\end{MessageOutput}
\begin{TeXOutput}
\begin{fricasmath}{2}
\STRING{1.0}%
\end{fricasmath}
\end{TeXOutput}
\formatResultType{Float}
\end{xtc}
\begin{xtc}
\begin{xtccomment}
If you want the library operation, we have to package-call it
(see \spadref{ugTypesPkgCall}
for more information).
\end{xtccomment}
\begin{spadsrc}
sin(4.3)$Float
\end{spadsrc}
\begin{TeXOutput}
\begin{fricasmath}{3}
-{\STRING{0.91616593674945498404}}%
\end{fricasmath}
\end{TeXOutput}
\formatResultType{Float}
\end{xtc}
\begin{xtc}
\begin{xtccomment}
\end{xtccomment}
\begin{spadsrc}
sin(34.6)$Float
\end{spadsrc}
\begin{TeXOutput}
\begin{fricasmath}{4}
-{\STRING{0.042468034716950101543}}%
\end{fricasmath}
\end{TeXOutput}
\formatResultType{Float}
\end{xtc}
\begin{xtc}
\begin{xtccomment}
Even worse, say we accidentally used the same name as a library
function in the function.
\end{xtccomment}
\begin{spadsrc}
sin x == sin x 
\end{spadsrc}
\begin{MessageOutput}
   Compiled code for sin has been cleared.
\end{MessageOutput}
\begin{MessageOutput}
   1 old definition(s) deleted for function or rule sin 
\end{MessageOutput}
\end{xtc}
\begin{xtc}
\begin{xtccomment}
Then \Language{} definitely does not understand us.
\end{xtccomment}
\begin{spadsrc}
sin 4.3 
\end{spadsrc}
\begin{MessageOutput}
   FriCAS cannot determine the type of sin because it cannot analyze 
      the non-recursive part, if that exists. This may be remedied by 
      declaring the function.
\end{MessageOutput}
\end{xtc}
\begin{xtc}
\begin{xtccomment}
Again, we could package-call the inside function.
\end{xtccomment}
\begin{spadsrc}
sin x == sin(x)$Float 
\end{spadsrc}
\begin{MessageOutput}
   1 old definition(s) deleted for function or rule sin 
\end{MessageOutput}
\end{xtc}
\begin{xtc}
\begin{xtccomment}
\end{xtccomment}
\begin{spadsrc}
sin 4.3 
\end{spadsrc}
\begin{MessageOutput}
   Compiling function sin with type Float -> Float 
\end{MessageOutput}
\begin{TeXOutput}
\begin{fricasmath}{7}
-{\STRING{0.91616593674945498404}}%
\end{fricasmath}
\end{TeXOutput}
\formatResultType{Float}
\end{xtc}
Of course, you are unlikely to make such obvious errors.
It is more probable that you would write a function and in the body use a
function that you think is a library function.
If you had also written a function by that same name, the library function
would be invisible.

How does \Language{} determine what library function to call?
It very much depends on the particular example, but the simple case of
creating the polynomial
\spad{x + 2/3} will give you an idea.
\begin{enumerate}
\item The \spad{x} is analyzed and its default type is
\spadtype{Variable(x)}.
\item The \spad{2} is analyzed and its default type is
\spadtype{PositiveInteger}.
\item The \spad{3} is analyzed and its default type is
\spadtype{PositiveInteger}.
\item Because the arguments to \spadop{/} are integers, \Language{}
gives the expression \spad{2/3} a default target type of
\spadtype{Fraction(Integer)}.
\item \Language{} looks in \spadtype{PositiveInteger} for \spadop{/}.
It is not found.
\item \Language{} looks in \spadtype{Fraction(Integer)} for \spadop{/}.
It is found for arguments of type \spadtype{Integer}.
\item The \spad{2} and \spad{3} are converted to objects of type
\spadtype{Integer} (this is trivial) and \spadop{/} is applied,
creating an object of type \spadtype{Fraction(Integer)}.
\item No \spadop{+} for arguments of types \spadtype{Variable(x)} and
\spadtype{Fraction(Integer)} are found in either domain.
\item \Language{} resolves
\index{resolve}
(see \spadref{ugTypesResolve})
the types and gets \spadtype{Polynomial (Fraction (Integer))}.
\item The \spad{x} and the \spad{2/3} are converted to objects of this
type and \spadop{+} is applied, yielding the answer, an object of type
\spadtype{Polynomial (Fraction (Integer))}.
\end{enumerate}

% *********************************************************************
\head{section}{Compiling vs. Interpreting}{ugUserCompInt}
% *********************************************************************

When possible, \Language{} completely determines the type of every object in
a function, then translates the function definition to \Lisp{} or
to machine code (see next section).
This translation,
\index{function!compiler}
called \spadglossSee{compilation}{compiler}, happens the first time you call
the function and results in a computational delay.
Subsequent function calls with the same argument types use the compiled
version of the code without delay.

If \Language{} cannot determine the type of everything, the
function may still be executed
\index{function!interpretation}
but
\index{interpret-code mode}
in \spadglossSee{interpret-code mode}{interpreter} :
each statement in the function is analyzed and executed as the control
flow indicates.
This process is slower than executing a compiled function, but it
allows the execution of code that may involve objects whose types
change.

% ----------------------------------------------------------------------
\beginImportant
If \Language{} decides that it cannot compile the code, it
issues a message stating the problem and then the following
message:
%
\begin{center}
{\bf We will attempt to step through and interpret the code.}
\end{center}
%
This is not a time to panic.
\index{panic!avoiding}
Rather, it just means that what you gave to \Language{}
is somehow ambiguous: either it is not specific enough to be analyzed
completely, or it is beyond \Language{}'s present interactive
compilation abilities.
\endImportant
% ----------------------------------------------------------------------

\begin{xtc}
\begin{xtccomment}
This function runs in interpret-code mode, but it does not compile.
\end{xtccomment}
\begin{spadsrc}
varPolys(vars) ==
  for var in vars repeat
    output(1 :: UnivariatePolynomial(var,Integer))
\end{spadsrc}
\end{xtc}
\begin{xtc}
\begin{xtccomment}
For \spad{vars} equal to \spad{['x, 'y, 'z]}, this function displays
\spad{1} three times.
\end{xtccomment}
\begin{spadsrc}
varPolys ['x,'y,'z] 
\end{spadsrc}
\begin{MessageOutput}
   Cannot compile conversion for types involving local variables. In 
      particular, could not compile the expression involving :: 
      UnivariatePolynomial(var,Integer) 
\end{MessageOutput}
\begin{MessageOutput}
   FriCAS will attempt to step through and interpret the code.
\end{MessageOutput}
\end{xtc}
\begin{xtc}
\begin{xtccomment}
The type of the argument to \spadfun{output} changes in each iteration,
so \Language{} cannot compile the function.
In this case, even the inner loop by itself would have a problem:
\end{xtccomment}
\begin{spadsrc}
for var in ['x,'y,'z] repeat
  output(1 :: UnivariatePolynomial(var,Integer))
\end{spadsrc}
\begin{MessageOutput}
   Cannot compile conversion for types involving local variables. In 
      particular, could not compile the expression involving :: 
      UnivariatePolynomial(var,Integer) 
\end{MessageOutput}
\begin{MessageOutput}
   FriCAS will attempt to step through and interpret the code.
\end{MessageOutput}
\end{xtc}

Sometimes you can help a function to compile by using an extra conversion
or by using \spad{pretend}.
\spadkey{pretend}
See \spadref{ugTypesSubdomains} for details.

When a function is compilable, you have the choice of whether it is
compiled to \Lisp{} and then interpreted by the \Lisp{}
interpreter or then further compiled from \Lisp{} to machine code.
\index{machine code}
The option is controlled via \spadsys{)set functions compile}.
\syscmdindex{set function compile}
Issue \spadsys{)set functions compile on} to compile all the way to
machine code.
With
the default setting \spadsys{)set functions compile off},
\Language{} has its \Lisp{} code interpreted
because the overhead of further compilation is larger than the run-time
of most of the functions our users have defined.
You may find that selectively turning this option on and off will
\index{performance}
give you the best performance in your particular application.
For example, if you are writing functions for graphics applications
where hundreds of points are being computed, it is almost certainly true
that you will get the best performance by issuing
\spadsys{)set functions compile on}.

% *********************************************************************
\head{section}{Piece-Wise Function Definitions}{ugUserPiece}
% *********************************************************************

To move beyond functions defined in one line, we introduce in this section
functions that are defined piece-by-piece.
That is, we say ``use this definition when the argument is such-and-such and
use this other definition when the argument is that-and-that.''

% *********************************************************************
\head{subsection}{A Basic Example}{ugUserPieceBasic}
% *********************************************************************

There are many other ways to define a factorial function for nonnegative
integers.
You might
\index{function!piece-wise definition}
say
\index{piece-wise function definition}
factorial of \spad{0} is \spad{1,} otherwise factorial of \spad{n} is
\spad{n} times factorial of \spad{n-1}.
Here is one way to do this in \Language{}.
%
\begin{xtc}
\begin{xtccomment}
Here is the value for \spad{n = 0}.
\end{xtccomment}
\begin{spadsrc}
fact(0) == 1 
\end{spadsrc}
\end{xtc}
\begin{xtc}
\begin{xtccomment}
Here is the value for \spad{n > 0}.
The vertical bar \spadSyntax{|} means
``such that''.
\end{xtccomment}
\begin{spadsrc}
fact(n | n > 0) == n * fact(n - 1) 
\end{spadsrc}
\end{xtc}
\index{such that}
%>> am moving this back
%The vertical bar \spadSyntax{|} is read as ``such that'' and so
%\index{such that}
%the second line means that that part of the definition for \userfun{fact}
%is for any \spad{n} such that \spad{n} is greater than 0.
%In fact, the first line is really just a shorthand expression for
%\spad{fact(n | n = 0) == 1}.
%>> prefer scratching next 4 lines
%We are implicitly using a \spadgloss{predicate} with a \spadSyntax{|} in
%this line (see \spadref{ugUserPiecePred} for more on predicates).
%So this piece of the function is applicable to all (the not so many!)
%values of \spad{n} that are equal to zero.
\begin{xtc}
\begin{xtccomment}
What is the value for \spad{n = 3}?
\end{xtccomment}
\begin{spadsrc}
fact(3) 
\end{spadsrc}
\begin{MessageOutput}
   Compiling function fact with type Integer -> Integer 
\end{MessageOutput}
\begin{MessageOutput}
   Compiling function fact as a recurrence relation.
\end{MessageOutput}
\begin{TeXOutput}
\begin{fricasmath}{3}
6%
\end{fricasmath}
\end{TeXOutput}
\formatResultType{PositiveInteger}
\end{xtc}
\begin{xtc}
\begin{xtccomment}
What is the value for \spad{n = -3}?
\end{xtccomment}
\begin{spadsrc}
fact(-3) 
\end{spadsrc}
\begin{MessageOutput}
   You did not define fact for argument -3 .
\end{MessageOutput}
\end{xtc}
\begin{xtc}
\begin{xtccomment}
Now for a second definition.
Here is the value for \spad{n = 0}.
\end{xtccomment}
\begin{spadsrc}
facto(0) == 1 
\end{spadsrc}
\end{xtc}
\begin{xtc}
\begin{xtccomment}
Give an error message if \spad{n < 0}.
\end{xtccomment}
\begin{spadsrc}
facto(n | n < 0) == error "arguments to facto must be non-negative" 
\end{spadsrc}
\end{xtc}
\begin{xtc}
\begin{xtccomment}
Here is the value otherwise.
\end{xtccomment}
\begin{spadsrc}
facto(n) == n * facto(n - 1) 
\end{spadsrc}
\end{xtc}
\begin{xtc}
\begin{xtccomment}
What is the value for \spad{n = 7}?
\end{xtccomment}
\begin{spadsrc}
facto(3) 
\end{spadsrc}
\begin{MessageOutput}
   Compiling function facto with type Integer -> Integer 
\end{MessageOutput}
\begin{TeXOutput}
\begin{fricasmath}{7}
6%
\end{fricasmath}
\end{TeXOutput}
\formatResultType{PositiveInteger}
\end{xtc}
\begin{xtc}
\begin{xtccomment}
What is the value for \spad{n = -7}?
\end{xtccomment}
\begin{spadsrc}
facto(-7) 
\end{spadsrc}
\begin{MessageOutput}
   Error signalled from user code in function facto: 
      arguments to facto must be non-negative
\end{MessageOutput}
\end{xtc}
\begin{xtc}
\begin{xtccomment}
To see the current piece-wise definition of a function,
use \spadsys{)display value}.
\end{xtccomment}
\begin{spadsrc}
)display value facto 
\end{spadsrc}
\begin{SysCmdOutput}
   Definition:
     facto 0 == 1
     facto (n | n < 0) == error(arguments to facto must be non-negative)
     facto n == n facto(n - 1)
\end{SysCmdOutput}
\end{xtc}

In general a {\it piece-wise definition} of a function consists of two or
more parts.
Each part gives a ``piece'' of the entire definition.
\Language{} collects the pieces of a function as you enter them.
When you ask for a value of the function, it then ``glues''
the pieces together to form a function.

The two piece-wise definitions for the factorial function
are examples of recursive functions, that is, functions that
are defined in terms of themselves.
Here is an interesting doubly-recursive function.
This function returns the value \spad{11} for all positive integer arguments.
\begin{xtc}
\begin{xtccomment}
Here is the first of two pieces.
\end{xtccomment}
\begin{spadsrc}
eleven(n | n < 1) == n + 11
\end{spadsrc}
\end{xtc}
\begin{xtc}
\begin{xtccomment}
And the general case.
\end{xtccomment}
\begin{spadsrc}
eleven(m) == eleven(eleven(m - 12))
\end{spadsrc}
\end{xtc}
\begin{xtc}
\begin{xtccomment}
Compute \spad{elevens}, the infinite stream
of values of \spad{eleven}.
\end{xtccomment}
\begin{spadsrc}
elevens := [eleven(i) for i in 0..]
\end{spadsrc}
\begin{MessageOutput}
   Compiling function eleven with type Integer -> Integer 
\end{MessageOutput}
\begin{TeXOutput}
\begin{fricasmath}{10}
\BRACKET{11\COMMA 11\COMMA 11\COMMA 11\COMMA 11\COMMA 11\COMMA 11\COMMA %
\STRING{...}}%
\end{fricasmath}
\end{TeXOutput}
\formatResultType{Stream(Integer)}
\end{xtc}
\begin{xtc}
\begin{xtccomment}
What is the value at \spad{n = 200}?
\end{xtccomment}
\begin{spadsrc}
elevens 200
\end{spadsrc}
\begin{TeXOutput}
\begin{fricasmath}{11}
11%
\end{fricasmath}
\end{TeXOutput}
\formatResultType{PositiveInteger}
\end{xtc}
\begin{xtc}
\begin{xtccomment}
What is the \Language{}'s definition of \spad{eleven}?
\end{xtccomment}
\begin{spadsrc}
)display value eleven
\end{spadsrc}
\begin{SysCmdOutput}
   Definition:
     eleven (m | m < 1) == m + 11
     eleven m == eleven(eleven(m - 12))
\end{SysCmdOutput}
\end{xtc}
% *********************************************************************
\head{subsection}{Picking Up the Pieces}{ugUserPiecePicking}
% *********************************************************************

Here are the details about how \Language{} creates a function from its
pieces.
\Language{} converts the \eth{i} piece of a function definition into a
conditional expression of the form: \spad{if} \pred{i} \spad{then}
\expr{i}.
If any new piece has a \pred{i} that is identical\footnote{after all
variables are uniformly named} to an earlier \pred{j}, the earlier piece is
removed.
Otherwise, the new piece is always added at the end.

% ----------------------------------------------------------------------
\beginImportant
If there are \spad{n} pieces to a function definition for \spad{f},
the function defined \spad{f} is: \newline
%
\hspace*{3pc}
{\tt if} \pred{1} {\tt then} \expr{1} {\tt else}\newline
\hspace*{6pc}. . . \newline
\hspace*{3pc}
{\tt if} \pred{n} {\tt then} \expr{n} {\tt else}\newline
\hspace*{3pc}
{\tt  error "You did not define f for argument <arg>."}
%
\endImportant
% ----------------------------------------------------------------------

You can give definitions of any number of mutually recursive function
definitions, piece-wise or otherwise.
No computation is done until you ask for a value.
When you do ask for a value, all the relevant definitions are gathered,
analyzed, and translated into separate functions and compiled.

\begin{xtc}
\begin{xtccomment}
Let's recall the definition of \userfun{eleven} from
the previous section.
\end{xtccomment}
\begin{spadsrc}
eleven(n | n < 1) == n + 11
\end{spadsrc}
\end{xtc}
\begin{xtc}
\begin{xtccomment}
\end{xtccomment}
\begin{spadsrc}
eleven(m) == eleven(eleven(m - 12))
\end{spadsrc}
\end{xtc}

A similar doubly-recursive function below produces \spad{-11} for all
negative positive integers.
If you haven't worked out why or how \userfun{eleven} works,
the structure of this definition gives a clue.
\begin{xtc}
\begin{xtccomment}
This definition we write as a block.
\end{xtccomment}
\begin{spadsrc}
minusEleven(n) ==
  n >= 0 => n - 11
  minusEleven (5 + minusEleven(n + 7))
\end{spadsrc}
\end{xtc}
\begin{xtc}
\begin{xtccomment}
Define \spad{s(n)} to be the
sum of plus and minus ``eleven'' functions divided by \spad{n}.
Since \spad{11 - 11 = 0}, we define \spad{s(0)} to be \spad{1}.
\end{xtccomment}
\begin{spadsrc}
s(0) == 1
\end{spadsrc}
\end{xtc}
\begin{xtc}
\begin{xtccomment}
And the general term.
\end{xtccomment}
\begin{spadsrc}
s(n) == (eleven(n) + minusEleven(n))/n
\end{spadsrc}
\end{xtc}
\begin{xtc}
\begin{xtccomment}
What are the first ten values of \spad{s}?
\end{xtccomment}
\begin{spadsrc}
[s(n) for n in 0..]
\end{spadsrc}
\begin{MessageOutput}
   Compiling function eleven with type Integer -> Integer 
\end{MessageOutput}
\begin{MessageOutput}
   Compiling function minusEleven with type Integer -> Integer 
\end{MessageOutput}
\begin{MessageOutput}
   Compiling function s with type NonNegativeInteger -> Fraction(
      Integer) 
\end{MessageOutput}
\begin{TeXOutput}
\begin{fricasmath}{6}
\BRACKET{1\COMMA 1\COMMA 1\COMMA 1\COMMA 1\COMMA 1\COMMA 1\COMMA \STRING{...}%
}%
\end{fricasmath}
\end{TeXOutput}
\formatResultType{Stream(Fraction(Integer))}
\end{xtc}
%% interpreter puts the rule at the end - should fix
% \xtc{
% Oops! Evidently \spad{s(0)} should be \spad{1}.
% Let's check the current definition of \userfun{s} using \spadsys{)display}.
% }{
% \spadcommand{)display value s\free{rf3}}
% }
% \xtc{
% Change the value at \spad{n = 0}.
% }{
% \spadcommand{s(0) == 1\free{rf3}\bound{rf4}}
% }
% \xtc{
% Now, what is the definition of \userfun{s}?
% Note: {\it you can only replace a given piece if you give exactly the same
% predicate!}
% }{
% \spadcommand{)display value s\free{rf4}}
% }
\Language{} can create infinite streams in the positive direction (for
example, for index values \mathOrSpad{0,1, \ldots}) or negative direction (for
example, for index values \mathOrSpad{0,-1,-2, \ldots}).
Here we would like a stream of values of \spad{s(n)} that is infinite in
both directions.
The function \spad{t(n)} below returns the \eth{n} term of the infinite
stream \mathOrSpad{[s(0), s(1), s(-1), s(2), s(-2), \ldots].}
Its definition has three pieces.
\begin{xtc}
\begin{xtccomment}
Define the initial term.
\end{xtccomment}
\begin{spadsrc}
t(1) == s(0)
\end{spadsrc}
\end{xtc}
\begin{xtc}
\begin{xtccomment}
The even numbered terms are the \spad{s(i)} for positive \spad{i}.
We use \spadop{quo} rather than \spadop{/}
since we want the result to be an integer.
\end{xtccomment}
\begin{spadsrc}
t(n | even?(n)) == s(n quo 2)
\end{spadsrc}
\end{xtc}
\begin{xtc}
\begin{xtccomment}
Finally, the odd numbered terms are the
\spad{s(i)} for negative \spad{i}.
In piece-wise definitions, you can use different variables
to define different pieces. \Language{} will not get confused.
\end{xtccomment}
\begin{spadsrc}
t(p) == s(- p quo 2)
\end{spadsrc}
\end{xtc}
\begin{xtc}
\begin{xtccomment}
Look at the definition of \spad{t}.
In the first piece, the variable \spad{n}
was used; in the second piece, \spad{p}.
\Language{} always uses
your last variable to display your definitions
back to you.
\end{xtccomment}
\begin{spadsrc}
)display value t
\end{spadsrc}
\begin{SysCmdOutput}
   Definition:
     t 1 == s(0)
     t (p | even?(p)) == s(p quo 2)
     t p == s(- p quo 2)
\end{SysCmdOutput}
\end{xtc}
\begin{xtc}
\begin{xtccomment}
Create a series of values of \spad{s} applied to
alternating positive and negative arguments.
\end{xtccomment}
\begin{spadsrc}
[t(i) for i in 1..]
\end{spadsrc}
\begin{MessageOutput}
   Compiling function s with type Integer -> Fraction(Integer) 
\end{MessageOutput}
\begin{MessageOutput}
   Compiling function t with type PositiveInteger -> Fraction(Integer) 
\end{MessageOutput}
\begin{TeXOutput}
\begin{fricasmath}{10}
\BRACKET{1\COMMA 1\COMMA 1\COMMA 1\COMMA 1\COMMA 1\COMMA 1\COMMA \STRING{...}%
}%
\end{fricasmath}
\end{TeXOutput}
\formatResultType{Stream(Fraction(Integer))}
\end{xtc}
\begin{xtc}
\begin{xtccomment}
Evidently \spad{t(n) = 1} for all \spad{i.}
Check it at \spad{n= 100}.
\end{xtccomment}
\begin{spadsrc}
t(100)
\end{spadsrc}
\begin{TeXOutput}
\begin{fricasmath}{11}
1%
\end{fricasmath}
\end{TeXOutput}
\formatResultType{Fraction(Integer)}
\end{xtc}

% *********************************************************************
\head{subsection}{Predicates}{ugUserPiecePred}
% *********************************************************************

We have already seen some examples of
\index{function!predicate}
predicates
\index{predicate!in function definition}
(\spadref{ugUserPieceBasic}).
Predicates are \spadtype{Boolean}-valued expressions and \Language{} uses them
for filtering collections
(see \spadref{ugLangIts})
and for placing
constraints on function arguments.
In this section we discuss their latter usage.

\begin{xtc}
\begin{xtccomment}
The simplest use of a predicate is one you don't see at all.
\end{xtccomment}
\begin{spadsrc}
opposite 'right == 'left
\end{spadsrc}
\end{xtc}
\begin{xtc}
\begin{xtccomment}
Here is a longer way to give the ``opposite definition.''
\end{xtccomment}
\begin{spadsrc}
opposite (x | x = 'left) == 'right
\end{spadsrc}
\end{xtc}
\begin{xtc}
\begin{xtccomment}
Try it out.
\end{xtccomment}
\begin{spadsrc}
for x in ['right, 'left] repeat output opposite x
\end{spadsrc}
\begin{MessageOutput}
   Compiling function opposite with type OrderedVariableList([right,
      left]) -> Symbol 
\end{MessageOutput}
\end{xtc}
\begin{xtc}
\begin{xtccomment}
We get an error if there is no definition for given argument.
\end{xtccomment}
\begin{spadsrc}
opposite('inbetween)
\end{spadsrc}
\begin{MessageOutput}
   Compiling function opposite with type Variable(inbetween) -> Symbol 
\end{MessageOutput}
\begin{MessageOutput}
   The function opposite is not defined for the given argument(s).
\end{MessageOutput}
\end{xtc}

Explicit predicates tell \Language{} that the given function definition
piece is to be applied if the predicate evaluates to {\tt true} for the
arguments to the function.
You can use such ``constant'' arguments for integers,
\index{function!constant argument}
strings, and quoted symbols.
\index{constant function argument}
The \spadtype{Boolean} values \spad{true} and \spad{false} can also be used
if qualified with ``\spad{@}'' or ``\spad{$}'' %$
and \spadtype{Boolean}.
The following are all valid function definition fragments using
constant arguments.
\begin{verbatim}
a(1) == ...
b("unramified") == ...
c('untested) == ...
d(true@Boolean) == ...
\end{verbatim}

If a function has more than one argument,
each argument can have its own predicate.
However, if a predicate involves two or more arguments, it must be given
{\it after} all the arguments mentioned in the predicate have been given.
You are always safe to give
a single predicate at the end of the argument list.
\begin{xtc}
\begin{xtccomment}
A function involving predicates on two arguments.
\end{xtccomment}
\begin{spadsrc}
inFirstHalfQuadrant(x | x > 0,y | y < x) == true
\end{spadsrc}
\end{xtc}
\begin{xtc}
\begin{xtccomment}
This is incorrect as it gives a predicate on \spad{y}
before the argument \spad{y} is given.
\end{xtccomment}
\begin{spadsrc}
inFirstHalfQuadrant(x | x > 0 and y < x,y) == true
\end{spadsrc}
\begin{MessageOutput}
   1 old definition(s) deleted for function or rule inFirstHalfQuadrant
      
\end{MessageOutput}
\end{xtc}
\begin{xtc}
\begin{xtccomment}
It is always correct to write the predicate at the end.
\end{xtccomment}
\begin{spadsrc}
inFirstHalfQuadrant(x,y | x > 0 and y < x) == true 
\end{spadsrc}
\begin{MessageOutput}
   1 old definition(s) deleted for function or rule inFirstHalfQuadrant
      
\end{MessageOutput}
\end{xtc}
\begin{xtc}
\begin{xtccomment}
Here is the rest of the definition.
\end{xtccomment}
\begin{spadsrc}
inFirstHalfQuadrant(x,y) == false 
\end{spadsrc}
\end{xtc}
\begin{xtc}
\begin{xtccomment}
Try it out.
\end{xtccomment}
\begin{spadsrc}
[inFirstHalfQuadrant(i,3) for i in 1..5]
\end{spadsrc}
\begin{MessageOutput}
   Compiling function inFirstHalfQuadrant with type (PositiveInteger,
      PositiveInteger) -> Boolean 
\end{MessageOutput}
\begin{TeXOutput}
\begin{fricasmath}{8}
\BRACKET{\STRING{false}\COMMA \STRING{false}\COMMA \STRING{false}\COMMA %
\STRING{true}\COMMA \STRING{true}}%
\end{fricasmath}
\end{TeXOutput}
\formatResultType{List(Boolean)}
\end{xtc}

{\bf Remark:} Very old versions of \Language{} allowed predicates
to be given after a {\tt when} keyword as in
{\tt inFirstHalfQuadrant(x ,y) == true when x >0 and y < x}.
This is no longer supported, is WRONG, and will cause a syntax
error or strange behavior.

% *********************************************************************
\head{section}{Caching Previously Computed Results}{ugUserCache}
% *********************************************************************

By default, \Language{} does not save the values of any function.
\index{function!caching values}
You can cause it to save values and not to recompute unnecessarily
\index{remembering function values}
by using \spadsys{)set functions cache}.
\syscmdindex{set functions cache}
This should be used before the functions are defined or, at least, before
they are executed.
The word following ``cache'' should be \spad{0} to turn off
caching, a positive integer \spad{n} to save the last \spad{n}
computed values or ``all'' to save all computed values.
If you then give a list of names of functions, the caching
only affects those functions.
Use no list of names or ``all'' when you want to define the default
behavior for functions not specifically mentioned in other
\spadsys{)set functions cache} statements.
If you give no list of names, all functions will have the caching behavior.
If you explicitly turn on caching for one or more names, you must
explicitly turn off caching for those names when you want to stop
saving their values.

\begin{xtc}
\begin{xtccomment}
This causes the functions \userfun{f} and \userfun{g} to have
the last three computed values saved.
\end{xtccomment}
\begin{spadsrc}
)set functions cache 3 f g 
\end{spadsrc}
\begin{SysCmdOutput}
   function f will cache the last 3 values.
   function g will cache the last 3 values.
\end{SysCmdOutput}
\end{xtc}
\begin{xtc}
\begin{xtccomment}
This is a sample definition for \userfun{f}.
\end{xtccomment}
\begin{spadsrc}
f x == factorial(2^x) 
\end{spadsrc}
\end{xtc}
\begin{xtc}
\begin{xtccomment}
A message is displayed stating what \userfun{f} will cache.
\end{xtccomment}
\begin{spadsrc}
f(4) 
\end{spadsrc}
\begin{MessageOutput}
   Compiling function f with type PositiveInteger -> Integer 
\end{MessageOutput}
\begin{MessageOutput}
   f will cache 3 most recently computed value(s).
\end{MessageOutput}
\begin{TeXOutput}
\begin{fricasmath}{2}
20922789888000%
\end{fricasmath}
\end{TeXOutput}
\formatResultType{PositiveInteger}
\end{xtc}
\begin{xtc}
\begin{xtccomment}
This causes all other functions to have all computed values saved by
default.
\end{xtccomment}
\begin{spadsrc}
)set functions cache all
\end{spadsrc}
\begin{SysCmdOutput}
   In general, interpreter functions will cache all values.
\end{SysCmdOutput}
\end{xtc}
\begin{xtc}
\begin{xtccomment}
This causes all functions that have not been specifically cached in some way
to have no computed values saved.
\end{xtccomment}
\begin{spadsrc}
)set functions cache 0
\end{spadsrc}
\begin{SysCmdOutput}
 In general, functions will cache no returned values.
\end{SysCmdOutput}
\end{xtc}
\begin{xtc}
\begin{xtccomment}
We also make \userfun{f} and \userfun{g} uncached.
\end{xtccomment}
\begin{spadsrc}
)set functions cache 0 f g
\end{spadsrc}
\begin{SysCmdOutput}
   Caching for function f is turned off
   Caching for function g is turned off
\end{SysCmdOutput}
\end{xtc}

\beginImportant
Be careful about caching functions that have
\spadglossSee{side effects}{side effect}.
Such a function might destructively modify the elements of an array or
issue a \spadfun{draw} command, for example.
A function that you expect to execute every time it is called should
not be cached.
Also, it is highly unlikely that a function with no arguments should
be cached.
\endImportant

You should also be careful about caching functions that depend on
free variables.
See \spadref{ugUserFreeLocal}
for an example.

% *********************************************************************
\head{section}{Recurrence Relations}{ugUserRecur}
% *********************************************************************

One of the most useful classes of function are those defined via a
``recurrence relation.''
A {\it recurrence relation} makes each successive
\index{recurrence relation}
value depend on some or all of the previous values.
A simple example is the ordinary ``factorial'' function:
\begin{verbatim}
fact(0) == 1
fact(n | n > 0) == n * fact(n-1)
\end{verbatim}

The value of
\spad{fact(10)} depends on the value of \spad{fact(9)}, \spad{fact(9)}
on \spad{fact(8)}, and so on.
Because it depends on only one previous value, it is usually called a
{\it first order recurrence relation.}
You can easily imagine a function based on two, three or more previous
values.
The Fibonacci numbers are probably the most famous function defined by a
\index{Fibonacci numbers}
second order recurrence relation.
\begin{xtc}
\begin{xtccomment}
The library function \spadfun{fibonacci} computes Fibonacci numbers.
It is obviously optimized for speed.
\end{xtccomment}
\begin{spadsrc}
[fibonacci(i) for i in 0..]
\end{spadsrc}
\begin{TeXOutput}
\begin{fricasmath}{1}
\BRACKET{0\COMMA 1\COMMA 1\COMMA 2\COMMA 3\COMMA 5\COMMA 8\COMMA \STRING{...}%
}%
\end{fricasmath}
\end{TeXOutput}
\formatResultType{Stream(Integer)}
\end{xtc}
\begin{xtc}
\begin{xtccomment}
Define the
Fibonacci numbers ourselves using a piece-wise definition.
\end{xtccomment}
\begin{spadsrc}
fib(1) == 1 
\end{spadsrc}
\end{xtc}
\begin{xtc}
\begin{xtccomment}
\end{xtccomment}
\begin{spadsrc}
fib(2) == 1 
\end{spadsrc}
\end{xtc}
\begin{xtc}
\begin{xtccomment}
\end{xtccomment}
\begin{spadsrc}
fib(n) == fib(n-1) + fib(n-2) 
\end{spadsrc}
\end{xtc}

As defined, this recurrence relation is obviously doubly-recursive.
To compute \spad{fib(10)}, we need to compute \spad{fib(9)} and
\spad{fib(8)}.
And to  \spad{fib(9)}, we need to compute \spad{fib(8)} and
\spad{fib(7)}.
And so on.
It seems that to compute \spad{fib(10)} we need to compute
\spad{fib(9)} once, \spad{fib(8)} twice, \spad{fib(7)} three times.
Look familiar?
The number of function calls needed to compute {\it any} second order
recurrence relation in the obvious way is exactly \spad{fib(n)}.
These numbers grow!
For example, if \Language{} actually did this, then \spad{fib(500)}
requires more than $10^{104}$ function calls.
And, given all this, our definition of \userfun{fib} obviously could not be
used to calculate the five-hundredth Fibonacci number.
\begin{xtc}
\begin{xtccomment}
Let's try it anyway.
\end{xtccomment}
\begin{spadsrc}
fib(500) 
\end{spadsrc}
\begin{MessageOutput}
   Compiling function fib with type Integer -> PositiveInteger 
\end{MessageOutput}
\begin{MessageOutput}
   Compiling function fib as a recurrence relation.
\end{MessageOutput}
\begin{TeXOutput}
\begin{fricasmath}{5}
13942322456169788013 97243828704072839500 70256587697307264108 96294832557162286329 06915576588762225212 94125%
\end{fricasmath}
\end{TeXOutput}
\formatResultType{PositiveInteger}
\end{xtc}

Since this takes a short time to compute, it obviously didn't do
as many as $10^{104}$ operations!
By default, \Language{} transforms any recurrence relation it recognizes
into an iteration.
Iterations are efficient.
To compute the value of the \eth{n}
term of a recurrence relation using an iteration requires only
\spad{n} function calls.\footnote{If
you compare the speed of our \userfun{fib} function
to the library function, our version is still slower.
This is because the library
\spadfunFrom{fibonacci}{IntegerNumberTheoryFunctions}
uses a ``powering algorithm'' with a computing time
proportional to $\log^3(n)$ to compute
\spad{fibonacci(n).}}

To turn off this special recurrence relation compilation, issue
\syscmdindex{set function recurrence}
\begin{verbatim}
)set functions recurrence off
\end{verbatim}
To turn it back on, substitute ``{\tt on}'' for ``{\tt off}''.

The transformations that \Language{} uses for \userfun{fib} caches the
last two values.\footnote{For a more general \eth{k} order recurrence
relation, \Language{} caches the last \spad{k} values.}
If, after computing a value for \userfun{fib}, you ask
for some larger value, \Language{} picks up the cached values
and continues computing from there.
See \spadref{ugUserFreeLocal}
for an example of a function definition that has this same behavior.
Also see \spadref{ugUserCache}
for a more general discussion of how you can cache function values.

Recurrence relations can be used for defining recurrence relations
involving polynomials, rational functions, or anything you like.
Here we compute the infinite stream of Legendre polynomials.
\begin{xtc}
\begin{xtccomment}
The Legendre polynomial of degree \spad{0.}
\end{xtccomment}
\begin{spadsrc}
p(0) == 1
\end{spadsrc}
\end{xtc}
\begin{xtc}
\begin{xtccomment}
The Legendre polynomial of degree \spad{1.}
\end{xtccomment}
\begin{spadsrc}
p(1) == x
\end{spadsrc}
\end{xtc}

\begin{xtc}
\begin{xtccomment}
The Legendre polynomial of degree \spad{n}.
\end{xtccomment}
\begin{spadsrc}
p(n) == ((2*n-1)*x*p(n-1) - (n-1)*p(n-2))/n
\end{spadsrc}
\end{xtc}
\begin{xtc}
\begin{xtccomment}
Compute the Legendre polynomial of degree \spad{6.}
\end{xtccomment}
\begin{spadsrc}
p(6)
\end{spadsrc}
\begin{MessageOutput}
   Compiling function p with type Integer -> Polynomial(Fraction(
      Integer)) 
\end{MessageOutput}
\begin{MessageOutput}
   Compiling function p as a recurrence relation.
\end{MessageOutput}
\begin{TeXOutput}
\begin{fricasmath}{9}
\frac{231}{16}\TIMES \SUPER{\SYMBOL{x}}{6}-{\frac{315}{16}\TIMES \SUPER{%
\SYMBOL{x}}{4}}+\frac{105}{16}\TIMES \SUPER{\SYMBOL{x}}{2}-{\frac{5}{16}}%
\end{fricasmath}
\end{TeXOutput}
\formatResultType{Polynomial(Fraction(Integer))}
\end{xtc}

% *********************************************************************
\head{section}{Making Functions from Objects}{ugUserMake}
% *********************************************************************

There are many times when you compute a complicated expression
and then wish to use that expression as the body of a function.
\Language{} provides an operation called \spadfun{function} to do
\index{function!from an object}
this.
\index{function!made by function @{made by {\bf function}}}
It creates a function object and places it into the workspace.
There are several versions, depending on how many arguments the function
has.
The first argument to \spadfun{function} is always the expression to be
converted into the function body, and the second is always the name to be
used for the function.
For more information, see \xmpref{MakeFunction}.

\begin{xtc}
\begin{xtccomment}
Start with a simple example of a polynomial in three variables.
\end{xtccomment}
\begin{spadsrc}
p := -x + y^2 - z^3 
\end{spadsrc}
\begin{TeXOutput}
\begin{fricasmath}{1}
-{\SUPER{\SYMBOL{z}}{3}}+\SUPER{\SYMBOL{y}}{2}-{\SYMBOL{x}}%
\end{fricasmath}
\end{TeXOutput}
\formatResultType{Polynomial(Integer)}
\end{xtc}
\begin{xtc}
\begin{xtccomment}
To make this into a function of no arguments that
simply returns the polynomial, use the two argument form of
\spadfun{function}.
\end{xtccomment}
\begin{spadsrc}
function(p,'f0) 
\end{spadsrc}
\begin{TeXOutput}
\begin{fricasmath}{2}
\SYMBOL{f0}%
\end{fricasmath}
\end{TeXOutput}
\formatResultType{Symbol}
\end{xtc}
\begin{xtc}
\begin{xtccomment}
To avoid possible conflicts (see below), it is a good idea to
quote always this second argument.
\end{xtccomment}
\begin{spadsrc}
f0 
\end{spadsrc}
\begin{TeXOutput}
\begin{fricasmath}{3}
\SYMBOL{f0}\ \PAREN{}\ \SYMBOL{==}\ -{\SUPER{\SYMBOL{z}}{3}}+\SUPER{\SYMBOL{y%
}}{2}-{\SYMBOL{x}}%
\end{fricasmath}
\end{TeXOutput}
\formatResultType{FunctionCalled(f0)}
\end{xtc}
\begin{xtc}
\begin{xtccomment}
This is what you get when you evaluate the function.
\end{xtccomment}
\begin{spadsrc}
f0() 
\end{spadsrc}
\begin{MessageOutput}
   Compiling function f0 with type () -> Polynomial(Integer) 
\end{MessageOutput}
\begin{TeXOutput}
\begin{fricasmath}{4}
-{\SUPER{\SYMBOL{z}}{3}}+\SUPER{\SYMBOL{y}}{2}-{\SYMBOL{x}}%
\end{fricasmath}
\end{TeXOutput}
\formatResultType{Polynomial(Integer)}
\end{xtc}
\begin{xtc}
\begin{xtccomment}
To make a function in \spad{x}, use a version of
\spadfun{function} that takes three arguments.
The last argument is the name of the variable to use as the parameter.
Typically, this variable occurs in the expression and, like the function
name, you should quote it to avoid possible confusion.
\end{xtccomment}
\begin{spadsrc}
function(p,'f1,'x) 
\end{spadsrc}
\begin{TeXOutput}
\begin{fricasmath}{5}
\SYMBOL{f1}%
\end{fricasmath}
\end{TeXOutput}
\formatResultType{Symbol}
\end{xtc}
\begin{xtc}
\begin{xtccomment}
This is what the new function looks like.
\end{xtccomment}
\begin{spadsrc}
f1 
\end{spadsrc}
\begin{TeXOutput}
\begin{fricasmath}{6}
\SYMBOL{f1}\ \SYMBOL{x}\ \SYMBOL{==}\ -{\SUPER{\SYMBOL{z}}{3}}+\SUPER{\SYMBOL%
{y}}{2}-{\SYMBOL{x}}%
\end{fricasmath}
\end{TeXOutput}
\formatResultType{FunctionCalled(f1)}
\end{xtc}
\begin{xtc}
\begin{xtccomment}
This is the value of \userfun{f1} at \spad{x = 3}.
Notice that the return type of the function is
\spadtype{Polynomial (Integer)}, the same as \spad{p}.
\end{xtccomment}
\begin{spadsrc}
f1(3) 
\end{spadsrc}
\begin{MessageOutput}
   Compiling function f1 with type PositiveInteger -> Polynomial(
      Integer) 
\end{MessageOutput}
\begin{TeXOutput}
\begin{fricasmath}{7}
-{\SUPER{\SYMBOL{z}}{3}}+\SUPER{\SYMBOL{y}}{2}-{3}%
\end{fricasmath}
\end{TeXOutput}
\formatResultType{Polynomial(Integer)}
\end{xtc}
\begin{xtc}
\begin{xtccomment}
To use \spad{x} and \spad{y} as parameters, use the
four argument form of \spadfun{function}.
\end{xtccomment}
\begin{spadsrc}
function(p,'f2,'x,'y) 
\end{spadsrc}
\begin{TeXOutput}
\begin{fricasmath}{8}
\SYMBOL{f2}%
\end{fricasmath}
\end{TeXOutput}
\formatResultType{Symbol}
\end{xtc}
\begin{xtc}
\begin{xtccomment}
\end{xtccomment}
\begin{spadsrc}
f2 
\end{spadsrc}
\begin{TeXOutput}
\begin{fricasmath}{9}
\SYMBOL{f2}\ \PAREN{\SYMBOL{x}\COMMA \SYMBOL{y}}\ \SYMBOL{==}\ -{\SUPER{%
\SYMBOL{z}}{3}}+\SUPER{\SYMBOL{y}}{2}-{\SYMBOL{x}}%
\end{fricasmath}
\end{TeXOutput}
\formatResultType{FunctionCalled(f2)}
\end{xtc}
\begin{xtc}
\begin{xtccomment}
Evaluate \spad{f2} at \spad{x = 3} and \spad{y = 0}.
The return type of \userfun{f2} is still
\spadtype{Polynomial(Integer)} because the variable \spad{z}
is still present and not one of the parameters.
\end{xtccomment}
\begin{spadsrc}
f2(3,0) 
\end{spadsrc}
\begin{MessageOutput}
   Compiling function f2 with type (PositiveInteger,NonNegativeInteger)
       -> Polynomial(Integer) 
\end{MessageOutput}
\begin{TeXOutput}
\begin{fricasmath}{10}
-{\SUPER{\SYMBOL{z}}{3}}-{3}%
\end{fricasmath}
\end{TeXOutput}
\formatResultType{Polynomial(Integer)}
\end{xtc}
\begin{xtc}
\begin{xtccomment}
Finally, use all three variables as parameters.
There is no five argument form of \spadfun{function}, so use the one with
three arguments, the third argument being a list of the parameters.
\end{xtccomment}
\begin{spadsrc}
function(p,'f3,['x,'y,'z]) 
\end{spadsrc}
\begin{TeXOutput}
\begin{fricasmath}{11}
\SYMBOL{f3}%
\end{fricasmath}
\end{TeXOutput}
\formatResultType{Symbol}
\end{xtc}
\begin{xtc}
\begin{xtccomment}
Evaluate this using the same values for \spad{x} and \spad{y}
as above, but let \spad{z} be \spad{-6}.
The result type of \userfun{f3} is \spadtype{Integer}.
\end{xtccomment}
\begin{spadsrc}
f3 
\end{spadsrc}
\begin{TeXOutput}
\begin{fricasmath}{12}
\SYMBOL{f3}\ \PAREN{\SYMBOL{x}\COMMA \SYMBOL{y}\COMMA \SYMBOL{z}}\ \SYMBOL{==%
}\ -{\SUPER{\SYMBOL{z}}{3}}+\SUPER{\SYMBOL{y}}{2}-{\SYMBOL{x}}%
\end{fricasmath}
\end{TeXOutput}
\formatResultType{FunctionCalled(f3)}
\end{xtc}
\begin{xtc}
\begin{xtccomment}
\end{xtccomment}
\begin{spadsrc}
f3(3,0,-6) 
\end{spadsrc}
\begin{MessageOutput}
   Compiling function f3 with type (PositiveInteger,NonNegativeInteger,
      Integer) -> Integer 
\end{MessageOutput}
\begin{TeXOutput}
\begin{fricasmath}{13}
213%
\end{fricasmath}
\end{TeXOutput}
\formatResultType{PositiveInteger}
\end{xtc}

The four functions we have defined via \spad{p} have been undeclared.
To declare a function whose body is to be generated by
\index{function!declaring}
\spadfun{function}, issue the declaration {\it before} the function is created.
\begin{xtc}
\begin{xtccomment}
\end{xtccomment}
\begin{spadsrc}
g: (Integer, Integer) -> Float 
\end{spadsrc}
\end{xtc}
\begin{xtc}
\begin{xtccomment}
\end{xtccomment}
\begin{spadsrc}
D(sin(x-y)/cos(x+y),x) 
\end{spadsrc}
\begin{TeXOutput}
\begin{fricasmath}{15}
\frac{-{\sin{\PAREN{\SYMBOL{y}-{\SYMBOL{x}}}}\TIMES \sin{\PAREN{\SYMBOL{y}+%
\SYMBOL{x}}}}+\cos{\PAREN{\SYMBOL{y}-{\SYMBOL{x}}}}\TIMES \cos{\PAREN{\SYMBOL%
{y}+\SYMBOL{x}}}}{\SUPER{\PAREN{\cos{\PAREN{\SYMBOL{y}+\SYMBOL{x}}}}}{2}}%
\end{fricasmath}
\end{TeXOutput}
\formatResultType{Expression(Integer)}
\end{xtc}
\begin{xtc}
\begin{xtccomment}
\end{xtccomment}
\begin{spadsrc}
function(%,'g,'x,'y) 
\end{spadsrc}
\begin{TeXOutput}
\begin{fricasmath}{16}
\SYMBOL{g}%
\end{fricasmath}
\end{TeXOutput}
\formatResultType{Symbol}
\end{xtc}
\begin{xtc}
\begin{xtccomment}
\end{xtccomment}
\begin{spadsrc}
g 
\end{spadsrc}
\begin{TeXOutput}
\begin{fricasmath}{17}
\SYMBOL{g}\ \PAREN{\SYMBOL{x}\COMMA \SYMBOL{y}}\ \SYMBOL{==}\ \frac{-{\sin{%
\PAREN{\SYMBOL{y}-{\SYMBOL{x}}}}\TIMES \sin{\PAREN{\SYMBOL{y}+\SYMBOL{x}}}}+%
\cos{\PAREN{\SYMBOL{y}-{\SYMBOL{x}}}}\TIMES \cos{\PAREN{\SYMBOL{y}+\SYMBOL{x}%
}}}{\SUPER{\PAREN{\cos{\PAREN{\SYMBOL{y}+\SYMBOL{x}}}}}{2}}%
\end{fricasmath}
\end{TeXOutput}
\formatResultType{FunctionCalled(g)}
\end{xtc}
It is an error to use \spad{g} without the quote in the
penultimate expression since \spad{g} had been declared but did not have
a value.
Similarly, since it is common to overuse variable names like \spad{x},
\spad{y}, and so on,
you avoid problems if you always quote the variable names
for \spadfun{function}.
In general,
if \spad{x} has a value and you use \spad{x} without a quote in a call to
\spadfun{function}, then
\Language{} does not know what you are trying to do.

What kind of object is allowable as the first argument to \spadfun{function}?
Let's use the \Browse{} facility of \HyperName{} to find out.
\index{Browse@\Browse{}}
At the main \Browse{} menu, enter the string {\tt function} and then
click on {\bf Operations.}
The exposed operations called \spadfun{function} all take an object
whose type belongs to category \spadtype{ConvertibleTo InputForm}.
What domains are those?
Go back to the main \Browse{} menu, erase {\tt function},
enter {\tt ConvertibleTo} in the
input area, and click on {\bf categories} on the {\bf Constructors} line.
At the bottom of the page, enter {\tt InputForm} in the input area
following {\bf S =}.
Click on {\bf Cross Reference} and then on {\bf Domains}.
The list you see contains over forty domains that belong to the
category \spadtype{ConvertibleTo InputForm}.
Thus you can use \spadfun{function} for \spadtype{Integer},
\spadtype{Float},
\spadtype{String},
\spadtype{Complex},
\spadtype{Expression}, and so on.

% *********************************************************************
\head{section}{Functions Defined with Blocks}{ugUserBlocks}
% *********************************************************************
% return, local and free variables

You need not restrict yourself to functions that only fit on one line
or are written in a piece-wise manner.
The body of the function can be a block, as discussed in
\spadref{ugLangBlocks}.

\begin{xtc}
\begin{xtccomment}
Here is a short function that swaps two elements of a list,
array or vector.
\end{xtccomment}
\begin{spadsrc}
swap(m,i,j) ==
  temp := m.i
  m.i := m.j
  m.j := temp
\end{spadsrc}
\end{xtc}
\begin{xtc}
\begin{xtccomment}
The significance of \userfun{swap} is that it has a destructive
effect on its first argument.
\end{xtccomment}
\begin{spadsrc}
k := [1,2,3,4,5] 
\end{spadsrc}
\begin{TeXOutput}
\begin{fricasmath}{2}
\BRACKET{1\COMMA 2\COMMA 3\COMMA 4\COMMA 5}%
\end{fricasmath}
\end{TeXOutput}
\formatResultType{List(PositiveInteger)}
\end{xtc}
\begin{xtc}
\begin{xtccomment}
\end{xtccomment}
\begin{spadsrc}
swap(k,2,4) 
\end{spadsrc}
\begin{MessageOutput}
   Compiling function swap with type (List(PositiveInteger),
      PositiveInteger,PositiveInteger) -> PositiveInteger 
\end{MessageOutput}
\begin{TeXOutput}
\begin{fricasmath}{3}
2%
\end{fricasmath}
\end{TeXOutput}
\formatResultType{PositiveInteger}
\end{xtc}
\begin{xtc}
\begin{xtccomment}
You see that the second and fourth elements are interchanged.
\end{xtccomment}
\begin{spadsrc}
k 
\end{spadsrc}
\begin{TeXOutput}
\begin{fricasmath}{4}
\BRACKET{1\COMMA 4\COMMA 3\COMMA 2\COMMA 5}%
\end{fricasmath}
\end{TeXOutput}
\formatResultType{List(PositiveInteger)}
\end{xtc}

\begin{xtc}
\begin{xtccomment}
Using this, we write a couple of different sort functions.
First, a simple bubble sort.
\index{sort!bubble}
The operation \spadopFrom{#}{List} returns the number of elements in
an aggregate.
\end{xtccomment}
\begin{spadsrc}
bubbleSort(m) ==
  n := #m
  for i in 1..(n-1) repeat
    for j in n..(i+1) by -1 repeat
      if m.j < m.(j-1) then swap(m,j,j-1)
  m
\end{spadsrc}
\end{xtc}
\begin{xtc}
\begin{xtccomment}
Let this be the list we want to sort.
\end{xtccomment}
\begin{spadsrc}
m := [8,4,-3,9] 
\end{spadsrc}
\begin{TeXOutput}
\begin{fricasmath}{6}
\BRACKET{8\COMMA 4\COMMA -{3}\COMMA 9}%
\end{fricasmath}
\end{TeXOutput}
\formatResultType{List(Integer)}
\end{xtc}
\begin{xtc}
\begin{xtccomment}
This is the result of sorting.
\end{xtccomment}
\begin{spadsrc}
bubbleSort(m) 
\end{spadsrc}
\begin{MessageOutput}
   Compiling function swap with type (List(Integer),Integer,Integer)
       -> Integer 
\end{MessageOutput}
\begin{MessageOutput}
   Compiling function bubbleSort with type List(Integer) -> List(
      Integer) 
\end{MessageOutput}
\begin{TeXOutput}
\begin{fricasmath}{7}
\BRACKET{-{3}\COMMA 4\COMMA 8\COMMA 9}%
\end{fricasmath}
\end{TeXOutput}
\formatResultType{List(Integer)}
\end{xtc}
\begin{xtc}
\begin{xtccomment}
Moreover, \spad{m} is destructively changed to be the sorted version.
\end{xtccomment}
\begin{spadsrc}
m 
\end{spadsrc}
\begin{TeXOutput}
\begin{fricasmath}{8}
\BRACKET{-{3}\COMMA 4\COMMA 8\COMMA 9}%
\end{fricasmath}
\end{TeXOutput}
\formatResultType{List(Integer)}
\end{xtc}

\begin{xtc}
\begin{xtccomment}
This function implements an insertion sort.
\index{sort!insertion}
The basic idea is to traverse the list and insert the \eth{i}
element in its correct position among the \spad{i-1} previous
elements.
Since we start at the beginning of the list, the list elements before the
\eth{i} element have already been placed in ascending order.
\end{xtccomment}
\begin{spadsrc}
insertionSort(m) ==
  for i in 2..#m repeat
    j := i
    while j > 1 and m.j < m.(j-1) repeat
      swap(m,j,j-1)
      j := j - 1
  m
\end{spadsrc}
\end{xtc}
\begin{xtc}
\begin{xtccomment}
As with our bubble sort, this is a destructive function.
\end{xtccomment}
\begin{spadsrc}
m := [8,4,-3,9] 
\end{spadsrc}
\begin{TeXOutput}
\begin{fricasmath}{10}
\BRACKET{8\COMMA 4\COMMA -{3}\COMMA 9}%
\end{fricasmath}
\end{TeXOutput}
\formatResultType{List(Integer)}
\end{xtc}
\begin{xtc}
\begin{xtccomment}
\end{xtccomment}
\begin{spadsrc}
insertionSort(m) 
\end{spadsrc}
\begin{MessageOutput}
   Compiling function swap with type (List(Integer),NonNegativeInteger,
      Integer) -> Integer 
\end{MessageOutput}
\begin{MessageOutput}
   Compiling function insertionSort with type List(Integer) -> List(
      Integer) 
\end{MessageOutput}
\begin{TeXOutput}
\begin{fricasmath}{11}
\BRACKET{-{3}\COMMA 4\COMMA 8\COMMA 9}%
\end{fricasmath}
\end{TeXOutput}
\formatResultType{List(Integer)}
\end{xtc}
\begin{xtc}
\begin{xtccomment}
\end{xtccomment}
\begin{spadsrc}
m 
\end{spadsrc}
\begin{TeXOutput}
\begin{fricasmath}{12}
\BRACKET{-{3}\COMMA 4\COMMA 8\COMMA 9}%
\end{fricasmath}
\end{TeXOutput}
\formatResultType{List(Integer)}
\end{xtc}

Neither of the above functions is efficient for sorting large lists since
they reference elements by asking for the \eth{j} element of the
structure \spad{m}.
%For lists, compute \spad{m.(j+1) = rest(m,j).first}, and thus, starting at
%the first node of \spad{m}, walk down to the \eth{j} node, then call
%\spadfun{first}.

\begin{xtc}
\begin{xtccomment}
Here is a more efficient bubble sort for lists.
\end{xtccomment}
\begin{spadsrc}
bubbleSort2(m: List Integer): List Integer ==
  null m => m
  l := m
  while not null (r := l.rest) repeat
     r := bubbleSort2 r
     x := l.first
     if x < r.first then
       l.first := r.first
       r.first := x
     l.rest := r
     l := l.rest
  m
\end{spadsrc}
\begin{MessageOutput}
   Function declaration bubbleSort2 : List(Integer) -> List(Integer) 
      has been added to workspace.
\end{MessageOutput}
\end{xtc}
\begin{xtc}
\begin{xtccomment}
Try it out.
\end{xtccomment}
\begin{spadsrc}
bubbleSort2 [3,7,2]
\end{spadsrc}
\begin{MessageOutput}
   Compiling function bubbleSort2 with type List(Integer) -> List(
      Integer) 
\end{MessageOutput}
\begin{TeXOutput}
\begin{fricasmath}{14}
\BRACKET{7\COMMA 3\COMMA 2}%
\end{fricasmath}
\end{TeXOutput}
\formatResultType{List(Integer)}
\end{xtc}

This definition is both recursive and iterative, and is tricky!
Unless you are {\it really} curious about this definition,
we suggest you skip immediately to the next section.

Here are the key points in the definition.
First notice that if you are sorting a list with less than two elements,
there is nothing to do: just return the list.
This definition returns immediately if there are zero elements, and skips
the entire \spad{while} loop if there is just one element.

The second point to realize is that on each outer iteration, the bubble sort
ensures that the minimum element is propagated leftmost.
Each iteration of the \spad{while} loop calls \userfun{bubbleSort2}
recursively to sort all but the first element.
When finished, the minimum element is either in the first or second position.
The conditional expression ensures that it comes first.
If it is in the second, then a swap occurs.
In any case, the \spadfun{rest} of the original list must be updated to hold
the result of the recursive call.

% *********************************************************************
\head{section}{Free and Local Variables}{ugUserFreeLocal}
% *********************************************************************

When you want to refer to a variable that is not local to your
function, use a ``\spad{free}'' declaration.
\spadkey{free}
Variables declared to be \spad{free}
\index{free variable}
are assumed to be defined globally
\index{variable!free}
in the
\index{variable!global}
workspace.
\index{global variable}

\begin{xtc}
\begin{xtccomment}
This is a global workspace variable.
\end{xtccomment}
\begin{spadsrc}
counter := 0 
\end{spadsrc}
\begin{TeXOutput}
\begin{fricasmath}{1}
0%
\end{fricasmath}
\end{TeXOutput}
\formatResultType{NonNegativeInteger}
\end{xtc}
\begin{xtc}
\begin{xtccomment}
This function refers to the global \spad{counter}.
\end{xtccomment}
\begin{spadsrc}
f() ==
  free counter
  counter := counter + 1
\end{spadsrc}
\end{xtc}
\begin{xtc}
\begin{xtccomment}
The global \spad{counter} is incremented by \spad{1}.
\end{xtccomment}
\begin{spadsrc}
f() 
\end{spadsrc}
\begin{MessageOutput}
   Compiling function f with type () -> NonNegativeInteger 
\end{MessageOutput}
\begin{TeXOutput}
\begin{fricasmath}{3}
1%
\end{fricasmath}
\end{TeXOutput}
\formatResultType{PositiveInteger}
\end{xtc}
\begin{xtc}
\begin{xtccomment}
\end{xtccomment}
\begin{spadsrc}
counter 
\end{spadsrc}
\begin{TeXOutput}
\begin{fricasmath}{4}
1%
\end{fricasmath}
\end{TeXOutput}
\formatResultType{NonNegativeInteger}
\end{xtc}

Usually \Language{} can tell that you mean to refer to a global
variable and so \spad{free} isn't always necessary.
However, for clarity and the sake of self-documentation, we encourage
you to use it.

Declare a variable to be ``\spad{local}'' when you do not want to refer to
\index{variable!local}
a global variable by the same name.
\index{local variable}

\begin{xtc}
\begin{xtccomment}
This function uses \spad{counter} as a local variable.
\end{xtccomment}
\begin{spadsrc}
g() ==
  local counter
  counter := 7
\end{spadsrc}
\end{xtc}
\begin{xtc}
\begin{xtccomment}
Apply the function.
\end{xtccomment}
\begin{spadsrc}
g() 
\end{spadsrc}
\begin{MessageOutput}
   Compiling function g with type () -> PositiveInteger 
\end{MessageOutput}
\begin{TeXOutput}
\begin{fricasmath}{6}
7%
\end{fricasmath}
\end{TeXOutput}
\formatResultType{PositiveInteger}
\end{xtc}
\begin{xtc}
\begin{xtccomment}
Check that the global value of \spad{counter} is unchanged.
\end{xtccomment}
\begin{spadsrc}
counter
\end{spadsrc}
\begin{TeXOutput}
\begin{fricasmath}{7}
1%
\end{fricasmath}
\end{TeXOutput}
\formatResultType{NonNegativeInteger}
\end{xtc}

Parameters to a function are local variables in the function.
Even if you issue a \spad{free} declaration for a parameter, it is
still local.

What happens if you do not declare that a variable \spad{x} in
the body of your function is \spad{local} or \spad{free}?
Well, \Language{} decides on this basis:

\begin{enumerate}
\item \Language{} scans your function line-by-line, from top-to-bottom.
The right-hand side of an assignment is looked at before the left-hand
side.
\item If \spad{x} is referenced before it is assigned a value, it is a
\spad{free} (global) variable.
\item If \spad{x} is assigned a value before it is referenced, it is a
\spad{local} variable.
\end{enumerate}

\begin{xtc}
\begin{xtccomment}
Set two global variables to 1.
\end{xtccomment}
\begin{spadsrc}
a := b := 1
\end{spadsrc}
\begin{TeXOutput}
\begin{fricasmath}{8}
1%
\end{fricasmath}
\end{TeXOutput}
\formatResultType{PositiveInteger}
\end{xtc}
\begin{xtc}
\begin{xtccomment}
Refer to \spad{a} before it is assigned a value, but
assign a value to \spad{b} before it is referenced.
\end{xtccomment}
\begin{spadsrc}
h() ==
  b := a + 1
  a := b + a
\end{spadsrc}
\end{xtc}
\begin{xtc}
\begin{xtccomment}
Can you predict this result?
\end{xtccomment}
\begin{spadsrc}
h() 
\end{spadsrc}
\begin{MessageOutput}
   Compiling function h with type () -> PositiveInteger 
\end{MessageOutput}
\begin{TeXOutput}
\begin{fricasmath}{10}
3%
\end{fricasmath}
\end{TeXOutput}
\formatResultType{PositiveInteger}
\end{xtc}
\begin{xtc}
\begin{xtccomment}
How about this one?
\end{xtccomment}
\begin{spadsrc}
[a, b] 
\end{spadsrc}
\begin{TeXOutput}
\begin{fricasmath}{11}
\BRACKET{3\COMMA 1}%
\end{fricasmath}
\end{TeXOutput}
\formatResultType{List(PositiveInteger)}
\end{xtc}

What happened?
In the first line of the function body for \spad{h}, \spad{a} is
referenced on the right-hand side of the assignment.
Thus \spad{a} is a free variable.
The variable \spad{b} is not referenced in that line, but it is
assigned a value.
Thus \spad{b} is a local variable and is given the value
\spad{a + 1 = 2}.
In the second line, the free variable \spad{a} is assigned the value
\spad{b + a} which equals \spad{2 + 1 = 3.}
This is the value returned by the function.
Since \spad{a} was free in \userfun{h}, the global variable \spad{a}
has value \spad{3.}
Since \spad{b} was local in \userfun{h}, the global variable \spad{b}
is unchanged---it still has the value \spad{1.}

It is good programming practice always to declare global variables.
However, by far the most common situation is to have local variables in
your functions.
No declaration is needed for this situation, but be sure to
initialize their values.

Be careful if you use free variables and you cache the value of
your function (see \spadref{ugUserCache}).
Caching {\it only} checks if the values of the function arguments
are the same as in a function call previously seen.
It does not check if any of the free variables on which the
function depends have changed between function calls.
\begin{xtc}
\begin{xtccomment}
Turn on caching for \userfun{p}.
\end{xtccomment}
\begin{spadsrc}
)set fun cache all p 
\end{spadsrc}
\begin{SysCmdOutput}
   function p will cache all values.
\end{SysCmdOutput}
\end{xtc}
\begin{xtc}
\begin{xtccomment}
Define \userfun{p} to depend on the free variable \spad{N}.
\end{xtccomment}
\begin{spadsrc}
p(i,x) == ( free N; reduce( + , [ (x-i)^n for n in 1..N ] ) ) 
\end{spadsrc}
\end{xtc}
\begin{xtc}
\begin{xtccomment}
Set the value of \spad{N}.
\end{xtccomment}
\begin{spadsrc}
N := 1 
\end{spadsrc}
\begin{TeXOutput}
\begin{fricasmath}{13}
1%
\end{fricasmath}
\end{TeXOutput}
\formatResultType{PositiveInteger}
\end{xtc}
\begin{xtc}
\begin{xtccomment}
Evaluate \userfun{p} the first time.
\end{xtccomment}
\begin{spadsrc}
p(0, x) 
\end{spadsrc}
\begin{MessageOutput}
   Compiling function p with type (NonNegativeInteger,Variable(x)) -> 
      Polynomial(Integer) 
\end{MessageOutput}
\begin{MessageOutput}
   p will cache all previously computed values.
\end{MessageOutput}
\begin{TeXOutput}
\begin{fricasmath}{14}
\SYMBOL{x}%
\end{fricasmath}
\end{TeXOutput}
\formatResultType{Polynomial(Integer)}
\end{xtc}
\begin{xtc}
\begin{xtccomment}
Change the value of \spad{N}.
\end{xtccomment}
\begin{spadsrc}
N := 2 
\end{spadsrc}
\begin{TeXOutput}
\begin{fricasmath}{15}
2%
\end{fricasmath}
\end{TeXOutput}
\formatResultType{PositiveInteger}
\end{xtc}
\begin{xtc}
\begin{xtccomment}
Evaluate \userfun{p} the second time.
\end{xtccomment}
\begin{spadsrc}
p(0, x) 
\end{spadsrc}
\begin{TeXOutput}
\begin{fricasmath}{16}
\SYMBOL{x}%
\end{fricasmath}
\end{TeXOutput}
\formatResultType{Polynomial(Integer)}
\end{xtc}
If caching had been turned off, the second evaluation would have
reflected the changed value of \spad{N}.
\begin{xtc}
\begin{xtccomment}
Turn off caching for \userfun{p}.
\end{xtccomment}
\begin{spadsrc}
)set fun cache 0 p
\end{spadsrc}
\begin{SysCmdOutput}
   Caching for function p is turned off
\end{SysCmdOutput}
\end{xtc}

\Language{} does not allow {\it fluid variables}, that is, variables
\index{variable!fluid}
\spadglossSee{bound}{binding} by a function \spad{f} that can be referenced by
functions called by \spad{f}.
\index{fluid variable}

Values are passed to functions by \spadgloss{reference}: a pointer
to the value is passed rather than a copy of the value or a pointer to
a copy.

\begin{xtc}
\begin{xtccomment}
This is a global variable that is bound to a record object.
\end{xtccomment}
\begin{spadsrc}
r : Record(i : Integer) := [1] 
\end{spadsrc}
\begin{TeXOutput}
\begin{fricasmath}{17}
\BRACKET{\SYMBOL{i}=1}%
\end{fricasmath}
\end{TeXOutput}
\formatResultType{Record(i: Integer)}
\end{xtc}
\begin{xtc}
\begin{xtccomment}
This function first modifies the one component of its
record argument and then rebinds the parameter to another
record.
\end{xtccomment}
\begin{spadsrc}
resetRecord rr ==
  rr.i := 2
  rr := [10]
\end{spadsrc}
\end{xtc}
\begin{xtc}
\begin{xtccomment}
Pass \spad{r} as an argument to \userfun{resetRecord}.
\end{xtccomment}
\begin{spadsrc}
resetRecord r 
\end{spadsrc}
\begin{MessageOutput}
   Compiling function resetRecord with type Record(i: Integer) -> 
      Record(i: Integer) 
\end{MessageOutput}
\begin{TeXOutput}
\begin{fricasmath}{19}
\BRACKET{\SYMBOL{i}=10}%
\end{fricasmath}
\end{TeXOutput}
\formatResultType{Record(i: Integer)}
\end{xtc}
\begin{xtc}
\begin{xtccomment}
The value of \spad{r} was changed by the expression
\spad{rr.i := 2} but not by \spad{rr := [10]}.
\end{xtccomment}
\begin{spadsrc}
r 
\end{spadsrc}
\begin{TeXOutput}
\begin{fricasmath}{20}
\BRACKET{\SYMBOL{i}=2}%
\end{fricasmath}
\end{TeXOutput}
\formatResultType{Record(i: Integer)}
\end{xtc}

To conclude this section, we give an iterative definition of
\index{Fibonacci numbers}
a function that computes Fibonacci numbers.
This definition approximates the definition into which \Language{}
transforms the recurrence relation definition of \userfun{fib} in
\spadref{ugUserRecur}.

\begin{xtc}
\begin{xtccomment}
Global variables
\spad{past} and \spad{present} are used
to hold the last computed Fibonacci numbers.
\end{xtccomment}
\begin{spadsrc}
past := present := 1
\end{spadsrc}
\begin{TeXOutput}
\begin{fricasmath}{21}
1%
\end{fricasmath}
\end{TeXOutput}
\formatResultType{PositiveInteger}
\end{xtc}
\begin{xtc}
\begin{xtccomment}
Global variable \spad{index} gives the
current index of \spad{present}.
\end{xtccomment}
\begin{spadsrc}
index := 2
\end{spadsrc}
\begin{TeXOutput}
\begin{fricasmath}{22}
2%
\end{fricasmath}
\end{TeXOutput}
\formatResultType{PositiveInteger}
\end{xtc}
\begin{xtc}
\begin{xtccomment}
Here is a recurrence relation defined in terms
of these three global variables.
\end{xtccomment}
\begin{spadsrc}
fib(n) ==
  free past, present, index
  n < 3 => 1
  n = index - 1 => past
  if n < index-1 then
    (past,present) := (1,1)
    index := 2
  while (index < n) repeat
    (past,present) := (present, past+present)
    index := index + 1
  present
\end{spadsrc}
\end{xtc}
\begin{xtc}
\begin{xtccomment}
Compute the infinite stream of Fibonacci numbers.
\end{xtccomment}
\begin{spadsrc}
fibs := [fib(n) for n in 1..] 
\end{spadsrc}
\begin{MessageOutput}
   Compiling function fib with type PositiveInteger -> PositiveInteger 
\end{MessageOutput}
\begin{TeXOutput}
\begin{fricasmath}{24}
\BRACKET{1\COMMA 1\COMMA 2\COMMA 3\COMMA 5\COMMA 8\COMMA 13\COMMA \STRING{...%
}}%
\end{fricasmath}
\end{TeXOutput}
\formatResultType{Stream(PositiveInteger)}
\end{xtc}
\begin{xtc}
\begin{xtccomment}
What is the 1000th Fibonacci number?
\end{xtccomment}
\begin{spadsrc}
fibs 1000 
\end{spadsrc}
\begin{TeXOutput}
\begin{fricasmath}{25}
43466557686937456435 68852767504062580256 46605173717804024817 29089536555417949051 89040387984007925516 92959225930803226347 75209689623239873322 47116164299644090653 31879382989696499285 16003704476137795166 849228875%
\end{fricasmath}
\end{TeXOutput}
\formatResultType{PositiveInteger}
\end{xtc}

As an exercise, we suggest you write a function in an iterative
style that computes the value of the recurrence relation
$p(n) = p(n-1) - 2 \, p(n-2) + 4 \, p(n-3)$
having the initial values
\mathOrSpad{p(1) = 1},
\mathOrSpad{p(2) = 3}, and
\mathOrSpad{p(3) = 9},
How would you write the function using an element
\spadtype{OneDimensionalArray} or \spadtype{Vector}
to hold the previously computed values?

% *********************************************************************
\head{section}{Anonymous Functions}{ugUserAnon}
% *********************************************************************

% ----------------------------------------------------------------------
\beginImportant
An {\it anonymous function} is a function that is
\index{function!anonymous}
defined
\index{anonymous function}
by giving a list of parameters, the ``maps-to'' compound
symbol \spadSyntax{+->}%
(from the mathematical symbol $\mapsto$)%
, and by an expression involving the parameters, the evaluation of
which determines the return value of the function.

\begin{center}
{\tt ( \subscriptIt{parm}{1}, \subscriptIt{parm}{2}, \ldots, \subscriptIt{parm}{N} ) +-> {\it expression}}
\end{center}
\endImportant
% ----------------------------------------------------------------------

You can apply an anonymous function in several ways.
\begin{enumerate}
\item Place the anonymous function definition in parentheses
directly followed by a list of arguments.
\item Assign the anonymous function to a variable and then
use the variable name when you would normally use a function name.
\item Use \spadSyntax{==} to use the anonymous function definition as
the arguments and body of a regular function definition.
\item Have a named function contain a declared anonymous function and
use the result returned by the named function.
\end{enumerate}

% *********************************************************************
\head{subsection}{Some Examples}{ugUserAnonExamp}
% *********************************************************************

Anonymous functions are particularly useful for defining functions
``on the fly.'' That is, they are handy for simple functions that
are used only in one place.
In the following examples, we show how to write some simple
anonymous functions.

\begin{xtc}
\begin{xtccomment}
This is a simple absolute value function.
\end{xtccomment}
\begin{spadsrc}
x +-> if x < 0 then -x else x 
\end{spadsrc}
\begin{TeXOutput}
\begin{fricasmath}{1}
\SYMBOL{x}\mapsto \SYMBOL{if}\ \SYMBOL{x}<0\ \begin{PILE}\SYMBOL{then}\ -{%
\SYMBOL{x}}\\\SYMBOL{else}\ \SYMBOL{x}\end{PILE}%
\end{fricasmath}
\end{TeXOutput}
\formatResultType{AnonymousFunction}
\end{xtc}
\begin{xtc}
\begin{xtccomment}
\end{xtccomment}
\begin{spadsrc}
abs1 := % 
\end{spadsrc}
\begin{TeXOutput}
\begin{fricasmath}{2}
\SYMBOL{x}\mapsto \SYMBOL{if}\ \SYMBOL{x}<0\ \begin{PILE}\SYMBOL{then}\ -{%
\SYMBOL{x}}\\\SYMBOL{else}\ \SYMBOL{x}\end{PILE}%
\end{fricasmath}
\end{TeXOutput}
\formatResultType{AnonymousFunction}
\end{xtc}
\begin{xtc}
\begin{xtccomment}
This function returns {\tt true} if the absolute value of
the first argument is greater than the absolute value of the
second, {\tt false} otherwise.
\end{xtccomment}
\begin{spadsrc}
(x,y) +-> abs1(x) > abs1(y) 
\end{spadsrc}
\begin{TeXOutput}
\begin{fricasmath}{3}
\PAREN{\SYMBOL{x}\COMMA \SYMBOL{y}}\mapsto \FUN{abs1}\PAREN{\SYMBOL{x}}>\FUN{%
abs1}\PAREN{\SYMBOL{y}}%
\end{fricasmath}
\end{TeXOutput}
\formatResultType{AnonymousFunction}
\end{xtc}
\begin{xtc}
\begin{xtccomment}
We use the above function to ``sort'' a list of integers.
\end{xtccomment}
\begin{spadsrc}
sort(%,[3,9,-4,10,-3,-1,-9,5]) 
\end{spadsrc}
\begin{TeXOutput}
\begin{fricasmath}{4}
\BRACKET{10\COMMA -{9}\COMMA 9\COMMA 5\COMMA -{4}\COMMA -{3}\COMMA 3\COMMA -{%
1}}%
\end{fricasmath}
\end{TeXOutput}
\formatResultType{List(Integer)}
\end{xtc}

\begin{xtc}
\begin{xtccomment}
This function returns \spad{1} if \spad{i + j} is even, \spad{-1} otherwise.
\end{xtccomment}
\begin{spadsrc}
ev := ( (i,j) +-> if even?(i+j) then 1 else -1) 
\end{spadsrc}
\begin{TeXOutput}
\begin{fricasmath}{5}
\PAREN{\SYMBOL{i}\COMMA \SYMBOL{j}}\mapsto \SYMBOL{if}\ \FUN{even?}\PAREN{%
\SYMBOL{i}+\SYMBOL{j}}\ \begin{PILE}\SYMBOL{then}\ 1\\\SYMBOL{else}\ -{1}%
\end{PILE}%
\end{fricasmath}
\end{TeXOutput}
\formatResultType{AnonymousFunction}
\end{xtc}
\begin{xtc}
\begin{xtccomment}
We create a four-by-four matrix containing \spad{1} or \spad{-1}
depending on whether the row plus the column index is even or not.
\end{xtccomment}
\begin{spadsrc}
matrix([[ev(row,col) for row in 1..4] for col in 1..4]) 
\end{spadsrc}
\begin{TeXOutput}
\begin{fricasmath}{6}
\begin{MATRIX}{4}1&-{1}&1&-{1}\\-{1}&1&-{1}&1\\1&-{1}&1&-{1}\\-{1}&1&-{1}&1%
\end{MATRIX}%
\end{fricasmath}
\end{TeXOutput}
\formatResultType{Matrix(Integer)}
\end{xtc}

\begin{xtc}
\begin{xtccomment}
This function returns {\tt true} if a polynomial in \spad{x} has multiple
roots, {\tt false} otherwise.
It is defined and applied in the same expression.
\end{xtccomment}
\begin{spadsrc}
( p +-> not one?(gcd(p,D(p,x))) )(x^2+4*x+4)
\end{spadsrc}
\begin{TeXOutput}
\begin{fricasmath}{7}
\STRING{true}%
\end{fricasmath}
\end{TeXOutput}
\formatResultType{Boolean}
\end{xtc}

\begin{xtc}
\begin{xtccomment}
This and the next expression are equivalent.
\end{xtccomment}
\begin{spadsrc}
g(x,y,z) == cos(x + sin(y + tan(z)))
\end{spadsrc}
\end{xtc}
\begin{xtc}
\begin{xtccomment}
The one you use is a matter of taste.
\end{xtccomment}
\begin{spadsrc}
g == (x,y,z) +-> cos(x + sin(y + tan(z)))
\end{spadsrc}
\begin{MessageOutput}
   1 old definition(s) deleted for function or rule g 
\end{MessageOutput}
\end{xtc}

% *********************************************************************
\head{subsection}{Declaring Anonymous Functions}{ugUserAnonDeclare}
% *********************************************************************

If you declare any of the arguments you must declare all of them.
Thus,
\begin{verbatim}
(x: INT,y): FRAC INT +-> (x + 2*y)/(y - 1)
\end{verbatim}
is not legal.

\begin{xtc}
\begin{xtccomment}
This is an example of a fully declared anonymous
\index{function!declaring}
function.
\index{function!anonymous!declaring}
The output shown just indicates that the object you created is a
particular kind of map, that is, function.
\end{xtccomment}
\begin{spadsrc}
(x: INT,y: INT): FRAC INT +-> (x + 2*y)/(y - 1)
\end{spadsrc}
\begin{TeXOutput}
\begin{fricasmath}{1}
\theMap{anonymousFunction}%
\end{fricasmath}
\end{TeXOutput}
\formatResultType{((Integer, Integer) -> Fraction(Integer))}
\end{xtc}
\begin{xtc}
\begin{xtccomment}
\Language{} allows you to declare the arguments and not declare
the return type.
\end{xtccomment}
\begin{spadsrc}
(x: INT,y: INT) +-> (x + 2*y)/(y - 1)
\end{spadsrc}
\begin{TeXOutput}
\begin{fricasmath}{2}
\theMap{anonymousFunction}%
\end{fricasmath}
\end{TeXOutput}
\formatResultType{((Integer, Integer) -> Fraction(Integer))}
\end{xtc}
The return type is computed from the types of the arguments and the
body of the function.
You cannot declare the return type if you do not declare the arguments.
Therefore,
\begin{verbatim}
(x,y): FRAC INT +-> (x + 2*y)/(y - 1)
\end{verbatim}
is not legal.

\begin{xtc}
\begin{xtccomment}
This and the next expression are equivalent.
\end{xtccomment}
\begin{spadsrc}
h(x: INT,y: INT): FRAC INT == (x + 2*y)/(y - 1)
\end{spadsrc}
\begin{MessageOutput}
   Function declaration h : (Integer,Integer) -> Fraction(Integer) has 
      been added to workspace.
\end{MessageOutput}
\end{xtc}
\begin{xtc}
\begin{xtccomment}
The one you use is a matter of taste.
\end{xtccomment}
\begin{spadsrc}
h == (x: INT,y: INT): FRAC INT +-> (x + 2*y)/(y - 1)
\end{spadsrc}
\begin{MessageOutput}
   Function declaration h : (Integer,Integer) -> Fraction(Integer) has 
      been added to workspace.
\end{MessageOutput}
\begin{MessageOutput}
   1 old definition(s) deleted for function or rule h 
\end{MessageOutput}
\end{xtc}

When should you declare an anonymous function?
\begin{enumerate}
\item If you use an anonymous function and \Language{} can't figure
out what you are trying to do, declare the function.
\item If the function has nontrivial argument types or a
nontrivial return type that
\Language{} may be able to determine eventually, but you are not
willing to wait that long, declare the function.
\item If the function will only be used for arguments of specific
types and it is not too much trouble to declare the function, do so.
\item If you are using the anonymous function as an argument to
another function (such as \spadfun{map} or \spadfun{sort}),
consider declaring the function.
\item If you define an anonymous function inside a named function,
you {\it must} declare the anonymous function.
\end{enumerate}

\begin{xtc}
\begin{xtccomment}
This is an example of a named function for integers that returns a
function.
\end{xtccomment}
\begin{spadsrc}
addx x == ((y: Integer): Integer +-> x + y) 
\end{spadsrc}
\end{xtc}
\begin{xtc}
\begin{xtccomment}
We define \userfun{g} to be a function that adds \spad{10} to its
argument.
\end{xtccomment}
\begin{spadsrc}
g := addx 10 
\end{spadsrc}
\begin{MessageOutput}
   Compiling function addx with type PositiveInteger -> (Integer -> 
      Integer) 
\end{MessageOutput}
\begin{TeXOutput}
\begin{fricasmath}{6}
\theMap{?}%
\end{fricasmath}
\end{TeXOutput}
\formatResultType{(Integer -> Integer)}
\end{xtc}
\begin{xtc}
\begin{xtccomment}
Try it out.
\end{xtccomment}
\begin{spadsrc}
g 3 
\end{spadsrc}
\begin{TeXOutput}
\begin{fricasmath}{7}
13%
\end{fricasmath}
\end{TeXOutput}
\formatResultType{PositiveInteger}
\end{xtc}
\begin{xtc}
\begin{xtccomment}
\end{xtccomment}
\begin{spadsrc}
g(-4) 
\end{spadsrc}
\begin{TeXOutput}
\begin{fricasmath}{8}
6%
\end{fricasmath}
\end{TeXOutput}
\formatResultType{PositiveInteger}
\end{xtc}

\index{function!anonymous!restrictions}
An anonymous function cannot be recursive: since it does not have a
name, you cannot even call it within itself!
If you place an anonymous function inside a named function, the
anonymous function must be declared.

% *********************************************************************
\head{section}{Example: A Database}{ugUserDatabase}
% *********************************************************************

This example shows how you can use \Language{} to organize a database of
lineage data and then query the database for relationships.

\begin{xtc}
\begin{xtccomment}
The database is entered as ``assertions'' that are really
pieces of a function definition.
\end{xtccomment}
\begin{spadsrc}
children("albert") == ["albertJr","richard","diane"]
\end{spadsrc}
\end{xtc}
\begin{xtc}
\begin{xtccomment}
Each piece
\spad{children(x) == y} means
``the children of \spad{x} are \spad{y}''.
\end{xtccomment}
\begin{spadsrc}
children("richard") == ["douglas","daniel","susan"]
\end{spadsrc}
\end{xtc}
\begin{xtc}
\begin{xtccomment}
This family tree thus spans four generations.
\end{xtccomment}
\begin{spadsrc}
children("douglas") == ["dougie","valerie"]
\end{spadsrc}
\end{xtc}
\begin{xtc}
\begin{xtccomment}
Say ``no one else has children.''
\end{xtccomment}
\begin{spadsrc}
children(x) == []
\end{spadsrc}
\end{xtc}

\begin{xtc}
\begin{xtccomment}
We need some functions for computing lineage.
Start with \spad{childOf}.
\end{xtccomment}
\begin{spadsrc}
childOf(x,y) == member?(x,children(y))
\end{spadsrc}
\end{xtc}
\begin{xtc}
\begin{xtccomment}
To find the \spad{parentOf} someone,
you have to scan the database of
people applying \spad{children}.
\end{xtccomment}
\begin{spadsrc}
parentOf(x) ==
  for y in people repeat
    (if childOf(x,y) then return y)
  "unknown"
\end{spadsrc}
\end{xtc}
\begin{xtc}
\begin{xtccomment}
And a grandparent of \spad{x} is just a parent of a parent of \spad{x}.
\end{xtccomment}
\begin{spadsrc}
grandParentOf(x) == parentOf parentOf x
\end{spadsrc}
\end{xtc}
\begin{xtc}
\begin{xtccomment}
The grandchildren of \spad{x}
are the people \spad{y} such that
\spad{x} is a grandparent of \spad{y}.
\end{xtccomment}
\begin{spadsrc}
grandchildren(x) == [y for y in people | grandParentOf(y) = x]
\end{spadsrc}
\end{xtc}
\begin{xtc}
\begin{xtccomment}
Suppose you want to make a list of all great-grandparents.
Well, a great-grandparent is a grandparent of a person who has children.
\end{xtccomment}
\begin{spadsrc}
greatGrandParents == [x for x in people |
  reduce(_or,[not empty? children(y) for y in grandchildren(x)],false)]
\end{spadsrc}
\end{xtc}
\begin{xtc}
\begin{xtccomment}
Define \spad{descendants} to include the parent as well.
\end{xtccomment}
\begin{spadsrc}
descendants(x) ==
  kids := children(x)
  null kids => [x]
  concat(x,reduce(concat,[descendants(y)
    for y in kids],[]))
\end{spadsrc}
\end{xtc}
\begin{xtc}
\begin{xtccomment}
Finally, we need a list of people.
Since all people are descendants of ``albert'', let's say so.
\end{xtccomment}
\begin{spadsrc}
people == descendants "albert"
\end{spadsrc}
\end{xtc}

We have used \spadSyntax{==} to define the database and some functions to
query the database.
But no computation is done until we ask for some information.
Then, once and for all, the functions are analyzed and compiled to machine
code for run-time efficiency.
Notice that no types are given anywhere in this example.
They are not needed.

\begin{xtc}
\begin{xtccomment}
Who are the grandchildren of ``richard''?
\end{xtccomment}
\begin{spadsrc}
grandchildren "richard"
\end{spadsrc}
\begin{MessageOutput}
   Compiling function children with type String -> List(String) 
\end{MessageOutput}
\begin{MessageOutput}
   Compiling function descendants with type String -> List(String) 
\end{MessageOutput}
\begin{MessageOutput}
   Compiling body of rule people to compute value of type List(String) 
\end{MessageOutput}
\begin{MessageOutput}
   Compiling function childOf with type (String,String) -> Boolean 
\end{MessageOutput}
\begin{MessageOutput}
   Compiling function parentOf with type String -> String 
\end{MessageOutput}
\begin{MessageOutput}
   Compiling function grandParentOf with type String -> String 
\end{MessageOutput}
\begin{MessageOutput}
   Compiling function grandchildren with type String -> List(String) 
\end{MessageOutput}
\begin{TeXOutput}
\begin{fricasmath}{12}
\BRACKET{\STRING{"dougie"}\COMMA \STRING{"valerie"}}%
\end{fricasmath}
\end{TeXOutput}
\formatResultType{List(String)}
\end{xtc}
\begin{xtc}
\begin{xtccomment}
Who are the great-grandparents?
\end{xtccomment}
\begin{spadsrc}
greatGrandParents
\end{spadsrc}
\begin{MessageOutput}
   Compiling body of rule greatGrandParents to compute value of type 
      List(String) 
\end{MessageOutput}
\begin{TeXOutput}
\begin{fricasmath}{13}
\BRACKET{\STRING{"albert"}}%
\end{fricasmath}
\end{TeXOutput}
\formatResultType{List(String)}
\end{xtc}

% *********************************************************************
\head{section}{Example: A Famous Triangle}{ugUserTriangle}
% *********************************************************************

In this example we write some functions that display
Pascal's triangle.
\index{Pascal's triangle}
It demonstrates the use of piece-wise definitions and some output
operations you probably haven't seen before.

\begin{xtc}
\begin{xtccomment}
To make these output operations
available, we have to \spadgloss{expose} the domain
\spadtype{OutputForm}.
\exptypeindex{OutputForm}
See \spadref{ugTypesExpose} for more information about exposing domains
and packages.
\end{xtccomment}
\begin{spadsrc}
)set expose add constructor OutputForm 
\end{spadsrc}
\begin{SysCmdOutput}
   OutputForm is now explicitly exposed in frame initial 
\end{SysCmdOutput}
\end{xtc}
\begin{xtc}
\begin{xtccomment}
Define the values along the first
row and any column \spad{i}.
\end{xtccomment}
\begin{spadsrc}
pascal(1,i) == 1 
\end{spadsrc}
\end{xtc}
\begin{xtc}
\begin{xtccomment}
Define the values for when the row
and column index \spad{i} are equal.
Repeating the argument name indicates that
the two index values are equal.
\end{xtccomment}
\begin{spadsrc}
pascal(n,n) == 1 
\end{spadsrc}
\end{xtc}
\begin{xtc}
\begin{xtccomment}
\end{xtccomment}
\begin{spadsrc}
pascal(i,j | 1 < i and i < j) ==
   pascal(i-1,j-1)+pascal(i,j-1)
\end{spadsrc}
\end{xtc}
Now that we have defined the coefficients in Pascal's triangle,
let's write a couple of one-liners to display it.
\begin{xtc}
\begin{xtccomment}
First, define a function that gives the \eth{n} row.
\end{xtccomment}
\begin{spadsrc}
pascalRow(n) == [pascal(i,n) for i in 1..n] 
\end{spadsrc}
\end{xtc}
\begin{xtc}
\begin{xtccomment}
Next, we write the function \userfun{displayRow}
to display the row, separating entries by blanks and centering.
\end{xtccomment}
\begin{spadsrc}
displayRow(n) == output center blankSeparate pascalRow(n) 
\end{spadsrc}
\end{xtc}
%
Here we have used three output operations.
Operation \spadfunFrom{output}{OutputForm}
displays the printable form of objects on the screen,
\spadfunFrom{center}{OutputForm} centers a printable form in the
width of the screen, and \spadfunFrom{blankSeparate}{OutputForm} takes a list of
printable forms and inserts a blank between successive elements.
\begin{xtc}
\begin{xtccomment}
Look at the result.
\end{xtccomment}
\begin{spadsrc}
for i in 1..7 repeat displayRow i 
\end{spadsrc}
\begin{MessageOutput}
   Compiling function pascal with type (Integer,Integer) -> 
      PositiveInteger 
\end{MessageOutput}
\begin{MessageOutput}
   Compiling function pascalRow with type PositiveInteger -> List(
      PositiveInteger) 
\end{MessageOutput}
\begin{MessageOutput}
   Compiling function displayRow with type PositiveInteger -> Void 
\end{MessageOutput}
\end{xtc}
Being purists, we find this less than satisfactory.
Traditionally, elements of Pascal's triangle are centered between
the left and right elements on the line above.
%
\begin{xtc}
\begin{xtccomment}
To fix this misalignment, we go back and
redefine \userfun{pascalRow} to right adjust the entries within the
triangle within a width of four characters.
\end{xtccomment}
\begin{spadsrc}
pascalRow(n) == [right(pascal(i,n),4) for i in 1..n] 
\end{spadsrc}
\begin{MessageOutput}
   Compiled code for pascalRow has been cleared.
\end{MessageOutput}
\begin{MessageOutput}
   Compiled code for displayRow has been cleared.
\end{MessageOutput}
\begin{MessageOutput}
   1 old definition(s) deleted for function or rule pascalRow 
\end{MessageOutput}
\end{xtc}
%
\begin{xtc}
\begin{xtccomment}
Finally let's look at our purely reformatted triangle.
\end{xtccomment}
\begin{spadsrc}
for i in 1..7 repeat displayRow i 
\end{spadsrc}
\begin{MessageOutput}
   Compiling function pascalRow with type PositiveInteger -> List(
      OutputForm) 
\end{MessageOutput}
\begin{MessageOutput}
   Compiling function displayRow with type PositiveInteger -> Void 
\end{MessageOutput}
\end{xtc}
\begin{xtc}
\begin{xtccomment}
Unexpose \spadtype{OutputForm} so we don't get unexpected
results later.
\end{xtccomment}
\begin{spadsrc}
)set expose drop constructor OutputForm
\end{spadsrc}
\begin{SysCmdOutput}
   OutputForm is now explicitly hidden in frame initial 
\end{SysCmdOutput}
\end{xtc}

% *********************************************************************
\head{section}{Example: Testing for Palindromes}{ugUserPal}
% *********************************************************************


In this section we define a function \userfun{pal?} that tests whether its
\index{palindrome}
argument is a {\it palindrome}, that is, something that reads the same
backwards and forwards.
For example, the string ``Madam I'm Adam'' is a palindrome (excluding blanks
and punctuation) and so is the number \spad{123454321.}
The definition works for any datatype that has \spad{n} components that
are accessed by the indices \mathOrSpad{1\ldots n}.

\begin{xtc}
\begin{xtccomment}
Here is the definition for \userfun{pal?}.
It is simply a call to an auxiliary function called
\userfun{palAux?}.
We are following the convention of ending a function's name with
\spadSyntax{?} if the function returns a \spadtype{Boolean} value.
\end{xtccomment}
\begin{spadsrc}
pal? s ==  palAux?(s,1,#s) 
\end{spadsrc}
\end{xtc}
\begin{xtc}
\begin{xtccomment}
Here is \userfun{palAux?}.
It works by comparing elements that are equidistant from the start and end
of the object.
\end{xtccomment}
\begin{spadsrc}
palAux?(s,i,j) ==
  j > i =>
    (s.i = s.j) and palAux?(s,i+1,i-1)
  true
\end{spadsrc}
\end{xtc}
\begin{xtc}
\begin{xtccomment}
Try \userfun{pal?} on some examples.
First, a string.
\end{xtccomment}
\begin{spadsrc}
pal? "Oxford"  
\end{spadsrc}
\begin{MessageOutput}
   Compiling function palAux? with type (String,Integer,Integer) -> 
      Boolean 
\end{MessageOutput}
\begin{MessageOutput}
   Compiling function pal? with type String -> Boolean 
\end{MessageOutput}
\begin{TeXOutput}
\begin{fricasmath}{3}
\STRING{false}%
\end{fricasmath}
\end{TeXOutput}
\formatResultType{Boolean}
\end{xtc}
\begin{xtc}
\begin{xtccomment}
A list of polynomials.
\end{xtccomment}
\begin{spadsrc}
pal? [4,a,x-1,0,x-1,a,4]  
\end{spadsrc}
\begin{MessageOutput}
   Compiling function palAux? with type (List(Polynomial(Integer)),
      Integer,Integer) -> Boolean 
\end{MessageOutput}
\begin{MessageOutput}
   Compiling function pal? with type List(Polynomial(Integer)) -> 
      Boolean 
\end{MessageOutput}
\begin{TeXOutput}
\begin{fricasmath}{4}
\STRING{true}%
\end{fricasmath}
\end{TeXOutput}
\formatResultType{Boolean}
\end{xtc}
\begin{xtc}
\begin{xtccomment}
A list of integers from the example in
the last section.
\end{xtccomment}
\begin{spadsrc}
pal? [1,6,15,20,15,6,1] 
\end{spadsrc}
\begin{MessageOutput}
   Compiling function palAux? with type (List(PositiveInteger),Integer,
      Integer) -> Boolean 
\end{MessageOutput}
\begin{MessageOutput}
   Compiling function pal? with type List(PositiveInteger) -> Boolean 
\end{MessageOutput}
\begin{TeXOutput}
\begin{fricasmath}{5}
\STRING{true}%
\end{fricasmath}
\end{TeXOutput}
\formatResultType{Boolean}
\end{xtc}
\begin{xtc}
\begin{xtccomment}
To use \userfun{pal?} on an integer, first convert it to a string.
\end{xtccomment}
\begin{spadsrc}
pal?(1441::String)
\end{spadsrc}
\begin{TeXOutput}
\begin{fricasmath}{6}
\STRING{true}%
\end{fricasmath}
\end{TeXOutput}
\formatResultType{Boolean}
\end{xtc}
\begin{xtc}
\begin{xtccomment}
Compute an infinite stream of decimal numbers,
each of which is an obvious palindrome.
\end{xtccomment}
\begin{spadsrc}
ones := [reduce(+,[10^j for j in 0..i]) for i in 1..]
\end{spadsrc}
\begin{TeXOutput}
\begin{fricasmath}{7}
\BRACKET{11\COMMA 111\COMMA 1111\COMMA 11111\COMMA 111111\COMMA 1111111%
\COMMA 11111111\COMMA \STRING{...}}%
\end{fricasmath}
\end{TeXOutput}
\formatResultType{Stream(PositiveInteger)}
\end{xtc}
\begin{discard}
\begin{noOutputXtc}
\begin{xtccomment}
\end{xtccomment}
\begin{spadsrc}
)set streams calculate 9
\end{spadsrc}
\end{noOutputXtc}
\end{discard}
\begin{xtc}
\begin{xtccomment}
How about their squares?
\end{xtccomment}
\begin{spadsrc}
squares := [x^2 for x in ones]
\end{spadsrc}
\begin{TeXOutput}
\begin{fricasmath}{8}
\BRACKET{121\COMMA 12321\COMMA 1234321\COMMA 123454321\COMMA 12345654321%
\COMMA 1234567654321\COMMA 123456787654321\COMMA 12345678987654321\COMMA %
1234567900987654321\COMMA \STRING{...}}%
\end{fricasmath}
\end{TeXOutput}
\formatResultType{Stream(PositiveInteger)}
\end{xtc}
\begin{xtc}
\begin{xtccomment}
Well, let's test them all!
\end{xtccomment}
\begin{spadsrc}
[pal?(x::String) for x in squares]
\end{spadsrc}
\begin{TeXOutput}
\begin{fricasmath}{9}
\BRACKET{\STRING{true}\COMMA \STRING{true}\COMMA \STRING{true}\COMMA \STRING{%
true}\COMMA \STRING{true}\COMMA \STRING{true}\COMMA \STRING{true}\COMMA %
\STRING{true}\COMMA \STRING{true}\COMMA \STRING{...}}%
\end{fricasmath}
\end{TeXOutput}
\formatResultType{Stream(Boolean)}
\end{xtc}
\begin{discard}
\begin{noOutputXtc}
\begin{xtccomment}
\end{xtccomment}
\begin{spadsrc}
)set streams calculate 7
\end{spadsrc}
\end{noOutputXtc}
\end{discard}

% *********************************************************************
\head{section}{Rules and Pattern Matching}{ugUserRules}
% *********************************************************************

A common mathematical formula is
\begin{displaymath}
\log(x) + \log(y) = \log(x y) \quad\forall \, x \hbox{\ and\ } y.
\end{displaymath}
The presence of
``$\forall$''
indicates that \spad{x} and \spad{y} can stand for arbitrary mathematical
expressions in the above formula.
You can use such mathematical formulas in \Language{} to specify ``rewrite
rules''.
Rewrite rules are objects in \Language{} that can be assigned to variables for
later use, often for the purpose of simplification.
Rewrite rules look like ordinary function definitions except that they are
preceded by the reserved word \spad{rule}.
\spadkey{rule}
For example, a rewrite rule for the above formula is:
\begin{verbatim}
rule log(x) + log(y) == log(x * y)
\end{verbatim}
Like function definitions, no action is taken when a rewrite rule is issued.
Think of rewrite rules as functions that take one argument.
When a rewrite rule \spad{A = B} is applied to an argument \spad{f}, its
meaning is: ``rewrite every subexpression of \spad{f} that {\it matches}
\spad{A} by \spad{B.}''
The left-hand side of a rewrite rule is called a \spadgloss{pattern}; its
right-side side is called its \spadgloss{substitution}.

\begin{xtc}
\begin{xtccomment}
Create a rewrite rule named \userfun{logrule}.
The generated symbol beginning with a \spadSyntax{%} is a place-holder
for any other terms that might occur in the sum.
\end{xtccomment}
\begin{spadsrc}
logrule := rule log(x) + log(y) == log(x * y) 
\end{spadsrc}
\begin{TeXOutput}
\begin{fricasmath}{1}
\log{\SYMBOL{y}}+\log{\SYMBOL{x}}+\SYMBOL{\%B}\SYMBOL{\ ==\ }\log{\PAREN{%
\SYMBOL{x}\TIMES \SYMBOL{y}}}+\SYMBOL{\%B}%
\end{fricasmath}
\end{TeXOutput}
\formatResultType{RewriteRule(Integer, Integer, Expression(Integer))}
\end{xtc}
\begin{xtc}
\begin{xtccomment}
Create an expression with logarithms.
\end{xtccomment}
\begin{spadsrc}
f := log sin x + log x 
\end{spadsrc}
\begin{TeXOutput}
\begin{fricasmath}{2}
\log{\PAREN{\sin{\SYMBOL{x}}}}+\log{\SYMBOL{x}}%
\end{fricasmath}
\end{TeXOutput}
\formatResultType{Expression(Integer)}
\end{xtc}
\begin{xtc}
\begin{xtccomment}
Apply \userfun{logrule} to \spad{f}.
\end{xtccomment}
\begin{spadsrc}
logrule f 
\end{spadsrc}
\begin{TeXOutput}
\begin{fricasmath}{3}
\log{\PAREN{\SYMBOL{x}\TIMES \sin{\SYMBOL{x}}}}%
\end{fricasmath}
\end{TeXOutput}
\formatResultType{Expression(Integer)}
\end{xtc}

The meaning of our example rewrite rule is:
``for all expressions \spad{x} and \spad{y}, rewrite
\spad{log(x) + log(y)} by \spad{log(x * y)}.''
Patterns generally have both operation names
(here, \spadfun{log} and \spadop{+})
and variables (here, \spad{x} and \spad{y}).
By default, every operation name stands for itself.
Thus \spadfun{log}  matches only ``\spad{log}'' and not any
other operation such as \spadfun{sin}.
On the other hand, variables do not stand for themselves.
Rather, a variable denotes a
{\it pattern variable} that is free to match any expression whatsoever.
\index{pattern!variables}

When a rewrite rule is applied, a process called
\spadgloss{pattern matching} goes to work by systematically
scanning
\index{pattern!matching}
the subexpressions of the argument.
When a subexpression is found that ``matches'' the pattern, the subexpression
is replaced by the right-hand side of the rule.
The details of what happens will be covered later.

The customary \Language{} notation for patterns is actually a shorthand for a
longer, more general notation.
Pattern variables can be made explicit by using a percent
(\spadSyntax{%}) as the first character of the variable name.
To say that a name stands for itself, you can prefix that name with a quote
operator (\spadSyntax{'}).
Although the current \Language{} parser does not let you quote an operation
name, this more general notation gives you an alternate way of giving the same
rewrite rule:
\begin{verbatim}
rule log(%x) + log(%y) == log(x * y)
\end{verbatim}
This longer notation gives you patterns that the
standard notation won't handle.
For example, the rule
\typeout{check this example}
\begin{verbatim}
rule %f(c * 'x) ==  c*%f(x)
\end{verbatim}
means ``for all \spad{f} and \spad{c}, replace \spad{f(y)} by
\spad{c * f(x)} when \spad{y} is the product of \spad{c}
and the explicit variable \spad{x}.''

Thus the pattern can have several adornments on the names that appear there.
Normally, all these adornments are dropped in the substitution on the
right-hand side.

To summarize:

% ----------------------------------------------------------------------
\beginImportant
To enter a single rule in \Language{}, use the following syntax:
\spadkey{rule}
\begin{center}
{\tt rule {\it leftHandSide} == {\it rightHandSide}}
\end{center}
The {\it leftHandSide} is a pattern to be matched and
the {\it rightHandSide} is its substitution.
The rule is an object of type \spadtype{RewriteRule} that can be
assigned to a variable and applied to expressions to transform them.
\endImportant
% ----------------------------------------------------------------------

Rewrite rules can be collected
into rulesets so that a set of rules can be applied at once.
Here is another simplification rule for logarithms.
\begin{displaymath}
y \log(x) = \log(x^y) \quad\forall \, x \text{ and } y.
\end{displaymath}
If instead of giving a single rule following the reserved word \spad{rule}
you give a ``pile'' of rules, you create
what is called a {\it ruleset.}
\index{ruleset}
Like rules, rulesets are objects in \Language{} and
can be assigned to variables.
You will find it useful to group commonly used rules into input files, and read
them in as needed.
\begin{xtc}
\begin{xtccomment}
Create a ruleset named \spad{logrules}.
\end{xtccomment}
\begin{spadsrc}
logrules := rule
  log(x) + log(y) == log(x * y)
  y * log x       == log(x ^ y)
\end{spadsrc}
\begin{TeXOutput}
\begin{fricasmath}{4}
\BRACE{\log{\SYMBOL{y}}+\log{\SYMBOL{x}}+\SYMBOL{\%C}\SYMBOL{\ ==\ }\log{%
\PAREN{\SYMBOL{x}\TIMES \SYMBOL{y}}}+\SYMBOL{\%C}\COMMA \SYMBOL{y}\TIMES \log%
{\SYMBOL{x}}\SYMBOL{\ ==\ }\log{\PAREN{\SUPER{\SYMBOL{x}}{\SYMBOL{y}}}}}%
\end{fricasmath}
\end{TeXOutput}
\formatResultType{Ruleset(Integer, Integer, Expression(Integer))}
\end{xtc}
\begin{xtc}
\begin{xtccomment}
Again, create an expression \spad{f} containing logarithms.
\end{xtccomment}
\begin{spadsrc}
f := a * log(sin x) - 2 * log x 
\end{spadsrc}
\begin{TeXOutput}
\begin{fricasmath}{5}
\SYMBOL{a}\TIMES \log{\PAREN{\sin{\SYMBOL{x}}}}-{2\TIMES \log{\SYMBOL{x}}}%
\end{fricasmath}
\end{TeXOutput}
\formatResultType{Expression(Integer)}
\end{xtc}
\begin{xtc}
\begin{xtccomment}
Apply the ruleset \userfun{logrules} to \spad{f}.
\end{xtccomment}
\begin{spadsrc}
logrules f 
\end{spadsrc}
\begin{TeXOutput}
\begin{fricasmath}{6}
\log{\PAREN{\frac{\SUPER{\PAREN{\sin{\SYMBOL{x}}}}{\SYMBOL{a}}}{\SUPER{%
\SYMBOL{x}}{2}}}}%
\end{fricasmath}
\end{TeXOutput}
\formatResultType{Expression(Integer)}
\end{xtc}

We have allowed pattern variables to match arbitrary expressions in the
above examples.
Often you want a variable only to match expressions
satisfying some predicate.
For example, we may want to apply the transformation
\begin{displaymath}
y \log(x) = \log(x^y)
\end{displaymath}
only when \spad{y} is an integer.
%
The way to restrict a pattern variable \spad{y} by a predicate \spad{f(y)}
\index{pattern!variable!predicate}
is by using a vertical bar \spadSyntax{|}, which means ``such that,'' in
\index{such that}
much the same way it is used in function definitions.
\index{predicate!on a pattern variable}
You do this only once, but at the earliest
(meaning deepest and leftmost) part of the pattern.
\begin{xtc}
\begin{xtccomment}
This restricts the logarithmic rule to create integer exponents only.
\end{xtccomment}
\begin{spadsrc}
logrules2 := rule
  log(x) + log(y)          == log(x * y)
  (y | integer? y) * log x == log(x ^ y)
\end{spadsrc}
\begin{TeXOutput}
\begin{fricasmath}{7}
\BRACE{\log{\SYMBOL{y}}+\log{\SYMBOL{x}}+\SYMBOL{\%E}\SYMBOL{\ ==\ }\log{%
\PAREN{\SYMBOL{x}\TIMES \SYMBOL{y}}}+\SYMBOL{\%E}\COMMA \SYMBOL{y}\TIMES \log%
{\SYMBOL{x}}\SYMBOL{\ ==\ }\log{\PAREN{\SUPER{\SYMBOL{x}}{\SYMBOL{y}}}}}%
\end{fricasmath}
\end{TeXOutput}
\formatResultType{Ruleset(Integer, Integer, Expression(Integer))}
\end{xtc}
\begin{xtc}
\begin{xtccomment}
Compare this with the result of applying the previous set of rules.
\end{xtccomment}
\begin{spadsrc}
f 
\end{spadsrc}
\begin{TeXOutput}
\begin{fricasmath}{8}
\SYMBOL{a}\TIMES \log{\PAREN{\sin{\SYMBOL{x}}}}-{2\TIMES \log{\SYMBOL{x}}}%
\end{fricasmath}
\end{TeXOutput}
\formatResultType{Expression(Integer)}
\end{xtc}
\begin{xtc}
\begin{xtccomment}
\end{xtccomment}
\begin{spadsrc}
logrules2 f 
\end{spadsrc}
\begin{TeXOutput}
\begin{fricasmath}{9}
\SYMBOL{a}\TIMES \log{\PAREN{\sin{\SYMBOL{x}}}}+\log{\PAREN{\frac{1}{\SUPER{%
\SYMBOL{x}}{2}}}}%
\end{fricasmath}
\end{TeXOutput}
\formatResultType{Expression(Integer)}
\end{xtc}
You should be aware that you might need to apply a function like
\spadfun{integer} within your predicate expression to actually apply
the test function.
\begin{xtc}
\begin{xtccomment}
Here we use \spadfun{integer} because \spad{n} has
type \spadtype{Expression Integer} but \spadfun{even?} is an operation
defined on integers.
\end{xtccomment}
\begin{spadsrc}
evenRule := rule cos(x)^(n | integer? n and even? integer n)==(1-sin(x)^2)^(n/2) 
\end{spadsrc}
\begin{TeXOutput}
\begin{fricasmath}{10}
\SUPER{\PAREN{\cos{\SYMBOL{x}}}}{\SYMBOL{n}}\SYMBOL{\ ==\ }\SUPER{\PAREN{-{%
\SUPER{\PAREN{\sin{\SYMBOL{x}}}}{2}}+1}}{\frac{\SYMBOL{n}}{2}}%
\end{fricasmath}
\end{TeXOutput}
\formatResultType{RewriteRule(Integer, Integer, Expression(Integer))}
\end{xtc}
\begin{xtc}
\begin{xtccomment}
Here is the application of the rule.
\end{xtccomment}
\begin{spadsrc}
evenRule( cos(x)^2 ) 
\end{spadsrc}
\begin{TeXOutput}
\begin{fricasmath}{11}
-{\SUPER{\PAREN{\sin{\SYMBOL{x}}}}{2}}+1%
\end{fricasmath}
\end{TeXOutput}
\formatResultType{Expression(Integer)}
\end{xtc}
\begin{xtc}
\begin{xtccomment}
This is an example of some of the usual identities involving products of
sines and cosines.
\end{xtccomment}
\begin{spadsrc}
sinCosProducts == rule
  sin(x) * sin(y) == (cos(x-y) - cos(x + y))/2
  cos(x) * cos(y) == (cos(x-y) + cos(x+y))/2
  sin(x) * cos(y) == (sin(x-y) + sin(x + y))/2
\end{spadsrc}
\end{xtc}
\begin{xtc}
\begin{xtccomment}
\end{xtccomment}
\begin{spadsrc}
g := sin(a)*sin(b) + cos(b)*cos(a) + sin(2*a)*cos(2*a) 
\end{spadsrc}
\begin{TeXOutput}
\begin{fricasmath}{13}
\sin{\SYMBOL{a}}\TIMES \sin{\SYMBOL{b}}+\cos{\PAREN{2\TIMES \SYMBOL{a}}}%
\TIMES \sin{\PAREN{2\TIMES \SYMBOL{a}}}+\cos{\SYMBOL{a}}\TIMES \cos{\SYMBOL{b%
}}%
\end{fricasmath}
\end{TeXOutput}
\formatResultType{Expression(Integer)}
\end{xtc}
\begin{xtc}
\begin{xtccomment}
\end{xtccomment}
\begin{spadsrc}
sinCosProducts g 
\end{spadsrc}
\begin{MessageOutput}
   Compiling body of rule sinCosProducts to compute value of type 
      Ruleset(Integer,Integer,Expression(Integer)) 
\end{MessageOutput}
\begin{TeXOutput}
\begin{fricasmath}{14}
\frac{\sin{\PAREN{4\TIMES \SYMBOL{a}}}+2\TIMES \cos{\PAREN{\SYMBOL{b}-{%
\SYMBOL{a}}}}}{2}%
\end{fricasmath}
\end{TeXOutput}
\formatResultType{Expression(Integer)}
\end{xtc}

Another qualification you will often want to use is to allow a pattern to
match an identity element.
Using the pattern \spad{x + y}, for example, neither \spad{x} nor \spad{y}
matches the expression \spad{0}.
Similarly, if a pattern contains a product \spad{x*y} or an exponentiation
\spad{x^y}, then neither \spad{x} or \spad{y} matches \spad{1}.
%
\begin{xtc}
\begin{xtccomment}
If identical elements were matched, pattern matching would generally loop.
Here is an expansion rule for exponentials.
\end{xtccomment}
\begin{spadsrc}
exprule := rule exp(a + b) == exp(a) * exp(b)
\end{spadsrc}
\begin{TeXOutput}
\begin{fricasmath}{15}
\SUPER{\EulerE }{\SYMBOL{b}+\SYMBOL{a}}\SYMBOL{\ ==\ }\SUPER{\EulerE }{%
\SYMBOL{a}}\TIMES \SUPER{\EulerE }{\SYMBOL{b}}%
\end{fricasmath}
\end{TeXOutput}
\formatResultType{RewriteRule(Integer, Integer, Expression(Integer))}
\end{xtc}
\begin{xtc}
\begin{xtccomment}
This rule would cause infinite rewriting on this if either \spad{a} or
\spad{b} were allowed to match \spad{0}.
\end{xtccomment}
\begin{spadsrc}
exprule exp x 
\end{spadsrc}
\begin{TeXOutput}
\begin{fricasmath}{16}
\SUPER{\EulerE }{\SYMBOL{x}}%
\end{fricasmath}
\end{TeXOutput}
\formatResultType{Expression(Integer)}
\end{xtc}
%
There are occasions when you do want a pattern variable in a sum or
product to match \spad{0} or \spad{1}.
If so, prefix its name
with a \spadSyntax{?} whenever it appears in a left-hand side of a rule.
For example, consider the following rule for the exponential integral:
\begin{displaymath}
  \int \left(\frac{y+e^x}{x}\right)\: dx = \int \frac{y}{x}\: dx + \hbox{\rm Ei}(x)
  \quad\forall \, x \hbox{\ and\ } y.
\end{displaymath}
This rule is valid for \spad{y = 0}.
One solution is to create a \spadtype{Ruleset} with two
rules, one with and one without \spad{y}.
A better solution is to use an ``optional'' pattern variable.
%
\begin{xtc}
\begin{xtccomment}
Define rule \spad{eirule} with
a pattern variable \spad{?y} to indicate
that an expression may or may not occur.
\end{xtccomment}
\begin{spadsrc}
eirule := rule integral((?y + exp x)/x,x) == integral(y/x,x) + Ei x 
\end{spadsrc}
\begin{TeXOutput}
\begin{fricasmath}{17}
\int^{\SYMBOL{x}} \frac{\SUPER{\EulerE }{\SYMBOL{\%D}}+\SYMBOL{y}}{\SYMBOL{%
\%D}}\TIMES \SYMBOL{d}\SYMBOL{\%D}\SYMBOL{\ ==\ }\QUOTE{\SYMBOL{integral}}%
\PAREN{\frac{\SYMBOL{y}}{\SYMBOL{x}},\SYMBOL{x}}+\QUOTE{\SYMBOL{Ei}}\PAREN{%
\SYMBOL{x}}%
\end{fricasmath}
\end{TeXOutput}
\formatResultType{RewriteRule(Integer, Integer, Expression(Integer))}
\end{xtc}
\begin{xtc}
\begin{xtccomment}
Apply rule \spad{eirule} to an integral without this term.
\end{xtccomment}
\begin{spadsrc}
eirule integral(exp u/u, u) 
\end{spadsrc}
\begin{TeXOutput}
\begin{fricasmath}{18}
\FUN{Ei}\PAREN{\SYMBOL{u}}%
\end{fricasmath}
\end{TeXOutput}
\formatResultType{Expression(Integer)}
\end{xtc}
\begin{xtc}
\begin{xtccomment}
Apply rule \spad{eirule} to an integral with this term.
\end{xtccomment}
\begin{spadsrc}
eirule integral(sin u + exp u/u, u) 
\end{spadsrc}
\begin{TeXOutput}
\begin{fricasmath}{19}
\int^{\SYMBOL{u}} \sin{\SYMBOL{\%D}}\TIMES \SYMBOL{d}\SYMBOL{\%D}+\FUN{Ei}%
\PAREN{\SYMBOL{u}}%
\end{fricasmath}
\end{TeXOutput}
\formatResultType{Expression(Integer)}
\end{xtc}

Here is one final adornment you will find useful.
When matching a pattern of the form \spad{x + y} to an expression containing a
long sum of the form \mathOrSpad{a +\ldots+ b}, there is no way to predict in
advance which subset of the sum  matches \spad{x} and which matches
\spad{y}.
Aside from efficiency, this is generally unimportant since the rule holds for
any possible combination of matches for \spad{x} and \spad{y}.
In some situations, however, you may want to say which pattern variable is a sum
(or product) of several terms, and which should match only a single term.
To do this, put a prefix colon \spadSyntax{:} before the pattern variable
that you want to match multiple terms.
\index{pattern!variable!matching several terms}
%
\begin{xtc}
\begin{xtccomment}
The remaining rules involve operators \spad{u} and \spad{v}.
\index{operator}
\end{xtccomment}
\begin{spadsrc}
u := operator 'u 
\end{spadsrc}
\begin{TeXOutput}
\begin{fricasmath}{20}
\SYMBOL{u}%
\end{fricasmath}
\end{TeXOutput}
\formatResultType{BasicOperator}
\end{xtc}
\begin{xtc}
\begin{xtccomment}
These definitions tell \Language{} that
\spad{u} and \spad{v} are formal operators to be used in expressions.
\end{xtccomment}
\begin{spadsrc}
v := operator 'v 
\end{spadsrc}
\begin{TeXOutput}
\begin{fricasmath}{21}
\SYMBOL{v}%
\end{fricasmath}
\end{TeXOutput}
\formatResultType{BasicOperator}
\end{xtc}
\begin{xtc}
\begin{xtccomment}
First define \spad{myRule}
with no restrictions on the pattern variables
\spad{x} and \spad{y}.
\end{xtccomment}
\begin{spadsrc}
myRule := rule u(x + y) == u x + v y 
\end{spadsrc}
\begin{TeXOutput}
\begin{fricasmath}{22}
\FUN{u}\PAREN{\SYMBOL{y}+\SYMBOL{x}}\SYMBOL{\ ==\ }\QUOTE{\SYMBOL{v}}\PAREN{%
\SYMBOL{y}}+\QUOTE{\SYMBOL{u}}\PAREN{\SYMBOL{x}}%
\end{fricasmath}
\end{TeXOutput}
\formatResultType{RewriteRule(Integer, Integer, Expression(Integer))}
\end{xtc}
\begin{xtc}
\begin{xtccomment}
Apply \spad{myRule} to an expression.
\end{xtccomment}
\begin{spadsrc}
myRule u(a + b + c + d) 
\end{spadsrc}
\begin{TeXOutput}
\begin{fricasmath}{23}
\FUN{v}\PAREN{\SYMBOL{d}+\SYMBOL{c}+\SYMBOL{b}}+\FUN{u}\PAREN{\SYMBOL{a}}%
\end{fricasmath}
\end{TeXOutput}
\formatResultType{Expression(Integer)}
\end{xtc}
\begin{xtc}
\begin{xtccomment}
Define \spad{myOtherRule} to match several terms
so that the rule gets applied recursively.
\end{xtccomment}
\begin{spadsrc}
myOtherRule := rule u(:x + y) == u x + v y 
\end{spadsrc}
\begin{TeXOutput}
\begin{fricasmath}{24}
\FUN{u}\PAREN{\SYMBOL{y}+\SYMBOL{x}}\SYMBOL{\ ==\ }\QUOTE{\SYMBOL{v}}\PAREN{%
\SYMBOL{y}}+\QUOTE{\SYMBOL{u}}\PAREN{\SYMBOL{x}}%
\end{fricasmath}
\end{TeXOutput}
\formatResultType{RewriteRule(Integer, Integer, Expression(Integer))}
\end{xtc}
\begin{xtc}
\begin{xtccomment}
Apply \spad{myOtherRule} to the same expression.
\end{xtccomment}
\begin{spadsrc}
myOtherRule u(a + b + c + d) 
\end{spadsrc}
\begin{TeXOutput}
\begin{fricasmath}{25}
\FUN{v}\PAREN{\SYMBOL{c}}+\FUN{v}\PAREN{\SYMBOL{b}}+\FUN{v}\PAREN{\SYMBOL{a}}%
+\FUN{u}\PAREN{\SYMBOL{d}}%
\end{fricasmath}
\end{TeXOutput}
\formatResultType{Expression(Integer)}
\end{xtc}

% ----------------------------------------------------------------------
\beginImportant
Summary of pattern variable adornments:

\vskip .5\baselineskip
\begin{tabular}{@{}ll}
{\tt (x | predicate?(x))} &
  means that the substitution \mathOrSpad{s} for \mathOrSpad{x}\\ &
  must satisfy \textspadexpr{predicate?(s) = true}. \\
{\tt ?x} &
  means that \mathOrSpad{x} can match an identity \\ & element (0 or 1). \\
{\tt :x} &
  means that \mathOrSpad{x} can match several terms \\ & in a sum.
\end{tabular}
\endImportant
% ----------------------------------------------------------------------

Here are some final remarks on pattern matching.
Pattern matching provides a very useful paradigm for solving
certain classes of problems, namely, those that involve
transformations of one form to another and back.
However, it is important to recognize its limitations.
\index{pattern!matching!caveats}

First, pattern matching slows down as the number of rules you have to apply
increases.
Thus it is good practice to organize the sets of rules you use optimally so
that irrelevant rules are never included.

Second, careless use of pattern matching can lead to wrong answers.
You should avoid using pattern matching to handle hidden algebraic
relationships that can go undetected by other programs.
As a simple example, a symbol such as ``J'' can easily be used to represent
the square root of \spad{-1} or some other important algebraic quantity.
Many algorithms branch on whether an expression is zero or not, then divide by
that expression if it is not.
If you fail to simplify an expression involving powers of
\spad{J} to \spad{-1,}
algorithms may incorrectly assume an expression is non-zero, take a wrong
branch, and produce a meaningless result.

Pattern matching should also not be used as a substitute for a domain.
In \Language{}, objects of one domain are transformed to objects of other
domains using well-defined \spadfun{coerce} operations.
Pattern matching should be used on objects that are all the same type.
Thus if your application can be handled by type \spadtype{Expression} in
\Language{} and you think you need pattern matching, consider this choice
carefully.
\exptypeindex{Expression}
You may well be better served by extending an existing domain
or by building a new domain of objects for your application.


%--rhx: TODO:
% Unfortunately ug07 uses \item[\spadfun{foo}] which doesn't work
% since \spadfun has a verbatim-like argument. Fortunately, that only
% happens inside the description environment. So we simply redefine it
% for this chapter. Better rewrite the text in ug07.
\begingroup
\let\spaddescriptionsave\description
\def\description{\def\spadfun{\textspadfun}\spaddescriptionsave}
% !! DO NOT MODIFY THIS FILE BY HAND !! Created by spool2tex.awk.

% Copyright (c) 1991-2002, The Numerical ALgorithms Group Ltd.
% All rights reserved.
%
% Redistribution and use in source and binary forms, with or without
% modification, are permitted provided that the following conditions are
% met:
%
%     - Redistributions of source code must retain the above copyright
%       notice, this list of conditions and the following disclaimer.
%
%     - Redistributions in binary form must reproduce the above copyright
%       notice, this list of conditions and the following disclaimer in
%       the documentation and/or other materials provided with the
%       distribution.
%
%     - Neither the name of The Numerical ALgorithms Group Ltd. nor the
%       names of its contributors may be used to endorse or promote products
%       derived from this software without specific prior written permission.
%
% THIS SOFTWARE IS PROVIDED BY THE COPYRIGHT HOLDERS AND CONTRIBUTORS "AS
% IS" AND ANY EXPRESS OR IMPLIED WARRANTIES, INCLUDING, BUT NOT LIMITED
% TO, THE IMPLIED WARRANTIES OF MERCHANTABILITY AND FITNESS FOR A
% PARTICULAR PURPOSE ARE DISCLAIMED. IN NO EVENT SHALL THE COPYRIGHT OWNER
% OR CONTRIBUTORS BE LIABLE FOR ANY DIRECT, INDIRECT, INCIDENTAL, SPECIAL,
% EXEMPLARY, OR CONSEQUENTIAL DAMAGES (INCLUDING, BUT NOT LIMITED TO,
% PROCUREMENT OF SUBSTITUTE GOODS OR SERVICES-- LOSS OF USE, DATA, OR
% PROFITS-- OR BUSINESS INTERRUPTION) HOWEVER CAUSED AND ON ANY THEORY OF
% LIABILITY, WHETHER IN CONTRACT, STRICT LIABILITY, OR TORT (INCLUDING
% NEGLIGENCE OR OTHERWISE) ARISING IN ANY WAY OUT OF THE USE OF THIS
% SOFTWARE, EVEN IF ADVISED OF THE POSSIBILITY OF SUCH DAMAGE.

% *********************************************************************
\head{chapter}{Graphics}{ugGraph}
% *********************************************************************

%
%
\begin{figure}[htbp]
{\epsfverbosetrue\epsfxsize=12.5pc%
\def\epsfsize#1#2{\epsfxsize}
\epsffile[10 0 300 300]{knot3.ps}}
\vskip -.5\baselineskip
\caption{Torus knot of type (15,17).}
\vskip .5\baselineskip
\end{figure}

This chapter shows how to use the \Language{} graphics facilities
\index{graphics}
under the X Window System.
\Language{} has \twodim{} and \threedim{} drawing and rendering
packages that allow the drawing, coloring, transforming, mapping,
clipping, and combining of graphic output from \Language{}
computations.
This facility is particularly useful for investigating problems in
areas such as topology.
The graphics package is capable of plotting functions of one or
more variables or plotting parametric surfaces and curves.
Various coordinate systems are also available, such as polar and
spherical.

A graph is displayed in a viewport window and it has a
\index{viewport}
control-panel that uses interactive mouse commands.
PostScript and other output forms are available so that \Language{}
\index{PostScript}
images can be printed or used by other programs.\footnote{PostScript
is a trademark of Adobe Systems Incorporated, registered in the United
States.}

% *********************************************************************
\head{section}{Two-Dimensional Graphics}{ugGraphTwoD}
% *********************************************************************
%
The \Language{} \twodim{} graphics package provides the ability to
\index{graphics!two-dimensional}
display
%
\begin{itemize}
%
\item curves defined by functions of a single real variable
%
\item curves defined by parametric equations
%
\item implicit non-singular curves defined by polynomial equations
%
\item planar graphs generated from lists of point components.
\end{itemize}
These graphs
can be modified by specifying various options, such as
calculating points in the polar
coordinate system or changing the size of the graph viewport window.

% *********************************************************************
\head{subsection}{Plotting Two-Dimensional Functions of One Variable}{ugGraphTwoDPlot}
% *********************************************************************

\index{curve!one variable function}
The first kind of \twodim{} graph is that of a curve defined by a function
\spad{y = f(x)} over a finite interval of the \spad{x} axis.

%
\beginImportant
The general format for drawing a function defined by a formula
\spad{f(x)} is:
%
\begin{center}
{\tt draw(f(x), x = a..b, {\it options})}
\end{center}
where \spad{a..b} defines the range of \spad{x}, and where
{\it options} prescribes zero or more options as described in
\spadref{ugGraphTwoDOptions}.
An example of an option is \spad{curveColor == bright red().}
An alternative format involving functions \spad{f} and \spad{g}
is also available.
\endImportant

A simple way to plot a function is to use a formula.
The first argument is the formula.
For the second argument, write the name of the independent variable (here, \spad{x}),
followed by an \spadSyntax{=}, and the range of values.

\begin{psXtc}
\begin{xtccomment}
Display this formula over the range
$0 \leq x \leq 6$.
\Language{} converts your formula to a compiled
function so that the results can be computed
quickly and efficiently.
\end{xtccomment}
\begin{spadsrc}
draw(sin(tan(x)) - tan(sin(x)),x = 0..6)
\end{spadsrc}
% window was 300 x 300
\epsffile[0 0 295 295]{2D1VarA.ps}
\end{psXtc}

Notice that \Language{} compiled the function before the graph was put
on the screen.

\begin{psXtc}
\begin{xtccomment}
Here is the same graph on a different interval.
This time we give the graph a title.
\end{xtccomment}
\begin{spadsrc}
draw(sin(tan(x)) - tan(sin(x)),x = 10..16)
\end{spadsrc}
%window was 300 x 300
\epsffile[0 0 295 295]{2D1VarB.ps}
\end{psXtc}
%
Once again the formula is converted to a compiled function before
any points were computed.
If you want to graph the same function on several intervals, it is
a good idea to define the function first so that the function has
to be compiled only once.
\begin{xtc}
\begin{xtccomment}
This time we first define the function.
\end{xtccomment}
\begin{spadsrc}
f(x) == (x-1)*(x-2)*(x-3) 
\end{spadsrc}
\end{xtc}
\begin{psXtc}
\begin{xtccomment}
To draw the function, the first argument is its name
and the second is just the range with no independent variable.
\end{xtccomment}
\begin{spadsrc}
draw(f, 0..4) 
\end{spadsrc}
% window was 300 x 300
\epsffile[0 0 295 295]{2D1VarD.ps}
\end{psXtc}

% *********************************************************************
\head{subsection}{Plotting Two-Dimensional Parametric Plane Curves}{ugGraphTwoDPar}
% *********************************************************************

The second kind of \twodim{} graph is that of
\index{parametric plane curve}
curves produced by parametric equations.
\index{curve!parametric plane}
Let \spad{x = f(t)} and \spad{y = g(t)} be formulas or two
functions \spad{f} and \spad{g} as the parameter \spad{t} ranges
over an interval \spad{[a,b]}.
The function \spadfun{curve} takes the two functions \spad{f} and
\spad{g} as its parameters.

\beginImportant
The general format for drawing a \twodim{} plane curve defined by
parametric formulas \spad{x = f(t)} and \spad{y = g(t)} is:
%
\begin{center}
{\tt draw(curve(f(t), g(t)), t = a..b, {\it options})}
\end{center}
where \spad{a..b} defines the range of the independent variable \spad{t},
and where {\it options} prescribes zero or more options as
described in \spadref{ugGraphThreeDOptions}.
An example of an option is \spad{curveColor == bright red().}
\endImportant

Here's an example:

\begin{psXtc}
\begin{xtccomment}
Define a parametric curve using a range involving
\spad{%pi}, \Language{}'s way of saying $\pi$.
For parametric curves, \Language{} compiles two
functions, one for each of the functions \spad{f} and \spad{g}.
\end{xtccomment}
\begin{spadsrc}
draw(curve(sin(t)*sin(2*t)*sin(3*t), sin(4*t)*sin(5*t)*sin(6*t)), t = 0..2*%pi)
\end{spadsrc}
% window was 300 x 300
\epsffile[0 0 295 295]{2DppcA.ps}
\end{psXtc}
%
%
\begin{psXtc}
\begin{xtccomment}
The title may be an arbitrary string and is an
optional argument to the \spadfun{draw} command.
\end{xtccomment}
\begin{spadsrc}
draw(curve(cos(t), sin(t)), t = 0..2*%pi)
\end{spadsrc}
% window was 300 x 300
\epsffile[0 0 295 295]{2DppcB.ps}
\end{psXtc}
%
If you plan on plotting \spad{x = f(t)}, \spad{y = g(t)} as \spad{t} ranges over
several intervals, you may want to define functions \spad{f} and \spad{g} first, so
that they need not be recompiled every time you create a new graph.
Here's an example:
\begin{xtc}
\begin{xtccomment}
As before, you can first define the functions you wish to draw.
\end{xtccomment}
\begin{spadsrc}
f(t:DFLOAT):DFLOAT == sin(3*t/4) 
\end{spadsrc}
\begin{MessageOutput}
   Function declaration f : DoubleFloat -> DoubleFloat has been added 
      to workspace.
\end{MessageOutput}
\end{xtc}
\begin{xtc}
\begin{xtccomment}
\Language{} compiles them to map \spadtype{DoubleFloat}
values to \spadtype{DoubleFloat} values.
\end{xtccomment}
\begin{spadsrc}
g(t:DFLOAT):DFLOAT == sin(t) 
\end{spadsrc}
\begin{MessageOutput}
   Function declaration g : DoubleFloat -> DoubleFloat has been added 
      to workspace.
\end{MessageOutput}
\end{xtc}

\begin{psXtc}
\begin{xtccomment}
Give to {\tt curve} the names of the functions,
then write the range without the name of the
independent variable.
\end{xtccomment}
\begin{spadsrc}
draw(curve(f,g),0..%pi) 
\end{spadsrc}
% window was 300 x 300
\epsffile[0 0 295 295]{2DppcC.ps}
\end{psXtc}
%
%
\begin{psXtc}
\begin{xtccomment}
Here is another look at the same curve but over a different
range. Notice that \spad{f} and \spad{g} are not recompiled.
Also note that \Language{} provides a default title based on
the first function specified in \spadfun{curve}.
\end{xtccomment}
\begin{spadsrc}
draw(curve(f,g),-4*%pi..4*%pi) 
\end{spadsrc}
% window was 300 x 300
\epsffile[0 0 295 295]{2DppcE.ps}
\end{psXtc}

% *********************************************************************
\head{subsection}{Plotting Plane Algebraic Curves}{ugGraphTwoDPlane}
% *********************************************************************

A third kind of \twodim{} graph is a non-singular ``solution curve''
\index{curve!plane algebraic}
in a rectangular region of the plane.
A solution curve is a curve defined by a polynomial equation
\spad{p(x,y) = 0}.
\index{plane algebraic curve}
Non-singular means that the curve is ``smooth'' in that it does not
cross itself or come to a point (cusp).
Algebraically, this means that for any point \spad{(x,y)} on the curve,
that is, a point such that \spad{p(x,y) = 0}, the partial derivatives
  $\frac{\partial p}{\partial x}(x,y)$ and
  $\frac{\partial p}{\partial y}(x,y)$
are not both zero.
\index{curve!smooth}
\index{curve!non-singular}
\index{smooth curve}
\index{non-singular curve}

%
\beginImportant
The general format for drawing a non-singular solution curve
given by a polynomial of the form \spad{p(x,y) = 0} is:
%
\begin{center}
{\tt draw(p(x,y) = 0, x, y, range == [a..b, c..d], {\it options})}
\end{center}
where the second and third arguments name the first and second
independent variables of \spad{p}.
A {\tt range} option is always given to designate a bounding
rectangular region of the plane
$a \leq x \leq b, c \leq y\leq d$.
Zero or more additional options as described in
\spadref{ugGraphTwoDOptions} may be given.
\endImportant

\begin{xtc}
\begin{xtccomment}
We require that the polynomial has rational or integral coefficients.
Here is an algebraic curve example (``Cartesian ovals''):
\index{Cartesian!ovals}
\end{xtccomment}
\begin{spadsrc}
p := ((x^2 + y^2 + 1) - 8*x)^2 - (8*(x^2 + y^2 + 1)-4*x-1) 
\end{spadsrc}
\begin{TeXOutput}
\begin{fricasmath}{1}
\SUPER{\SYMBOL{y}}{4}+\PAREN{2\TIMES \SUPER{\SYMBOL{x}}{2}-{16\TIMES \SYMBOL{%
x}}-{6}}\TIMES \SUPER{\SYMBOL{y}}{2}+\SUPER{\SYMBOL{x}}{4}-{16\TIMES \SUPER{%
\SYMBOL{x}}{3}}+58\TIMES \SUPER{\SYMBOL{x}}{2}-{12\TIMES \SYMBOL{x}}-{6}%
\end{fricasmath}
\end{TeXOutput}
\formatResultType{Polynomial(Integer)}
\end{xtc}

\begin{psXtc}
\begin{xtccomment}
The first argument is always expressed as an equation of the form \spad{p = 0}
where \spad{p} is a polynomial.
\end{xtccomment}
\begin{spadsrc}
draw(p = 0, x, y, range == [-1..11, -7..7]) 
\end{spadsrc}
% window was 300 x 300
\epsffile[0 0 295 295]{2DpacA.ps}
\end{psXtc}

% *********************************************************************
\head{subsection}{Two-Dimensional Options}{ugGraphTwoDOptions}
% *********************************************************************

The \spadfun{draw} commands take an optional list of options,
such as {\tt title} shown above.
Each option is given by the syntax: {\it name} {\tt ==} {\it value}.
Here is a list of the available options in the order that they are
described below.

\begin{tabular}{llll}
adaptive&clip&unit\\
clip&curveColor&range\\
toScale&pointColor&coordinates\\
\end{tabular}

The \spad{adaptive} option turns adaptive plotting on or off.
\index{adaptive plotting}
Adaptive plotting uses an algorithm that traverses a graph and computes
more points for those parts of the graph with high curvature.
The higher the curvature of a region is, the more points the algorithm
computes.
\index{graphics!2D options!adaptive}
%
%
\begin{psXtc}
\begin{xtccomment}
The {\tt adaptive} option is normally on.
Here we turn it off.
\end{xtccomment}
\begin{spadsrc}
draw(sin(1/x),x=-2*%pi..2*%pi, adaptive == false)
\end{spadsrc}
% window was 300 x 300
\epsffile[0 0 295 295]{2DOptAd.ps}
\end{psXtc}
%
%
\begin{psXtc}
\begin{xtccomment}
The {\tt clip} option turns clipping on or off.
\index{graphics!2D options!clipping}
If on, large values are cut off according to
\spadfunFrom{clipPointsDefault}{GraphicsDefaults}.
\end{xtccomment}
\begin{spadsrc}
draw(tan(x),x=-2*%pi..2*%pi, clip == true)
\end{spadsrc}
% window was 300 x 300
\epsffile[0 0 295 295]{2DOptCp.ps}
\end{psXtc}
%
%
\begin{psXtc}
\begin{xtccomment}
Option {\tt toScale} does plotting to scale if {\tt true} or uses
the entire viewport if {\tt false}.
The default can be determined using
\spadfunFrom{drawToScale}{GraphicsDefaults}.
\index{graphics!2D options!to scale}
\end{xtccomment}
\begin{spadsrc}
draw(sin(x),x=-%pi..%pi, toScale == true, unit == [1.0,1.0])
\end{spadsrc}
% window was 300 x 300
\epsffile[0 0 295 295]{2DOptSc.ps}
\end{psXtc}
%
%
\begin{psXtc}
\begin{xtccomment}
Option {\tt clip} with a range sets point clipping of a graph within the
\index{graphics!2D options!clip in a range}
ranges specified in the list \spad{[x range,y range]}.
\index{clipping}
If only one range is specified, clipping applies to the y-axis.
\end{xtccomment}
\begin{spadsrc}
draw(sec(x),x=-2*%pi..2*%pi, clip == [-2*%pi..2*%pi,-%pi..%pi], unit == [1.0,1.0])
\end{spadsrc}
% window was 300 x 300
\epsffile[0 0 295 295]{2DOptCpR.ps}
\end{psXtc}
%
\begin{psXtc}
\begin{xtccomment}
Option {\tt curveColor} sets the color of the graph curves or lines to be the
\index{graphics!2D options!curve color}
indicated palette color
\index{curve!color}
(see \spadref{ugGraphColor} and \spadref{ugGraphColorPalette}).
\index{color!curve}
\end{xtccomment}
\begin{spadsrc}
draw(sin(x),x=-%pi..%pi, curveColor == bright red())
\end{spadsrc}
% window was 300 x 300
\epsffile[0 0 295 295]{2DOptCvC.ps}
\end{psXtc}
%
\begin{psXtc}
\begin{xtccomment}
Option {\tt pointColor}
sets the color of the graph points to the indicated
\index{graphics!2D options!point color}
palette color
(see \spadref{ugGraphColor} and \spadref{ugGraphColorPalette}).
\index{color!point}
\end{xtccomment}
\begin{spadsrc}
draw(sin(x),x=-%pi..%pi, pointColor == pastel yellow())
\end{spadsrc}
% window was 300 x 300
\epsffile[0 0 295 295]{2DOptPtC.ps}
\end{psXtc}
%
\begin{psXtc}
\begin{xtccomment}
Option {\tt unit} sets the intervals at which the axis units are plotted
\index{graphics!2D options!set units}
according to the indicated steps [\spad{x} interval, \spad{y} interval].
\end{xtccomment}
\begin{spadsrc}
draw(curve(9*sin(3*t/4),8*sin(t)), t = -4*%pi..4*%pi, unit == [2.0,1.0])
\end{spadsrc}
% window was 300 x 300
\epsffile[0 0 295 295]{2DOptUt.ps}
\end{psXtc}
%
%
\begin{psXtc}
\begin{xtccomment}
Option {\tt range} sets the range of variables in a graph to be
within the ranges
\index{graphics!2D options!range}
for solving plane algebraic curve plots.
\end{xtccomment}
\begin{spadsrc}
draw(y^2 + y - (x^3 - x) = 0, x, y, range == [-2..2,-2..1], unit==[1.0,1.0])
\end{spadsrc}
% window was 300 x 300
\epsffile[0 0 295 295]{2DOptRgA.ps}
\end{psXtc}
%
%
\begin{psXtc}
\begin{xtccomment}
A second example of a solution plot.
\end{xtccomment}
\begin{spadsrc}
draw(x^2 + y^2 = 1, x, y, range == [-3/2..3/2,-3/2..3/2], unit==[0.5,0.5])
\end{spadsrc}
% window was 300 x 300
\epsffile[0 0 295 295]{2DOptRgB.ps}
\end{psXtc}
%
%
\begin{psXtc}
\begin{xtccomment}
Option \spad{coordinates} indicates the coordinate system
in which the graph
\index{graphics!2D options!coordinates}
is plotted.
The default is to use the Cartesian coordinate system.
\index{Cartesian!coordinate system}
For more details, see \spadref{ugGraphCoord}%
.
\index{coordinate system!Cartesian}
\end{xtccomment}
\begin{spadsrc}
draw(curve(sin(5*t),t),t=0..2*%pi, coordinates == polar)
\end{spadsrc}
% window was 300 x 300
\epsffile[0 0 295 295]{2DOptPlr.ps}
\end{psXtc}

% *********************************************************************
\head{subsection}{Color}{ugGraphColor}
% *********************************************************************

The domain \spadtype{Color}
\exptypeindex{Color}
provides operations for manipulating
\index{graphics!color}
colors in \twodim{} graphs.
\index{color}
Colors are objects of \spadtype{Color}.
Each color has a {\it hue} and a {\it weight}.
\index{hue}
Hues are represented by integers that range from \spad{1} to the
\spadfunFrom{numberOfHues()}{Color}, normally
\index{graphics!color!number of hues}
\spad{27}.
%\footnote{Use \spadfun{colorDef} to
%change these values to any range you want for a given \threedim{} viewport}
\index{weight}
Weights are floats and  have the value \spad{1.0} by default.
%
\begin{description}
%
\item[\spadfun{color}]\funArgs{integer}
creates a color of hue {\it integer} and weight \spad{1.0}.
%
\item[\spadfun{hue}]\funArgs{color}
returns the hue of {\it color} as an integer.
\index{graphics!color!hue function}
%
\item[\spadfun{red}]\texttt{()}, \spadfun{blue}\texttt{()}
\spadfun{green}\texttt{()}, and \spadfun{yellow}\texttt{()}
\index{graphics!color!primary color functions}
\newline
create colors of that hue with weight \spad{1.0}.
%
\item[\subscriptIt{color}{1} {\tt +} \subscriptIt{color}{2}] returns the
color that results from additively combining the indicated
\subscriptIt{color}{1} and \subscriptIt{color}{2}.
Color addition is not commutative: changing the order of the arguments
produces different results.
%
\item[{\it integer} {\tt *} {\it color}]
changes the weight of {\it color} by {\it integer}
without affecting its hue.
\index{graphics!color!multiply function}
For example,
\spad{red() + 3*yellow()} produces a color closer to yellow than to red.
Color multiplication is not associative: changing the order of grouping
\index{color!multiplication}
produces different results.
\end{description}
%
\begin{psXtc}
\begin{xtccomment}
These functions can be used to change the point and curve colors
for two- and \threedim{} graphs.
Use the {\tt pointColor} option for points.
\end{xtccomment}
\begin{spadsrc}
draw(x^2,x=-1..1,pointColor == green())
\end{spadsrc}
% window was 300 x 300
\epsffile[0 0 295 295]{23DColA.ps}
\end{psXtc}
%
\begin{psXtc}
\begin{xtccomment}
Use the {\tt curveColor} option for curves.
\end{xtccomment}
\begin{spadsrc}
draw(x^2,x=-1..1,curveColor == color(13) + 2*blue())
\end{spadsrc}
% window was 300 x 300
\epsffile[0 0 295 295]{23DColB.ps}
\end{psXtc}

% *********************************************************************
\head{subsection}{Palette}{ugGraphColorPalette}
% *********************************************************************
\index{graphics!palette}

Domain \spadtype{Palette} is the domain of shades of colors:
\spadfun{dark}, \spadfun{dim}, \spadfun{bright}, \spadfun{pastel}, and \spadfun{light},
designated by the integers \spad{1} through \spad{5}, respectively.
\exptypeindex{Palette}
\begin{xtc}
\begin{xtccomment}
Colors are normally ``bright.''
\end{xtccomment}
\begin{spadsrc}
shade red()
\end{spadsrc}
\begin{TeXOutput}
\begin{fricasmath}{1}
3%
\end{fricasmath}
\end{TeXOutput}
\formatResultType{PositiveInteger}
\end{xtc}
\begin{xtc}
\begin{xtccomment}
To change the shade of a color, apply the name of a shade to it.
\index{color!shade}
\index{shade}
\end{xtccomment}
\begin{spadsrc}
myFavoriteColor := dark blue() 
\end{spadsrc}
\begin{TeXOutput}
\begin{fricasmath}{2}
\STRING{[}\STRING{Hue:\ }22\STRING{\ \ Weight:\ }\STRING{1.0}\STRING{%
]\ from\ the\ }\STRING{Dark}\STRING{\ palette}%
\end{fricasmath}
\end{TeXOutput}
\formatResultType{Palette}
\end{xtc}
\begin{xtc}
\begin{xtccomment}
The expression \spad{shade(color)}
returns the value of a shade of \spad{color}.
\end{xtccomment}
\begin{spadsrc}
shade myFavoriteColor 
\end{spadsrc}
\begin{TeXOutput}
\begin{fricasmath}{3}
1%
\end{fricasmath}
\end{TeXOutput}
\formatResultType{PositiveInteger}
\end{xtc}
\begin{xtc}
\begin{xtccomment}
The expression \spad{hue(color)} returns its hue.
\end{xtccomment}
\begin{spadsrc}
hue myFavoriteColor 
\end{spadsrc}
\begin{TeXOutput}
\begin{fricasmath}{4}
\STRING{Hue:\ }22\STRING{\ \ Weight:\ }\STRING{1.0}%
\end{fricasmath}
\end{TeXOutput}
\formatResultType{Color}
\end{xtc}
\begin{psXtc}
\begin{xtccomment}
Palettes can be used in specifying colors in \twodim{} graphs.
\end{xtccomment}
\begin{spadsrc}
draw(x^2,x=-1..1,curveColor == dark blue())
\end{spadsrc}
% window was 300 x 300
\epsffile[0 0 295 295]{23DPal.ps}
\end{psXtc}

% *********************************************************************
\head{subsection}{Two-Dimensional Control-Panel}{ugGraphTwoDControl}
% *********************************************************************
\index{graphics!2D control-panel}
Once you have created a viewport, move your mouse to the viewport and click
with your left mouse button to display a control-panel.
The panel is displayed on the side of the viewport closest to
where you clicked.  Each of the buttons which toggle on and off show the
current state of the graph.

\typeout{2D control-panel.}
\begin{figure}[htbp]
%{\epsfverbosetrue\epsfxsize=2in%
%\def\epsfsize#1#2{\epsfxsize}\hspace*{\baseLeftSkip}%
%\epsffile[0 0 144 289]{2Dctrl.ps}}
\begin{picture}(147,252)%(-143,0)
\special{psfile=2Dctrl.ps}
\end{picture}
\caption{Two-dimensional control-panel.}
\end{figure}

% *********************************************************************
\subsubsection{Transformations}
% *********************************************************************
\index{graphics!2D control-panel!transformations}

Object transformations are executed from the control-panel by mouse-activated
potentiometer windows.
%
\begin{description}
%
\item[Scale:] To scale a graph, click on a mouse button
\index{graphics!2D control-panel!scale}
within the {\bf Scale} window in the upper left corner of the control-panel.
The axes along which the scaling is to occur are indicated by setting the
toggles above the arrow.
With {\tt X On} and {\tt Y On} appearing, both axes are selected and scaling
is uniform.
If either is not selected, for example, if {\tt X Off} appears, scaling is
non-uniform.
%
\item[Translate:] To translate a graph, click the mouse in the
\index{graphics!2D control-panel!translate}
{\bf Translate} window in the direction you wish the graph to move.
This window is located in the upper right corner of the control-panel.
Along the top of the {\bf Translate} window are two buttons for selecting
the direction of translation.
Translation along both coordinate axes results when {\tt X On} and {\tt Y
On} appear or along one axis when one is on, for example, {\tt X On} and
{\tt Y Off} appear.
\end{description}

% *********************************************************************
\subsubsection{Messages}
% *********************************************************************
\index{graphics!2D control-panel!messages}

The window directly below the transformation potentiometer windows is
used to display system messages relating to the viewport and the control-panel.
The following format is displayed: \newline
%
\begin{center}
[scaleX, scaleY] \spad{>}graph\spad{<} [translateX, translateY] \newline
\end{center}
The two values to the left show the scale factor along the {\tt X} and
{\tt Y} coordinate axes.  The two values to the right show the distance of
translation from the center in the {\tt X} and {\tt Y} directions.  The number
in the center shows which graph in the viewport this data pertains to.
When multiple graphs exist in the same viewport,
the graph must be selected (see ``Multiple Graphs,'' below) in
order for its transformation data to be shown, otherwise the number
is 1.

% *********************************************************************
\subsubsection{Multiple Graphs}
% *********************************************************************

\index{graphics!2D control-panel!multiple graphs}
The {\bf Graphs} window contains buttons that allow the placement
of \twodim{} graphs into one of nine available slots in any other
\twodim{} viewport.
In the center of the window are numeral buttons from one to nine
that show whether a graph is displayed in the viewport.
Below each number button is a button showing whether a graph
that is present is selected for application of some
transformation.
When the caret symbol is displayed, then the graph in that slot
will be manipulated.
Initially, the graph for which the viewport is created occupies
the first slot, is displayed, and is selected.
%
%
\begin{description}
%
\item[Clear:]  The {\bf Clear} button deselects every viewport graph slot.
\index{graphics!2D control-panel!clear}
A graph slot is reselected by selecting the button below its number.
%
\item[Query:]  The {\bf Query} button is used to display the scale and
\index{graphics!2D control-panel!query}
translate data for the indicated graph.  When this button is selected the
message ``Click on the graph to query'' appears.  Select a slot
number button from the {\bf Graphs} window. The scaling factor and translation
offset of the graph are then displayed in the message window.
%
\item[Pick:]  The {\bf Pick} button is used to select a graph
\index{graphics!2D control-panel!pick}
to be placed or dropped into the indicated viewport.  When this button is
selected, the message ``Click on the graph to pick'' appears.
Click on the slot with the graph number of the desired
graph.  The graph information is held waiting for
you to execute a {\bf Drop} in some other graph.
%
\item[Drop:]  Once a graph has been picked up using the {\bf Pick} button,
\index{graphics!2D control-panel!drop}
the {\bf Drop} button places it into a new viewport slot.
The message ``Click on the graph to drop'' appears in the message
window when the {\bf Drop} button is selected.
By selecting one of the slot number buttons in the {\bf Graphs}
window, the graph currently being held is dropped into this slot
and displayed.
\end{description}

% *********************************************************************
\subsubsection{Buttons}
% *********************************************************************
\index{graphics!2D control-panel!buttons}

%
\begin{description}
%
\item[Axes] turns the coordinate axes on or off.
\index{graphics!2D control-panel!axes}
%
\item[Units] turns the units along the {\tt x}
and {\tt y} axis on or off.
\index{graphics!2D control-panel!units}
%
\item[Box] encloses the area of the viewport graph
in a bounding box, or removes the box if already enclosed.
\index{graphics!2D control-panel!box}
%
\item[Pts] turns on or off the display of points.
\index{graphics!2D control-panel!points}
%
\item[Lines] turns on or off the display
of lines connecting points.
\index{graphics!2D control-panel!lines}
%
\item[PS] writes the current viewport contents to
\index{graphics!2D control-panel!ps}
a file {\bf axiom2D.ps} or to a name specified in the user's {\bf
\index{graphics!.Xdefaults!PostScript file name}
.Xdefaults} file.
\index{file!.Xdefaults @{\bf .Xdefaults}}
The file is placed in the directory from which \Language{} or the {\bf
viewAlone} program was invoked.
\index{PostScript}
%
\item[Reset] resets the object transformation
characteristics and attributes back to their initial states.
\index{graphics!2D control-panel!reset}
%
\item[Hide] makes the control-panel disappear.
\index{graphics!2D control-panel!hide}
%
\item[Quit] queries whether the current viewport
\index{graphics!2D control-panel!quit}
session should be terminated.
\end{description}

% *********************************************************************
\head{subsection}{Operations for Two-Dimensional Graphics}{ugGraphTwoDops}
% *********************************************************************

Here is a summary of useful \Language{} operations for \twodim{}
graphics.
Each operation name is followed by a list of arguments.
Each argument is written as a variable informally named according
to the type of the argument (for example, {\it integer}).
If appropriate, a default value for an argument is given in
parentheses immediately following the name.

%
\bgroup\hbadness = 10001\sloppy
\begin{description}
%
\item[\spadfun{adaptive}]\funArgs{\optArg{boolean\argDef{true}}}
\index{adaptive plotting}
sets or indicates whether graphs are plotted
\index{graphics!set 2D defaults!adaptive}
according to the adaptive refinement algorithm.
%
\item[\spadfun{axesColorDefault}]\funArgs{\optArg{color\argDef{dark blue()}}}
sets or indicates the default color of the
\index{graphics!set 2D defaults!axes color}
axes in a \twodim{} graph viewport.
%
\item[\spadfun{clipPointsDefault}]\funArgs{\optArg{boolean\argDef{false}}}
sets or
indicates whether point clipping is
\index{graphics!set 2D defaults!clip points}
to be applied as the default for graph plots.
%
\item[\spadfun{drawToScale}]\funArgs{\optArg{boolean\argDef{false}}}
sets or
indicates whether the plot of a graph
\index{graphics!set 2D defaults!to scale}
is ``to scale'' or uses the entire viewport space as the default.
%
\item[\spadfun{lineColorDefault}]\funArgs{\optArg{color\argDef{pastel yellow()}}}
sets or indicates the default color of the
\index{graphics!set 2D defaults!line color}
lines or curves in a \twodim{} graph viewport.
%
\item[\spadfun{maxPoints}]\funArgs{\optArg{integer\argDef{500}}}
sets or indicates
the default maximum number of
\index{graphics!set 2D defaults!max points}
possible points to be used when constructing a \twodim{} graph.
%
\item[\spadfun{minPoints}]\funArgs{\optArg{integer\argDef{21}}}
sets or indicates the default minimum number of
\index{graphics!set 2D defaults!min points}
possible points to be used when constructing a \twodim{} graph.
%
\item[\spadfun{pointColorDefault}]\funArgs{\optArg{color\argDef{bright red()}}}
sets or indicates the default color of the
\index{graphics!set 2D defaults!point color}
points in a \twodim{} graph viewport.
%
\item[\spadfun{pointSizeDefault}]\funArgs{\optArg{integer\argDef{5}}}
sets or indicates the default size of the
\index{graphics!set 2D defaults!point size}
dot used to plot points in a \twodim{} graph.
%
\item[\spadfun{screenResolution}]\funArgs{\optArg{integer\argDef{600}}}
sets or indicates the default screen
\index{graphics!set 2D defaults!screen resolution}
resolution constant used in setting the computation limit of adaptively
\index{adaptive plotting}
generated curve plots.
%
\item[\spadfun{unitsColorDefault}]\funArgs{\optArg{color\argDef{dim green()}}}
sets or indicates the default color of the
\index{graphics!set 2D defaults!units color}
unit labels in a \twodim{} graph viewport.
%
\item[\spadfun{viewDefaults}]\funArgs{}
resets the default settings for the following
\index{graphics!set 2D defaults!reset viewport}
attributes:  point color, line color, axes color, units color, point size,
viewport upper left-hand corner position, and the viewport size.
%
\item[\spadfun{viewPosDefault}]\funArgs{\optArg{list\argDef{[100,100]}}}
sets or indicates the default position of the
\index{graphics!set 2D defaults!viewport position}
upper left-hand corner of a \twodim{} viewport, relative to the
display root window.
The upper left-hand corner of the display is considered to be at the
(0, 0) position.
%
\item[\spadfun{viewSizeDefault}]\funArgs{\optArg{list\argDef{[200,200]}}}
sets or
indicates the default size in which two
\index{graphics!set 2D defaults!viewport size}
dimensional viewport windows are shown.
It is defined by a width and then a height.
%
\item[\spadfun{viewWriteAvailable}]\funArgs{\optArg{list\argDef{["pixmap",
"bitmap", "postscript", \"image"}}}
indicates the possible file types
\index{graphics!2D defaults!available viewport writes}
that can be created with the \spadfunFrom{write}{TwoDimensionalViewport} function.
%
\item[\spadfun{viewWriteDefault}]
\funArgs{\optArg{list\argDef{[]}}}
sets or indicates the default types of files, in
\index{graphics!set 2D defaults!write viewport}
addition to the {\bf data} file, that are created when a
\spadfun{write} function is executed on a viewport.
%
\item[\spadfun{units}]\funArgs{viewport, integer\argDef{1}, string\argDef{"off"}}
turns the units on or off for the graph with index {\it integer}.
%
\item[\spadfun{axes}]\funArgs{viewport, integer\argDef{1}, string\argDef{"on"}}
turns the axes on
\index{graphics!2D commands!axes}
or off for the graph with index {\it integer}.
%
\item[\spadfun{close}]\funArgs{viewport}
closes {\it viewport}.
\index{graphics!2D commands!close}
%
\item[\spadfun{connect}]\funArgs{viewport, integer\argDef{1}, string\argDef{"on"}}
declares whether lines
\index{graphics!2D commands!connect}
connecting the points are displayed or not.
%
\item[\spadfun{controlPanel}]\funArgs{viewport, string\argDef{"off"}}
declares
whether the \twodim{} control-panel is automatically displayed
or not.
%
\item[\spadfun{graphs}]\funArgs{viewport}
returns a list
\index{graphics!2D commands!graphs}
describing the state of each graph.
If the graph state is not being used this is shown by {\tt "undefined"},
otherwise a description of the graph's contents is shown.
%
\item[\spadfun{graphStates}]\funArgs{viewport}
displays
\index{graphics!2D commands!state of graphs}
a list of all the graph states available for {\it viewport}, giving the
values for every property.
%
\item[\spadfun{key}]\funArgs{viewport}
returns the process
\index{graphics!2D commands!key}
ID number for {\it viewport}.
%
\item[\spadfun{move}]\funArgs{viewport,
\subscriptText{integer}{x}(viewPosDefault),
\subscriptText{integer}{y}(viewPosDefault)}
moves {\it viewport} on the screen so that the
\index{graphics!2D commands!move}
upper left-hand corner of {\it viewport} is at the position {\it (x,y)}.
%
\item[\spadfun{options}]\funArgs{\it viewport}
returns a list
\index{graphics!2D commands!options}
of all the \spadtype{DrawOption}s used by {\it viewport}.
%
\item[\spadfun{points}]\funArgs{viewport, integer\argDef{1}, string\argDef{"on"}}
specifies whether the graph points for graph {\it integer} are
\index{graphics!2D commands!points}
to be displayed or not.
%
\item[\spadfun{region}]\funArgs{viewport, integer\argDef{1}, string\argDef{"off"}}
declares whether graph {\it integer} is or is not to be displayed
with a bounding rectangle.
%
\item[\spadfun{reset}]\funArgs{viewport}
resets all the properties of {\it viewport}.
%
\item[\spadfun{resize}]\funArgs{viewport,
\subscriptText{integer}{width}, \subscriptText{integer}{height}}
\index{graphics!2D commands!resize}
resizes {\it viewport} with a new {\it width} and {\it height}.
%
\item[\spadfun{scale}]\funArgs{viewport, \subscriptText{integer}{n}\argDef{1},
\subscriptText{integer}{x}\argDef{0.9}, \subscriptText{integer}{y}\argDef{0.9}}
scales values for the
\index{graphics!2D commands!scale}
{\it x} and {\it y} coordinates of graph {\it n}.
%
\item[\spadfun{show}]\funArgs{viewport, \subscriptText{integer}{n}\argDef{1},
string\argDef{"on"}}
indicates if graph {\it n} is shown or not.
%
\item[\spadfun{title}]\funArgs{viewport, string\argDef{"Axiom 2D"}}
designates the title for {\it viewport}.
%
\item[\spadfun{translate}]\funArgs{viewport,
\subscriptText{integer}{n}\argDef{1},
\subscriptText{float}{x}\argDef{0.0}, \subscriptText{float}{y}\argDef{0.0}}
\index{graphics!2D commands!translate}
causes graph {\it n} to be moved {\it x} and {\it y} units in the respective directions.
%
\item[\spadfun{write}]\funArgs{viewport, \subscriptText{string}{directory},
\optArg{strings}}
if no third argument is given, writes the {\bf data} file onto the directory
with extension {\bf data}.
The third argument can be a single string or a list of strings with some or
all the entries {\tt "pixmap"}, {\tt "bitmap"}, {\tt "postscript"}, and
{\tt "image"}.
\end{description}
\egroup

% *********************************************************************
\head{subsection}{Addendum: Building Two-Dimensional Graphs}{ugGraphTwoDbuild}
% *********************************************************************

In this section we demonstrate how to create \twodim{} graphs from
lists of points and give an example showing how to read the lists
of points from a file.

% *********************************************************************
\subsubsection{Creating a Two-Dimensional Viewport from a List of Points}
% *********************************************************************

\Language{} creates lists of points in a \twodim{} viewport by utilizing
the \spadtype{GraphImage} and \spadtype{TwoDimensionalViewport} domains.
In this example, the \spadfunFrom{makeGraphImage}{GraphImage}
function takes a list of lists of points parameter, a list of colors for
each point in the graph, a list of colors for each line in the graph, and
a list of sizes for each point in the graph.
%
\begin{xtc}
\begin{xtccomment}
The following expressions create a list of lists of points which will be read
by \Language{} and made into a \twodim{} viewport.
\end{xtccomment}
\begin{spadsrc}
p1 := point [1,1]$(Point DFLOAT) 
\end{spadsrc}
\begin{TeXOutput}
\begin{fricasmath}{1}
\BRACKET{\STRING{1.0}\COMMA \STRING{1.0}}%
\end{fricasmath}
\end{TeXOutput}
\formatResultType{Point(DoubleFloat)}
\end{xtc}
\begin{xtc}
\begin{xtccomment}
\end{xtccomment}
\begin{spadsrc}
p2 := point [0,1]$(Point DFLOAT) 
\end{spadsrc}
\begin{TeXOutput}
\begin{fricasmath}{2}
\BRACKET{\STRING{0.0}\COMMA \STRING{1.0}}%
\end{fricasmath}
\end{TeXOutput}
\formatResultType{Point(DoubleFloat)}
\end{xtc}
\begin{xtc}
\begin{xtccomment}
\end{xtccomment}
\begin{spadsrc}
p3 := point [0,0]$(Point DFLOAT) 
\end{spadsrc}
\begin{TeXOutput}
\begin{fricasmath}{3}
\BRACKET{\STRING{0.0}\COMMA \STRING{0.0}}%
\end{fricasmath}
\end{TeXOutput}
\formatResultType{Point(DoubleFloat)}
\end{xtc}
\begin{xtc}
\begin{xtccomment}
\end{xtccomment}
\begin{spadsrc}
p4 := point [1,0]$(Point DFLOAT) 
\end{spadsrc}
\begin{TeXOutput}
\begin{fricasmath}{4}
\BRACKET{\STRING{1.0}\COMMA \STRING{0.0}}%
\end{fricasmath}
\end{TeXOutput}
\formatResultType{Point(DoubleFloat)}
\end{xtc}
\begin{xtc}
\begin{xtccomment}
\end{xtccomment}
\begin{spadsrc}
p5 := point [1,.5]$(Point DFLOAT) 
\end{spadsrc}
\begin{TeXOutput}
\begin{fricasmath}{5}
\BRACKET{\STRING{1.0}\COMMA \STRING{0.5}}%
\end{fricasmath}
\end{TeXOutput}
\formatResultType{Point(DoubleFloat)}
\end{xtc}
\begin{xtc}
\begin{xtccomment}
\end{xtccomment}
\begin{spadsrc}
p6 := point [.5,0]$(Point DFLOAT) 
\end{spadsrc}
\begin{TeXOutput}
\begin{fricasmath}{6}
\BRACKET{\STRING{0.5}\COMMA \STRING{0.0}}%
\end{fricasmath}
\end{TeXOutput}
\formatResultType{Point(DoubleFloat)}
\end{xtc}
\begin{xtc}
\begin{xtccomment}
\end{xtccomment}
\begin{spadsrc}
p7 := point [0,0.5]$(Point DFLOAT) 
\end{spadsrc}
\begin{TeXOutput}
\begin{fricasmath}{7}
\BRACKET{\STRING{0.0}\COMMA \STRING{0.5}}%
\end{fricasmath}
\end{TeXOutput}
\formatResultType{Point(DoubleFloat)}
\end{xtc}
\begin{xtc}
\begin{xtccomment}
\end{xtccomment}
\begin{spadsrc}
p8 := point [.5,1]$(Point DFLOAT) 
\end{spadsrc}
\begin{TeXOutput}
\begin{fricasmath}{8}
\BRACKET{\STRING{0.5}\COMMA \STRING{1.0}}%
\end{fricasmath}
\end{TeXOutput}
\formatResultType{Point(DoubleFloat)}
\end{xtc}
\begin{xtc}
\begin{xtccomment}
\end{xtccomment}
\begin{spadsrc}
p9 := point [.25,.25]$(Point DFLOAT) 
\end{spadsrc}
\begin{TeXOutput}
\begin{fricasmath}{9}
\BRACKET{\STRING{0.25}\COMMA \STRING{0.25}}%
\end{fricasmath}
\end{TeXOutput}
\formatResultType{Point(DoubleFloat)}
\end{xtc}
\begin{xtc}
\begin{xtccomment}
\end{xtccomment}
\begin{spadsrc}
p10 := point [.25,.75]$(Point DFLOAT) 
\end{spadsrc}
\begin{TeXOutput}
\begin{fricasmath}{10}
\BRACKET{\STRING{0.25}\COMMA \STRING{0.75}}%
\end{fricasmath}
\end{TeXOutput}
\formatResultType{Point(DoubleFloat)}
\end{xtc}
\begin{xtc}
\begin{xtccomment}
\end{xtccomment}
\begin{spadsrc}
p11 := point [.75,.75]$(Point DFLOAT) 
\end{spadsrc}
\begin{TeXOutput}
\begin{fricasmath}{11}
\BRACKET{\STRING{0.75}\COMMA \STRING{0.75}}%
\end{fricasmath}
\end{TeXOutput}
\formatResultType{Point(DoubleFloat)}
\end{xtc}
\begin{xtc}
\begin{xtccomment}
\end{xtccomment}
\begin{spadsrc}
p12 := point [.75,.25]$(Point DFLOAT) 
\end{spadsrc}
\begin{TeXOutput}
\begin{fricasmath}{12}
\BRACKET{\STRING{0.75}\COMMA \STRING{0.25}}%
\end{fricasmath}
\end{TeXOutput}
\formatResultType{Point(DoubleFloat)}
\end{xtc}
\begin{xtc}
\begin{xtccomment}
Finally, here is the list.
\end{xtccomment}
\begin{spadsrc}
llp := [[p1,p2], [p2,p3], [p3,p4], [p4,p1], [p5,p6], [p6,p7], [p7,p8], [p8,p5], [p9,p10], [p10,p11], [p11,p12], [p12,p9]] 
\end{spadsrc}
\begin{TeXOutput}
\begin{fricasmath}{13}
\BRACKET{\BRACKET{\BRACKET{\STRING{1.0}\COMMA \STRING{1.0}}\COMMA \BRACKET{%
\STRING{0.0}\COMMA \STRING{1.0}}}\COMMA \BRACKET{\BRACKET{\STRING{0.0}\COMMA %
\STRING{1.0}}\COMMA \BRACKET{\STRING{0.0}\COMMA \STRING{0.0}}}\COMMA \BRACKET%
{\BRACKET{\STRING{0.0}\COMMA \STRING{0.0}}\COMMA \BRACKET{\STRING{1.0}\COMMA %
\STRING{0.0}}}\COMMA \BRACKET{\BRACKET{\STRING{1.0}\COMMA \STRING{0.0}}%
\COMMA \BRACKET{\STRING{1.0}\COMMA \STRING{1.0}}}\COMMA \BRACKET{\BRACKET{%
\STRING{1.0}\COMMA \STRING{0.5}}\COMMA \BRACKET{\STRING{0.5}\COMMA \STRING{%
0.0}}}\COMMA \BRACKET{\BRACKET{\STRING{0.5}\COMMA \STRING{0.0}}\COMMA %
\BRACKET{\STRING{0.0}\COMMA \STRING{0.5}}}\COMMA \BRACKET{\BRACKET{\STRING{%
0.0}\COMMA \STRING{0.5}}\COMMA \BRACKET{\STRING{0.5}\COMMA \STRING{1.0}}}%
\COMMA \BRACKET{\BRACKET{\STRING{0.5}\COMMA \STRING{1.0}}\COMMA \BRACKET{%
\STRING{1.0}\COMMA \STRING{0.5}}}\COMMA \BRACKET{\BRACKET{\STRING{0.25}%
\COMMA \STRING{0.25}}\COMMA \BRACKET{\STRING{0.25}\COMMA \STRING{0.75}}}%
\COMMA \BRACKET{\BRACKET{\STRING{0.25}\COMMA \STRING{0.75}}\COMMA \BRACKET{%
\STRING{0.75}\COMMA \STRING{0.75}}}\COMMA \BRACKET{\BRACKET{\STRING{0.75}%
\COMMA \STRING{0.75}}\COMMA \BRACKET{\STRING{0.75}\COMMA \STRING{0.25}}}%
\COMMA \BRACKET{\BRACKET{\STRING{0.75}\COMMA \STRING{0.25}}\COMMA \BRACKET{%
\STRING{0.25}\COMMA \STRING{0.25}}}}%
\end{fricasmath}
\end{TeXOutput}
\formatResultType{List(List(Point(DoubleFloat)))}
\end{xtc}
\begin{xtc}
\begin{xtccomment}
Now we set the point sizes for all components of the graph.
\end{xtccomment}
\begin{spadsrc}
size1 := 6::PositiveInteger 
\end{spadsrc}
\begin{TeXOutput}
\begin{fricasmath}{14}
6%
\end{fricasmath}
\end{TeXOutput}
\formatResultType{PositiveInteger}
\end{xtc}
\begin{xtc}
\begin{xtccomment}
\end{xtccomment}
\begin{spadsrc}
size2 := 8::PositiveInteger 
\end{spadsrc}
\begin{TeXOutput}
\begin{fricasmath}{15}
8%
\end{fricasmath}
\end{TeXOutput}
\formatResultType{PositiveInteger}
\end{xtc}
\begin{xtc}
\begin{xtccomment}
\end{xtccomment}
\begin{spadsrc}
size3 := 10::PositiveInteger 
\end{spadsrc}
\begin{TeXOutput}
\begin{fricasmath}{16}
10%
\end{fricasmath}
\end{TeXOutput}
\formatResultType{PositiveInteger}
\end{xtc}
\begin{xtc}
\begin{xtccomment}
\end{xtccomment}
\begin{spadsrc}
lsize := [size1, size1, size1, size1, size2, size2, size2, size2, size3, size3, size3, size3] 
\end{spadsrc}
\begin{TeXOutput}
\begin{fricasmath}{17}
\BRACKET{6\COMMA 6\COMMA 6\COMMA 6\COMMA 8\COMMA 8\COMMA 8\COMMA 8\COMMA 10%
\COMMA 10\COMMA 10\COMMA 10}%
\end{fricasmath}
\end{TeXOutput}
\formatResultType{List(PositiveInteger)}
\end{xtc}
\begin{xtc}
\begin{xtccomment}
Here are the colors for the points.
\end{xtccomment}
\begin{spadsrc}
pc1 := pastel red() 
\end{spadsrc}
\begin{TeXOutput}
\begin{fricasmath}{18}
\STRING{[}\STRING{Hue:\ }1\STRING{\ \ Weight:\ }\STRING{1.0}\STRING{%
]\ from\ the\ }\STRING{Pastel}\STRING{\ palette}%
\end{fricasmath}
\end{TeXOutput}
\formatResultType{Palette}
\end{xtc}
\begin{xtc}
\begin{xtccomment}
\end{xtccomment}
\begin{spadsrc}
pc2 := dim green() 
\end{spadsrc}
\begin{TeXOutput}
\begin{fricasmath}{19}
\STRING{[}\STRING{Hue:\ }14\STRING{\ \ Weight:\ }\STRING{1.0}\STRING{%
]\ from\ the\ }\STRING{Dim}\STRING{\ palette}%
\end{fricasmath}
\end{TeXOutput}
\formatResultType{Palette}
\end{xtc}
\begin{xtc}
\begin{xtccomment}
\end{xtccomment}
\begin{spadsrc}
pc3 := pastel yellow() 
\end{spadsrc}
\begin{TeXOutput}
\begin{fricasmath}{20}
\STRING{[}\STRING{Hue:\ }11\STRING{\ \ Weight:\ }\STRING{1.0}\STRING{%
]\ from\ the\ }\STRING{Pastel}\STRING{\ palette}%
\end{fricasmath}
\end{TeXOutput}
\formatResultType{Palette}
\end{xtc}
\begin{xtc}
\begin{xtccomment}
\end{xtccomment}
\begin{spadsrc}
lpc := [pc1, pc1, pc1, pc1, pc2, pc2, pc2, pc2, pc3, pc3, pc3, pc3] 
\end{spadsrc}
\begin{TeXOutput}
\begin{fricasmath}{21}
\BRACKET{\STRING{[}\STRING{Hue:\ }1\STRING{\ \ Weight:\ }\STRING{1.0}\STRING{%
]\ from\ the\ }\STRING{Pastel}\STRING{\ palette}\COMMA \STRING{[}\STRING{%
Hue:\ }1\STRING{\ \ Weight:\ }\STRING{1.0}\STRING{]\ from\ the\ }\STRING{%
Pastel}\STRING{\ palette}\COMMA \STRING{[}\STRING{Hue:\ }1\STRING{%
\ \ Weight:\ }\STRING{1.0}\STRING{]\ from\ the\ }\STRING{Pastel}\STRING{%
\ palette}\COMMA \STRING{[}\STRING{Hue:\ }1\STRING{\ \ Weight:\ }\STRING{1.0}%
\STRING{]\ from\ the\ }\STRING{Pastel}\STRING{\ palette}\COMMA \STRING{[}%
\STRING{Hue:\ }14\STRING{\ \ Weight:\ }\STRING{1.0}\STRING{]\ from\ the\ }%
\STRING{Dim}\STRING{\ palette}\COMMA \STRING{[}\STRING{Hue:\ }14\STRING{%
\ \ Weight:\ }\STRING{1.0}\STRING{]\ from\ the\ }\STRING{Dim}\STRING{%
\ palette}\COMMA \STRING{[}\STRING{Hue:\ }14\STRING{\ \ Weight:\ }\STRING{1.0%
}\STRING{]\ from\ the\ }\STRING{Dim}\STRING{\ palette}\COMMA \STRING{[}%
\STRING{Hue:\ }14\STRING{\ \ Weight:\ }\STRING{1.0}\STRING{]\ from\ the\ }%
\STRING{Dim}\STRING{\ palette}\COMMA \STRING{[}\STRING{Hue:\ }11\STRING{%
\ \ Weight:\ }\STRING{1.0}\STRING{]\ from\ the\ }\STRING{Pastel}\STRING{%
\ palette}\COMMA \STRING{[}\STRING{Hue:\ }11\STRING{\ \ Weight:\ }\STRING{1.0%
}\STRING{]\ from\ the\ }\STRING{Pastel}\STRING{\ palette}\COMMA \STRING{[}%
\STRING{Hue:\ }11\STRING{\ \ Weight:\ }\STRING{1.0}\STRING{]\ from\ the\ }%
\STRING{Pastel}\STRING{\ palette}\COMMA \STRING{[}\STRING{Hue:\ }11\STRING{%
\ \ Weight:\ }\STRING{1.0}\STRING{]\ from\ the\ }\STRING{Pastel}\STRING{%
\ palette}}%
\end{fricasmath}
\end{TeXOutput}
\formatResultType{List(Palette)}
\end{xtc}
\begin{xtc}
\begin{xtccomment}
Here are the colors for the lines.
\end{xtccomment}
\begin{spadsrc}
lc := [pastel blue(), light yellow(), dim green(), bright red(), light green(), dim yellow(), bright blue(), dark red(), pastel red(), light blue(), dim green(), light yellow()] 
\end{spadsrc}
\begin{TeXOutput}
\begin{fricasmath}{22}
\BRACKET{\STRING{[}\STRING{Hue:\ }22\STRING{\ \ Weight:\ }\STRING{1.0}\STRING%
{]\ from\ the\ }\STRING{Pastel}\STRING{\ palette}\COMMA \STRING{[}\STRING{%
Hue:\ }11\STRING{\ \ Weight:\ }\STRING{1.0}\STRING{]\ from\ the\ }\STRING{%
Light}\STRING{\ palette}\COMMA \STRING{[}\STRING{Hue:\ }14\STRING{%
\ \ Weight:\ }\STRING{1.0}\STRING{]\ from\ the\ }\STRING{Dim}\STRING{%
\ palette}\COMMA \STRING{[}\STRING{Hue:\ }1\STRING{\ \ Weight:\ }\STRING{1.0}%
\STRING{]\ from\ the\ }\STRING{Bright}\STRING{\ palette}\COMMA \STRING{[}%
\STRING{Hue:\ }14\STRING{\ \ Weight:\ }\STRING{1.0}\STRING{]\ from\ the\ }%
\STRING{Light}\STRING{\ palette}\COMMA \STRING{[}\STRING{Hue:\ }11\STRING{%
\ \ Weight:\ }\STRING{1.0}\STRING{]\ from\ the\ }\STRING{Dim}\STRING{%
\ palette}\COMMA \STRING{[}\STRING{Hue:\ }22\STRING{\ \ Weight:\ }\STRING{1.0%
}\STRING{]\ from\ the\ }\STRING{Bright}\STRING{\ palette}\COMMA \STRING{[}%
\STRING{Hue:\ }1\STRING{\ \ Weight:\ }\STRING{1.0}\STRING{]\ from\ the\ }%
\STRING{Dark}\STRING{\ palette}\COMMA \STRING{[}\STRING{Hue:\ }1\STRING{%
\ \ Weight:\ }\STRING{1.0}\STRING{]\ from\ the\ }\STRING{Pastel}\STRING{%
\ palette}\COMMA \STRING{[}\STRING{Hue:\ }22\STRING{\ \ Weight:\ }\STRING{1.0%
}\STRING{]\ from\ the\ }\STRING{Light}\STRING{\ palette}\COMMA \STRING{[}%
\STRING{Hue:\ }14\STRING{\ \ Weight:\ }\STRING{1.0}\STRING{]\ from\ the\ }%
\STRING{Dim}\STRING{\ palette}\COMMA \STRING{[}\STRING{Hue:\ }11\STRING{%
\ \ Weight:\ }\STRING{1.0}\STRING{]\ from\ the\ }\STRING{Light}\STRING{%
\ palette}}%
\end{fricasmath}
\end{TeXOutput}
\formatResultType{List(Palette)}
\end{xtc}
\begin{xtc}
\begin{xtccomment}
Now the \spadtype{GraphImage} is created according to the component
specifications indicated above.
\end{xtccomment}
\begin{spadsrc}
g := makeGraphImage(llp,lpc,lc,lsize)$GRIMAGE 
\end{spadsrc}
\begin{TeXOutput}
\begin{fricasmath}{23}
\STRING{Graph\ with\ }12\STRING{\ point\ lists}%
\end{fricasmath}
\end{TeXOutput}
\formatResultType{GraphImage}
\end{xtc}
\begin{psXtc}
\begin{xtccomment}
The \spadfunFrom{makeViewport2D}{TwoDimensionalViewport} function now
creates a \spadtype{TwoDimensionalViewport} for this graph according to the
list of options specified within the brackets.
\end{xtccomment}
\begin{spadsrc}
makeViewport2D(g,[title("Lines")])$VIEW2D 
\end{spadsrc}
%
\end{psXtc}
%See Figure #.#.
\begin{xtc}
\begin{xtccomment}
This example demonstrates the use of the \spadtype{GraphImage} functions
\spadfunFrom{component}{GraphImage} and \spadfunFrom{appendPoint}{GraphImage}
in adding points to an empty \spadtype{GraphImage}.
\end{xtccomment}
\begin{spadsrc}
)clear all 
\end{spadsrc}
\begin{MessageOutput}
   All user variables and function definitions have been cleared.
\end{MessageOutput}
\end{xtc}
\begin{xtc}
\begin{xtccomment}
\end{xtccomment}
\begin{spadsrc}
g := graphImage()$GRIMAGE 
\end{spadsrc}
\begin{TeXOutput}
\begin{fricasmath}{1}
\STRING{Graph\ with\ }0\STRING{\ point\ lists}%
\end{fricasmath}
\end{TeXOutput}
\formatResultType{GraphImage}
\end{xtc}
\begin{xtc}
\begin{xtccomment}
\end{xtccomment}
\begin{spadsrc}
p1 := point [0,0]$(Point DFLOAT) 
\end{spadsrc}
\begin{TeXOutput}
\begin{fricasmath}{2}
\BRACKET{\STRING{0.0}\COMMA \STRING{0.0}}%
\end{fricasmath}
\end{TeXOutput}
\formatResultType{Point(DoubleFloat)}
\end{xtc}
\begin{xtc}
\begin{xtccomment}
\end{xtccomment}
\begin{spadsrc}
p2 := point [.25,.25]$(Point DFLOAT) 
\end{spadsrc}
\begin{TeXOutput}
\begin{fricasmath}{3}
\BRACKET{\STRING{0.25}\COMMA \STRING{0.25}}%
\end{fricasmath}
\end{TeXOutput}
\formatResultType{Point(DoubleFloat)}
\end{xtc}
\begin{xtc}
\begin{xtccomment}
\end{xtccomment}
\begin{spadsrc}
p3 := point [.5,.5]$(Point DFLOAT) 
\end{spadsrc}
\begin{TeXOutput}
\begin{fricasmath}{4}
\BRACKET{\STRING{0.5}\COMMA \STRING{0.5}}%
\end{fricasmath}
\end{TeXOutput}
\formatResultType{Point(DoubleFloat)}
\end{xtc}
\begin{xtc}
\begin{xtccomment}
\end{xtccomment}
\begin{spadsrc}
p4 := point [.75,.75]$(Point DFLOAT) 
\end{spadsrc}
\begin{TeXOutput}
\begin{fricasmath}{5}
\BRACKET{\STRING{0.75}\COMMA \STRING{0.75}}%
\end{fricasmath}
\end{TeXOutput}
\formatResultType{Point(DoubleFloat)}
\end{xtc}
\begin{xtc}
\begin{xtccomment}
\end{xtccomment}
\begin{spadsrc}
p5 := point [1,1]$(Point DFLOAT) 
\end{spadsrc}
\begin{TeXOutput}
\begin{fricasmath}{6}
\BRACKET{\STRING{1.0}\COMMA \STRING{1.0}}%
\end{fricasmath}
\end{TeXOutput}
\formatResultType{Point(DoubleFloat)}
\end{xtc}
\begin{xtc}
\begin{xtccomment}
\end{xtccomment}
\begin{spadsrc}
component(g,p1)$GRIMAGE
\end{spadsrc}
\end{xtc}
\begin{xtc}
\begin{xtccomment}
\end{xtccomment}
\begin{spadsrc}
component(g,p2)$GRIMAGE
\end{spadsrc}
\end{xtc}
\begin{xtc}
\begin{xtccomment}
\end{xtccomment}
\begin{spadsrc}
appendPoint(g,p3)$GRIMAGE
\end{spadsrc}
\end{xtc}
\begin{xtc}
\begin{xtccomment}
\end{xtccomment}
\begin{spadsrc}
appendPoint(g,p4)$GRIMAGE
\end{spadsrc}
\end{xtc}
\begin{xtc}
\begin{xtccomment}
\end{xtccomment}
\begin{spadsrc}
appendPoint(g,p5)$GRIMAGE
\end{spadsrc}
\end{xtc}
\begin{psXtc}
\begin{xtccomment}
Here is the graph.
\end{xtccomment}
\begin{spadsrc}
makeViewport2D(g,[title("Graph Points")])$VIEW2D 
\end{spadsrc}
%
\end{psXtc}
%
%See Figure #.#.
%
\begin{xtc}
\begin{xtccomment}
A list of points can also be made into a \spadtype{GraphImage} by using
the operation \spadfunFrom{coerce}{GraphImage}.  It is equivalent to adding
each point to \spad{g2} using \spadfunFrom{component}{GraphImage}.
\end{xtccomment}
\begin{spadsrc}
g2 := coerce([[p1],[p2],[p3],[p4],[p5]])$GRIMAGE  
\end{spadsrc}
\begin{TeXOutput}
\begin{fricasmath}{12}
\STRING{Graph\ with\ }5\STRING{\ point\ lists}%
\end{fricasmath}
\end{TeXOutput}
\formatResultType{GraphImage}
\end{xtc}
\begin{xtc}
\begin{xtccomment}
Now, create an empty \spadtype{TwoDimensionalViewport}.
\end{xtccomment}
\begin{spadsrc}
v := viewport2D()$VIEW2D 
\end{spadsrc}
\begin{TeXOutput}
\begin{fricasmath}{13}
\STRING{Closed\ or\ Undefined\ TwoDimensionalViewport:\ }\STRING{"FriCAS2D"}%
\end{fricasmath}
\end{TeXOutput}
\formatResultType{TwoDimensionalViewport}
\end{xtc}
\begin{xtc}
\begin{xtccomment}
\end{xtccomment}
\begin{spadsrc}
options(v,[title("Just Points")])$VIEW2D 
\end{spadsrc}
\begin{TeXOutput}
\begin{fricasmath}{14}
\STRING{Closed\ or\ Undefined\ TwoDimensionalViewport:\ }\STRING{"FriCAS2D"}%
\end{fricasmath}
\end{TeXOutput}
\formatResultType{TwoDimensionalViewport}
\end{xtc}
\begin{xtc}
\begin{xtccomment}
Place the graph into the viewport.
\end{xtccomment}
\begin{spadsrc}
putGraph(v,g2,1)$VIEW2D 
\end{spadsrc}
\end{xtc}
\begin{psXtc}
\begin{xtccomment}
Take a look.
\end{xtccomment}
\begin{spadsrc}
makeViewport2D(v)$VIEW2D 
\end{spadsrc}
%
\end{psXtc}

%See Figure #.#.

% *********************************************************************
\subsubsection{Creating a Two-Dimensional Viewport of a List of Points from a File}
% *********************************************************************

The following three functions read a list of points from a
file and then draw the points and the connecting lines. The
points are stored in the file in readable form as floating point numbers
(specifically, \spadtype{DoubleFloat} values) as an alternating
stream of \spad{x}- and \spad{y}-values. For example,
\begin{verbatim}
0.0 0.0     1.0 1.0     2.0 4.0
3.0 9.0     4.0 16.0    5.0 25.0
\end{verbatim}

\begin{xmpLines}
drawPoints(lp:List Point DoubleFloat):VIEW2D ==
  g := graphImage()$GRIMAGE
  for p in lp repeat
    component(g,p,pointColorDefault(),lineColorDefault(),
      pointSizeDefault())
  makeViewport2D(g,[title("Points")])$VIEW2D

drawLines(lp:List Point DoubleFloat):VIEW2D ==
  g := graphImage()$GRIMAGE
  component(g, lp, pointColorDefault(), lineColorDefault(),
    pointSizeDefault())$GRIMAGE
  makeViewport2D(g,[title("Points")])$VIEW2D

plotData2D(name, title) ==
  f:File(DFLOAT) := open(name,"input")
  lp:LIST(Point DFLOAT) := empty()
  while ((x := readIfCan!(f)) case DFLOAT) repeat
    y : DFLOAT := read!(f)
    lp := cons(point [x,y]$(Point DFLOAT), lp)
    lp
  close!(f)
  drawPoints(lp)
  drawLines(lp)
\end{xmpLines}
%
This command will actually create the viewport and the graph if
the point data is in the file \spad{"file.data"}.
\begin{xmpLines}
plotData2D("file.data", "2D Data Plot")
\end{xmpLines}

% *********************************************************************
\head{subsection}{Addendum: Appending a Graph to a Viewport Window Containing a Graph}{ugGraphTwoDappend}
% *********************************************************************

This section demonstrates how to append a \twodim{} graph to a viewport
already containing other graphs.
The default \spadfun{draw} command places a graph into the first
\spadtype{GraphImage} slot position of the \spadtype{TwoDimensionalViewport}.

\begin{xtc}
\begin{xtccomment}
This graph is in the first slot in its viewport.
\end{xtccomment}
\begin{spadsrc}
v1 := draw(sin(x),x=0..2*%pi) 
\end{spadsrc}
\begin{MessageOutput}
   Compiling function %B with type DoubleFloat -> DoubleFloat 
\end{MessageOutput}
\begin{TeXOutput}
\begin{fricasmath}{1}
\STRING{TwoDimensionalViewport:\ }\STRING{"sin(x)"}%
\end{fricasmath}
\end{TeXOutput}
\formatResultType{TwoDimensionalViewport}
\end{xtc}
\begin{xtc}
\begin{xtccomment}
So is this graph.
\end{xtccomment}
\begin{spadsrc}
v2 := draw(cos(x),x=0..2*%pi, curveColor==light red()) 
\end{spadsrc}
\begin{MessageOutput}
   Compiling function %D with type DoubleFloat -> DoubleFloat 
\end{MessageOutput}
\begin{TeXOutput}
\begin{fricasmath}{2}
\STRING{TwoDimensionalViewport:\ }\STRING{"cos(x)"}%
\end{fricasmath}
\end{TeXOutput}
\formatResultType{TwoDimensionalViewport}
\end{xtc}
\begin{xtc}
\begin{xtccomment}
The operation \spadfunFrom{getGraph}{TwoDimensionalViewport}
retrieves the \spadtype{GraphImage} \spad{g1} from the first slot position
in the viewport \spad{v1}.
\end{xtccomment}
\begin{spadsrc}
g1 := getGraph(v1,1) 
\end{spadsrc}
\begin{TeXOutput}
\begin{fricasmath}{3}
\STRING{Graph\ with\ }1\STRING{\ point\ list}%
\end{fricasmath}
\end{TeXOutput}
\formatResultType{GraphImage}
\end{xtc}
\begin{xtc}
\begin{xtccomment}
Now \spadfunFrom{putGraph}{TwoDimensionalViewport}
places \spad{g1} into the the second slot position of \spad{v2}.
\end{xtccomment}
\begin{spadsrc}
putGraph(v2,g1,2) 
\end{spadsrc}
\end{xtc}
\begin{psXtc}
\begin{xtccomment}
Display the new \spadtype{TwoDimensionalViewport} containing both graphs.
\end{xtccomment}
\begin{spadsrc}
makeViewport2D(v2) 
\end{spadsrc}
%
\end{psXtc}
Instead of using \spadfun{draw} to draw a graph and then extract
graph data we can use \spadfun{makeObject}.
\begin{xtc}
\begin{xtccomment}
First graph.
\end{xtccomment}
\begin{spadsrc}
g3 := makeObject(sin(x),x=-1..%pi,[]) 
\end{spadsrc}
\begin{MessageOutput}
   Compiling function %F with type DoubleFloat -> DoubleFloat 
\end{MessageOutput}
\begin{TeXOutput}
\begin{fricasmath}{5}
\STRING{Graph\ with\ }1\STRING{\ point\ list}%
\end{fricasmath}
\end{TeXOutput}
\formatResultType{GraphImage}
\end{xtc}
\begin{xtc}
\begin{xtccomment}
This graph is in the first slot in its viewport.
\end{xtccomment}
\begin{spadsrc}
v3 := draw(cos(x),x=-1..%pi, curveColor==light red()) 
\end{spadsrc}
\begin{MessageOutput}
   Compiling function %H with type DoubleFloat -> DoubleFloat 
\end{MessageOutput}
\begin{TeXOutput}
\begin{fricasmath}{6}
\STRING{TwoDimensionalViewport:\ }\STRING{"cos(x)"}%
\end{fricasmath}
\end{TeXOutput}
\formatResultType{TwoDimensionalViewport}
\end{xtc}
\begin{xtc}
\begin{xtccomment}
Now \spadfunFrom{putGraph}{TwoDimensionalViewport}
places \spad{g3} into the the second slot position of \spad{v3}.
\end{xtccomment}
\begin{spadsrc}
putGraph(v3,g3,2) 
\end{spadsrc}
\end{xtc}
\begin{psXtc}
\begin{xtccomment}
Display the new \spadtype{TwoDimensionalViewport} containing both graphs.
\end{xtccomment}
\begin{spadsrc}
makeViewport2D(v3) 
\end{spadsrc}
%
\end{psXtc}
% XXX: Without close we have trouble during documentation build
\begin{nullXtc}
\begin{xtccomment}
The viewports \spad{v1}, \spad{v2} and \spad{v3} are no longer needed so
we close them.
\end{xtccomment}
\begin{spadsrc}
close(v1); close(v2); close(v3) 
\end{spadsrc}
\end{nullXtc}

% XXX: We should wait for close to finish, but since there is
% other content on the page this should happen automatically
%
%See Figure #.#.
%

% *********************************************************************
\head{section}{Three-Dimensional Graphics}{ugGraphThreeD}
% *********************************************************************
%
The \Language{} \threedim{} graphics package provides the ability to
\index{graphics!three-dimensional}
%
\begin{itemize}
%
\item generate surfaces defined by a function of two real variables
%
\item generate space curves and tubes defined by parametric equations
%
\item generate surfaces defined by parametric equations
\end{itemize}
These graphs can be modified by using various options, such as calculating
points in the spherical coordinate system or changing the polygon grid size
of a surface.

% *********************************************************************
\head{subsection}{Plotting Three-Dimensional Functions of Two Variables}{ugGraphThreeDPlot}
% *********************************************************************

\index{surface!two variable function}
The simplest \threedim{} graph is that of a surface defined by a function
of two variables, \spad{z = f(x,y)}.

%
\beginImportant
The general format for drawing a surface defined by a formula \spad{f(x,y)}
of two variables \spad{x} and \spad{y} is:
%
\begin{center}
{\tt draw(f(x,y), x = a..b, y = c..d, {\it options})}
\end{center}
where \spad{a..b} and \spad{c..d} define the range of \spad{x}
and \spad{y}, and where {\it options} prescribes zero or more
options as described in \spadref{ugGraphThreeDOptions}.
An example of an option is \spad{title == "Title of Graph".}
An alternative format involving a function \spad{f} is also
available.
\endImportant

%
\begin{psXtc}
\begin{xtccomment}
The simplest way to plot a function of two variables is to use a formula.
With formulas you always precede the range specifications with
the variable name and an \spadSyntax{=} sign.
\end{xtccomment}
\begin{spadsrc}
draw(cos(x*y),x=-3..3,y=-3..3)
\end{spadsrc}
% window was 300 x 300
\epsffile[0 0 295 295]{3D2VarA.ps}
\end{psXtc}
%
\begin{xtc}
\begin{xtccomment}
If you intend to use a function more than once,
or it is long and complex, then first
give its definition to \Language{}.
\end{xtccomment}
\begin{spadsrc}
f(x,y) == sin(x)*cos(y) 
\end{spadsrc}
\end{xtc}
%
%
\begin{psXtc}
\begin{xtccomment}
To draw the function, just give its name and drop the variables
from the range specifications.
\Language{} compiles your function for efficient computation
of data for the graph.
Notice that \Language{} uses the text of your function as a
default title.
\end{xtccomment}
\begin{spadsrc}
draw(f,-%pi..%pi,-%pi..%pi) 
\end{spadsrc}
% window was 300 x 300
\epsffile[0 0 295 295]{3D2VarB.ps}
\end{psXtc}

% *********************************************************************
\head{subsection}{Plotting Three-Dimensional Parametric Space Curves}{ugGraphThreeDParm}
% *********************************************************************

A second kind of \threedim{} graph is a \threedim{} space curve
\index{curve!parametric space}
defined by the parametric equations for \spad{x(t)}, \spad{y(t)},
\index{parametric space curve}
and \spad{z(t)} as a function of an independent variable \spad{t}.

%
\beginImportant
The general format for drawing a \threedim{} space curve defined by
parametric formulas \spad{x = f(t)}, \spad{y = g(t)}, and
\spad{z = h(t)} is:
%
\begin{center}
{\tt draw(curve(f(t),g(t),h(t)), t = a..b, {\it options})}
\end{center}
where \spad{a..b} defines the range of the independent variable
\spad{t}, and where {\it options} prescribes zero or more options
as described in \spadref{ugGraphThreeDOptions}.
An example of an option is \spad{title == "Title of Graph".}
An alternative format involving functions \spad{f}, \spad{g} and
\spad{h} is also available.
\endImportant

%
\begin{psXtc}
\begin{xtccomment}
If you use explicit formulas to draw a space curve, always precede
the range specification with the variable name and an
\spadSyntax{=} sign.
\end{xtccomment}
\begin{spadsrc}
draw(curve(5*cos(t), 5*sin(t),t), t=-12..12)
\end{spadsrc}
% window was 300 x 300
\epsffile[0 0 295 295]{3DpscA.ps}
\end{psXtc}
%
\begin{xtc}
\begin{xtccomment}
Alternatively, you can draw space curves by referring to functions.
\end{xtccomment}
\begin{spadsrc}
i1(t:DFLOAT):DFLOAT == sin(t)*cos(3*t/5) 
\end{spadsrc}
\begin{MessageOutput}
   Function declaration i1 : DoubleFloat -> DoubleFloat has been added 
      to workspace.
\end{MessageOutput}
\end{xtc}
\begin{xtc}
\begin{xtccomment}
This is useful if the functions are to be used more than once \ldots
\end{xtccomment}
\begin{spadsrc}
i2(t:DFLOAT):DFLOAT == cos(t)*cos(3*t/5) 
\end{spadsrc}
\begin{MessageOutput}
   Function declaration i2 : DoubleFloat -> DoubleFloat has been added 
      to workspace.
\end{MessageOutput}
\end{xtc}
\begin{xtc}
\begin{xtccomment}
or if the functions are long and complex.
\end{xtccomment}
\begin{spadsrc}
i3(t:DFLOAT):DFLOAT == cos(t)*sin(3*t/5) 
\end{spadsrc}
\begin{MessageOutput}
   Function declaration i3 : DoubleFloat -> DoubleFloat has been added 
      to workspace.
\end{MessageOutput}
\end{xtc}
%
%
\begin{psXtc}
\begin{xtccomment}
Give the names of the functions and
drop the variable name specification in the second argument.
Again, \Language{} supplies a default title.
\end{xtccomment}
\begin{spadsrc}
draw(curve(i1,i2,i3),0..15*%pi) 
\end{spadsrc}
% window was 300 x 300
\epsffile[0 0 295 295]{3DpscB.ps}
\end{psXtc}

% *********************************************************************
\head{subsection}{Plotting Three-Dimensional Parametric Surfaces}{ugGraphThreeDPar}
% *********************************************************************

\index{surface!parametric}
A third kind of \threedim{} graph is a surface defined by
\index{parametric surface}
parametric equations for \spad{x(u,v)}, \spad{y(u,v)}, and
\spad{z(u,v)} of two independent variables \spad{u} and \spad{v}.

%
\beginImportant
The general format for drawing a \threedim{} graph defined by
parametric formulas \spad{x = f(u,v)}, \spad{y = g(u,v)},
and \spad{z = h(u,v)} is:
%
\begin{center}
{\tt draw(surface(f(u,v),g(u,v),h(u,v)), u = a..b, v = c..d, {\it options})}
\end{center}
where \spad{a..b} and \spad{c..d} define the range of the
independent variables \spad{u} and \spad{v}, and where
{\it options} prescribes zero or more options as described in
\spadref{ugGraphThreeDOptions}.
An example of an option is \spad{title == "Title of Graph".}
An alternative format involving functions \spad{f}, \spad{g} and
\spad{h} is also available.
\endImportant

%
\begin{psXtc}
\begin{xtccomment}
This example draws a graph of a surface plotted using the
parabolic cylindrical coordinate system option.
\index{coordinate system!parabolic cylindrical}
The values of the functions supplied to \spadfun{surface} are
\index{parabolic cylindrical coordinate system}
interpreted in coordinates as given by a {\tt coordinates} option,
here as parabolic cylindrical coordinates (see
\spadref{ugGraphCoord}).
\end{xtccomment}
\begin{spadsrc}
draw(surface(u*cos(v), u*sin(v), v*cos(u)), u=-4..4, v=0..%pi, coordinates== parabolicCylindrical)
\end{spadsrc}
% window was 300 x 300
\epsffile[0 0 295 295]{3DpsA.ps}
\end{psXtc}
%
Again, you can graph these parametric surfaces using functions,
if the functions are long and complex.
\begin{xtc}
\begin{xtccomment}
Here we declare the types of arguments and values to be of type
\spadtype{DoubleFloat}.
\end{xtccomment}
\begin{spadsrc}
n1(u:DFLOAT,v:DFLOAT):DFLOAT == u*cos(v) 
\end{spadsrc}
\begin{MessageOutput}
   Function declaration n1 : (DoubleFloat,DoubleFloat) -> DoubleFloat 
      has been added to workspace.
\end{MessageOutput}
\end{xtc}
\begin{xtc}
\begin{xtccomment}
As shown by previous examples, these declarations are necessary.
\end{xtccomment}
\begin{spadsrc}
n2(u:DFLOAT,v:DFLOAT):DFLOAT == u*sin(v) 
\end{spadsrc}
\begin{MessageOutput}
   Function declaration n2 : (DoubleFloat,DoubleFloat) -> DoubleFloat 
      has been added to workspace.
\end{MessageOutput}
\end{xtc}
\begin{xtc}
\begin{xtccomment}
In either case, \Language{} compiles the functions
when needed to graph a result.
\end{xtccomment}
\begin{spadsrc}
n3(u:DFLOAT,v:DFLOAT):DFLOAT == u 
\end{spadsrc}
\begin{MessageOutput}
   Function declaration n3 : (DoubleFloat,DoubleFloat) -> DoubleFloat 
      has been added to workspace.
\end{MessageOutput}
\end{xtc}
\begin{xtc}
\begin{xtccomment}
Without these declarations, you have to suffix floats
with \spad{@DFLOAT} to get a \spadtype{DoubleFloat} result.
However, a call here with an unadorned float produces a \spadtype{DoubleFloat}.
\end{xtccomment}
\begin{spadsrc}
n3(0.5,1.0)
\end{spadsrc}
\begin{MessageOutput}
   Compiling function n3 with type (DoubleFloat,DoubleFloat) -> 
      DoubleFloat 
\end{MessageOutput}
\begin{TeXOutput}
\begin{fricasmath}{4}
\STRING{0.5}%
\end{fricasmath}
\end{TeXOutput}
\formatResultType{DoubleFloat}
\end{xtc}
%
%
\begin{psXtc}
\begin{xtccomment}
Draw the surface by referencing the function names, this time
choosing the toroidal coordinate system.
\index{coordinate system!toroidal}
\index{toroidal coordinate system}
\end{xtccomment}
\begin{spadsrc}
draw(surface(n1,n2,n3), 1..4, 1..2*%pi, coordinates == toroidal(1$DFLOAT)) 
\end{spadsrc}
% window was 300 x 300
\epsffile[0 0 295 295]{3DpsB.ps}
\end{psXtc}

% *********************************************************************
\head{subsection}{Three-Dimensional Options}{ugGraphThreeDOptions}
% *********************************************************************

\index{graphics!3D options}
The \spadfun{draw} commands optionally take an optional list of options such
as {\tt coordinates} as shown in the last example.
Each option is given by the syntax: \spad{name} {\tt ==} \spad{value}.
Here is a list of the available options in the order that they are
described below:

\begin{tabular}{llll}
title&coordinates&var1Steps\\
style&tubeRadius&var2Steps\\
colorFunction&tubePoints&space\\
\end{tabular}

\begin{psXtc}
\begin{xtccomment}
The option \spad{title} gives your graph a title.
\index{graphics!3D options!title}
\end{xtccomment}
\begin{spadsrc}
draw(cos(x*y),x=0..2*%pi,y=0..%pi,title == "Title of Graph") 
\end{spadsrc}
% window was 300 x 300
\epsffile[0 0 295 295]{3DOptTtl.ps}
\end{psXtc}
%
\begin{psXtc}
\begin{xtccomment}
The \spad{style} determines which of four rendering algorithms is used for
\index{rendering}
the graph.
The choices are
{\tt "wireMesh"}, {\tt "solid"}, {\tt "shade"}, and {\tt "smooth"}.
\end{xtccomment}
\begin{spadsrc}
draw(cos(x*y),x=-3..3,y=-3..3, style=="smooth", title=="Smooth Option")
\end{spadsrc}
% window was 300 x 300
\epsffile[0 0 295 295]{3DOptSty.ps}
\end{psXtc}
%

In all but the wire-mesh style, polygons in a surface or tube plot
are normally colored in a graph according to their
\spad{z}-coordinate value.  Space curves are colored according to their
parametric variable value.
\index{graphics!3D options!color function}
To change this, you can give a coloring function.
\index{function!coloring}
The coloring function is sampled across the range of its arguments, then
normalized onto the standard \Language{} colormap.

\begin{xtc}
\begin{xtccomment}
A function of one variable  makes the color depend on the
value of the parametric variable specified for a tube plot.
\end{xtccomment}
\begin{spadsrc}
color1(t) == t 
\end{spadsrc}
\end{xtc}
\begin{psXtc}
\begin{xtccomment}
\end{xtccomment}
\begin{spadsrc}
draw(curve(sin(t), cos(t),0), t=0..2*%pi, tubeRadius == .3, colorFunction == color1) 
\end{spadsrc}
% window was 300 x 300
\epsffile[0 0 295 295]{3DOptCf1.ps}
\end{psXtc}
%
\begin{xtc}
\begin{xtccomment}
A function of two variables makes the color depend on the
values of the independent variables.
\end{xtccomment}
\begin{spadsrc}
color2(u,v) == u^2 - v^2 
\end{spadsrc}
\end{xtc}
\begin{psXtc}
\begin{xtccomment}
Use the option {\tt colorFunction} for special coloring.
\end{xtccomment}
\begin{spadsrc}
draw(cos(u*v), u=-3..3, v=-3..3, colorFunction == color2) 
\end{spadsrc}
% window was 300 x 300
\epsffile[0 0 295 295]{3DOptCf2.ps}
\end{psXtc}
%
\begin{xtc}
\begin{xtccomment}
With a three variable function, the
color also depends on the value of the function.
\end{xtccomment}
\begin{spadsrc}
color3(x,y,fxy) == sin(x*fxy) + cos(y*fxy) 
\end{spadsrc}
\end{xtc}
\begin{psXtc}
\begin{xtccomment}
\end{xtccomment}
\begin{spadsrc}
draw(cos(x*y), x=-3..3, y=-3..3, colorFunction == color3) 
\end{spadsrc}
% window was 300 x 300
\epsffile[0 0 295 295]{3DOptCf3.ps}
\end{psXtc}
%
Normally the Cartesian coordinate system is used.
\index{Cartesian!coordinate system}
To change this, use the {\tt coordinates} option.
\index{coordinate system!Cartesian}
For details, see \spadref{ugGraphCoord}.
%
%
\begin{xtc}
\begin{xtccomment}
\end{xtccomment}
\begin{spadsrc}
m(u:DFLOAT,v:DFLOAT):DFLOAT == 1 
\end{spadsrc}
\begin{MessageOutput}
   Function declaration m : (DoubleFloat,DoubleFloat) -> DoubleFloat 
      has been added to workspace.
\end{MessageOutput}
\end{xtc}
\begin{psXtc}
\begin{xtccomment}
Use the spherical
\index{spherical coordinate system}
coordinate system.
\index{coordinate system!spherical}
\end{xtccomment}
\begin{spadsrc}
draw(m, 0..2*%pi,0..%pi, coordinates == spherical, style=="shade") 
\end{spadsrc}
% window was 300 x 300
\epsffile[0 0 295 295]{3DOptCrd.ps}
\end{psXtc}
%
Space curves may be displayed as tubes with polygonal cross sections.
\index{tube}
Two options, {\tt tubeRadius} and {\tt tubePoints},  control the size and
shape of this cross section.
%
\begin{psXtc}
\begin{xtccomment}
The {\tt tubeRadius} option specifies the radius of the tube that
\index{tube!radius}
encircles the specified space curve.
\end{xtccomment}
\begin{spadsrc}
draw(curve(sin(t),cos(t),0),t=0..2*%pi, style=="shade", tubeRadius == .3)
\end{spadsrc}
% window was 300 x 300
\epsffile[0 0 295 295]{3DOptRad.ps}
\end{psXtc}
%
%
\begin{psXtc}
\begin{xtccomment}
The {\tt tubePoints} option specifies the number of vertices
\index{tube!points in polygon}
defining the polygon that is used to create a tube around the
specified space curve.
The larger this number is, the more cylindrical the tube becomes.
\end{xtccomment}
\begin{spadsrc}
draw(curve(sin(t), cos(t), 0), t=0..2*%pi, style=="shade", tubeRadius == .25, tubePoints == 3)
\end{spadsrc}
% window was 300 x 300
\epsffile[0 0 295 295]{3DOptPts.ps}
\end{psXtc}
%
\index{graphics!3D options!variable steps}
%
%
\begin{psXtc}
\begin{xtccomment}
Options \spadfunFrom{var1Steps}{DrawOption} and
\spadfunFrom{var2Steps}{DrawOption} specify the number of intervals into
which the grid defining a surface plot is subdivided with respect to the
first and second parameters of the surface function(s).
\end{xtccomment}
\begin{spadsrc}
draw(cos(x*y),x=-3..3,y=-3..3, style=="shade", var1Steps == 30, var2Steps == 30)
\end{spadsrc}
% window was 300 x 300
\epsffile[0 0 295 295]{3DOptvB.ps}
\end{psXtc}
%
The {\tt space} option
of a \spadfun{draw} command lets you build multiple graphs in three space.
To use this option, first create an empty three-space object,
then use the {\tt space} option thereafter.
There is no restriction as to the number or kinds
of graphs that can be combined this way.
\begin{xtc}
\begin{xtccomment}
Create an empty three-space object.
\end{xtccomment}
\begin{spadsrc}
s := create3Space()$(ThreeSpace DFLOAT) 
\end{spadsrc}
\begin{TeXOutput}
\begin{fricasmath}{5}
\STRING{3-Space\ with\ }0\STRING{\ components}%
\end{fricasmath}
\end{TeXOutput}
\formatResultType{ThreeSpace(DoubleFloat)}
\end{xtc}
%
%
\begin{xtc}
\begin{xtccomment}
\end{xtccomment}
\begin{spadsrc}
m(u:DFLOAT,v:DFLOAT):DFLOAT == 1 
\end{spadsrc}
\begin{MessageOutput}
   Function declaration m : (DoubleFloat,DoubleFloat) -> DoubleFloat 
      has been added to workspace.
\end{MessageOutput}
\begin{MessageOutput}
   1 old definition(s) deleted for function or rule m 
\end{MessageOutput}
\end{xtc}
\begin{psXtc}
\begin{xtccomment}
Add a graph to this three-space object.
The new graph destructively inserts the graph
into \spad{s}.
\end{xtccomment}
\begin{spadsrc}
draw(m,0..%pi,0..2*%pi, coordinates == spherical, space == s) 
\end{spadsrc}
% window was 300 x 300
\epsffile[0 0 295 295]{3Dmult1A.ps}
\end{psXtc}
%
%
\begin{psXtc}
\begin{xtccomment}
Add a second graph to \spad{s}.
\end{xtccomment}
\begin{spadsrc}
v := draw(curve(1.5*sin(t), 1.5*cos(t),0), t=0..2*%pi, tubeRadius == .25, space == s) 
\end{spadsrc}
% window was 300 x 300
\epsffile[0 0 295 295]{3Dmult1B.ps}
\end{psXtc}
%
A three-space object can also be obtained from an existing \threedim{} viewport
using the \spadfunFrom{subspace}{ThreeSpace} command.
You can then use \spadfun{makeViewport3D} to create a viewport window.
\begin{noOutputXtc}
\begin{xtccomment}
Assign to \spad{subsp} the three-space object in viewport \spad{v}.
\end{xtccomment}
\begin{spadsrc}
subsp := subspace v 
\end{spadsrc}
\begin{MessageOutput}
   There are 2 exposed and 0 unexposed library operations named 
      subspace having 1 argument(s) but none was determined to be 
      applicable. Use HyperDoc Browse, or issue
                            )display op subspace
      to learn more about the available operations. Perhaps 
      package-calling the operation or using coercions on the arguments
      will allow you to apply the operation.
\end{MessageOutput}
\begin{MessageOutput}
   Cannot find a definition or applicable library operation named 
      subspace with argument type(s) 
                                 Variable(v)
      
      Perhaps you should use "@" to indicate the required return type, 
      or "$" to specify which version of the function you need.
\end{MessageOutput}
\end{noOutputXtc}
\begin{noOutputXtc}
\begin{xtccomment}
Reset the space component of \spad{v} to the value of \spad{subsp}.
\end{xtccomment}
\begin{spadsrc}
subspace(v, subsp) 
\end{spadsrc}
\begin{MessageOutput}
   There are 1 exposed and 0 unexposed library operations named 
      subspace having 2 argument(s) but none was determined to be 
      applicable. Use HyperDoc Browse, or issue
                            )display op subspace
      to learn more about the available operations. Perhaps 
      package-calling the operation or using coercions on the arguments
      will allow you to apply the operation.
\end{MessageOutput}
\begin{MessageOutput}
   Cannot find a definition or applicable library operation named 
      subspace with argument type(s) 
                                 Variable(v)
                               Variable(subsp)
      
      Perhaps you should use "@" to indicate the required return type, 
      or "$" to specify which version of the function you need.
\end{MessageOutput}
\end{noOutputXtc}
\begin{noOutputXtc}
\begin{xtccomment}
Create a viewport window from a three-space object.
\end{xtccomment}
\begin{spadsrc}
makeViewport3D(subsp,"Graphs") 
\end{spadsrc}
\begin{MessageOutput}
   There are 2 exposed and 0 unexposed library operations named 
      makeViewport3D having 2 argument(s) but none was determined to be
      applicable. Use HyperDoc Browse, or issue
                         )display op makeViewport3D
      to learn more about the available operations. Perhaps 
      package-calling the operation or using coercions on the arguments
      will allow you to apply the operation.
\end{MessageOutput}
\begin{MessageOutput}
   Cannot find a definition or applicable library operation named 
      makeViewport3D with argument type(s) 
                               Variable(subsp)
                                   String
      
      Perhaps you should use "@" to indicate the required return type, 
      or "$" to specify which version of the function you need.
\end{MessageOutput}
\end{noOutputXtc}

% *********************************************************************
\head{subsection}{The makeObject Command}{ugGraphMakeObject}
% *********************************************************************

An alternate way to create multiple graphs is to use
\spadfun{makeObject}.
The \spadfun{makeObject} command is similar to the \spadfun{draw}
command, except that it returns a three-space object rather than a
\spadtype{ThreeDimensionalViewport}.
In fact, \spadfun{makeObject} is called by the \spadfun{draw}
command to create the \spadtype{ThreeSpace} then
\spadfunFrom{makeViewport3D}{ThreeDimensionalViewport} to create a
viewport window.

\begin{xtc}
\begin{xtccomment}
\end{xtccomment}
\begin{spadsrc}
m(u:DFLOAT,v:DFLOAT):DFLOAT == 1 
\end{spadsrc}
\begin{MessageOutput}
   Function declaration m : (DoubleFloat,DoubleFloat) -> DoubleFloat 
      has been added to workspace.
\end{MessageOutput}
\end{xtc}
\begin{xtc}
\begin{xtccomment}
Do the last example a new way.
First use \spadfun{makeObject} to
create a three-space object \spad{sph}.
\end{xtccomment}
\begin{spadsrc}
sph := makeObject(m, 0..%pi, 0..2*%pi, coordinates==spherical)
\end{spadsrc}
\begin{MessageOutput}
   Compiling function m with type (DoubleFloat,DoubleFloat) -> 
      DoubleFloat 
\end{MessageOutput}
\begin{TeXOutput}
\begin{fricasmath}{2}
\STRING{3-Space\ with\ }1\STRING{\ component}%
\end{fricasmath}
\end{TeXOutput}
\formatResultType{ThreeSpace(DoubleFloat)}
\end{xtc}
\begin{xtc}
\begin{xtccomment}
Add a second object to \spad{sph}.
\end{xtccomment}
\begin{spadsrc}
makeObject(curve(1.5*sin(t), 1.5*cos(t), 0), t=0..2*%pi, space == sph, tubeRadius == .25) 
\end{spadsrc}
\begin{MessageOutput}
   Compiling function %K with type DoubleFloat -> DoubleFloat 
\end{MessageOutput}
\begin{MessageOutput}
   Compiling function %M with type DoubleFloat -> DoubleFloat 
\end{MessageOutput}
\begin{MessageOutput}
   Compiling function %O with type DoubleFloat -> DoubleFloat 
\end{MessageOutput}
\begin{TeXOutput}
\begin{fricasmath}{3}
\STRING{3-Space\ with\ }2\STRING{\ components}%
\end{fricasmath}
\end{TeXOutput}
\formatResultType{ThreeSpace(DoubleFloat)}
\end{xtc}
\begin{noOutputXtc}
\begin{xtccomment}
Create and display a viewport
containing \spad{sph}.
\end{xtccomment}
\begin{spadsrc}
makeViewport3D(sph,"Multiple Objects") 
\end{spadsrc}
\begin{TeXOutput}
\begin{fricasmath}{4}
\STRING{ThreeDimensionalViewport:\ }\STRING{"Multiple\ Objects"}%
\end{fricasmath}
\end{TeXOutput}
\formatResultType{ThreeDimensionalViewport}
\end{noOutputXtc}

Note that an undefined \spadtype{ThreeSpace} parameter declared in a
\spadfun{makeObject} or \spadfun{draw} command results in an error.
Use the \spadfunFrom{create3Space}{ThreeSpace} function to define a
\spadtype{ThreeSpace}, or obtain a \spadtype{ThreeSpace} that has been
previously generated before including it in a command line.

% *********************************************************************
\head{subsection}{Building Three-Dimensional Objects From Primitives}{ugGraphThreeDBuild}
% *********************************************************************

Rather than using the \spadfun{draw} and \spadfun{makeObject} commands,
\index{graphics!advanced!build 3D objects}
you can create \threedim{} graphs from primitives.
Operation \spadfunFrom{create3Space}{ThreeSpace} creates a
three-space object to which points, curves and polygons
can be added using the operations from the \spadtype{ThreeSpace}
domain.
The resulting object can then be displayed in a viewport using
\spadfunFrom{makeViewport3D}{ThreeDimensionalViewport}.

\begin{xtc}
\begin{xtccomment}
Create the empty three-space object \spad{space}.
\end{xtccomment}
\begin{spadsrc}
space := create3Space()$(ThreeSpace DFLOAT) 
\end{spadsrc}
\begin{TeXOutput}
\begin{fricasmath}{1}
\STRING{3-Space\ with\ }0\STRING{\ components}%
\end{fricasmath}
\end{TeXOutput}
\formatResultType{ThreeSpace(DoubleFloat)}
\end{xtc}

Objects can be sent to this \spad{space} using the operations
exported by the \spadtype{ThreeSpace} domain.
\exptypeindex{ThreeSpace}
The following examples place curves into \spad{space}.

\begin{xtc}
\begin{xtccomment}
Add these eight curves to the space.
\end{xtccomment}
\begin{spadsrc}
closedCurve(space,[[0,30,20], [0,30,30], [0,40,30], [0,40,100], [0,30,100],[0,30,110], [0,60,110], [0,60,100], [0,50,100], [0,50,30], [0,60,30], [0,60,20]]) 
\end{spadsrc}
\begin{TeXOutput}
\begin{fricasmath}{2}
\STRING{3-Space\ with\ }1\STRING{\ component}%
\end{fricasmath}
\end{TeXOutput}
\formatResultType{ThreeSpace(DoubleFloat)}
\end{xtc}
\begin{xtc}
\begin{xtccomment}
\end{xtccomment}
\begin{spadsrc}
closedCurve(space,[[80,0,30], [80,0,100], [70,0,110], [40,0,110], [30,0,100], [30,0,90], [40,0,90], [40,0,95], [45,0,100], [65,0,100], [70,0,95], [70,0,35]]) 
\end{spadsrc}
\begin{TeXOutput}
\begin{fricasmath}{3}
\STRING{3-Space\ with\ }2\STRING{\ components}%
\end{fricasmath}
\end{TeXOutput}
\formatResultType{ThreeSpace(DoubleFloat)}
\end{xtc}
\begin{xtc}
\begin{xtccomment}
\end{xtccomment}
\begin{spadsrc}
closedCurve(space,[[70,0,35], [65,0,30], [45,0,30], [40,0,35], [40,0,60], [50,0,60], [50,0,70], [30,0,70], [30,0,30], [40,0,20], [70,0,20], [80,0,30]]) 
\end{spadsrc}
\begin{TeXOutput}
\begin{fricasmath}{4}
\STRING{3-Space\ with\ }3\STRING{\ components}%
\end{fricasmath}
\end{TeXOutput}
\formatResultType{ThreeSpace(DoubleFloat)}
\end{xtc}
\begin{xtc}
\begin{xtccomment}
\end{xtccomment}
\begin{spadsrc}
closedCurve(space,[[0,70,20], [0,70,110], [0,110,110], [0,120,100], [0,120,70], [0,115,65], [0,120,60], [0,120,30], [0,110,20], [0,80,20], [0,80,30], [0,80,20]]) 
\end{spadsrc}
\begin{TeXOutput}
\begin{fricasmath}{5}
\STRING{3-Space\ with\ }4\STRING{\ components}%
\end{fricasmath}
\end{TeXOutput}
\formatResultType{ThreeSpace(DoubleFloat)}
\end{xtc}
\begin{xtc}
\begin{xtccomment}
\end{xtccomment}
\begin{spadsrc}
closedCurve(space,[[0,105,30], [0,110,35], [0,110,55], [0,105,60], [0,80,60], [0,80,70], [0,105,70], [0,110,75], [0,110,95], [0,105,100], [0,80,100], [0,80,20], [0,80,30]]) 
\end{spadsrc}
\begin{TeXOutput}
\begin{fricasmath}{6}
\STRING{3-Space\ with\ }5\STRING{\ components}%
\end{fricasmath}
\end{TeXOutput}
\formatResultType{ThreeSpace(DoubleFloat)}
\end{xtc}
\begin{xtc}
\begin{xtccomment}
\end{xtccomment}
\begin{spadsrc}
closedCurve(space,[[140,0,20], [140,0,110], [130,0,110], [90,0,20], [101,0,20],[114,0,50], [130,0,50], [130,0,60], [119,0,60], [130,0,85], [130,0,20]]) 
\end{spadsrc}
\begin{TeXOutput}
\begin{fricasmath}{7}
\STRING{3-Space\ with\ }6\STRING{\ components}%
\end{fricasmath}
\end{TeXOutput}
\formatResultType{ThreeSpace(DoubleFloat)}
\end{xtc}
\begin{xtc}
\begin{xtccomment}
\end{xtccomment}
\begin{spadsrc}
closedCurve(space,[[0,140,20], [0,140,110], [0,150,110], [0,170,50], [0,190,110], [0,200,110], [0,200,20], [0,190,20], [0,190,75], [0,175,35], [0,165,35],[0,150,75], [0,150,20]]) 
\end{spadsrc}
\begin{TeXOutput}
\begin{fricasmath}{8}
\STRING{3-Space\ with\ }7\STRING{\ components}%
\end{fricasmath}
\end{TeXOutput}
\formatResultType{ThreeSpace(DoubleFloat)}
\end{xtc}
\begin{xtc}
\begin{xtccomment}
\end{xtccomment}
\begin{spadsrc}
closedCurve(space,[[200,0,20], [200,0,110], [189,0,110], [160,0,45], [160,0,110], [150,0,110], [150,0,20], [161,0,20], [190,0,85], [190,0,20]]) 
\end{spadsrc}
\begin{TeXOutput}
\begin{fricasmath}{9}
\STRING{3-Space\ with\ }8\STRING{\ components}%
\end{fricasmath}
\end{TeXOutput}
\formatResultType{ThreeSpace(DoubleFloat)}
\end{xtc}
\begin{psXtc}
\begin{xtccomment}
Create and display the viewport using \spadfun{makeViewport3D}.
Options may also be given but here are displayed as a list with values
enclosed in parentheses.
\end{xtccomment}
\begin{spadsrc}
makeViewport3D(space, title == "Letters") 
\end{spadsrc}
% window was 300 x 300
\epsffile[0 0 295 295]{3DBuildA.ps}
\end{psXtc}

% *********************************************************************
\subsubsection{Cube Example}
% *********************************************************************

As a second example of the use of primitives, we generate a cube using a
polygon mesh.
It is important to use a consistent orientation of the polygons for
correct generation of \threedim{} objects.

\begin{xtc}
\begin{xtccomment}
Again start with an empty three-space object.
\end{xtccomment}
\begin{spadsrc}
spaceC := create3Space()$(ThreeSpace DFLOAT) 
\end{spadsrc}
\begin{TeXOutput}
\begin{fricasmath}{10}
\STRING{3-Space\ with\ }0\STRING{\ components}%
\end{fricasmath}
\end{TeXOutput}
\formatResultType{ThreeSpace(DoubleFloat)}
\end{xtc}
\begin{xtc}
\begin{xtccomment}
For convenience,
give \spadtype{DoubleFloat} values \spad{+1} and \spad{-1} names.
\end{xtccomment}
\begin{spadsrc}
x: DFLOAT := 1 
\end{spadsrc}
\begin{TeXOutput}
\begin{fricasmath}{11}
\STRING{1.0}%
\end{fricasmath}
\end{TeXOutput}
\formatResultType{DoubleFloat}
\end{xtc}
\begin{xtc}
\begin{xtccomment}
\end{xtccomment}
\begin{spadsrc}
y: DFLOAT := -1 
\end{spadsrc}
\begin{TeXOutput}
\begin{fricasmath}{12}
-{\STRING{1.0}}%
\end{fricasmath}
\end{TeXOutput}
\formatResultType{DoubleFloat}
\end{xtc}
\begin{xtc}
\begin{xtccomment}
Define the vertices of the cube.
\end{xtccomment}
\begin{spadsrc}
a := point [x,x,y,1::DFLOAT]$(Point DFLOAT) 
\end{spadsrc}
\begin{TeXOutput}
\begin{fricasmath}{13}
\BRACKET{\STRING{1.0}\COMMA \STRING{1.0}\COMMA -{\STRING{1.0}}\COMMA \STRING{%
1.0}}%
\end{fricasmath}
\end{TeXOutput}
\formatResultType{Point(DoubleFloat)}
\end{xtc}
\begin{xtc}
\begin{xtccomment}
\end{xtccomment}
\begin{spadsrc}
b := point [y,x,y,4::DFLOAT]$(Point DFLOAT) 
\end{spadsrc}
\begin{TeXOutput}
\begin{fricasmath}{14}
\BRACKET{-{\STRING{1.0}}\COMMA \STRING{1.0}\COMMA -{\STRING{1.0}}\COMMA %
\STRING{4.0}}%
\end{fricasmath}
\end{TeXOutput}
\formatResultType{Point(DoubleFloat)}
\end{xtc}
\begin{xtc}
\begin{xtccomment}
\end{xtccomment}
\begin{spadsrc}
c := point [y,x,x,8::DFLOAT]$(Point DFLOAT) 
\end{spadsrc}
\begin{TeXOutput}
\begin{fricasmath}{15}
\BRACKET{-{\STRING{1.0}}\COMMA \STRING{1.0}\COMMA \STRING{1.0}\COMMA \STRING{%
8.0}}%
\end{fricasmath}
\end{TeXOutput}
\formatResultType{Point(DoubleFloat)}
\end{xtc}
\begin{xtc}
\begin{xtccomment}
\end{xtccomment}
\begin{spadsrc}
d := point [x,x,x,12::DFLOAT]$(Point DFLOAT) 
\end{spadsrc}
\begin{TeXOutput}
\begin{fricasmath}{16}
\BRACKET{\STRING{1.0}\COMMA \STRING{1.0}\COMMA \STRING{1.0}\COMMA \STRING{%
12.0}}%
\end{fricasmath}
\end{TeXOutput}
\formatResultType{Point(DoubleFloat)}
\end{xtc}
\begin{xtc}
\begin{xtccomment}
\end{xtccomment}
\begin{spadsrc}
e := point [x,y,y,16::DFLOAT]$(Point DFLOAT) 
\end{spadsrc}
\begin{TeXOutput}
\begin{fricasmath}{17}
\BRACKET{\STRING{1.0}\COMMA -{\STRING{1.0}}\COMMA -{\STRING{1.0}}\COMMA %
\STRING{16.0}}%
\end{fricasmath}
\end{TeXOutput}
\formatResultType{Point(DoubleFloat)}
\end{xtc}
\begin{xtc}
\begin{xtccomment}
\end{xtccomment}
\begin{spadsrc}
f := point [y,y,y,20::DFLOAT]$(Point DFLOAT) 
\end{spadsrc}
\begin{TeXOutput}
\begin{fricasmath}{18}
\BRACKET{-{\STRING{1.0}}\COMMA -{\STRING{1.0}}\COMMA -{\STRING{1.0}}\COMMA %
\STRING{20.0}}%
\end{fricasmath}
\end{TeXOutput}
\formatResultType{Point(DoubleFloat)}
\end{xtc}
\begin{xtc}
\begin{xtccomment}
\end{xtccomment}
\begin{spadsrc}
g := point [y,y,x,24::DFLOAT]$(Point DFLOAT) 
\end{spadsrc}
\begin{TeXOutput}
\begin{fricasmath}{19}
\BRACKET{-{\STRING{1.0}}\COMMA -{\STRING{1.0}}\COMMA \STRING{1.0}\COMMA %
\STRING{24.0}}%
\end{fricasmath}
\end{TeXOutput}
\formatResultType{Point(DoubleFloat)}
\end{xtc}
\begin{xtc}
\begin{xtccomment}
\end{xtccomment}
\begin{spadsrc}
h := point [x,y,x,27::DFLOAT]$(Point DFLOAT) 
\end{spadsrc}
\begin{TeXOutput}
\begin{fricasmath}{20}
\BRACKET{\STRING{1.0}\COMMA -{\STRING{1.0}}\COMMA \STRING{1.0}\COMMA \STRING{%
27.0}}%
\end{fricasmath}
\end{TeXOutput}
\formatResultType{Point(DoubleFloat)}
\end{xtc}
\begin{xtc}
\begin{xtccomment}
Add the faces of the cube as polygons to the space using a
consistent orientation.
\end{xtccomment}
\begin{spadsrc}
polygon(spaceC,[d,c,g,h]) 
\end{spadsrc}
\begin{TeXOutput}
\begin{fricasmath}{21}
\STRING{3-Space\ with\ }1\STRING{\ component}%
\end{fricasmath}
\end{TeXOutput}
\formatResultType{ThreeSpace(DoubleFloat)}
\end{xtc}
\begin{xtc}
\begin{xtccomment}
\end{xtccomment}
\begin{spadsrc}
polygon(spaceC,[d,h,e,a]) 
\end{spadsrc}
\begin{TeXOutput}
\begin{fricasmath}{22}
\STRING{3-Space\ with\ }2\STRING{\ components}%
\end{fricasmath}
\end{TeXOutput}
\formatResultType{ThreeSpace(DoubleFloat)}
\end{xtc}
\begin{xtc}
\begin{xtccomment}
\end{xtccomment}
\begin{spadsrc}
polygon(spaceC,[c,d,a,b]) 
\end{spadsrc}
\begin{TeXOutput}
\begin{fricasmath}{23}
\STRING{3-Space\ with\ }3\STRING{\ components}%
\end{fricasmath}
\end{TeXOutput}
\formatResultType{ThreeSpace(DoubleFloat)}
\end{xtc}
\begin{xtc}
\begin{xtccomment}
\end{xtccomment}
\begin{spadsrc}
polygon(spaceC,[g,c,b,f]) 
\end{spadsrc}
\begin{TeXOutput}
\begin{fricasmath}{24}
\STRING{3-Space\ with\ }4\STRING{\ components}%
\end{fricasmath}
\end{TeXOutput}
\formatResultType{ThreeSpace(DoubleFloat)}
\end{xtc}
\begin{xtc}
\begin{xtccomment}
\end{xtccomment}
\begin{spadsrc}
polygon(spaceC,[h,g,f,e]) 
\end{spadsrc}
\begin{TeXOutput}
\begin{fricasmath}{25}
\STRING{3-Space\ with\ }5\STRING{\ components}%
\end{fricasmath}
\end{TeXOutput}
\formatResultType{ThreeSpace(DoubleFloat)}
\end{xtc}
\begin{xtc}
\begin{xtccomment}
\end{xtccomment}
\begin{spadsrc}
polygon(spaceC,[e,f,b,a]) 
\end{spadsrc}
\begin{TeXOutput}
\begin{fricasmath}{26}
\STRING{3-Space\ with\ }6\STRING{\ components}%
\end{fricasmath}
\end{TeXOutput}
\formatResultType{ThreeSpace(DoubleFloat)}
\end{xtc}
\begin{psXtc}
\begin{xtccomment}
Create and display the viewport.
\end{xtccomment}
\begin{spadsrc}
makeViewport3D(spaceC, title == "Cube") 
\end{spadsrc}
% window was 300 x 300
\epsffile[0 0 295 295]{3DBuildB.ps}
\end{psXtc}

% *********************************************************************
\head{subsection}{Coordinate System Transformations}{ugGraphCoord}
% *********************************************************************
\index{graphics!advanced!coordinate systems}

The \spadtype{CoordinateSystems} package provides coordinate transformation
functions that map a given data point from the coordinate system specified
into the Cartesian coordinate system.
\exptypeindex{CoordinateSystems}
The default coordinate system, given a triplet \spad{(f(u,v), u, v)}, assumes
that \spad{z = f(u, v)}, \spad{x = u} and \spad{y = v},
that is, reads the coordinates in \spad{(z, x, y)} order.

\begin{xtc}
\begin{xtccomment}
\end{xtccomment}
\begin{spadsrc}
m(u:DFLOAT,v:DFLOAT):DFLOAT == u^2 
\end{spadsrc}
\begin{MessageOutput}
   Function declaration m : (DoubleFloat,DoubleFloat) -> DoubleFloat 
      has been added to workspace.
\end{MessageOutput}
\end{xtc}
%
\begin{psXtc}
\begin{xtccomment}
Graph plotted in default coordinate system.
\end{xtccomment}
\begin{spadsrc}
draw(m,0..3,0..5) 
\end{spadsrc}
% window was 300 x 300
\epsffile[0 0 295 295]{defcoord.ps}
\end{psXtc}

The \spad{z} coordinate comes first since the first argument of
the \spadfun{draw} command gives its values.
In general, the coordinate systems \Language{} provides, or any
that you make up, must provide a map to an \spad{(x, y, z)} triplet in
order to be compatible with the
\spadfunFrom{coordinates}{DrawOption} \spadtype{DrawOption}.
\exptypeindex{DrawOption}
Here is an example.

\begin{xtc}
\begin{xtccomment}
Define the identity function.
\end{xtccomment}
\begin{spadsrc}
cartesian(point:Point DFLOAT):Point DFLOAT == point 
\end{spadsrc}
\begin{MessageOutput}
   Function declaration cartesian : Point(DoubleFloat) -> Point(
      DoubleFloat) has been added to workspace.
\end{MessageOutput}
\end{xtc}
\begin{psXtc}
\begin{xtccomment}
Pass \spad{cartesian} as the \spadfunFrom{coordinates}{DrawOption}
parameter to the \spadfun{draw} command.
\end{xtccomment}
\begin{spadsrc}
draw(m,0..3,0..5,coordinates==cartesian) 
\end{spadsrc}
% window was 300 x 300
\epsffile[0 0 295 295]{cartcoord.ps}
\end{psXtc}
%

What happened?
The option {\tt coordinates == cartesian} directs \Language{} to
treat the dependent variable \spad{m} defined by
$m=u^2$ as the \spad{x} coordinate.
Thus the triplet of values \spad{(m, u, v)} is transformed to
coordinates \spad{(x, y, z)} and so we get the graph of
$x=y^2$.

Here is another example.
The \spadfunFrom{cylindrical}{CoordinateSystems} transform takes
\index{coordinate system!cylindrical}
input of the form \spad{(w,u,v)}, interprets it in the order
\index{cylindrical coordinate system}
($r$,$\theta$,$z$)
and maps it to the Cartesian coordinates
$x=r\cos(\theta)$, $y=r\sin(\theta)$, $z=z$
in which
$r$ is the radius,
$\theta$ is the angle and
$z$ is the z-coordinate.
\begin{xtc}
\begin{xtccomment}
An example using the \spadfunFrom{cylindrical}{CoordinateSystems}
coordinates for the constant \spad{r = 3}.
\end{xtccomment}
\begin{spadsrc}
f(u:DFLOAT,v:DFLOAT):DFLOAT == 3 
\end{spadsrc}
\begin{MessageOutput}
   Function declaration f : (DoubleFloat,DoubleFloat) -> DoubleFloat 
      has been added to workspace.
\end{MessageOutput}
\end{xtc}
\begin{psXtc}
\begin{xtccomment}
Graph plotted in cylindrical coordinates.
\end{xtccomment}
\begin{spadsrc}
draw(f,0..%pi,0..6,coordinates==cylindrical) 
\end{spadsrc}
% window was 300 x 300
\epsffile[0 0 295 295]{cylCoord.ps}
\end{psXtc}

Suppose you would like to specify \smath{z} as a function of
\smath{r} and $\theta$ instead of just
\smath{r}?
Well, you still can use the \spadfun{cylindrical} \Language{}
transformation but we have to reorder the triplet before
passing it to the transformation.

\begin{xtc}
\begin{xtccomment}
First, let's create a point to
work with and call it \spad{pt} with some color \spad{col}.
\end{xtccomment}
\begin{spadsrc}
col := 5 
\end{spadsrc}
\begin{TeXOutput}
\begin{fricasmath}{4}
5%
\end{fricasmath}
\end{TeXOutput}
\formatResultType{PositiveInteger}
\end{xtc}
\begin{xtc}
\begin{xtccomment}
\end{xtccomment}
\begin{spadsrc}
pt := point[1,2,3,col]$(Point DFLOAT) 
\end{spadsrc}
\begin{TeXOutput}
\begin{fricasmath}{5}
\BRACKET{\STRING{1.0}\COMMA \STRING{2.0}\COMMA \STRING{3.0}\COMMA \STRING{5.0%
}}%
\end{fricasmath}
\end{TeXOutput}
\formatResultType{Point(DoubleFloat)}
\end{xtc}
The reordering you want is
$(z,r, \theta)$ to
$(r, \theta,z)$
so that the first element is moved to the third element, while the second
and third elements move forward and the color element does not change.
\begin{xtc}
\begin{xtccomment}
Define a function \userfun{reorder} to reorder the point elements.
\end{xtccomment}
\begin{spadsrc}
reorder(p:Point DFLOAT):Point DFLOAT == point[p.2, p.3, p.1, p.4] 
\end{spadsrc}
\begin{MessageOutput}
   Function declaration reorder : Point(DoubleFloat) -> Point(
      DoubleFloat) has been added to workspace.
\end{MessageOutput}
\end{xtc}
\begin{xtc}
\begin{xtccomment}
The function moves the second and third elements
forward but the color does not change.
\end{xtccomment}
\begin{spadsrc}
reorder pt 
\end{spadsrc}
\begin{MessageOutput}
   Compiling function reorder with type Point(DoubleFloat) -> Point(
      DoubleFloat) 
\end{MessageOutput}
\begin{TeXOutput}
\begin{fricasmath}{7}
\BRACKET{\STRING{2.0}\COMMA \STRING{3.0}\COMMA \STRING{1.0}\COMMA \STRING{5.0%
}}%
\end{fricasmath}
\end{TeXOutput}
\formatResultType{Point(DoubleFloat)}
\end{xtc}
\begin{xtc}
\begin{xtccomment}
The function \userfun{newmap} converts our reordered version of
the cylindrical coordinate system to the standard
$(x,y,z)$ Cartesian system.
\end{xtccomment}
\begin{spadsrc}
newmap(pt:Point DFLOAT):Point DFLOAT == cylindrical(reorder pt) 
\end{spadsrc}
\begin{MessageOutput}
   Function declaration newmap : Point(DoubleFloat) -> Point(
      DoubleFloat) has been added to workspace.
\end{MessageOutput}
\end{xtc}
\begin{xtc}
\begin{xtccomment}
\end{xtccomment}
\begin{spadsrc}
newmap pt 
\end{spadsrc}
\begin{MessageOutput}
   Compiling function newmap with type Point(DoubleFloat) -> Point(
      DoubleFloat) 
\end{MessageOutput}
\begin{TeXOutput}
\begin{fricasmath}{9}
\BRACKET{-{\STRING{1.9799849932008908}}\COMMA \STRING{0.2822400161197344}%
\COMMA \STRING{1.0}\COMMA \STRING{5.0}}%
\end{fricasmath}
\end{TeXOutput}
\formatResultType{Point(DoubleFloat)}
\end{xtc}
%
\begin{psXtc}
\begin{xtccomment}
Graph the same function \spad{f} using the coordinate mapping of the function
\spad{newmap}, so it is now interpreted as
$z=3$:
\end{xtccomment}
\begin{spadsrc}
draw(f,0..3,0..2*%pi,coordinates==newmap) 
\end{spadsrc}
% window was 300 x 300
\epsffile[0 0 295 295]{newmap.ps}
\end{psXtc}

% I think this is good to say here: it shows a lot of depth. RSS
{\sloppy
The \spadtype{CoordinateSystems} package exports the following
\index{coordinate system}
operations:
\spadfun{bipolar},
\spadfun{bipolarCylindrical},
\spadfun{cartesian},
\spadfun{conical},
\spadfun{cylindrical},
\spadfun{elliptic},
\spadfun{ellipticCylindrical},
\spadfun{oblateSpheroidal},
\spadfun{parabolic},
\spadfun{parabolicCylindrical},
\spadfun{paraboloidal},
\spadfun{polar},
\spadfun{prolateSpheroidal},
\spadfun{spherical}, and
\spadfun{toroidal}.
Use \Browse{} or the \spadsys{)show} system command
\syscmdindex{show}
to get more information.

}

% *********************************************************************
\head{subsection}{Three-Dimensional Clipping}{ugGraphClip}
% *********************************************************************

A \threedim{} graph can be explicitly clipped within the \spadfun{draw}
\index{graphics!advanced!clip}
command by indicating a minimum and maximum threshold for the
\index{clipping}
given function definition.
These thresholds can be defined using the \Language{} \spadfun{min}
and \spadfun{max} functions.
\begin{xtc}
\begin{xtccomment}
\end{xtccomment}
\begin{spadsrc}
gamma(x,y) ==
  g := Gamma complex(x,y)
  point [x, y, max( min(real g, 4), -4), argument g]
\end{spadsrc}
\end{xtc}
\begin{psXtc}
\begin{xtccomment}
Here is an example that clips
the gamma function in order to eliminate the extreme divergence it creates.
\end{xtccomment}
\begin{spadsrc}
draw(gamma,-%pi..%pi,-%pi..%pi,var1Steps==50,var2Steps==50) 
\end{spadsrc}
% window was 300 x 300
\epsffile[0 0 295 295]{clipGamma.ps}
\end{psXtc}

% *********************************************************************
\head{subsection}{Three-Dimensional Control-Panel}{ugGraphThreeDControl}
% *********************************************************************
\index{graphics!3D control-panel}
Once you have created a viewport, move your mouse to the viewport
and click with your left mouse button.
This displays a control-panel on the side of the viewport
that is closest to where you clicked.

\typeout{3D control-panel.}
\begin{figure}[htbp]
%{\epsfverbosetrue\epsfxsize=2in%
%\def\epsfsize#1#2{\epsfxsize}\hspace*{\baseLeftSkip}%
%\epsffile[0 0 144 289]{3Dctrl.ps}}
\begin{picture}(183,252)%(-125,0)
\special{psfile=3Dctrl.ps}
\end{picture}
\caption{Three-dimensional control-panel.}
\end{figure}

% *********************************************************************
\subsubsection{Transformations}
% *********************************************************************

We recommend you first select the {\bf Bounds} button while
\index{graphics!3D control-panel!transformations}
executing transformations since the bounding box displayed
indicates the object's position as it changes.
%
\begin{description}
%
\item[Rotate:]  A rotation transformation occurs by clicking the mouse
\index{graphics!3D control-panel!rotate}
within the {\bf Rotate} window in the upper left corner of the
control-panel.
The rotation is computed in spherical coordinates, using the
horizontal mouse position to increment or decrement the value of
the longitudinal angle $\theta$ within the
range of 0 to 2$\pi$ and the vertical mouse position
to increment or decrement the value of the latitudinal angle
\spad{phi} within the range of -$\pi$
to $\pi$.
The active mode of rotation is displayed in green on a color
monitor or in clear text on a black and white monitor, while the
inactive mode is displayed in red for color display or a mottled
pattern for black and white.
%
\begin{description}
%
\item[origin:]  The {\bf origin} button indicates that the
rotation is to occur with respect to the origin of the viewing space, that is
indicated by the axes.
%
\item[object:]  The {\bf object} button indicates that the
rotation is to occur with respect to the center of volume of the object,
independent of the axes' origin position.
\end{description}
%
\item[Scale:]  A scaling transformation occurs by clicking the mouse
\index{graphics!3D control-panel!scale}
within the {\bf Scale} window in the upper center of the
control-panel, containing a zoom arrow.
The axes along which the scaling is to occur are indicated by
selecting the appropriate button above the zoom arrow window.
The selected axes are displayed in green on a color monitor or in
clear text on a black and white monitor, while the unselected axes
are displayed in red for a color display or a mottled pattern for
black and white.
%
\begin{description}
%
\item[uniform:]  Uniform scaling along the {\tt x}, {\tt y}
and {\tt z} axes occurs when all the axes buttons are selected.
%
\item[non-uniform:]  If any of the axes buttons are
not selected, non-uniform scaling occurs, that is, scaling occurs only in the
direction of the axes that are selected.
\end{description}
%
\item[Translate:]  Translation occurs by indicating with the mouse in the
\index{graphics!3D control-panel!translate}
{\bf Translate} window the direction you want the graph to move.
This window is located in the upper right corner of the
control-panel and contains a potentiometer with crossed arrows
pointing up, down, left and right.
Along the top of the {\bf Translate} window are three buttons
({\bf XY},
{\bf XZ}, and {\bf YZ}) indicating the three orthographic projection planes.
Each orientates the group as a view into that plane.
Any translation of the graph occurs only along this plane.
\end{description}

% *********************************************************************
\subsubsection{Messages}
% *********************************************************************
\index{graphics!3D control-panel!messages}

The window directly below the potentiometer windows for transformations is
used to display system messages relating to the viewport, the control-panel
and the current graph displaying status.

% *********************************************************************
\subsubsection{Colormap}
% *********************************************************************
\index{graphics!3D control-panel!color map}

Directly below the message window is the colormap range indicator
window.
\index{colormap}
The \Language{} Colormap shows a sampling of the spectrum from
which hues can be drawn to represent the colors of a surface.
The Colormap is composed of five shades for each of the hues along
this spectrum.
By moving the markers above and below the Colormap, the range of
hues that are used to color the existing surface are set.
The bottom marker shows the hue for the low end of the color range
and the top marker shows the hue for the upper end of the range.
Setting the bottom and top markers at the same hue results in
monochromatic smooth shading of the graph when {\bf Smooth} mode is selected.
At each end of the Colormap are {\bf +} and {\bf -} buttons.
When clicked on, these increment or decrement the top or bottom
marker.

% *********************************************************************
\subsubsection{Buttons}
% *********************************************************************
\index{graphics!3D control-panel!buttons}

Below the Colormap window and to the left are located various
buttons that determine the characteristics of a graph.
The buttons along the bottom and right hand side all have special
meanings; the remaining buttons in the first row indicate the mode
or style used to display the graph.
The second row are toggles that turn on or off a property of the
graph.
On a color monitor, the property is on if green (clear text, on a
monochrome monitor) and off if red (mottled pattern, on a
monochrome monitor).
Here is a list of their functions.
%
\begin{description}
%
\item[Wire] displays surface and tube plots as a
\index{graphics!3D control-panel!wire}
wireframe image in a single color (blue) with no hidden surfaces removed,
or displays space curve plots in colors based upon their parametric variables.
This is the fastest mode for displaying a graph.
This is very useful when you
want to find a good orientation of your graph.
%
\item[Solid] displays the graph with hidden
\index{graphics!3D control-panel!solid}
surfaces removed, drawing each polygon beginning with the furthest
from the viewer.
The edges of the polygons are displayed in the hues specified by
the range in the Colormap window.
%
\item[Shade] displays the graph with hidden
\index{graphics!3D control-panel!shade}
surfaces removed and with the polygons shaded, drawing each
polygon beginning with the furthest from the viewer.
Polygons are shaded in the hues specified by the range in the
Colormap window using the Phong illumination model.
\index{Phong!illumination model}
%
\item[Smooth] displays the graph using a
\index{graphics!3D control-panel!smooth}
renderer that computes the graph one line at a time.
The location and color of the graph at each visible point on the
screen are determined and displayed using the Phong illumination
\index{Phong!illumination model}
model.
Smooth shading is done in one of two ways, depending on the range
selected in the colormap window and the number of colors available
from the hardware and/or window manager.
When the top and bottom markers of the colormap range are set to
different hues, the graph is rendered by dithering between the
\index{dithering}
transitions in color hue.
When the top and bottom markers of the colormap range are set to
the same hue, the graph is rendered using the Phong smooth shading
model.
\index{Phong!smooth shading model}
However, if enough colors cannot be allocated for this purpose,
the renderer reverts to the color dithering method until a
sufficient color supply is available.
For this reason, it may not be possible to render multiple Phong
smooth shaded graphs at the same time on some systems.
%
\item[Bounds] encloses the entire volume of the
viewgraph within a bounding box, or removes the box if previously selected.
\index{graphics!3D control-panel!bounds}
The region that encloses the entire volume of the viewport graph is displayed.
%
\item[Axes] displays Cartesian
\index{graphics!3D control-panel!axes}
coordinate axes of the space, or turns them off if previously selected.
%
\item[Outline] causes
\index{graphics!3D control-panel!outline}
quadrilateral polygons forming the graph surface to be outlined in black when
the graph is displayed in {\bf Shade} mode.
%
\item[BW] converts a color viewport to black and white, or vice-versa.
\index{graphics!3D control-panel!bw}
When this button is selected the
control-panel and viewport switch to an immutable colormap composed of a range
of grey scale patterns or tiles that are used wherever shading is necessary.
%
\item[Light] takes you to a control-panel described below.
%
\item[ViewVolume] takes you to another control-panel as described below.
\index{graphics!3D control-panel!save}
%
\item[Save] creates a menu of the possible file types that can
be written using the control-panel.
The {\bf Exit} button leaves the save menu.
The {\bf Pixmap} button writes an \Language{} pixmap of
\index{graphics!3D control-panel!pixmap}
the current viewport contents.  The file is called {\bf axiom3D.pixmap} and is
located in the directory from which \Language{} or {\bf viewAlone} was
started.
The {\bf PS} button writes the current viewport contents to
\index{graphics!3D control-panel!ps}
PostScript output rather than to the viewport window.
By default the file is called {\bf axiom3D.ps}; however, if a file
\index{file!.Xdefaults @{\bf .Xdefaults}}
name is specified in the user's {\bf .Xdefaults} file it is
\index{graphics!.Xdefaults!PostScript file name}
used.
The file is placed in the directory from which the \Language{} or
{\bf viewAlone} session was begun.
See also the \spadfunFrom{write}{ThreeDimensionalViewport}
function.
\index{PostScript}
%
\item[Reset] returns the object transformation
\index{graphics!3D control-panel!reset}
characteristics back to their initial states.
%
\item[Hide] causes the control-panel for the
\index{graphics!3D control-panel!hide}
corresponding viewport to disappear from the screen.
%
\item[Quit]  queries whether the current viewport
\index{graphics!3D control-panel!quit}
session should be terminated.
\end{description}

% *********************************************************************
\subsubsection{Light}
% *********************************************************************
\index{graphics!3D control-panel!light}

%
%>>>\begin{texonly}
%
%>>>\begin{figure}[htbp]
%>>>\begin{picture}(183,252)(-125,0)
%>>>\special{psfile=3Dlight.ps}
%>>>\end{picture}
%>>>\caption{Three-Dimensional Lighting Panel.}
%>>>\end{figure}
%>>>\end{texonly}
%

The {\bf Light} button changes the control-panel into the
{\bf Lighting Control-Panel}.  At the top of this panel, the three axes
are shown with the same orientation as the object.  A light vector from
the origin of the axes shows the current position of the light source
relative to the object.  At the bottom of the panel is an {\bf Abort}
button that cancels any changes to the lighting that were made, and a
{\bf Return} button that carries out the current set of lighting changes
on the graph.
%
\begin{description}
%
\item[XY:]  The {\bf XY} lighting axes window is below the
\index{graphics!3D control-panel!move xy}
{\bf Lighting Control-Panel} title and to the left.
This changes the light vector within the {\bf XY} view plane.
%
\item[Z:]  The {\bf Z} lighting axis window is below the
\index{graphics!3D control-panel!move z}
{\bf Lighting Control-Panel} title and in the center.  This
changes the {\bf Z}
location of the light vector.
%
\item[Intensity:]
Below the {\bf Lighting Control-Panel} title
\index{graphics!3D control-panel!intensity}
and to the right is the light intensity meter.
Moving the intensity indicator down decreases the amount of
light emitted from the light source.
When the indicator is at the top of the meter the light source is
emitting at 100\% intensity.
At the bottom of the meter the light source is emitting at a level
slightly above ambient lighting.
\end{description}

% *********************************************************************
\subsubsection{View Volume}
% *********************************************************************
\index{graphics!3D control-panel!view volume}

The {\bf View Volume} button changes the control-panel into
the {\bf Viewing Volume Panel}.
At the bottom of the viewing panel is an {\bf Abort} button that
cancels any changes to the viewing volume that were made and a
{\it Return} button that carries out the current set of
viewing changes to the graph.
%
%>>>\begin{texonly}
%
%>>>\begin{figure}[htbp]
%>>>\begin{picture}(183,252)(-125,0)
%>>>\special{psfile=3Dvolume.ps}
%>>>\end{picture}
%>>>\caption{Three-Dimensional Volume Panel.}
%>>>\end{figure}
%>>>\end{texonly}
%
\begin{description}
%
\item[Eye Reference:]  At the top of this panel is the
\index{graphics!3D control-panel!eye reference}
{\bf Eye Reference} window.
It shows a planar projection of the viewing pyramid from the eye
of the viewer relative to the location of the object.
This has a bounding region represented by the rectangle on the
left.
Below the object rectangle is the {\bf Hither} window.
By moving the slider in this window the hither clipping plane sets
\index{hither clipping plane}
the front of the view volume.
As a result of this depth clipping all points of the object closer
to the eye than this hither plane are not shown.
The {\bf Eye Distance} slider to the right of the {\bf Hither}
slider is used to change the degree of perspective in the image.
%
\item[Clip Volume:]  The {\bf Clip Volume} window is at the
\index{graphics!3D control-panel!clip volume}
bottom of the {\bf Viewing Volume Panel}.
On the right is a {\bf Settings} menu.
In this menu are buttons to select viewing attributes.
Selecting the {\bf Perspective} button computes the image using
perspective projection.
\index{graphics!3D control-panel!perspective}
The {\bf Show Region} button indicates whether the clipping region
of the
\index{graphics!3D control-panel!show clip region}
volume is to be drawn in the viewport and the {\bf Clipping On}
button shows whether the view volume clipping is to be in effect
when the image
\index{graphics!3D control-panel!clipping on}
is drawn.
The left side of the {\bf Clip Volume} window shows the clipping
\index{graphics!3D control-panel!clip volume}
boundary of the graph.
Moving the knobs along the {\bf X}, {\bf Y}, and {\bf Z} sliders
adjusts the volume of the clipping region accordingly.
\end{description}

% *********************************************************************
\head{subsection}{Operations for Three-Dimensional Graphics}{ugGraphThreeDops}
% *********************************************************************

Here is a summary of useful \Language{} operations for \threedim{}
graphics.
Each operation name is followed by a list of arguments.
Each argument is written as a variable informally named according
to the type of the argument (for example, {\it integer}).
If appropriate, a default value for an argument is given in
parentheses immediately following the name.

%
\bgroup\hbadness = 10001\sloppy
\begin{description}
%
\item[\spadfun{adaptive3D?}]\funArgs{}
tests whether space curves are to be plotted
\index{graphics!plot3d defaults!adaptive}
according to the
\index{adaptive plotting}
adaptive refinement algorithm.

%
\item[\spadfun{axes}]\funArgs{viewport, string\argDef{"on"}}
turns the axes on and off.
\index{graphics!3D commands!axes}

%
\item[\spadfun{close}]\funArgs{viewport}
closes the viewport.
\index{graphics!3D commands!close}

%
\item[\spadfun{colorDef}]\funArgs{viewport,
\subscriptIt{color}{1}\argDef{1}, \subscriptIt{color}{2}\argDef{27}}
sets the colormap
\index{graphics!3D commands!define color}
range to be from
\subscriptIt{color}{1} to \subscriptIt{color}{2}.

%
\item[\spadfun{controlPanel}]\funArgs{viewport, string\argDef{"off"}}
declares whether the
\index{graphics!3D commands!control-panel}
control-panel for the viewport is to be displayed or not.

%
\item[\spadfun{diagonals}]\funArgs{viewport, string\argDef{"off"}}
declares whether the
\index{graphics!3D commands!diagonals}
polygon outline includes the diagonals or not.

%
\item[\spadfun{drawStyle}]\funArgs{viewport, style}
selects which of four drawing styles
\index{graphics!3D commands!drawing style}
are used: {\tt "wireMesh", "solid", "shade",} or {\tt "smooth".}

%
\item[\spadfun{eyeDistance}]\funArgs{viewport,float\argDef{500}}
sets the distance of the eye from the origin of the object
\index{graphics!3D commands!eye distance}
for use in the \spadfunFrom{perspective}{ThreeDimensionalViewport}.

%
\item[\spadfun{key}]\funArgs{viewport}
returns the operating
\index{graphics!3D commands!key}
system process ID number for the viewport.

%
\item[\spadfun{lighting}]\funArgs{viewport,
\subscriptText{float}{x}\argDef{-0.5},
\subscriptText{float}{y}\argDef{0.5}, \subscriptText{float}{z}\argDef{0.5}}
sets the Cartesian
\index{graphics!3D commands!lighting}
coordinates of the light source.

%
\item[\spadfun{modifyPointData}]\funArgs{viewport,integer,point}
replaces the coordinates of the point with
\index{graphics!3D commands!modify point data}
the index {\it integer} with {\it point}.

%
\item[\spadfun{move}]\funArgs{viewport,
\subscriptText{integer}{x}\argDef{viewPosDefault},
\subscriptText{integer}{y}\argDef{viewPosDefault}}
moves the upper
\index{graphics!3D commands!move}
left-hand corner of the viewport to screen position
\allowbreak
({\small \subscriptText{integer}{x}, \subscriptText{integer}{y}}).

%
\item[\spadfun{options}]\funArgs{viewport}
returns a list of all current draw options.

%
\item[\spadfun{outlineRender}]\funArgs{viewport, string\argDef{"off"}}
turns polygon outlining
\index{graphics!3D commands!outline}
off or on when drawing in {\tt "shade"} mode.

%
\item[\spadfun{perspective}]\funArgs{viewport, string\argDef{"on"}}
turns perspective
\index{graphics!3D commands!perspective}
viewing on and off.

%
\item[\spadfun{reset}]\funArgs{viewport}
resets the attributes of a viewport to their
\index{graphics!3D commands!reset}
initial settings.

%
\item[\spadfun{resize}]\funArgs{viewport,
\subscriptText{integer}{width} \argDef{viewSizeDefault},
\subscriptText{integer}{height} \argDef{viewSizeDefault}}
resets the width and height
\index{graphics!3D commands!resize}
values for a viewport.

%
\item[\spadfun{rotate}]\funArgs{viewport,
\subscriptText{number}{$\theta$}\argDef{viewThetaDefault},
\subscriptText{number}{$\varphi$}\argDef{viewPhiDefault}}
rotates the viewport by rotation angles for longitude
({\it $\theta$}) and
latitude ({\it $\varphi$}).
Angles designate radians if given as floats, or degrees if given
\index{graphics!3D commands!rotate}
as integers.

%
\item[\spadfun{setAdaptive3D}]\funArgs{boolean\argDef{true}}
sets whether space curves are to be plotted
\index{graphics!plot3d defaults!set adaptive}
according to the adaptive
\index{adaptive plotting}
refinement algorithm.

%
\item[\spadfun{setMaxPoints3D}]\funArgs{integer\argDef{1000}}
 sets the default maximum number of possible
\index{graphics!plot3d defaults!set max points}
points to be used when constructing a \threedim{} space curve.

%
\item[\spadfun{setMinPoints3D}]\funArgs{integer\argDef{49}}
sets the default minimum number of possible
\index{graphics!plot3d defaults!set min points}
points to be used when constructing a \threedim{} space curve.

%
\item[\spadfun{setScreenResolution3D}]\funArgs{integer\argDef{500}}
sets the default screen resolution constant
\index{graphics!plot3d defaults!set screen resolution}
used in setting the computation limit of adaptively
\index{adaptive plotting}
generated \threedim{} space curve plots.

%
\item[\spadfun{showRegion}]\funArgs{viewport, string\argDef{"off"}}
declares whether the bounding
\index{graphics!3D commands!showRegion}
box of a graph is shown or not.
%
\item[\spadfun{subspace}]\funArgs{viewport}
returns the space component.
%
\item[\spadfun{subspace}]\funArgs{viewport, subspace}
resets the space component
\index{graphics!3D commands!subspace}
to {\it subspace}.

%
\item[\spadfun{title}]\funArgs{viewport, string}
gives the viewport the
\index{graphics!3D commands!title}
title {\it string}.

%
\item[\spadfun{translate}]\funArgs{viewport,
\subscriptText{float}{x}\argDef{viewDeltaXDefault},
\subscriptText{float}{y}\argDef{viewDeltaYDefault}}
translates
\index{graphics!3D commands!translate}
the object horizontally and vertically relative to the center of the viewport.

%
\item[\spadfun{intensity}]\funArgs{viewport,float\argDef{1.0}}
resets the intensity {\it I} of the light source,
\index{graphics!3D commands!intensity}
$0 \le I \le 1.$

%
\item[\spadfun{tubePointsDefault}]\funArgs{\optArg{integer\argDef{6}}}
sets or indicates the default number of
\index{graphics!3D defaults!tube points}
vertices defining the polygon that is used to create a tube around
a space curve.

%
\item[\spadfun{tubeRadiusDefault}]\funArgs{\optArg{float\argDef{0.5}}}
sets or indicates the default radius of
\index{graphics!3D defaults!tube radius}
the tube that encircles a space curve.

%
\item[\spadfun{var1StepsDefault}]\funArgs{\optArg{integer\argDef{27}}}
sets or indicates the default number of
\index{graphics!3D defaults!var1 steps}
increments into which the grid defining a surface plot is subdivided with
respect to the first parameter declared in the surface function.

%
\item[\spadfun{var2StepsDefault}]\funArgs{\optArg{integer\argDef{27}}}
sets or indicates the default number of
\index{graphics!3D defaults!var2 steps}
increments into which the grid defining a surface plot is subdivided with
respect to the second parameter declared in the surface function.

%
\item[\spadfun{viewDefaults}]\funArgs{{\tt [}\subscriptText{integer}{%
point}, \subscriptText{integer}{line}, \subscriptText{integer}{axes},
\subscriptText{integer}{units}, \subscriptText{float}{point},
\allowbreak\subscriptText{list}{position},
\subscriptText{list}{size}{\tt ]}}
resets the default settings for the
\index{graphics!3D defaults!reset viewport defaults}
point color, line color, axes color, units color, point size,
viewport upper left-hand corner position, and the viewport size.

%
\item[\spadfun{viewDeltaXDefault}]\funArgs{\optArg{float\argDef{0}}}
resets the default horizontal offset
\index{graphics!3D commands!deltaX default}
from the center of the viewport, or returns the current default offset if no argument is given.

%
\item[\spadfun{viewDeltaYDefault}]\funArgs{\optArg{float\argDef{0}}}
resets the default vertical offset
\index{graphics!3D commands!deltaY default}
from the center of the viewport, or returns the current default offset if no argument is given.

%
\item[\spadfun{viewPhiDefault}]\funArgs{\optArg{float\argDef{-$\pi$/4}}}
resets the default latitudinal view angle,
or returns the current default angle if no argument is given.
\index{graphics!3D commands!phi default}
$\varphi$ is set to this value.

%
\item[\spadfun{viewpoint}]\funArgs{viewport, \subscriptText{float}{x},
\subscriptText{float}{y}, \subscriptText{float}{z}}
sets the viewing position in Cartesian coordinates.

%
\item[\spadfun{viewpoint}]\funArgs{viewport,
\subscriptText{float}{$\theta$},
\subscriptText{Float}{$\varphi$}}
sets the viewing position in spherical coordinates.

%
\item[\spadfun{viewpoint}]\funArgs{viewport,
\subscriptText{Float}{$\theta$},
\subscriptText{Float}{$\varphi$},
\subscriptText{Float}{scaleFactor},
\subscriptText{Float}{xOffset}, \subscriptText{Float}{yOffset}}
sets the viewing position in spherical coordinates,
the scale factor, and offsets.
\index{graphics!3D commands!viewpoint}
$\theta$ (longitude) and
$\varphi$ (latitude) are in radians.

%
\item[\spadfun{viewPosDefault}]\funArgs{\optArg{list\argDef{[0,0]}}}
sets or indicates the position of the upper
\index{graphics!3D defaults!viewport position}
left-hand corner of a \twodim{} viewport, relative to the display root
window (the upper left-hand corner of the display is \spad{[0, 0]}).

%
\item[\spadfun{viewSizeDefault}]\funArgs{\optArg{list\argDef{[400,400]}}}
sets or indicates the width and height dimensions
\index{graphics!3D defaults!viewport size}
of a viewport.

%
\item[\spadfun{viewThetaDefault}]\funArgs{\optArg{float\argDef{$\pi$/4}}}
resets the default longitudinal view angle,
or returns the current default angle if no argument is given.
\index{graphics!3D commands!theta default}
When a parameter is specified, the default longitudinal view angle
$\theta$ is set to this value.

%
\item[\spadfun{viewWriteAvailable}]\funArgs{\optArg{list\argDef{["pixmap",
"bitmap", "postscript", "image"}}}
indicates the possible file types
\index{graphics!3D defaults!available viewport writes}
that can be created with the \spadfunFrom{write}{ThreeDimensionalViewport} function.

%
\item[\spadfun{viewWriteDefault}]\funArgs{\optArg{list\argDef{[]}}}
sets or indicates the default types of files
that are created in addition to the {\bf data} file when a
\spadfunFrom{write}{ThreeDimensionalViewport} command
\index{graphics!3D defaults!viewport writes}
is executed on a viewport.

%
\item[\spadfun{viewScaleDefault}]\funArgs{\optArg{float}}
sets the default scaling factor, or returns
\index{graphics!3D commands!scale default}
the current factor if no argument is given.

%
\item[\spadfun{write}]\funArgs{viewport, directory, \optArg{option}}
writes the file {\bf data} for {\it viewport}
in the directory {\it directory}.
An optional third argument specifies a file type (one of {\tt
pixmap}, {\tt bitmap}, {\tt postscript}, or {\tt image}), or a
list of file types.
An additional file is written for each file type listed.

%
\item[\spadfun{scale}]\funArgs{viewport, float\argDef{2.5}}
specifies the scaling factor.
\index{graphics!3D commands!scale}
\index{scaling graphs}
\end{description}
\egroup

% *********************************************************************
\head{subsection}{Customization using .Xdefaults}{ugXdefaults}
% *********************************************************************
\index{graphics!.Xdefaults}

Both the \twodim{} and \threedim{} drawing facilities consult
the {\bf .Xdefaults} file for various defaults.
\index{file!.Xdefaults @{\bf .Xdefaults}}
The list of defaults that are recognized by the graphing routines
is discussed in this section.
These defaults are preceded by {\tt Axiom.3D.}
for \threedim{} viewport defaults, {\tt Axiom.2D.}
for \twodim{} viewport defaults, or {\tt Axiom*} (no dot) for
those defaults that are acceptable to either viewport type.

%
\begin{description}
%
\item[{\tt Axiom*buttonFont:\ \it font}] \ \newline
This indicates which
\index{graphics!.Xdefaults!button font}
font type is used for the button text on the control-panel.
\xdefault{Rom11}
%
\item[{\tt Axiom.2D.graphFont:\ \it font}] \quad (2D only) \newline
This indicates
\index{graphics!.Xdefaults!graph number font}
which font type is used for displaying the graph numbers and
slots in the {\bf Graphs} section of the \twodim{} control-panel.
\xdefault{Rom22}
%
\item[{\tt Axiom.3D.headerFont:\ \it font}] \ \newline
This indicates which
\index{graphics!.Xdefaults!graph label font}
font type is used for the axes labels and potentiometer
header names on \threedim{} viewport windows.
This is also used for \twodim{} control-panels for indicating
which font type is used for potentionmeter header names and
multiple graph title headers.
%for example, {\tt Axiom.2D.headerFont: 8x13}.
\xdefault{Itl14}
%
\item[{\tt Axiom*inverse:\ \it switch}] \ \newline
This indicates whether the
\index{graphics!.Xdefaults!inverting background}
background color is to be inverted from white to black.
If {\tt on}, the graph viewports use black as the background
color.
If {\tt off} or no declaration is made, the graph viewports use a
white background.
\xdefault{off}
%
\item[{\tt Axiom.3D.lightingFont:\ \it font}] \quad (3D only) \newline
This indicates which font type is used for the {\bf x},
\index{graphics!.Xdefaults!lighting font}
{\bf y}, and {\bf z} labels of the two lighting axes potentiometers, and for
the {\bf Intensity} title on the lighting control-panel.
\xdefault{Rom10}
%
\item[{\tt Axiom.2D.messageFont, Axiom.3D.messageFont:\ \it font}] \ \newline
These indicate the font type
\index{graphics!.Xdefaults!message font}
to be used for the text in the control-panel message window.
\xdefault{Rom14}
%
\item[{\tt Axiom*monochrome:\ \it switch}] \ \newline
This indicates whether the
\index{graphics!.Xdefaults!monochrome}
graph viewports are to be displayed as if the monitor is black and
white, that is, a 1 bit plane.
If {\tt on} is specified, the viewport display is black and white.
If {\tt off} is specified, or no declaration for this default is
given, the viewports are displayed in the normal fashion for the
monitor in use.
\xdefault{off}
%
\item[{\tt Axiom.2D.postScript:\ \it filename}] \ \newline
This specifies
\index{graphics!.Xdefaults!PostScript file name}
the name of the file that is generated when a 2D PostScript graph
\index{PostScript}
is saved.
\xdefault{axiom2D.ps}
%
\item[{\tt Axiom.3D.postScript:\ \it filename}] \ \newline
This specifies
\index{graphics!.Xdefaults!PostScript file name}
the name of the file that is generated when a 3D PostScript graph
\index{PostScript}
is saved.
\xdefault{axiom3D.ps}
%
\item[{\tt Axiom*titleFont \it font}] \ \newline
This
\index{graphics!.Xdefaults!title font}
indicates which font type is used
for the title text and, for \threedim{} graphs,
in the lighting and viewing-volume control-panel windows.
\index{graphics!Xdefaults!2d}
\xdefault{Rom14}
%
\item[{\tt Axiom.2D.unitFont:\ \it font}] \quad (2D only) \newline
This indicates
\index{graphics!.Xdefaults!unit label font}
which font type is used for displaying the unit labels on
\twodim{} viewport graphs.
\xdefault{6x10}
%
\item[{\tt Axiom.3D.volumeFont:\ \it font}] \quad (3D only) \newline
This indicates which font type is used for the {\bf x},
\index{graphics!.Xdefaults!volume label font}
{\bf y}, and {\bf z} labels of the clipping region sliders; for the
{\bf Perspective}, {\bf Show Region}, and {\bf Clipping On} buttons under
{\bf Settings}, and above the windows for the {\bf Hither} and
{\bf Eye Distance} sliders in the {\bf Viewing Volume Panel} of the
\threedim{} control-panel.
\xdefault{Rom8}
\end{description}

\endgroup

\part{Advanced Problem Solving and Examples}
%
% !! DO NOT MODIFY THIS FILE BY HAND !! Created by spool2tex.awk.

% Copyright (c) 1991-2002, The Numerical ALgorithms Group Ltd.
% All rights reserved.
%
% Redistribution and use in source and binary forms, with or without
% modification, are permitted provided that the following conditions are
% met:
%
%     - Redistributions of source code must retain the above copyright
%       notice, this list of conditions and the following disclaimer.
%
%     - Redistributions in binary form must reproduce the above copyright
%       notice, this list of conditions and the following disclaimer in
%       the documentation and/or other materials provided with the
%       distribution.
%
%     - Neither the name of The Numerical ALgorithms Group Ltd. nor the
%       names of its contributors may be used to endorse or promote products
%       derived from this software without specific prior written permission.
%
% THIS SOFTWARE IS PROVIDED BY THE COPYRIGHT HOLDERS AND CONTRIBUTORS "AS
% IS" AND ANY EXPRESS OR IMPLIED WARRANTIES, INCLUDING, BUT NOT LIMITED
% TO, THE IMPLIED WARRANTIES OF MERCHANTABILITY AND FITNESS FOR A
% PARTICULAR PURPOSE ARE DISCLAIMED. IN NO EVENT SHALL THE COPYRIGHT OWNER
% OR CONTRIBUTORS BE LIABLE FOR ANY DIRECT, INDIRECT, INCIDENTAL, SPECIAL,
% EXEMPLARY, OR CONSEQUENTIAL DAMAGES (INCLUDING, BUT NOT LIMITED TO,
% PROCUREMENT OF SUBSTITUTE GOODS OR SERVICES-- LOSS OF USE, DATA, OR
% PROFITS-- OR BUSINESS INTERRUPTION) HOWEVER CAUSED AND ON ANY THEORY OF
% LIABILITY, WHETHER IN CONTRACT, STRICT LIABILITY, OR TORT (INCLUDING
% NEGLIGENCE OR OTHERWISE) ARISING IN ANY WAY OUT OF THE USE OF THIS
% SOFTWARE, EVEN IF ADVISED OF THE POSSIBILITY OF SUCH DAMAGE.

% *********************************************************************
\head{chapter}{Advanced Problem Solving}{ugProblem}
% *********************************************************************

In this chapter we describe techniques useful in solving advanced problems
with \Language{}.

% *********************************************************************
\head{section}{Numeric Functions}{ugProblemNumeric}
% *********************************************************************
%
\Language{} provides two basic floating-point types: \spadtype{Float} and
\spadtype{DoubleFloat}.  This section describes how to use numerical
\index{function!numeric}
operations defined on these types and the related complex types.
\index{numeric operations}
%
As we mentioned in \chapref{ugIntro}, the \spadtype{Float} type is a software
implementation of floating-point numbers in which the exponent and the
\index{floating-point number}
significand may have any number of digits.
\index{number!floating-point}
See \xmpref{Float} for detailed information about this domain.
The \spadtype{DoubleFloat} (see \xmpref{DoubleFloat}) is usually a hardware
implementation of floating point numbers, corresponding to machine double
precision.
The types \spadtype{Complex Float} and \spadtype{Complex DoubleFloat} are
\index{floating-point number!complex}
the corresponding software implementations of complex floating-point numbers.
\index{complex!floating-point number}
In this section the term {\it floating-point type}  means any of these
\index{number!complex floating-point}
four types.
%
The floating-point types implement the basic elementary functions.
These include (where \spadSyntax{%} means
\spadtype{DoubleFloat},
\spadtype{Float},
\spadtype{Complex DoubleFloat}, or
\spadtype{Complex Float}):

\noindent
\spadfun{exp},  \spadfun{log}: \spad \newline
\spadfun{sin},  \spadfun{cos}, \spadfun{tan}, \spadfun{cot}, \spadfun{sec}, \spadfun{csc}: \spad \newline
\spadfun{asin}, \spadfun{acos}, \spadfun{atan}, \spadfun{acot}, \spadfun{asec}, \spadfun{acsc}: \spad  \newline
\spadfun{sinh},  \spadfun{cosh}, \spadfun{tanh}, \spadfun{coth}, \spadfun{sech}, \spadfun{csch}: \spad  \newline
\spadfun{asinh}, \spadfun{acosh}, \spadfun{atanh}, \spadfun{acoth}, \spadfun{asech}, \spadfun{acsch}: \spad  \newline
\spadfun{pi}: \spad{() -> %}  \newline
\spadfun{sqrt}: \spad \newline
\spadfun{nthRoot}: \spad{(%, Integer) -> %}  \newline
\spadopFrom{^}{Float}: \spad{(%, Fraction Integer) -> %} \newline
\spadopFrom{^}{Float}: \spad{(%, %) -> %}  \newline

The handling of roots depends on whether the floating-point type
\index{root!numeric approximation}
is real or complex: for the real floating-point types,
\spadtype{DoubleFloat} and \spadtype{Float}, if a real root exists
the one with the same sign as the radicand is returned; for the
complex floating-point types, the principal value is returned.
\index{principal value}
Also, for real floating-point types the inverse functions
produce errors if the results are not real.
This includes cases such as \spad{asin(1.2)}, \spad{log(-3.2)},
\spad{sqrt(-1.1)}.
%
\begin{xtc}
\begin{xtccomment}
The default floating-point type is \spadtype{Float} so to evaluate
functions using \spadtype{Float} or \spadtype{Complex Float}, just use
normal decimal notation.
\end{xtccomment}
\begin{spadsrc}
exp(3.1)
\end{spadsrc}
\begin{TeXOutput}
\begin{fricasmath}{1}
\STRING{22.197951281441633405}%
\end{fricasmath}
\end{TeXOutput}
\formatResultType{Float}
\end{xtc}
\begin{xtc}
\begin{xtccomment}
\end{xtccomment}
\begin{spadsrc}
exp(3.1 + 4.5 * %i)
\end{spadsrc}
\begin{TeXOutput}
\begin{fricasmath}{2}
-{\STRING{4.6792348860969899118}}-{\STRING{21.699165928071731864}\TIMES %
\ImaginaryI }%
\end{fricasmath}
\end{TeXOutput}
\formatResultType{Complex(Float)}
\end{xtc}
\begin{xtc}
\begin{xtccomment}
To evaluate functions using \spadtype{DoubleFloat}
or \spadtype{Complex DoubleFloat},
a declaration or conversion is required.
\end{xtccomment}
\begin{spadsrc}
r: DFLOAT := 3.1; t: DFLOAT := 4.5; exp(r + t*%i)
\end{spadsrc}
\begin{TeXOutput}
\begin{fricasmath}{3}
-{\STRING{4.679234886096988}}-{\STRING{21.69916592807172}\TIMES \ImaginaryI }%
\end{fricasmath}
\end{TeXOutput}
\formatResultType{Complex(DoubleFloat)}
\end{xtc}
\begin{xtc}
\begin{xtccomment}
\end{xtccomment}
\begin{spadsrc}
exp(3.1::DFLOAT + 4.5::DFLOAT * %i)
\end{spadsrc}
\begin{TeXOutput}
\begin{fricasmath}{4}
-{\STRING{4.679234886096988}}-{\STRING{21.69916592807172}\TIMES \ImaginaryI }%
\end{fricasmath}
\end{TeXOutput}
\formatResultType{Complex(DoubleFloat)}
\end{xtc}
%
A number of special functions are provided by the package
\spadtype{DoubleFloatSpecialFunctions} for the machine-precision
\index{special functions}
floating-point types.
\exptypeindex{DoubleFloatSpecialFunctions}
The special functions provided are listed below, where \spad{F} stands for
the types \spadtype{DoubleFloat} and \spadtype{Complex DoubleFloat}.
The real versions of the functions yield an error if the result is not real.
\index{function!special}

\noindent
\spadfun{Gamma}: \spad{F -> F}\hfill\newline
\spad{Gamma(z)} is the Euler gamma function,
\index{function!Gamma}
   $\Gamma(z)$,
   defined by
\index{Euler!gamma function}
\begin{displaymath}
\Gamma(z) = \int_{0}^{\infty} t^{z-1} e^{-t} dt.
\end{displaymath}

\noindent
\spadfun{Beta}: \spad{F -> F}\hfill\newline
   \spad{Beta(u, v)} is the Euler Beta function,
\index{function!Euler Beta}
   \mathOrSpad{B(u,v)}, defined by
\index{Euler!Beta function}
\begin{displaymath}
B(u,v) = \int_{0}^{1} t^{u-1} (1-t)^{v-1} dt.
\end{displaymath}
   This is related to $\Gamma(z)$ by
\begin{displaymath}
B(u,v) = \frac{\Gamma(u) \Gamma(v)}{\Gamma(u + v)}.
\end{displaymath}

\noindent
\spadfun{logGamma}: \spad{F -> F}\hfill\newline
   \spad{logGamma(z)} is the natural logarithm of
$\Gamma(z)$.
   This can often be computed even if $\Gamma(z)$
cannot.
%

\noindent
\spadfun{digamma}: \spad{F -> F}\hfill\newline
   \spad{digamma(z)}, also called \spad{psi(z)},
\index{psi @ $\psi$}
is the function $\psi(z)$,
\index{function!digamma}
   defined by
\begin{displaymath}
\psi(z) = \Gamma'(z)/\Gamma(z).
\end{displaymath}

\noindent
\spadfun{polygamma}: \spad{(NonNegativeInteger, F) -> F}\hfill\newline
   \spad{polygamma(n, z)} is the \eth{n} derivative of
\index{function!polygamma}
   $\psi(z)$, written $\psi^{(n)}(z)$.

\noindent
\spadfun{besselJ}: \spad{(F,F) -> F}\hfill\newline
   \spad{besselJ(v,z)} is the Bessel function of the first kind,
\index{function!Bessel}
   $J_\nu (z)$.
   This function satisfies the differential equation
\begin{displaymath}
z^2 w''(z) + z w'(z) + (z^2-\nu^2)w(z) = 0.
\end{displaymath}

\noindent
\spadfun{besselY}: \spad{(F,F) -> F}\hfill\newline
   \spad{besselY(v,z)} is the Bessel function of the second kind,
\index{function!Bessel}
   $Y_\nu (z)$.
   This function satisfies the same differential equation as
   \spadfun{besselJ}.
   The implementation simply uses the relation
\begin{displaymath}
Y_\nu (z) = \frac{J_\nu (z) \cos(\nu \pi) - J_{-\nu} (z)}{\sin(\nu \pi)}.
\end{displaymath}

\noindent
\spadfun{besselI}: \spad{(F,F) -> F}\hfill\newline
   \spad{besselI(v,z)} is the modified Bessel function of the first kind,
\index{function!Bessel}
   $I_\nu (z)$.
   This function satisfies the differential equation
\begin{displaymath}
z^2 w''(z) + z w'(z) - (z^2+\nu^2)w(z) = 0.
\end{displaymath}

\noindent
\spadfun{besselK}: \spad{(F,F) -> F}\hfill\newline
   \spad{besselK(v,z)} is the modified Bessel function of the second kind,
\index{function!Bessel}
   $K_\nu (z)$.
   This function satisfies the same differential equation as \spadfun{besselI}.
\index{Bessel function}
   The implementation simply uses the relation
\begin{displaymath}
K_\nu (z) = \pi \frac{I_{-\nu} (z) - I_{\nu} (z)}{2 \sin(\nu \pi)}.
\end{displaymath}

\noindent
\spadfun{airyAi}: \spad{F -> F}\hfill\newline
   \spad{airyAi(z)} is the Airy function $Ai(z)$.
\index{function!Airy Ai}
   This function satisfies the differential equation
   $w''(z) - z w(z) = 0.$
   The implementation simply uses the relation
\begin{displaymath}
Ai(-z) = \frac{1}{3}\sqrt{z} ( J_{-1/3} (\frac{2}{3}z^{3/2}) + J_{1/3} (\frac{2}{3}z^{3/2}) ).
\end{displaymath}

\noindent
\spadfun{airyBi}: \spad{F -> F}\hfill\newline
   \spad{airyBi(z)} is the Airy function $Bi(z)$.
\index{function!Airy Bi}
   This function satisfies the same differential equation as \spadfun{airyAi}.
\index{Airy function}
   The implementation simply uses the relation
\begin{displaymath}
Bi(-z) = \frac{1}{3}\sqrt{3 z} ( J_{-1/3} (\frac{2}{3}z^{3/2}) - J_{1/3} (\frac{2}{3}z^{3/2}) ).
\end{displaymath}

\noindent
\spadfun{hypergeometric0F1}: \spad{(F,F) -> F}\hfill\newline
   \spad{hypergeometric0F1(c,z)} is the hypergeometric function
\index{function!hypergeometric}
   ${}_0 F_1 ( ; c; z)$.%

\begin{xtc}
\begin{xtccomment}
The above special functions are defined only for small floating-point types.
If you give \spadtype{Float} arguments, they are converted to
\spadtype{DoubleFloat} by \Language{}.
\end{xtccomment}
\begin{spadsrc}
Gamma(0.5)^2
\end{spadsrc}
\begin{TeXOutput}
\begin{fricasmath}{5}
\STRING{3.1415926535897932385}%
\end{fricasmath}
\end{TeXOutput}
\formatResultType{Float}
\end{xtc}
\begin{xtc}
\begin{xtccomment}
\end{xtccomment}
\begin{spadsrc}
a := 2.1; b := 1.1; besselI(a + %i*b, b*a + 1)
\end{spadsrc}
\begin{TeXOutput}
\begin{fricasmath}{6}
\STRING{2.4894824175473698}-{\STRING{2.365846038146814}\TIMES \ImaginaryI }%
\end{fricasmath}
\end{TeXOutput}
\formatResultType{Complex(DoubleFloat)}
\end{xtc}
%
A number of additional operations may be used to compute numerical values.
These are special polynomial functions that can be evaluated for values in
any commutative ring \spad{R}, and in particular for values in any
floating-point type.
The following operations are provided by the package
\spadtype{OrthogonalPolynomialFunctions}:
\exptypeindex{OrthogonalPolynomialFunctions}

\noindent
\spadfun{chebyshevT}: \spad{(NonNegativeInteger, R) -> R}\hbox{}\hfill\newline
   \spad{chebyshevT(n,z)} is the \eth{n} Chebyshev polynomial of the first
   kind, $T_n (z)$.  These are defined by
\begin{displaymath}
\frac{1-t z}{1-2 t z+t^2} = \sum_{n=0}^{\infty} T_n (z) t^n.
\end{displaymath}

\noindent
\spadfun{chebyshevU}: \spad{(NonNegativeInteger, R) -> R}\hbox{}\hfill\newline
   \spad{chebyshevU(n,z)} is the \eth{n} Chebyshev polynomial of the second
   kind, $U_n (z)$. These are defined by
\begin{displaymath}
\frac{1}{1-2 t z+t^2} = \sum_{n=0}^{\infty} U_n (z) t^n.
\end{displaymath}

\noindent
\spadfun{hermiteH}:   \spad{(NonNegativeInteger, R) -> R}\hbox{}\hfill\newline
   \spad{hermiteH(n,z)} is the \eth{n} Hermite polynomial,
   $H_n (z)$.
   These are defined by
\begin{displaymath}
e^{2 t z - t^2} = \sum_{n=0}^{\infty} H_n (z) \frac{t^n}{n!}.
\end{displaymath}

\noindent
\spadfun{laguerreL}:  \spad{(NonNegativeInteger, R) -> R}\hbox{}\hfill\newline
   \spad{laguerreL(n,z)} is the \eth{n} Laguerre polynomial,
   $L_n (z)$.
   These are defined by
\begin{displaymath}
\frac{e^{-\frac{t z}{1-t}}}{1-t} = \sum_{n=0}^{\infty} L_n (z) \frac{t^n}{n!}.
\end{displaymath}

\noindent
\spadfun{laguerreL}:  \spad{(NonNegativeInteger, NonNegativeInteger, R) -> R}\hbox{}\hfill\newline
   \spad{laguerreL(m,n,z)} is the associated Laguerre polynomial,
   $L^m_n (z)$.
   This is the \eth{m} derivative of $L_n (z)$.

\noindent
\spadfun{legendreP}:  \spad{(NonNegativeInteger, R) -> R}\hbox{}\hfill\newline
   \spad{legendreP(n,z)} is the \eth{n} Legendre polynomial,
   $P_n (z)$.  These are defined by
\begin{displaymath}
\frac{1}{\sqrt{1-2 t z+t^2}} = \sum_{n=0}^{\infty} P_n (z) t^n.
\end{displaymath}

%
\begin{xtc}
\begin{xtccomment}
These operations require non-negative integers for the indices, but otherwise
the argument can be given as desired.
\end{xtccomment}
\begin{spadsrc}
[chebyshevT(i, z) for i in 0..5]
\end{spadsrc}
\begin{TeXOutput}
\begin{fricasmath}{7}
\BRACKET{1\COMMA \SYMBOL{z}\COMMA 2\TIMES \SUPER{\SYMBOL{z}}{2}-{1}\COMMA 4%
\TIMES \SUPER{\SYMBOL{z}}{3}-{3\TIMES \SYMBOL{z}}\COMMA 8\TIMES \SUPER{%
\SYMBOL{z}}{4}-{8\TIMES \SUPER{\SYMBOL{z}}{2}}+1\COMMA 16\TIMES \SUPER{%
\SYMBOL{z}}{5}-{20\TIMES \SUPER{\SYMBOL{z}}{3}}+5\TIMES \SYMBOL{z}}%
\end{fricasmath}
\end{TeXOutput}
\formatResultType{List(Polynomial(Integer))}
\end{xtc}
\begin{xtc}
\begin{xtccomment}
The expression \spad{chebyshevT(n,z)} evaluates to the \eth{n} Chebyshev
\index{polynomial!Chebyshev!of the first kind}
polynomial of the first kind.
\end{xtccomment}
\begin{spadsrc}
chebyshevT(3, 5.0 + 6.0*%i)
\end{spadsrc}
\begin{TeXOutput}
\begin{fricasmath}{8}
-{\STRING{1675.0}}+\STRING{918.0}\TIMES \ImaginaryI %
\end{fricasmath}
\end{TeXOutput}
\formatResultType{Complex(Float)}
\end{xtc}
\begin{xtc}
\begin{xtccomment}
\end{xtccomment}
\begin{spadsrc}
chebyshevT(3, 5.0::DoubleFloat)
\end{spadsrc}
\begin{TeXOutput}
\begin{fricasmath}{9}
\STRING{485.0}%
\end{fricasmath}
\end{TeXOutput}
\formatResultType{DoubleFloat}
\end{xtc}
\begin{xtc}
\begin{xtccomment}
The expression \spad{chebyshevU(n,z)} evaluates to the \eth{n} Chebyshev
\index{polynomial!Chebyshev!of the second kind}
polynomial of the second kind.
\end{xtccomment}
\begin{spadsrc}
[chebyshevU(i, z) for i in 0..5]
\end{spadsrc}
\begin{TeXOutput}
\begin{fricasmath}{10}
\BRACKET{1\COMMA 2\TIMES \SYMBOL{z}\COMMA 4\TIMES \SUPER{\SYMBOL{z}}{2}-{1}%
\COMMA 8\TIMES \SUPER{\SYMBOL{z}}{3}-{4\TIMES \SYMBOL{z}}\COMMA 16\TIMES %
\SUPER{\SYMBOL{z}}{4}-{12\TIMES \SUPER{\SYMBOL{z}}{2}}+1\COMMA 32\TIMES %
\SUPER{\SYMBOL{z}}{5}-{32\TIMES \SUPER{\SYMBOL{z}}{3}}+6\TIMES \SYMBOL{z}}%
\end{fricasmath}
\end{TeXOutput}
\formatResultType{List(Polynomial(Integer))}
\end{xtc}
\begin{xtc}
\begin{xtccomment}
\end{xtccomment}
\begin{spadsrc}
chebyshevU(3, 0.2)
\end{spadsrc}
\begin{TeXOutput}
\begin{fricasmath}{11}
-{\STRING{0.736}}%
\end{fricasmath}
\end{TeXOutput}
\formatResultType{Float}
\end{xtc}
\begin{xtc}
\begin{xtccomment}
The expression \spad{hermiteH(n,z)} evaluates to the \eth{n} Hermite
\index{polynomial!Hermite}
polynomial.
\end{xtccomment}
\begin{spadsrc}
[hermiteH(i, z) for i in 0..5]
\end{spadsrc}
\begin{TeXOutput}
\begin{fricasmath}{12}
\BRACKET{1\COMMA 2\TIMES \SYMBOL{z}\COMMA 4\TIMES \SUPER{\SYMBOL{z}}{2}-{2}%
\COMMA 8\TIMES \SUPER{\SYMBOL{z}}{3}-{12\TIMES \SYMBOL{z}}\COMMA 16\TIMES %
\SUPER{\SYMBOL{z}}{4}-{48\TIMES \SUPER{\SYMBOL{z}}{2}}+12\COMMA 32\TIMES %
\SUPER{\SYMBOL{z}}{5}-{160\TIMES \SUPER{\SYMBOL{z}}{3}}+120\TIMES \SYMBOL{z}}%
\end{fricasmath}
\end{TeXOutput}
\formatResultType{List(Polynomial(Integer))}
\end{xtc}
\begin{xtc}
\begin{xtccomment}
\end{xtccomment}
\begin{spadsrc}
hermiteH(100, 1.0)
\end{spadsrc}
\begin{TeXOutput}
\begin{fricasmath}{13}
-{\STRING{0.1448706729337934088E93}}%
\end{fricasmath}
\end{TeXOutput}
\formatResultType{Float}
\end{xtc}
\begin{xtc}
\begin{xtccomment}
The expression \spad{laguerreL(n,z)} evaluates to the \eth{n} Laguerre
\index{polynomial!Laguerre}
polynomial.
\end{xtccomment}
\begin{spadsrc}
[laguerreL(i, z) for i in 0..4]
\end{spadsrc}
\begin{TeXOutput}
\begin{fricasmath}{14}
\BRACKET{1\COMMA -{\SYMBOL{z}}+1\COMMA \SUPER{\SYMBOL{z}}{2}-{4\TIMES \SYMBOL%
{z}}+2\COMMA -{\SUPER{\SYMBOL{z}}{3}}+9\TIMES \SUPER{\SYMBOL{z}}{2}-{18%
\TIMES \SYMBOL{z}}+6\COMMA \SUPER{\SYMBOL{z}}{4}-{16\TIMES \SUPER{\SYMBOL{z}%
}{3}}+72\TIMES \SUPER{\SYMBOL{z}}{2}-{96\TIMES \SYMBOL{z}}+24}%
\end{fricasmath}
\end{TeXOutput}
\formatResultType{List(Polynomial(Integer))}
\end{xtc}
\begin{xtc}
\begin{xtccomment}
\end{xtccomment}
\begin{spadsrc}
laguerreL(4, 1.2)
\end{spadsrc}
\begin{TeXOutput}
\begin{fricasmath}{15}
-{\STRING{13.0944}}%
\end{fricasmath}
\end{TeXOutput}
\formatResultType{Float}
\end{xtc}
\begin{xtc}
\begin{xtccomment}
\end{xtccomment}
\begin{spadsrc}
[laguerreL(j, 3, z) for j in 0..4]
\end{spadsrc}
\begin{TeXOutput}
\begin{fricasmath}{16}
\BRACKET{-{\SUPER{\SYMBOL{z}}{3}}+9\TIMES \SUPER{\SYMBOL{z}}{2}-{18\TIMES %
\SYMBOL{z}}+6\COMMA -{3\TIMES \SUPER{\SYMBOL{z}}{2}}+18\TIMES \SYMBOL{z}-{18}%
\COMMA -{6\TIMES \SYMBOL{z}}+18\COMMA -{6}\COMMA 0}%
\end{fricasmath}
\end{TeXOutput}
\formatResultType{List(Polynomial(Integer))}
\end{xtc}
\begin{xtc}
\begin{xtccomment}
\end{xtccomment}
\begin{spadsrc}
laguerreL(1, 3, 2.1)
\end{spadsrc}
\begin{TeXOutput}
\begin{fricasmath}{17}
\STRING{6.57}%
\end{fricasmath}
\end{TeXOutput}
\formatResultType{Float}
\end{xtc}
\begin{xtc}
\begin{xtccomment}
The expression
\index{polynomial!Legendre}
\spad{legendreP(n,z)} evaluates to the \eth{n} Legendre polynomial,
\end{xtccomment}
\begin{spadsrc}
[legendreP(i,z) for i in 0..5]
\end{spadsrc}
\begin{TeXOutput}
\begin{fricasmath}{18}
\BRACKET{1\COMMA \SYMBOL{z}\COMMA \frac{3}{2}\TIMES \SUPER{\SYMBOL{z}}{2}-{%
\frac{1}{2}}\COMMA \frac{5}{2}\TIMES \SUPER{\SYMBOL{z}}{3}-{\frac{3}{2}%
\TIMES \SYMBOL{z}}\COMMA \frac{35}{8}\TIMES \SUPER{\SYMBOL{z}}{4}-{\frac{15}{%
4}\TIMES \SUPER{\SYMBOL{z}}{2}}+\frac{3}{8}\COMMA \frac{63}{8}\TIMES \SUPER{%
\SYMBOL{z}}{5}-{\frac{35}{4}\TIMES \SUPER{\SYMBOL{z}}{3}}+\frac{15}{8}\TIMES %
\SYMBOL{z}}%
\end{fricasmath}
\end{TeXOutput}
\formatResultType{List(Polynomial(Fraction(Integer)))}
\end{xtc}
\begin{xtc}
\begin{xtccomment}
\end{xtccomment}
\begin{spadsrc}
legendreP(3, 3.0*%i)
\end{spadsrc}
\begin{TeXOutput}
\begin{fricasmath}{19}
-{\STRING{72.0}\TIMES \ImaginaryI }%
\end{fricasmath}
\end{TeXOutput}
\formatResultType{Complex(Float)}
\end{xtc}
%

Finally, three number-theoretic polynomial operations may be evaluated.
\index{number theory}
The following operations are provided by the package
\spadtype{NumberTheoreticPolynomialFunctions}.
\exptypeindex{NumberTheoreticPolynomialFunctions}.

\noindent
\spadfun{bernoulliB}: \spad{(NonNegativeInteger, R) -> R} \hbox{}\hfill\newline
   \spad{bernoulliB(n,z)} is the \eth{n} Bernoulli polynomial,
\index{polynomial!Bernoulli}
   $B_n (z)$.  These are defined by
\begin{displaymath}
\frac{t e^{z t}}{e^t - 1} = \sum_{n=0}^{\infty} B_n (z) \frac{t^n}{n!}
\end{displaymath}

\noindent
\spadfun{eulerE}: \spad{(NonNegativeInteger, R) -> R} \hbox{}\hfill\newline
   \spad{eulerE(n,z)} is the \eth{n} Euler polynomial,
\index{Euler!polynomial}
   $E_n (z)$.  These are defined by
\index{polynomial!Euler}
\begin{displaymath}
\frac{2 e^{z t}}{e^t + 1} = \sum_{n=0}^{\infty} E_n (z) \frac{t^n}{n!}.
\end{displaymath}

\noindent
\spadfun{cyclotomic}: \spad{(NonNegativeInteger, R) -> R}\hbox{}\hfill\newline
   \spad{cyclotomic(n,z)} is the \eth{n} cyclotomic polynomial
   $\Phi_n (z)$.  This is the polynomial whose
   roots are precisely the primitive \eth{n} roots of unity.
\index{Euler!totient function}
   This polynomial has degree given by the Euler totient function
\index{function!totient}
   $\varphi(n)$.

\begin{xtc}
\begin{xtccomment}
The expression \spad{bernoulliB(n,z)} evaluates to the \eth{n} Bernoulli
\index{polynomial!Bernouilli}
polynomial.
\end{xtccomment}
\begin{spadsrc}
bernoulliB(3, z)
\end{spadsrc}
\begin{TeXOutput}
\begin{fricasmath}{20}
\SUPER{\SYMBOL{z}}{3}-{\frac{3}{2}\TIMES \SUPER{\SYMBOL{z}}{2}}+\frac{1}{2}%
\TIMES \SYMBOL{z}%
\end{fricasmath}
\end{TeXOutput}
\formatResultType{Polynomial(Fraction(Integer))}
\end{xtc}
\begin{xtc}
\begin{xtccomment}
\end{xtccomment}
\begin{spadsrc}
bernoulliB(3, 0.7 + 0.4 * %i)
\end{spadsrc}
\begin{TeXOutput}
\begin{fricasmath}{21}
-{\STRING{0.138}}-{\STRING{0.116}\TIMES \ImaginaryI }%
\end{fricasmath}
\end{TeXOutput}
\formatResultType{Complex(Float)}
\end{xtc}
\begin{xtc}
\begin{xtccomment}
The expression
\index{polynomial!Euler}
\spad{eulerE(n,z)} evaluates to the \eth{n} Euler polynomial.
\end{xtccomment}
\begin{spadsrc}
eulerE(3, z)
\end{spadsrc}
\begin{TeXOutput}
\begin{fricasmath}{22}
\SUPER{\SYMBOL{z}}{3}-{\frac{3}{2}\TIMES \SUPER{\SYMBOL{z}}{2}}+\frac{1}{4}%
\end{fricasmath}
\end{TeXOutput}
\formatResultType{Polynomial(Fraction(Integer))}
\end{xtc}
\begin{xtc}
\begin{xtccomment}
\end{xtccomment}
\begin{spadsrc}
eulerE(3, 0.7 + 0.4 * %i)
\end{spadsrc}
\begin{TeXOutput}
\begin{fricasmath}{23}
-{\STRING{0.238}}-{\STRING{0.316}\TIMES \ImaginaryI }%
\end{fricasmath}
\end{TeXOutput}
\formatResultType{Complex(Float)}
\end{xtc}
\begin{xtc}
\begin{xtccomment}
The expression
\index{polynomial!cyclotomic}
\spad{cyclotomic(n,z)} evaluates to the \eth{n} cyclotomic polynomial.
\index{cyclotomic polynomial}
\end{xtccomment}
\begin{spadsrc}
cyclotomic(3, z)
\end{spadsrc}
\begin{TeXOutput}
\begin{fricasmath}{24}
\SUPER{\SYMBOL{z}}{2}+\SYMBOL{z}+1%
\end{fricasmath}
\end{TeXOutput}
\formatResultType{Polynomial(Integer)}
\end{xtc}
\begin{xtc}
\begin{xtccomment}
\end{xtccomment}
\begin{spadsrc}
cyclotomic(3, (-1.0 + 0.0 * %i)^(2/3))
\end{spadsrc}
\begin{TeXOutput}
\begin{fricasmath}{25}
\STRING{0.0}%
\end{fricasmath}
\end{TeXOutput}
\formatResultType{Complex(Float)}
\end{xtc}

Drawing complex functions in \Language{} is presently somewhat
awkward compared to drawing real functions.
It is necessary to use the \spadfun{draw} operations that operate
on functions rather than expressions.

\begin{psXtc}
\begin{xtccomment}
This is the complex exponential function (rotated interactively).
\index{function!complex exponential}
When this is displayed in color, the height is the value of the real part of
the function and the color is the imaginary part.
Red indicates large negative imaginary values, green indicates imaginary
values near zero and blue/violet indicates large positive imaginary values.
\end{xtccomment}
\begin{spadsrc}
draw((x,y)+-> real exp complex(x,y), -2..2, -2*%pi..2*%pi, colorFunction == (x, y) +->  imag exp complex(x,y), title=="exp(x+%i*y)", style=="smooth")
\end{spadsrc}
\epsffile[0 0 295 295]{compexp.ps}
\end{psXtc}

\begin{psXtc}
\begin{xtccomment}
This is the complex arctangent function.
\index{function!complex arctangent}
Again, the height is the real part of the function value but here
the color indicates the function value's phase.
The position of the branch cuts are clearly visible and one can
see that the function is real only for a real argument.
\end{xtccomment}
\begin{spadsrc}
vp := draw((x,y) +-> real atan complex(x,y), -%pi..%pi, -%pi..%pi, colorFunction==(x,y) +->argument atan complex(x,y), title=="atan(x+%i*y)", style=="shade"); rotate(vp,-160,-45); vp
\end{spadsrc}
\epsffile[0 0 295 295]{compatan.ps}
\end{psXtc}

\begin{psXtc}
\begin{xtccomment}
This is the complex Gamma function.
\end{xtccomment}
\begin{spadsrc}
draw((x,y) +-> max(min(real Gamma complex(x,y),4),-4), -%pi..%pi, -%pi..%pi, style=="shade", colorFunction == (x,y) +-> argument Gamma complex(x,y), title == "Gamma(x+%i*y)", var1Steps == 50, var2Steps== 50)
\end{spadsrc}
\epsffile[0 0 295 295]{compgamm.ps}
\end{psXtc}

\begin{psXtc}
\begin{xtccomment}
This shows the real Beta function near the origin.
\end{xtccomment}
\begin{spadsrc}
draw(Beta(x,y)/100, x=-1.6..1.7, y = -1.6..1.7, style=="shade", title=="Beta(x,y)", var1Steps==40, var2Steps==40)
\end{spadsrc}
\epsffile[0 0 295 295]{realbeta.ps}
\end{psXtc}

\begin{psXtc}
\begin{xtccomment}
This is the Bessel function $J_\alpha (x)$
for index $\alpha$ in the range \spad{-6..4} and
argument $x$ in the range \spad{2..14}.
\end{xtccomment}
\begin{spadsrc}
draw((alpha,x) +-> min(max(besselJ(alpha, x+8), -6), 6), -6..4, -6..6, title=="besselJ(alpha,x)", style=="shade", var1Steps==40, var2Steps==40)
\end{spadsrc}
\epsffile[0 0 295 295]{bessel.ps}
\end{psXtc}

\begin{psXtc}
\begin{xtccomment}
This is the modified Bessel function
$I_\alpha (x)$
evaluated for various real values of the index $\alpha$
and fixed argument $x = 5$.
\end{xtccomment}
\begin{spadsrc}
draw(besselI(alpha, 5), alpha = -12..12, unit==[5,20])
\end{spadsrc}
\epsffile[0 0 295 295]{modbess.ps}
\end{psXtc}

\begin{psXtc}
\begin{xtccomment}
This is similar to the last example
except the index $\alpha$
takes on complex values in a \spad{6 x 6} rectangle  centered on the origin.
\end{xtccomment}
\begin{spadsrc}
draw((x,y) +-> real besselI(complex(x/20, y/20),5), -60..60, -60..60, colorFunction == (x,y)+-> argument besselI(complex(x/20,y/20),5), title=="besselI(x+i*y,5)", style=="shade")
\end{spadsrc}
\epsffile[0 0 295 295]{modbessc.ps}
\end{psXtc}

% *********************************************************************
\head{section}{Polynomial Factorization}{ugProblemFactor}
% *********************************************************************
%
The \Language{} polynomial factorization
\index{polynomial!factorization}
facilities are available for all polynomial types and a wide variety of
coefficient domains.
\index{factorization}
Here are some examples.

% *********************************************************************
\head{subsection}{Integer and Rational Number Coefficients}{ugProblemFactorIntRat}
% *********************************************************************

\begin{xtc}
\begin{xtccomment}
Polynomials with integer
\index{polynomial!factorization!integer coefficients}
coefficients can be be factored.
\end{xtccomment}
\begin{spadsrc}
v := (4*x^3+2*y^2+1)*(12*x^5-x^3*y+12) 
\end{spadsrc}
\begin{TeXOutput}
\begin{fricasmath}{1}
-{2\TIMES \SUPER{\SYMBOL{x}}{3}\TIMES \SUPER{\SYMBOL{y}}{3}}+\PAREN{24\TIMES %
\SUPER{\SYMBOL{x}}{5}+24}\TIMES \SUPER{\SYMBOL{y}}{2}+\PAREN{-{4\TIMES \SUPER%
{\SYMBOL{x}}{6}}-{\SUPER{\SYMBOL{x}}{3}}}\TIMES \SYMBOL{y}+48\TIMES \SUPER{%
\SYMBOL{x}}{8}+12\TIMES \SUPER{\SYMBOL{x}}{5}+48\TIMES \SUPER{\SYMBOL{x}}{3}+%
12%
\end{fricasmath}
\end{TeXOutput}
\formatResultType{Polynomial(Integer)}
\end{xtc}
\begin{xtc}
\begin{xtccomment}
\end{xtccomment}
\begin{spadsrc}
factor v 
\end{spadsrc}
\begin{TeXOutput}
\begin{fricasmath}{2}
-{\PAREN{\SUPER{\SYMBOL{x}}{3}\TIMES \SYMBOL{y}-{12\TIMES \SUPER{\SYMBOL{x}}{%
5}}-{12}}\TIMES \PAREN{2\TIMES \SUPER{\SYMBOL{y}}{2}+4\TIMES \SUPER{\SYMBOL{x%
}}{3}+1}}%
\end{fricasmath}
\end{TeXOutput}
\formatResultType{Factored(Polynomial(Integer))}
\end{xtc}
\begin{xtc}
\begin{xtccomment}
Also, \Language{} can factor polynomials with
\index{polynomial!factorization!rational number coefficients}
rational number coefficients.
\end{xtccomment}
\begin{spadsrc}
w := (4*x^3+(2/3)*x^2+1)*(12*x^5-(1/2)*x^3+12) 
\end{spadsrc}
\begin{TeXOutput}
\begin{fricasmath}{3}
48\TIMES \SUPER{\SYMBOL{x}}{8}+8\TIMES \SUPER{\SYMBOL{x}}{7}-{2\TIMES \SUPER{%
\SYMBOL{x}}{6}}+\frac{35}{3}\TIMES \SUPER{\SYMBOL{x}}{5}+\frac{95}{2}\TIMES %
\SUPER{\SYMBOL{x}}{3}+8\TIMES \SUPER{\SYMBOL{x}}{2}+12%
\end{fricasmath}
\end{TeXOutput}
\formatResultType{Polynomial(Fraction(Integer))}
\end{xtc}
\begin{xtc}
\begin{xtccomment}
\end{xtccomment}
\begin{spadsrc}
factor w 
\end{spadsrc}
\begin{TeXOutput}
\begin{fricasmath}{4}
48\TIMES \PAREN{\SUPER{\SYMBOL{x}}{3}+\frac{1}{6}\TIMES \SUPER{\SYMBOL{x}}{2}%
+\frac{1}{4}}\TIMES \PAREN{\SUPER{\SYMBOL{x}}{5}-{\frac{1}{24}\TIMES \SUPER{%
\SYMBOL{x}}{3}}+1}%
\end{fricasmath}
\end{TeXOutput}
\formatResultType{Factored(Polynomial(Fraction(Integer)))}
\end{xtc}

% *********************************************************************
\head{subsection}{Finite Field Coefficients}{ugProblemFactorFF}
% *********************************************************************

Polynomials with coefficients in a finite field
\index{polynomial!factorization!finite field coefficients}
can be also be factored.
\index{finite field!factoring polynomial with coefficients in}

\begin{xtc}
\begin{xtccomment}
\end{xtccomment}
\begin{spadsrc}
u : POLY(PF(19)) :=3*x^4+2*x^2+15*x+18 
\end{spadsrc}
\begin{TeXOutput}
\begin{fricasmath}{1}
3\TIMES \SUPER{\SYMBOL{x}}{4}+2\TIMES \SUPER{\SYMBOL{x}}{2}+15\TIMES \SYMBOL{%
x}+18%
\end{fricasmath}
\end{TeXOutput}
\formatResultType{Polynomial(PrimeField(19))}
\end{xtc}
\begin{xtc}
\begin{xtccomment}
These include the integers mod \spad{p}, where \spad{p} is prime, and
extensions of these fields.
\end{xtccomment}
\begin{spadsrc}
factor u 
\end{spadsrc}
\begin{TeXOutput}
\begin{fricasmath}{2}
3\TIMES \PAREN{\SUPER{\SYMBOL{x}}{3}+\SUPER{\SYMBOL{x}}{2}+8\TIMES \SYMBOL{x}%
+13}\TIMES \PAREN{\SYMBOL{x}+18}%
\end{fricasmath}
\end{TeXOutput}
\formatResultType{Factored(Polynomial(PrimeField(19)))}
\end{xtc}
\begin{xtc}
\begin{xtccomment}
Convert this to have coefficients in the finite
field with $19^3$ elements.
See \spadref{ugProblemFinite} for more information
about finite fields.
\end{xtccomment}
\begin{spadsrc}
factor(u :: POLY FFX(PF 19,3)) 
\end{spadsrc}
\begin{TeXOutput}
\begin{fricasmath}{3}
3\TIMES \PAREN{\SYMBOL{x}+18}\TIMES \PAREN{\SYMBOL{x}+5\TIMES \SUPER{\SYMBOL{%
\%A}}{2}+3\TIMES \SYMBOL{\%A}+13}\TIMES \PAREN{\SYMBOL{x}+16\TIMES \SUPER{%
\SYMBOL{\%A}}{2}+14\TIMES \SYMBOL{\%A}+13}\TIMES \PAREN{\SYMBOL{x}+17\TIMES %
\SUPER{\SYMBOL{\%A}}{2}+2\TIMES \SYMBOL{\%A}+13}%
\end{fricasmath}
\end{TeXOutput}
\formatResultType{Factored(Polynomial(FiniteFieldExtension(PrimeField(19), 3)))}
\end{xtc}
%

% *********************************************************************
\head{subsection}{Simple Algebraic Extension Field Coefficients}{ugProblemFactorAlg}
% *********************************************************************

Polynomials with coefficients in simple algebraic extensions
\index{polynomial!factorization!algebraic extension field coefficients}
of the rational numbers can be factored.
\index{algebraic number}
\index{number!algebraic}

\begin{xtc}
\begin{xtccomment}
Here, \spad{aa} and \spad{bb} are symbolic roots of polynomials.
\end{xtccomment}
\begin{spadsrc}
aa := rootOf(aa^2+aa+1) 
\end{spadsrc}
\begin{TeXOutput}
\begin{fricasmath}{1}
\SYMBOL{aa}%
\end{fricasmath}
\end{TeXOutput}
\formatResultType{AlgebraicNumber}
\end{xtc}
\begin{xtc}
\begin{xtccomment}
\end{xtccomment}
\begin{spadsrc}
p:=(x^3+aa^2*x+y)*(aa*x^2+aa*x+aa*y^2)^2 
\end{spadsrc}
\begin{TeXOutput}
\begin{fricasmath}{2}
\PAREN{-{\SYMBOL{aa}}-{1}}\TIMES \SUPER{\SYMBOL{y}}{5}+\PAREN{\PAREN{-{%
\SYMBOL{aa}}-{1}}\TIMES \SUPER{\SYMBOL{x}}{3}+\SYMBOL{aa}\TIMES \SYMBOL{x}}%
\TIMES \SUPER{\SYMBOL{y}}{4}+\PAREN{\PAREN{-{2\TIMES \SYMBOL{aa}}-{2}}\TIMES %
\SUPER{\SYMBOL{x}}{2}+\PAREN{-{2\TIMES \SYMBOL{aa}}-{2}}\TIMES \SYMBOL{x}}%
\TIMES \SUPER{\SYMBOL{y}}{3}+\PAREN{\PAREN{-{2\TIMES \SYMBOL{aa}}-{2}}\TIMES %
\SUPER{\SYMBOL{x}}{5}+\PAREN{-{2\TIMES \SYMBOL{aa}}-{2}}\TIMES \SUPER{\SYMBOL%
{x}}{4}+2\TIMES \SYMBOL{aa}\TIMES \SUPER{\SYMBOL{x}}{3}+2\TIMES \SYMBOL{aa}%
\TIMES \SUPER{\SYMBOL{x}}{2}}\TIMES \SUPER{\SYMBOL{y}}{2}+\PAREN{\PAREN{-{%
\SYMBOL{aa}}-{1}}\TIMES \SUPER{\SYMBOL{x}}{4}+\PAREN{-{2\TIMES \SYMBOL{aa}}-{%
2}}\TIMES \SUPER{\SYMBOL{x}}{3}+\PAREN{-{\SYMBOL{aa}}-{1}}\TIMES \SUPER{%
\SYMBOL{x}}{2}}\TIMES \SYMBOL{y}+\PAREN{-{\SYMBOL{aa}}-{1}}\TIMES \SUPER{%
\SYMBOL{x}}{7}+\PAREN{-{2\TIMES \SYMBOL{aa}}-{2}}\TIMES \SUPER{\SYMBOL{x}}{6}%
-{\SUPER{\SYMBOL{x}}{5}}+2\TIMES \SYMBOL{aa}\TIMES \SUPER{\SYMBOL{x}}{4}+%
\SYMBOL{aa}\TIMES \SUPER{\SYMBOL{x}}{3}%
\end{fricasmath}
\end{TeXOutput}
\formatResultType{Polynomial(AlgebraicNumber)}
\end{xtc}
\begin{xtc}
\begin{xtccomment}
Note that the second argument to factor can be a list of
algebraic extensions to factor over.
\end{xtccomment}
\begin{spadsrc}
factor(p,[aa]) 
\end{spadsrc}
\begin{TeXOutput}
\begin{fricasmath}{3}
\PAREN{-{\SYMBOL{aa}}-{1}}\TIMES \PAREN{\SYMBOL{y}+\SUPER{\SYMBOL{x}}{3}+%
\PAREN{-{\SYMBOL{aa}}-{1}}\TIMES \SYMBOL{x}}\TIMES \SUPER{\PAREN{\SUPER{%
\SYMBOL{y}}{2}+\SUPER{\SYMBOL{x}}{2}+\SYMBOL{x}}}{2}%
\end{fricasmath}
\end{TeXOutput}
\formatResultType{Factored(Polynomial(AlgebraicNumber))}
\end{xtc}
\begin{xtc}
\begin{xtccomment}
This factors \spad{x^2+3} over the integers.
\end{xtccomment}
\begin{spadsrc}
factor(x^2+3)
\end{spadsrc}
\begin{TeXOutput}
\begin{fricasmath}{4}
\SUPER{\SYMBOL{x}}{2}+3%
\end{fricasmath}
\end{TeXOutput}
\formatResultType{Factored(Polynomial(Integer))}
\end{xtc}
\begin{xtc}
\begin{xtccomment}
Factor the same polynomial over the field obtained by adjoining
\spad{aa} to the rational numbers.
\end{xtccomment}
\begin{spadsrc}
factor(x^2+3,[aa]) 
\end{spadsrc}
\begin{TeXOutput}
\begin{fricasmath}{5}
\PAREN{\SYMBOL{x}-{2\TIMES \SYMBOL{aa}}-{1}}\TIMES \PAREN{\SYMBOL{x}+2\TIMES %
\SYMBOL{aa}+1}%
\end{fricasmath}
\end{TeXOutput}
\formatResultType{Factored(Polynomial(AlgebraicNumber))}
\end{xtc}
\begin{xtc}
\begin{xtccomment}
Factor \spad{x^6+108} over the same field.
\end{xtccomment}
\begin{spadsrc}
factor(x^6+108,[aa]) 
\end{spadsrc}
\begin{TeXOutput}
\begin{fricasmath}{6}
\PAREN{\SUPER{\SYMBOL{x}}{3}-{12\TIMES \SYMBOL{aa}}-{6}}\TIMES \PAREN{\SUPER{%
\SYMBOL{x}}{3}+12\TIMES \SYMBOL{aa}+6}%
\end{fricasmath}
\end{TeXOutput}
\formatResultType{Factored(Polynomial(AlgebraicNumber))}
\end{xtc}
\begin{xtc}
\begin{xtccomment}
\end{xtccomment}
\begin{spadsrc}
bb:=rootOf(bb^3-2) 
\end{spadsrc}
\begin{TeXOutput}
\begin{fricasmath}{7}
\SYMBOL{bb}%
\end{fricasmath}
\end{TeXOutput}
\formatResultType{AlgebraicNumber}
\end{xtc}
\begin{xtc}
\begin{xtccomment}
\end{xtccomment}
\begin{spadsrc}
factor(x^6+108,[bb]) 
\end{spadsrc}
\begin{TeXOutput}
\begin{fricasmath}{8}
\PAREN{\SUPER{\SYMBOL{x}}{2}-{3\TIMES \SYMBOL{bb}\TIMES \SYMBOL{x}}+3\TIMES %
\SUPER{\SYMBOL{bb}}{2}}\TIMES \PAREN{\SUPER{\SYMBOL{x}}{2}+3\TIMES \SUPER{%
\SYMBOL{bb}}{2}}\TIMES \PAREN{\SUPER{\SYMBOL{x}}{2}+3\TIMES \SYMBOL{bb}%
\TIMES \SYMBOL{x}+3\TIMES \SUPER{\SYMBOL{bb}}{2}}%
\end{fricasmath}
\end{TeXOutput}
\formatResultType{Factored(Polynomial(AlgebraicNumber))}
\end{xtc}
\begin{xtc}
\begin{xtccomment}
Factor again over the field obtained by adjoining both \spad{aa}
and \spad{bb} to the rational numbers.
\end{xtccomment}
\begin{spadsrc}
factor(x^6+108,[aa,bb]) 
\end{spadsrc}
\begin{TeXOutput}
\begin{fricasmath}{9}
\PAREN{\SYMBOL{x}+\PAREN{-{2\TIMES \SYMBOL{aa}}-{1}}\TIMES \SYMBOL{bb}}%
\TIMES \PAREN{\SYMBOL{x}+\PAREN{-{\SYMBOL{aa}}-{2}}\TIMES \SYMBOL{bb}}\TIMES %
\PAREN{\SYMBOL{x}+\PAREN{-{\SYMBOL{aa}}+1}\TIMES \SYMBOL{bb}}\TIMES \PAREN{%
\SYMBOL{x}+\PAREN{\SYMBOL{aa}-{1}}\TIMES \SYMBOL{bb}}\TIMES \PAREN{\SYMBOL{x}%
+\PAREN{\SYMBOL{aa}+2}\TIMES \SYMBOL{bb}}\TIMES \PAREN{\SYMBOL{x}+\PAREN{2%
\TIMES \SYMBOL{aa}+1}\TIMES \SYMBOL{bb}}%
\end{fricasmath}
\end{TeXOutput}
\formatResultType{Factored(Polynomial(AlgebraicNumber))}
\end{xtc}

% *********************************************************************
\head{subsection}{Factoring Rational Functions}{ugProblemFactorRatFun}
% *********************************************************************

Since fractions of polynomials form a field, every element (other than zero)
\index{rational function!factoring}
divides any other, so there is no useful notion of irreducible factors.
Thus the \spadfun{factor} operation is not very useful for fractions
of polynomials.

\begin{xtc}
\begin{xtccomment}
There is, instead, a specific operation \spadfun{factorFraction}
that separately factors the numerator and denominator and returns
a fraction of the factored results.
\end{xtccomment}
\begin{spadsrc}
factorFraction((x^2-4)/(y^2-4))
\end{spadsrc}
\begin{TeXOutput}
\begin{fricasmath}{1}
\frac{\PAREN{\SYMBOL{x}-{2}}\TIMES \PAREN{\SYMBOL{x}+2}}{\PAREN{\SYMBOL{y}-{2%
}}\TIMES \PAREN{\SYMBOL{y}+2}}%
\end{fricasmath}
\end{TeXOutput}
\formatResultType{Fraction(Factored(Polynomial(Integer)))}
\end{xtc}
\begin{xtc}
\begin{xtccomment}
You can also use \spadfun{map}. This expression
applies the \spadfun{factor} operation
to the numerator and denominator.
\end{xtccomment}
\begin{spadsrc}
map(factor,(x^2-4)/(y^2-4))
\end{spadsrc}
\begin{TeXOutput}
\begin{fricasmath}{2}
\frac{\PAREN{\SYMBOL{x}-{2}}\TIMES \PAREN{\SYMBOL{x}+2}}{\PAREN{\SYMBOL{y}-{2%
}}\TIMES \PAREN{\SYMBOL{y}+2}}%
\end{fricasmath}
\end{TeXOutput}
\formatResultType{Fraction(Factored(Polynomial(Integer)))}
\end{xtc}

% *********************************************************************
\head{section}{Manipulating Symbolic Roots of a Polynomial}{ugProblemSymRoot}
% *********************************************************************
%
In this section we show you how to work with one root or all roots
\index{root!symbolic}
of a polynomial.
These roots are represented symbolically (as opposed to being
numeric approximations).
See \spadref{ugxProblemOnePol} and \spadref{ugxProblemPolSys} for
information about solving for the roots of one or more
polynomials.

% *********************************************************************
\head{subsection}{Using a Single Root of a Polynomial}{ugxProblemSymRootOne}
% *********************************************************************

Use \spadfun{rootOf} to get a symbolic root of a polynomial:
\spad{rootOf(p, x)} returns a root of \spad{p(x)}.

\begin{xtc}
\begin{xtccomment}
This creates an algebraic number \spad{a}.
\index{algebraic number}
\index{number!algebraic}
\end{xtccomment}
\begin{spadsrc}
a := rootOf(a^4+1,a) 
\end{spadsrc}
\begin{TeXOutput}
\begin{fricasmath}{1}
\SYMBOL{a}%
\end{fricasmath}
\end{TeXOutput}
\formatResultType{Expression(Integer)}
\end{xtc}
\begin{xtc}
\begin{xtccomment}
To find the algebraic relation that defines \spad{a},
use \spadfun{definingPolynomial}.
\end{xtccomment}
\begin{spadsrc}
definingPolynomial a 
\end{spadsrc}
\begin{TeXOutput}
\begin{fricasmath}{2}
\SUPER{\SYMBOL{a}}{4}+1%
\end{fricasmath}
\end{TeXOutput}
\formatResultType{Expression(Integer)}
\end{xtc}
\begin{xtc}
\begin{xtccomment}
You can use \spad{a} in any further expression,
including a nested \spadfun{rootOf}.
\end{xtccomment}
\begin{spadsrc}
b := rootOf(b^2-a-1,b) 
\end{spadsrc}
\begin{TeXOutput}
\begin{fricasmath}{3}
\SYMBOL{b}%
\end{fricasmath}
\end{TeXOutput}
\formatResultType{Expression(Integer)}
\end{xtc}
\begin{xtc}
\begin{xtccomment}
Higher powers of the roots are automatically reduced during
calculations.
\end{xtccomment}
\begin{spadsrc}
a + b 
\end{spadsrc}
\begin{TeXOutput}
\begin{fricasmath}{4}
\SYMBOL{b}+\SYMBOL{a}%
\end{fricasmath}
\end{TeXOutput}
\formatResultType{Expression(Integer)}
\end{xtc}
\begin{xtc}
\begin{xtccomment}
\end{xtccomment}
\begin{spadsrc}
% ^ 5 
\end{spadsrc}
\begin{TeXOutput}
\begin{fricasmath}{5}
\PAREN{10\TIMES \SUPER{\SYMBOL{a}}{3}+11\TIMES \SUPER{\SYMBOL{a}}{2}+2\TIMES %
\SYMBOL{a}-{4}}\TIMES \SYMBOL{b}+15\TIMES \SUPER{\SYMBOL{a}}{3}+10\TIMES %
\SUPER{\SYMBOL{a}}{2}+4\TIMES \SYMBOL{a}-{10}%
\end{fricasmath}
\end{TeXOutput}
\formatResultType{Expression(Integer)}
\end{xtc}
\begin{xtc}
\begin{xtccomment}
The operation \spadfun{zeroOf} is similar to \spadfun{rootOf},
except that it may express the root using radicals in some cases.
\index{radical}
\end{xtccomment}
\begin{spadsrc}
rootOf(c^2+c+1,c)
\end{spadsrc}
\begin{TeXOutput}
\begin{fricasmath}{6}
\SYMBOL{c}%
\end{fricasmath}
\end{TeXOutput}
\formatResultType{Expression(Integer)}
\end{xtc}
\begin{xtc}
\begin{xtccomment}
\end{xtccomment}
\begin{spadsrc}
zeroOf(d^2+d+1,d)
\end{spadsrc}
\begin{TeXOutput}
\begin{fricasmath}{7}
\frac{\sqrt{-{3}}-{1}}{2}%
\end{fricasmath}
\end{TeXOutput}
\formatResultType{Expression(Integer)}
\end{xtc}
\begin{xtc}
\begin{xtccomment}
\end{xtccomment}
\begin{spadsrc}
rootOf(e^5-2,e)
\end{spadsrc}
\begin{TeXOutput}
\begin{fricasmath}{8}
\SYMBOL{e}%
\end{fricasmath}
\end{TeXOutput}
\formatResultType{Expression(Integer)}
\end{xtc}
\begin{xtc}
\begin{xtccomment}
\end{xtccomment}
\begin{spadsrc}
zeroOf(f^5-2,f)
\end{spadsrc}
\begin{TeXOutput}
\begin{fricasmath}{9}
\nthroot{2}{5}%
\end{fricasmath}
\end{TeXOutput}
\formatResultType{Expression(Integer)}
\end{xtc}

% *********************************************************************
\head{subsection}{Using All Roots of a Polynomial}{ugxProblemSymRootAll}
% *********************************************************************

Use \spadfun{rootsOf} to get all symbolic roots of a polynomial:
\spad{rootsOf(p, x)} returns a
list of all the roots of \spad{p(x)}.
If \spad{p(x)} has a multiple root of order \spad{n}, then that root
\index{root!multiple}
appears \spad{n} times in the list.
\typeout{Make sure these variables are x0 etc}

\begin{xtc}
\begin{xtccomment}
Compute all the roots of \spad{x^4 + x + 1}.
\end{xtccomment}
\begin{spadsrc}
l := rootsOf(x^4 + x + 1,x) 
\end{spadsrc}
\begin{TeXOutput}
\begin{fricasmath}{1}
\BRACKET{\SYMBOL{\%x0}\COMMA \SYMBOL{\%x1}\COMMA \SYMBOL{\%x2}\COMMA -{%
\SYMBOL{\%x2}}-{\SYMBOL{\%x1}}-{\SYMBOL{\%x0}}}%
\end{fricasmath}
\end{TeXOutput}
\formatResultType{List(Expression(Integer))}
\end{xtc}
\begin{xtc}
\begin{xtccomment}
As a side effect, the variables \spad{%x0, %x1} and \spad{%x2} are bound
to the first three roots of \spad{x^4 + x + 1}.
\end{xtccomment}
\begin{spadsrc}
%x0^5 
\end{spadsrc}
\begin{TeXOutput}
\begin{fricasmath}{2}
-{\SUPER{\SYMBOL{\%x0}}{2}}-{\SYMBOL{\%x0}}%
\end{fricasmath}
\end{TeXOutput}
\formatResultType{Expression(Integer)}
\end{xtc}
\begin{xtc}
\begin{xtccomment}
Although they all satisfy \spad{x^4 + x + 1 = 0, %x0, %x1},
and \spad{%x2} are different algebraic numbers.
To find the algebraic relation that defines each of them,
use \spadfun{definingPolynomial}.
\end{xtccomment}
\begin{spadsrc}
definingPolynomial %x0 
\end{spadsrc}
\begin{TeXOutput}
\begin{fricasmath}{3}
\SUPER{\SYMBOL{\%x0}}{4}+\SYMBOL{\%x0}+1%
\end{fricasmath}
\end{TeXOutput}
\formatResultType{Expression(Integer)}
\end{xtc}
\begin{xtc}
\begin{xtccomment}
\end{xtccomment}
\begin{spadsrc}
definingPolynomial %x1 
\end{spadsrc}
\begin{TeXOutput}
\begin{fricasmath}{4}
\SUPER{\SYMBOL{\%x0}}{3}+\SYMBOL{\%x1}\TIMES \SUPER{\SYMBOL{\%x0}}{2}+\SUPER{%
\SYMBOL{\%x1}}{2}\TIMES \SYMBOL{\%x0}+\SUPER{\SYMBOL{\%x1}}{3}+1%
\end{fricasmath}
\end{TeXOutput}
\formatResultType{Expression(Integer)}
\end{xtc}
\begin{xtc}
\begin{xtccomment}
\end{xtccomment}
\begin{spadsrc}
definingPolynomial %x2 
\end{spadsrc}
\begin{TeXOutput}
\begin{fricasmath}{5}
\SUPER{\SYMBOL{\%x1}}{2}+\PAREN{\SYMBOL{\%x0}+\SYMBOL{\%x2}}\TIMES \SYMBOL{%
\%x1}+\SUPER{\SYMBOL{\%x0}}{2}+\SYMBOL{\%x2}\TIMES \SYMBOL{\%x0}+\SUPER{%
\SYMBOL{\%x2}}{2}%
\end{fricasmath}
\end{TeXOutput}
\formatResultType{Expression(Integer)}
\end{xtc}
\begin{xtc}
\begin{xtccomment}
We can check that the sum and product of the roots of \spad{x^4 + x + 1} are
its trace and norm.
\end{xtccomment}
\begin{spadsrc}
x3 := last l 
\end{spadsrc}
\begin{TeXOutput}
\begin{fricasmath}{6}
-{\SYMBOL{\%x2}}-{\SYMBOL{\%x1}}-{\SYMBOL{\%x0}}%
\end{fricasmath}
\end{TeXOutput}
\formatResultType{Expression(Integer)}
\end{xtc}
\begin{xtc}
\begin{xtccomment}
\end{xtccomment}
\begin{spadsrc}
%x0 + %x1 + %x2 + x3 
\end{spadsrc}
\begin{TeXOutput}
\begin{fricasmath}{7}
0%
\end{fricasmath}
\end{TeXOutput}
\formatResultType{Expression(Integer)}
\end{xtc}
\begin{xtc}
\begin{xtccomment}
\end{xtccomment}
\begin{spadsrc}
%x0 * %x1 * %x2 * x3 
\end{spadsrc}
\begin{TeXOutput}
\begin{fricasmath}{8}
1%
\end{fricasmath}
\end{TeXOutput}
\formatResultType{Expression(Integer)}
\end{xtc}
\begin{xtc}
\begin{xtccomment}
Note, that in general roots are expressions in new symbols. For
example for \spad{x^4 + 1} the second root is a product.
\end{xtccomment}
\begin{spadsrc}
rootsOf(x^4 + 1, x)
\end{spadsrc}
\begin{TeXOutput}
\begin{fricasmath}{9}
\BRACKET{\SYMBOL{\%x4}\COMMA \SYMBOL{\%x4}\TIMES \SYMBOL{\%x5}\COMMA -{%
\SYMBOL{\%x4}}\COMMA -{\SYMBOL{\%x4}\TIMES \SYMBOL{\%x5}}}%
\end{fricasmath}
\end{TeXOutput}
\formatResultType{List(Expression(Integer))}
\end{xtc}
\begin{xtc}
\begin{xtccomment}
Corresponding to the pair of operations
\spadfun{rootOf}/\spadfun{zeroOf} in
\spadref{ugxProblemOnePol}, there is
an operation \spadfun{zerosOf} that, like \spadfun{rootsOf},
computes all the roots
of a given polynomial, but which expresses some of them in terms of
radicals.
\end{xtccomment}
\begin{spadsrc}
zerosOf(y^4 + y + 1, y) 
\end{spadsrc}
\begin{TeXOutput}
\begin{fricasmath}{10}
\BRACKET{\SYMBOL{\%x0}\COMMA \SYMBOL{\%x1}\COMMA \frac{\sqrt{-{3\TIMES \SUPER%
{\SYMBOL{\%x1}}{2}}-{2\TIMES \SYMBOL{\%x0}\TIMES \SYMBOL{\%x1}}-{3\TIMES %
\SUPER{\SYMBOL{\%x0}}{2}}}-{\SYMBOL{\%x1}}-{\SYMBOL{\%x0}}}{2}\COMMA \frac{-{%
\sqrt{-{3\TIMES \SUPER{\SYMBOL{\%x1}}{2}}-{2\TIMES \SYMBOL{\%x0}\TIMES %
\SYMBOL{\%x1}}-{3\TIMES \SUPER{\SYMBOL{\%x0}}{2}}}}-{\SYMBOL{\%x1}}-{\SYMBOL{%
\%x0}}}{2}}%
\end{fricasmath}
\end{TeXOutput}
\formatResultType{List(Expression(Integer))}
\end{xtc}
\begin{xtc}
\begin{xtccomment}
As you see, only two implicit algebraic numbers were created
(\spad{%y0,%y1}), and its defining equations are below.
The other two roots are expressed in radicals.
\end{xtccomment}
\begin{spadsrc}
definingPolynomial %y0 
\end{spadsrc}
\begin{TeXOutput}
\begin{fricasmath}{11}
\SUPER{\SYMBOL{\%x0}}{4}+\SYMBOL{\%x0}+1%
\end{fricasmath}
\end{TeXOutput}
\formatResultType{Expression(Integer)}
\end{xtc}
\begin{xtc}
\begin{xtccomment}
\end{xtccomment}
\begin{spadsrc}
definingPolynomial %y1 
\end{spadsrc}
\begin{TeXOutput}
\begin{fricasmath}{12}
\SUPER{\SYMBOL{\%x0}}{3}+\SYMBOL{\%x1}\TIMES \SUPER{\SYMBOL{\%x0}}{2}+\SUPER{%
\SYMBOL{\%x1}}{2}\TIMES \SYMBOL{\%x0}+\SUPER{\SYMBOL{\%x1}}{3}+1%
\end{fricasmath}
\end{TeXOutput}
\formatResultType{Expression(Integer)}
\end{xtc}
\begin{xtc}
\begin{xtccomment}
For \spad{x^4 + 1} all roots can be expressied in radicals.
\end{xtccomment}
\begin{spadsrc}
zerosOf(x^4 + 1)
\end{spadsrc}
\begin{TeXOutput}
\begin{fricasmath}{13}
\BRACKET{\frac{\sqrt{-{1}}+1}{\sqrt{2}}\COMMA \frac{\sqrt{-{1}}-{1}}{\sqrt{2}%
}\COMMA \frac{-{\sqrt{-{1}}}-{1}}{\sqrt{2}}\COMMA \frac{-{\sqrt{-{1}}}+1}{%
\sqrt{2}}}%
\end{fricasmath}
\end{TeXOutput}
\formatResultType{List(AlgebraicNumber)}
\end{xtc}
% *********************************************************************
\head{section}{Computation of Eigenvalues and Eigenvectors}{ugProblemEigen}
% *********************************************************************
%
In this section we show you
some of \Language{}'s facilities for computing and
\index{eigenvalue}
manipulating eigenvalues and eigenvectors, also called
\index{eigenvector}
characteristic values and characteristic vectors,
\index{characteristic!value}
respectively.
\index{characteristic!vector}

\begin{xtc}
\begin{xtccomment}
Let's first create a matrix with integer entries.
\end{xtccomment}
\begin{spadsrc}
m1 := matrix [[1,2,1],[2,1,-2],[1,-2,4]] 
\end{spadsrc}
\begin{TeXOutput}
\begin{fricasmath}{1}
\begin{MATRIX}{3}1&2&1\\2&1&-{2}\\1&-{2}&4\end{MATRIX}%
\end{fricasmath}
\end{TeXOutput}
\formatResultType{Matrix(Integer)}
\end{xtc}
\begin{xtc}
\begin{xtccomment}
To get a list of the {\it rational} eigenvalues,
use the operation \spadfun{eigenvalues}.
\end{xtccomment}
\begin{spadsrc}
leig := eigenvalues(m1) 
\end{spadsrc}
\begin{TeXOutput}
\begin{fricasmath}{2}
\BRACKET{5\COMMA \SYMBOL{\%P}\mid \SUPER{\SYMBOL{\%P}}{2}-{\SYMBOL{\%P}}-{5}}%
\end{fricasmath}
\end{TeXOutput}
\formatResultType{List(Union(Fraction(Polynomial(Integer)), SuchThat(Symbol, Polynomial(Integer))))}
\end{xtc}
\begin{xtc}
\begin{xtccomment}
Given an explicit eigenvalue, \spadfun{eigenvector} computes the eigenvectors
corresponding to it.
\end{xtccomment}
\begin{spadsrc}
eigenvector(first(leig),m1) 
\end{spadsrc}
\begin{TeXOutput}
\begin{fricasmath}{3}
\BRACKET{\begin{MATRIX}{1}0\\-{\frac{1}{2}}\\1\end{MATRIX}}%
\end{fricasmath}
\end{TeXOutput}
\formatResultType{List(Matrix(Fraction(Polynomial(Fraction(Integer)))))}
\end{xtc}

The operation \spadfun{eigenvectors} returns a list of pairs of values and
vectors. When an eigenvalue is rational, \Language{} gives you
the value explicitly; otherwise, its minimal polynomial is given,
(the polynomial of lowest degree with the eigenvalues as roots),
together with a parametric representation of the eigenvector using the
eigenvalue.
This means that if you ask \Language{} to \spadfun{solve}
the minimal polynomial, then you can substitute these roots
\index{polynomial!minimal}
into the parametric form of the corresponding eigenvectors.
\index{minimal polynomial}

\begin{xtc}
\begin{xtccomment}
You must be aware that unless an exact eigenvalue has been computed,
the eigenvector may be badly in error.
\end{xtccomment}
\begin{spadsrc}
eigenvectors(m1) 
\end{spadsrc}
\begin{TeXOutput}
\begin{fricasmath}{4}
\BRACKET{\BRACKET{\SYMBOL{eigval}=5\COMMA \SYMBOL{eigmult}=1\COMMA \SYMBOL{%
eigvec}=\BRACKET{\begin{MATRIX}{1}0\\-{\frac{1}{2}}\\1\end{MATRIX}}}\COMMA %
\BRACKET{\SYMBOL{eigval}=\PAREN{\SYMBOL{\%R}\mid \SUPER{\SYMBOL{\%R}}{2}-{%
\SYMBOL{\%R}}-{5}}\COMMA \SYMBOL{eigmult}=1\COMMA \SYMBOL{eigvec}=\BRACKET{%
\begin{MATRIX}{1}\SYMBOL{\%R}\\2\\1\end{MATRIX}}}}%
\end{fricasmath}
\end{TeXOutput}
\formatResultType{List(Record(eigval: Union(Fraction(Polynomial(Integer)), SuchThat(Symbol, Polynomial(Integer))), eigmult: NonNegativeInteger, eigvec: List(Matrix(Fraction(Polynomial(Integer))))))}
\end{xtc}
\begin{xtc}
\begin{xtccomment}
Another possibility is to use the operation
\spadfun{radicalEigenvectors}
tries to compute explicitly the eigenvectors
in terms of radicals.
\index{radical}
\end{xtccomment}
\begin{spadsrc}
radicalEigenvectors(m1) 
\end{spadsrc}
\begin{TeXOutput}
\begin{fricasmath}{5}
\BRACKET{\BRACKET{\SYMBOL{radval}=\frac{\sqrt{21}+1}{2}\COMMA \SYMBOL{radmult%
}=1\COMMA \SYMBOL{radvect}=\BRACKET{\begin{MATRIX}{1}\frac{\sqrt{21}+1}{2}\\2%
\\1\end{MATRIX}}}\COMMA \BRACKET{\SYMBOL{radval}=\frac{-{\sqrt{21}}+1}{2}%
\COMMA \SYMBOL{radmult}=1\COMMA \SYMBOL{radvect}=\BRACKET{\begin{MATRIX}{1}%
\frac{-{\sqrt{21}}+1}{2}\\2\\1\end{MATRIX}}}\COMMA \BRACKET{\SYMBOL{radval}=5%
\COMMA \SYMBOL{radmult}=1\COMMA \SYMBOL{radvect}=\BRACKET{\begin{MATRIX}{1}0%
\\-{\frac{1}{2}}\\1\end{MATRIX}}}}%
\end{fricasmath}
\end{TeXOutput}
\formatResultType{List(Record(radval: Expression(Integer), radmult: Integer, radvect: List(Matrix(Expression(Integer)))))}
\end{xtc}

Alternatively, \Language{} can compute real or complex approximations to the
\index{approximation}
eigenvectors and eigenvalues using the operations \spadfun{realEigenvectors}
or \spadfun{complexEigenvectors}.
They each take an additional argument $\epsilon$
to specify the ``precision'' required.
\index{precision}
In the real case, this means that each approximation will be within
$\pm\epsilon$ of the actual
result.
In the complex case, this means that each approximation will be within
$\pm\epsilon$ of the actual result
in each of the real and imaginary parts.

\begin{xtc}
\begin{xtccomment}
The precision can be specified as a \spadtype{Float} if the results are
desired in floating-point notation, or as \spadtype{Fraction Integer} if the
results are to be expressed using rational (or complex rational) numbers.
\end{xtccomment}
\begin{spadsrc}
realEigenvectors(m1,1/1000) 
\end{spadsrc}
\begin{TeXOutput}
\begin{fricasmath}{6}
\BRACKET{\BRACKET{\SYMBOL{outval}=5\COMMA \SYMBOL{outmult}=1\COMMA \SYMBOL{%
outvect}=\BRACKET{\begin{MATRIX}{1}0\\-{\frac{1}{2}}\\1\end{MATRIX}}}\COMMA %
\BRACKET{\SYMBOL{outval}=\frac{5717}{2048}\COMMA \SYMBOL{outmult}=1\COMMA %
\SYMBOL{outvect}=\BRACKET{\begin{MATRIX}{1}\frac{5717}{2048}\\2\\1%
\end{MATRIX}}}\COMMA \BRACKET{\SYMBOL{outval}=-{\frac{3669}{2048}}\COMMA %
\SYMBOL{outmult}=1\COMMA \SYMBOL{outvect}=\BRACKET{\begin{MATRIX}{1}-{\frac{%
3669}{2048}}\\2\\1\end{MATRIX}}}}%
\end{fricasmath}
\end{TeXOutput}
\formatResultType{List(Record(outval: Fraction(Integer), outmult: Integer, outvect: List(Matrix(Fraction(Integer)))))}
\end{xtc}
\begin{xtc}
\begin{xtccomment}
If an \spad{n} by \spad{n} matrix has \spad{n} distinct eigenvalues (and
therefore \spad{n} eigenvectors) the operation \spadfun{eigenMatrix}
gives you a matrix of the eigenvectors.
\end{xtccomment}
\begin{spadsrc}
eigenMatrix(m1) 
\end{spadsrc}
\begin{TeXOutput}
\begin{fricasmath}{7}
\begin{MATRIX}{3}\frac{\sqrt{21}+1}{2}&\frac{-{\sqrt{21}}+1}{2}&0\\2&2&-{%
\frac{1}{2}}\\1&1&1\end{MATRIX}%
\end{fricasmath}
\end{TeXOutput}
\formatResultType{Union(Matrix(Expression(Integer)), ...)}
\end{xtc}
\begin{xtc}
\begin{xtccomment}
\end{xtccomment}
\begin{spadsrc}
m2 := matrix [[-5,-2],[18,7]] 
\end{spadsrc}
\begin{TeXOutput}
\begin{fricasmath}{8}
\begin{MATRIX}{2}-{5}&-{2}\\18&7\end{MATRIX}%
\end{fricasmath}
\end{TeXOutput}
\formatResultType{Matrix(Integer)}
\end{xtc}
\begin{xtc}
\begin{xtccomment}
\end{xtccomment}
\begin{spadsrc}
eigenMatrix(m2) 
\end{spadsrc}
\begin{TeXOutput}
\begin{fricasmath}{9}
\STRING{"failed"}%
\end{fricasmath}
\end{TeXOutput}
\formatResultType{Union("failed", ...)}
\end{xtc}
%
%
\begin{xtc}
\begin{xtccomment}
If a symmetric matrix
\index{matrix!symmetric}
has a basis of orthonormal eigenvectors, then
\index{basis!orthonormal}
\spadfun{orthonormalBasis} computes a list of these vectors.
\index{orthonormal basis}
\end{xtccomment}
\begin{spadsrc}
m3 := matrix [[1,2],[2,1]] 
\end{spadsrc}
\begin{TeXOutput}
\begin{fricasmath}{10}
\begin{MATRIX}{2}1&2\\2&1\end{MATRIX}%
\end{fricasmath}
\end{TeXOutput}
\formatResultType{Matrix(Integer)}
\end{xtc}
\begin{xtc}
\begin{xtccomment}
\end{xtccomment}
\begin{spadsrc}
orthonormalBasis(m3) 
\end{spadsrc}
\begin{TeXOutput}
\begin{fricasmath}{11}
\BRACKET{\begin{MATRIX}{1}-{\frac{1}{\sqrt{2}}}\\\frac{1}{\sqrt{2}}%
\end{MATRIX}\COMMA \begin{MATRIX}{1}\frac{1}{\sqrt{2}}\\\frac{1}{\sqrt{2}}%
\end{MATRIX}}%
\end{fricasmath}
\end{TeXOutput}
\formatResultType{List(Matrix(Expression(Integer)))}
\end{xtc}

% *********************************************************************
\head{section}{Solution of Linear and Polynomial Equations}{ugProblemLinPolEqn}
% *********************************************************************
%
In this section we discuss the \Language{} facilities for solving
systems of linear equations, finding the roots of polynomials and
\index{linear equation}
solving systems of polynomial equations.
For a discussion of the solution of differential equations, see
\spadref{ugProblemDEQ}.

% *********************************************************************
\head{subsection}{Solution of Systems of Linear Equations}{ugxProblemLinSys}
% *********************************************************************

You can use the operation \spadfun{solve} to solve systems of linear equations.
\index{equation!linear!solving}

The operation \spadfun{solve} takes two arguments, the list of equations and the
list of the unknowns to be solved for.
A system of linear equations need not have a unique solution.

\begin{xtc}
\begin{xtccomment}
To solve the linear system:
$$
\begin{array}{rcrcrcr}
  x &+&   y &+&   z &=& 8 \\
3 x &-& 2 y &+&   z &=& 0 \\
  x &+& 2 y &+& 2 z &=& 17
\end{array}
$$
evaluate this expression.
\end{xtccomment}
\begin{spadsrc}
solve([x+y+z=8,3*x-2*y+z=0,x+2*y+2*z=17],[x,y,z])
\end{spadsrc}
\begin{TeXOutput}
\begin{fricasmath}{1}
\BRACKET{\BRACKET{\SYMBOL{x}=-{1}\COMMA \SYMBOL{y}=2\COMMA \SYMBOL{z}=7}}%
\end{fricasmath}
\end{TeXOutput}
\formatResultType{List(List(Equation(Fraction(Polynomial(Integer)))))}
\end{xtc}

Parameters are given as new variables starting with a percent sign and
\spadSyntax{%} and
the variables are expressed in terms of the parameters.
If the system has no solutions then the empty list is returned.

\begin{xtc}
\begin{xtccomment}
When you solve the linear system
$$
\begin{array}{rcrcrcr}
  x&+&2 y&+&3 z&=&2 \\
2 x&+&3 y&+&4 z&=&2 \\
3 x&+&4 y&+&5 z&=&2
\end{array}
$$
with this expression
you get a solution involving a parameter.
\end{xtccomment}
\begin{spadsrc}
solve([x+2*y+3*z=2,2*x+3*y+4*z=2,3*x+4*y+5*z=2],[x,y,z])
\end{spadsrc}
\begin{TeXOutput}
\begin{fricasmath}{2}
\BRACKET{\BRACKET{\SYMBOL{x}=\SYMBOL{\%W}-{2}\COMMA \SYMBOL{y}=-{2\TIMES %
\SYMBOL{\%W}}+2\COMMA \SYMBOL{z}=\SYMBOL{\%W}}}%
\end{fricasmath}
\end{TeXOutput}
\formatResultType{List(List(Equation(Fraction(Polynomial(Integer)))))}
\end{xtc}

The system can also be presented as a matrix and a vector.
The matrix contains the coefficients of the linear equations and
the vector contains the numbers appearing on the right-hand sides
of the equations.
You may input the matrix as a list of rows and the vector as a
list of its elements.

\begin{xtc}
\begin{xtccomment}
To solve the system:
$$
\begin{array}{rcrcrcr}
  x&+&  y&+&  z&=&8  \\
3 x&-&2 y&+&  z&=&0  \\
  x&+&2 y&+&2 z&=&17
\end{array}
$$
in matrix form you would evaluate this expression.
\end{xtccomment}
\begin{spadsrc}
solve([[1,1,1],[3,-2,1],[1,2,2]],[8,0,17])
\end{spadsrc}
\begin{TeXOutput}
\begin{fricasmath}{3}
\BRACKET{\SYMBOL{particular}=\BRACKET{-{1}\COMMA 2\COMMA 7}\COMMA \SYMBOL{%
basis}=\BRACKET{}}%
\end{fricasmath}
\end{TeXOutput}
\formatResultType{Record(particular: Union(Vector(Fraction(Integer)), "failed"), basis: List(Vector(Fraction(Integer))))}
\end{xtc}

The solutions are presented as a \pspadtype{Record} with two
components: the component
{\it particular}
contains a particular solution of the given system or
the item {\tt "failed"} if there are no solutions, the component
{\it basis} contains a list of vectors that
are a basis for the space of solutions of the corresponding
homogeneous system.
If the system of linear equations does not have a unique solution,
then the {\it basis} component contains
non-trivial vectors.

\begin{xtc}
\begin{xtccomment}
This happens when you solve the linear system
$$
\begin{array}{rcrcrcr}
  x&+&2 y&+&3 z&=&2 \\
2 x&+&3 y&+&4 z&=&2 \\
3 x&+&4 y&+&5 z&=&2
\end{array}
$$
with this command.
\end{xtccomment}
\begin{spadsrc}
solve([[1,2,3],[2,3,4],[3,4,5]],[2,2,2])
\end{spadsrc}
\begin{TeXOutput}
\begin{fricasmath}{4}
\BRACKET{\SYMBOL{particular}=\BRACKET{-{2}\COMMA 2\COMMA 0}\COMMA \SYMBOL{%
basis}=\BRACKET{\BRACKET{1\COMMA -{2}\COMMA 1}}}%
\end{fricasmath}
\end{TeXOutput}
\formatResultType{Record(particular: Union(Vector(Fraction(Integer)), "failed"), basis: List(Vector(Fraction(Integer))))}
\end{xtc}

All solutions of this system are obtained by adding the particular
solution with a linear combination of the
{\it basis} vectors.

When no solution exists then {\tt "failed"} is returned as the
{\it particular} component.

\begin{xtc}
\begin{xtccomment}
For example:
\end{xtccomment}
\begin{spadsrc}
solve([[1,2,3],[2,3,4],[3,4,5]],[2,3,2])
\end{spadsrc}
\begin{TeXOutput}
\begin{fricasmath}{5}
\BRACKET{\SYMBOL{particular}=\STRING{"failed"}\COMMA \SYMBOL{basis}=\BRACKET{%
\BRACKET{1\COMMA -{2}\COMMA 1}}}%
\end{fricasmath}
\end{TeXOutput}
\formatResultType{Record(particular: Union(Vector(Fraction(Integer)), "failed"), basis: List(Vector(Fraction(Integer))))}
\end{xtc}

When you want to solve a system of homogeneous equations (that is,
a system where the numbers on the right-hand sides of the
\index{nullspace}
equations are all zero) in the matrix form you can omit the second
argument and use the \spadfun{nullSpace} operation.

\begin{xtc}
\begin{xtccomment}
This computes the solutions of the following system of equations:
$$
\begin{array}{rcrcrcr}
  x&+&2 y&+&3 z&=&0  \\
2 x&+&3 y&+&4 z&=&0  \\
3 x&+&4 y&+&5 z&=&0
\end{array}
$$
The result is given as a list of vectors and
these vectors form a basis for the solution space.
\end{xtccomment}
\begin{spadsrc}
nullSpace([[1,2,3],[2,3,4],[3,4,5]])
\end{spadsrc}
\begin{TeXOutput}
\begin{fricasmath}{6}
\BRACKET{\BRACKET{1\COMMA -{2}\COMMA 1}}%
\end{fricasmath}
\end{TeXOutput}
\formatResultType{List(Vector(Integer))}
\end{xtc}

% *********************************************************************
\head{subsection}{Solution of a Single Polynomial Equation}{ugxProblemOnePol}
% *********************************************************************

\Language{} can solve polynomial equations producing either approximate
\index{polynomial!root finding}
or exact solutions.
\index{equation!polynomial!solving}
Exact solutions are either members of the ground
field or can be presented symbolically as roots of irreducible polynomials.

\begin{xtc}
\begin{xtccomment}
This returns the one rational root along with an irreducible
polynomial describing the other solutions.
\end{xtccomment}
\begin{spadsrc}
solve(x^3  = 8,x)
\end{spadsrc}
\begin{TeXOutput}
\begin{fricasmath}{1}
\BRACKET{\SYMBOL{x}=2\COMMA \SUPER{\SYMBOL{x}}{2}+2\TIMES \SYMBOL{x}+4=0}%
\end{fricasmath}
\end{TeXOutput}
\formatResultType{List(Equation(Fraction(Polynomial(Integer))))}
\end{xtc}
\begin{xtc}
\begin{xtccomment}
If you want solutions expressed in terms of radicals you would use this
instead.
\index{radical}
\end{xtccomment}
\begin{spadsrc}
radicalSolve(x^3  = 8,x)
\end{spadsrc}
\begin{TeXOutput}
\begin{fricasmath}{2}
\BRACKET{\SYMBOL{x}=-{\sqrt{-{3}}}-{1}\COMMA \SYMBOL{x}=\sqrt{-{3}}-{1}%
\COMMA \SYMBOL{x}=2}%
\end{fricasmath}
\end{TeXOutput}
\formatResultType{List(Equation(Expression(Integer)))}
\end{xtc}

The \spadfun{solve} command always returns a value but
\spadfun{radicalSolve} returns only the solutions that it is
able to express in terms of radicals.
\index{radical}

If the polynomial equation has rational coefficients
you can ask for approximations to its real roots by calling
solve with a second argument that specifies the ``precision''
\index{precision}
$\epsilon$.
This means that each approximation will be within
$\pm\epsilon$ of the actual
result.

\begin{xtc}
\begin{xtccomment}
Notice that the type of second argument controls the type of the result.
\end{xtccomment}
\begin{spadsrc}
solve(x^4 - 10*x^3 + 35*x^2 - 50*x + 25,.0001)
\end{spadsrc}
\begin{TeXOutput}
\begin{fricasmath}{3}
\BRACKET{\SYMBOL{x}=\STRING{3.618011474609375}\COMMA \SYMBOL{x}=\STRING{%
1.381988525390625}}%
\end{fricasmath}
\end{TeXOutput}
\formatResultType{List(Equation(Polynomial(Float)))}
\end{xtc}
\begin{xtc}
\begin{xtccomment}
If you give a floating-point precision you get a floating-point result;
if you give the precision as a rational number you get a rational result.
\end{xtccomment}
\begin{spadsrc}
solve(x^3-2,1/1000)
\end{spadsrc}
\begin{TeXOutput}
\begin{fricasmath}{4}
\BRACKET{\SYMBOL{x}=\frac{2581}{2048}}%
\end{fricasmath}
\end{TeXOutput}
\formatResultType{List(Equation(Polynomial(Fraction(Integer))))}
\end{xtc}
\begin{xtc}
\begin{xtccomment}
If you want approximate complex results you should use the
\index{approximation}
command \spadfun{complexSolve} that takes the same precision argument
$\epsilon$.
\end{xtccomment}
\begin{spadsrc}
complexSolve(x^3-2,.0001)
\end{spadsrc}
\begin{TeXOutput}
\begin{fricasmath}{5}
\BRACKET{\SYMBOL{x}=\STRING{1.259921049815602600574493408203125}\COMMA %
\SYMBOL{x}=-{\STRING{0.62996052473711410282}}-{\STRING{1.0911236358806490898}%
\TIMES \ImaginaryI }\COMMA \SYMBOL{x}=-{\STRING{0.62996052473711410282}}+%
\STRING{1.091123635880649089813232421875}\TIMES \ImaginaryI }%
\end{fricasmath}
\end{TeXOutput}
\formatResultType{List(Equation(Polynomial(Complex(Float))))}
\end{xtc}
\begin{xtc}
\begin{xtccomment}
Each approximation will be within
$\pm\epsilon$ of the actual result
in each of the real and imaginary parts.
\end{xtccomment}
\begin{spadsrc}
complexSolve(x^2-2*%i+1,1/100)
\end{spadsrc}
\begin{TeXOutput}
\begin{fricasmath}{6}
\BRACKET{\SYMBOL{x}=-{\frac{%
29413428673197503666 55494440259707227933 20109632199}{%
37414441915671114706 01433171753684530319 18731001856}}-{\frac{10670475}{%
8388608}\TIMES \ImaginaryI }\COMMA \SYMBOL{x}=\frac{%
29413428673197503666 55494440259707227933 20109632199}{%
37414441915671114706 01433171753684530319 18731001856}+\frac{10670475}{%
8388608}\TIMES \ImaginaryI }%
\end{fricasmath}
\end{TeXOutput}
\formatResultType{List(Equation(Polynomial(Complex(Fraction(Integer)))))}
\end{xtc}

Note that if you omit the \spadop{=} from the first argument
\Language{} generates an equation by equating the first argument to zero.
Also, when only one variable is present in the equation, you
do not need to specify the variable to be solved for, that is,
you can omit the second argument.

\begin{xtc}
\begin{xtccomment}
\Language{} can also solve equations involving rational functions.
Solutions where the denominator vanishes are discarded.
\end{xtccomment}
\begin{spadsrc}
radicalSolve(1/x^3 + 1/x^2 + 1/x = 0,x)
\end{spadsrc}
\begin{TeXOutput}
\begin{fricasmath}{7}
\BRACKET{\SYMBOL{x}=\frac{-{\sqrt{-{3}}}-{1}}{2}\COMMA \SYMBOL{x}=\frac{\sqrt%
{-{3}}-{1}}{2}}%
\end{fricasmath}
\end{TeXOutput}
\formatResultType{List(Equation(Expression(Integer)))}
\end{xtc}

% *********************************************************************
\head{subsection}{Solution of Systems of Polynomial Equations}{ugxProblemPolSys}
% *********************************************************************

Given a system of equations of rational functions with exact coefficients:
\index{equation!polynomial!solving}
\begin{displaymath}
\begin{array}{c}
p_1(x_1, \ldots, x_n) \\ \vdots \\ p_m(x_1,\ldots,x_n)
\end{array}
\end{displaymath}
\Language{} can find
numeric or symbolic solutions.
The system is first split into irreducible components, then for
each component, a triangular system of equations is found that reduces
the problem to sequential solution of univariate polynomials resulting
from substitution of partial solutions from the previous stage.
\begin{displaymath}
\begin{array}{c}
q_1(x_1, \ldots, x_n) \\ \vdots \\ q_m(x_n)
\end{array}
\end{displaymath}

Symbolic solutions can be presented using ``implicit'' algebraic numbers
defined as roots of irreducible polynomials or in terms of radicals.
\Language{} can also find approximations to the real or complex roots
of a system of polynomial equations to any user-specified accuracy.

The operation \spadfun{solve} for systems is used in a way similar
to \spadfun{solve} for single equations.
Instead of a polynomial equation, one has to give a list of
equations and instead of a single variable to solve for, a list of
variables.
For solutions of single equations see \spadref{ugxProblemOnePol}.

%
\begin{xtc}
\begin{xtccomment}
Use the operation \spadfun{solve} if you want implicitly presented
solutions.
\end{xtccomment}
\begin{spadsrc}
solve([3*x^3 + y + 1,y^2 -4],[x,y])
\end{spadsrc}
\begin{TeXOutput}
\begin{fricasmath}{1}
\BRACKET{\BRACKET{\SYMBOL{x}=-{1}\COMMA \SYMBOL{y}=2}\COMMA \BRACKET{\SUPER{%
\SYMBOL{x}}{2}-{\SYMBOL{x}}+1=0\COMMA \SYMBOL{y}=2}\COMMA \BRACKET{3\TIMES %
\SUPER{\SYMBOL{x}}{3}-{1}=0\COMMA \SYMBOL{y}=-{2}}}%
\end{fricasmath}
\end{TeXOutput}
\formatResultType{List(List(Equation(Fraction(Polynomial(Integer)))))}
\end{xtc}
\begin{xtc}
\begin{xtccomment}
\end{xtccomment}
\begin{spadsrc}
solve([x = y^2-19,y = z^2+x+3,z = 3*x],[x,y,z])
\end{spadsrc}
\begin{TeXOutput}
\begin{fricasmath}{2}
\BRACKET{\BRACKET{\SYMBOL{x}=\frac{\SYMBOL{z}}{3}\COMMA \SYMBOL{y}=\frac{3%
\TIMES \SUPER{\SYMBOL{z}}{2}+\SYMBOL{z}+9}{3}\COMMA 9\TIMES \SUPER{\SYMBOL{z}%
}{4}+6\TIMES \SUPER{\SYMBOL{z}}{3}+55\TIMES \SUPER{\SYMBOL{z}}{2}+15\TIMES %
\SYMBOL{z}-{90}=0}}%
\end{fricasmath}
\end{TeXOutput}
\formatResultType{List(List(Equation(Fraction(Polynomial(Integer)))))}
\end{xtc}
\begin{xtc}
\begin{xtccomment}
Use \spadfun{radicalSolve} if you want your solutions expressed
in terms of radicals.
\end{xtccomment}
\begin{spadsrc}
radicalSolve([3*x^3 + y + 1,y^2 -4],[x,y])
\end{spadsrc}
\begin{TeXOutput}
\begin{fricasmath}{3}
\BRACKET{\BRACKET{\SYMBOL{x}=\frac{\sqrt{-{3}}+1}{2}\COMMA \SYMBOL{y}=2}%
\COMMA \BRACKET{\SYMBOL{x}=\frac{-{\sqrt{-{3}}}+1}{2}\COMMA \SYMBOL{y}=2}%
\COMMA \BRACKET{\SYMBOL{x}=\frac{-{\sqrt{-{1}}\TIMES \sqrt{3}}-{1}}{2\TIMES %
\nthroot{3}{3}}\COMMA \SYMBOL{y}=-{2}}\COMMA \BRACKET{\SYMBOL{x}=\frac{\sqrt{%
-{1}}\TIMES \sqrt{3}-{1}}{2\TIMES \nthroot{3}{3}}\COMMA \SYMBOL{y}=-{2}}%
\COMMA \BRACKET{\SYMBOL{x}=\frac{1}{\nthroot{3}{3}}\COMMA \SYMBOL{y}=-{2}}%
\COMMA \BRACKET{\SYMBOL{x}=-{1}\COMMA \SYMBOL{y}=2}}%
\end{fricasmath}
\end{TeXOutput}
\formatResultType{List(List(Equation(Expression(Integer))))}
\end{xtc}

To get numeric solutions you only need to give the list of
equations and the precision desired.
The list of variables would be redundant information since there
can be no parameters for the numerical solver.

\begin{xtc}
\begin{xtccomment}
If the precision is expressed as a floating-point number you get
results expressed as floats.
\end{xtccomment}
\begin{spadsrc}
solve([x^2*y - 1,x*y^2 - 2],.01)
\end{spadsrc}
\begin{TeXOutput}
\begin{fricasmath}{4}
\BRACKET{\BRACKET{\SYMBOL{y}=\STRING{1.5874011516571044921875}\COMMA \SYMBOL{%
x}=\STRING{0.79370057582855224609375}}}%
\end{fricasmath}
\end{TeXOutput}
\formatResultType{List(List(Equation(Polynomial(Float))))}
\end{xtc}
\begin{xtc}
\begin{xtccomment}
To get complex numeric solutions, use the operation \spadfun{complexSolve},
which takes the same arguments as in the real case.
\end{xtccomment}
\begin{spadsrc}
complexSolve([x^2*y - 1,x*y^2 - 2],1/1000)
\end{spadsrc}
\begin{TeXOutput}
\begin{fricasmath}{5}
\BRACKET{\BRACKET{\SYMBOL{y}=\frac{6981463658331}{4398046511104}\COMMA %
\SYMBOL{x}=\frac{6981463658331}{8796093022208}}\COMMA \BRACKET{\SYMBOL{y}=-{%
\frac{92799569462155921798 03150679423538973037 814343717}{%
11692013098647223345 62947866173026415724 7460343808}}-{\frac{3023062441857}{%
2199023255552}\TIMES \ImaginaryI }\COMMA \SYMBOL{x}=-{\frac{%
92799569462155921798 03150679423538973037 814343717}{%
23384026197294446691 25895732346052831449 4920687616}}-{\frac{3023062441857}{%
4398046511104}\TIMES \ImaginaryI }}\COMMA \BRACKET{\SYMBOL{y}=-{\frac{%
92799569462155921798 03150679423538973037 814343717}{%
11692013098647223345 62947866173026415724 7460343808}}+\frac{3023062441857}{%
2199023255552}\TIMES \ImaginaryI \COMMA \SYMBOL{x}=-{\frac{%
92799569462155921798 03150679423538973037 814343717}{%
23384026197294446691 25895732346052831449 4920687616}}+\frac{3023062441857}{%
4398046511104}\TIMES \ImaginaryI }}%
\end{fricasmath}
\end{TeXOutput}
\formatResultType{List(List(Equation(Polynomial(Complex(Fraction(Integer))))))}
\end{xtc}
\begin{xtc}
\begin{xtccomment}
It is also possible to solve systems of equations in rational functions
over the rational numbers.
Note that \spad{[x = 0.0, a = 0.0]} is not returned as a solution since
the denominator vanishes there.
\end{xtccomment}
\begin{spadsrc}
solve([x^2/a = a,a = a*x],.001)
\end{spadsrc}
\begin{TeXOutput}
\begin{fricasmath}{6}
\BRACKET{\BRACKET{\SYMBOL{x}=\STRING{1.0}\COMMA \SYMBOL{a}=-{\STRING{1.0}}}%
\COMMA \BRACKET{\SYMBOL{x}=\STRING{1.0}\COMMA \SYMBOL{a}=\STRING{1.0}}}%
\end{fricasmath}
\end{TeXOutput}
\formatResultType{List(List(Equation(Polynomial(Float))))}
\end{xtc}
\begin{xtc}
\begin{xtccomment}
When solving equations with
denominators, all solutions where the denominator vanishes are
discarded.
\end{xtccomment}
\begin{spadsrc}
radicalSolve([x^2/a + a + y^3 - 1,a*y + a + 1],[x,y])
\end{spadsrc}
\begin{TeXOutput}
\begin{fricasmath}{7}
\BRACKET{\BRACKET{\SYMBOL{x}=-{\sqrt{\frac{-{\SUPER{\SYMBOL{a}}{4}}+2\TIMES %
\SUPER{\SYMBOL{a}}{3}+3\TIMES \SUPER{\SYMBOL{a}}{2}+3\TIMES \SYMBOL{a}+1}{%
\SUPER{\SYMBOL{a}}{2}}}}\COMMA \SYMBOL{y}=\frac{-{\SYMBOL{a}}-{1}}{\SYMBOL{a}%
}}\COMMA \BRACKET{\SYMBOL{x}=\sqrt{\frac{-{\SUPER{\SYMBOL{a}}{4}}+2\TIMES %
\SUPER{\SYMBOL{a}}{3}+3\TIMES \SUPER{\SYMBOL{a}}{2}+3\TIMES \SYMBOL{a}+1}{%
\SUPER{\SYMBOL{a}}{2}}}\COMMA \SYMBOL{y}=\frac{-{\SYMBOL{a}}-{1}}{\SYMBOL{a}}%
}}%
\end{fricasmath}
\end{TeXOutput}
\formatResultType{List(List(Equation(Expression(Integer))))}
\end{xtc}

% *********************************************************************
\head{section}{Limits}{ugProblemLimits}
% *********************************************************************
%
To compute a limit, you must specify a functional expression,
\index{limit}
a variable, and a limiting value for that variable.
If you do not specify a direction, \Language{} attempts to
compute a two-sided limit.

\begin{xtc}
\begin{xtccomment}
Issue this to compute the limit
\begin{displaymath}
\lim_{x \rightarrow 1}\frac{\displaystyle x^2 - 3x + 2}{\displaystyle x^2 - 1}.
\end{displaymath}
\end{xtccomment}
\begin{spadsrc}
limit((x^2 - 3*x + 2)/(x^2 - 1),x = 1)
\end{spadsrc}
\begin{TeXOutput}
\begin{fricasmath}{1}
-{\frac{1}{2}}%
\end{fricasmath}
\end{TeXOutput}
\formatResultType{Union(OrderedCompletion(Fraction(Polynomial(Integer))), ...)}
\end{xtc}
% answer := -1/2

Sometimes the limit when approached from the left is different from the
limit from the right and, in this case, you may wish to ask for a
one-sided limit.
Also,
if you have a function that is only defined on one side of a particular value,
\index{limit!one-sided vs. two-sided}
you can compute a one-sided limit.

\begin{xtc}
\begin{xtccomment}
The function \spad{log(x)} is real only to the right of zero,
that is, for \spad{x > 0}.
Thus, when computing limits of functions involving \spad{log(x)},
you probably want a ``right-hand'' limit.
\end{xtccomment}
\begin{spadsrc}
limit(x * log(x),x = 0,"right")
\end{spadsrc}
\begin{TeXOutput}
\begin{fricasmath}{2}
0%
\end{fricasmath}
\end{TeXOutput}
\formatResultType{Union(OrderedCompletion(Expression(Integer)), ...)}
\end{xtc}
% answer := 0
\begin{xtc}
\begin{xtccomment}
When you do not specify \spad{"right"} or \spad{"left"} as the
optional fourth argument, \spadfun{limit} tries to compute a
two-sided limit.
Here the limit from the left does not exist, as \Language{}
indicates when you try to take a two-sided limit.
\end{xtccomment}
\begin{spadsrc}
limit(sin(1/x)*exp(1/x), x=0)
\end{spadsrc}
\begin{TeXOutput}
\begin{fricasmath}{3}
\BRACKET{\SYMBOL{leftHandLimit}=0\COMMA \SYMBOL{rightHandLimit}=\STRING{%
"failed"}}%
\end{fricasmath}
\end{TeXOutput}
\formatResultType{Union(Record(leftHandLimit: Union(OrderedCompletion(Expression(Integer)), "failed"), rightHandLimit: Union(OrderedCompletion(Expression(Integer)), "failed")), ...)}
\end{xtc}
% answer := [left = "failed",right = 0]

A function can be defined on both sides of a particular value, but
tend to different limits as its variable approaches that value from the
left and from the right.
We can construct an example of this as follows:
Since
$\sqrt{y^2}$
is simply the absolute value of \spad{y},
the function
$\sqrt{y^2} / y$
is simply the sign (\spad{+1} or \spad{-1}) of the nonzero
real number \spad{y}.
Therefore,
$\sqrt{y^2} / y = -1$
for \spad{y < 0} and
$\sqrt{y^2} / y = +1$
for \spad{y > 0}.
\begin{xtc}
\begin{xtccomment}
This is what happens when we take the limit at \spad{y = 0}.
The answer returned by \Language{} gives both a
``left-hand'' and a ``right-hand'' limit.
\end{xtccomment}
\begin{spadsrc}
limit(sqrt(y^2)/y,y = 0)
\end{spadsrc}
\begin{TeXOutput}
\begin{fricasmath}{4}
\BRACKET{\SYMBOL{leftHandLimit}=-{1}\COMMA \SYMBOL{rightHandLimit}=1}%
\end{fricasmath}
\end{TeXOutput}
\formatResultType{Union(Record(leftHandLimit: Union(OrderedCompletion(Expression(Integer)), "failed"), rightHandLimit: Union(OrderedCompletion(Expression(Integer)), "failed")), ...)}
\end{xtc}
% answer := [left = -1,right = 1]
\begin{xtc}
\begin{xtccomment}
Here is another example, this time using a more complicated function.
\end{xtccomment}
\begin{spadsrc}
limit(sqrt(1 - cos(t))/t,t = 0)
\end{spadsrc}
\begin{TeXOutput}
\begin{fricasmath}{5}
\BRACKET{\SYMBOL{leftHandLimit}=-{\frac{1}{\sqrt{2}}}\COMMA \SYMBOL{%
rightHandLimit}=\frac{1}{\sqrt{2}}}%
\end{fricasmath}
\end{TeXOutput}
\formatResultType{Union(Record(leftHandLimit: Union(OrderedCompletion(Expression(Integer)), "failed"), rightHandLimit: Union(OrderedCompletion(Expression(Integer)), "failed")), ...)}
\end{xtc}
% answer := [left = -sqrt(1/2),right = sqrt(1/2)]

You can compute limits at infinity by passing either
\index{limit!at infinity}
$+\infty$ or $-\infty$
as the third argument of \spadfun{limit}.
\begin{xtc}
\begin{xtccomment}
To do this, use the constants \spad{%plusInfinity} and \spad{%minusInfinity}.
\end{xtccomment}
\begin{spadsrc}
limit(sqrt(3*x^2 + 1)/(5*x),x = %plusInfinity)
\end{spadsrc}
\begin{TeXOutput}
\begin{fricasmath}{6}
\frac{\sqrt{3}}{5}%
\end{fricasmath}
\end{TeXOutput}
\formatResultType{Union(OrderedCompletion(Expression(Integer)), ...)}
\end{xtc}
\begin{xtc}
\begin{xtccomment}
\end{xtccomment}
\begin{spadsrc}
limit(sqrt(3*x^2 + 1)/(5*x),x = %minusInfinity)
\end{spadsrc}
\begin{TeXOutput}
\begin{fricasmath}{7}
-{\frac{\sqrt{3}}{5}}%
\end{fricasmath}
\end{TeXOutput}
\formatResultType{Union(OrderedCompletion(Expression(Integer)), ...)}
\end{xtc}

\begin{xtc}
\begin{xtccomment}
You can take limits of functions with parameters.
\index{limit!of function with parameters}
As you can see, the limit is expressed in terms of the parameters.
\end{xtccomment}
\begin{spadsrc}
limit(sinh(a*x)/tan(b*x),x = 0)
\end{spadsrc}
\begin{TeXOutput}
\begin{fricasmath}{8}
\frac{\SYMBOL{a}}{\SYMBOL{b}}%
\end{fricasmath}
\end{TeXOutput}
\formatResultType{Union(OrderedCompletion(Expression(Integer)), ...)}
\end{xtc}
% answer := a/b

When you use \spadfun{limit}, you are taking the limit of a real
function of a real variable.
\begin{xtc}
\begin{xtccomment}
When you compute this,
\Language{} returns \spad{0} because, as a function of a real variable,
\spad{sin(1/z)} is always between \spad{-1} and \spad{1}, so \spad{z * sin(1/z)}
tends to \spad{0} as \spad{z} tends to \spad{0}.
\end{xtccomment}
\begin{spadsrc}
limit(z * sin(1/z),z = 0)
\end{spadsrc}
\begin{TeXOutput}
\begin{fricasmath}{9}
0%
\end{fricasmath}
\end{TeXOutput}
\formatResultType{Union(OrderedCompletion(Expression(Integer)), ...)}
\end{xtc}
However, as a function of a {\it complex} variable, \spad{sin(1/z)} is badly
\index{limit!real vs. complex}
behaved near \spad{0} (one says that \spad{sin(1/z)} has an
\index{essential singularity}
{\it essential singularity} at \spad{z = 0}).
\index{singularity!essential}
\begin{xtc}
\begin{xtccomment}
When viewed as a function of a complex variable, \spad{z * sin(1/z)}
does not approach any limit as \spad{z} tends to \spad{0} in the complex plane.
\Language{} indicates this when we call \spadfun{complexLimit}.
\end{xtccomment}
\begin{spadsrc}
complexLimit(z * sin(1/z),z = 0)
\end{spadsrc}
\begin{TeXOutput}
\begin{fricasmath}{10}
\STRING{"failed"}%
\end{fricasmath}
\end{TeXOutput}
\formatResultType{Union("failed", ...)}
\end{xtc}

%% This is used in chapter 1
% Here is another example.
% As \spad{x} approaches \spad{0} along the real axis, \spad{exp(-1/x^2)}
% tends to \spad{0}.
% % This works in the newest version of the limit code.  - cjw 11/15/91
% \spadcommand{limit(exp(-1/x^2),x = 0)}
% However, if \spad{x} is allowed to approach \spad{0} along any path in the
% complex plane, the limiting value of \spad{exp(-1/x^2)} depends on the
% path taken because the function has an essential singularity at \spad{x=0}.
% This is reflected in the error message returned by the function.
% \spadcommand{complexLimit(exp(-1/x^2),x = 0)}

You can also take complex limits at infinity, that is, limits of a function of
\spad{z} as \spad{z} approaches infinity on the Riemann sphere.
Use the symbol \spad{%infinity} to denote ``complex infinity.''
\begin{xtc}
\begin{xtccomment}
As above, to compute complex limits rather than real limits, use
\spadfun{complexLimit}.
\end{xtccomment}
\begin{spadsrc}
complexLimit((2 + z)/(1 - z),z = %infinity)
\end{spadsrc}
\begin{TeXOutput}
\begin{fricasmath}{11}
-{1}%
\end{fricasmath}
\end{TeXOutput}
\formatResultType{OnePointCompletion(Fraction(Polynomial(Integer)))}
\end{xtc}
\begin{xtc}
\begin{xtccomment}
In many cases, a limit of a real function of a real variable
exists when the corresponding complex limit does not.
This limit exists.
\end{xtccomment}
\begin{spadsrc}
limit(sin(x)/x,x = %plusInfinity)
\end{spadsrc}
\begin{TeXOutput}
\begin{fricasmath}{12}
0%
\end{fricasmath}
\end{TeXOutput}
\formatResultType{Union(OrderedCompletion(Expression(Integer)), ...)}
\end{xtc}
\begin{xtc}
\begin{xtccomment}
But this limit does not.
\end{xtccomment}
\begin{spadsrc}
complexLimit(sin(x)/x,x = %infinity)
\end{spadsrc}
\begin{TeXOutput}
\begin{fricasmath}{13}
\STRING{"failed"}%
\end{fricasmath}
\end{TeXOutput}
\formatResultType{Union("failed", ...)}
\end{xtc}

% *********************************************************************
\head{section}{Laplace Transforms}{ugProblemLaplace}
% *********************************************************************
%
\Language{} can compute some forward Laplace transforms, mostly
\index{Laplace transform}
of elementary
\index{function!elementary}
functions
\index{transform!Laplace}
not involving logarithms, although some cases of
special functions are handled.
\begin{xtc}
\begin{xtccomment}
To compute the forward Laplace transform of \spad{F(t)} with respect to
\spad{t} and express the result as \spad{f(s)}, issue the command
\spad{laplace(F(t), t, s)}.
\end{xtccomment}
\begin{spadsrc}
laplace(sin(a*t)*cosh(a*t)-cos(a*t)*sinh(a*t), t, s)
\end{spadsrc}
\begin{TeXOutput}
\begin{fricasmath}{1}
\frac{4\TIMES \SUPER{\SYMBOL{a}}{3}}{\SUPER{\SYMBOL{s}}{4}+4\TIMES \SUPER{%
\SYMBOL{a}}{4}}%
\end{fricasmath}
\end{TeXOutput}
\formatResultType{Expression(Integer)}
\end{xtc}
\begin{xtc}
\begin{xtccomment}
Here are some other non-trivial examples.
\end{xtccomment}
\begin{spadsrc}
laplace((exp(a*t) - exp(b*t))/t, t, s)
\end{spadsrc}
\begin{TeXOutput}
\begin{fricasmath}{2}
-{\log{\PAREN{\SYMBOL{s}-{\SYMBOL{a}}}}}+\log{\PAREN{\SYMBOL{s}-{\SYMBOL{b}}}%
}%
\end{fricasmath}
\end{TeXOutput}
\formatResultType{Expression(Integer)}
\end{xtc}
\begin{xtc}
\begin{xtccomment}
\end{xtccomment}
\begin{spadsrc}
laplace(2/t * (1 - cos(a*t)), t, s)
\end{spadsrc}
\begin{TeXOutput}
\begin{fricasmath}{3}
\log{\PAREN{\SUPER{\SYMBOL{s}}{2}+\SUPER{\SYMBOL{a}}{2}}}-{2\TIMES \log{%
\SYMBOL{s}}}%
\end{fricasmath}
\end{TeXOutput}
\formatResultType{Expression(Integer)}
\end{xtc}
\begin{xtc}
\begin{xtccomment}
\end{xtccomment}
\begin{spadsrc}
laplace(exp(-a*t) * sin(b*t) / b^2, t, s)
\end{spadsrc}
\begin{TeXOutput}
\begin{fricasmath}{4}
\frac{1}{\SYMBOL{b}\TIMES \SUPER{\SYMBOL{s}}{2}+2\TIMES \SYMBOL{a}\TIMES %
\SYMBOL{b}\TIMES \SYMBOL{s}+\SUPER{\SYMBOL{b}}{3}+\SUPER{\SYMBOL{a}}{2}%
\TIMES \SYMBOL{b}}%
\end{fricasmath}
\end{TeXOutput}
\formatResultType{Expression(Integer)}
\end{xtc}
\begin{xtc}
\begin{xtccomment}
\end{xtccomment}
\begin{spadsrc}
laplace((cos(a*t) - cos(b*t))/t, t, s)
\end{spadsrc}
\begin{TeXOutput}
\begin{fricasmath}{5}
\frac{\log{\PAREN{\SUPER{\SYMBOL{s}}{2}+\SUPER{\SYMBOL{b}}{2}}}-{\log{\PAREN{%
\SUPER{\SYMBOL{s}}{2}+\SUPER{\SYMBOL{a}}{2}}}}}{2}%
\end{fricasmath}
\end{TeXOutput}
\formatResultType{Expression(Integer)}
\end{xtc}
\begin{xtc}
\begin{xtccomment}
\Language{} also knows about a few special functions.
\end{xtccomment}
\begin{spadsrc}
laplace(exp(a*t+b)*Ei(c*t), t, s)
\end{spadsrc}
\begin{TeXOutput}
\begin{fricasmath}{6}
\frac{\SUPER{\EulerE }{\SYMBOL{b}}\TIMES \log{\PAREN{\frac{\SYMBOL{s}+\SYMBOL%
{c}-{\SYMBOL{a}}}{\SYMBOL{c}}}}}{\SYMBOL{s}-{\SYMBOL{a}}}%
\end{fricasmath}
\end{TeXOutput}
\formatResultType{Expression(Integer)}
\end{xtc}
\begin{xtc}
\begin{xtccomment}
\end{xtccomment}
\begin{spadsrc}
laplace(a*Ci(b*t) + c*Si(d*t), t, s)
\end{spadsrc}
\begin{TeXOutput}
\begin{fricasmath}{7}
\frac{\SYMBOL{a}\TIMES \log{\PAREN{\frac{\SUPER{\SYMBOL{s}}{2}+\SUPER{\SYMBOL%
{b}}{2}}{\SUPER{\SYMBOL{b}}{2}}}}+2\TIMES \SYMBOL{c}\TIMES \arctan{\PAREN{%
\frac{\SYMBOL{d}}{\SYMBOL{s}}}}}{2\TIMES \SYMBOL{s}}%
\end{fricasmath}
\end{TeXOutput}
\formatResultType{Expression(Integer)}
\end{xtc}
\begin{xtc}
\begin{xtccomment}
When \Language{} does not know about a particular transform,
it keeps it as a formal transform in the answer.
\end{xtccomment}
\begin{spadsrc}
laplace(sin(a*t) - a*t*cos(a*t) + exp(t^2), t, s)
\end{spadsrc}
\begin{TeXOutput}
\begin{fricasmath}{8}
\frac{\PAREN{\SUPER{\SYMBOL{s}}{4}+2\TIMES \SUPER{\SYMBOL{a}}{2}\TIMES \SUPER%
{\SYMBOL{s}}{2}+\SUPER{\SYMBOL{a}}{4}}\TIMES \FUN{laplace}\PAREN{\SUPER{%
\EulerE }{\SUPER{\SYMBOL{t}}{2}},\SYMBOL{t},\SYMBOL{s}}+2\TIMES \SUPER{%
\SYMBOL{a}}{3}}{\SUPER{\SYMBOL{s}}{4}+2\TIMES \SUPER{\SYMBOL{a}}{2}\TIMES %
\SUPER{\SYMBOL{s}}{2}+\SUPER{\SYMBOL{a}}{4}}%
\end{fricasmath}
\end{TeXOutput}
\formatResultType{Expression(Integer)}
\end{xtc}

% *********************************************************************
\head{section}{Integration}{ugProblemIntegration}
% *********************************************************************
%
Integration is the reverse process of differentiation, that is,
\index{integration}
an {\it integral} of a function \spad{f} with respect to a variable
\spad{x} is any function \spad{g} such that \spad{D(g,x)}
is equal to \spad{f}.
\begin{xtc}
\begin{xtccomment}
The package \spadtype{FunctionSpaceIntegration} provides the top-level
integration operation, \spadfunFrom{integrate}{FunctionSpaceIntegration},
for integrating real-valued elementary functions.
\exptypeindex{FunctionSpaceIntegration}
\end{xtccomment}
\begin{spadsrc}
integrate(cosh(a*x)*sinh(a*x), x)
\end{spadsrc}
\begin{TeXOutput}
\begin{fricasmath}{1}
\frac{\SUPER{\PAREN{\sinh{\PAREN{\SYMBOL{a}\TIMES \SYMBOL{x}}}}}{2}+\SUPER{%
\PAREN{\cosh{\PAREN{\SYMBOL{a}\TIMES \SYMBOL{x}}}}}{2}}{4\TIMES \SYMBOL{a}}%
\end{fricasmath}
\end{TeXOutput}
\formatResultType{Union(Expression(Integer), ...)}
\end{xtc}
\begin{xtc}
\begin{xtccomment}
Unfortunately, antiderivatives of most functions cannot be expressed in
terms of elementary functions.
\end{xtccomment}
\begin{spadsrc}
integrate(log(1 + sqrt(a * x + b)) / x, x)
\end{spadsrc}
\begin{TeXOutput}
\begin{fricasmath}{2}
\int^{\SYMBOL{x}} \frac{\log{\PAREN{\sqrt{\SYMBOL{b}+\SYMBOL{\%BH}\TIMES %
\SYMBOL{a}}+1}}}{\SYMBOL{\%BH}}\TIMES \SYMBOL{d}\SYMBOL{\%BH}%
\end{fricasmath}
\end{TeXOutput}
\formatResultType{Union(Expression(Integer), ...)}
\end{xtc}
Given an elementary function to integrate, \Language{} returns a formal
integral as above only when it can prove that the integral is not
elementary and not when it cannot determine the integral.
In this rare case it prints a message that it cannot
determine if an elementary integral exists.
%
\begin{xtc}
\begin{xtccomment}
Similar functions may have antiderivatives
\index{antiderivative}
that look quite different because the form of the antiderivative
depends on the sign of a constant that appears in the function.
\end{xtccomment}
\begin{spadsrc}
integrate(1/(x^2 - 2),x)
\end{spadsrc}
\begin{TeXOutput}
\begin{fricasmath}{3}
\frac{\log{\PAREN{\frac{\PAREN{\SUPER{\SYMBOL{x}}{2}+2}\TIMES \sqrt{2}-{4%
\TIMES \SYMBOL{x}}}{\SUPER{\SYMBOL{x}}{2}-{2}}}}}{2\TIMES \sqrt{2}}%
\end{fricasmath}
\end{TeXOutput}
\formatResultType{Union(Expression(Integer), ...)}
\end{xtc}
\begin{xtc}
\begin{xtccomment}
\end{xtccomment}
\begin{spadsrc}
integrate(1/(x^2 + 2),x)
\end{spadsrc}
\begin{TeXOutput}
\begin{fricasmath}{4}
\frac{\arctan{\PAREN{\frac{\SYMBOL{x}\TIMES \sqrt{2}}{2}}}}{\sqrt{2}}%
\end{fricasmath}
\end{TeXOutput}
\formatResultType{Union(Expression(Integer), ...)}
\end{xtc}
%
If the integrand contains parameters, then there may be several possible
antiderivatives, depending on the signs of expressions of the parameters.
\begin{xtc}
\begin{xtccomment}
In this case \Language{} returns a list of answers that cover all
the possible cases.
Here you
use the answer involving the square root of \spad{a} when \spad{a > 0} and
\index{integration!result as list of real functions}
the answer involving the square root of \spad{-a} when \spad{a < 0}.
\end{xtccomment}
\begin{spadsrc}
integrate(x^2 / (x^4 - a^2), x)
\end{spadsrc}
\begin{TeXOutput}
\begin{fricasmath}{5}
\BRACKET{\frac{\log{\PAREN{\frac{\PAREN{\SUPER{\SYMBOL{x}}{2}+\SYMBOL{a}}%
\TIMES \sqrt{\SYMBOL{a}}-{2\TIMES \SYMBOL{a}\TIMES \SYMBOL{x}}}{\SUPER{%
\SYMBOL{x}}{2}-{\SYMBOL{a}}}}}+2\TIMES \arctan{\PAREN{\frac{\SYMBOL{x}\TIMES %
\sqrt{\SYMBOL{a}}}{\SYMBOL{a}}}}}{4\TIMES \sqrt{\SYMBOL{a}}}\COMMA \frac{\log%
{\PAREN{\frac{\PAREN{\SUPER{\SYMBOL{x}}{2}-{\SYMBOL{a}}}\TIMES \sqrt{-{%
\SYMBOL{a}}}+2\TIMES \SYMBOL{a}\TIMES \SYMBOL{x}}{\SUPER{\SYMBOL{x}}{2}+%
\SYMBOL{a}}}}-{2\TIMES \arctan{\PAREN{\frac{\SYMBOL{x}\TIMES \sqrt{-{\SYMBOL{%
a}}}}{\SYMBOL{a}}}}}}{4\TIMES \sqrt{-{\SYMBOL{a}}}}}%
\end{fricasmath}
\end{TeXOutput}
\formatResultType{Union(List(Expression(Integer)), ...)}
\end{xtc}

If the parameters and the variables of integration can be complex
numbers rather than real, then the notion of sign is not defined.
In this case all the possible answers can be expressed as one
complex function.
To get that function, rather than a list of real functions, use
\spadfunFrom{complexIntegrate}{FunctionSpaceComplexIntegration},
which is provided by the package
\index{integration!result as a complex functions}
\spadtype{FunctionSpaceComplexIntegration}.
\exptypeindex{FunctionSpaceComplexIntegration}

\begin{xtc}
\begin{xtccomment}
This operation is used for
integrating complex-valued elementary functions.
\end{xtccomment}
\begin{spadsrc}
complexIntegrate(x^2 / (x^4 - a^2), x)
\end{spadsrc}
\begin{TeXOutput}
\begin{fricasmath}{6}
\frac{-{\sqrt{\frac{1}{4\TIMES \SYMBOL{a}}}\TIMES \log{\PAREN{2\TIMES \SYMBOL%
{a}\TIMES \sqrt{\frac{1}{4\TIMES \SYMBOL{a}}}+\SYMBOL{x}}}}+\sqrt{-{\frac{1}{%
4\TIMES \SYMBOL{a}}}}\TIMES \log{\PAREN{2\TIMES \SYMBOL{a}\TIMES \sqrt{-{%
\frac{1}{4\TIMES \SYMBOL{a}}}}+\SYMBOL{x}}}-{\sqrt{-{\frac{1}{4\TIMES \SYMBOL%
{a}}}}\TIMES \log{\PAREN{-{2\TIMES \SYMBOL{a}\TIMES \sqrt{-{\frac{1}{4\TIMES %
\SYMBOL{a}}}}}+\SYMBOL{x}}}}+\sqrt{\frac{1}{4\TIMES \SYMBOL{a}}}\TIMES \log{%
\PAREN{-{2\TIMES \SYMBOL{a}\TIMES \sqrt{\frac{1}{4\TIMES \SYMBOL{a}}}}+%
\SYMBOL{x}}}}{2}%
\end{fricasmath}
\end{TeXOutput}
\formatResultType{Expression(Integer)}
\end{xtc}
\begin{xtc}
\begin{xtccomment}
As with the real case,
antiderivatives for most complex-valued functions cannot be expressed
in terms of elementary functions.
\end{xtccomment}
\begin{spadsrc}
complexIntegrate(log(1 + sqrt(a * x + b)) / x, x)
\end{spadsrc}
\begin{TeXOutput}
\begin{fricasmath}{7}
\int^{\SYMBOL{x}} \frac{\log{\PAREN{\sqrt{\SYMBOL{b}+\SYMBOL{\%BH}\TIMES %
\SYMBOL{a}}+1}}}{\SYMBOL{\%BH}}\TIMES \SYMBOL{d}\SYMBOL{\%BH}%
\end{fricasmath}
\end{TeXOutput}
\formatResultType{Expression(Integer)}
\end{xtc}

Sometimes \spadfun{integrate} can involve symbolic algebraic numbers
such as those returned by \spadfunFrom{rootOf}{Expression}.
To see how to work with these strange generated symbols (such as
\spad{%%a0}), see \spadref{ugxProblemSymRootAll}.

Definite integration is the process of computing the area between
\index{integration!definite}
the \spad{x}-axis and the curve of a function \spad{f(x)}.
The fundamental theorem of calculus states that if \spad{f} is
continuous on an interval \spad{a..b} and if there exists a function \spad{g}
that is differentiable on \spad{a..b} and such that \spad{D(g, x)}
is equal to \spad{f}, then the definite integral of \spad{f}
for \spad{x} in the interval \spad{a..b} is equal to \spad{g(b) - g(a)}.

\begin{xtc}
\begin{xtccomment}
The package \spadtype{RationalFunctionDefiniteIntegration} provides
the top-level definite integration operation,
\spadfunFrom{integrate}{RationalFunctionDefiniteIntegration},
for integrating real-valued rational functions.
\end{xtccomment}
\begin{spadsrc}
integrate((x^4 - 3*x^2 + 6)/(x^6-5*x^4+5*x^2+4), x = 1..2)
\end{spadsrc}
\begin{TeXOutput}
\begin{fricasmath}{8}
\frac{2\TIMES \arctan{8}+2\TIMES \arctan{5}+2\TIMES \arctan{2}+2\TIMES %
\arctan{\PAREN{\frac{1}{2}}}-{\pi }}{2}%
\end{fricasmath}
\end{TeXOutput}
\formatResultType{Union(f1: OrderedCompletion(Expression(Integer)), ...)}
\end{xtc}
\Language{} checks beforehand that the function you are integrating is
defined on the interval \spad{a..b}, and prints an error message if it
finds that this is not case, as in the following example:
\begin{verbatim}
integrate(1/(x^2-2), x = 1..2)

    >> Error detected within library code:
       Pole in path of integration
       You are being returned to the top level
       of the interpreter.
\end{verbatim}
When parameters are present in the function, the function may or may not be
defined on the interval of integration.

\begin{xtc}
\begin{xtccomment}
If this is the case, \Language{} issues a warning that a pole might
lie in the path of integration, and does not compute the integral.
\end{xtccomment}
\begin{spadsrc}
integrate(1/(x^2-a), x = 1..2)
\end{spadsrc}
\begin{TeXOutput}
\begin{fricasmath}{9}
\STRING{"potentialPole"}%
\end{fricasmath}
\end{TeXOutput}
\formatResultType{Union(pole: potentialPole, ...)}
\end{xtc}

If you know that you are using values of the parameter for which
the function has no pole in the interval of integration, use the
string {\tt "noPole"} as a third argument to
\spadfunFrom{integrate}{RationalFunctionDefiniteIntegration}:

%
\begin{xtc}
\begin{xtccomment}
The value here is, of course, incorrect if \spad{sqrt(a)} is between
\spad{1} and \spad{2.}
\end{xtccomment}
\begin{spadsrc}
integrate(1/(x^2-a), x = 1..2, "noPole")
\end{spadsrc}
\begin{TeXOutput}
\begin{fricasmath}{10}
\BRACKET{\frac{-{\log{\PAREN{\frac{\PAREN{-{4\TIMES \SUPER{\SYMBOL{a}}{2}}-{4%
\TIMES \SYMBOL{a}}}\TIMES \sqrt{\SYMBOL{a}}+\SUPER{\SYMBOL{a}}{3}+6\TIMES %
\SUPER{\SYMBOL{a}}{2}+\SYMBOL{a}}{\SUPER{\SYMBOL{a}}{2}-{2\TIMES \SYMBOL{a}}+%
1}}}}+\log{\PAREN{\frac{\PAREN{-{8\TIMES \SUPER{\SYMBOL{a}}{2}}-{32\TIMES %
\SYMBOL{a}}}\TIMES \sqrt{\SYMBOL{a}}+\SUPER{\SYMBOL{a}}{3}+24\TIMES \SUPER{%
\SYMBOL{a}}{2}+16\TIMES \SYMBOL{a}}{\SUPER{\SYMBOL{a}}{2}-{8\TIMES \SYMBOL{a}%
}+16}}}}{4\TIMES \sqrt{\SYMBOL{a}}}\COMMA \frac{-{\arctan{\PAREN{\frac{2%
\TIMES \sqrt{-{\SYMBOL{a}}}}{\SYMBOL{a}}}}}+\arctan{\PAREN{\frac{\sqrt{-{%
\SYMBOL{a}}}}{\SYMBOL{a}}}}}{\sqrt{-{\SYMBOL{a}}}}}%
\end{fricasmath}
\end{TeXOutput}
\formatResultType{Union(f2: List(OrderedCompletion(Expression(Integer))), ...)}
\end{xtc}

% *********************************************************************
\head{section}{Working with Power Series}{ugProblemSeries}
% *********************************************************************
%
\Language{} has very sophisticated facilities for working with power
\index{series}
series.
\index{power series}
Infinite series are represented by a list of the
coefficients that have already been determined, together with a
function for computing the additional coefficients if needed.
%
The system command that determines how many terms of a series is displayed
is \spadsys{)set streams calculate}.
For the purposes of this book, we have used this system command to display
fewer than ten terms.
\syscmdindex{set streams calculate}
Series can be created from expressions, from functions for the
series coefficients, and from applications of operations on
existing series.
The most general function for creating a series is called
\spadfun{series}, although you can also use \spadfun{taylor},
\spadfun{laurent} and \spadfun{puiseux} in situations where you
know what kind of exponents are involved.

For information about solving differential equations in terms of
power series, see \spadref{ugxProblemDEQSeries}.

% *********************************************************************
\head{subsection}{Creation of Power Series}{ugxProblemSeriesCreate}
% *********************************************************************

\begin{xtc}
\begin{xtccomment}
This is the easiest way to create a power series.
This tells \Language{} that \spad{x} is to be treated as a power series,
\index{series!creating}
so functions of \spad{x} are again power series.
\end{xtccomment}
\begin{spadsrc}
x := series 'x 
\end{spadsrc}
\begin{TeXOutput}
\begin{fricasmath}{1}
\SYMBOL{x}%
\end{fricasmath}
\end{TeXOutput}
\formatResultType{UnivariatePuiseuxSeries(Expression(Integer), x, 0)}
\end{xtc}
%
We didn't say anything about the coefficients of the power
series, so the coefficients are general expressions over the integers.
This allows us to introduce denominators, symbolic constants, and other
variables as needed.
\begin{xtc}
\begin{xtccomment}
Here the coefficients are integers (note that the coefficients
are the Fibonacci
\index{Fibonacci numbers}
numbers).
\end{xtccomment}
\begin{spadsrc}
1/(1 - x - x^2) 
\end{spadsrc}
\begin{TeXOutput}
\begin{fricasmath}{2}
1+\SYMBOL{x}+2\TIMES \SUPER{\SYMBOL{x}}{2}+3\TIMES \SUPER{\SYMBOL{x}}{3}+5%
\TIMES \SUPER{\SYMBOL{x}}{4}+8\TIMES \SUPER{\SYMBOL{x}}{5}+13\TIMES \SUPER{%
\SYMBOL{x}}{6}+21\TIMES \SUPER{\SYMBOL{x}}{7}+\FUN{O}\PAREN{\SUPER{\SYMBOL{x}%
}{8}}%
\end{fricasmath}
\end{TeXOutput}
\formatResultType{UnivariatePuiseuxSeries(Expression(Integer), x, 0)}
\end{xtc}
\begin{xtc}
\begin{xtccomment}
This series has coefficients that are rational numbers.
\end{xtccomment}
\begin{spadsrc}
sin(x) 
\end{spadsrc}
\begin{TeXOutput}
\begin{fricasmath}{3}
\SYMBOL{x}-{\frac{1}{6}\TIMES \SUPER{\SYMBOL{x}}{3}}+\frac{1}{120}\TIMES %
\SUPER{\SYMBOL{x}}{5}-{\frac{1}{5040}\TIMES \SUPER{\SYMBOL{x}}{7}}+\FUN{O}%
\PAREN{\SUPER{\SYMBOL{x}}{9}}%
\end{fricasmath}
\end{TeXOutput}
\formatResultType{UnivariatePuiseuxSeries(Expression(Integer), x, 0)}
\end{xtc}
\begin{xtc}
\begin{xtccomment}
When you enter this expression
you introduce the symbolic constants \spad{sin(1)} and \spad{cos(1).}
\end{xtccomment}
\begin{spadsrc}
sin(1 + x) 
\end{spadsrc}
\begin{TeXOutput}
\begin{fricasmath}{4}
\sin{1}+\cos{1}\TIMES \SYMBOL{x}-{\frac{\sin{1}}{2}\TIMES \SUPER{\SYMBOL{x}}{%
2}}-{\frac{\cos{1}}{6}\TIMES \SUPER{\SYMBOL{x}}{3}}+\frac{\sin{1}}{24}\TIMES %
\SUPER{\SYMBOL{x}}{4}+\frac{\cos{1}}{120}\TIMES \SUPER{\SYMBOL{x}}{5}-{\frac{%
\sin{1}}{720}\TIMES \SUPER{\SYMBOL{x}}{6}}-{\frac{\cos{1}}{5040}\TIMES \SUPER%
{\SYMBOL{x}}{7}}+\FUN{O}\PAREN{\SUPER{\SYMBOL{x}}{8}}%
\end{fricasmath}
\end{TeXOutput}
\formatResultType{UnivariatePuiseuxSeries(Expression(Integer), x, 0)}
\end{xtc}
\begin{xtc}
\begin{xtccomment}
When you enter the expression
the variable \spad{a} appears in the resulting series expansion.
\end{xtccomment}
\begin{spadsrc}
sin(a * x) 
\end{spadsrc}
\begin{TeXOutput}
\begin{fricasmath}{5}
\SYMBOL{a}\TIMES \SYMBOL{x}-{\frac{\SUPER{\SYMBOL{a}}{3}}{6}\TIMES \SUPER{%
\SYMBOL{x}}{3}}+\frac{\SUPER{\SYMBOL{a}}{5}}{120}\TIMES \SUPER{\SYMBOL{x}}{5}%
-{\frac{\SUPER{\SYMBOL{a}}{7}}{5040}\TIMES \SUPER{\SYMBOL{x}}{7}}+\FUN{O}%
\PAREN{\SUPER{\SYMBOL{x}}{9}}%
\end{fricasmath}
\end{TeXOutput}
\formatResultType{UnivariatePuiseuxSeries(Expression(Integer), x, 0)}
\end{xtc}

\begin{xtc}
\begin{xtccomment}
You can also convert an expression into a series expansion.
This expression creates the series expansion of \spad{1/log(y)}
about \spad{y = 1}.
For details and more examples, see
\spadref{ugxProblemSeriesConversions}.
\end{xtccomment}
\begin{spadsrc}
series(1/log(y),y = 1)
\end{spadsrc}
\begin{TeXOutput}
\begin{fricasmath}{6}
\SUPER{\PAREN{\SYMBOL{y}-{1}}}{-{1}}+\frac{1}{2}-{\frac{1}{12}\TIMES \PAREN{%
\SYMBOL{y}-{1}}}+\frac{1}{24}\TIMES \SUPER{\PAREN{\SYMBOL{y}-{1}}}{2}-{\frac{%
19}{720}\TIMES \SUPER{\PAREN{\SYMBOL{y}-{1}}}{3}}+\frac{3}{160}\TIMES \SUPER{%
\PAREN{\SYMBOL{y}-{1}}}{4}-{\frac{863}{60480}\TIMES \SUPER{\PAREN{\SYMBOL{y}-%
{1}}}{5}}+\frac{275}{24192}\TIMES \SUPER{\PAREN{\SYMBOL{y}-{1}}}{6}+\FUN{O}%
\PAREN{\SUPER{\PAREN{\SYMBOL{y}-{1}}}{7}}%
\end{fricasmath}
\end{TeXOutput}
\formatResultType{UnivariatePuiseuxSeries(Expression(Integer), y, 1)}
\end{xtc}

You can create power series with more general coefficients.
You normally accomplish this via a type declaration (see
\spadref{ugTypesDeclare}).
See \spadref{ugxProblemSeriesFunctions} for some warnings about
working with declared series.

\begin{xtc}
\begin{xtccomment}
We declare that \spad{y} is a one-variable Taylor series
\index{series!Taylor}
(\spadtype{UTS} is the abbreviation for \spadtype{UnivariateTaylorSeries})
in the variable \spad{z} with \spadtype{FLOAT}
(that is, floating-point) coefficients, centered about \spad{0.}
Then, by assignment, we obtain the Taylor expansion of
\spad{exp(z)} with floating-point coefficients.
\exptypeindex{UnivariateTaylorSeries}
\end{xtccomment}
\begin{spadsrc}
y : UTS(FLOAT,'z,0) := exp(z) 
\end{spadsrc}
\begin{TeXOutput}
\begin{fricasmath}{7}
\STRING{1.0}+\SYMBOL{z}+\STRING{0.5}\TIMES \SUPER{\SYMBOL{z}}{2}+\STRING{%
0.16666666666666666667}\TIMES \SUPER{\SYMBOL{z}}{3}+\STRING{%
0.041666666666666666667}\TIMES \SUPER{\SYMBOL{z}}{4}+\STRING{%
0.0083333333333333333334}\TIMES \SUPER{\SYMBOL{z}}{5}+\STRING{%
0.0013888888888888888889}\TIMES \SUPER{\SYMBOL{z}}{6}+\STRING{%
0.0001984126984126984127}\TIMES \SUPER{\SYMBOL{z}}{7}+\FUN{O}\PAREN{\SUPER{%
\SYMBOL{z}}{8}}%
\end{fricasmath}
\end{TeXOutput}
\formatResultType{UnivariateTaylorSeries(Float, z, 0.0)}
\end{xtc}

You can also create a power series by giving an explicit formula
for its \eth{n} coefficient.
For details and more examples, see
\spadref{ugxProblemSeriesFormula}.

\begin{xtc}
\begin{xtccomment}
To create a series about
\spad{w = 0} whose \eth{n} Taylor coefficient
is \spad{1/n!}, you can evaluate this expression.
This is the Taylor expansion of \spad{exp(w)} at \spad{w = 0}.
\end{xtccomment}
\begin{spadsrc}
series(1/factorial(n),n,w = 0)
\end{spadsrc}
\begin{TeXOutput}
\begin{fricasmath}{8}
1+\SYMBOL{w}+\frac{1}{2}\TIMES \SUPER{\SYMBOL{w}}{2}+\frac{1}{6}\TIMES \SUPER%
{\SYMBOL{w}}{3}+\frac{1}{24}\TIMES \SUPER{\SYMBOL{w}}{4}+\frac{1}{120}\TIMES %
\SUPER{\SYMBOL{w}}{5}+\frac{1}{720}\TIMES \SUPER{\SYMBOL{w}}{6}+\frac{1}{5040%
}\TIMES \SUPER{\SYMBOL{w}}{7}+\FUN{O}\PAREN{\SUPER{\SYMBOL{w}}{8}}%
\end{fricasmath}
\end{TeXOutput}
\formatResultType{UnivariatePuiseuxSeries(Expression(Integer), w, 0)}
\end{xtc}
%

% *********************************************************************
\head{subsection}{Coefficients of Power Series}{ugxProblemSeriesCoefficients}
% *********************************************************************

You can extract any coefficient from a power series---even one
that hasn't been computed yet.
This is possible because in \Language{}, infinite series are
represented by a list of the coefficients that have already been
determined, together with a function for computing the additional
coefficients.
(This is known as {\it lazy evaluation}.) When you ask for a
\index{series!lazy evaluation}
coefficient that hasn't yet been computed, \Language{} computes
\index{lazy evaluation}
whatever additional coefficients it needs and then stores them in
the representation of the power series.

\begin{xtc}
\begin{xtccomment}
Here's an example of how to extract the coefficients of a power series.
\index{series!extracting coefficients}
\end{xtccomment}
\begin{spadsrc}
x := series(x) 
\end{spadsrc}
\begin{TeXOutput}
\begin{fricasmath}{1}
\SYMBOL{x}%
\end{fricasmath}
\end{TeXOutput}
\formatResultType{UnivariatePuiseuxSeries(Expression(Integer), x, 0)}
\end{xtc}
\begin{xtc}
\begin{xtccomment}
\end{xtccomment}
\begin{spadsrc}
y := exp(x) * sin(x) 
\end{spadsrc}
\begin{TeXOutput}
\begin{fricasmath}{2}
\SYMBOL{x}+\SUPER{\SYMBOL{x}}{2}+\frac{1}{3}\TIMES \SUPER{\SYMBOL{x}}{3}-{%
\frac{1}{30}\TIMES \SUPER{\SYMBOL{x}}{5}}-{\frac{1}{90}\TIMES \SUPER{\SYMBOL{%
x}}{6}}-{\frac{1}{630}\TIMES \SUPER{\SYMBOL{x}}{7}}+\FUN{O}\PAREN{\SUPER{%
\SYMBOL{x}}{9}}%
\end{fricasmath}
\end{TeXOutput}
\formatResultType{UnivariatePuiseuxSeries(Expression(Integer), x, 0)}
\end{xtc}
\begin{xtc}
\begin{xtccomment}
This coefficient is readily available.
\end{xtccomment}
\begin{spadsrc}
coefficient(y,6) 
\end{spadsrc}
\begin{TeXOutput}
\begin{fricasmath}{3}
-{\frac{1}{90}}%
\end{fricasmath}
\end{TeXOutput}
\formatResultType{Expression(Integer)}
\end{xtc}
\begin{xtc}
\begin{xtccomment}
But let's get the fifteenth coefficient of \spad{y}.
\end{xtccomment}
\begin{spadsrc}
coefficient(y,15) 
\end{spadsrc}
\begin{TeXOutput}
\begin{fricasmath}{4}
-{\frac{1}{10216206000}}%
\end{fricasmath}
\end{TeXOutput}
\formatResultType{Expression(Integer)}
\end{xtc}
\begin{xtc}
\begin{xtccomment}
If you look at \spad{y}
then you see that the coefficients up to order \spad{15}
have all been computed.
\end{xtccomment}
\begin{spadsrc}
y 
\end{spadsrc}
\begin{TeXOutput}
\begin{fricasmath}{5}
\SYMBOL{x}+\SUPER{\SYMBOL{x}}{2}+\frac{1}{3}\TIMES \SUPER{\SYMBOL{x}}{3}-{%
\frac{1}{30}\TIMES \SUPER{\SYMBOL{x}}{5}}-{\frac{1}{90}\TIMES \SUPER{\SYMBOL{%
x}}{6}}-{\frac{1}{630}\TIMES \SUPER{\SYMBOL{x}}{7}}+\FUN{O}\PAREN{\SUPER{%
\SYMBOL{x}}{9}}%
\end{fricasmath}
\end{TeXOutput}
\formatResultType{UnivariatePuiseuxSeries(Expression(Integer), x, 0)}
\end{xtc}

% *********************************************************************
\head{subsection}{Power Series Arithmetic}{ugxProblemSeriesArithmetic}
% *********************************************************************

You can manipulate power series using the usual arithmetic operations
\index{series!arithmetic}
\spadopFrom{+}{UnivariatePuiseuxSeries},
\spadopFrom{-}{UnivariatePuiseuxSeries},
\spadopFrom{*}{UnivariatePuiseuxSeries}, and
\spadopFrom{/}{UnivariatePuiseuxSeries}.
%

\begin{xtc}
\begin{xtccomment}
The results of these operations are also power series.
\end{xtccomment}
\begin{spadsrc}
x := series x 
\end{spadsrc}
\begin{TeXOutput}
\begin{fricasmath}{1}
\SYMBOL{x}%
\end{fricasmath}
\end{TeXOutput}
\formatResultType{UnivariatePuiseuxSeries(Expression(Integer), x, 0)}
\end{xtc}
\begin{xtc}
\begin{xtccomment}
\end{xtccomment}
\begin{spadsrc}
(3 + x) / (1 + 7*x)
\end{spadsrc}
\begin{TeXOutput}
\begin{fricasmath}{2}
3-{20\TIMES \SYMBOL{x}}+140\TIMES \SUPER{\SYMBOL{x}}{2}-{980\TIMES \SUPER{%
\SYMBOL{x}}{3}}+6860\TIMES \SUPER{\SYMBOL{x}}{4}-{48020\TIMES \SUPER{\SYMBOL{%
x}}{5}}+336140\TIMES \SUPER{\SYMBOL{x}}{6}-{2352980\TIMES \SUPER{\SYMBOL{x}}{%
7}}+\FUN{O}\PAREN{\SUPER{\SYMBOL{x}}{8}}%
\end{fricasmath}
\end{TeXOutput}
\formatResultType{UnivariatePuiseuxSeries(Expression(Integer), x, 0)}
\end{xtc}
\begin{xtc}
\begin{xtccomment}
You can also compute \spad{f(x) ^ g(x)}, where \spad{f(x)} and \spad{g(x)}
are two power series.
\end{xtccomment}
\begin{spadsrc}
base := 1 / (1 - x) 
\end{spadsrc}
\begin{TeXOutput}
\begin{fricasmath}{3}
1+\SYMBOL{x}+\SUPER{\SYMBOL{x}}{2}+\SUPER{\SYMBOL{x}}{3}+\SUPER{\SYMBOL{x}}{4%
}+\SUPER{\SYMBOL{x}}{5}+\SUPER{\SYMBOL{x}}{6}+\SUPER{\SYMBOL{x}}{7}+\FUN{O}%
\PAREN{\SUPER{\SYMBOL{x}}{8}}%
\end{fricasmath}
\end{TeXOutput}
\formatResultType{UnivariatePuiseuxSeries(Expression(Integer), x, 0)}
\end{xtc}
%
\begin{xtc}
\begin{xtccomment}
\end{xtccomment}
\begin{spadsrc}
expon := x * base 
\end{spadsrc}
\begin{TeXOutput}
\begin{fricasmath}{4}
\SYMBOL{x}+\SUPER{\SYMBOL{x}}{2}+\SUPER{\SYMBOL{x}}{3}+\SUPER{\SYMBOL{x}}{4}+%
\SUPER{\SYMBOL{x}}{5}+\SUPER{\SYMBOL{x}}{6}+\SUPER{\SYMBOL{x}}{7}+\SUPER{%
\SYMBOL{x}}{8}+\FUN{O}\PAREN{\SUPER{\SYMBOL{x}}{9}}%
\end{fricasmath}
\end{TeXOutput}
\formatResultType{UnivariatePuiseuxSeries(Expression(Integer), x, 0)}
\end{xtc}
%
\begin{xtc}
\begin{xtccomment}
\end{xtccomment}
\begin{spadsrc}
base ^ expon 
\end{spadsrc}
\begin{TeXOutput}
\begin{fricasmath}{5}
1+\SUPER{\SYMBOL{x}}{2}+\frac{3}{2}\TIMES \SUPER{\SYMBOL{x}}{3}+\frac{7}{3}%
\TIMES \SUPER{\SYMBOL{x}}{4}+\frac{43}{12}\TIMES \SUPER{\SYMBOL{x}}{5}+\frac{%
649}{120}\TIMES \SUPER{\SYMBOL{x}}{6}+\frac{241}{30}\TIMES \SUPER{\SYMBOL{x}%
}{7}+\FUN{O}\PAREN{\SUPER{\SYMBOL{x}}{8}}%
\end{fricasmath}
\end{TeXOutput}
\formatResultType{UnivariatePuiseuxSeries(Expression(Integer), x, 0)}
\end{xtc}

% *********************************************************************
\head{subsection}{Functions on Power Series}{ugxProblemSeriesFunctions}
% *********************************************************************

Once you have created a power series, you can apply transcendental
functions
(for example, \spadfun{exp}, \spadfun{log}, \spadfun{sin}, \spadfun{tan},
\spadfun{cosh}, etc.) to it.

%
\begin{xtc}
\begin{xtccomment}
To demonstrate this, we first create the power series
expansion of the rational function
$\frac{\displaystyle x^2}{\displaystyle 1 - 6x + x^2}$
about \spad{x = 0}.
\end{xtccomment}
\begin{spadsrc}
x := series 'x 
\end{spadsrc}
\begin{TeXOutput}
\begin{fricasmath}{1}
\SYMBOL{x}%
\end{fricasmath}
\end{TeXOutput}
\formatResultType{UnivariatePuiseuxSeries(Expression(Integer), x, 0)}
\end{xtc}
%
\begin{xtc}
\begin{xtccomment}
\end{xtccomment}
\begin{spadsrc}
rat := x^2 / (1 - 6*x + x^2) 
\end{spadsrc}
\begin{TeXOutput}
\begin{fricasmath}{2}
\SUPER{\SYMBOL{x}}{2}+6\TIMES \SUPER{\SYMBOL{x}}{3}+35\TIMES \SUPER{\SYMBOL{x%
}}{4}+204\TIMES \SUPER{\SYMBOL{x}}{5}+1189\TIMES \SUPER{\SYMBOL{x}}{6}+6930%
\TIMES \SUPER{\SYMBOL{x}}{7}+40391\TIMES \SUPER{\SYMBOL{x}}{8}+235416\TIMES %
\SUPER{\SYMBOL{x}}{9}+\FUN{O}\PAREN{\SUPER{\SYMBOL{x}}{10}}%
\end{fricasmath}
\end{TeXOutput}
\formatResultType{UnivariatePuiseuxSeries(Expression(Integer), x, 0)}
\end{xtc}
%
%
\begin{xtc}
\begin{xtccomment}
If you want to compute the series expansion of
$\sin\left(\frac{\displaystyle x^2}{\displaystyle 1 - 6x + x^2}\right)$
you simply compute the sine of \spad{rat}.
\end{xtccomment}
\begin{spadsrc}
sin(rat) 
\end{spadsrc}
\begin{TeXOutput}
\begin{fricasmath}{3}
\SUPER{\SYMBOL{x}}{2}+6\TIMES \SUPER{\SYMBOL{x}}{3}+35\TIMES \SUPER{\SYMBOL{x%
}}{4}+204\TIMES \SUPER{\SYMBOL{x}}{5}+\frac{7133}{6}\TIMES \SUPER{\SYMBOL{x}%
}{6}+6927\TIMES \SUPER{\SYMBOL{x}}{7}+\frac{80711}{2}\TIMES \SUPER{\SYMBOL{x}%
}{8}+235068\TIMES \SUPER{\SYMBOL{x}}{9}+\FUN{O}\PAREN{\SUPER{\SYMBOL{x}}{10}}%
\end{fricasmath}
\end{TeXOutput}
\formatResultType{UnivariatePuiseuxSeries(Expression(Integer), x, 0)}
\end{xtc}
%

\beginImportant
\noindent {\bf Warning:}
the type of the coefficients of a power series may
affect the kind of computations that you can do with that series.
This can only happen when you have made a declaration to
specify a series domain with a certain type of coefficient.
\endImportant

\begin{xtc}
\begin{xtccomment}
If you evaluate
then you have declared that \spad{y} is a one variable Taylor series
\index{series!Taylor}
(\spadtype{UTS} is the abbreviation for \spadtype{UnivariateTaylorSeries})
in the variable \spad{y} with \spadtype{FRAC INT}
(that is, fractions of integer) coefficients, centered about \spad{0}.
\end{xtccomment}
\begin{spadsrc}
y : UTS(FRAC INT,y,0) := y 
\end{spadsrc}
\begin{TeXOutput}
\begin{fricasmath}{4}
\SYMBOL{y}%
\end{fricasmath}
\end{TeXOutput}
\formatResultType{UnivariateTaylorSeries(Fraction(Integer), y, 0)}
\end{xtc}
%
\begin{xtc}
\begin{xtccomment}
You can now compute certain power series in \spad{y},
{\it provided} that these series have rational coefficients.
\end{xtccomment}
\begin{spadsrc}
exp(y) 
\end{spadsrc}
\begin{TeXOutput}
\begin{fricasmath}{5}
1+\SYMBOL{y}+\frac{1}{2}\TIMES \SUPER{\SYMBOL{y}}{2}+\frac{1}{6}\TIMES \SUPER%
{\SYMBOL{y}}{3}+\frac{1}{24}\TIMES \SUPER{\SYMBOL{y}}{4}+\frac{1}{120}\TIMES %
\SUPER{\SYMBOL{y}}{5}+\frac{1}{720}\TIMES \SUPER{\SYMBOL{y}}{6}+\frac{1}{5040%
}\TIMES \SUPER{\SYMBOL{y}}{7}+\FUN{O}\PAREN{\SUPER{\SYMBOL{y}}{8}}%
\end{fricasmath}
\end{TeXOutput}
\formatResultType{UnivariateTaylorSeries(Fraction(Integer), y, 0)}
\end{xtc}
%
\begin{xtc}
\begin{xtccomment}
You can get examples of such series
by applying transcendental functions to
series in \spad{y} that have no constant terms.
\end{xtccomment}
\begin{spadsrc}
tan(y^2) 
\end{spadsrc}
\begin{TeXOutput}
\begin{fricasmath}{6}
\SUPER{\SYMBOL{y}}{2}+\frac{1}{3}\TIMES \SUPER{\SYMBOL{y}}{6}+\FUN{O}\PAREN{%
\SUPER{\SYMBOL{y}}{8}}%
\end{fricasmath}
\end{TeXOutput}
\formatResultType{UnivariateTaylorSeries(Fraction(Integer), y, 0)}
\end{xtc}
%
\begin{xtc}
\begin{xtccomment}
\end{xtccomment}
\begin{spadsrc}
cos(y + y^5) 
\end{spadsrc}
\begin{TeXOutput}
\begin{fricasmath}{7}
1-{\frac{1}{2}\TIMES \SUPER{\SYMBOL{y}}{2}}+\frac{1}{24}\TIMES \SUPER{\SYMBOL%
{y}}{4}-{\frac{721}{720}\TIMES \SUPER{\SYMBOL{y}}{6}}+\FUN{O}\PAREN{\SUPER{%
\SYMBOL{y}}{8}}%
\end{fricasmath}
\end{TeXOutput}
\formatResultType{UnivariateTaylorSeries(Fraction(Integer), y, 0)}
\end{xtc}
%
%
\begin{xtc}
\begin{xtccomment}
Similarly, you can compute the logarithm of a power series with rational
coefficients if the constant coefficient is \spad{1.}
\end{xtccomment}
\begin{spadsrc}
log(1 + sin(y)) 
\end{spadsrc}
\begin{TeXOutput}
\begin{fricasmath}{8}
\SYMBOL{y}-{\frac{1}{2}\TIMES \SUPER{\SYMBOL{y}}{2}}+\frac{1}{6}\TIMES \SUPER%
{\SYMBOL{y}}{3}-{\frac{1}{12}\TIMES \SUPER{\SYMBOL{y}}{4}}+\frac{1}{24}%
\TIMES \SUPER{\SYMBOL{y}}{5}-{\frac{1}{45}\TIMES \SUPER{\SYMBOL{y}}{6}}+\frac%
{61}{5040}\TIMES \SUPER{\SYMBOL{y}}{7}+\FUN{O}\PAREN{\SUPER{\SYMBOL{y}}{8}}%
\end{fricasmath}
\end{TeXOutput}
\formatResultType{UnivariateTaylorSeries(Fraction(Integer), y, 0)}
\end{xtc}
%
If you wanted to apply, say, the operation \spadfun{exp} to a power
series with a nonzero constant coefficient $a_0$,
then the constant coefficient of the result would be
$e^{a_0}$, which is {\it not} a rational number.
Therefore, evaluating \spad{exp(2 + tan(y))} would generate an error
message.

If you want to compute the Taylor expansion of \spad{exp(2 + tan(y))}, you must
ensure that the coefficient domain has an operation \spadfun{exp} defined
for it.
An example of such a domain is \spadtype{Expression Integer}, the type
of formal functional expressions over the integers.
%
\begin{xtc}
\begin{xtccomment}
When working with coefficients of this type,
\end{xtccomment}
\begin{spadsrc}
z : UTS(EXPR INT,z,0) := z 
\end{spadsrc}
\begin{TeXOutput}
\begin{fricasmath}{9}
\SYMBOL{z}%
\end{fricasmath}
\end{TeXOutput}
\formatResultType{UnivariateTaylorSeries(Expression(Integer), z, 0)}
\end{xtc}
\begin{xtc}
\begin{xtccomment}
this presents no problems.
\end{xtccomment}
\begin{spadsrc}
exp(2 + tan(z)) 
\end{spadsrc}
\begin{TeXOutput}
\begin{fricasmath}{10}
\SUPER{\EulerE }{2}+\SUPER{\EulerE }{2}\TIMES \SYMBOL{z}+\frac{\SUPER{%
\EulerE }{2}}{2}\TIMES \SUPER{\SYMBOL{z}}{2}+\frac{\SUPER{\EulerE }{2}}{2}%
\TIMES \SUPER{\SYMBOL{z}}{3}+\frac{3\TIMES \SUPER{\EulerE }{2}}{8}\TIMES %
\SUPER{\SYMBOL{z}}{4}+\frac{37\TIMES \SUPER{\EulerE }{2}}{120}\TIMES \SUPER{%
\SYMBOL{z}}{5}+\frac{59\TIMES \SUPER{\EulerE }{2}}{240}\TIMES \SUPER{\SYMBOL{%
z}}{6}+\frac{137\TIMES \SUPER{\EulerE }{2}}{720}\TIMES \SUPER{\SYMBOL{z}}{7}+%
\FUN{O}\PAREN{\SUPER{\SYMBOL{z}}{8}}%
\end{fricasmath}
\end{TeXOutput}
\formatResultType{UnivariateTaylorSeries(Expression(Integer), z, 0)}
\end{xtc}
%
Another way to create Taylor series whose coefficients are expressions over
the integers is to use \spadfun{taylor} which works similarly to
\index{series!Taylor}
\spadfun{series}.
%
\begin{xtc}
\begin{xtccomment}
This is equivalent to the previous computation, except that now we
are using the variable \spad{w} instead of \spad{z}.
\end{xtccomment}
\begin{spadsrc}
w := taylor 'w 
\end{spadsrc}
\begin{TeXOutput}
\begin{fricasmath}{11}
\SYMBOL{w}%
\end{fricasmath}
\end{TeXOutput}
\formatResultType{UnivariateTaylorSeries(Expression(Integer), w, 0)}
\end{xtc}
\begin{xtc}
\begin{xtccomment}
\end{xtccomment}
\begin{spadsrc}
exp(2 + tan(w)) 
\end{spadsrc}
\begin{TeXOutput}
\begin{fricasmath}{12}
\SUPER{\EulerE }{2}+\SUPER{\EulerE }{2}\TIMES \SYMBOL{w}+\frac{\SUPER{%
\EulerE }{2}}{2}\TIMES \SUPER{\SYMBOL{w}}{2}+\frac{\SUPER{\EulerE }{2}}{2}%
\TIMES \SUPER{\SYMBOL{w}}{3}+\frac{3\TIMES \SUPER{\EulerE }{2}}{8}\TIMES %
\SUPER{\SYMBOL{w}}{4}+\frac{37\TIMES \SUPER{\EulerE }{2}}{120}\TIMES \SUPER{%
\SYMBOL{w}}{5}+\frac{59\TIMES \SUPER{\EulerE }{2}}{240}\TIMES \SUPER{\SYMBOL{%
w}}{6}+\frac{137\TIMES \SUPER{\EulerE }{2}}{720}\TIMES \SUPER{\SYMBOL{w}}{7}+%
\FUN{O}\PAREN{\SUPER{\SYMBOL{w}}{8}}%
\end{fricasmath}
\end{TeXOutput}
\formatResultType{UnivariateTaylorSeries(Expression(Integer), w, 0)}
\end{xtc}

% *********************************************************************
\head{subsection}{Converting to Power Series}{ugxProblemSeriesConversions}
% *********************************************************************

The \spadtype{ExpressionToUnivariatePowerSeries} package provides
operations for computing series expansions of functions.
\exptypeindex{ExpressionToUnivariatePowerSeries}

\begin{xtc}
\begin{xtccomment}
Evaluate this
to compute the Taylor expansion of \spad{sin x} about
\index{series!Taylor}
\spad{x = 0}.
The first argument, \spad{sin(x)}, specifies the function whose series
expansion is to be computed and the second argument, \spad{x = 0},
specifies that the series is to be expanded in power of \spad{(x - 0)},
that is, in power of \spad{x}.
\end{xtccomment}
\begin{spadsrc}
taylor(sin(x),x = 0)
\end{spadsrc}
\begin{TeXOutput}
\begin{fricasmath}{1}
\SYMBOL{x}-{\frac{1}{6}\TIMES \SUPER{\SYMBOL{x}}{3}}+\frac{1}{120}\TIMES %
\SUPER{\SYMBOL{x}}{5}-{\frac{1}{5040}\TIMES \SUPER{\SYMBOL{x}}{7}}+\FUN{O}%
\PAREN{\SUPER{\SYMBOL{x}}{8}}%
\end{fricasmath}
\end{TeXOutput}
\formatResultType{UnivariateTaylorSeries(Expression(Integer), x, 0)}
\end{xtc}
\begin{xtc}
\begin{xtccomment}
Here is the Taylor expansion of \spad{sin x} about
$x = \frac{\pi}{6}$:
\end{xtccomment}
\begin{spadsrc}
taylor(sin(x),x = %pi/6)
\end{spadsrc}
\begin{TeXOutput}
\begin{fricasmath}{2}
\frac{1}{2}+\frac{\sqrt{3}}{2}\TIMES \PAREN{\SYMBOL{x}-{\frac{\pi }{6}}}-{%
\frac{1}{4}\TIMES \SUPER{\PAREN{\SYMBOL{x}-{\frac{\pi }{6}}}}{2}}-{\frac{%
\sqrt{3}}{12}\TIMES \SUPER{\PAREN{\SYMBOL{x}-{\frac{\pi }{6}}}}{3}}+\frac{1}{%
48}\TIMES \SUPER{\PAREN{\SYMBOL{x}-{\frac{\pi }{6}}}}{4}+\frac{\sqrt{3}}{240}%
\TIMES \SUPER{\PAREN{\SYMBOL{x}-{\frac{\pi }{6}}}}{5}-{\frac{1}{1440}\TIMES %
\SUPER{\PAREN{\SYMBOL{x}-{\frac{\pi }{6}}}}{6}}-{\frac{\sqrt{3}}{10080}%
\TIMES \SUPER{\PAREN{\SYMBOL{x}-{\frac{\pi }{6}}}}{7}}+\FUN{O}\PAREN{\SUPER{%
\PAREN{\SYMBOL{x}-{\frac{\pi }{6}}}}{8}}%
\end{fricasmath}
\end{TeXOutput}
\formatResultType{UnivariateTaylorSeries(Expression(Integer), x, \%pi/6)}
\end{xtc}

The function to be expanded into a series may have variables other than
\index{series!multiple variables}
the series variable.
%
\begin{xtc}
\begin{xtccomment}
For example, we may expand \spad{tan(x*y)} as a Taylor series in
\spad{x}
\end{xtccomment}
\begin{spadsrc}
taylor(tan(x*y),x = 0)
\end{spadsrc}
\begin{TeXOutput}
\begin{fricasmath}{3}
\SYMBOL{y}\TIMES \SYMBOL{x}+\frac{\SUPER{\SYMBOL{y}}{3}}{3}\TIMES \SUPER{%
\SYMBOL{x}}{3}+\frac{2\TIMES \SUPER{\SYMBOL{y}}{5}}{15}\TIMES \SUPER{\SYMBOL{%
x}}{5}+\frac{17\TIMES \SUPER{\SYMBOL{y}}{7}}{315}\TIMES \SUPER{\SYMBOL{x}}{7}%
+\FUN{O}\PAREN{\SUPER{\SYMBOL{x}}{8}}%
\end{fricasmath}
\end{TeXOutput}
\formatResultType{UnivariateTaylorSeries(Expression(Integer), x, 0)}
\end{xtc}
%
\begin{xtc}
\begin{xtccomment}
or as a Taylor series in \spad{y}.
\end{xtccomment}
\begin{spadsrc}
taylor(tan(x*y),y = 0)
\end{spadsrc}
\begin{TeXOutput}
\begin{fricasmath}{4}
\SYMBOL{x}\TIMES \SYMBOL{y}+\frac{\SUPER{\SYMBOL{x}}{3}}{3}\TIMES \SUPER{%
\SYMBOL{y}}{3}+\frac{2\TIMES \SUPER{\SYMBOL{x}}{5}}{15}\TIMES \SUPER{\SYMBOL{%
y}}{5}+\frac{17\TIMES \SUPER{\SYMBOL{x}}{7}}{315}\TIMES \SUPER{\SYMBOL{y}}{7}%
+\FUN{O}\PAREN{\SUPER{\SYMBOL{y}}{8}}%
\end{fricasmath}
\end{TeXOutput}
\formatResultType{UnivariateTaylorSeries(Expression(Integer), y, 0)}
\end{xtc}
\begin{xtc}
\begin{xtccomment}
A more interesting function is
$\displaystyle\frac{t e^{x t}}{e^t - 1}$.

When we expand this function as a Taylor series in \spad{t}
the \eth{n} order coefficient is the \eth{n} Bernoulli
\index{Bernoulli!polynomial}
polynomial
\index{polynomial!Bernoulli}
divided by \spad{n!}.
\end{xtccomment}
\begin{spadsrc}
bern := taylor(t*exp(x*t)/(exp(t) - 1),t = 0) 
\end{spadsrc}
\begin{TeXOutput}
\begin{fricasmath}{5}
1+\frac{2\TIMES \SYMBOL{x}-{1}}{2}\TIMES \SYMBOL{t}+\frac{6\TIMES \SUPER{%
\SYMBOL{x}}{2}-{6\TIMES \SYMBOL{x}}+1}{12}\TIMES \SUPER{\SYMBOL{t}}{2}+\frac{%
2\TIMES \SUPER{\SYMBOL{x}}{3}-{3\TIMES \SUPER{\SYMBOL{x}}{2}}+\SYMBOL{x}}{12}%
\TIMES \SUPER{\SYMBOL{t}}{3}+\frac{30\TIMES \SUPER{\SYMBOL{x}}{4}-{60\TIMES %
\SUPER{\SYMBOL{x}}{3}}+30\TIMES \SUPER{\SYMBOL{x}}{2}-{1}}{720}\TIMES \SUPER{%
\SYMBOL{t}}{4}+\frac{6\TIMES \SUPER{\SYMBOL{x}}{5}-{15\TIMES \SUPER{\SYMBOL{x%
}}{4}}+10\TIMES \SUPER{\SYMBOL{x}}{3}-{\SYMBOL{x}}}{720}\TIMES \SUPER{\SYMBOL%
{t}}{5}+\frac{42\TIMES \SUPER{\SYMBOL{x}}{6}-{126\TIMES \SUPER{\SYMBOL{x}}{5}%
}+105\TIMES \SUPER{\SYMBOL{x}}{4}-{21\TIMES \SUPER{\SYMBOL{x}}{2}}+1}{30240}%
\TIMES \SUPER{\SYMBOL{t}}{6}+\frac{6\TIMES \SUPER{\SYMBOL{x}}{7}-{21\TIMES %
\SUPER{\SYMBOL{x}}{6}}+21\TIMES \SUPER{\SYMBOL{x}}{5}-{7\TIMES \SUPER{\SYMBOL%
{x}}{3}}+\SYMBOL{x}}{30240}\TIMES \SUPER{\SYMBOL{t}}{7}+\FUN{O}\PAREN{\SUPER{%
\SYMBOL{t}}{8}}%
\end{fricasmath}
\end{TeXOutput}
\formatResultType{UnivariateTaylorSeries(Expression(Integer), t, 0)}
\end{xtc}
\begin{xtc}
\begin{xtccomment}
Therefore, this and the next expression
produce the same result.
\end{xtccomment}
\begin{spadsrc}
factorial(6) * coefficient(bern,6) 
\end{spadsrc}
\begin{TeXOutput}
\begin{fricasmath}{6}
\frac{42\TIMES \SUPER{\SYMBOL{x}}{6}-{126\TIMES \SUPER{\SYMBOL{x}}{5}}+105%
\TIMES \SUPER{\SYMBOL{x}}{4}-{21\TIMES \SUPER{\SYMBOL{x}}{2}}+1}{42}%
\end{fricasmath}
\end{TeXOutput}
\formatResultType{Expression(Integer)}
\end{xtc}
\begin{xtc}
\begin{xtccomment}
\end{xtccomment}
\begin{spadsrc}
bernoulliB(6,x)
\end{spadsrc}
\begin{TeXOutput}
\begin{fricasmath}{7}
\SUPER{\SYMBOL{x}}{6}-{3\TIMES \SUPER{\SYMBOL{x}}{5}}+\frac{5}{2}\TIMES %
\SUPER{\SYMBOL{x}}{4}-{\frac{1}{2}\TIMES \SUPER{\SYMBOL{x}}{2}}+\frac{1}{42}%
\end{fricasmath}
\end{TeXOutput}
\formatResultType{Polynomial(Fraction(Integer))}
\end{xtc}

Technically, a series with terms of negative degree is not considered to
be a Taylor series, but, rather, a
\index{series!Laurent}
{\it Laurent series}.
\index{Laurent series}
If you try to compute a Taylor series expansion of
$\frac{x}{\log x}$
at \spad{x = 1} via \spad{taylor(x/log(x),x = 1)}
you get an error message.
The reason is that the function has a {\it pole} at \spad{x = 1},
meaning that
its series expansion about this point has terms of negative degree.
A series with finitely many terms of negative degree is called a Laurent
series.
\begin{xtc}
\begin{xtccomment}
You get the desired series expansion by issuing this.
\end{xtccomment}
\begin{spadsrc}
laurent(x/log(x),x = 1)
\end{spadsrc}
\begin{TeXOutput}
\begin{fricasmath}{8}
\SUPER{\PAREN{\SYMBOL{x}-{1}}}{-{1}}+\frac{3}{2}+\frac{5}{12}\TIMES \PAREN{%
\SYMBOL{x}-{1}}-{\frac{1}{24}\TIMES \SUPER{\PAREN{\SYMBOL{x}-{1}}}{2}}+\frac{%
11}{720}\TIMES \SUPER{\PAREN{\SYMBOL{x}-{1}}}{3}-{\frac{11}{1440}\TIMES %
\SUPER{\PAREN{\SYMBOL{x}-{1}}}{4}}+\frac{271}{60480}\TIMES \SUPER{\PAREN{%
\SYMBOL{x}-{1}}}{5}-{\frac{13}{4480}\TIMES \SUPER{\PAREN{\SYMBOL{x}-{1}}}{6}}%
+\FUN{O}\PAREN{\SUPER{\PAREN{\SYMBOL{x}-{1}}}{7}}%
\end{fricasmath}
\end{TeXOutput}
\formatResultType{UnivariateLaurentSeries(Expression(Integer), x, 1)}
\end{xtc}

Similarly, a series with terms of fractional degree is neither a Taylor
series nor a Laurent series.
Such a series is called a
\index{series!Puiseux}
{\it Puiseux series}.
\index{Puiseux series}
The expression \spad{laurent(sqrt(sec(x)),x = 3 * %pi/2)}
results in an error message
because the series expansion about this point has terms of fractional degree.
\begin{xtc}
\begin{xtccomment}
However, this command produces what you want.
\end{xtccomment}
\begin{spadsrc}
puiseux(sqrt(sec(x)),x = 3 * %pi/2)
\end{spadsrc}
\begin{TeXOutput}
\begin{fricasmath}{9}
\SUPER{\PAREN{\SYMBOL{x}-{\frac{3\TIMES \pi }{2}}}}{-{\frac{1}{2}}}+\frac{1}{%
12}\TIMES \SUPER{\PAREN{\SYMBOL{x}-{\frac{3\TIMES \pi }{2}}}}{\frac{3}{2}}+%
\FUN{O}\PAREN{\SUPER{\PAREN{\SYMBOL{x}-{\frac{3\TIMES \pi }{2}}}}{\frac{7}{2}%
}}%
\end{fricasmath}
\end{TeXOutput}
\formatResultType{UnivariatePuiseuxSeries(Expression(Integer), x, (3*\%pi)/2)}
\end{xtc}

Finally, consider the case of functions that do not have Puiseux
expansions about certain points.
An example of this is $x^x$ about \spad{x = 0}.
\spad{puiseux(x^x,x=0)}
produces an error message because of the
type of singularity of the function at \spad{x = 0}.
\begin{xtc}
\begin{xtccomment}
The general function \spadfun{series} can be used in this case.
Notice that the series returned is not, strictly speaking, a power series
because of the \spad{log(x)} in the expansion.
\end{xtccomment}
\begin{spadsrc}
series(x^x,x=0)
\end{spadsrc}
\begin{TeXOutput}
\begin{fricasmath}{10}
1+\log{\SYMBOL{x}}\TIMES \SYMBOL{x}+\frac{\SUPER{\PAREN{\log{\SYMBOL{x}}}}{2}%
}{2}\TIMES \SUPER{\SYMBOL{x}}{2}+\frac{\SUPER{\PAREN{\log{\SYMBOL{x}}}}{3}}{6%
}\TIMES \SUPER{\SYMBOL{x}}{3}+\frac{\SUPER{\PAREN{\log{\SYMBOL{x}}}}{4}}{24}%
\TIMES \SUPER{\SYMBOL{x}}{4}+\frac{\SUPER{\PAREN{\log{\SYMBOL{x}}}}{5}}{120}%
\TIMES \SUPER{\SYMBOL{x}}{5}+\frac{\SUPER{\PAREN{\log{\SYMBOL{x}}}}{6}}{720}%
\TIMES \SUPER{\SYMBOL{x}}{6}+\frac{\SUPER{\PAREN{\log{\SYMBOL{x}}}}{7}}{5040}%
\TIMES \SUPER{\SYMBOL{x}}{7}+\FUN{O}\PAREN{\SUPER{\SYMBOL{x}}{8}}%
\end{fricasmath}
\end{TeXOutput}
\formatResultType{GeneralUnivariatePowerSeries(Expression(Integer), x, 0)}
\end{xtc}

\beginImportant
The operation \spadfun{series} returns the most general type of
infinite series.
The user who is not interested in distinguishing
between various types of infinite series may wish to use this operation
exclusively.
\endImportant

% *********************************************************************
\head{subsection}{Power Series from Formulas}{ugxProblemSeriesFormula}
% *********************************************************************

The \spadtype{GenerateUnivariatePowerSeries} package enables you to
\index{series!giving formula for coefficients}
create power series from explicit formulas for their
\eth{n} coefficients.
In what follows, we construct series expansions for certain
transcendental functions by giving formulas for their
coefficients.
You can also compute such series expansions directly simply by
specifying the function and the point about which the series is to
be expanded.
\exptypeindex{GenerateUnivariatePowerSeries}
See \spadref{ugxProblemSeriesConversions} for more information.

Consider the Taylor expansion of $e^x$
\index{series!Taylor}
about \spad{x = 0}:
\begin{displaymath}
\begin{array}{ccl}
e^x &=& \displaystyle 1 + x + \frac{x^2}{2} + \frac{x^3}{6} + \cdots \\ \\
    &=& \displaystyle\sum_{n=0}^\infty \frac{x^n}{n!}
\end{array}
\end{displaymath}
The \eth{n} Taylor coefficient is \spad{1/n!}.
%
\begin{xtc}
\begin{xtccomment}
This is how you create this series in \Language{}.
\end{xtccomment}
\begin{spadsrc}
series(n +-> 1/factorial(n),x = 0)
\end{spadsrc}
\begin{TeXOutput}
\begin{fricasmath}{1}
1+\SYMBOL{x}+\frac{1}{2}\TIMES \SUPER{\SYMBOL{x}}{2}+\frac{1}{6}\TIMES \SUPER%
{\SYMBOL{x}}{3}+\frac{1}{24}\TIMES \SUPER{\SYMBOL{x}}{4}+\frac{1}{120}\TIMES %
\SUPER{\SYMBOL{x}}{5}+\frac{1}{720}\TIMES \SUPER{\SYMBOL{x}}{6}+\frac{1}{5040%
}\TIMES \SUPER{\SYMBOL{x}}{7}+\FUN{O}\PAREN{\SUPER{\SYMBOL{x}}{8}}%
\end{fricasmath}
\end{TeXOutput}
\formatResultType{UnivariatePuiseuxSeries(Expression(Integer), x, 0)}
\end{xtc}

The first argument specifies a formula for the \eth{n}
coefficient by giving a function that maps \spad{n} to
\spad{1/n!}.
The second argument specifies that the series is to be expanded in
powers of \spad{(x - 0)}, that is, in powers of \spad{x}.
Since we did not specify an initial degree, the first term in the
series was the term of degree 0 (the constant term).
Note that the formula was given as an anonymous function.
These are discussed in \spadref{ugUserAnon}.

Consider the Taylor expansion of \spad{log x} about \spad{x = 1}:
\begin{displaymath}
\begin{array}{ccl}
\log(x) &=& \displaystyle (x - 1) - \frac{(x - 1)^2}{2} + \frac{(x - 1)^3}{3} - \cdots \\ \\
        &=& \displaystyle\sum_{n = 1}^\infty (-1)^{n-1} \frac{(x - 1)^n}{n}
\end{array}
\end{displaymath}
If you were to evaluate the expression
\spad{series(n +-> (-1)^(n-1) / n, x = 1)}
you would get an error message because \Language{} would try to
calculate a term of degree \spad{0} and therefore divide by \spad{0.}

\begin{xtc}
\begin{xtccomment}
Instead, evaluate this.
The third argument, \spad{1..}, indicates that only terms of degree
\spad{n = 1, ...} are to be computed.
\end{xtccomment}
\begin{spadsrc}
series(n +-> (-1)^(n-1)/n,x = 1,1..)
\end{spadsrc}
\begin{TeXOutput}
\begin{fricasmath}{2}
\PAREN{\SYMBOL{x}-{1}}-{\frac{1}{2}\TIMES \SUPER{\PAREN{\SYMBOL{x}-{1}}}{2}}+%
\frac{1}{3}\TIMES \SUPER{\PAREN{\SYMBOL{x}-{1}}}{3}-{\frac{1}{4}\TIMES \SUPER%
{\PAREN{\SYMBOL{x}-{1}}}{4}}+\frac{1}{5}\TIMES \SUPER{\PAREN{\SYMBOL{x}-{1}}%
}{5}-{\frac{1}{6}\TIMES \SUPER{\PAREN{\SYMBOL{x}-{1}}}{6}}+\frac{1}{7}\TIMES %
\SUPER{\PAREN{\SYMBOL{x}-{1}}}{7}-{\frac{1}{8}\TIMES \SUPER{\PAREN{\SYMBOL{x}%
-{1}}}{8}}+\FUN{O}\PAREN{\SUPER{\PAREN{\SYMBOL{x}-{1}}}{9}}%
\end{fricasmath}
\end{TeXOutput}
\formatResultType{UnivariatePuiseuxSeries(Expression(Integer), x, 1)}
\end{xtc}
%

Next consider the Taylor expansion of an odd function, say,
\spad{sin(x)}:
\begin{displaymath}
\sin(x) = x - \frac{x^3}{3!} + \frac{x^5}{5!} - \cdots
\end{displaymath}
Here every other coefficient is zero and we would like to give an
explicit formula only for the odd Taylor coefficients.
%
\begin{xtc}
\begin{xtccomment}
This is one way to do it.
The third argument, \spad{1..}, specifies that the first term to be computed
is the term of degree 1.
The fourth argument, \spad{2}, specifies that we
increment by \spad{2} to find the degrees of subsequent terms, that is, the next
term
is of degree \spad{1 + 2}, the next of degree \spad{1 + 2 + 2}, etc.
\end{xtccomment}
\begin{spadsrc}
series(n +-> (-1)^((n-1)/2)/factorial(n),x = 0,1..,2)
\end{spadsrc}
\begin{TeXOutput}
\begin{fricasmath}{3}
\SYMBOL{x}-{\frac{1}{6}\TIMES \SUPER{\SYMBOL{x}}{3}}+\frac{1}{120}\TIMES %
\SUPER{\SYMBOL{x}}{5}-{\frac{1}{5040}\TIMES \SUPER{\SYMBOL{x}}{7}}+\FUN{O}%
\PAREN{\SUPER{\SYMBOL{x}}{9}}%
\end{fricasmath}
\end{TeXOutput}
\formatResultType{UnivariatePuiseuxSeries(Expression(Integer), x, 0)}
\end{xtc}
%

\begin{xtc}
\begin{xtccomment}
The initial degree and the increment do not have to be integers.
For example, this expression produces a series expansion of
$\sin(x^{\frac{1}{3}})$.
\end{xtccomment}
\begin{spadsrc}
series(n +-> (-1)^((3*n-1)/2)/factorial(3*n),x = 0,1/3..,2/3)
\end{spadsrc}
\begin{TeXOutput}
\begin{fricasmath}{4}
\SUPER{\SYMBOL{x}}{\frac{1}{3}}-{\frac{1}{6}\TIMES \SYMBOL{x}}+\frac{1}{120}%
\TIMES \SUPER{\SYMBOL{x}}{\frac{5}{3}}-{\frac{1}{5040}\TIMES \SUPER{\SYMBOL{x%
}}{\frac{7}{3}}}+\FUN{O}\PAREN{\SUPER{\SYMBOL{x}}{3}}%
\end{fricasmath}
\end{TeXOutput}
\formatResultType{UnivariatePuiseuxSeries(Expression(Integer), x, 0)}
\end{xtc}
\begin{xtc}
\begin{xtccomment}
While the increment must be positive, the initial degree may be negative.
This yields the Laurent expansion of \spad{csc(x)} at
\spad{x = 0}.
% bernoulli(numer(n+1)) is necessary because bernoulli takes integer arguments
% It looks disgusting, though.
%
\end{xtccomment}
\begin{spadsrc}
cscx := series(n +-> (-1)^((n-1)/2) * 2 * (2^n-1) * bernoulli(numer(n+1)) / factorial(n+1), x=0, -1..,2) 
\end{spadsrc}
\begin{TeXOutput}
\begin{fricasmath}{5}
\SUPER{\SYMBOL{x}}{-{1}}+\frac{1}{6}\TIMES \SYMBOL{x}+\frac{7}{360}\TIMES %
\SUPER{\SYMBOL{x}}{3}+\frac{31}{15120}\TIMES \SUPER{\SYMBOL{x}}{5}+\FUN{O}%
\PAREN{\SUPER{\SYMBOL{x}}{7}}%
\end{fricasmath}
\end{TeXOutput}
\formatResultType{UnivariatePuiseuxSeries(Expression(Integer), x, 0)}
\end{xtc}
\begin{xtc}
\begin{xtccomment}
Of course, the reciprocal of this power series is the Taylor expansion
of \spad{sin(x)}.
\end{xtccomment}
\begin{spadsrc}
1/cscx 
\end{spadsrc}
\begin{TeXOutput}
\begin{fricasmath}{6}
\SYMBOL{x}-{\frac{1}{6}\TIMES \SUPER{\SYMBOL{x}}{3}}+\frac{1}{120}\TIMES %
\SUPER{\SYMBOL{x}}{5}-{\frac{1}{5040}\TIMES \SUPER{\SYMBOL{x}}{7}}+\FUN{O}%
\PAREN{\SUPER{\SYMBOL{x}}{9}}%
\end{fricasmath}
\end{TeXOutput}
\formatResultType{UnivariatePuiseuxSeries(Expression(Integer), x, 0)}
\end{xtc}
%
\begin{xtc}
\begin{xtccomment}
As a final example,
here is the Taylor expansion of \spad{asin(x)} about \spad{x = 0}.
\end{xtccomment}
\begin{spadsrc}
asinx := series(n +-> binomial(n-1,(n-1)/2)/(n*2^(n-1)),x=0,1..,2) 
\end{spadsrc}
\begin{TeXOutput}
\begin{fricasmath}{7}
\SYMBOL{x}+\frac{1}{6}\TIMES \SUPER{\SYMBOL{x}}{3}+\frac{3}{40}\TIMES \SUPER{%
\SYMBOL{x}}{5}+\frac{5}{112}\TIMES \SUPER{\SYMBOL{x}}{7}+\FUN{O}\PAREN{\SUPER%
{\SYMBOL{x}}{9}}%
\end{fricasmath}
\end{TeXOutput}
\formatResultType{UnivariatePuiseuxSeries(Expression(Integer), x, 0)}
\end{xtc}
\begin{xtc}
\begin{xtccomment}
When we compute the \spad{sin} of this series, we get \spad{x}
(in the sense that all higher terms computed so far are zero).
\end{xtccomment}
\begin{spadsrc}
sin(asinx) 
\end{spadsrc}
\begin{TeXOutput}
\begin{fricasmath}{8}
\SYMBOL{x}+\FUN{O}\PAREN{\SUPER{\SYMBOL{x}}{9}}%
\end{fricasmath}
\end{TeXOutput}
\formatResultType{UnivariatePuiseuxSeries(Expression(Integer), x, 0)}
\end{xtc}

As we discussed in \spadref{ugxProblemSeriesConversions},
you can also use the operations \spadfun{taylor}, \spadfun{laurent} and
\spadfun{puiseux} instead of \spadfun{series} if you know ahead of time
what kind of exponents a series has.
You can't go wrong using \spadfun{series}, though.

% *********************************************************************
\head{subsection}{Substituting Numerical Values in Power Series}{ugxProblemSeriesSubstitute}
% *********************************************************************

Use \spadfunFrom{eval}{UnivariatePowerSeriesCategory}
\index{approximation}
to substitute a numerical value for a variable in
\index{series!numerical approximation}
a power series.
For example, here's a way to obtain numerical approximations of
\spad{%e} from the Taylor series
expansion of \spad{exp(x)}.

\begin{xtc}
\begin{xtccomment}
First you create the desired Taylor expansion.
\end{xtccomment}
\begin{spadsrc}
f := taylor(exp(x)) 
\end{spadsrc}
\begin{TeXOutput}
\begin{fricasmath}{1}
1+\SYMBOL{x}+\frac{1}{2}\TIMES \SUPER{\SYMBOL{x}}{2}+\frac{1}{6}\TIMES \SUPER%
{\SYMBOL{x}}{3}+\frac{1}{24}\TIMES \SUPER{\SYMBOL{x}}{4}+\frac{1}{120}\TIMES %
\SUPER{\SYMBOL{x}}{5}+\frac{1}{720}\TIMES \SUPER{\SYMBOL{x}}{6}+\frac{1}{5040%
}\TIMES \SUPER{\SYMBOL{x}}{7}+\FUN{O}\PAREN{\SUPER{\SYMBOL{x}}{8}}%
\end{fricasmath}
\end{TeXOutput}
\formatResultType{UnivariateTaylorSeries(Expression(Integer), x, 0)}
\end{xtc}
\begin{xtc}
\begin{xtccomment}
Then you evaluate the series at the value \spad{1.0}.
The result is a sequence of the partial sums.
\end{xtccomment}
\begin{spadsrc}
eval(f,1.0) 
\end{spadsrc}
\begin{TeXOutput}
\begin{fricasmath}{2}
\BRACKET{\STRING{1.0}\COMMA \STRING{2.0}\COMMA \STRING{2.5}\COMMA \STRING{%
2.6666666666666666667}\COMMA \STRING{2.7083333333333333333}\COMMA \STRING{%
2.7166666666666666667}\COMMA \STRING{2.7180555555555555556}\COMMA \STRING{...%
}}%
\end{fricasmath}
\end{TeXOutput}
\formatResultType{Stream(Expression(Float))}
\end{xtc}

% *********************************************************************
\head{subsection}{Example: Bernoulli Polynomials and Sums of Powers}{ugxProblemSeriesBernoulli}
% *********************************************************************

\Language{} provides operations for computing definite and
\index{summation!definite}
indefinite sums.
\index{summation!indefinite}

\begin{xtc}
\begin{xtccomment}
You can compute the sum of the first
ten fourth powers by evaluating this.
This creates a list whose entries are
$m^4$ as $m$ ranges from 1
to 10, and then computes the sum of the entries of that list.
\end{xtccomment}
\begin{spadsrc}
reduce(+,[m^4 for m in 1..10])
\end{spadsrc}
\begin{TeXOutput}
\begin{fricasmath}{1}
25333%
\end{fricasmath}
\end{TeXOutput}
\formatResultType{PositiveInteger}
\end{xtc}
\begin{xtc}
\begin{xtccomment}
You can also compute a formula for the sum of the first
$k$ fourth powers, where $k$ is an
unspecified positive integer.
\end{xtccomment}
\begin{spadsrc}
sum4 := sum(m^4, m = 1..k) 
\end{spadsrc}
\begin{TeXOutput}
\begin{fricasmath}{2}
\frac{6\TIMES \SUPER{\SYMBOL{k}}{5}+15\TIMES \SUPER{\SYMBOL{k}}{4}+10\TIMES %
\SUPER{\SYMBOL{k}}{3}-{\SYMBOL{k}}}{30}%
\end{fricasmath}
\end{TeXOutput}
\formatResultType{Fraction(Polynomial(Integer))}
\end{xtc}
\begin{xtc}
\begin{xtccomment}
This formula is valid for any positive integer $k$.
For instance, if we replace $k$ by 10,
\index{summation!definite}
we obtain the number we computed earlier.
\end{xtccomment}
\begin{spadsrc}
eval(sum4, k = 10) 
\end{spadsrc}
\begin{TeXOutput}
\begin{fricasmath}{3}
25333%
\end{fricasmath}
\end{TeXOutput}
\formatResultType{Fraction(Polynomial(Integer))}
\end{xtc}

You can compute a formula for the sum of the first
$k$ \eth{n} powers in a similar fashion.
Just replace the \spad{4} in the definition of \userfun{sum4} by
any expression not involving $k$.
\Language{} computes these formulas using Bernoulli polynomials;
\index{Bernoulli!polynomial}
we
\index{polynomial!Bernoulli}
use the rest of this section to describe this method.

%
\begin{xtc}
\begin{xtccomment}
First consider this function of \spad{t} and \spad{x}.
\end{xtccomment}
\begin{spadsrc}
f := t*exp(x*t) / (exp(t) - 1) 
\end{spadsrc}
\begin{TeXOutput}
\begin{fricasmath}{4}
\frac{\SYMBOL{t}\TIMES \SUPER{\EulerE }{\SYMBOL{t}\TIMES \SYMBOL{x}}}{\SUPER{%
\EulerE }{\SYMBOL{t}}-{1}}%
\end{fricasmath}
\end{TeXOutput}
\formatResultType{Expression(Integer)}
\end{xtc}
\begin{noOutputXtc}
\begin{xtccomment}
Since the expressions involved get quite large, we tell
\Language{} to show us only terms of degree up to \spad{5.}
\end{xtccomment}
\begin{spadsrc}
)set streams calculate 5 
\end{spadsrc}
\end{noOutputXtc}
\syscmdindex{set streams calculate}
%
%
\begin{xtc}
\begin{xtccomment}
If we look at the Taylor expansion of \spad{f(x, t)} about \spad{t = 0,}
we see that the coefficients of the powers of \spad{t} are polynomials
in \spad{x}.
\end{xtccomment}
\begin{spadsrc}
ff := taylor(f,t = 0) 
\end{spadsrc}
\begin{TeXOutput}
\begin{fricasmath}{5}
1+\frac{2\TIMES \SYMBOL{x}-{1}}{2}\TIMES \SYMBOL{t}+\frac{6\TIMES \SUPER{%
\SYMBOL{x}}{2}-{6\TIMES \SYMBOL{x}}+1}{12}\TIMES \SUPER{\SYMBOL{t}}{2}+\frac{%
2\TIMES \SUPER{\SYMBOL{x}}{3}-{3\TIMES \SUPER{\SYMBOL{x}}{2}}+\SYMBOL{x}}{12}%
\TIMES \SUPER{\SYMBOL{t}}{3}+\frac{30\TIMES \SUPER{\SYMBOL{x}}{4}-{60\TIMES %
\SUPER{\SYMBOL{x}}{3}}+30\TIMES \SUPER{\SYMBOL{x}}{2}-{1}}{720}\TIMES \SUPER{%
\SYMBOL{t}}{4}+\frac{6\TIMES \SUPER{\SYMBOL{x}}{5}-{15\TIMES \SUPER{\SYMBOL{x%
}}{4}}+10\TIMES \SUPER{\SYMBOL{x}}{3}-{\SYMBOL{x}}}{720}\TIMES \SUPER{\SYMBOL%
{t}}{5}+\FUN{O}\PAREN{\SUPER{\SYMBOL{t}}{6}}%
\end{fricasmath}
\end{TeXOutput}
\formatResultType{UnivariateTaylorSeries(Expression(Integer), t, 0)}
\end{xtc}
%
In fact, the \eth{n} coefficient in this series is essentially
the \eth{n} Bernoulli polynomial:
the \eth{n} coefficient of the series is
${\frac{1}{n!}} B_n(x)$, where
$B_n(x)$
is the \eth{n} Bernoulli polynomial.
Thus, to obtain the \eth{n} Bernoulli polynomial, we multiply
the \eth{n} coefficient
of the series \spad{ff} by \spad{n!}.
%
\begin{xtc}
\begin{xtccomment}
For example, the sixth Bernoulli polynomial is this.
\end{xtccomment}
\begin{spadsrc}
factorial(6) * coefficient(ff,6) 
\end{spadsrc}
\begin{TeXOutput}
\begin{fricasmath}{6}
\frac{42\TIMES \SUPER{\SYMBOL{x}}{6}-{126\TIMES \SUPER{\SYMBOL{x}}{5}}+105%
\TIMES \SUPER{\SYMBOL{x}}{4}-{21\TIMES \SUPER{\SYMBOL{x}}{2}}+1}{42}%
\end{fricasmath}
\end{TeXOutput}
\formatResultType{Expression(Integer)}
\end{xtc}
%
\begin{xtc}
\begin{xtccomment}
We derive some properties of the function \spad{f(x,t)}.
First we compute \spad{f(x + 1,t) - f(x,t)}.
\end{xtccomment}
\begin{spadsrc}
g := eval(f, x = x + 1) - f 
\end{spadsrc}
\begin{TeXOutput}
\begin{fricasmath}{7}
\frac{\SYMBOL{t}\TIMES \SUPER{\EulerE }{\SYMBOL{t}\TIMES \SYMBOL{x}+\SYMBOL{t%
}}-{\SYMBOL{t}\TIMES \SUPER{\EulerE }{\SYMBOL{t}\TIMES \SYMBOL{x}}}}{\SUPER{%
\EulerE }{\SYMBOL{t}}-{1}}%
\end{fricasmath}
\end{TeXOutput}
\formatResultType{Expression(Integer)}
\end{xtc}
%
\begin{xtc}
\begin{xtccomment}
If we normalize \spad{g}, we see that it has a particularly simple form.
\end{xtccomment}
\begin{spadsrc}
normalize(g) 
\end{spadsrc}
\begin{TeXOutput}
\begin{fricasmath}{8}
\SYMBOL{t}\TIMES \SUPER{\EulerE }{\SYMBOL{t}\TIMES \SYMBOL{x}}%
\end{fricasmath}
\end{TeXOutput}
\formatResultType{Expression(Integer)}
\end{xtc}
%
From this it follows that the \eth{n}
coefficient in the Taylor expansion of
\spad{g(x,t)} at \spad{t = 0} is
$\frac{1}{(n-1)!}\:x^{n-1}$.
\begin{xtc}
\begin{xtccomment}
If you want to check this, evaluate the next expression.
\end{xtccomment}
\begin{spadsrc}
taylor(g,t = 0) 
\end{spadsrc}
\begin{TeXOutput}
\begin{fricasmath}{9}
\SYMBOL{t}+\SYMBOL{x}\TIMES \SUPER{\SYMBOL{t}}{2}+\frac{\SUPER{\SYMBOL{x}}{2}%
}{2}\TIMES \SUPER{\SYMBOL{t}}{3}+\frac{\SUPER{\SYMBOL{x}}{3}}{6}\TIMES \SUPER%
{\SYMBOL{t}}{4}+\frac{\SUPER{\SYMBOL{x}}{4}}{24}\TIMES \SUPER{\SYMBOL{t}}{5}+%
\FUN{O}\PAREN{\SUPER{\SYMBOL{t}}{6}}%
\end{fricasmath}
\end{TeXOutput}
\formatResultType{UnivariateTaylorSeries(Expression(Integer), t, 0)}
\end{xtc}
%
However, since \spad{g(x,t) = f(x+1,t)-f(x,t)}, it follows that the
\eth{n} coefficient is
$\frac{1}{n!}\:(B_n(x+1)-B_n(x))$.
Equating coefficients, we see that
$\frac{1}{(n-1)\:!}\:x^{n-1} = \frac{1}{n!}\:(B_n(x + 1) - B_n(x))$
and, therefore,
$x^{n-1} = {\frac{1}{n}}\:(B_n(x + 1) - B_n(x))$.
Let's apply this formula repeatedly, letting \spad{x} vary between two
integers \spad{a} and \spad{b}, with \spad{a < b}:
%
\begin{displaymath}
\begin{array}{lcl}
  a^{n-1}       & = & \frac{1}{n}   (B_n(a + 1) - B_n(a))       \\
  (a + 1)^{n-1} & = & \frac{1}{n}   (B_n(a + 2) - B_n(a + 1))   \\
  (a + 2)^{n-1} & = & \frac{1}{n}   (B_n(a + 3) - B_n(a + 2))   \\
  & \vdots &                                                    \\
  (b - 1)^{n-1} & = & \frac{1}{n}   (B_n(b) - B_n(b - 1))       \\
  b^{n-1}       & = & \frac{1}{n}   (B_n(b + 1) - B_n(b))
\end{array}
\end{displaymath}

When we add these equations we find that
the sum of the left-hand sides is
$\sum_{m=a}^{b} m^{n-1},$%
the sum of the
$(n-1)^{\hbox{\small\rm st}}$
powers from \spad{a} to \spad{b}.
The sum of the right-hand sides is a ``telescoping series.''
After cancellation, the sum is simply
$\frac{1}{n}\:(B_n(b + 1) - B_n(a))$.

Replacing \spad{n} by \spad{n + 1}, we have shown that
\begin{displaymath}
\sum_{m = a}^{b} m^n = \frac{1}{\displaystyle n + 1} \:
(B_{n+1}(b + 1) - B_{n+1}(a)).
\end{displaymath}

Let's use this to obtain the formula for the sum of fourth powers.
\begin{xtc}
\begin{xtccomment}
First we obtain the Bernoulli polynomial $B_5$.
\end{xtccomment}
\begin{spadsrc}
B5 := factorial(5) * coefficient(ff,5) 
\end{spadsrc}
\begin{TeXOutput}
\begin{fricasmath}{10}
\frac{6\TIMES \SUPER{\SYMBOL{x}}{5}-{15\TIMES \SUPER{\SYMBOL{x}}{4}}+10%
\TIMES \SUPER{\SYMBOL{x}}{3}-{\SYMBOL{x}}}{6}%
\end{fricasmath}
\end{TeXOutput}
\formatResultType{Expression(Integer)}
\end{xtc}
%
\begin{xtc}
\begin{xtccomment}
To find the sum of the first $k$ 4th powers,
we multiply \spad{1/5} by
$B_5(k+1) - B_5(1)$.
\end{xtccomment}
\begin{spadsrc}
1/5 * (eval(B5, x = k + 1) - eval(B5, x = 1)) 
\end{spadsrc}
\begin{TeXOutput}
\begin{fricasmath}{11}
\frac{6\TIMES \SUPER{\SYMBOL{k}}{5}+15\TIMES \SUPER{\SYMBOL{k}}{4}+10\TIMES %
\SUPER{\SYMBOL{k}}{3}-{\SYMBOL{k}}}{30}%
\end{fricasmath}
\end{TeXOutput}
\formatResultType{Expression(Integer)}
\end{xtc}
%
\begin{xtc}
\begin{xtccomment}
This is the same formula that we obtained via \spad{sum(m^4, m = 1..k)}.
\end{xtccomment}
\begin{spadsrc}
sum4 
\end{spadsrc}
\begin{TeXOutput}
\begin{fricasmath}{12}
\frac{6\TIMES \SUPER{\SYMBOL{k}}{5}+15\TIMES \SUPER{\SYMBOL{k}}{4}+10\TIMES %
\SUPER{\SYMBOL{k}}{3}-{\SYMBOL{k}}}{30}%
\end{fricasmath}
\end{TeXOutput}
\formatResultType{Fraction(Polynomial(Integer))}
\end{xtc}

At this point you may want to do the same computation, but with an
exponent other than \spad{4.}
For example, you might try to find a formula for the sum of the
first $k$ 20th powers.

% *********************************************************************
\head{section}{Solution of Differential Equations}{ugProblemDEQ}
% *********************************************************************
%
In this section we discuss \Language{}'s facilities for
\index{equation!differential!solving}
solving
\index{differential equation}
differential equations in closed-form and in series.

\Language{} provides facilities for closed-form solution of
\index{equation!differential!solving in closed-form}
single differential equations of the following kinds:
\begin{itemize}
\item linear ordinary differential equations, and
\item non-linear first order ordinary differential equations
when integrating factors can be found just by integration.
\end{itemize}

For a discussion of the solution of systems of linear and polynomial
equations, see
\spadref{ugProblemLinPolEqn}.

% *********************************************************************
\head{subsection}{Closed-Form Solutions of Linear Differential Equations}{ugxProblemLDEQClosed}
% *********************************************************************

A {\it differential equation} is an equation involving an unknown {\it
function} and one or more of its derivatives.
\index{differential equation}
The equation is called {\it ordinary} if derivatives with respect to
\index{equation!differential}
only one dependent variable appear in the equation (it is called {\it
partial} otherwise).
The package \spadtype{ElementaryFunctionODESolver} provides the
top-level operation \spadfun {solve} for finding closed-form solutions
of ordinary differential equations.
\exptypeindex{ElementaryFunctionODESolver}

To solve a differential equation, you must first create an operator for
\index{operator}
the unknown function.
%
\begin{xtc}
\begin{xtccomment}
We let \spad{y} be the unknown function in terms of \spad{x}.
\end{xtccomment}
\begin{spadsrc}
y := operator 'y 
\end{spadsrc}
\begin{TeXOutput}
\begin{fricasmath}{1}
\SYMBOL{y}%
\end{fricasmath}
\end{TeXOutput}
\formatResultType{BasicOperator}
\end{xtc}
%
You then type the equation using \spadfun{D} to create the
derivatives of the unknown function \spad{y(x)} where \spad{x} is any
symbol you choose (the so-called {\it dependent variable}).
%
\begin{xtc}
\begin{xtccomment}
This is how you enter
the equation \spad{y'' + y' + y = 0}.
\end{xtccomment}
\begin{spadsrc}
deq := D(y x, x, 2) + D(y x, x) + y x = 0
\end{spadsrc}
\begin{TeXOutput}
\begin{fricasmath}{2}
\PRIME{\SYMBOL{y}}{\STRING{,,}}\PAREN{\SYMBOL{x}}+\PRIME{\SYMBOL{y}}{\STRING{%
,}}\PAREN{\SYMBOL{x}}+\FUN{y}\PAREN{\SYMBOL{x}}=0%
\end{fricasmath}
\end{TeXOutput}
\formatResultType{Equation(Expression(Integer))}
\end{xtc}
%
The simplest way to invoke the \spadfun{solve} command is with three
arguments.
\begin{items}
\item the differential equation,
\item the operator representing the unknown function,
\item the dependent variable.
\end{items}
%
\begin{xtc}
\begin{xtccomment}
So, to solve the above equation, we enter this.
\end{xtccomment}
\begin{spadsrc}
solve(deq, y, x) 
\end{spadsrc}
\begin{TeXOutput}
\begin{fricasmath}{3}
\BRACKET{\SYMBOL{particular}=0\COMMA \SYMBOL{basis}=\BRACKET{\cos{\PAREN{%
\frac{\SYMBOL{x}\TIMES \sqrt{3}}{2}}}\TIMES \SUPER{\EulerE }{-{\frac{\SYMBOL{%
x}}{2}}}\COMMA \SUPER{\EulerE }{-{\frac{\SYMBOL{x}}{2}}}\TIMES \sin{\PAREN{%
\frac{\SYMBOL{x}\TIMES \sqrt{3}}{2}}}}}%
\end{fricasmath}
\end{TeXOutput}
\formatResultType{Union(Record(particular: Expression(Integer), basis: List(Expression(Integer))), ...)}
\end{xtc}
%
Since linear ordinary differential equations have infinitely many
solutions, \spadfun{solve} returns a {\it particular solution}
$f_p$
and a basis
$f_1,\dots,f_n$
for the solutions of the corresponding homogenuous equation.
Any expression of the form
$f_p + c_1 f_1 + \dots c_n f_n$
where the $c_i$ do not involve
the dependent variable is also a solution.
This is similar to what you get when you solve systems of linear
algebraic equations.

A way to select a unique solution is to specify {\it initial
conditions}: choose a value \spad{a} for the dependent variable
and specify the values of the unknown function and its derivatives
at \spad{a}.
If the number of initial conditions is equal to the order of the
equation, then the solution is unique (if it exists in closed
form!) and \spadfun{solve} tries to find it.
To specify initial conditions to \spadfun{solve}, use an
\spadtype{Equation} of the form \spad{x = a} for the third
parameter instead of the dependent variable, and add a fourth
parameter consisting of the list of values \spad{y(a), y'(a), ...}.

\begin{xtc}
\begin{xtccomment}
To find the solution of \spad{y'' + y = 0} satisfying \spad{y(0) = y'(0) = 1},
do this.
\end{xtccomment}
\begin{spadsrc}
deq := D(y x, x, 2) + y x 
\end{spadsrc}
\begin{TeXOutput}
\begin{fricasmath}{4}
\PRIME{\SYMBOL{y}}{\STRING{,,}}\PAREN{\SYMBOL{x}}+\FUN{y}\PAREN{\SYMBOL{x}}%
\end{fricasmath}
\end{TeXOutput}
\formatResultType{Expression(Integer)}
\end{xtc}
\begin{xtc}
\begin{xtccomment}
You can omit the \spad{= 0} when you enter the equation to be solved.
\end{xtccomment}
\begin{spadsrc}
solve(deq, y, x = 0, [1, 1]) 
\end{spadsrc}
\begin{TeXOutput}
\begin{fricasmath}{5}
\sin{\SYMBOL{x}}+\cos{\SYMBOL{x}}%
\end{fricasmath}
\end{TeXOutput}
\formatResultType{Union(Expression(Integer), ...)}
\end{xtc}
%

\Language{} is not limited to linear differential equations with
constant coefficients.
It can also find solutions when the coefficients are rational or
algebraic functions of the dependent variable.
Furthermore, \Language{} is not limited by the order of the equation.
%
\begin{xtc}
\begin{xtccomment}
\Language{} can solve the following third order equations with
polynomial coefficients.
\end{xtccomment}
\begin{spadsrc}
deq := x^3 * D(y x, x, 3) + x^2 * D(y x, x, 2) - 2 * x * D(y x, x) + 2 * y x = 2 * x^4 
\end{spadsrc}
\begin{TeXOutput}
\begin{fricasmath}{6}
\SUPER{\SYMBOL{x}}{3}\TIMES \PRIME{\SYMBOL{y}}{\STRING{,,,}}\PAREN{\SYMBOL{x}%
}+\SUPER{\SYMBOL{x}}{2}\TIMES \PRIME{\SYMBOL{y}}{\STRING{,,}}\PAREN{\SYMBOL{x%
}}-{2\TIMES \SYMBOL{x}\TIMES \PRIME{\SYMBOL{y}}{\STRING{,}}\PAREN{\SYMBOL{x}}%
}+2\TIMES \FUN{y}\PAREN{\SYMBOL{x}}=2\TIMES \SUPER{\SYMBOL{x}}{4}%
\end{fricasmath}
\end{TeXOutput}
\formatResultType{Equation(Expression(Integer))}
\end{xtc}
\begin{xtc}
\begin{xtccomment}
\end{xtccomment}
\begin{spadsrc}
solve(deq, y, x) 
\end{spadsrc}
\begin{TeXOutput}
\begin{fricasmath}{7}
\BRACKET{\SYMBOL{particular}=\frac{\SUPER{\SYMBOL{x}}{5}-{10\TIMES \SUPER{%
\SYMBOL{x}}{3}}+20\TIMES \SUPER{\SYMBOL{x}}{2}+4}{15\TIMES \SYMBOL{x}}\COMMA %
\SYMBOL{basis}=\BRACKET{\frac{2\TIMES \SUPER{\SYMBOL{x}}{3}-{3\TIMES \SUPER{%
\SYMBOL{x}}{2}}+1}{\SYMBOL{x}}\COMMA \frac{\SUPER{\SYMBOL{x}}{3}-{1}}{\SYMBOL%
{x}}\COMMA \frac{\SUPER{\SYMBOL{x}}{3}-{3\TIMES \SUPER{\SYMBOL{x}}{2}}-{1}}{%
\SYMBOL{x}}}}%
\end{fricasmath}
\end{TeXOutput}
\formatResultType{Union(Record(particular: Expression(Integer), basis: List(Expression(Integer))), ...)}
\end{xtc}
%
%
\begin{xtc}
\begin{xtccomment}
Here we are solving a homogeneous equation.
\end{xtccomment}
\begin{spadsrc}
deq := (x^9+x^3) * D(y x, x, 3) + 18 * x^8 * D(y x, x, 2) - 90 * x * D(y x, x) - 30 * (11 * x^6 - 3) * y x 
\end{spadsrc}
\begin{TeXOutput}
\begin{fricasmath}{8}
\PAREN{\SUPER{\SYMBOL{x}}{9}+\SUPER{\SYMBOL{x}}{3}}\TIMES \PRIME{\SYMBOL{y}}{%
\STRING{,,,}}\PAREN{\SYMBOL{x}}+18\TIMES \SUPER{\SYMBOL{x}}{8}\TIMES \PRIME{%
\SYMBOL{y}}{\STRING{,,}}\PAREN{\SYMBOL{x}}-{90\TIMES \SYMBOL{x}\TIMES \PRIME{%
\SYMBOL{y}}{\STRING{,}}\PAREN{\SYMBOL{x}}}+\PAREN{-{330\TIMES \SUPER{\SYMBOL{%
x}}{6}}+90}\TIMES \FUN{y}\PAREN{\SYMBOL{x}}%
\end{fricasmath}
\end{TeXOutput}
\formatResultType{Expression(Integer)}
\end{xtc}
\begin{xtc}
\begin{xtccomment}
\end{xtccomment}
\begin{spadsrc}
solve(deq, y, x) 
\end{spadsrc}
\begin{TeXOutput}
\begin{fricasmath}{9}
\BRACKET{\SYMBOL{particular}=0\COMMA \SYMBOL{basis}=\BRACKET{\frac{\SYMBOL{x}%
}{\SUPER{\SYMBOL{x}}{6}+1}\COMMA \frac{\SYMBOL{x}\TIMES \SUPER{\EulerE }{-{%
\sqrt{91}\TIMES \log{\SYMBOL{x}}}}}{\SUPER{\SYMBOL{x}}{6}+1}\COMMA \frac{%
\SYMBOL{x}\TIMES \SUPER{\EulerE }{\sqrt{91}\TIMES \log{\SYMBOL{x}}}}{\SUPER{%
\SYMBOL{x}}{6}+1}}}%
\end{fricasmath}
\end{TeXOutput}
\formatResultType{Union(Record(particular: Expression(Integer), basis: List(Expression(Integer))), ...)}
\end{xtc}
%
On the other hand, and in contrast with the operation
\spadfun{integrate}, it can happen that \Language{} finds no solution
and that some closed-form solution still exists.
While it is mathematically complicated to describe exactly when the
solutions are guaranteed to be found, the following statements are
correct and form good guidelines for linear ordinary differential
equations:
\begin{items}
\item If the coefficients are constants, \Language{} finds a complete basis
of solutions (i,e, all solutions).
\item If the coefficients are rational functions in the dependent variable,
\Language{} at least finds all solutions that do not involve algebraic
functions.
\end{items}
%
Note that this last statement does not mean that \Language{} does not find
the solutions that are algebraic functions.
It means that it is not
guaranteed that the algebraic function solutions will be found.
%
\begin{xtc}
\begin{xtccomment}
This is an example where all the algebraic solutions are found.
\end{xtccomment}
\begin{spadsrc}
deq := (x^2 + 1) * D(y x, x, 2) + 3 * x * D(y x, x) + y x = 0 
\end{spadsrc}
\begin{TeXOutput}
\begin{fricasmath}{10}
\PAREN{\SUPER{\SYMBOL{x}}{2}+1}\TIMES \PRIME{\SYMBOL{y}}{\STRING{,,}}\PAREN{%
\SYMBOL{x}}+3\TIMES \SYMBOL{x}\TIMES \PRIME{\SYMBOL{y}}{\STRING{,}}\PAREN{%
\SYMBOL{x}}+\FUN{y}\PAREN{\SYMBOL{x}}=0%
\end{fricasmath}
\end{TeXOutput}
\formatResultType{Equation(Expression(Integer))}
\end{xtc}
\begin{xtc}
\begin{xtccomment}
\end{xtccomment}
\begin{spadsrc}
solve(deq, y, x) 
\end{spadsrc}
\begin{TeXOutput}
\begin{fricasmath}{11}
\BRACKET{\SYMBOL{particular}=0\COMMA \SYMBOL{basis}=\BRACKET{\frac{1}{\sqrt{%
\SUPER{\SYMBOL{x}}{2}+1}}\COMMA \frac{\log{\PAREN{\sqrt{\SUPER{\SYMBOL{x}}{2}%
+1}-{\SYMBOL{x}}}}}{\sqrt{\SUPER{\SYMBOL{x}}{2}+1}}}}%
\end{fricasmath}
\end{TeXOutput}
\formatResultType{Union(Record(particular: Expression(Integer), basis: List(Expression(Integer))), ...)}
\end{xtc}

% *********************************************************************
\head{subsection}{Closed-Form Solutions of Non-Linear Differential Equations}{ugxProblemNLDEQClosed}
% *********************************************************************

This is an example that shows how to solve a non-linear
first order ordinary differential equation manually when an integrating
factor can be found just by integration.
At the end, we show you how to solve it directly.

Let's solve the differential equation \spad{y' = y / (x + y log y)}.
%
\begin{xtc}
\begin{xtccomment}
Using the notation
\spad{m(x, y) + n(x, y) y' = 0},
we have \spad{m = -y} and \spad{n = x + y log y}.
\end{xtccomment}
\begin{spadsrc}
m := -y 
\end{spadsrc}
\begin{TeXOutput}
\begin{fricasmath}{1}
-{\SYMBOL{y}}%
\end{fricasmath}
\end{TeXOutput}
\formatResultType{Polynomial(Integer)}
\end{xtc}
\begin{xtc}
\begin{xtccomment}
\end{xtccomment}
\begin{spadsrc}
n := x + y * log y 
\end{spadsrc}
\begin{TeXOutput}
\begin{fricasmath}{2}
\SYMBOL{y}\TIMES \log{\SYMBOL{y}}+\SYMBOL{x}%
\end{fricasmath}
\end{TeXOutput}
\formatResultType{Expression(Integer)}
\end{xtc}
%
\begin{xtc}
\begin{xtccomment}
We first check for exactness, that is, does \spad{dm/dy = dn/dx}?
\end{xtccomment}
\begin{spadsrc}
D(m, y) - D(n, x) 
\end{spadsrc}
\begin{TeXOutput}
\begin{fricasmath}{3}
-{2}%
\end{fricasmath}
\end{TeXOutput}
\formatResultType{Expression(Integer)}
\end{xtc}
%
This is not zero, so the equation is not exact.
Therefore we must look for
an integrating factor: a function \spad{mu(x,y)} such that
\spad{d(mu m)/dy = d(mu n)/dx}.
Normally, we first search for \spad{mu(x,y)} depending only on
\spad{x} or only on \spad{y}.
%
\begin{xtc}
\begin{xtccomment}
Let's search for such a \spad{mu(x)} first.
\end{xtccomment}
\begin{spadsrc}
mu := operator 'mu 
\end{spadsrc}
\begin{TeXOutput}
\begin{fricasmath}{4}
\SYMBOL{mu}%
\end{fricasmath}
\end{TeXOutput}
\formatResultType{BasicOperator}
\end{xtc}
\begin{xtc}
\begin{xtccomment}
\end{xtccomment}
\begin{spadsrc}
a := D(mu(x) * m, y) - D(mu(x) * n, x) 
\end{spadsrc}
\begin{TeXOutput}
\begin{fricasmath}{5}
\PAREN{-{\SYMBOL{y}\TIMES \log{\SYMBOL{y}}}-{\SYMBOL{x}}}\TIMES \PRIME{%
\SYMBOL{mu}}{\STRING{,}}\PAREN{\SYMBOL{x}}-{2\TIMES \FUN{mu}\PAREN{\SYMBOL{x}%
}}%
\end{fricasmath}
\end{TeXOutput}
\formatResultType{Expression(Integer)}
\end{xtc}
%
%
\begin{xtc}
\begin{xtccomment}
If the above is zero for a function
\spad{mu} that does {\it not} depend on \spad{y}, then
\spad{mu(x)} is an integrating factor.
\end{xtccomment}
\begin{spadsrc}
solve(a = 0, mu, x) 
\end{spadsrc}
\begin{TeXOutput}
\begin{fricasmath}{6}
\BRACKET{\SYMBOL{particular}=0\COMMA \SYMBOL{basis}=\BRACKET{\frac{1}{\SUPER{%
\SYMBOL{y}}{2}\TIMES \SUPER{\PAREN{\log{\SYMBOL{y}}}}{2}+2\TIMES \SYMBOL{x}%
\TIMES \SYMBOL{y}\TIMES \log{\SYMBOL{y}}+\SUPER{\SYMBOL{x}}{2}}}}%
\end{fricasmath}
\end{TeXOutput}
\formatResultType{Union(Record(particular: Expression(Integer), basis: List(Expression(Integer))), ...)}
\end{xtc}
%
The solution depends on \spad{y}, so there is no integrating
factor that depends on \spad{x} only.
%
\begin{xtc}
\begin{xtccomment}
Let's look for one that depends on \spad{y} only.
\end{xtccomment}
\begin{spadsrc}
b := D(mu(y) * m, y) - D(mu(y) * n, x) 
\end{spadsrc}
\begin{TeXOutput}
\begin{fricasmath}{7}
-{\SYMBOL{y}\TIMES \PRIME{\SYMBOL{mu}}{\STRING{,}}\PAREN{\SYMBOL{y}}}-{2%
\TIMES \FUN{mu}\PAREN{\SYMBOL{y}}}%
\end{fricasmath}
\end{TeXOutput}
\formatResultType{Expression(Integer)}
\end{xtc}
\begin{xtc}
\begin{xtccomment}
\end{xtccomment}
\begin{spadsrc}
sb := solve(b = 0, mu, y) 
\end{spadsrc}
\begin{TeXOutput}
\begin{fricasmath}{8}
\BRACKET{\SYMBOL{particular}=0\COMMA \SYMBOL{basis}=\BRACKET{\frac{1}{\SUPER{%
\SYMBOL{y}}{2}}}}%
\end{fricasmath}
\end{TeXOutput}
\formatResultType{Union(Record(particular: Expression(Integer), basis: List(Expression(Integer))), ...)}
\end{xtc}
\noindent
We've found one!
%
\begin{xtc}
\begin{xtccomment}
The above \spad{mu(y)} is an integrating factor.
We must multiply our initial equation
(that is, \spad{m} and \spad{n}) by the integrating factor.
\end{xtccomment}
\begin{spadsrc}
intFactor := sb.basis.1 
\end{spadsrc}
\begin{TeXOutput}
\begin{fricasmath}{9}
\frac{1}{\SUPER{\SYMBOL{y}}{2}}%
\end{fricasmath}
\end{TeXOutput}
\formatResultType{Expression(Integer)}
\end{xtc}
\begin{xtc}
\begin{xtccomment}
\end{xtccomment}
\begin{spadsrc}
m := intFactor * m 
\end{spadsrc}
\begin{TeXOutput}
\begin{fricasmath}{10}
-{\frac{1}{\SYMBOL{y}}}%
\end{fricasmath}
\end{TeXOutput}
\formatResultType{Expression(Integer)}
\end{xtc}
\begin{xtc}
\begin{xtccomment}
\end{xtccomment}
\begin{spadsrc}
n := intFactor * n 
\end{spadsrc}
\begin{TeXOutput}
\begin{fricasmath}{11}
\frac{\SYMBOL{y}\TIMES \log{\SYMBOL{y}}+\SYMBOL{x}}{\SUPER{\SYMBOL{y}}{2}}%
\end{fricasmath}
\end{TeXOutput}
\formatResultType{Expression(Integer)}
\end{xtc}
%
\begin{xtc}
\begin{xtccomment}
Let's check for exactness.
\end{xtccomment}
\begin{spadsrc}
D(m, y) - D(n, x) 
\end{spadsrc}
\begin{TeXOutput}
\begin{fricasmath}{12}
0%
\end{fricasmath}
\end{TeXOutput}
\formatResultType{Expression(Integer)}
\end{xtc}
%
We must solve the exact equation, that is, find a function
\spad{s(x,y)} such that
\spad{ds/dx = m}  and \spad{ds/dy = n}.
%
\begin{xtc}
\begin{xtccomment}
We start by writing \spad{s(x, y) = h(y) + integrate(m, x)}
where \spad{h(y)} is an unknown function of \spad{y}.
This guarantees that \spad{ds/dx = m}.
\end{xtccomment}
\begin{spadsrc}
h := operator 'h 
\end{spadsrc}
\begin{TeXOutput}
\begin{fricasmath}{13}
\SYMBOL{h}%
\end{fricasmath}
\end{TeXOutput}
\formatResultType{BasicOperator}
\end{xtc}
\begin{xtc}
\begin{xtccomment}
\end{xtccomment}
\begin{spadsrc}
sol := h y + integrate(m, x) 
\end{spadsrc}
\begin{TeXOutput}
\begin{fricasmath}{14}
\frac{\SYMBOL{y}\TIMES \FUN{h}\PAREN{\SYMBOL{y}}-{\SYMBOL{x}}}{\SYMBOL{y}}%
\end{fricasmath}
\end{TeXOutput}
\formatResultType{Expression(Integer)}
\end{xtc}
%
%
\begin{xtc}
\begin{xtccomment}
All we want is to find \spad{h(y)} such that
\spad{ds/dy = n}.
\end{xtccomment}
\begin{spadsrc}
dsol := D(sol, y) 
\end{spadsrc}
\begin{TeXOutput}
\begin{fricasmath}{15}
\frac{\SUPER{\SYMBOL{y}}{2}\TIMES \PRIME{\SYMBOL{h}}{\STRING{,}}\PAREN{%
\SYMBOL{y}}+\SYMBOL{x}}{\SUPER{\SYMBOL{y}}{2}}%
\end{fricasmath}
\end{TeXOutput}
\formatResultType{Expression(Integer)}
\end{xtc}
\begin{xtc}
\begin{xtccomment}
\end{xtccomment}
\begin{spadsrc}
nsol := solve(dsol = n, h, y) 
\end{spadsrc}
\begin{TeXOutput}
\begin{fricasmath}{16}
\BRACKET{\SYMBOL{particular}=\frac{\SUPER{\PAREN{\log{\SYMBOL{y}}}}{2}}{2}%
\COMMA \SYMBOL{basis}=\BRACKET{1}}%
\end{fricasmath}
\end{TeXOutput}
\formatResultType{Union(Record(particular: Expression(Integer), basis: List(Expression(Integer))), ...)}
\end{xtc}
%
\begin{xtc}
\begin{xtccomment}
The above particular solution is the \spad{h(y)} we want, so we just replace
\spad{h(y)} by it in the implicit solution.
\end{xtccomment}
\begin{spadsrc}
eval(sol, h y = nsol.particular) 
\end{spadsrc}
\begin{TeXOutput}
\begin{fricasmath}{17}
\frac{\SYMBOL{y}\TIMES \SUPER{\PAREN{\log{\SYMBOL{y}}}}{2}-{2\TIMES \SYMBOL{x%
}}}{2\TIMES \SYMBOL{y}}%
\end{fricasmath}
\end{TeXOutput}
\formatResultType{Expression(Integer)}
\end{xtc}
%
A first integral of the initial equation is obtained by setting
this result equal to an arbitrary constant.

Now that we've seen how to solve the equation ``by hand,''
we show you how to do it with the \spadfun{solve} operation.
\begin{xtc}
\begin{xtccomment}
First define \spad{y} to be an operator.
\end{xtccomment}
\begin{spadsrc}
y := operator 'y 
\end{spadsrc}
\begin{TeXOutput}
\begin{fricasmath}{18}
\SYMBOL{y}%
\end{fricasmath}
\end{TeXOutput}
\formatResultType{BasicOperator}
\end{xtc}
\begin{xtc}
\begin{xtccomment}
Next we create the differential equation.
\end{xtccomment}
\begin{spadsrc}
deq := D(y x, x) = y(x) / (x + y(x) * log y x) 
\end{spadsrc}
\begin{TeXOutput}
\begin{fricasmath}{19}
\PRIME{\SYMBOL{y}}{\STRING{,}}\PAREN{\SYMBOL{x}}=\frac{\FUN{y}\PAREN{\SYMBOL{%
x}}}{\FUN{y}\PAREN{\SYMBOL{x}}\TIMES \log{\FUN{y}\PAREN{\SYMBOL{x}}}+\SYMBOL{%
x}}%
\end{fricasmath}
\end{TeXOutput}
\formatResultType{Equation(Expression(Integer))}
\end{xtc}
\begin{xtc}
\begin{xtccomment}
Finally, we solve it.
\end{xtccomment}
\begin{spadsrc}
solve(deq, y, x) 
\end{spadsrc}
\begin{TeXOutput}
\begin{fricasmath}{20}
\frac{\FUN{y}\PAREN{\SYMBOL{x}}\TIMES \SUPER{\PAREN{\log{\FUN{y}\PAREN{%
\SYMBOL{x}}}}}{2}-{2\TIMES \SYMBOL{x}}}{2\TIMES \FUN{y}\PAREN{\SYMBOL{x}}}%
\end{fricasmath}
\end{TeXOutput}
\formatResultType{Union(Expression(Integer), ...)}
\end{xtc}

% *********************************************************************
\head{subsection}{Power Series Solutions of Differential Equations}{ugxProblemDEQSeries}
% *********************************************************************

The command to solve differential equations in power
\index{equation!differential!solving in power series}
series
\index{power series}
around
\index{series!power}
a particular initial point with specific initial conditions is called
\spadfun{seriesSolve}.
It can take a variety of parameters, so we illustrate
its use with some examples.

\begin{noOutputXtc}
\begin{xtccomment}
Since the coefficients of some solutions
are quite large, we reset the default to compute only seven terms.
\end{xtccomment}
\begin{spadsrc}
)set streams calculate 7 
\end{spadsrc}
\end{noOutputXtc}

You can solve a single nonlinear equation of any order. For example,
we solve  \spad{y''' = sin(y'') * exp(y) + cos(x)}
subject to \spad{y(0) = 1, y'(0) = 0, y''(0) = 0}.

\begin{xtc}
\begin{xtccomment}
We first tell \Language{}
that the symbol \spad{'y} denotes a new operator.
\end{xtccomment}
\begin{spadsrc}
y := operator 'y 
\end{spadsrc}
\begin{TeXOutput}
\begin{fricasmath}{1}
\SYMBOL{y}%
\end{fricasmath}
\end{TeXOutput}
\formatResultType{BasicOperator}
\end{xtc}
\begin{xtc}
\begin{xtccomment}
Enter the differential equation using \spad{y} like any system
function.
\end{xtccomment}
\begin{spadsrc}
eq := D(y(x), x, 3) - sin(D(y(x), x, 2))*exp(y(x)) = cos(x)
\end{spadsrc}
\begin{TeXOutput}
\begin{fricasmath}{2}
\PRIME{\SYMBOL{y}}{\STRING{,,,}}\PAREN{\SYMBOL{x}}-{\SUPER{\EulerE }{\FUN{y}%
\PAREN{\SYMBOL{x}}}\TIMES \sin{\PRIME{\SYMBOL{y}}{\STRING{,,}}\PAREN{\SYMBOL{%
x}}}}=\cos{\SYMBOL{x}}%
\end{fricasmath}
\end{TeXOutput}
\formatResultType{Equation(Expression(Integer))}
\end{xtc}
%
\begin{xtc}
\begin{xtccomment}
Solve it around \spad{x = 0} with the initial conditions
\spad{y(0) = 1, y'(0) = y''(0) = 0}.
\end{xtccomment}
\begin{spadsrc}
seriesSolve(eq, y, x = 0, [1, 0, 0])
\end{spadsrc}
\begin{MessageOutput}
   Compiling function %JG with type List(UnivariateTaylorSeries(
      Expression(Integer),x,0)) -> UnivariateTaylorSeries(Expression(
      Integer),x,0) 
\end{MessageOutput}
\begin{TeXOutput}
\begin{fricasmath}{3}
1+\frac{1}{6}\TIMES \SUPER{\SYMBOL{x}}{3}+\frac{\EulerE }{24}\TIMES \SUPER{%
\SYMBOL{x}}{4}+\frac{\SUPER{\EulerE }{2}-{1}}{120}\TIMES \SUPER{\SYMBOL{x}}{5%
}+\frac{\SUPER{\EulerE }{3}-{2\TIMES \EulerE }}{720}\TIMES \SUPER{\SYMBOL{x}%
}{6}+\frac{\SUPER{\EulerE }{4}-{8\TIMES \SUPER{\EulerE }{2}}+4\TIMES \EulerE %
+1}{5040}\TIMES \SUPER{\SYMBOL{x}}{7}+\FUN{O}\PAREN{\SUPER{\SYMBOL{x}}{8}}%
\end{fricasmath}
\end{TeXOutput}
\formatResultType{UnivariateTaylorSeries(Expression(Integer), x, 0)}
\end{xtc}

You can also solve a system of nonlinear first order equations.
For example, we solve a system that has \spad{tan(t)} and
\spad{sec(t)} as solutions.

\begin{xtc}
\begin{xtccomment}
We tell \Language{} that \spad{x} is also an operator.
\end{xtccomment}
\begin{spadsrc}
x := operator 'x
\end{spadsrc}
\begin{MessageOutput}
   Compiled code for %JG has been cleared.
\end{MessageOutput}
\begin{TeXOutput}
\begin{fricasmath}{4}
\SYMBOL{x}%
\end{fricasmath}
\end{TeXOutput}
\formatResultType{BasicOperator}
\end{xtc}
\begin{xtc}
\begin{xtccomment}
Enter the two equations forming our system.
\end{xtccomment}
\begin{spadsrc}
eq1 := D(x(t), t) = 1 + x(t)^2
\end{spadsrc}
\begin{TeXOutput}
\begin{fricasmath}{5}
\PRIME{\SYMBOL{x}}{\STRING{,}}\PAREN{\SYMBOL{t}}=\SUPER{\FUN{x}\PAREN{\SYMBOL%
{t}}}{2}+1%
\end{fricasmath}
\end{TeXOutput}
\formatResultType{Equation(Expression(Integer))}
\end{xtc}
%
\begin{xtc}
\begin{xtccomment}
\end{xtccomment}
\begin{spadsrc}
eq2 := D(y(t), t) = x(t) * y(t)
\end{spadsrc}
\begin{TeXOutput}
\begin{fricasmath}{6}
\PRIME{\SYMBOL{y}}{\STRING{,}}\PAREN{\SYMBOL{t}}=\FUN{x}\PAREN{\SYMBOL{t}}%
\TIMES \FUN{y}\PAREN{\SYMBOL{t}}%
\end{fricasmath}
\end{TeXOutput}
\formatResultType{Equation(Expression(Integer))}
\end{xtc}
%
\begin{xtc}
\begin{xtccomment}
Solve the system around \spad{t = 0} with the initial conditions
\spad{x(0) = 0} and \spad{y(0) = 1}.
Notice that since we give the unknowns in the
order \spad{[x, y]}, the answer is a list of two series in the order
\spad{[series for x(t), series for y(t)]}.
\end{xtccomment}
\begin{spadsrc}
seriesSolve([eq2, eq1], [x, y], t = 0, [y(0) = 1, x(0) = 0])
\end{spadsrc}
\begin{MessageOutput}
   Compiling function %JM with type List(UnivariateTaylorSeries(
      Expression(Integer),t,0)) -> UnivariateTaylorSeries(Expression(
      Integer),t,0) 
\end{MessageOutput}
\begin{MessageOutput}
   Compiling function %JN with type List(UnivariateTaylorSeries(
      Expression(Integer),t,0)) -> UnivariateTaylorSeries(Expression(
      Integer),t,0) 
\end{MessageOutput}
\begin{TeXOutput}
\begin{fricasmath}{7}
\BRACKET{\SYMBOL{t}+\frac{1}{3}\TIMES \SUPER{\SYMBOL{t}}{3}+\frac{2}{15}%
\TIMES \SUPER{\SYMBOL{t}}{5}+\frac{17}{315}\TIMES \SUPER{\SYMBOL{t}}{7}+\FUN{%
O}\PAREN{\SUPER{\SYMBOL{t}}{8}}\COMMA 1+\frac{1}{2}\TIMES \SUPER{\SYMBOL{t}}{%
2}+\frac{5}{24}\TIMES \SUPER{\SYMBOL{t}}{4}+\frac{61}{720}\TIMES \SUPER{%
\SYMBOL{t}}{6}+\FUN{O}\PAREN{\SUPER{\SYMBOL{t}}{8}}}%
\end{fricasmath}
\end{TeXOutput}
\formatResultType{List(UnivariateTaylorSeries(Expression(Integer), t, 0))}
\end{xtc}
\noindent
The order in which we give the
equations and the initial conditions has no effect on the order of
the solution.

% *********************************************************************
\head{section}{Finite Fields}{ugProblemFinite}
% *********************************************************************
%

A {\it finite field} (also called a {\it Galois field}) is a
finite algebraic structure where one can add, multiply and divide
under the same laws (for example, commutativity, associativity or
distributivity) as apply to the rational, real or complex numbers.
Unlike those three fields, for any finite field there exists a
positive prime integer \smath{p}, called the
\spadfun{characteristic}, such that
$p \: x = 0$
for any element \smath{x} in the finite field.
In fact, the number of elements in a finite field is a power of
the characteristic and for each prime \smath{p} and positive
integer \smath{n} there exists exactly one finite field with
\mathOrSpad{p^n} elements, up to
isomorphism.\footnote{For more information about the algebraic
structure and properties of finite fields, see, for example, S.
Lang, {\it Algebra}, Second Edition, New York: Addison-Wesley
Publishing Company, Inc., 1984, ISBN 0 201 05487 6; or R.
Lidl, H.
Niederreiter, {\it Finite Fields}, Encyclopedia of Mathematics and
Its Applications, Vol.
20, Cambridge: Cambridge Univ.
Press, 1983, ISBN 0 521 30240 4.}

When \spad{n = 1,} the field has \smath{p} elements and is
called a {\it prime field}, discussed in
the next section.
There are several ways of implementing extensions of finite
fields, and \Language{} provides quite a bit of freedom to allow
you to choose the one that is best for your application.
Moreover, we provide operations for converting among the different
representations of extensions and different extensions of a single
field.
Finally, note that you usually need to package-call operations
from finite fields if the operations do not take as an argument an
object of the field.
See \spadref{ugTypesPkgCall} for more information on
package-calling.

% *********************************************************************
\head{subsection}{Modular Arithmetic and Prime Fields}{ugxProblemFinitePrime}
% *********************************************************************
\index{finite field}
\index{Galois!field}
\index{field!finite!prime}
\index{field!prime}
\index{field!Galois}
\index{prime field}
\index{modular arithmetic}
\index{arithmetic!modular}

Let \smath{n} be a positive integer.
It is well known that you can get the same result if you perform
addition, subtraction or multiplication of integers and then take
the remainder on dividing by \smath{n} as if
you had first done such remaindering on the operands, performed the
arithmetic and then (if necessary) done remaindering again.
This allows us to speak of arithmetic
{\it modulo} \smath{n} or, more simply
{\it mod} \smath{n}.
\begin{xtc}
\begin{xtccomment}
In \Language{}, you use \spadtype{IntegerMod} to do such arithmetic.
\end{xtccomment}
\begin{spadsrc}
(a,b) : IntegerMod 12 
\end{spadsrc}
\end{xtc}
\begin{xtc}
\begin{xtccomment}
\end{xtccomment}
\begin{spadsrc}
(a, b) := (16, 7) 
\end{spadsrc}
\begin{TeXOutput}
\begin{fricasmath}{2}
7%
\end{fricasmath}
\end{TeXOutput}
\formatResultType{IntegerMod(12)}
\end{xtc}
\begin{xtc}
\begin{xtccomment}
\end{xtccomment}
\begin{spadsrc}
[a - b, a * b] 
\end{spadsrc}
\begin{TeXOutput}
\begin{fricasmath}{3}
\BRACKET{9\COMMA 4}%
\end{fricasmath}
\end{TeXOutput}
\formatResultType{List(IntegerMod(12))}
\end{xtc}
\begin{xtc}
\begin{xtccomment}
If \smath{n} is not prime, there is only a limited notion of
reciprocals and division.
\end{xtccomment}
\begin{spadsrc}
a / b 
\end{spadsrc}
\begin{MessageOutput}
   There are 12 exposed and 12 unexposed library operations named / 
      having 2 argument(s) but none was determined to be applicable. 
      Use HyperDoc Browse, or issue
                                )display op /
      to learn more about the available operations. Perhaps 
      package-calling the operation or using coercions on the arguments
      will allow you to apply the operation.
\end{MessageOutput}
\begin{MessageOutput}
   Cannot find a definition or applicable library operation named / 
      with argument type(s) 
                               IntegerMod(12)
                               IntegerMod(12)
      
      Perhaps you should use "@" to indicate the required return type, 
      or "$" to specify which version of the function you need.
\end{MessageOutput}
\end{xtc}
\begin{xtc}
\begin{xtccomment}
\end{xtccomment}
\begin{spadsrc}
recip a 
\end{spadsrc}
\begin{TeXOutput}
\begin{fricasmath}{4}
\STRING{"failed"}%
\end{fricasmath}
\end{TeXOutput}
\formatResultType{Union("failed", ...)}
\end{xtc}
\begin{xtc}
\begin{xtccomment}
Here \spad{7} and \spad{12} are relatively prime, so \spad{7}
has a multiplicative inverse modulo \spad{12}.
\end{xtccomment}
\begin{spadsrc}
recip b 
\end{spadsrc}
\begin{TeXOutput}
\begin{fricasmath}{5}
7%
\end{fricasmath}
\end{TeXOutput}
\formatResultType{Union(IntegerMod(12), ...)}
\end{xtc}

If we take \smath{n} to be a prime number \smath{p},
then taking inverses and, therefore, division are generally defined.
\begin{xtc}
\begin{xtccomment}
Use \spadtype{PrimeField} instead of \spadtype{IntegerMod}
for \smath{n} prime.
\end{xtccomment}
\begin{spadsrc}
c : PrimeField 11 := 8 
\end{spadsrc}
\begin{TeXOutput}
\begin{fricasmath}{6}
8%
\end{fricasmath}
\end{TeXOutput}
\formatResultType{PrimeField(11)}
\end{xtc}
\begin{xtc}
\begin{xtccomment}
\end{xtccomment}
\begin{spadsrc}
inv c 
\end{spadsrc}
\begin{TeXOutput}
\begin{fricasmath}{7}
7%
\end{fricasmath}
\end{TeXOutput}
\formatResultType{PrimeField(11)}
\end{xtc}
\begin{xtc}
\begin{xtccomment}
You can also use \spad{1/c} and \spad{c^(-1)} for the inverse of
\smath{c}.
\end{xtccomment}
\begin{spadsrc}
9/c 
\end{spadsrc}
\begin{TeXOutput}
\begin{fricasmath}{8}
8%
\end{fricasmath}
\end{TeXOutput}
\formatResultType{PrimeField(11)}
\end{xtc}

\spadtype{PrimeField} (abbreviation \spadtype{PF}) checks if its
argument is prime when you try to use an operation from it.
If you know the argument is prime (particularly if it is large),
\spadtype{InnerPrimeField} (abbreviation \spadtype{IPF}) assumes
the argument has already been verified to be prime.
If you do use a number that is not prime, you will eventually get
an error message, most likely a division by zero message.
For computer science applications, the most important finite fields
are \spadtype{PrimeField 2} and its extensions.

\begin{xtc}
\begin{xtccomment}
In the following examples, we work with the finite field with
\smath{p = 101} elements.
\end{xtccomment}
\begin{spadsrc}
GF101 := PF 101 
\end{spadsrc}
\begin{TeXOutput}
\begin{fricasmath}{9}
\STRING{PrimeField(101)}%
\end{fricasmath}
\end{TeXOutput}
\formatResultType{Type}
\end{xtc}
\begin{xtc}
\begin{xtccomment}
Like many domains in \Language{}, finite fields provide an operation
for returning a random element of the domain.
\end{xtccomment}
\begin{spadsrc}
x := random()$GF101 
\end{spadsrc}
\begin{TeXOutput}
\begin{fricasmath}{10}
44%
\end{fricasmath}
\end{TeXOutput}
\formatResultType{PrimeField(101)}
\end{xtc}
\begin{xtc}
\begin{xtccomment}
\end{xtccomment}
\begin{spadsrc}
y : GF101 := 37 
\end{spadsrc}
\begin{TeXOutput}
\begin{fricasmath}{11}
37%
\end{fricasmath}
\end{TeXOutput}
\formatResultType{PrimeField(101)}
\end{xtc}
\begin{xtc}
\begin{xtccomment}
\end{xtccomment}
\begin{spadsrc}
z := x/y 
\end{spadsrc}
\begin{TeXOutput}
\begin{fricasmath}{12}
94%
\end{fricasmath}
\end{TeXOutput}
\formatResultType{PrimeField(101)}
\end{xtc}
\begin{xtc}
\begin{xtccomment}
\end{xtccomment}
\begin{spadsrc}
z * y - x 
\end{spadsrc}
\begin{TeXOutput}
\begin{fricasmath}{13}
0%
\end{fricasmath}
\end{TeXOutput}
\formatResultType{PrimeField(101)}
\end{xtc}
%
\begin{xtc}
\begin{xtccomment}
The element \spad{2} is a {\it primitive element} of this field,
\index{primitive element}
\index{element!primitive}
\end{xtccomment}
\begin{spadsrc}
pe := primitiveElement()$GF101 
\end{spadsrc}
\begin{TeXOutput}
\begin{fricasmath}{14}
2%
\end{fricasmath}
\end{TeXOutput}
\formatResultType{PrimeField(101)}
\end{xtc}
%
\begin{xtc}
\begin{xtccomment}
in the sense that its powers enumerate all nonzero elements.
\end{xtccomment}
\begin{spadsrc}
[pe^i for i in 0..99] 
\end{spadsrc}
\begin{TeXOutput}
\begin{fricasmath}{15}
\BRACKET{1\COMMA 2\COMMA 4\COMMA 8\COMMA 16\COMMA 32\COMMA 64\COMMA 27\COMMA %
54\COMMA 7\COMMA 14\COMMA 28\COMMA 56\COMMA 11\COMMA 22\COMMA 44\COMMA 88%
\COMMA 75\COMMA 49\COMMA 98\COMMA 95\COMMA 89\COMMA 77\COMMA 53\COMMA 5%
\COMMA 10\COMMA 20\COMMA 40\COMMA 80\COMMA 59\COMMA 17\COMMA 34\COMMA 68%
\COMMA 35\COMMA 70\COMMA 39\COMMA 78\COMMA 55\COMMA 9\COMMA 18\COMMA 36%
\COMMA 72\COMMA 43\COMMA 86\COMMA 71\COMMA 41\COMMA 82\COMMA 63\COMMA 25%
\COMMA 50\COMMA 100\COMMA 99\COMMA 97\COMMA 93\COMMA 85\COMMA 69\COMMA 37%
\COMMA 74\COMMA 47\COMMA 94\COMMA 87\COMMA 73\COMMA 45\COMMA 90\COMMA 79%
\COMMA 57\COMMA 13\COMMA 26\COMMA 52\COMMA 3\COMMA 6\COMMA 12\COMMA 24\COMMA %
48\COMMA 96\COMMA 91\COMMA 81\COMMA 61\COMMA 21\COMMA 42\COMMA 84\COMMA 67%
\COMMA 33\COMMA 66\COMMA 31\COMMA 62\COMMA 23\COMMA 46\COMMA 92\COMMA 83%
\COMMA 65\COMMA 29\COMMA 58\COMMA 15\COMMA 30\COMMA 60\COMMA 19\COMMA 38%
\COMMA 76\COMMA 51}%
\end{fricasmath}
\end{TeXOutput}
\formatResultType{List(PrimeField(101))}
\end{xtc}
%
%
\begin{xtc}
\begin{xtccomment}
If every nonzero element is a power of a primitive element, how do you
determine what the exponent is?
Use
\index{discrete logarithm}
\spadfun{discreteLog}.
\index{logarithm!discrete}
\end{xtccomment}
\begin{spadsrc}
ex := discreteLog(y) 
\end{spadsrc}
\begin{TeXOutput}
\begin{fricasmath}{16}
56%
\end{fricasmath}
\end{TeXOutput}
\formatResultType{PositiveInteger}
\end{xtc}
\begin{xtc}
\begin{xtccomment}
\end{xtccomment}
\begin{spadsrc}
pe ^ ex 
\end{spadsrc}
\begin{TeXOutput}
\begin{fricasmath}{17}
37%
\end{fricasmath}
\end{TeXOutput}
\formatResultType{PrimeField(101)}
\end{xtc}
%
\begin{xtc}
\begin{xtccomment}
The \spadfun{order} of a nonzero element \smath{x} is the
smallest positive integer \smath{t} such
$x^t = 1$.
\end{xtccomment}
\begin{spadsrc}
order y 
\end{spadsrc}
\begin{TeXOutput}
\begin{fricasmath}{18}
25%
\end{fricasmath}
\end{TeXOutput}
\formatResultType{PositiveInteger}
\end{xtc}
\begin{xtc}
\begin{xtccomment}
The order of a primitive element is the defining \smath{p-1}.
\end{xtccomment}
\begin{spadsrc}
order pe 
\end{spadsrc}
\begin{TeXOutput}
\begin{fricasmath}{19}
100%
\end{fricasmath}
\end{TeXOutput}
\formatResultType{PositiveInteger}
\end{xtc}

% *********************************************************************
\head{subsection}{Extensions of Finite Fields}{ugxProblemFiniteExtensionFinite}
% *********************************************************************
\index{finite field}
\index{field!finite!extension of}

When you want to work with an extension of a finite field in \Language{},
you have three choices to make:
\begin{enumerate}
\item Do you want to generate an extension of the prime field
(for example, \spadtype{PrimeField 2}) or an extension of a given field?
%
\item Do you want to use a representation that is particularly
efficient for multiplication, exponentiation and addition but
uses a lot of computer memory (a representation that models the cyclic
group structure of the multiplicative group of the field extension
and uses a Zech logarithm table),  one that
\index{Zech logarithm}
uses a normal basis for the vector space structure of the field
extension, or one that performs arithmetic modulo an irreducible
polynomial?
The cyclic group representation is only usable up to ``medium''
(relative to your machine's performance)
sized fields.
If the field is large and the normal basis is relatively simple,
the normal basis representation is more efficient for exponentiation than
the irreducible polynomial representation.
%
\item Do you want to provide a polynomial explicitly, a root of which
``generates'' the extension in one of the three senses in (2),
or do you wish to have the polynomial generated for you?
\end{enumerate}

This illustrates one of the most important features of \Language{}:
you can choose exactly the right data-type and representation to
suit your application best.

We first tell you what domain constructors to use for each case
above, and then give some examples.

\hangafter=1\hangindent=2pc
Constructors that automatically generate extensions of the prime field:
\newline
\spadtype{FiniteField} \newline
\spadtype{FiniteFieldCyclicGroup} \newline
\spadtype{FiniteFieldNormalBasis}

\hangafter=1\hangindent=2pc
Constructors that generate extensions of an arbitrary field:
\newline
\spadtype{FiniteFieldExtension} \newline
\spadtype{FiniteFieldExtensionByPolynomial} \newline
\spadtype{FiniteFieldCyclicGroupExtension} \newline
\spadtype{FiniteFieldCyclicGroupExtensionByPolynomial} \newline
\spadtype{FiniteFieldNormalBasisExtension} \newline
\spadtype{FiniteFieldNormalBasisExtensionByPolynomial}

\hangafter=1\hangindent=2pc
Constructors that use a cyclic group representation:
\newline
\spadtype{FiniteFieldCyclicGroup} \newline
\spadtype{FiniteFieldCyclicGroupExtension} \newline
\spadtype{FiniteFieldCyclicGroupExtensionByPolynomial}

\hangafter=1\hangindent=2pc
Constructors that use a normal basis representation:
\newline
\spadtype{FiniteFieldNormalBasis} \newline
\spadtype{FiniteFieldNormalBasisExtension} \newline
\spadtype{FiniteFieldNormalBasisExtensionByPolynomial}

\hangafter=1\hangindent=2pc
Constructors that use an irreducible modulus polynomial representation:
\newline
\spadtype{FiniteField} \newline
\spadtype{FiniteFieldExtension} \newline
\spadtype{FiniteFieldExtensionByPolynomial}

\hangafter=1\hangindent=2pc
Constructors that generate a polynomial for you:
\newline
\spadtype{FiniteField} \newline
\spadtype{FiniteFieldExtension} \newline
\spadtype{FiniteFieldCyclicGroup} \newline
\spadtype{FiniteFieldCyclicGroupExtension} \newline
\spadtype{FiniteFieldNormalBasis} \newline
\spadtype{FiniteFieldNormalBasisExtension}

\hangafter=1\hangindent=2pc
Constructors for which you provide a polynomial:
\newline
\spadtype{FiniteFieldExtensionByPolynomial} \newline
\spadtype{FiniteFieldCyclicGroupExtensionByPolynomial} \newline
\spadtype{FiniteFieldNormalBasisExtensionByPolynomial}

These constructors are discussed in the following sections where
we collect together descriptions of extension fields that have the
same underlying representation.\footnote{For
more information on the implementation aspects of finite
fields, see
J. Grabmeier, A. Scheerhorn, {\it Finite Fields in AXIOM,}
Technical Report, IBM Heidelberg Scientific Center, 1992.}

If you don't really care about all this detail, just use
\spadtype{FiniteField}.
As your knowledge of your application and its \Language{} implementation
grows, you can come back and choose an alternative constructor that
may improve the efficiency of your code.
Note that the exported operations are almost the same for all constructors
of finite field extensions and include the operations exported by
\spadtype{PrimeField}.

% *********************************************************************
\head{subsection}{Irreducible Modulus Polynomial Representations}{ugxProblemFiniteModulus}
% *********************************************************************

All finite field extension constructors discussed in this
\index{finite field}
section
\index{field!finite!extension of}
use a representation that performs arithmetic with univariate
(one-variable) polynomials modulo an irreducible polynomial.
This polynomial may be given explicitly by you or automatically
generated.
The ground field may be the prime field or one you specify.
See \spadref{ugxProblemFiniteExtensionFinite} for general
information about finite field extensions.

For \spadtype{FiniteField} (abbreviation \spadtype{FF}) you provide a
prime number \smath{p} and an extension degree \smath{n}.
This degree can be 1.
%
\begin{xtc}
\begin{xtccomment}
\Language{} uses the prime field \spadtype{PrimeField(p)},
here \spadtype{PrimeField 2},
and it chooses an irreducible polynomial of degree \smath{n},
here 12, over the ground field.
\end{xtccomment}
\begin{spadsrc}
GF4096 := FF(2,12); 
\end{spadsrc}
\formatResultType{Type}
\end{xtc}
%

The objects in the generated field extension are polynomials
of degree at most \smath{n-1} with coefficients in the
prime field.
The polynomial indeterminate is automatically chosen by \Language{} and
is typically something like \spad{%A} or \spad{%D}.
These (strange) variables are {\it only} for output display;
there are several ways to construct elements of this field.

The operation \spadfun{index} enumerates the elements of the field
extension and accepts as argument the integers from 1 to
\smath{p^n}.
%
\begin{xtc}
\begin{xtccomment}
The expression
\spad{index(p)} always gives the indeterminate.
\end{xtccomment}
\begin{spadsrc}
a := index(2)$GF4096 
\end{spadsrc}
\begin{TeXOutput}
\begin{fricasmath}{2}
\SYMBOL{\%JO}%
\end{fricasmath}
\end{TeXOutput}
\formatResultType{FiniteField(2, 12)}
\end{xtc}
%
%
\begin{xtc}
\begin{xtccomment}
You can build polynomials in \smath{a} and calculate in
\spad{GF4096}.
\end{xtccomment}
\begin{spadsrc}
b := a^12 - a^5 + a 
\end{spadsrc}
\begin{TeXOutput}
\begin{fricasmath}{3}
\SUPER{\SYMBOL{\%JO}}{5}+\SUPER{\SYMBOL{\%JO}}{3}+\SYMBOL{\%JO}+1%
\end{fricasmath}
\end{TeXOutput}
\formatResultType{FiniteField(2, 12)}
\end{xtc}
\begin{xtc}
\begin{xtccomment}
\end{xtccomment}
\begin{spadsrc}
b ^ 1000 
\end{spadsrc}
\begin{TeXOutput}
\begin{fricasmath}{4}
\SUPER{\SYMBOL{\%JO}}{10}+\SUPER{\SYMBOL{\%JO}}{9}+\SUPER{\SYMBOL{\%JO}}{7}+%
\SUPER{\SYMBOL{\%JO}}{5}+\SUPER{\SYMBOL{\%JO}}{4}+\SUPER{\SYMBOL{\%JO}}{3}+%
\SYMBOL{\%JO}%
\end{fricasmath}
\end{TeXOutput}
\formatResultType{FiniteField(2, 12)}
\end{xtc}
\begin{xtc}
\begin{xtccomment}
\end{xtccomment}
\begin{spadsrc}
c := a/b 
\end{spadsrc}
\begin{TeXOutput}
\begin{fricasmath}{5}
\SUPER{\SYMBOL{\%JO}}{11}+\SUPER{\SYMBOL{\%JO}}{8}+\SUPER{\SYMBOL{\%JO}}{7}+%
\SUPER{\SYMBOL{\%JO}}{5}+\SUPER{\SYMBOL{\%JO}}{4}+\SUPER{\SYMBOL{\%JO}}{3}+%
\SUPER{\SYMBOL{\%JO}}{2}%
\end{fricasmath}
\end{TeXOutput}
\formatResultType{FiniteField(2, 12)}
\end{xtc}
%
\begin{xtc}
\begin{xtccomment}
Among the available operations are \spadfun{norm} and \spadfun{trace}.
\end{xtccomment}
\begin{spadsrc}
norm c 
\end{spadsrc}
\begin{TeXOutput}
\begin{fricasmath}{6}
1%
\end{fricasmath}
\end{TeXOutput}
\formatResultType{PrimeField(2)}
\end{xtc}
\begin{xtc}
\begin{xtccomment}
\end{xtccomment}
\begin{spadsrc}
trace c 
\end{spadsrc}
\begin{TeXOutput}
\begin{fricasmath}{7}
0%
\end{fricasmath}
\end{TeXOutput}
\formatResultType{PrimeField(2)}
\end{xtc}
%
%

Since any nonzero element is a power of a primitive element, how
do we discover what the exponent is?
%
\begin{xtc}
\begin{xtccomment}
The operation \spadfun{discreteLog} calculates
\index{discrete logarithm}
the exponent and,
\index{logarithm!discrete}
if it is called with only one argument, always refers to the primitive
element returned by \spadfun{primitiveElement}.
\end{xtccomment}
\begin{spadsrc}
dL := discreteLog a 
\end{spadsrc}
\begin{TeXOutput}
\begin{fricasmath}{8}
1729%
\end{fricasmath}
\end{TeXOutput}
\formatResultType{PositiveInteger}
\end{xtc}
\begin{xtc}
\begin{xtccomment}
\end{xtccomment}
\begin{spadsrc}
g ^ dL 
\end{spadsrc}
\begin{TeXOutput}
\begin{fricasmath}{9}
\SUPER{\SYMBOL{g}}{1729}%
\end{fricasmath}
\end{TeXOutput}
\formatResultType{Polynomial(Integer)}
\end{xtc}

\spadtype{FiniteFieldExtension} (abbreviation \spadtype{FFX}) is
similar to \spadtype{FiniteField} except that the
ground-field for \spadtype{FiniteFieldExtension} is arbitrary and
chosen by you.
%
\begin{xtc}
\begin{xtccomment}
In case you select the prime field as ground field, there is
essentially no difference between the constructed two finite field
extensions.
\end{xtccomment}
\begin{spadsrc}
GF16 := FF(2,4); 
\end{spadsrc}
\formatResultType{Type}
\end{xtc}
\begin{xtc}
\begin{xtccomment}
\end{xtccomment}
\begin{spadsrc}
GF4096 := FFX(GF16,3); 
\end{spadsrc}
\formatResultType{Type}
\end{xtc}
\begin{xtc}
\begin{xtccomment}
\end{xtccomment}
\begin{spadsrc}
r := (random()$GF4096) ^ 20 
\end{spadsrc}
\begin{TeXOutput}
\begin{fricasmath}{12}
\SUPER{\SYMBOL{\%JP}}{3}\TIMES \SUPER{\SYMBOL{\%JQ}}{2}+\SYMBOL{\%JQ}+\SYMBOL%
{\%JP}+1%
\end{fricasmath}
\end{TeXOutput}
\formatResultType{FiniteFieldExtension(FiniteField(2, 4), 3)}
\end{xtc}
\begin{xtc}
\begin{xtccomment}
\end{xtccomment}
\begin{spadsrc}
norm(r) 
\end{spadsrc}
\begin{TeXOutput}
\begin{fricasmath}{13}
\SUPER{\SYMBOL{\%JP}}{2}+\SYMBOL{\%JP}%
\end{fricasmath}
\end{TeXOutput}
\formatResultType{FiniteField(2, 4)}
\end{xtc}
%

\spadtype{FiniteFieldExtensionByPolynomial} (abbreviation \spadtype{FFP})
is similar to \spadtype{FiniteField} and \spadtype{FiniteFieldExtension}
but is more general.
%
\begin{xtc}
\begin{xtccomment}
\end{xtccomment}
\begin{spadsrc}
GF4 := FF(2,2); 
\end{spadsrc}
\formatResultType{Type}
\end{xtc}
\begin{xtc}
\begin{xtccomment}
\end{xtccomment}
\begin{spadsrc}
f := nextIrreduciblePoly(random(6)$FFPOLY(GF4))$FFPOLY(GF4) 
\end{spadsrc}
\begin{TeXOutput}
\begin{fricasmath}{15}
\SUPER{\STRING{?}}{6}+\PAREN{\SYMBOL{\%JR}+1}\TIMES \SUPER{\STRING{?}}{4}+%
\PAREN{\SYMBOL{\%JR}+1}\TIMES \STRING{?}+\SYMBOL{\%JR}%
\end{fricasmath}
\end{TeXOutput}
\formatResultType{Union(SparseUnivariatePolynomial(FiniteField(2, 2)), ...)}
\end{xtc}
\begin{xtc}
\begin{xtccomment}
For \spadtype{FFP} you choose both the
ground field and the irreducible polynomial used in the representation.
The degree of the extension is the degree of the polynomial.
\end{xtccomment}
\begin{spadsrc}
GF4096 := FFP(GF4,f); 
\end{spadsrc}
\formatResultType{Type}
\end{xtc}
\begin{xtc}
\begin{xtccomment}
\end{xtccomment}
\begin{spadsrc}
discreteLog random()$GF4096 
\end{spadsrc}
\begin{TeXOutput}
\begin{fricasmath}{17}
2798%
\end{fricasmath}
\end{TeXOutput}
\formatResultType{PositiveInteger}
\end{xtc}
%

% *********************************************************************
\head{subsection}{Cyclic Group Representations}{ugxProblemFiniteCyclic}
% *********************************************************************
\index{finite field}
\index{field!finite!extension of}

In every finite field there exist elements whose powers are all the
nonzero elements of the field.
Such an element is called a {\it primitive element}.

In \spadtype{FiniteFieldCyclicGroup} (abbreviation \spadtype{FFCG})
\index{group!cyclic}
the nonzero elements are represented by the
powers of a fixed primitive
\index{element!primitive}
element
\index{primitive element}
of the field (that is, a generator of its
cyclic multiplicative group).
Multiplication (and hence exponentiation) using this representation is easy.
To do addition, we consider our primitive element as the root of a primitive
polynomial (an irreducible polynomial whose
roots are all primitive).
See \spadref{ugxProblemFiniteUtility} for examples of how to
compute such a polynomial.

%
\begin{xtc}
\begin{xtccomment}
To use \spadtype{FiniteFieldCyclicGroup} you provide a prime number and an
extension degree.
\end{xtccomment}
\begin{spadsrc}
GF81 := FFCG(3,4); 
\end{spadsrc}
\formatResultType{Type}
\end{xtc}
%
%
\begin{xtc}
\begin{xtccomment}
\Language{} uses the prime field, here \spadtype{PrimeField 3}, as the
ground field and it chooses a primitive polynomial of degree
\smath{n}, here 4, over the prime field.
\end{xtccomment}
\begin{spadsrc}
a := primitiveElement()$GF81 
\end{spadsrc}
\begin{TeXOutput}
\begin{fricasmath}{2}
\SUPER{\SYMBOL{\%JT}}{1}%
\end{fricasmath}
\end{TeXOutput}
\formatResultType{FiniteFieldCyclicGroup(3, 4)}
\end{xtc}
%
%
\begin{xtc}
\begin{xtccomment}
You can calculate in \spad{GF81}.
\end{xtccomment}
\begin{spadsrc}
b  := a^12 - a^5 + a 
\end{spadsrc}
\begin{TeXOutput}
\begin{fricasmath}{3}
\SUPER{\SYMBOL{\%JT}}{72}%
\end{fricasmath}
\end{TeXOutput}
\formatResultType{FiniteFieldCyclicGroup(3, 4)}
\end{xtc}
%
\begin{xtc}
\begin{xtccomment}
In this representation of finite fields the discrete logarithm
of an element can be seen directly in its output form.
\end{xtccomment}
\begin{spadsrc}
b 
\end{spadsrc}
\begin{TeXOutput}
\begin{fricasmath}{4}
\SUPER{\SYMBOL{\%JT}}{72}%
\end{fricasmath}
\end{TeXOutput}
\formatResultType{FiniteFieldCyclicGroup(3, 4)}
\end{xtc}
\begin{xtc}
\begin{xtccomment}
\end{xtccomment}
\begin{spadsrc}
discreteLog b 
\end{spadsrc}
\begin{TeXOutput}
\begin{fricasmath}{5}
72%
\end{fricasmath}
\end{TeXOutput}
\formatResultType{PositiveInteger}
\end{xtc}
%

\spadtype{FiniteFieldCyclicGroupExtension} (abbreviation
\spadtype{FFCGX}) is similar to \spadtype{FiniteFieldCyclicGroup}
except that the ground field for
\spadtype{FiniteFieldCyclicGroupExtension} is arbitrary and chosen
by you.
In case you select the prime field as ground field, there is
essentially no difference between the constructed two finite field
extensions.
%
\begin{xtc}
\begin{xtccomment}
\end{xtccomment}
\begin{spadsrc}
GF9 := FF(3,2); 
\end{spadsrc}
\formatResultType{Type}
\end{xtc}
\begin{xtc}
\begin{xtccomment}
\end{xtccomment}
\begin{spadsrc}
GF729 := FFCGX(GF9,3); 
\end{spadsrc}
\formatResultType{Type}
\end{xtc}
\begin{xtc}
\begin{xtccomment}
\end{xtccomment}
\begin{spadsrc}
r := (random()$GF729) ^ 20 
\end{spadsrc}
\begin{TeXOutput}
\begin{fricasmath}{8}
\SUPER{\SYMBOL{\%JV}}{624}%
\end{fricasmath}
\end{TeXOutput}
\formatResultType{FiniteFieldCyclicGroupExtension(FiniteField(3, 2), 3)}
\end{xtc}
\begin{xtc}
\begin{xtccomment}
\end{xtccomment}
\begin{spadsrc}
trace(r) 
\end{spadsrc}
\begin{TeXOutput}
\begin{fricasmath}{9}
2\TIMES \SYMBOL{\%JU}+1%
\end{fricasmath}
\end{TeXOutput}
\formatResultType{FiniteField(3, 2)}
\end{xtc}
%

\spadtype{FiniteFieldCyclicGroupExtensionByPolynomial}
(abbreviation \spadtype{FFCGP})
is similar to \spadtype{FiniteFieldCyclicGroup} and
\spadtype{FiniteFieldCyclicGroupExtension}
but is more general.
For \spadtype{FiniteFieldCyclicGroupExtensionByPolynomial} you choose both the
ground field and the irreducible polynomial used in the representation.
The degree of the extension is the degree of the polynomial.
%
\begin{xtc}
\begin{xtccomment}
\end{xtccomment}
\begin{spadsrc}
GF3  := PrimeField 3; 
\end{spadsrc}
\formatResultType{Type}
\end{xtc}
\begin{xtc}
\begin{xtccomment}
We use a utility operation to generate an irreducible primitive
polynomial (see \spadref{ugxProblemFiniteUtility}).
The polynomial has one variable that is ``anonymous'': it displays
as a question mark.
\end{xtccomment}
\begin{spadsrc}
f := createPrimitivePoly(4)$FFPOLY(GF3) 
\end{spadsrc}
\begin{TeXOutput}
\begin{fricasmath}{11}
\SUPER{\STRING{?}}{4}+\STRING{?}+2%
\end{fricasmath}
\end{TeXOutput}
\formatResultType{SparseUnivariatePolynomial(PrimeField(3))}
\end{xtc}
\begin{xtc}
\begin{xtccomment}
\end{xtccomment}
\begin{spadsrc}
GF81 := FFCGP(GF3,f); 
\end{spadsrc}
\formatResultType{Type}
\end{xtc}
\begin{xtc}
\begin{xtccomment}
Let's look at a random element from this field.
\end{xtccomment}
\begin{spadsrc}
random()$GF81 
\end{spadsrc}
\begin{TeXOutput}
\begin{fricasmath}{13}
\SUPER{\SYMBOL{\%JT}}{79}%
\end{fricasmath}
\end{TeXOutput}
\formatResultType{FiniteFieldCyclicGroupExtensionByPolynomial(PrimeField(3), ?^4+?+2)}
\end{xtc}
%

% *********************************************************************
\head{subsection}{Normal Basis Representations}{ugxProblemFiniteNormal}
% *********************************************************************
\index{finite field}
\index{field!finite!extension of}
\index{basis!normal}
\index{normal basis}

Let \smath{K} be a finite extension of degree \smath{n} of the
finite field \smath{F} and let \smath{F} have \smath{q}
elements.
An element \smath{x} of \smath{K} is said to be
{\it normal} over \smath{F} if the elements
\begin{displaymath}1, x^q, x^{q^2}, \ldots, x^{q^{n-1}}
\end{displaymath}
form a basis of \smath{K} as a vector space over \smath{F}.
Such a basis is called a {\it normal basis}.\footnote{This
agrees with the general definition of a normal basis because the
\smath{n} distinct powers of the automorphism
$x \mapsto x^q$
constitute the Galois group of \smath{K/F}.}

If \smath{x} is normal over \smath{F}, its minimal
\index{polynomial!minimal}
polynomial is also said to be {\it normal} over \smath{F}.
\index{minimal polynomial}
There exist normal bases for all finite extensions of arbitrary
finite fields.

In \spadtype{FiniteFieldNormalBasis} (abbreviation
\spadtype{FFNB}), the elements of the finite field are represented
by coordinate vectors with respect to a normal basis.

\begin{xtc}
\begin{xtccomment}
You provide a prime \smath{p} and an extension degree
\smath{n}.
\end{xtccomment}
\begin{spadsrc}
K := FFNB(3,8) 
\end{spadsrc}
\begin{TeXOutput}
\begin{fricasmath}{1}
\STRING{FiniteFieldNormalBasis(3,8)}%
\end{fricasmath}
\end{TeXOutput}
\formatResultType{Type}
\end{xtc}
%
\Language{} uses the prime field \spadtype{PrimeField(p)},
here \spadtype{PrimeField 3},
and it chooses a normal polynomial of degree
\smath{n}, here 8, over the ground field.
The remainder class of the indeterminate is used
as the normal element.
The polynomial indeterminate is automatically chosen by \Language{} and
is typically something like \spad{%A} or \spad{%D}.
These (strange) variables are only for output display;
there are several ways to construct elements of this field.
The output of the basis elements is something like
$\%A^{q^i}$.
%
\begin{xtc}
\begin{xtccomment}
\end{xtccomment}
\begin{spadsrc}
a := normalElement()$K 
\end{spadsrc}
\begin{TeXOutput}
\begin{fricasmath}{2}
\SYMBOL{\%JW}%
\end{fricasmath}
\end{TeXOutput}
\formatResultType{FiniteFieldNormalBasis(3, 8)}
\end{xtc}
%
%
\begin{xtc}
\begin{xtccomment}
You can calculate in \smath{K} using \smath{a}.
\end{xtccomment}
\begin{spadsrc}
b  := a^12 - a^5 + a 
\end{spadsrc}
\begin{TeXOutput}
\begin{fricasmath}{3}
2\TIMES \SUPER{\SYMBOL{\%JW}}{\SUPER{\SYMBOL{q}}{7}}+\SUPER{\SYMBOL{\%JW}}{%
\SUPER{\SYMBOL{q}}{5}}+\SUPER{\SYMBOL{\%JW}}{\SYMBOL{q}}%
\end{fricasmath}
\end{TeXOutput}
\formatResultType{FiniteFieldNormalBasis(3, 8)}
\end{xtc}

\spadtype{FiniteFieldNormalBasisExtension} (abbreviation
\spadtype{FFNBX}) is
similar to \spadtype{FiniteFieldNormalBasis} except that the
groundfield for \spadtype{FiniteFieldNormalBasisExtension} is arbitrary and
chosen by you.
In case you select the prime field as ground field, there is
essentially no difference between the constructed two finite field
extensions.
\begin{xtc}
\begin{xtccomment}
\end{xtccomment}
\begin{spadsrc}
GF9 := FFNB(3,2); 
\end{spadsrc}
\formatResultType{Type}
\end{xtc}
\begin{xtc}
\begin{xtccomment}
\end{xtccomment}
\begin{spadsrc}
GF729 := FFNBX(GF9,3); 
\end{spadsrc}
\formatResultType{Type}
\end{xtc}
\begin{xtc}
\begin{xtccomment}
\end{xtccomment}
\begin{spadsrc}
r := random()$GF729 
\end{spadsrc}
\begin{TeXOutput}
\begin{fricasmath}{6}
\SUPER{\SYMBOL{\%JX}}{\SYMBOL{q}}\TIMES \SUPER{\SYMBOL{\%JY}}{\SUPER{\SYMBOL{%
q}}{2}}+\SYMBOL{\%JX}\TIMES \SUPER{\SYMBOL{\%JY}}{\SYMBOL{q}}+2\TIMES \SYMBOL%
{\%JX}\TIMES \SYMBOL{\%JY}%
\end{fricasmath}
\end{TeXOutput}
\formatResultType{FiniteFieldNormalBasisExtension(FiniteFieldNormalBasis(3, 2), 3)}
\end{xtc}
\begin{xtc}
\begin{xtccomment}
\end{xtccomment}
\begin{spadsrc}
r + r^3 + r^9 + r^27 
\end{spadsrc}
\begin{TeXOutput}
\begin{fricasmath}{7}
2\TIMES \SYMBOL{\%JX}\TIMES \SUPER{\SYMBOL{\%JY}}{\SUPER{\SYMBOL{q}}{2}}+%
\SUPER{\SYMBOL{\%JY}}{\SYMBOL{q}}+\PAREN{\SUPER{\SYMBOL{\%JX}}{\SYMBOL{q}}+2%
\TIMES \SYMBOL{\%JX}}\TIMES \SYMBOL{\%JY}%
\end{fricasmath}
\end{TeXOutput}
\formatResultType{FiniteFieldNormalBasisExtension(FiniteFieldNormalBasis(3, 2), 3)}
\end{xtc}

\spadtype{FiniteFieldNormalBasisExtensionByPolynomial}
(abbreviation \spadtype{FFNBP}) is similar to
\spadtype{FiniteFieldNormalBasis} and
\spadtype{FiniteFieldNormalBasisExtension} but is more general.
For \spadtype{FiniteFieldNormalBasisExtensionByPolynomial} you
choose both the ground field and the irreducible polynomial used
in the representation.
The degree of the extension is the degree of the polynomial.

%
\begin{xtc}
\begin{xtccomment}
\end{xtccomment}
\begin{spadsrc}
GF3 := PrimeField 3; 
\end{spadsrc}
\formatResultType{Type}
\end{xtc}
\begin{xtc}
\begin{xtccomment}
We use a utility operation to generate an irreducible normal
polynomial (see \spadref{ugxProblemFiniteUtility}).
The polynomial has one variable that is ``anonymous'': it displays
as a question mark.
\end{xtccomment}
\begin{spadsrc}
f := createNormalPoly(4)$FFPOLY(GF3) 
\end{spadsrc}
\begin{TeXOutput}
\begin{fricasmath}{9}
\SUPER{\STRING{?}}{4}+2\TIMES \SUPER{\STRING{?}}{3}+2%
\end{fricasmath}
\end{TeXOutput}
\formatResultType{SparseUnivariatePolynomial(PrimeField(3))}
\end{xtc}
\begin{xtc}
\begin{xtccomment}
\end{xtccomment}
\begin{spadsrc}
GF81 := FFNBP(GF3,f); 
\end{spadsrc}
\formatResultType{Type}
\end{xtc}
\begin{xtc}
\begin{xtccomment}
Let's look at a random element from this field.
\end{xtccomment}
\begin{spadsrc}
r := random()$GF81 
\end{spadsrc}
\begin{TeXOutput}
\begin{fricasmath}{11}
2\TIMES \SUPER{\SYMBOL{\%JZ}}{\SUPER{\SYMBOL{q}}{3}}+\SUPER{\SYMBOL{\%JZ}}{%
\SUPER{\SYMBOL{q}}{2}}+\SUPER{\SYMBOL{\%JZ}}{\SYMBOL{q}}+2\TIMES \SYMBOL{\%JZ%
}%
\end{fricasmath}
\end{TeXOutput}
\formatResultType{FiniteFieldNormalBasisExtensionByPolynomial(PrimeField(3), ?^4+2*?^3+2)}
\end{xtc}
\begin{xtc}
\begin{xtccomment}
\end{xtccomment}
\begin{spadsrc}
r * r^3 * r^9 * r^27 
\end{spadsrc}
\begin{TeXOutput}
\begin{fricasmath}{12}
2\TIMES \SUPER{\SYMBOL{\%JZ}}{\SUPER{\SYMBOL{q}}{3}}+2\TIMES \SUPER{\SYMBOL{%
\%JZ}}{\SUPER{\SYMBOL{q}}{2}}+2\TIMES \SUPER{\SYMBOL{\%JZ}}{\SYMBOL{q}}+2%
\TIMES \SYMBOL{\%JZ}%
\end{fricasmath}
\end{TeXOutput}
\formatResultType{FiniteFieldNormalBasisExtensionByPolynomial(PrimeField(3), ?^4+2*?^3+2)}
\end{xtc}
\begin{xtc}
\begin{xtccomment}
\end{xtccomment}
\begin{spadsrc}
norm r 
\end{spadsrc}
\begin{TeXOutput}
\begin{fricasmath}{13}
2%
\end{fricasmath}
\end{TeXOutput}
\formatResultType{PrimeField(3)}
\end{xtc}

% *********************************************************************
\head{subsection}{Conversion Operations for Finite Fields}{ugxProblemFiniteConversion}
% *********************************************************************
\index{field!finite!conversions}
%
\begin{xtc}
\begin{xtccomment}
Let $K$ be a finite field.
\end{xtccomment}
\begin{spadsrc}
K := PrimeField 3 
\end{spadsrc}
\begin{TeXOutput}
\begin{fricasmath}{1}
\STRING{PrimeField(3)}%
\end{fricasmath}
\end{TeXOutput}
\formatResultType{Type}
\end{xtc}
%
An extension field $K_m$ of degree
\smath{m} over $K$ is a subfield of an
extension field $K_n$ of degree \smath{n}
over $K$ if and only if \smath{m} divides
\smath{n}.
\begin{center}
\begin{tabular}{ccc}
$K_n$ \\
$|$ \\
$K_m$ & $\Longleftrightarrow$ & $m | n$ \\
$|$ \\
K
\end{tabular}
\end{center}
\spadtype{FiniteFieldHomomorphisms} provides conversion operations
between different extensions of one
fixed finite ground field and between different representations of
these finite fields.
\begin{xtc}
\begin{xtccomment}
Let's choose \smath{m} and \smath{n},
\end{xtccomment}
\begin{spadsrc}
(m,n) := (4,8) 
\end{spadsrc}
\begin{TeXOutput}
\begin{fricasmath}{2}
8%
\end{fricasmath}
\end{TeXOutput}
\formatResultType{PositiveInteger}
\end{xtc}
\begin{xtc}
\begin{xtccomment}
build the field extensions,
\end{xtccomment}
\begin{spadsrc}
Km := FiniteFieldExtension(K,m) 
\end{spadsrc}
\begin{TeXOutput}
\begin{fricasmath}{3}
\STRING{FiniteFieldExtension(PrimeField(3),4)}%
\end{fricasmath}
\end{TeXOutput}
\formatResultType{Type}
\end{xtc}
\begin{xtc}
\begin{xtccomment}
and pick two random elements from the smaller field.
\end{xtccomment}
\begin{spadsrc}
Kn := FiniteFieldExtension(K,n) 
\end{spadsrc}
\begin{TeXOutput}
\begin{fricasmath}{4}
\STRING{FiniteFieldExtension(PrimeField(3),8)}%
\end{fricasmath}
\end{TeXOutput}
\formatResultType{Type}
\end{xtc}
\begin{xtc}
\begin{xtccomment}
\end{xtccomment}
\begin{spadsrc}
a1 := random()$Km 
\end{spadsrc}
\begin{TeXOutput}
\begin{fricasmath}{5}
\SUPER{\SYMBOL{\%KA}}{3}+\SYMBOL{\%KA}%
\end{fricasmath}
\end{TeXOutput}
\formatResultType{FiniteFieldExtension(PrimeField(3), 4)}
\end{xtc}
\begin{xtc}
\begin{xtccomment}
\end{xtccomment}
\begin{spadsrc}
b1 := random()$Km 
\end{spadsrc}
\begin{TeXOutput}
\begin{fricasmath}{6}
2\TIMES \SUPER{\SYMBOL{\%KA}}{3}+2\TIMES \SUPER{\SYMBOL{\%KA}}{2}+2\TIMES %
\SYMBOL{\%KA}%
\end{fricasmath}
\end{TeXOutput}
\formatResultType{FiniteFieldExtension(PrimeField(3), 4)}
\end{xtc}
%
\begin{xtc}
\begin{xtccomment}
Since \smath{m} divides \smath{n},
$K_m$ is a subfield of $K_n$.
\end{xtccomment}
\begin{spadsrc}
a2 := a1 :: Kn 
\end{spadsrc}
\begin{TeXOutput}
\begin{fricasmath}{7}
\SUPER{\SYMBOL{\%KB}}{6}+2\TIMES \SUPER{\SYMBOL{\%KB}}{4}+2\TIMES \SUPER{%
\SYMBOL{\%KB}}{2}%
\end{fricasmath}
\end{TeXOutput}
\formatResultType{FiniteFieldExtension(PrimeField(3), 8)}
\end{xtc}
\begin{xtc}
\begin{xtccomment}
Therefore we can convert the elements of $K_m$
into elements of $K_n$.
\end{xtccomment}
\begin{spadsrc}
b2 := b1 :: Kn 
\end{spadsrc}
\begin{TeXOutput}
\begin{fricasmath}{8}
2\TIMES \SUPER{\SYMBOL{\%KB}}{6}%
\end{fricasmath}
\end{TeXOutput}
\formatResultType{FiniteFieldExtension(PrimeField(3), 8)}
\end{xtc}
%
%
\begin{xtc}
\begin{xtccomment}
To check this, let's do some arithmetic.
\end{xtccomment}
\begin{spadsrc}
a1+b1 - ((a2+b2) :: Km) 
\end{spadsrc}
\begin{TeXOutput}
\begin{fricasmath}{9}
0%
\end{fricasmath}
\end{TeXOutput}
\formatResultType{FiniteFieldExtension(PrimeField(3), 4)}
\end{xtc}
\begin{xtc}
\begin{xtccomment}
\end{xtccomment}
\begin{spadsrc}
a1*b1 - ((a2*b2) :: Km) 
\end{spadsrc}
\begin{TeXOutput}
\begin{fricasmath}{10}
0%
\end{fricasmath}
\end{TeXOutput}
\formatResultType{FiniteFieldExtension(PrimeField(3), 4)}
\end{xtc}
%
There are also conversions available for the
situation, when $K_m$ and $K_n$
are represented in different ways (see
\spadref{ugxProblemFiniteExtensionFinite}).
For example let's choose $K_m$ where the
representation is 0 plus the cyclic multiplicative group and
$K_n$ with a normal basis representation.
\begin{xtc}
\begin{xtccomment}
\end{xtccomment}
\begin{spadsrc}
Km := FFCGX(K,m) 
\end{spadsrc}
\begin{TeXOutput}
\begin{fricasmath}{11}
\STRING{FiniteFieldCyclicGroupExtension(PrimeField(3),4)}%
\end{fricasmath}
\end{TeXOutput}
\formatResultType{Type}
\end{xtc}
\begin{xtc}
\begin{xtccomment}
\end{xtccomment}
\begin{spadsrc}
Kn := FFNBX(K,n) 
\end{spadsrc}
\begin{TeXOutput}
\begin{fricasmath}{12}
\STRING{FiniteFieldNormalBasisExtension(PrimeField(3),8)}%
\end{fricasmath}
\end{TeXOutput}
\formatResultType{Type}
\end{xtc}
\begin{xtc}
\begin{xtccomment}
\end{xtccomment}
\begin{spadsrc}
(a1,b1) := (random()$Km,random()$Km) 
\end{spadsrc}
\begin{TeXOutput}
\begin{fricasmath}{13}
\SUPER{\SYMBOL{\%JT}}{4}%
\end{fricasmath}
\end{TeXOutput}
\formatResultType{FiniteFieldCyclicGroupExtension(PrimeField(3), 4)}
\end{xtc}
\begin{xtc}
\begin{xtccomment}
\end{xtccomment}
\begin{spadsrc}
a2 := a1 :: Kn 
\end{spadsrc}
\begin{TeXOutput}
\begin{fricasmath}{14}
\SUPER{\SYMBOL{\%KC}}{\SUPER{\SYMBOL{q}}{6}}+\SUPER{\SYMBOL{\%KC}}{\SUPER{%
\SYMBOL{q}}{2}}%
\end{fricasmath}
\end{TeXOutput}
\formatResultType{FiniteFieldNormalBasisExtension(PrimeField(3), 8)}
\end{xtc}
\begin{xtc}
\begin{xtccomment}
\end{xtccomment}
\begin{spadsrc}
b2 := b1 :: Kn 
\end{spadsrc}
\begin{TeXOutput}
\begin{fricasmath}{15}
\SUPER{\SYMBOL{\%KC}}{\SUPER{\SYMBOL{q}}{7}}+\SUPER{\SYMBOL{\%KC}}{\SUPER{%
\SYMBOL{q}}{3}}%
\end{fricasmath}
\end{TeXOutput}
\formatResultType{FiniteFieldNormalBasisExtension(PrimeField(3), 8)}
\end{xtc}
%
\begin{xtc}
\begin{xtccomment}
Check the arithmetic again.
\end{xtccomment}
\begin{spadsrc}
a1+b1 - ((a2+b2) :: Km) 
\end{spadsrc}
\begin{TeXOutput}
\begin{fricasmath}{16}
\STRING{0}%
\end{fricasmath}
\end{TeXOutput}
\formatResultType{FiniteFieldCyclicGroupExtension(PrimeField(3), 4)}
\end{xtc}
\begin{xtc}
\begin{xtccomment}
\end{xtccomment}
\begin{spadsrc}
a1*b1 - ((a2*b2) :: Km) 
\end{spadsrc}
\begin{TeXOutput}
\begin{fricasmath}{17}
\STRING{0}%
\end{fricasmath}
\end{TeXOutput}
\formatResultType{FiniteFieldCyclicGroupExtension(PrimeField(3), 4)}
\end{xtc}

% *********************************************************************
\head{subsection}{Utility Operations for Finite Fields}{ugxProblemFiniteUtility}
% *********************************************************************

\spadtype{FiniteFieldPolynomialPackage} (abbreviation
\spadtype{FFPOLY})
provides operations for generating, counting and testing polynomials
over finite fields. Let's start with a couple of definitions:
\begin{itemize}
\item A polynomial is {\it primitive} if its roots are primitive
\index{polynomial!primitive}
elements in an extension of the coefficient field of degree equal
to the degree of the polynomial.
\item A polynomial is {\it normal} over its coefficient field
\index{polynomial!normal}
if its roots are linearly independent
elements in an extension of the coefficient field of degree equal
to the degree of the polynomial.
\end{itemize}
In what follows, many of the generated polynomials have one
``anonymous'' variable.
This indeterminate is displayed as a question mark (\spadSyntax{?}).

\begin{xtc}
\begin{xtccomment}
To fix ideas, let's use the field with five elements for the first
few examples.
\end{xtccomment}
\begin{spadsrc}
GF5 := PF 5; 
\end{spadsrc}
\formatResultType{Type}
\end{xtc}
%
%
\begin{xtc}
\begin{xtccomment}
You can generate irreducible polynomials of any (positive) degree
\index{polynomial!irreducible}
(within the storage capabilities of the computer and your ability
to wait) by using
\spadfunFrom{createIrreduciblePoly}{FiniteFieldPolynomialPackage}.
\end{xtccomment}
\begin{spadsrc}
f := createIrreduciblePoly(8)$FFPOLY(GF5) 
\end{spadsrc}
\begin{TeXOutput}
\begin{fricasmath}{2}
\SUPER{\STRING{?}}{8}+\SUPER{\STRING{?}}{4}+2%
\end{fricasmath}
\end{TeXOutput}
\formatResultType{SparseUnivariatePolynomial(PrimeField(5))}
\end{xtc}
%
\begin{xtc}
\begin{xtccomment}
Does this polynomial have other important properties? Use
\spadfun{primitive?} to test whether it is a primitive polynomial.
\end{xtccomment}
\begin{spadsrc}
primitive?(f)$FFPOLY(GF5) 
\end{spadsrc}
\begin{TeXOutput}
\begin{fricasmath}{3}
\STRING{false}%
\end{fricasmath}
\end{TeXOutput}
\formatResultType{Boolean}
\end{xtc}
\begin{xtc}
\begin{xtccomment}
Use \spadfun{normal?} to test whether it is a normal polynomial.
\end{xtccomment}
\begin{spadsrc}
normal?(f)$FFPOLY(GF5) 
\end{spadsrc}
\begin{TeXOutput}
\begin{fricasmath}{4}
\STRING{false}%
\end{fricasmath}
\end{TeXOutput}
\formatResultType{Boolean}
\end{xtc}
\noindent
Note that this is actually a trivial case,
because a normal polynomial of degree \smath{n}
must have a nonzero term of degree \smath{n-1}.
We will refer back to this later.

\begin{xtc}
\begin{xtccomment}
To get a primitive polynomial of degree 8 just issue this.
\end{xtccomment}
\begin{spadsrc}
p := createPrimitivePoly(8)$FFPOLY(GF5) 
\end{spadsrc}
\begin{TeXOutput}
\begin{fricasmath}{5}
\SUPER{\STRING{?}}{8}+\SUPER{\STRING{?}}{3}+\SUPER{\STRING{?}}{2}+\STRING{?}+%
2%
\end{fricasmath}
\end{TeXOutput}
\formatResultType{SparseUnivariatePolynomial(PrimeField(5))}
\end{xtc}
\begin{xtc}
\begin{xtccomment}
\end{xtccomment}
\begin{spadsrc}
primitive?(p)$FFPOLY(GF5) 
\end{spadsrc}
\begin{TeXOutput}
\begin{fricasmath}{6}
\STRING{true}%
\end{fricasmath}
\end{TeXOutput}
\formatResultType{Boolean}
\end{xtc}
\begin{xtc}
\begin{xtccomment}
This polynomial is not normal,
\end{xtccomment}
\begin{spadsrc}
normal?(p)$FFPOLY(GF5) 
\end{spadsrc}
\begin{TeXOutput}
\begin{fricasmath}{7}
\STRING{false}%
\end{fricasmath}
\end{TeXOutput}
\formatResultType{Boolean}
\end{xtc}
\begin{xtc}
\begin{xtccomment}
but if you want a normal one simply write this.
\end{xtccomment}
\begin{spadsrc}
n := createNormalPoly(8)$FFPOLY(GF5) 
\end{spadsrc}
\begin{TeXOutput}
\begin{fricasmath}{8}
\SUPER{\STRING{?}}{8}+4\TIMES \SUPER{\STRING{?}}{7}+\SUPER{\STRING{?}}{3}+1%
\end{fricasmath}
\end{TeXOutput}
\formatResultType{SparseUnivariatePolynomial(PrimeField(5))}
\end{xtc}
\begin{xtc}
\begin{xtccomment}
This polynomial is not primitive!
\end{xtccomment}
\begin{spadsrc}
primitive?(n)$FFPOLY(GF5) 
\end{spadsrc}
\begin{TeXOutput}
\begin{fricasmath}{9}
\STRING{false}%
\end{fricasmath}
\end{TeXOutput}
\formatResultType{Boolean}
\end{xtc}
This could have been seen directly, as
the constant term is 1 here, which is not a primitive
element up to the factor (\spad{-1}) raised to the degree of the
polynomial.\footnote{Cf. Lidl, R. \& Niederreiter,
H., {\it Finite Fields,} Encycl. of Math. 20, (Addison-Wesley, 1983),
p.90, Th. 3.18.}

What about
polynomials that are both primitive and normal?
The existence of such a polynomial is by no means obvious.
\footnote{The existence of such polynomials is proved in
Lenstra, H. W. \& Schoof, R. J., {\it Primitive
Normal Bases for Finite Fields,} Math. Comp. 48, 1987, pp. 217-231.}
%
\begin{xtc}
\begin{xtccomment}
If you really need one use either
\spadfunFrom{createPrimitiveNormalPoly}{FiniteFieldPolynomialPackage} or
\spadfunFrom{createNormalPrimitivePoly}{FiniteFieldPolynomialPackage}.
\end{xtccomment}
\begin{spadsrc}
createPrimitiveNormalPoly(8)$FFPOLY(GF5) 
\end{spadsrc}
\begin{TeXOutput}
\begin{fricasmath}{10}
\SUPER{\STRING{?}}{8}+4\TIMES \SUPER{\STRING{?}}{7}+2\TIMES \SUPER{\STRING{?}%
}{5}+2%
\end{fricasmath}
\end{TeXOutput}
\formatResultType{SparseUnivariatePolynomial(PrimeField(5))}
\end{xtc}
%

If you want to obtain additional polynomials of the various types above
as given by the {\bf create...} operations above, you can use the {\bf
next...} operations.
For instance,
\spadfunFrom{nextIrreduciblePoly}{FiniteFieldPolynomialPackage} yields
the next monic irreducible polynomial with the same degree as the input
polynomial.
By ``next'' we mean ``next in a natural order using the terms and
coefficients.''
This will become more clear in the following examples.

\begin{xtc}
\begin{xtccomment}
This is the field with five elements.
\end{xtccomment}
\begin{spadsrc}
GF5 := PF 5; 
\end{spadsrc}
\formatResultType{Type}
\end{xtc}
%
\begin{xtc}
\begin{xtccomment}
Our first example irreducible polynomial, say
of degree 3, must be ``greater'' than this.
\end{xtccomment}
\begin{spadsrc}
h := monomial(1,8)$SUP(GF5) 
\end{spadsrc}
\begin{TeXOutput}
\begin{fricasmath}{12}
\SUPER{\STRING{?}}{8}%
\end{fricasmath}
\end{TeXOutput}
\formatResultType{SparseUnivariatePolynomial(PrimeField(5))}
\end{xtc}
\begin{xtc}
\begin{xtccomment}
You can generate it by doing this.
\end{xtccomment}
\begin{spadsrc}
nh := nextIrreduciblePoly(h)$FFPOLY(GF5) 
\end{spadsrc}
\begin{TeXOutput}
\begin{fricasmath}{13}
\SUPER{\STRING{?}}{8}+2%
\end{fricasmath}
\end{TeXOutput}
\formatResultType{Union(SparseUnivariatePolynomial(PrimeField(5)), ...)}
\end{xtc}
%
\begin{xtc}
\begin{xtccomment}
Notice that this polynomial is not the same as the one
\spadfunFrom{createIrreduciblePoly}{FiniteFieldPolynomialPackage}.
\end{xtccomment}
\begin{spadsrc}
createIrreduciblePoly(3)$FFPOLY(GF5) 
\end{spadsrc}
\begin{TeXOutput}
\begin{fricasmath}{14}
\SUPER{\STRING{?}}{3}+\STRING{?}+1%
\end{fricasmath}
\end{TeXOutput}
\formatResultType{SparseUnivariatePolynomial(PrimeField(5))}
\end{xtc}
\begin{xtc}
\begin{xtccomment}
You can step through all irreducible polynomials of degree 8 over
the field with 5 elements by repeatedly issuing this.
\end{xtccomment}
\begin{spadsrc}
nh := nextIrreduciblePoly(nh)$FFPOLY(GF5) 
\end{spadsrc}
\begin{TeXOutput}
\begin{fricasmath}{15}
\SUPER{\STRING{?}}{8}+3%
\end{fricasmath}
\end{TeXOutput}
\formatResultType{Union(SparseUnivariatePolynomial(PrimeField(5)), ...)}
\end{xtc}
\begin{xtc}
\begin{xtccomment}
You could also ask for the total number of these.
\end{xtccomment}
\begin{spadsrc}
numberOfIrreduciblePoly(5)$FFPOLY(GF5) 
\end{spadsrc}
\begin{TeXOutput}
\begin{fricasmath}{16}
624%
\end{fricasmath}
\end{TeXOutput}
\formatResultType{PositiveInteger}
\end{xtc}

We hope that ``natural order'' on polynomials is now clear:
first we compare the number of monomials of
two polynomials (``more'' is ``greater'');
then, if necessary, the degrees of these monomials (lexicographically),
and lastly their coefficients (also
lexicographically, and using the operation \spadfun{lookup} if
our field is not a prime field).
Also note that we make both polynomials monic before looking at the
coefficients:
multiplying either polynomial  by a nonzero constant
produces the same result.

%
\begin{xtc}
\begin{xtccomment}
The package
\spadtype{FiniteFieldPolynomialPackage} also provides similar
operations for primitive and normal polynomials. With
the exception of the number of primitive normal polynomials;
we're not aware of any known formula for this.
\end{xtccomment}
\begin{spadsrc}
numberOfPrimitivePoly(3)$FFPOLY(GF5) 
\end{spadsrc}
\begin{TeXOutput}
\begin{fricasmath}{17}
20%
\end{fricasmath}
\end{TeXOutput}
\formatResultType{PositiveInteger}
\end{xtc}
%
%
\begin{xtc}
\begin{xtccomment}
Take these,
\end{xtccomment}
\begin{spadsrc}
m := monomial(1,1)$SUP(GF5) 
\end{spadsrc}
\begin{TeXOutput}
\begin{fricasmath}{18}
\STRING{?}%
\end{fricasmath}
\end{TeXOutput}
\formatResultType{SparseUnivariatePolynomial(PrimeField(5))}
\end{xtc}
\begin{xtc}
\begin{xtccomment}
\end{xtccomment}
\begin{spadsrc}
f := m^3 + 4*m^2 + m + 2 
\end{spadsrc}
\begin{TeXOutput}
\begin{fricasmath}{19}
\SUPER{\STRING{?}}{3}+4\TIMES \SUPER{\STRING{?}}{2}+\STRING{?}+2%
\end{fricasmath}
\end{TeXOutput}
\formatResultType{SparseUnivariatePolynomial(PrimeField(5))}
\end{xtc}
%
%
\begin{xtc}
\begin{xtccomment}
and then we have:
\end{xtccomment}
\begin{spadsrc}
f1 := nextPrimitivePoly(f)$FFPOLY(GF5) 
\end{spadsrc}
\begin{TeXOutput}
\begin{fricasmath}{20}
\SUPER{\STRING{?}}{3}+4\TIMES \SUPER{\STRING{?}}{2}+4\TIMES \STRING{?}+2%
\end{fricasmath}
\end{TeXOutput}
\formatResultType{Union(SparseUnivariatePolynomial(PrimeField(5)), ...)}
\end{xtc}
\begin{xtc}
\begin{xtccomment}
What happened?
\end{xtccomment}
\begin{spadsrc}
nextPrimitivePoly(f1)$FFPOLY(GF5) 
\end{spadsrc}
\begin{TeXOutput}
\begin{fricasmath}{21}
\SUPER{\STRING{?}}{3}+2\TIMES \SUPER{\STRING{?}}{2}+3%
\end{fricasmath}
\end{TeXOutput}
\formatResultType{Union(SparseUnivariatePolynomial(PrimeField(5)), ...)}
\end{xtc}
%
Well, for the ordering used in
\spadfunFrom{nextPrimitivePoly}{FiniteFieldPolynomialPackage} we
use as first criterion a comparison of the constant terms of the
polynomials.
Analogously, in
\spadfunFrom{nextNormalPoly}{FiniteFieldPolynomialPackage} we first
compare the monomials of degree 1 less than the degree of the
polynomials (which is nonzero, by an earlier remark).
%
\begin{xtc}
\begin{xtccomment}
\end{xtccomment}
\begin{spadsrc}
f := m^3 + m^2 + 4*m + 1 
\end{spadsrc}
\begin{TeXOutput}
\begin{fricasmath}{22}
\SUPER{\STRING{?}}{3}+\SUPER{\STRING{?}}{2}+4\TIMES \STRING{?}+1%
\end{fricasmath}
\end{TeXOutput}
\formatResultType{SparseUnivariatePolynomial(PrimeField(5))}
\end{xtc}
\begin{xtc}
\begin{xtccomment}
\end{xtccomment}
\begin{spadsrc}
f1 := nextNormalPoly(f)$FFPOLY(GF5) 
\end{spadsrc}
\begin{TeXOutput}
\begin{fricasmath}{23}
\SUPER{\STRING{?}}{3}+\SUPER{\STRING{?}}{2}+4\TIMES \STRING{?}+3%
\end{fricasmath}
\end{TeXOutput}
\formatResultType{Union(SparseUnivariatePolynomial(PrimeField(5)), ...)}
\end{xtc}
\begin{xtc}
\begin{xtccomment}
\end{xtccomment}
\begin{spadsrc}
nextNormalPoly(f1)$FFPOLY(GF5) 
\end{spadsrc}
\begin{TeXOutput}
\begin{fricasmath}{24}
\SUPER{\STRING{?}}{3}+2\TIMES \SUPER{\STRING{?}}{2}+1%
\end{fricasmath}
\end{TeXOutput}
\formatResultType{Union(SparseUnivariatePolynomial(PrimeField(5)), ...)}
\end{xtc}
%
\noindent
We don't have to restrict ourselves to prime fields.
%
\begin{xtc}
\begin{xtccomment}
Let's consider, say, a field with 16 elements.
\end{xtccomment}
\begin{spadsrc}
GF16 := FFX(FFX(PF 2,2),2); 
\end{spadsrc}
\formatResultType{Type}
\end{xtc}
%
%
\begin{xtc}
\begin{xtccomment}
We can apply any of the operations described above.
\end{xtccomment}
\begin{spadsrc}
createIrreduciblePoly(5)$FFPOLY(GF16) 
\end{spadsrc}
\begin{TeXOutput}
\begin{fricasmath}{26}
\SUPER{\STRING{?}}{5}+\SYMBOL{\%KE}%
\end{fricasmath}
\end{TeXOutput}
\formatResultType{SparseUnivariatePolynomial(FiniteFieldExtension(FiniteFieldExtension(PrimeField(2), 2), 2))}
\end{xtc}

\begin{xtc}
\begin{xtccomment}
\Language{} also provides operations
for producing random polynomials of a given degree
\end{xtccomment}
\begin{spadsrc}
random(5)$FFPOLY(GF16) 
\end{spadsrc}
\begin{TeXOutput}
\begin{fricasmath}{27}
\SUPER{\STRING{?}}{5}+\SYMBOL{\%KE}\TIMES \SUPER{\STRING{?}}{4}+\SYMBOL{\%JR}%
\TIMES \SYMBOL{\%KE}\TIMES \SUPER{\STRING{?}}{2}+\PAREN{\PAREN{\SYMBOL{\%JR}+%
1}\TIMES \SYMBOL{\%KE}+1}\TIMES \STRING{?}+\PAREN{\SYMBOL{\%JR}+1}\TIMES %
\SYMBOL{\%KE}+\SYMBOL{\%JR}+1%
\end{fricasmath}
\end{TeXOutput}
\formatResultType{SparseUnivariatePolynomial(FiniteFieldExtension(FiniteFieldExtension(PrimeField(2), 2), 2))}
\end{xtc}
\begin{xtc}
\begin{xtccomment}
or with degree between two given bounds.
\end{xtccomment}
\begin{spadsrc}
random(3,9)$FFPOLY(GF16) 
\end{spadsrc}
\begin{TeXOutput}
\begin{fricasmath}{28}
\SUPER{\STRING{?}}{8}+\PAREN{\SYMBOL{\%JR}\TIMES \SYMBOL{\%KE}+1}\TIMES %
\SUPER{\STRING{?}}{7}+\PAREN{\SYMBOL{\%KE}+1}\TIMES \SUPER{\STRING{?}}{6}+%
\PAREN{\PAREN{\SYMBOL{\%JR}+1}\TIMES \SYMBOL{\%KE}+\SYMBOL{\%JR}+1}\TIMES %
\SUPER{\STRING{?}}{5}+\PAREN{\SYMBOL{\%KE}+1}\TIMES \SUPER{\STRING{?}}{4}+%
\PAREN{\SYMBOL{\%JR}+1}\TIMES \SUPER{\STRING{?}}{3}+\PAREN{\SYMBOL{\%KE}+1}%
\TIMES \SUPER{\STRING{?}}{2}+\PAREN{\SYMBOL{\%KE}+\SYMBOL{\%JR}}\TIMES %
\STRING{?}+\PAREN{\SYMBOL{\%JR}+1}\TIMES \SYMBOL{\%KE}+\SYMBOL{\%JR}+1%
\end{fricasmath}
\end{TeXOutput}
\formatResultType{SparseUnivariatePolynomial(FiniteFieldExtension(FiniteFieldExtension(PrimeField(2), 2), 2))}
\end{xtc}

\spadtype{FiniteFieldPolynomialPackage2} (abbreviation
\spadtype{FFPOLY2})
exports an operation \spadfun{rootOfIrreduciblePoly}
for finding one root of an irreducible polynomial \spad{f}
\index{polynomial!root of}
in an extension field of the coefficient field.
The degree of the extension has to be a multiple of the degree of \spad{f}.
It is not checked whether \spad{f} actually is irreducible.

%
\begin{xtc}
\begin{xtccomment}
To illustrate this operation, we fix a ground field \spad{GF}
\end{xtccomment}
\begin{spadsrc}
GF2 := PrimeField 2; 
\end{spadsrc}
\formatResultType{Type}
\end{xtc}
%
%
\begin{xtc}
\begin{xtccomment}
and then an extension field.
\end{xtccomment}
\begin{spadsrc}
F := FFX(GF2,12) 
\end{spadsrc}
\begin{TeXOutput}
\begin{fricasmath}{30}
\STRING{FiniteFieldExtension(PrimeField(2),12)}%
\end{fricasmath}
\end{TeXOutput}
\formatResultType{Type}
\end{xtc}
%
%
\begin{xtc}
\begin{xtccomment}
We construct an irreducible polynomial over \spad{GF2}.
\end{xtccomment}
\begin{spadsrc}
f := createIrreduciblePoly(6)$FFPOLY(GF2) 
\end{spadsrc}
\begin{TeXOutput}
\begin{fricasmath}{31}
\SUPER{\STRING{?}}{6}+\STRING{?}+1%
\end{fricasmath}
\end{TeXOutput}
\formatResultType{SparseUnivariatePolynomial(PrimeField(2))}
\end{xtc}
%
%
\begin{xtc}
\begin{xtccomment}
We compute a root of \spad{f}.
\end{xtccomment}
\begin{spadsrc}
root := rootOfIrreduciblePoly(f)$FFPOLY2(F,GF2) 
\end{spadsrc}
\begin{TeXOutput}
\begin{fricasmath}{32}
\SUPER{\SYMBOL{\%JO}}{11}+\SUPER{\SYMBOL{\%JO}}{8}+\SUPER{\SYMBOL{\%JO}}{7}+%
\SUPER{\SYMBOL{\%JO}}{5}+\SYMBOL{\%JO}+1%
\end{fricasmath}
\end{TeXOutput}
\formatResultType{FiniteFieldExtension(PrimeField(2), 12)}
\end{xtc}
%
%and check the result
%\spadcommand{eval(f, monomial(1,1)\$SUP(F) = root) \free{fz F root}}

%*********************************************************************
\head{section}{Primary Decomposition of Ideals}{ugProblemIdeal}
%*********************************************************************
%
\Language{} provides a facility for the primary decomposition
\index{ideal!primary decomposition}
of
\index{primary decomposition of ideal}
polynomial ideals over fields of characteristic zero.
The algorithm
%is discussed in \cite{gtz:gbpdpi} and
works in essentially two steps:
\begin{enumerate}
\item the problem is solved for 0-dimensional ideals by ``generic''
projection on the last coordinate
\item a ``reduction process'' uses localization and ideal quotients
to reduce the general case to the 0-dimensional one.
\end{enumerate}
The \Language{} constructor \spadtype{PolynomialIdeal}
represents ideals with coefficients in any field and
supports the basic ideal operations,
including intersection, sum and quotient.
\spadtype{IdealDecompositionPackage} contains the specific
operations for the primary decomposition and the computation of the
radical of an ideal with polynomial
coefficients in a field of characteristic 0 with
an effective algorithm for factoring polynomials.

The following examples illustrate the capabilities of this facility.
%
\begin{xtc}
\begin{xtccomment}
First consider the ideal generated by
$x^2 + y^2 - 1$
(which defines a circle in the \spad{(x,y)}-plane) and the ideal
generated by $x^2 - y^2$ (corresponding to the
straight lines \spad{x = y} and \spad{x = -y}.
\end{xtccomment}
\begin{spadsrc}
(n,m) : List DMP([x,y],FRAC INT) 
\end{spadsrc}
\end{xtc}
\begin{xtc}
\begin{xtccomment}
\end{xtccomment}
\begin{spadsrc}
m := [x^2+y^2-1] 
\end{spadsrc}
\begin{TeXOutput}
\begin{fricasmath}{2}
\BRACKET{\SUPER{\SYMBOL{x}}{2}+\SUPER{\SYMBOL{y}}{2}-{1}}%
\end{fricasmath}
\end{TeXOutput}
\formatResultType{List(DistributedMultivariatePolynomial([x, y], Fraction(Integer)))}
\end{xtc}
\begin{xtc}
\begin{xtccomment}
\end{xtccomment}
\begin{spadsrc}
n := [x^2-y^2] 
\end{spadsrc}
\begin{TeXOutput}
\begin{fricasmath}{3}
\BRACKET{\SUPER{\SYMBOL{x}}{2}-{\SUPER{\SYMBOL{y}}{2}}}%
\end{fricasmath}
\end{TeXOutput}
\formatResultType{List(DistributedMultivariatePolynomial([x, y], Fraction(Integer)))}
\end{xtc}
%
%
\begin{xtc}
\begin{xtccomment}
We find the equations defining the intersection of the two loci.
This correspond to the sum of the associated ideals.
\end{xtccomment}
\begin{spadsrc}
id := ideal m  + ideal n 
\end{spadsrc}
\begin{TeXOutput}
\begin{fricasmath}{4}
\BRACKET{\SUPER{\SYMBOL{x}}{2}-{\frac{1}{2}}\COMMA \SUPER{\SYMBOL{y}}{2}-{%
\frac{1}{2}}}%
\end{fricasmath}
\end{TeXOutput}
\formatResultType{PolynomialIdeal(Fraction(Integer), DirectProduct(2, NonNegativeInteger), OrderedVariableList([x, y]), DistributedMultivariatePolynomial([x, y], Fraction(Integer)))}
\end{xtc}
%
%
\begin{xtc}
\begin{xtccomment}
We can check if the locus contains only a finite number of points,
that is, if the ideal is zero-dimensional.
\end{xtccomment}
\begin{spadsrc}
zeroDim? id 
\end{spadsrc}
\begin{TeXOutput}
\begin{fricasmath}{5}
\STRING{true}%
\end{fricasmath}
\end{TeXOutput}
\formatResultType{Boolean}
\end{xtc}
\begin{xtc}
\begin{xtccomment}
\end{xtccomment}
\begin{spadsrc}
zeroDim?(ideal m) 
\end{spadsrc}
\begin{TeXOutput}
\begin{fricasmath}{6}
\STRING{false}%
\end{fricasmath}
\end{TeXOutput}
\formatResultType{Boolean}
\end{xtc}
\begin{xtc}
\begin{xtccomment}
\end{xtccomment}
\begin{spadsrc}
dimension ideal m 
\end{spadsrc}
\begin{TeXOutput}
\begin{fricasmath}{7}
1%
\end{fricasmath}
\end{TeXOutput}
\formatResultType{PositiveInteger}
\end{xtc}
\begin{xtc}
\begin{xtccomment}
We can find polynomial relations among the generators
(\spad{f} and \spad{g} are the parametric equations of the knot).
\end{xtccomment}
\begin{spadsrc}
(f,g):DMP([x,y],FRAC INT) 
\end{spadsrc}
\end{xtc}
\begin{xtc}
\begin{xtccomment}
\end{xtccomment}
\begin{spadsrc}
f := x^2-1 
\end{spadsrc}
\begin{TeXOutput}
\begin{fricasmath}{9}
\SUPER{\SYMBOL{x}}{2}-{1}%
\end{fricasmath}
\end{TeXOutput}
\formatResultType{DistributedMultivariatePolynomial([x, y], Fraction(Integer))}
\end{xtc}
\begin{xtc}
\begin{xtccomment}
\end{xtccomment}
\begin{spadsrc}
g := x*(x^2-1) 
\end{spadsrc}
\begin{TeXOutput}
\begin{fricasmath}{10}
\SUPER{\SYMBOL{x}}{3}-{\SYMBOL{x}}%
\end{fricasmath}
\end{TeXOutput}
\formatResultType{DistributedMultivariatePolynomial([x, y], Fraction(Integer))}
\end{xtc}
\begin{xtc}
\begin{xtccomment}
\end{xtccomment}
\begin{spadsrc}
relationsIdeal [f,g] 
\end{spadsrc}
\begin{TeXOutput}
\begin{fricasmath}{11}
\BRACKET{-{\SUPER{\SYMBOL{\%KG}}{2}}+\SUPER{\SYMBOL{\%KF}}{3}+\SUPER{\SYMBOL{%
\%KF}}{2}}\mid \BRACKET{\SYMBOL{\%KF}=\SUPER{\SYMBOL{x}}{2}-{1}\COMMA \SYMBOL%
{\%KG}=\SUPER{\SYMBOL{x}}{3}-{\SYMBOL{x}}}%
\end{fricasmath}
\end{TeXOutput}
\formatResultType{SuchThat(List(Polynomial(Fraction(Integer))), List(Equation(Polynomial(Fraction(Integer)))))}
\end{xtc}

\begin{xtc}
\begin{xtccomment}
We can compute the primary decomposition of an ideal.
\end{xtccomment}
\begin{spadsrc}
l: List DMP([x,y,z],FRAC INT) := [x^2+2*y^2,x*z^2-y*z,z^2-4] 
\end{spadsrc}
\begin{TeXOutput}
\begin{fricasmath}{12}
\BRACKET{\SUPER{\SYMBOL{x}}{2}+2\TIMES \SUPER{\SYMBOL{y}}{2}\COMMA \SYMBOL{x}%
\TIMES \SUPER{\SYMBOL{z}}{2}-{\SYMBOL{y}\TIMES \SYMBOL{z}}\COMMA \SUPER{%
\SYMBOL{z}}{2}-{4}}%
\end{fricasmath}
\end{TeXOutput}
\formatResultType{List(DistributedMultivariatePolynomial([x, y, z], Fraction(Integer)))}
\end{xtc}
\begin{xtc}
\begin{xtccomment}
\end{xtccomment}
\begin{spadsrc}
ld:=primaryDecomp(ideal l)$IdealDecompositionPackage([x,y,z]) 
\end{spadsrc}
\begin{TeXOutput}
\begin{fricasmath}{13}
\BRACKET{\BRACKET{\SYMBOL{x}+\frac{1}{2}\TIMES \SYMBOL{y}\COMMA \SUPER{%
\SYMBOL{y}}{2}\COMMA \SYMBOL{z}+2}\COMMA \BRACKET{\SYMBOL{x}-{\frac{1}{2}%
\TIMES \SYMBOL{y}}\COMMA \SUPER{\SYMBOL{y}}{2}\COMMA \SYMBOL{z}-{2}}}%
\end{fricasmath}
\end{TeXOutput}
\formatResultType{List(PolynomialIdeal(Fraction(Integer), DirectProduct(3, NonNegativeInteger), OrderedVariableList([x, y, z]), DistributedMultivariatePolynomial([x, y, z], Fraction(Integer))))}
\end{xtc}
\begin{xtc}
\begin{xtccomment}
We can intersect back.
\end{xtccomment}
\begin{spadsrc}
reduce(intersect,ld) 
\end{spadsrc}
\begin{TeXOutput}
\begin{fricasmath}{14}
\BRACKET{\SYMBOL{x}-{\frac{1}{4}\TIMES \SYMBOL{y}\TIMES \SYMBOL{z}}\COMMA %
\SUPER{\SYMBOL{y}}{2}\COMMA \SUPER{\SYMBOL{z}}{2}-{4}}%
\end{fricasmath}
\end{TeXOutput}
\formatResultType{PolynomialIdeal(Fraction(Integer), DirectProduct(3, NonNegativeInteger), OrderedVariableList([x, y, z]), DistributedMultivariatePolynomial([x, y, z], Fraction(Integer)))}
\end{xtc}

\begin{xtc}
\begin{xtccomment}
We can compute the radical of every primary component.
\end{xtccomment}
\begin{spadsrc}
reduce(intersect,[radical(ld.i)$IdealDecompositionPackage([x,y,z]) for i in 1..2]) 
\end{spadsrc}
\begin{TeXOutput}
\begin{fricasmath}{15}
\BRACKET{\SYMBOL{x}\COMMA \SYMBOL{y}\COMMA \SUPER{\SYMBOL{z}}{2}-{4}}%
\end{fricasmath}
\end{TeXOutput}
\formatResultType{PolynomialIdeal(Fraction(Integer), DirectProduct(3, NonNegativeInteger), OrderedVariableList([x, y, z]), DistributedMultivariatePolynomial([x, y, z], Fraction(Integer)))}
\end{xtc}
\begin{xtc}
\begin{xtccomment}
Their intersection is equal to the radical of the ideal of \spad{l}.
\end{xtccomment}
\begin{spadsrc}
radical(ideal l)$IdealDecompositionPackage([x,y,z]) 
\end{spadsrc}
\begin{TeXOutput}
\begin{fricasmath}{16}
\BRACKET{\SYMBOL{x}\COMMA \SYMBOL{y}\COMMA \SUPER{\SYMBOL{z}}{2}-{4}}%
\end{fricasmath}
\end{TeXOutput}
\formatResultType{PolynomialIdeal(Fraction(Integer), DirectProduct(3, NonNegativeInteger), OrderedVariableList([x, y, z]), DistributedMultivariatePolynomial([x, y, z], Fraction(Integer)))}
\end{xtc}

% *********************************************************************
\head{section}{Computation of Galois Groups}{ugProblemGalois}
% *********************************************************************
%
As a sample use of \Language{}'s algebraic number facilities,
\index{group!Galois}
we compute
\index{Galois!group}
the Galois group of the polynomial
$p(x) = x^5 - 5 x + 12$.
%
\begin{xtc}
\begin{xtccomment}
\end{xtccomment}
\begin{spadsrc}
p := x^5 - 5*x + 12 
\end{spadsrc}
\begin{TeXOutput}
\begin{fricasmath}{1}
\SUPER{\SYMBOL{x}}{5}-{5\TIMES \SYMBOL{x}}+12%
\end{fricasmath}
\end{TeXOutput}
\formatResultType{Polynomial(Integer)}
\end{xtc}
%
We would like to construct a polynomial \smath{f(x)}
such that the splitting
\index{field!splitting}
field
\index{splitting field}
of \smath{p(x)} is generated by one root of \smath{f(x)}.
First we construct a polynomial \smath{r = r(x)} such that one
root of \smath{r(x)} generates the field generated by two roots of
the polynomial \smath{p(x)}.
(As it will turn out, the field generated by two roots of
\smath{p(x)} is, in fact, the splitting field of
\smath{p(x)}.)

From the proof of the primitive element theorem we know that
if \smath{a} and \smath{b} are
algebraic numbers, then the field
${\bf Q}(a,b)$ is equal to
${\bf Q}(a+kb)$ for an
appropriately chosen integer \smath{k}.
In our case, we construct the minimal polynomial of
$a_i - a_j$, where
$a_i$ and
$a_j$ are two roots of \smath{p(x)}.
We construct this polynomial using \spadfun{resultant}.
The main result we  need is the following:
If \smath{f(x)} is a polynomial with roots
$a_i \ldots a_m$ and
\smath{g(x)} is a polynomial
with roots
$b_i \ldots b_n$, then the polynomial
\spad{h(x) = resultant(f(y), g(x-y), y)}
is a polynomial of degree \smath{m*n} with
roots
$a_i + b_j, i = 1 \ldots m, j = 1 \ldots n$.

\begin{xtc}
\begin{xtccomment}
For \smath{f(x)} we use the polynomial \smath{p(x)}.
For \smath{g(x)} we use the polynomial \smath{-p(-x)}.
Thus, the polynomial we  first construct is
\spad{resultant(p(y), -p(y-x), y)}.
\end{xtccomment}
\begin{spadsrc}
q := resultant(eval(p,x,y),-eval(p,x,y-x),y) 
\end{spadsrc}
\begin{TeXOutput}
\begin{fricasmath}{2}
\SUPER{\SYMBOL{x}}{25}-{50\TIMES \SUPER{\SYMBOL{x}}{21}}-{2375\TIMES \SUPER{%
\SYMBOL{x}}{17}}+90000\TIMES \SUPER{\SYMBOL{x}}{15}-{5000\TIMES \SUPER{%
\SYMBOL{x}}{13}}+2700000\TIMES \SUPER{\SYMBOL{x}}{11}+250000\TIMES \SUPER{%
\SYMBOL{x}}{9}+18000000\TIMES \SUPER{\SYMBOL{x}}{7}+64000000\TIMES \SUPER{%
\SYMBOL{x}}{5}%
\end{fricasmath}
\end{TeXOutput}
\formatResultType{Polynomial(Integer)}
\end{xtc}
%
The roots of \smath{q(x)} are
$a_i - a_j, i \leq 1, j \leq 5$.
Of course, there are five pairs \smath{(i,j)} with \smath{i = j},
so \spad{0} is a 5-fold root of \smath{q(x)}.
%
\begin{xtc}
\begin{xtccomment}
Let's get rid of this factor.
\end{xtccomment}
\begin{spadsrc}
q1 := exquo(q, x^5) 
\end{spadsrc}
\begin{TeXOutput}
\begin{fricasmath}{3}
\SUPER{\SYMBOL{x}}{20}-{50\TIMES \SUPER{\SYMBOL{x}}{16}}-{2375\TIMES \SUPER{%
\SYMBOL{x}}{12}}+90000\TIMES \SUPER{\SYMBOL{x}}{10}-{5000\TIMES \SUPER{%
\SYMBOL{x}}{8}}+2700000\TIMES \SUPER{\SYMBOL{x}}{6}+250000\TIMES \SUPER{%
\SYMBOL{x}}{4}+18000000\TIMES \SUPER{\SYMBOL{x}}{2}+64000000%
\end{fricasmath}
\end{TeXOutput}
\formatResultType{Union(Polynomial(Integer), ...)}
\end{xtc}
\begin{xtc}
\begin{xtccomment}
Factor the polynomial \spad{q1}.
\end{xtccomment}
\begin{spadsrc}
factoredQ := factor q1 
\end{spadsrc}
\begin{TeXOutput}
\begin{fricasmath}{4}
\PAREN{\SUPER{\SYMBOL{x}}{10}-{10\TIMES \SUPER{\SYMBOL{x}}{8}}-{75\TIMES %
\SUPER{\SYMBOL{x}}{6}}+1500\TIMES \SUPER{\SYMBOL{x}}{4}-{5500\TIMES \SUPER{%
\SYMBOL{x}}{2}}+16000}\TIMES \PAREN{\SUPER{\SYMBOL{x}}{10}+10\TIMES \SUPER{%
\SYMBOL{x}}{8}+125\TIMES \SUPER{\SYMBOL{x}}{6}+500\TIMES \SUPER{\SYMBOL{x}}{4%
}+2500\TIMES \SUPER{\SYMBOL{x}}{2}+4000}%
\end{fricasmath}
\end{TeXOutput}
\formatResultType{Factored(Polynomial(Integer))}
\end{xtc}
%
We see that \spad{q1} has two irreducible factors, each of degree \spad{10}.
(The fact that the polynomial \spad{q1} has two factors of
degree \spad{10} is enough to show
that the Galois group of \smath{p(x)} is the dihedral group of
order \spad{10}.\footnote{See McKay, Soicher,  Computing Galois Groups
over the Rationals, Journal of Number Theory 20, 273-281 (1983).
We do not assume the results of this paper, however, and we continue with
the computation.}
Note that the type of \spad{factoredQ} is \spadtype{FR POLY INT}, that is,
\spadtype{Factored Polynomial Integer}.
\exptypeindex{Factored}
This is a special data type for recording factorizations of polynomials with
integer coefficients (see \xmpref{Factored}).
%
\begin{xtc}
\begin{xtccomment}
We can access the individual factors using the operation
\spadfunFrom{nthFactor}{Factored}.
\end{xtccomment}
\begin{spadsrc}
r := nthFactor(factoredQ,1) 
\end{spadsrc}
\begin{TeXOutput}
\begin{fricasmath}{5}
\SUPER{\SYMBOL{x}}{10}-{10\TIMES \SUPER{\SYMBOL{x}}{8}}-{75\TIMES \SUPER{%
\SYMBOL{x}}{6}}+1500\TIMES \SUPER{\SYMBOL{x}}{4}-{5500\TIMES \SUPER{\SYMBOL{x%
}}{2}}+16000%
\end{fricasmath}
\end{TeXOutput}
\formatResultType{Polynomial(Integer)}
\end{xtc}
%

Consider the polynomial \smath{r = r(x)}.
This is the minimal polynomial of the difference of two roots of
\smath{p(x)}.
Thus, the splitting field of \smath{p(x)} contains a subfield of
degree \spad{10}.
We show that this subfield is, in fact, the splitting field of
\smath{p(x)} by showing that \smath{p(x)} factors completely
over this field.
%
\begin{xtc}
\begin{xtccomment}
First we create a symbolic root of the polynomial \smath{r(x)}.
(We replaced \spad{x} by \spad{b} in the
polynomial \spad{r} so that our symbolic root would be
printed as \spad{b}.)
\end{xtccomment}
\begin{spadsrc}
beta:AN := rootOf(eval(r,x,b)) 
\end{spadsrc}
\begin{TeXOutput}
\begin{fricasmath}{6}
\SYMBOL{b}%
\end{fricasmath}
\end{TeXOutput}
\formatResultType{AlgebraicNumber}
\end{xtc}
\begin{xtc}
\begin{xtccomment}
We next tell \Language{} to view \smath{p(x)} as a univariate polynomial
in \spad{x}
with algebraic number coefficients.
This is accomplished with this type declaration.
\end{xtccomment}
\begin{spadsrc}
p := p::UP(x,INT)::UP(x,AN) 
\end{spadsrc}
\begin{TeXOutput}
\begin{fricasmath}{7}
\SUPER{\SYMBOL{x}}{5}-{5\TIMES \SYMBOL{x}}+12%
\end{fricasmath}
\end{TeXOutput}
\formatResultType{UnivariatePolynomial(x, AlgebraicNumber)}
\end{xtc}
%
%
\begin{xtc}
\begin{xtccomment}
Factor \smath{p(x)} over the field
${\bf Q}(\beta)$.
(This computation will take some time!)
\end{xtccomment}
\begin{spadsrc}
algFactors := factor(p,[beta]) 
\end{spadsrc}
\begin{TeXOutput}
\begin{fricasmath}{8}
\PAREN{\SYMBOL{x}+\frac{-{85\TIMES \SUPER{\SYMBOL{b}}{9}}-{116\TIMES \SUPER{%
\SYMBOL{b}}{8}}+780\TIMES \SUPER{\SYMBOL{b}}{7}+2640\TIMES \SUPER{\SYMBOL{b}%
}{6}+14895\TIMES \SUPER{\SYMBOL{b}}{5}-{8820\TIMES \SUPER{\SYMBOL{b}}{4}}-{%
127050\TIMES \SUPER{\SYMBOL{b}}{3}}-{327000\TIMES \SUPER{\SYMBOL{b}}{2}}-{%
405200\TIMES \SYMBOL{b}}+2062400}{1339200}}\TIMES \PAREN{\SYMBOL{x}+\frac{-{%
17\TIMES \SUPER{\SYMBOL{b}}{8}}+156\TIMES \SUPER{\SYMBOL{b}}{6}+2979\TIMES %
\SUPER{\SYMBOL{b}}{4}-{25410\TIMES \SUPER{\SYMBOL{b}}{2}}-{14080}}{66960}}%
\TIMES \PAREN{\SYMBOL{x}+\frac{143\TIMES \SUPER{\SYMBOL{b}}{8}-{2100\TIMES %
\SUPER{\SYMBOL{b}}{6}}-{10485\TIMES \SUPER{\SYMBOL{b}}{4}}+290550\TIMES %
\SUPER{\SYMBOL{b}}{2}-{334800\TIMES \SYMBOL{b}}-{960800}}{669600}}\TIMES %
\PAREN{\SYMBOL{x}+\frac{143\TIMES \SUPER{\SYMBOL{b}}{8}-{2100\TIMES \SUPER{%
\SYMBOL{b}}{6}}-{10485\TIMES \SUPER{\SYMBOL{b}}{4}}+290550\TIMES \SUPER{%
\SYMBOL{b}}{2}+334800\TIMES \SYMBOL{b}-{960800}}{669600}}\TIMES \PAREN{%
\SYMBOL{x}+\frac{85\TIMES \SUPER{\SYMBOL{b}}{9}-{116\TIMES \SUPER{\SYMBOL{b}%
}{8}}-{780\TIMES \SUPER{\SYMBOL{b}}{7}}+2640\TIMES \SUPER{\SYMBOL{b}}{6}-{%
14895\TIMES \SUPER{\SYMBOL{b}}{5}}-{8820\TIMES \SUPER{\SYMBOL{b}}{4}}+127050%
\TIMES \SUPER{\SYMBOL{b}}{3}-{327000\TIMES \SUPER{\SYMBOL{b}}{2}}+405200%
\TIMES \SYMBOL{b}+2062400}{1339200}}%
\end{fricasmath}
\end{TeXOutput}
\formatResultType{Factored(UnivariatePolynomial(x, AlgebraicNumber))}
\end{xtc}
%
When factoring over number fields, it is important to specify the
field over which the polynomial is to be factored, as polynomials
have different factorizations over different fields.
When you use the operation \spadfun{factor}, the field over which
the polynomial is factored is the field generated by
\begin{enumerate}
\item the algebraic numbers that appear
in the coefficients of the polynomial, and
\item the algebraic numbers that
appear in a list passed as an optional second argument of the operation.
\end{enumerate}
In our case, the coefficients of \spad{p}
are all rational integers and only \spad{beta}
appears in the list, so the field is simply
${\bf Q}(\beta)$.
%
\begin{xtc}
\begin{xtccomment}
It was necessary to give the list \spad{[beta]}
as a second argument of the operation
because otherwise the polynomial would have been factored over the field
generated by its coefficients, namely the rational numbers.
\end{xtccomment}
\begin{spadsrc}
factor(p) 
\end{spadsrc}
\begin{TeXOutput}
\begin{fricasmath}{9}
\SUPER{\SYMBOL{x}}{5}-{5\TIMES \SYMBOL{x}}+12%
\end{fricasmath}
\end{TeXOutput}
\formatResultType{Factored(UnivariatePolynomial(x, AlgebraicNumber))}
\end{xtc}
%
We have shown that the splitting field of \smath{p(x)} has degree
\spad{10}.
Since the symmetric group of degree 5 has only one transitive subgroup
of order \spad{10}, we know that the Galois group of \smath{p(x)} must be
this group, the dihedral group
\index{group!dihedral}
of order \spad{10}.
Rather than stop here, we explicitly compute the action of the Galois
group on the roots of \smath{p(x)}.

First we assign the roots of \smath{p(x)} as the values of five
\index{root}
variables.
\begin{xtc}
\begin{xtccomment}
We can obtain an individual root by negating the constant coefficient of
one of the factors of \smath{p(x)}.
\end{xtccomment}
\begin{spadsrc}
factor1 := nthFactor(algFactors,1) 
\end{spadsrc}
\begin{TeXOutput}
\begin{fricasmath}{10}
\SYMBOL{x}+\frac{-{85\TIMES \SUPER{\SYMBOL{b}}{9}}-{116\TIMES \SUPER{\SYMBOL{%
b}}{8}}+780\TIMES \SUPER{\SYMBOL{b}}{7}+2640\TIMES \SUPER{\SYMBOL{b}}{6}+%
14895\TIMES \SUPER{\SYMBOL{b}}{5}-{8820\TIMES \SUPER{\SYMBOL{b}}{4}}-{127050%
\TIMES \SUPER{\SYMBOL{b}}{3}}-{327000\TIMES \SUPER{\SYMBOL{b}}{2}}-{405200%
\TIMES \SYMBOL{b}}+2062400}{1339200}%
\end{fricasmath}
\end{TeXOutput}
\formatResultType{UnivariatePolynomial(x, AlgebraicNumber)}
\end{xtc}
\begin{xtc}
\begin{xtccomment}
\end{xtccomment}
\begin{spadsrc}
root1 := -coefficient(factor1,0) 
\end{spadsrc}
\begin{TeXOutput}
\begin{fricasmath}{11}
\frac{85\TIMES \SUPER{\SYMBOL{b}}{9}+116\TIMES \SUPER{\SYMBOL{b}}{8}-{780%
\TIMES \SUPER{\SYMBOL{b}}{7}}-{2640\TIMES \SUPER{\SYMBOL{b}}{6}}-{14895%
\TIMES \SUPER{\SYMBOL{b}}{5}}+8820\TIMES \SUPER{\SYMBOL{b}}{4}+127050\TIMES %
\SUPER{\SYMBOL{b}}{3}+327000\TIMES \SUPER{\SYMBOL{b}}{2}+405200\TIMES \SYMBOL%
{b}-{2062400}}{1339200}%
\end{fricasmath}
\end{TeXOutput}
\formatResultType{AlgebraicNumber}
\end{xtc}
%
%
\begin{xtc}
\begin{xtccomment}
We can obtain a list of all the roots in this way.
\end{xtccomment}
\begin{spadsrc}
roots := [-coefficient(nthFactor(algFactors,i),0) for i in 1..5] 
\end{spadsrc}
\begin{TeXOutput}
\begin{fricasmath}{12}
\BRACKET{\frac{85\TIMES \SUPER{\SYMBOL{b}}{9}+116\TIMES \SUPER{\SYMBOL{b}}{8}%
-{780\TIMES \SUPER{\SYMBOL{b}}{7}}-{2640\TIMES \SUPER{\SYMBOL{b}}{6}}-{14895%
\TIMES \SUPER{\SYMBOL{b}}{5}}+8820\TIMES \SUPER{\SYMBOL{b}}{4}+127050\TIMES %
\SUPER{\SYMBOL{b}}{3}+327000\TIMES \SUPER{\SYMBOL{b}}{2}+405200\TIMES \SYMBOL%
{b}-{2062400}}{1339200}\COMMA \frac{17\TIMES \SUPER{\SYMBOL{b}}{8}-{156%
\TIMES \SUPER{\SYMBOL{b}}{6}}-{2979\TIMES \SUPER{\SYMBOL{b}}{4}}+25410\TIMES %
\SUPER{\SYMBOL{b}}{2}+14080}{66960}\COMMA \frac{-{143\TIMES \SUPER{\SYMBOL{b}%
}{8}}+2100\TIMES \SUPER{\SYMBOL{b}}{6}+10485\TIMES \SUPER{\SYMBOL{b}}{4}-{%
290550\TIMES \SUPER{\SYMBOL{b}}{2}}+334800\TIMES \SYMBOL{b}+960800}{669600}%
\COMMA \frac{-{143\TIMES \SUPER{\SYMBOL{b}}{8}}+2100\TIMES \SUPER{\SYMBOL{b}%
}{6}+10485\TIMES \SUPER{\SYMBOL{b}}{4}-{290550\TIMES \SUPER{\SYMBOL{b}}{2}}-{%
334800\TIMES \SYMBOL{b}}+960800}{669600}\COMMA \frac{-{85\TIMES \SUPER{%
\SYMBOL{b}}{9}}+116\TIMES \SUPER{\SYMBOL{b}}{8}+780\TIMES \SUPER{\SYMBOL{b}}{%
7}-{2640\TIMES \SUPER{\SYMBOL{b}}{6}}+14895\TIMES \SUPER{\SYMBOL{b}}{5}+8820%
\TIMES \SUPER{\SYMBOL{b}}{4}-{127050\TIMES \SUPER{\SYMBOL{b}}{3}}+327000%
\TIMES \SUPER{\SYMBOL{b}}{2}-{405200\TIMES \SYMBOL{b}}-{2062400}}{1339200}}%
\end{fricasmath}
\end{TeXOutput}
\formatResultType{List(AlgebraicNumber)}
\end{xtc}

The expression
\begin{verbatim}
- coefficient(nthFactor(algFactors, i), 0)}
\end{verbatim}
is the \eth{i} root
of \smath{p(x)} and the elements of \spad{roots} are the \eth{i}
roots of \smath{p(x)} as \spad{i} ranges from \spad{1} to \spad{5}.

\begin{xtc}
\begin{xtccomment}
Assign the roots as the values of the variables \spad{a1,...,a5}.
\end{xtccomment}
\begin{spadsrc}
(a1,a2,a3,a4,a5) := (roots.1,roots.2,roots.3,roots.4,roots.5) 
\end{spadsrc}
\begin{TeXOutput}
\begin{fricasmath}{13}
\frac{-{85\TIMES \SUPER{\SYMBOL{b}}{9}}+116\TIMES \SUPER{\SYMBOL{b}}{8}+780%
\TIMES \SUPER{\SYMBOL{b}}{7}-{2640\TIMES \SUPER{\SYMBOL{b}}{6}}+14895\TIMES %
\SUPER{\SYMBOL{b}}{5}+8820\TIMES \SUPER{\SYMBOL{b}}{4}-{127050\TIMES \SUPER{%
\SYMBOL{b}}{3}}+327000\TIMES \SUPER{\SYMBOL{b}}{2}-{405200\TIMES \SYMBOL{b}}-%
{2062400}}{1339200}%
\end{fricasmath}
\end{TeXOutput}
\formatResultType{AlgebraicNumber}
\end{xtc}
%

Next we express the roots of \smath{r(x)} as polynomials in
\spad{beta}.
We could obtain these roots by calling the operation \spadfun{factor}:
\spad{factor(r, [beta])} factors \spad{r(x)} over
${\bf Q}(\beta)$.
However, this is a lengthy computation and we can obtain the roots of
\smath{r(x)} as differences of the roots \spad{a1,...,a5} of
\smath{p(x)}.
Only ten of these differences are roots of \smath{r(x)} and the
other ten are roots
of the other irreducible factor of \spad{q1}.
We can determine if a given value is a root of \smath{r(x)} by evaluating
\smath{r(x)} at that particular value.
(Of course, the order in which factors are returned by the
operation \spadfun{factor}
is unimportant and may change with different implementations of the operation.
Therefore, we cannot predict in advance which differences are roots of
\smath{r(x)} and which are not.)
%
\begin{xtc}
\begin{xtccomment}
Let's look at four examples (two are roots of \smath{r(x)} and
two are not).
\end{xtccomment}
\begin{spadsrc}
eval(r,x,a1 - a2) 
\end{spadsrc}
\begin{TeXOutput}
\begin{fricasmath}{14}
0%
\end{fricasmath}
\end{TeXOutput}
\formatResultType{Polynomial(AlgebraicNumber)}
\end{xtc}
\begin{xtc}
\begin{xtccomment}
\end{xtccomment}
\begin{spadsrc}
eval(r,x,a1 - a3) 
\end{spadsrc}
\begin{TeXOutput}
\begin{fricasmath}{15}
\frac{47905\TIMES \SUPER{\SYMBOL{b}}{9}+66920\TIMES \SUPER{\SYMBOL{b}}{8}-{%
536100\TIMES \SUPER{\SYMBOL{b}}{7}}-{980400\TIMES \SUPER{\SYMBOL{b}}{6}}-{%
3345075\TIMES \SUPER{\SYMBOL{b}}{5}}-{5787000\TIMES \SUPER{\SYMBOL{b}}{4}}+%
75572250\TIMES \SUPER{\SYMBOL{b}}{3}+161688000\TIMES \SUPER{\SYMBOL{b}}{2}-{%
184600000\TIMES \SYMBOL{b}}-{710912000}}{4464}%
\end{fricasmath}
\end{TeXOutput}
\formatResultType{Polynomial(AlgebraicNumber)}
\end{xtc}
\begin{xtc}
\begin{xtccomment}
\end{xtccomment}
\begin{spadsrc}
eval(r,x,a1 - a4) 
\end{spadsrc}
\begin{TeXOutput}
\begin{fricasmath}{16}
0%
\end{fricasmath}
\end{TeXOutput}
\formatResultType{Polynomial(AlgebraicNumber)}
\end{xtc}
\begin{xtc}
\begin{xtccomment}
\end{xtccomment}
\begin{spadsrc}
eval(r,x,a1 - a5) 
\end{spadsrc}
\begin{TeXOutput}
\begin{fricasmath}{17}
\frac{405\TIMES \SUPER{\SYMBOL{b}}{8}+3450\TIMES \SUPER{\SYMBOL{b}}{6}-{19875%
\TIMES \SUPER{\SYMBOL{b}}{4}}-{198000\TIMES \SUPER{\SYMBOL{b}}{2}}-{588000}}{%
31}%
\end{fricasmath}
\end{TeXOutput}
\formatResultType{Polynomial(AlgebraicNumber)}
\end{xtc}
%

Take one of the differences that was a root of \smath{r(x)}
and assign it to the variable \spad{bb}.
\begin{xtc}
\begin{xtccomment}
For example, if \spad{eval(r,x,a1 - a4)} returned \spad{0}, you would
enter this.
\end{xtccomment}
\begin{spadsrc}
bb := a1 - a4 
\end{spadsrc}
\begin{TeXOutput}
\begin{fricasmath}{18}
\frac{85\TIMES \SUPER{\SYMBOL{b}}{9}+402\TIMES \SUPER{\SYMBOL{b}}{8}-{780%
\TIMES \SUPER{\SYMBOL{b}}{7}}-{6840\TIMES \SUPER{\SYMBOL{b}}{6}}-{14895%
\TIMES \SUPER{\SYMBOL{b}}{5}}-{12150\TIMES \SUPER{\SYMBOL{b}}{4}}+127050%
\TIMES \SUPER{\SYMBOL{b}}{3}+908100\TIMES \SUPER{\SYMBOL{b}}{2}+1074800%
\TIMES \SYMBOL{b}-{3984000}}{1339200}%
\end{fricasmath}
\end{TeXOutput}
\formatResultType{AlgebraicNumber}
\end{xtc}
Of course, if the difference is, in fact, equal to the root \spad{beta},
you should choose another root of \smath{r(x)}.

Automorphisms of the splitting field are given by mapping a generator of
the field, namely \spad{beta}, to other roots of its minimal polynomial.
Let's see what happens when \spad{beta} is mapped to \spad{bb}.
%
\begin{xtc}
\begin{xtccomment}
We compute the images of the roots \spad{a1,...,a5}
under this automorphism:
\end{xtccomment}
\begin{spadsrc}
aa1 := subst(a1,beta = bb) 
\end{spadsrc}
\begin{TeXOutput}
\begin{fricasmath}{19}
\frac{-{143\TIMES \SUPER{\SYMBOL{b}}{8}}+2100\TIMES \SUPER{\SYMBOL{b}}{6}+%
10485\TIMES \SUPER{\SYMBOL{b}}{4}-{290550\TIMES \SUPER{\SYMBOL{b}}{2}}+334800%
\TIMES \SYMBOL{b}+960800}{669600}%
\end{fricasmath}
\end{TeXOutput}
\formatResultType{AlgebraicNumber}
\end{xtc}
\begin{xtc}
\begin{xtccomment}
\end{xtccomment}
\begin{spadsrc}
aa2 := subst(a2,beta = bb) 
\end{spadsrc}
\begin{TeXOutput}
\begin{fricasmath}{20}
\frac{-{85\TIMES \SUPER{\SYMBOL{b}}{9}}+116\TIMES \SUPER{\SYMBOL{b}}{8}+780%
\TIMES \SUPER{\SYMBOL{b}}{7}-{2640\TIMES \SUPER{\SYMBOL{b}}{6}}+14895\TIMES %
\SUPER{\SYMBOL{b}}{5}+8820\TIMES \SUPER{\SYMBOL{b}}{4}-{127050\TIMES \SUPER{%
\SYMBOL{b}}{3}}+327000\TIMES \SUPER{\SYMBOL{b}}{2}-{405200\TIMES \SYMBOL{b}}-%
{2062400}}{1339200}%
\end{fricasmath}
\end{TeXOutput}
\formatResultType{AlgebraicNumber}
\end{xtc}
\begin{xtc}
\begin{xtccomment}
\end{xtccomment}
\begin{spadsrc}
aa3 := subst(a3,beta = bb) 
\end{spadsrc}
\begin{TeXOutput}
\begin{fricasmath}{21}
\frac{85\TIMES \SUPER{\SYMBOL{b}}{9}+116\TIMES \SUPER{\SYMBOL{b}}{8}-{780%
\TIMES \SUPER{\SYMBOL{b}}{7}}-{2640\TIMES \SUPER{\SYMBOL{b}}{6}}-{14895%
\TIMES \SUPER{\SYMBOL{b}}{5}}+8820\TIMES \SUPER{\SYMBOL{b}}{4}+127050\TIMES %
\SUPER{\SYMBOL{b}}{3}+327000\TIMES \SUPER{\SYMBOL{b}}{2}+405200\TIMES \SYMBOL%
{b}-{2062400}}{1339200}%
\end{fricasmath}
\end{TeXOutput}
\formatResultType{AlgebraicNumber}
\end{xtc}
\begin{xtc}
\begin{xtccomment}
\end{xtccomment}
\begin{spadsrc}
aa4 := subst(a4,beta = bb) 
\end{spadsrc}
\begin{TeXOutput}
\begin{fricasmath}{22}
\frac{-{143\TIMES \SUPER{\SYMBOL{b}}{8}}+2100\TIMES \SUPER{\SYMBOL{b}}{6}+%
10485\TIMES \SUPER{\SYMBOL{b}}{4}-{290550\TIMES \SUPER{\SYMBOL{b}}{2}}-{%
334800\TIMES \SYMBOL{b}}+960800}{669600}%
\end{fricasmath}
\end{TeXOutput}
\formatResultType{AlgebraicNumber}
\end{xtc}
\begin{xtc}
\begin{xtccomment}
\end{xtccomment}
\begin{spadsrc}
aa5 := subst(a5,beta = bb) 
\end{spadsrc}
\begin{TeXOutput}
\begin{fricasmath}{23}
\frac{17\TIMES \SUPER{\SYMBOL{b}}{8}-{156\TIMES \SUPER{\SYMBOL{b}}{6}}-{2979%
\TIMES \SUPER{\SYMBOL{b}}{4}}+25410\TIMES \SUPER{\SYMBOL{b}}{2}+14080}{66960}%
\end{fricasmath}
\end{TeXOutput}
\formatResultType{AlgebraicNumber}
\end{xtc}
%
Of course, the values \spad{aa1,...,aa5} are simply a permutation of the values
\spad{a1,...,a5}.
%
\begin{xtc}
\begin{xtccomment}
Let's find the value of \spad{aa1} (execute as many of the following five commands
as necessary).
\end{xtccomment}
\begin{spadsrc}
(aa1 = a1) :: Boolean 
\end{spadsrc}
\begin{TeXOutput}
\begin{fricasmath}{24}
\STRING{false}%
\end{fricasmath}
\end{TeXOutput}
\formatResultType{Boolean}
\end{xtc}
\begin{xtc}
\begin{xtccomment}
\end{xtccomment}
\begin{spadsrc}
(aa1 = a2) :: Boolean 
\end{spadsrc}
\begin{TeXOutput}
\begin{fricasmath}{25}
\STRING{false}%
\end{fricasmath}
\end{TeXOutput}
\formatResultType{Boolean}
\end{xtc}
\begin{xtc}
\begin{xtccomment}
\end{xtccomment}
\begin{spadsrc}
(aa1 = a3) :: Boolean 
\end{spadsrc}
\begin{TeXOutput}
\begin{fricasmath}{26}
\STRING{true}%
\end{fricasmath}
\end{TeXOutput}
\formatResultType{Boolean}
\end{xtc}
\begin{xtc}
\begin{xtccomment}
\end{xtccomment}
\begin{spadsrc}
(aa1 = a4) :: Boolean 
\end{spadsrc}
\begin{TeXOutput}
\begin{fricasmath}{27}
\STRING{false}%
\end{fricasmath}
\end{TeXOutput}
\formatResultType{Boolean}
\end{xtc}
\begin{xtc}
\begin{xtccomment}
\end{xtccomment}
\begin{spadsrc}
(aa1 = a5) :: Boolean 
\end{spadsrc}
\begin{TeXOutput}
\begin{fricasmath}{28}
\STRING{false}%
\end{fricasmath}
\end{TeXOutput}
\formatResultType{Boolean}
\end{xtc}
%
Proceeding in this fashion, you can find the values of
\spad{aa2,...aa5}.\footnote{Here you should use the
\Clef{} line editor.
See \spadref{ugAvailCLEF} for more information about \Clef{}.}
You have represented the automorphism \spad{beta -> bb}
as a permutation of the roots \spad{a1,...,a5}.
If you wish, you can repeat this computation for all the roots of
\smath{r(x)} and represent the Galois group of
\smath{p(x)} as a subgroup of the symmetric group on five letters.

Here are two other problems that you may attack in a similar fashion:
\begin{enumerate}
\item Show that the Galois group of
$p(x) = x^4 + 2 x^3 - 2 x^2 - 3 x + 1$
is the dihedral group of order eight.
\index{group!dihedral}
(The splitting field of this polynomial is the Hilbert class field
\index{Hilbert class field}
of
\index{field!Hilbert class}
the quadratic field
${\bf Q}(\sqrt{145})$.)
\item Show that the Galois group of
$p(x) = x^6 + 108$
has order 6 and is
isomorphic to $S_3,$ the symmetric group on three letters.
\index{group!symmetric}
(The splitting field of this polynomial is the splitting field of
$x^3 - 2$.)
\end{enumerate}

% *********************************************************************
\head{section}{Non-Associative Algebras and Modelling Genetic Laws}{ugProblemGenetic}
% *********************************************************************

Many algebraic structures of mathematics and \Language{}
have a multiplication operation \spadop{*} that satisfies
the associativity law
\index{associativity law}
$a*(b*c) = (a*b)*c$
for all \smath{a}, \smath{b} and \smath{c}.
The octonions (see \xmpref{Octonion}) are a well known exception.
There are many other interesting non-associative structures, such as the
class of
\index{Lie algebra}
Lie algebras.\footnote{Two \Language{} implementations of Lie algebras
are \spadtype{LieSquareMatrix} and \spadtype{FreeNilpotentLie}.}
Lie algebras can be used, for example, to analyse Lie symmetry algebras of
\index{symmetry}
partial differential
\index{differential equation!partial}
equations.
\index{partial differential equation}
In this section we show a different application of non-associative algebras,
\index{non-associative algebra}
the modelling of genetic laws.
\index{algebra!non-associative}

The \Language{} library contains several constructors for
creating non-assoc\-i\-a\-tive structures,
ranging from the categories \spadtype{Monad},
\spadtype{NonAssociativeRng}, and
\spadtype{FramedNonAssociativeAlgebra}, to the domains
\spadtype{AlgebraGivenByStructuralConstants} and
\spadtype{GenericNonAssociativeAlgebra}.
Furthermore, the package \spadtype{AlgebraPackage} provides
operations for analysing the structure of such algebras.\footnote{%
The interested reader can learn more about these aspects of the
\Language{} library from the paper
``Computations in Algebras of Finite Rank,''
by Johannes Grabmeier and Robert Wisbauer,
Technical Report, IBM Heidelberg Scientific Center, 1992.}

Mendel's genetic laws are often written in a form like
\begin{displaymath}
Aa \times Aa = \frac{1}{4}AA + \frac{1}{2}Aa + \frac{1}{4}aa.
\end{displaymath}
The implementation of general algebras in \Language{} allows us to
\index{Mendel's genetic laws}
use this as the definition for multiplication in an algebra.
\index{genetics}
Hence, it is possible to study
questions of genetic inheritance using \Language{}.
To demonstrate this more precisely,  we discuss one example from a
monograph of A. W\"orz-Busekros,
where you can also find a general setting of this theory.\footnote{%
W\"{o}rz-Busekros, A.,
{\it Algebras in Genetics},
Springer Lectures Notes in Biomathematics 36, Berlin e.a. (1980).
In particular, see example 1.3.}

We assume that there is an infinitely large random mating population.
Random mating of two gametes $a_i$  and
$a_j$ gives zygotes
\index{zygote}
$a_ia_j$, which produce new gametes.
\index{gamete}
In classical Mendelian segregation we have
$a_ia_j = \frac{1}{2}a_i+\frac{1}{2}a_j$.
In general, we have
\begin{displaymath}
a_ia_j = \sum_{k=1}^n \gamma_{i,j}^k\ a_k.
\end{displaymath}
The segregation rates $\gamma_{i,j}$
are  the structural constants of an
\smath{n}-dimensional algebra.
This is provided in \Language{} by
the constructor \spadtype{AlgebraGivenByStructuralConstants}
(abbreviation \spadtype{ALGSC}).

Consider two coupled autosomal loci with alleles
\smath{A},\smath{a}, \smath{B}, and \smath{b}, building four
different gametes
$a_1 =  AB, a_2 =  Ab, a_3 =  aB,$ and $a_4 =  ab$.
The zygotes $a_ia_j$
produce gametes $a_i$ and
$a_j$
with classical Mendelian segregation.
Zygote $a_1a_4$ undergoes transition
to $a_2a_3$
and vice versa with probability
$0 \le \theta \le \frac{1}{2}$.

\begin{xtc}
\begin{xtccomment}
Define a list
$[(\gamma_{i,j}^k) 1 \le k \le 4]$
of four four-by-four matrices giving the segregation
rates.
We use the value \smath{1/10} for \smath{\theta}.
\end{xtccomment}
\begin{spadsrc}
segregationRates : List SquareMatrix(4,FRAC INT) := [matrix [ [1, 1/2, 1/2, 9/20], [1/2, 0, 1/20, 0], [1/2, 1/20, 0, 0], [9/20, 0, 0, 0] ], matrix [ [0, 1/2, 0, 1/20], [1/2, 1, 9/20, 1/2], [0, 9/20, 0, 0], [1/20, 1/2, 0, 0] ], matrix [ [0, 0, 1/2, 1/20], [0, 0, 9/20, 0], [1/2, 9/20, 1, 1/2], [1/20, 0, 1/2, 0] ], matrix [ [0, 0, 0, 9/20], [0, 0, 1/20, 1/2], [0, 1/20, 0, 1/2], [9/20, 1/2, 1/2, 1] ] ] 
\end{spadsrc}
\begin{TeXOutput}
\begin{fricasmath}{1}
\BRACKET{\begin{MATRIX}{4}1&\frac{1}{2}&\frac{1}{2}&\frac{9}{20}\\\frac{1}{2}%
&0&\frac{1}{20}&0\\\frac{1}{2}&\frac{1}{20}&0&0\\\frac{9}{20}&0&0&0%
\end{MATRIX}\COMMA \begin{MATRIX}{4}0&\frac{1}{2}&0&\frac{1}{20}\\\frac{1}{2}%
&1&\frac{9}{20}&\frac{1}{2}\\0&\frac{9}{20}&0&0\\\frac{1}{20}&\frac{1}{2}&0&0%
\end{MATRIX}\COMMA \begin{MATRIX}{4}0&0&\frac{1}{2}&\frac{1}{20}\\0&0&\frac{9%
}{20}&0\\\frac{1}{2}&\frac{9}{20}&1&\frac{1}{2}\\\frac{1}{20}&0&\frac{1}{2}&0%
\end{MATRIX}\COMMA \begin{MATRIX}{4}0&0&0&\frac{9}{20}\\0&0&\frac{1}{20}&%
\frac{1}{2}\\0&\frac{1}{20}&0&\frac{1}{2}\\\frac{9}{20}&\frac{1}{2}&\frac{1}{%
2}&1\end{MATRIX}}%
\end{fricasmath}
\end{TeXOutput}
\formatResultType{List(SquareMatrix(4, Fraction(Integer)))}
\end{xtc}
\begin{xtc}
\begin{xtccomment}
Choose the appropriate symbols for the basis of gametes,
\end{xtccomment}
\begin{spadsrc}
gametes := ['AB,'Ab,'aB,'ab]  
\end{spadsrc}
\begin{TeXOutput}
\begin{fricasmath}{2}
\BRACKET{\SYMBOL{AB}\COMMA \SYMBOL{Ab}\COMMA \SYMBOL{aB}\COMMA \SYMBOL{ab}}%
\end{fricasmath}
\end{TeXOutput}
\formatResultType{List(OrderedVariableList([AB, Ab, aB, ab]))}
\end{xtc}
\begin{xtc}
\begin{xtccomment}
Define the algebra.
\end{xtccomment}
\begin{spadsrc}
A := ALGSC(FRAC INT, 4, gametes, segregationRates);
\end{spadsrc}
\formatResultType{Type}
\end{xtc}

\begin{xtc}
\begin{xtccomment}
What are the probabilities for zygote
$a_1a_4$ to produce the different gametes?
\end{xtccomment}
\begin{spadsrc}
a := basis()$A;  a.1*a.4
\end{spadsrc}
\begin{TeXOutput}
\begin{fricasmath}{4}
\frac{9}{20}\TIMES \SYMBOL{ab}+\frac{1}{20}\TIMES \SYMBOL{aB}+\frac{1}{20}%
\TIMES \SYMBOL{Ab}+\frac{9}{20}\TIMES \SYMBOL{AB}%
\end{fricasmath}
\end{TeXOutput}
\formatResultType{AlgebraGivenByStructuralConstants(Fraction(Integer), 4, [AB, Ab, aB, ab], [[[1, 1/2, 1/2, 9/20], [1/2, 0, 1/20, 0], [1/2, 1/20, 0, 0], [9/20, 0, 0, 0]], [[0, 1/2, 0, 1/20], [1/2, 1, 9/20, 1/2], [0, 9/20, 0, 0], [1/20, 1/2, 0, 0]], [[0, 0, 1/2, 1/20], [0, 0, 9/20, 0], [1/2, 9/20, 1, 1/2], [1/20, 0, 1/2, 0]], [[0, 0, 0, 9/20], [0, 0, 1/20, 1/2], [0, 1/20, 0, 1/2], [9/20, 1/2, 1/2, 1]]])}
\end{xtc}

Elements in this algebra whose coefficients sum to one play a
distinguished role.
They represent a population with the distribution of gametes
reflected by the coefficients with respect to the basis of
gametes.

Random mating of different populations \smath{x} and \smath{y} is described by
their product \smath{x*y}.

\begin{xtc}
\begin{xtccomment}
This product is commutative only
if the gametes are not sex-dependent, as in our example.
\end{xtccomment}
\begin{spadsrc}
commutative?()$A 
\end{spadsrc}
\begin{TeXOutput}
\begin{fricasmath}{5}
\STRING{true}%
\end{fricasmath}
\end{TeXOutput}
\formatResultType{Boolean}
\end{xtc}
\begin{xtc}
\begin{xtccomment}
In general, it is not associative.
\end{xtccomment}
\begin{spadsrc}
associative?()$A 
\end{spadsrc}
\begin{TeXOutput}
\begin{fricasmath}{6}
\STRING{false}%
\end{fricasmath}
\end{TeXOutput}
\formatResultType{Boolean}
\end{xtc}

Random mating within a population \smath{x} is described by
\smath{x*x.}
The next generation is \smath{(x*x)*(x*x).}

\begin{xtc}
\begin{xtccomment}
Use decimal numbers to compare the distributions more easily.
\end{xtccomment}
\begin{spadsrc}
x : ALGSC(DECIMAL, 4, gametes, segregationRates) :=  convert [3/10, 1/5, 1/10, 2/5]
\end{spadsrc}
\begin{TeXOutput}
\begin{fricasmath}{7}
0\STRING{.}4\TIMES \SYMBOL{ab}+0\STRING{.}1\TIMES \SYMBOL{aB}+0\STRING{.}2%
\TIMES \SYMBOL{Ab}+0\STRING{.}3\TIMES \SYMBOL{AB}%
\end{fricasmath}
\end{TeXOutput}
\formatResultType{AlgebraGivenByStructuralConstants(DecimalExpansion, 4, [AB, Ab, aB, ab], [[[1, CONCAT(0, ., 5), CONCAT(0, ., 5), CONCAT(0, ., CONCAT(4, 5))], [CONCAT(0, ., 5), 0, CONCAT(0, ., CONCAT(0, 5)), 0], [CONCAT(0, ., 5), CONCAT(0, ., CONCAT(0, 5)), 0, 0], [CONCAT(0, ., CONCAT(4, 5)), 0, 0, 0]], [[0, CONCAT(0, ., 5), 0, CONCAT(0, ., CONCAT(0, 5))], [CONCAT(0, ., 5), 1, CONCAT(0, ., CONCAT(4, 5)), CONCAT(0, ., 5)], [0, CONCAT(0, ., CONCAT(4, 5)), 0, 0], [CONCAT(0, ., CONCAT(0, 5)), CONCAT(0, ., 5), 0, 0]], [[0, 0, CONCAT(0, ., 5), CONCAT(0, ., CONCAT(0, 5))], [0, 0, CONCAT(0, ., CONCAT(4, 5)), 0], [CONCAT(0, ., 5), CONCAT(0, ., CONCAT(4, 5)), 1, CONCAT(0, ., 5)], [CONCAT(0, ., CONCAT(0, 5)), 0, CONCAT(0, ., 5), 0]], [[0, 0, 0, CONCAT(0, ., CONCAT(4, 5))], [0, 0, CONCAT(0, ., CONCAT(0, 5)), CONCAT(0, ., 5)], [0, CONCAT(0, ., CONCAT(0, 5)), 0, CONCAT(0, ., 5)], [CONCAT(0, ., CONCAT(4, 5)), CONCAT(0, ., 5), CONCAT(0, ., 5), 1]]])}
\end{xtc}
\begin{xtc}
\begin{xtccomment}
To compute directly the gametic distribution in the fifth
generation, we use \spadfun{plenaryPower}.
\end{xtccomment}
\begin{spadsrc}
plenaryPower(x,5) 
\end{spadsrc}
\begin{TeXOutput}
\begin{fricasmath}{8}
0\STRING{.}36561\TIMES \SYMBOL{ab}+0\STRING{.}13439\TIMES \SYMBOL{aB}+0%
\STRING{.}23439\TIMES \SYMBOL{Ab}+0\STRING{.}26561\TIMES \SYMBOL{AB}%
\end{fricasmath}
\end{TeXOutput}
\formatResultType{AlgebraGivenByStructuralConstants(DecimalExpansion, 4, [AB, Ab, aB, ab], [[[1, CONCAT(0, ., 5), CONCAT(0, ., 5), CONCAT(0, ., CONCAT(4, 5))], [CONCAT(0, ., 5), 0, CONCAT(0, ., CONCAT(0, 5)), 0], [CONCAT(0, ., 5), CONCAT(0, ., CONCAT(0, 5)), 0, 0], [CONCAT(0, ., CONCAT(4, 5)), 0, 0, 0]], [[0, CONCAT(0, ., 5), 0, CONCAT(0, ., CONCAT(0, 5))], [CONCAT(0, ., 5), 1, CONCAT(0, ., CONCAT(4, 5)), CONCAT(0, ., 5)], [0, CONCAT(0, ., CONCAT(4, 5)), 0, 0], [CONCAT(0, ., CONCAT(0, 5)), CONCAT(0, ., 5), 0, 0]], [[0, 0, CONCAT(0, ., 5), CONCAT(0, ., CONCAT(0, 5))], [0, 0, CONCAT(0, ., CONCAT(4, 5)), 0], [CONCAT(0, ., 5), CONCAT(0, ., CONCAT(4, 5)), 1, CONCAT(0, ., 5)], [CONCAT(0, ., CONCAT(0, 5)), 0, CONCAT(0, ., 5), 0]], [[0, 0, 0, CONCAT(0, ., CONCAT(4, 5))], [0, 0, CONCAT(0, ., CONCAT(0, 5)), CONCAT(0, ., 5)], [0, CONCAT(0, ., CONCAT(0, 5)), 0, CONCAT(0, ., 5)], [CONCAT(0, ., CONCAT(4, 5)), CONCAT(0, ., 5), CONCAT(0, ., 5), 1]]])}
\end{xtc}

We now ask two questions:
Does this distribution converge to an equilibrium state?
What are the distributions that are stable?

\begin{xtc}
\begin{xtccomment}
This is an invariant of the algebra and it is used to answer
the first question.
The new indeterminates describe a symbolic distribution.
\end{xtccomment}
\begin{spadsrc}
q := leftRankPolynomial()$GCNAALG(FRAC INT, 4, gametes, segregationRates) :: UP(Y, POLY FRAC INT)
\end{spadsrc}
\begin{TeXOutput}
\begin{fricasmath}{9}
\SUPER{\SYMBOL{Y}}{3}+\PAREN{-{\frac{29}{20}\TIMES \SYMBOL{\%x4}}-{\frac{29}{%
20}\TIMES \SYMBOL{\%x3}}-{\frac{29}{20}\TIMES \SYMBOL{\%x2}}-{\frac{29}{20}%
\TIMES \SYMBOL{\%x1}}}\TIMES \SUPER{\SYMBOL{Y}}{2}+\PAREN{\frac{9}{20}\TIMES %
\SUPER{\SYMBOL{\%x4}}{2}+\PAREN{\frac{9}{10}\TIMES \SYMBOL{\%x3}+\frac{9}{10}%
\TIMES \SYMBOL{\%x2}+\frac{9}{10}\TIMES \SYMBOL{\%x1}}\TIMES \SYMBOL{\%x4}+%
\frac{9}{20}\TIMES \SUPER{\SYMBOL{\%x3}}{2}+\PAREN{\frac{9}{10}\TIMES \SYMBOL%
{\%x2}+\frac{9}{10}\TIMES \SYMBOL{\%x1}}\TIMES \SYMBOL{\%x3}+\frac{9}{20}%
\TIMES \SUPER{\SYMBOL{\%x2}}{2}+\frac{9}{10}\TIMES \SYMBOL{\%x1}\TIMES %
\SYMBOL{\%x2}+\frac{9}{20}\TIMES \SUPER{\SYMBOL{\%x1}}{2}}\TIMES \SYMBOL{Y}%
\end{fricasmath}
\end{TeXOutput}
\formatResultType{UnivariatePolynomial(Y, Polynomial(Fraction(Integer)))}
\end{xtc}
\begin{xtc}
\begin{xtccomment}
Because the coefficient $\frac{9}{20}$ has absolute
value less than 1, all distributions do converge,
by a theorem of this theory.
\end{xtccomment}
\begin{spadsrc}
factor(q :: POLY FRAC INT) 
\end{spadsrc}
\begin{TeXOutput}
\begin{fricasmath}{10}
\PAREN{\SYMBOL{Y}-{\SYMBOL{\%x4}}-{\SYMBOL{\%x3}}-{\SYMBOL{\%x2}}-{\SYMBOL{%
\%x1}}}\TIMES \PAREN{\SYMBOL{Y}-{\frac{9}{20}\TIMES \SYMBOL{\%x4}}-{\frac{9}{%
20}\TIMES \SYMBOL{\%x3}}-{\frac{9}{20}\TIMES \SYMBOL{\%x2}}-{\frac{9}{20}%
\TIMES \SYMBOL{\%x1}}}\TIMES \SYMBOL{Y}%
\end{fricasmath}
\end{TeXOutput}
\formatResultType{Factored(Polynomial(Fraction(Integer)))}
\end{xtc}
\begin{xtc}
\begin{xtccomment}
The second question is answered by searching for idempotents in the
algebra.
\end{xtccomment}
\begin{spadsrc}
cI := conditionsForIdempotents()$GCNAALG(FRAC INT, 4, gametes, segregationRates)
\end{spadsrc}
\begin{TeXOutput}
\begin{fricasmath}{11}
\BRACKET{\frac{9}{10}\TIMES \SYMBOL{\%x1}\TIMES \SYMBOL{\%x4}+\PAREN{\frac{1%
}{10}\TIMES \SYMBOL{\%x2}+\SYMBOL{\%x1}}\TIMES \SYMBOL{\%x3}+\SYMBOL{\%x1}%
\TIMES \SYMBOL{\%x2}+\SUPER{\SYMBOL{\%x1}}{2}-{\SYMBOL{\%x1}}\COMMA \PAREN{%
\SYMBOL{\%x2}+\frac{1}{10}\TIMES \SYMBOL{\%x1}}\TIMES \SYMBOL{\%x4}+\frac{9}{%
10}\TIMES \SYMBOL{\%x2}\TIMES \SYMBOL{\%x3}+\SUPER{\SYMBOL{\%x2}}{2}+\PAREN{%
\SYMBOL{\%x1}-{1}}\TIMES \SYMBOL{\%x2}\COMMA \PAREN{\SYMBOL{\%x3}+\frac{1}{10%
}\TIMES \SYMBOL{\%x1}}\TIMES \SYMBOL{\%x4}+\SUPER{\SYMBOL{\%x3}}{2}+\PAREN{%
\frac{9}{10}\TIMES \SYMBOL{\%x2}+\SYMBOL{\%x1}-{1}}\TIMES \SYMBOL{\%x3}%
\COMMA \SUPER{\SYMBOL{\%x4}}{2}+\PAREN{\SYMBOL{\%x3}+\SYMBOL{\%x2}+\frac{9}{%
10}\TIMES \SYMBOL{\%x1}-{1}}\TIMES \SYMBOL{\%x4}+\frac{1}{10}\TIMES \SYMBOL{%
\%x2}\TIMES \SYMBOL{\%x3}}%
\end{fricasmath}
\end{TeXOutput}
\formatResultType{List(Polynomial(Fraction(Integer)))}
\end{xtc}
\begin{xtc}
\begin{xtccomment}
Solve these equations and look at the first solution.
\end{xtccomment}
\begin{spadsrc}
gbs:= groebnerFactorize cI; gbs.1
\end{spadsrc}
\begin{TeXOutput}
\begin{fricasmath}{12}
\BRACKET{\SYMBOL{\%x4}+\SYMBOL{\%x3}+\SYMBOL{\%x2}+\SYMBOL{\%x1}-{1}\COMMA %
\PAREN{\SYMBOL{\%x2}+\SYMBOL{\%x1}}\TIMES \SYMBOL{\%x3}+\SYMBOL{\%x1}\TIMES %
\SYMBOL{\%x2}+\SUPER{\SYMBOL{\%x1}}{2}-{\SYMBOL{\%x1}}}%
\end{fricasmath}
\end{TeXOutput}
\formatResultType{List(Polynomial(Fraction(Integer)))}
\end{xtc}

Further analysis using the package \spadtype{PolynomialIdeal}
shows that there is a two-dimensional variety of equilibrium states and all
other solutions are contained in it.

\begin{xtc}
\begin{xtccomment}
Choose one equilibrium state by setting two indeterminates to
concrete values.
\end{xtccomment}
\begin{spadsrc}
sol := solve concat(gbs.1,[%x1-1/10,%x2-1/10])
\end{spadsrc}
\begin{TeXOutput}
\begin{fricasmath}{13}
\BRACKET{\BRACKET{\SYMBOL{\%x4}=\frac{2}{5}\COMMA \SYMBOL{\%x3}=\frac{2}{5}%
\COMMA \SYMBOL{\%x2}=\frac{1}{10}\COMMA \SYMBOL{\%x1}=\frac{1}{10}}}%
\end{fricasmath}
\end{TeXOutput}
\formatResultType{List(List(Equation(Fraction(Polynomial(Integer)))))}
\end{xtc}
\begin{xtc}
\begin{xtccomment}
\end{xtccomment}
\begin{spadsrc}
e : A := represents reverse (map(rhs, sol.1) :: List FRAC INT)
\end{spadsrc}
\begin{TeXOutput}
\begin{fricasmath}{14}
\frac{2}{5}\TIMES \SYMBOL{ab}+\frac{2}{5}\TIMES \SYMBOL{aB}+\frac{1}{10}%
\TIMES \SYMBOL{Ab}+\frac{1}{10}\TIMES \SYMBOL{AB}%
\end{fricasmath}
\end{TeXOutput}
\formatResultType{AlgebraGivenByStructuralConstants(Fraction(Integer), 4, [AB, Ab, aB, ab], [[[1, 1/2, 1/2, 9/20], [1/2, 0, 1/20, 0], [1/2, 1/20, 0, 0], [9/20, 0, 0, 0]], [[0, 1/2, 0, 1/20], [1/2, 1, 9/20, 1/2], [0, 9/20, 0, 0], [1/20, 1/2, 0, 0]], [[0, 0, 1/2, 1/20], [0, 0, 9/20, 0], [1/2, 9/20, 1, 1/2], [1/20, 0, 1/2, 0]], [[0, 0, 0, 9/20], [0, 0, 1/20, 1/2], [0, 1/20, 0, 1/2], [9/20, 1/2, 1/2, 1]]])}
\end{xtc}
\begin{xtc}
\begin{xtccomment}
Verify the result.
\end{xtccomment}
\begin{spadsrc}
e*e-e 
\end{spadsrc}
\begin{TeXOutput}
\begin{fricasmath}{15}
0%
\end{fricasmath}
\end{TeXOutput}
\formatResultType{AlgebraGivenByStructuralConstants(Fraction(Integer), 4, [AB, Ab, aB, ab], [[[1, 1/2, 1/2, 9/20], [1/2, 0, 1/20, 0], [1/2, 1/20, 0, 0], [9/20, 0, 0, 0]], [[0, 1/2, 0, 1/20], [1/2, 1, 9/20, 1/2], [0, 9/20, 0, 0], [1/20, 1/2, 0, 0]], [[0, 0, 1/2, 1/20], [0, 0, 9/20, 0], [1/2, 9/20, 1, 1/2], [1/20, 0, 1/2, 0]], [[0, 0, 0, 9/20], [0, 0, 1/20, 1/2], [0, 1/20, 0, 1/2], [9/20, 1/2, 1/2, 1]]])}
\end{xtc}

% *********************************************************************
\head{section}{Matrix Manipulation}{ugMatrixManipulation}
% *********************************************************************

This section shows some examples on selecting various (rectangular) submatrices
of matrices. In the numerics literature, these operations are usually referred
to as slicing. Apart from indexing matrices by two integers for retrieving single
elements, it is possible to use lists of integers (\spadtype{List(Integer)}),
segments (\spadtype{Segment(Integer)}) and list of segments (\spadtype{List(Segment(Integer))})
to select slices like whole rows, columns or submatrices.

\begin{xtc}
\begin{xtccomment}
First, we build a simple test matrix to show the above-mentioned manipulations:
\end{xtccomment}
\begin{spadsrc}
m := matrix([[11, 12, 13, 14], [21, 22,23, 24], [31, 32, 33, 34]]) 
\end{spadsrc}
\begin{TeXOutput}
\begin{fricasmath}{1}
\begin{MATRIX}{4}11&12&13&14\\21&22&23&24\\31&32&33&34\end{MATRIX}%
\end{fricasmath}
\end{TeXOutput}
\formatResultType{Matrix(Integer)}
\end{xtc}

\begin{xtc}
\begin{xtccomment}
Select the top right two by two submatrix by slicing using segments:
\end{xtccomment}
\begin{spadsrc}
m(1..2, 3..4) 
\end{spadsrc}
\begin{TeXOutput}
\begin{fricasmath}{2}
\begin{MATRIX}{2}13&14\\23&24\end{MATRIX}%
\end{fricasmath}
\end{TeXOutput}
\formatResultType{Matrix(Integer)}
\end{xtc}

\begin{xtc}
\begin{xtccomment}
Having a nonzero step size in the segment is also supported:
\end{xtccomment}
\begin{spadsrc}
m(1..2, 1..3 by 2) 
\end{spadsrc}
\begin{TeXOutput}
\begin{fricasmath}{3}
\begin{MATRIX}{2}11&13\\21&23\end{MATRIX}%
\end{fricasmath}
\end{TeXOutput}
\formatResultType{Matrix(Integer)}
\end{xtc}

\begin{xtc}
\begin{xtccomment}
Indexing by lists works as expected, returning all elements having index pairs
from the outer product of both lists:
\end{xtccomment}
\begin{spadsrc}
m([1,3], [2,4]) 
\end{spadsrc}
\begin{TeXOutput}
\begin{fricasmath}{4}
\begin{MATRIX}{2}12&14\\32&34\end{MATRIX}%
\end{fricasmath}
\end{TeXOutput}
\formatResultType{Matrix(Integer)}
\end{xtc}

\begin{xtc}
\begin{xtccomment}
Selecting single elements by any index type other than \spadtype{Integer}
for both, the row and column index, will not give the respective element but
a 1 times 1 matrix containing it:
\end{xtccomment}
\begin{spadsrc}
m(1, 2) 
\end{spadsrc}
\begin{TeXOutput}
\begin{fricasmath}{5}
12%
\end{fricasmath}
\end{TeXOutput}
\formatResultType{PositiveInteger}
\begin{spadsrc}
m([1], [2]) 
\end{spadsrc}
\begin{TeXOutput}
\begin{fricasmath}{6}
\begin{MATRIX}{1}12\end{MATRIX}%
\end{fricasmath}
\end{TeXOutput}
\formatResultType{Matrix(Integer)}
\begin{spadsrc}
m(1, [2]) 
\end{spadsrc}
\begin{TeXOutput}
\begin{fricasmath}{7}
\begin{MATRIX}{1}12\end{MATRIX}%
\end{fricasmath}
\end{TeXOutput}
\formatResultType{Matrix(Integer)}
\begin{spadsrc}
m(1, 2..2) 
\end{spadsrc}
\begin{TeXOutput}
\begin{fricasmath}{8}
\begin{MATRIX}{1}12\end{MATRIX}%
\end{fricasmath}
\end{TeXOutput}
\formatResultType{Matrix(Integer)}
\begin{spadsrc}
m(1, [2..2]) 
\end{spadsrc}
\begin{TeXOutput}
\begin{fricasmath}{9}
\begin{MATRIX}{1}12\end{MATRIX}%
\end{fricasmath}
\end{TeXOutput}
\formatResultType{Matrix(Integer)}
\end{xtc}

\begin{xtc}
\begin{xtccomment}
It is possible to use lists of segments to select multiple
submatrices which get stacked together forming the result returned:
\end{xtccomment}
\begin{spadsrc}
m([1..2], [1, 3..4]) 
\end{spadsrc}
\begin{TeXOutput}
\begin{fricasmath}{10}
\begin{MATRIX}{3}11&13&14\\21&23&24\end{MATRIX}%
\end{fricasmath}
\end{TeXOutput}
\formatResultType{Matrix(Integer)}
\end{xtc}

\begin{xtc}
\begin{xtccomment}
Use overlapping segments to repeat elements:
\end{xtccomment}
\begin{spadsrc}
m([1..2], [3..4, 3..4]) 
\end{spadsrc}
\begin{TeXOutput}
\begin{fricasmath}{11}
\begin{MATRIX}{4}13&14&13&14\\23&24&23&24\end{MATRIX}%
\end{fricasmath}
\end{TeXOutput}
\formatResultType{Matrix(Integer)}
\end{xtc}

\begin{xtc}
\begin{xtccomment}
It is even possible to mix any of the valid index constructs in the selection
of rows and columns:
\end{xtccomment}
\begin{spadsrc}
m(2, [1,4]) 
\end{spadsrc}
\begin{TeXOutput}
\begin{fricasmath}{12}
\begin{MATRIX}{2}21&24\end{MATRIX}%
\end{fricasmath}
\end{TeXOutput}
\formatResultType{Matrix(Integer)}
\begin{spadsrc}
m([1,2,3], 2..3) 
\end{spadsrc}
\begin{TeXOutput}
\begin{fricasmath}{13}
\begin{MATRIX}{2}12&13\\22&23\\32&33\end{MATRIX}%
\end{fricasmath}
\end{TeXOutput}
\formatResultType{Matrix(Integer)}
\begin{spadsrc}
m([1,2], [1..3, 4]) 
\end{spadsrc}
\begin{TeXOutput}
\begin{fricasmath}{14}
\begin{MATRIX}{4}11&12&13&14\\21&22&23&24\end{MATRIX}%
\end{fricasmath}
\end{TeXOutput}
\formatResultType{Matrix(Integer)}
\end{xtc}

\begin{xtc}
\begin{xtccomment}
Assignment to a submatrix using slicing syntax is supported, too:
\end{xtccomment}
\begin{spadsrc}
m([1..2], [3..3]) := m([1,2], [2]) 
\end{spadsrc}
\begin{TeXOutput}
\begin{fricasmath}{15}
\begin{MATRIX}{1}12\\22\end{MATRIX}%
\end{fricasmath}
\end{TeXOutput}
\formatResultType{Matrix(Integer)}
\begin{spadsrc}
m 
\end{spadsrc}
\begin{TeXOutput}
\begin{fricasmath}{16}
\begin{MATRIX}{4}11&12&12&14\\21&22&22&24\\31&32&33&34\end{MATRIX}%
\end{fricasmath}
\end{TeXOutput}
\formatResultType{Matrix(Integer)}
\end{xtc}

Note that assignment currently does not check for overlapping segments,
and the last assignments wins. However, overlapping case should be
considered undefined anyway.

Another caveat shows up when assigning single elements.

\begin{xtc}
\begin{xtccomment}
Selecting the entry by any index type other than two times an \spadtype{Integer}
requires assignment of a matrix type:
\end{xtccomment}
\begin{spadsrc}
m([2], [2]) := matrix([[4]]) 
\end{spadsrc}
\begin{TeXOutput}
\begin{fricasmath}{17}
\begin{MATRIX}{1}4\end{MATRIX}%
\end{fricasmath}
\end{TeXOutput}
\formatResultType{Matrix(Integer)}
\begin{spadsrc}
m 
\end{spadsrc}
\begin{TeXOutput}
\begin{fricasmath}{18}
\begin{MATRIX}{4}11&12&12&14\\21&4&22&24\\31&32&33&34\end{MATRIX}%
\end{fricasmath}
\end{TeXOutput}
\formatResultType{Matrix(Integer)}
\end{xtc}

By using the functions \spadfun{rowSlice} and \spadfun{colSlice} it is
possible to obtain for a given matrix two special slicing objects that
when used will select all elements along a colum or row respectively
(\spadfun{rowSlice} varies row index giving a column).
The advantage of using these is that no information about the actual
matrix size in necessary.

\begin{xtc}
\begin{xtccomment}
It is easily possible to select the second and fourth columns
of a given matrix:
\end{xtccomment}
\begin{spadsrc}
r := rowSlice(m) 
\end{spadsrc}
\begin{TeXOutput}
\begin{fricasmath}{19}
\SEGMENTii{1}{3}%
\end{fricasmath}
\end{TeXOutput}
\formatResultType{Segment(Integer)}
\begin{spadsrc}
m(r, 2) 
\end{spadsrc}
\begin{TeXOutput}
\begin{fricasmath}{20}
\begin{MATRIX}{1}12\\4\\32\end{MATRIX}%
\end{fricasmath}
\end{TeXOutput}
\formatResultType{Matrix(Integer)}
\begin{spadsrc}
m(r, 4) 
\end{spadsrc}
\begin{TeXOutput}
\begin{fricasmath}{21}
\begin{MATRIX}{1}14\\24\\34\end{MATRIX}%
\end{fricasmath}
\end{TeXOutput}
\formatResultType{Matrix(Integer)}
\end{xtc}

\begin{xtc}
\begin{xtccomment}
Assignment of course works the same way. The following snippet shows
simple row operations as used in Gaussian elimination:
\end{xtccomment}
\begin{spadsrc}
c := colSlice(m) 
\end{spadsrc}
\begin{TeXOutput}
\begin{fricasmath}{22}
\SEGMENTii{1}{4}%
\end{fricasmath}
\end{TeXOutput}
\formatResultType{Segment(Integer)}
\begin{spadsrc}
m := m :: Matrix(Fraction(Integer))
\end{spadsrc}
\begin{TeXOutput}
\begin{fricasmath}{23}
\begin{MATRIX}{4}11&12&12&14\\21&4&22&24\\31&32&33&34\end{MATRIX}%
\end{fricasmath}
\end{TeXOutput}
\formatResultType{Matrix(Fraction(Integer))}
\begin{spadsrc}
m(2, c) := m(2, c) - m(2,1)/m(1,1) * m(1, c) 
\end{spadsrc}
\begin{TeXOutput}
\begin{fricasmath}{24}
\begin{MATRIX}{4}0&-{\frac{208}{11}}&-{\frac{10}{11}}&-{\frac{30}{11}}%
\end{MATRIX}%
\end{fricasmath}
\end{TeXOutput}
\formatResultType{Matrix(Fraction(Integer))}
\begin{spadsrc}
m(3, c) := m(3, c) - m(3,1)/m(1,1) * m(1, c) 
\end{spadsrc}
\begin{TeXOutput}
\begin{fricasmath}{25}
\begin{MATRIX}{4}0&-{\frac{20}{11}}&-{\frac{9}{11}}&-{\frac{60}{11}}%
\end{MATRIX}%
\end{fricasmath}
\end{TeXOutput}
\formatResultType{Matrix(Fraction(Integer))}
\begin{spadsrc}
m(3, c) := m(3, c) - m(3,2)/m(2,2) * m(2, c) 
\end{spadsrc}
\begin{TeXOutput}
\begin{fricasmath}{26}
\begin{MATRIX}{4}0&0&-{\frac{19}{26}}&-{\frac{135}{26}}\end{MATRIX}%
\end{fricasmath}
\end{TeXOutput}
\formatResultType{Matrix(Fraction(Integer))}
\begin{spadsrc}
m 
\end{spadsrc}
\begin{TeXOutput}
\begin{fricasmath}{27}
\begin{MATRIX}{4}11&12&12&14\\0&-{\frac{208}{11}}&-{\frac{10}{11}}&-{\frac{30%
}{11}}\\0&0&-{\frac{19}{26}}&-{\frac{135}{26}}\end{MATRIX}%
\end{fricasmath}
\end{TeXOutput}
\formatResultType{Matrix(Fraction(Integer))}
\end{xtc}

\begin{xtc}
\begin{xtccomment}
Selecting the whole matrix:
\end{xtccomment}
\begin{spadsrc}
r := rowSlice(m) 
\end{spadsrc}
\begin{TeXOutput}
\begin{fricasmath}{28}
\SEGMENTii{1}{3}%
\end{fricasmath}
\end{TeXOutput}
\formatResultType{Segment(Integer)}
\begin{spadsrc}
c := colSlice(m) 
\end{spadsrc}
\begin{TeXOutput}
\begin{fricasmath}{29}
\SEGMENTii{1}{4}%
\end{fricasmath}
\end{TeXOutput}
\formatResultType{Segment(Integer)}
\begin{spadsrc}
m(r,c) 
\end{spadsrc}
\begin{TeXOutput}
\begin{fricasmath}{30}
\begin{MATRIX}{4}11&12&12&14\\0&-{\frac{208}{11}}&-{\frac{10}{11}}&-{\frac{30%
}{11}}\\0&0&-{\frac{19}{26}}&-{\frac{135}{26}}\end{MATRIX}%
\end{fricasmath}
\end{TeXOutput}
\formatResultType{Matrix(Fraction(Integer))}
\end{xtc}

\input{ug09}

\part{Advanced Programming in \Language{}}
%
% !! DO NOT MODIFY THIS FILE BY HAND !! Created by spool2tex.awk.

% Copyright (c) 1991-2002, The Numerical ALgorithms Group Ltd.
% All rights reserved.
%
% Redistribution and use in source and binary forms, with or without
% modification, are permitted provided that the following conditions are
% met:
%
%     - Redistributions of source code must retain the above copyright
%       notice, this list of conditions and the following disclaimer.
%
%     - Redistributions in binary form must reproduce the above copyright
%       notice, this list of conditions and the following disclaimer in
%       the documentation and/or other materials provided with the
%       distribution.
%
%     - Neither the name of The Numerical ALgorithms Group Ltd. nor the
%       names of its contributors may be used to endorse or promote products
%       derived from this software without specific prior written permission.
%
% THIS SOFTWARE IS PROVIDED BY THE COPYRIGHT HOLDERS AND CONTRIBUTORS "AS
% IS" AND ANY EXPRESS OR IMPLIED WARRANTIES, INCLUDING, BUT NOT LIMITED
% TO, THE IMPLIED WARRANTIES OF MERCHANTABILITY AND FITNESS FOR A
% PARTICULAR PURPOSE ARE DISCLAIMED. IN NO EVENT SHALL THE COPYRIGHT OWNER
% OR CONTRIBUTORS BE LIABLE FOR ANY DIRECT, INDIRECT, INCIDENTAL, SPECIAL,
% EXEMPLARY, OR CONSEQUENTIAL DAMAGES (INCLUDING, BUT NOT LIMITED TO,
% PROCUREMENT OF SUBSTITUTE GOODS OR SERVICES-- LOSS OF USE, DATA, OR
% PROFITS-- OR BUSINESS INTERRUPTION) HOWEVER CAUSED AND ON ANY THEORY OF
% LIABILITY, WHETHER IN CONTRACT, STRICT LIABILITY, OR TORT (INCLUDING
% NEGLIGENCE OR OTHERWISE) ARISING IN ANY WAY OUT OF THE USE OF THIS
% SOFTWARE, EVEN IF ADVISED OF THE POSSIBILITY OF SUCH DAMAGE.

% *********************************************************************
\head{chapter}{Interactive Programming}{ugIntProg}
% *********************************************************************

Programming in the interpreter is easy.
So is the use of \Language{}'s graphics facility.
Both are rather flexible and allow you to use them for many
interesting applications.
However, both require learning some basic ideas and skills.

All graphics examples in the \Gallery{} section are either
produced directly by interactive commands or by interpreter
programs.
Four of these programs are introduced here.
By the end of this chapter you will know enough about graphics and
programming in the interpreter to not only understand all these
examples, but to tackle interesting and difficult problems on your
own.
\appxref{ugAppGraphics} lists all the remaining commands and
programs used to create these images.

% *********************************************************************
\head{section}{Drawing Ribbons Interactively}{ugIntProgDrawing}
% *********************************************************************
%

We begin our discussion of interactive graphics with the creation
of a useful facility: plotting ribbons of two-graphs in
three-space.
Suppose you want to draw the \twodim{} graphs of \smath{n}
functions
$f_i(x), 1 \leq i \leq n,$
all over some fixed range of \smath{x}.
One approach is to create a \twodim{} graph for each one, then
superpose one on top of the other.
What you will more than likely get is a jumbled mess.
Even if you make each function a different color, the result is
likely to be confusing.

A better approach is to display each of the \smath{f_i(x)} in three
\index{ribbon}
dimensions as a ``ribbon'' of some appropriate width along the
\smath{y}-direction, laying down each  ribbon next to the
previous one.
A ribbon is simply a function of \smath{x} and \smath{y} depending
only on \smath{x.}

We illustrate this for \smath{f_i(x)} defined as simple powers of
\smath{x} for \smath{x} ranging between \smath{-1} and \smath{1}.

\begin{psXtc}
\begin{xtccomment}
Draw the ribbon for \mathOrSpad{z=x ^ 2}.
\end{xtccomment}
\begin{spadsrc}
draw(x^2,x=-1..1,y=0..1)
\end{spadsrc}
\epsffile[0 0 295 295]{ribbon1.ps}
\end{psXtc}

Now that was easy!
What you get is a ``wire-mesh'' rendition of the ribbon.
That's fine for now.
Notice that the mesh-size is small in both the \smath{x} and the
\smath{y} directions.
\Language{} normally computes points in both these directions.
This is unnecessary.
One step is all we need in the \smath{y}-direction.
To have \Language{} economize on \spad{y}-points, we re-draw the
ribbon with option \spad{var2Steps == 1}.

\begin{psXtc}
\begin{xtccomment}
Re-draw the ribbon, but with option \spad{var2Steps == 1}
so that only \spad{1} step is computed in the
\smath{y} direction.
\end{xtccomment}
\begin{spadsrc}
vp := draw(x^2,x=-1..1,y=0..1,var2Steps==1) 
\end{spadsrc}
\epsffile[0 0 295 295]{ribbon2.ps}
\end{psXtc}

The operation has created a viewport, that is, a graphics window
on your screen.
We assigned the viewport to \spad{vp} and now we manipulate
its contents.


Graphs are objects, like numbers and algebraic expressions.
You may want to do some experimenting with graphs.
For example, say
\begin{verbatim}
showRegion(vp, "on")
\end{verbatim}
to put a bounding box around the ribbon.
Try it!
Issue \spad{rotate(vp, -45, 90)} to rotate the
figure \smath{-45} longitudinal degrees and \smath{90} latitudinal
degrees.

\begin{psXtc}
\begin{xtccomment}
Here is a different rotation.
This turns the graph so you can view it along the \smath{y}-axis.
\end{xtccomment}
\begin{spadsrc}
rotate(vp, 0, -90)
\end{spadsrc}
\epsffile[0 0 295 295]{ribbon2r.ps}
\end{psXtc}

There are many other things you can do.
In fact, most everything you can do interactively using the
\threedim{} control panel (such as translating, zooming, resizing,
coloring, perspective and lighting selections) can also be done
directly by operations (see \chapref{ugGraph} for more details).

When you are done experimenting, say \spad{reset(vp)} to restore the
picture to its original position and settings.


Let's add another ribbon to our picture---one
for \mathOrSpad{x^3}.
Since \smath{y} ranges from \smath{0} to \smath{1} for the
first ribbon, now let \smath{y} range from \smath{1} to
\smath{2.}
This puts the second ribbon next to the first one.

How do you add a second ribbon to the viewport?
One method is
to extract the ``space'' component from the
viewport using the operation
\spadfunFrom{subspace}{ThreeDimensionalViewport}.
You can think of the space component as the object inside the
window (here, the ribbon).
Let's call it \spad{sp}.
To add the second ribbon, you draw the second ribbon using the
option \spad{space == sp}.

\begin{xtc}
\begin{xtccomment}
Extract the space component of \spad{vp}.
\end{xtccomment}
\begin{spadsrc}
sp := subspace(vp)
\end{spadsrc}
\begin{MessageOutput}
   There are 2 exposed and 0 unexposed library operations named 
      subspace having 1 argument(s) but none was determined to be 
      applicable. Use HyperDoc Browse, or issue
                            )display op subspace
      to learn more about the available operations. Perhaps 
      package-calling the operation or using coercions on the arguments
      will allow you to apply the operation.
\end{MessageOutput}
\begin{MessageOutput}
   Cannot find a definition or applicable library operation named 
      subspace with argument type(s) 
                                Variable(vp)
      
      Perhaps you should use "@" to indicate the required return type, 
      or "$" to specify which version of the function you need.
\end{MessageOutput}
\end{xtc}

\begin{psXtc}
\begin{xtccomment}
Add the ribbon for
\mathOrSpad{x^3} alongside that for
\mathOrSpad{x^2}.
\end{xtccomment}
\begin{spadsrc}
vp := draw(x^3,x=-1..1,y=1..2,var2Steps==1, space==sp)
\end{spadsrc}
\epsffile[0 0 295 295]{ribbons.ps}
\end{psXtc}

Unless you moved the original viewport, the new viewport covers
the old one.
You might want to check that the old object is still there by
moving the top window.

Let's show quadrilateral polygon outlines on the ribbons and then
enclose the ribbons in a box.

\begin{psXtc}
\begin{xtccomment}
Show quadrilateral polygon outlines.
\end{xtccomment}
\begin{spadsrc}
drawStyle(vp,"shade");outlineRender(vp,"on")
\end{spadsrc}
\epsffile[0 0 295 295]{ribbons2.ps}
\end{psXtc}
\begin{psXtc}
\begin{xtccomment}
Enclose the ribbons in a box.
\end{xtccomment}
\begin{spadsrc}
rotate(vp,20,-60); showRegion(vp,"on")
\end{spadsrc}
\epsffile[0 0 295 295]{ribbons2b.ps}
\end{psXtc}

This process has become tedious!
If we had to add two or three more ribbons, we would have to
repeat the above steps several more times.
It is time to write an interpreter program to help us take care of
the details.

% *********************************************************************
\head{section}{A Ribbon Program}{ugIntProgRibbon}
% *********************************************************************
%

The above approach creates a new viewport for each additional
ribbon.
A better approach is to build one object composed of all ribbons
before creating a viewport.
To do this, use \spadfun{makeObject} rather than \spadfun{draw}.
The operations have similar formats, but
\spadfun{draw} returns a viewport and
\spadfun{makeObject} returns a space object.

We now create a function \userfun{drawRibbons} of two arguments:
\spad{flist}, a list of formulas for the ribbons you want to draw,
and \spad{xrange}, the range over which you want them drawn.
Using this function, you can just say
\begin{verbatim}
drawRibbons([x^2, x^3], x=-1..1)
\end{verbatim}
to do all of the work required in the last section.
Here is the \userfun{drawRibbons} program.
Invoke your favorite editor and create a file called {\bf ribbon.input}
containing the following program.

\begin{figXmpLines}[caption={The first \protect\pspadfun{drawRibbons} function.}]
drawRibbons(flist, xrange) ==
  sp := createThreeSpace()               -- Create empty space \spad{sp}.
  y0 := 0                                -- The initial ribbon position.
  for f in flist repeat                  -- For each function \spad{f},
    makeObject(f, xrange, y=y0..y0+1,    -- \quad{}create and add a ribbon
       space==sp, var2Steps == 1)        -- \quad{}for \spad{f} to the space \spad{sp}.
    y0 := y0 + 1                         -- The next ribbon position.
  vp := makeViewport3D(sp, "Ribbons")    -- Create viewport.
  drawStyle(vp, "shade")                 -- Select shading style.
  outlineRender(vp, "on")                -- Show polygon outlines.
  showRegion(vp,"on")                    -- Enclose in a box.
  n := # flist                           -- The number of ribbons
  zoom(vp,n,1,n)                         -- Zoom in x- and z-directions.
  rotate(vp,0,75)                        -- Change the angle of view.
  vp                                     -- Return the viewport.
\end{figXmpLines}

Here are some remarks on the syntax used in the \pspadfun{drawRibbons} function
(consult \chapref{ugUser} for more details).
Unlike most other programming languages which use semicolons,
parentheses, or {\it begin}--{\it end} brackets to delineate the
structure of programs, the structure of an \Language{} program is
determined by indentation.
The first line of the function definition always begins in column 1.
All other lines of the function are indented with respect to the first
line and form a \spadgloss{pile} (see \spadref{ugLangBlocks}).

The definition of \userfun{drawRibbons}
consists of a pile of expressions to be executed one after
another.
Each expression of the pile is indented at the same level.
Lines 4-7 designate one single expression:
since lines 5-7 are indented with respect to the others, these
lines are treated as a continuation of line 4.
Also since lines 5 and 7 have the same indentation level, these
lines designate a pile within the outer pile.

The last line of a pile usually gives the value returned by the
pile.
Here it is also the value returned by the function.
\Language{} knows this is the last line of the function because it
is the last line of the file.
In other cases, a new expression beginning in column one signals
the end of a function.

The line \spad{drawStyle(vp,"shade")} is given after the viewport
has been created to select the draw style.
We have also used the \spadfunFrom{zoom}{ThreeDimensionalViewport}
option.
Without the zoom, the viewport region would be scaled equally in
all three coordinate directions.

Let's try the function \userfun{drawRibbons}.
First you must read the file to give \Language{} the function definition.

\begin{xtc}
\begin{xtccomment}
Read the input file.
\end{xtccomment}
\begin{spadsrc}
)read ribbon 
\end{spadsrc}
\end{xtc}
\begin{psXtc}
\begin{xtccomment}
Draw ribbons for $x, x^2,\dots, x^5$
for $-1 \leq x \leq 1$
\end{xtccomment}
\begin{spadsrc}
drawRibbons([x^i for i in 1..5],x=-1..1) 
\end{spadsrc}
\epsffile[0 0 295 295]{ribbons5.ps}
\end{psXtc}


% *********************************************************************
\head{section}{Coloring and Positioning Ribbons}{ugIntProgColor}
% *********************************************************************
%

Before leaving the ribbon example, we  make two improvements.
Normally, the color given to each point in the space is a
function of its height within a bounding box.
The points at the bottom of the
box are red, those at the top are purple.

To change the normal coloring, you can give
an option \textspadexpr{colorFunction == {\it function}}.
When \Language{} goes about displaying the data, it
determines the range of colors used for all points within the box.
\Language{} then distributes these numbers uniformly over the number of hues.
Here we use the simple color function
$(x,y) \mapsto i$ for the
\eth{i} ribbon.

Also, we add an argument \spad{yrange} so you can give the range of
\spad{y} occupied by the ribbons.
For example, if the \spad{yrange} is given as
\spad{y=0..1} and there are \smath{5} ribbons to be displayed, each
ribbon would have width \smath{0.2} and would appear in the
range $0 \leq y \leq 1$.

Refer to lines 4-9.
Line 4 assigns to \spad{yVar} the variable part of the
\spad{yrange} (after all, it need not be \spad{y}).
Suppose that \spad{yrange} is given as \spad{t = a..b} where \spad{a} and
\spad{b} have numerical values.
Then line 5 assigns the value of \spad{a} to the variable \spad{y0}.
Line 6 computes the width of the ribbon by dividing the difference of
\spad{a} and \spad{b} by the number, \spad{num}, of ribbons.
The result is assigned to the variable \spad{width}.
Note that in the for-loop in line 7, we are iterating in parallel; it is
not a nested loop.

\begin{figXmpLines}[caption={The final \protect\pspadfun{drawRibbons} function.}]
drawRibbons(flist, xrange, yrange) ==
  sp := createThreeSpace()                     -- Create empty space \spad{sp}.
  num := # flist                               -- The number of ribbons.
  yVar := variable yrange                      -- The ribbon variable.
  y0:Float    := low segment yrange             - The first ribbon coordinate.
  width:Float := (high segment yrange - y0)/num   The width of a ribbon.
  for f in flist for color in 1..num repeat    -- For each function \spad{f},
    makeObject(f, xrange, yVar = y0..y0+width, -- \quad{}create and add ribbon to
      var2Steps == 1, _                        -- \quad{}\spad{sp} of a different color.
      colorFunction == (x,y) +-> color, _
      space == sp)
    y0 := y0 + width                           -- The next ribbon coordinate.
  vp := makeViewport3D(sp, "Ribbons")          -- Create viewport.
  drawStyle(vp, "shade")                       -- Select shading style.
  outlineRender(vp, "on")                      -- Show polygon outlines.
  showRegion(vp, "on")                         -- Enclose -- in a box.
  vp                                           -- Return the viewport.
\end{figXmpLines}


% *********************************************************************
\head{section}{Points, Lines, and Curves}{ugIntProgPLC}
% *********************************************************************
%
What you have seen so far is a high-level program using the
graphics facility.
We now turn to the more basic notions of points, lines, and curves
in \threedim{} graphs.
These facilities use small floats (objects
of type \spadtype{DoubleFloat}) for data.
Let us first give names to the small float values \smath{0} and
\smath{1}.
\begin{xtc}
\begin{xtccomment}
The small float 0.
\end{xtccomment}
\begin{spadsrc}
zero := 0.0@DFLOAT 
\end{spadsrc}
\begin{TeXOutput}
\begin{fricasmath}{1}
\STRING{0.0}%
\end{fricasmath}
\end{TeXOutput}
\formatResultType{DoubleFloat}
\end{xtc}
\begin{xtc}
\begin{xtccomment}
The small float 1.
\end{xtccomment}
\begin{spadsrc}
one  := 1.0@DFLOAT 
\end{spadsrc}
\begin{TeXOutput}
\begin{fricasmath}{2}
\STRING{1.0}%
\end{fricasmath}
\end{TeXOutput}
\formatResultType{DoubleFloat}
\end{xtc}
The \spadSyntax{@} sign means ``of the type.'' Thus \spad{zero} is
\smath{0.0} of the type \spadtype{DoubleFloat}.
You can also say \spad{0.0::DFLOAT}.

Points can have four small float components: \smath{x, y, z} coordinates and an
optional color.
A ``curve'' is simply a list of points connected by straight line
segments.
\begin{xtc}
\begin{xtccomment}
Create the point \spad{origin} with color zero, that is, the lowest color
on the color map.
\end{xtccomment}
\begin{spadsrc}
origin := point [zero,zero,zero,zero] 
\end{spadsrc}
\begin{TeXOutput}
\begin{fricasmath}{3}
\BRACKET{\STRING{0.0}\COMMA \STRING{0.0}\COMMA \STRING{0.0}\COMMA \STRING{0.0%
}}%
\end{fricasmath}
\end{TeXOutput}
\formatResultType{Point(DoubleFloat)}
\end{xtc}
\begin{xtc}
\begin{xtccomment}
Create the point \spad{unit} with color zero.
\end{xtccomment}
\begin{spadsrc}
unit := point [one,one,one,zero] 
\end{spadsrc}
\begin{TeXOutput}
\begin{fricasmath}{4}
\BRACKET{\STRING{1.0}\COMMA \STRING{1.0}\COMMA \STRING{1.0}\COMMA \STRING{0.0%
}}%
\end{fricasmath}
\end{TeXOutput}
\formatResultType{Point(DoubleFloat)}
\end{xtc}
\begin{xtc}
\begin{xtccomment}
Create the curve (well, here, a line) from
\spad{origin} to \spad{unit}.
\end{xtccomment}
\begin{spadsrc}
line := [origin, unit] 
\end{spadsrc}
\begin{TeXOutput}
\begin{fricasmath}{5}
\BRACKET{\BRACKET{\STRING{0.0}\COMMA \STRING{0.0}\COMMA \STRING{0.0}\COMMA %
\STRING{0.0}}\COMMA \BRACKET{\STRING{1.0}\COMMA \STRING{1.0}\COMMA \STRING{%
1.0}\COMMA \STRING{0.0}}}%
\end{fricasmath}
\end{TeXOutput}
\formatResultType{List(Point(DoubleFloat))}
\end{xtc}

We make this line segment into an arrow by adding an arrowhead.
The arrowhead extends to,
say, \spad{p3} on the left, and to, say, \spad{p4} on the right.
To describe an arrow, you tell \Language{} to draw the two curves
\spad{[p1, p2, p3]} and \spad{[p2, p4].}
We also decide through experimentation on
values for \spad{arrowScale}, the ratio of the size of
the arrowhead to the stem of the arrow, and \spad{arrowAngle},
the angle between the arrowhead and the arrow.

Invoke your favorite editor and create
an input file called {\bf arrows.input}.
This input file first defines the values of
%\spad{origin},\spad{unit},
\spad{arrowAngle} and \spad{arrowScale}, then
defines the function \userfun{makeArrow}$(p_1, p_2)$ to
draw an arrow from point $p_1$ to $p_2$.

%origin := point [0.0@DFLOAT,0.0@DFLOAT,0.0@DFLOAT,0.0@DFLOAT]       -- The point 0 with color 0.
%unit := point [1.0@DFLOAT,1.0@DFLOAT,1.0@DFLOAT,0.0@DFLOAT]         -- A second point with color 0.
%\xmpLine{}{}
\begin{xmpLines}
arrowAngle := %pi-%pi/10.0@DFLOAT              -- The angle of the arrowhead.
arrowScale := 0.2@DFLOAT                       -- The size of the arrowhead
                                               -- \quad{}relative to the stem.
makeArrow(p1, p2) ==
  delta := p2 - p1                             -- The arrow.
  len := arrowScale * length delta             -- The length of the arrowhead.
  theta := atan(delta.1, delta.2)              -- The angle from the x-axis
  c1 := len*cos(theta + arrowAngle)            -- The x-coord of left endpoint.
  s1 := len*sin(theta + arrowAngle)            -- The y-coord of left endpoint.
  c2 := len*cos(theta - arrowAngle)            -- The x-coord of right endpoint.
  s2 := len*sin(theta - arrowAngle)            -- The y-coord of right endpoint.
  z  := p2.3*(1 - arrowScale)                  -- The z-coord of both endpoints.
  p3 := point [p2.1 + c1, p2.2 + s1, z, p2.4]  -- The left endpoint of head.
  p4 := point [p2.1 + c2, p2.2 + s2, z, p2.4]  -- The right endpoint of head.
  [[p1, p2, p3], [p2, p4]]                     -- The arrow as a list of curves.
\end{xmpLines}

Read the file and then create
an arrow from the point \spad{origin} to the point \spad{unit}.
\begin{xtc}
\begin{xtccomment}
Read the input file defining \userfun{makeArrow}.
\end{xtccomment}
\begin{spadsrc}
)read arrows
\end{spadsrc}
\begin{TeXOutput}
\begin{fricasmath}{6}
\STRING{2.827433388230814}%
\end{fricasmath}
\end{TeXOutput}
\formatResultType{DoubleFloat}
\begin{TeXOutput}
\begin{fricasmath}{7}
\STRING{0.2}%
\end{fricasmath}
\end{TeXOutput}
\formatResultType{DoubleFloat}
\end{xtc}
\begin{xtc}
\begin{xtccomment}
Construct the arrow (a list of two curves).
\end{xtccomment}
\begin{spadsrc}
arrow := makeArrow(origin,unit)
\end{spadsrc}
\begin{MessageOutput}
   Compiling function makeArrow with type (Point(DoubleFloat),Point(
      DoubleFloat)) -> List(List(Point(DoubleFloat))) 
\end{MessageOutput}
\begin{TeXOutput}
\begin{fricasmath}{9}
\BRACKET{\BRACKET{\BRACKET{\STRING{0.0}\COMMA \STRING{0.0}\COMMA \STRING{0.0}%
\COMMA \STRING{0.0}}\COMMA \BRACKET{\STRING{1.0}\COMMA \STRING{1.0}\COMMA %
\STRING{1.0}\COMMA \STRING{0.0}}\COMMA \BRACKET{\STRING{0.6913462860460797}%
\COMMA \STRING{0.842733077659504}\COMMA \STRING{0.8}\COMMA \STRING{0.0}}}%
\COMMA \BRACKET{\BRACKET{\STRING{1.0}\COMMA \STRING{1.0}\COMMA \STRING{1.0}%
\COMMA \STRING{0.0}}\COMMA \BRACKET{\STRING{0.842733077659504}\COMMA \STRING{%
0.6913462860460797}\COMMA \STRING{0.8}\COMMA \STRING{0.0}}}}%
\end{fricasmath}
\end{TeXOutput}
\formatResultType{List(List(Point(DoubleFloat)))}
\end{xtc}
\begin{xtc}
\begin{xtccomment}
Create an empty object \spad{sp} of type \spad{ThreeSpace}.
\end{xtccomment}
\begin{spadsrc}
sp := createThreeSpace()
\end{spadsrc}
\begin{TeXOutput}
\begin{fricasmath}{10}
\STRING{3-Space\ with\ }0\STRING{\ components}%
\end{fricasmath}
\end{TeXOutput}
\formatResultType{ThreeSpace(DoubleFloat)}
\end{xtc}
\begin{xtc}
\begin{xtccomment}
Add each curve of the arrow to the space \spad{sp}.
\end{xtccomment}
\begin{spadsrc}
for a in arrow repeat sp := curve(sp,a)
\end{spadsrc}
\end{xtc}
\begin{psXtc}
\begin{xtccomment}
Create a \threedim{} viewport containing that space.
\end{xtccomment}
\begin{spadsrc}
vp := makeViewport3D(sp,"Arrow")
\end{spadsrc}
\epsffile[0 0 295 295]{arrow.ps}
\end{psXtc}
\begin{psXtc}
\begin{xtccomment}
Here is a better viewing angle.
\end{xtccomment}
\begin{spadsrc}
rotate(vp,200,-60)
\end{spadsrc}
\epsffile[0 0 295 295]{arrowr.ps}
\end{psXtc}


% *********************************************************************
\head{section}{A Bouquet of Arrows}{ugIntProgColorArr}
% *********************************************************************

%\Language{} gathers up all the points of a graph and looks at the range
%of color values given as integers.
%If theses color values range from a minimum value of \spad{a} to a maximum
%value of \spad{b}, then the \spad{a} values are colored red (the
%lowest color in our spectrum), and \spad{b} values are colored
%purple (the highest color), and those in the middle are colored
%green.
%When all the points are the same color as above, \Language{}
%chooses green.

Let's draw a ``bouquet'' of arrows.
Each arrow is identical. The arrowheads are
uniformly placed on a circle parallel to the \smath{xy}-plane.
Thus the position of each arrow differs only
by the angle $\theta$,
$0 \leq \theta < 2\pi$,
between the arrow and
the \smath{x}-axis on the \smath{xy}-plane.

Our bouquet is rather special: each arrow has a different
color (which won't be evident here, unfortunately).
This is arranged by letting the color of each successive arrow be
denoted by $\theta$.
In this way, the color of arrows ranges from red to green to violet.
Here is a program to draw a bouquet of \smath{n} arrows.

\begin{xmpLines}
drawBouquet(n,title) ==
  z := 0.0@DFLOAT
  e := 1.0@DFLOAT
  angle := z                                        -- The initial angle.
  sp := createThreeSpace()                          -- Create empty space \spad{sp}.
  for i in 0..n-1 repeat                            -- For each index i, create:
    start := point [z,z,z,angle]                    -- point at base of arrow;
    end   := point [cos angle, sin angle, e, angle] -- point at tip of arrow;
    arrow := makeArrow(start,end)                   -- \spad{i}th arrow.
    for a in makeArrow(start,end) repeat            -- For each arrow component,
      curve(sp,a)                                   -- \quad{}add the component to \spad{sp}.
    angle := angle + 2*%pi/n                        -- The next angle.
  makeViewport3D(sp,title)                          -- Create the viewport from \spad{sp}.
\end{xmpLines}

\begin{xtc}
\begin{xtccomment}
Read the input file.
\end{xtccomment}
\begin{spadsrc}
)read bouquet
\end{spadsrc}
relative size of the arrow head compared to the length of the arrow
\begin{TeXOutput}
\begin{fricasmath}{1}
\STRING{0.2}%
\end{fricasmath}
\end{TeXOutput}
\formatResultType{DoubleFloat}
angle of the arrow head
\begin{TeXOutput}
\begin{fricasmath}{2}
\STRING{2.827433388230814}%
\end{fricasmath}
\end{TeXOutput}
\formatResultType{DoubleFloat}
Add an arrow head to a line segment, which starts at 'p1', ends at 'p2',
has length 'len', and and angle 'arg'.  We pass 'len' and 'arg' as
arguments since they were already computed by the calling program
\end{xtc}
\begin{psXtc}
\begin{xtccomment}
A bouquet of a dozen arrows.
\end{xtccomment}
\begin{spadsrc}
drawBouquet(12,"A Dozen Arrows")
\end{spadsrc}
\epsffile[0 0 295 295]{bouquet.ps}
\end{psXtc}
\

% *********************************************************************
%\head{section}{Diversion: When Things Go Wrong}{ugIntProgDivTwo}
% *********************************************************************
%
%Up to now, if you have typed in all the programs exactly as they are in
%the book, you have encountered no errors.
%In practice, however, it is easy to make mistakes.
%Computers are unforgiving: your program must be letter-for-letter correct
%or you will encounter some error.
%
%One thing that can go wrong is that you can create a syntactically
%incorrect program.
%As pointed out in Diversion 1, the meaning of \Language{} programs is
%affected by indentation.
%
%The \Language{} parser will ensure that all parentheses, brackets, and
%braces balance, and that commas and operators appear in the correct
%context.
%For example, change line ??
%to ??
%and run.
%
%A common mistake is to misspell an identifier or operation name.
%These are generally easy to spot since the interpreter will tell you the
%name of the operation together with the type and number of arguments which
%it is trying to find.
%
%Another mistake is to either to omit an argument or to give too many.
%Again \Language{} will notify you of the offending operation.
%
%Indentation makes your programs more readable.
%However there are several ways to create a syntactically valid program.
%A most common problem occurs when a line is either indented improperly.
%% either or what?
%If this is a first line of a pile, then all the other lines will act as an
%inner pile to the first line.
%If it is a line of the pile other than the first line, \Language{} then
%thinks that this line is a continuation of the previous line.
%More frequently than not, a syntactically correct expression is created.
%Almost never however will this be a semantically correct.
%Only when the program is run will an error be discovered.
%For example, change line ??
%to ??
%and run.

% *********************************************************************
\head{section}{Drawing Complex Vector Fields}{ugIntProgVecFields}
% *********************************************************************

We now put our arrows to good use drawing complex vector fields.
These vector fields give a representation of complex-valued
functions of complex variables.
Consider a Cartesian coordinate grid of points \smath{(x, y)} in
the plane, and some complex-valued function \smath{f} defined on
this grid.
At every point on this grid, compute the value of
$f(x + iy)$ and call it \smath{z}.
Since \smath{z} has both a real and imaginary value for a given
\smath{(x,y)} grid point, there are four dimensions to plot.
What do we do?
We represent the values of \smath{z} by arrows planted at each
grid point.
Each arrow represents the value of \smath{z} in polar coordinates
$(r,\theta)$.
The length of the arrow is proportional to \smath{r}.
Its direction is given by $\theta$.

The code for drawing vector fields is in the file {\bf vectors.input}.
We discuss its contents from top to bottom.

Before showing you the code, we have two small
matters to take care of.
First, what if the function has large spikes, say, ones that go off
to infinity?
We define a variable \spad{clipValue} for this purpose. When
\spad{r} exceeds the value of \spad{clipValue}, then the value of
\spad{clipValue} is used instead of that for \spad{r}.
For convenience, we define a function \spad{clipFun(x)} which uses
\spad{clipValue} to ``clip'' the value of \spad{x}.

%
\begin{xmpLines}
clipValue : DFLOAT := 6                              -- Maximum value allowed.
clipFun(x) == min(max(x,-clipValue),clipValue)
\end{xmpLines}

Notice that we identify \spad{clipValue} as a small float but do
not declare the type of the function \userfun{clipFun}.
As it turns out, \userfun{clipFun} is called with a
small float value.
This declaration ensures that \userfun{clipFun} never does a
conversion when it is called.

The second matter concerns the possible ``poles'' of a
function, the actual points where the spikes have infinite
values.
\Language{} uses normal \spadtype{DoubleFloat} arithmetic  which
does not directly handle infinite values.
If your function has poles, you must adjust your step size to
avoid landing directly on them (\Language{} calls \spadfun{error}
when asked to divide a value by \spad{0}, for example).

We set the variables \spad{realSteps} and \spad{imagSteps} to
hold the number of steps taken in the real and imaginary
directions, respectively.
Most examples will have ranges centered around the origin.
To avoid a pole at the origin, the number of points is taken
to be odd.

\begin{xmpLinesNoReset}
realSteps: INT := 25                                 -- Number of real steps.
imagSteps: INT := 25                                 -- Number of imaginary steps.
)read arrows
\end{xmpLinesNoReset}

Now define the function \userfun{drawComplexVectorField} to draw the arrows.
It is good practice to declare the type of the main function in
the file.
This one declaration is usually sufficient to ensure that other
lower-level functions are compiled with the correct types.

\begin{xmpLinesNoReset}
C := Complex DoubleFloat
S := Segment DoubleFloat
drawComplexVectorField: (C -> C, S, S) -> VIEW3D
\end{xmpLinesNoReset}

The first argument is a function mapping complex small floats into
complex small floats.
The second and third arguments give the range of real and
imaginary values as segments like \spad{a..b}.
The result is a \threedim{} viewport.
Here is the full function definition:

\begin{xmpLinesNoReset}
drawComplexVectorField(f, realRange,imagRange) ==
  delReal := (high(realRange)-low(realRange))/realSteps The real step size.
  delImag := (high(imagRange)-low(imagRange))/imagSteps The imaginary step size.
  sp := createThreeSpace()                           -- Create empty space \spad{sp}.
  real := low(realRange)                             -- The initial real value.
  for i in 1..realSteps+1 repeat                     -- Begin real iteration.
    imag := low(imagRange)                           -- The initial imaginary value.
    for j in 1..imagSteps+1 repeat                   -- Begin imaginary iteration.
      z := f complex(real,imag)                      -- The value of \spad{f} at the point.
      arg := argument z                              -- The direction of the arrow.
      len := clipFun sqrt norm z                     -- The length of the arrow.
      p1 :=  point [real, imag, 0.0@DFLOAT, arg]     -- The base point of the arrow.
      scaleLen := delReal * len                      -- The scaled length of the arrow.
      p2 := point [p1.1 + scaleLen*cos(arg),         -- The tip point of the arrow.
                   p1.2 + scaleLen*sin(arg),0.0@DFLOAT, arg]
      arrow := makeArrow(p1, p2)                     -- Create the arrow.
      for a in arrow repeat curve(sp, a)             -- Add arrow to the space \spad{sp}.
      imag := imag + delImag                         -- The next imaginary value.
    real := real + delReal                           -- The next real value.
  makeViewport3D(sp, "Complex Vector Field")         -- Draw it!
\end{xmpLinesNoReset}

As a first example, let us draw \spad{f(z) == sin(z)}.
There is no need to create a user function: just pass the
\spadfunFrom{sin}{Complex DoubleFloat} from \spadtype{Complex DoubleFloat}.
\begin{xtc}
\begin{xtccomment}
Read the file.
\end{xtccomment}
\begin{spadsrc}
)read vectors 
\end{spadsrc}
\begin{TeXOutput}
\begin{fricasmath}{1}
\STRING{2.827433388230814}%
\end{fricasmath}
\end{TeXOutput}
\formatResultType{DoubleFloat}
\begin{TeXOutput}
\begin{fricasmath}{2}
\STRING{0.2}%
\end{fricasmath}
\end{TeXOutput}
\formatResultType{DoubleFloat}
\begin{TeXOutput}
\begin{fricasmath}{4}
\STRING{6.0}%
\end{fricasmath}
\end{TeXOutput}
\formatResultType{DoubleFloat}
\begin{TeXOutput}
\begin{fricasmath}{6}
25%
\end{fricasmath}
\end{TeXOutput}
\formatResultType{Integer}
\begin{TeXOutput}
\begin{fricasmath}{7}
25%
\end{fricasmath}
\end{TeXOutput}
\formatResultType{Integer}
\begin{TeXOutput}
\begin{fricasmath}{8}
\STRING{Complex(DoubleFloat)}%
\end{fricasmath}
\end{TeXOutput}
\formatResultType{Type}
\begin{TeXOutput}
\begin{fricasmath}{9}
\STRING{Segment(DoubleFloat)}%
\end{fricasmath}
\end{TeXOutput}
\formatResultType{Type}
\end{xtc}
\begin{psXtc}
\begin{xtccomment}
Draw the complex vector field of \spad{sin(x)}.
\end{xtccomment}
\begin{spadsrc}
drawComplexVectorField(sin,-2..2,-2..2) 
\end{spadsrc}
\epsffile[0 0 295 295]{vectorSin.ps}
\end{psXtc}
\

% *********************************************************************
\head{section}{Drawing Complex Functions}{ugIntProgCompFuns}
% *********************************************************************

Here is another way to graph a complex function of complex
arguments.
For each complex value \smath{z}, compute \smath{f(z)}, again
expressing the value in polar coordinates \smath{(r,\theta{})}.
We draw the complex valued function, again considering the
\smath{(x,y)}-plane as the complex plane, using \smath{r} as the
height (or \smath{z}-coordinate) and \smath{\theta} as the color.
This is a standard plot---we learned how to do this in
\chapref{ugGraph}---but here we write a new program to illustrate
the creation of polygon meshes, or grids.

Call this function \userfun{drawComplex}.
It displays the points using the ``mesh'' of points.
The function definition is in three parts.

\begin{xmpLines}
drawComplex: (C -> C, S, S) -> VIEW3D
drawComplex(f, realRange, imagRange) ==              -- The first part.
  delReal := (high(realRange)-low(realRange))/realSteps The real step size.
  delImag := (high(imagRange)-low(imagRange))/imagSteps The imaginary step size.
  llp:List List Point DFLOAT := []                   -- Initial list of list of points \spad{llp}.
\end{xmpLines}

Variables \spad{delReal} and \spad{delImag} give the step
sizes along the real and imaginary directions as computed by the values
of the global variables \spad{realSteps} and \spad{imagSteps}.
The mesh is represented by a list of lists of points \spad{llp},
initially empty.
Now \spad{[ ]} alone is ambiguous, so
to set this initial value
you have to tell \Language{} what type of empty list it is.
Next comes the loop which builds \spad{llp}.

\begin{xmpLinesNoReset}
  real := low(realRange)                              - The initial real value.
  for i in 1..realSteps+1 repeat                     -- Begin real iteration.
    imag := low(imagRange)                           -- The initial imaginary value.
    lp := []$(List Point DFLOAT)                     -- The initial list of points \spad{lp}.
    for j in 1..imagSteps+1 repeat                   -- Begin imaginary iteration.
      z := f complex(real,imag)                      -- The value of \spad{f} at the point.
      pt := point [real,imag, clipFun sqrt norm z,   -- Create a point.
                   argument z]
      lp := cons(pt,lp)                              -- Add the point to \spad{lp}.
      imag := imag + delImag                         -- The next imaginary value.
    real := real + delReal                           -- The next real value.
    llp := cons(lp, llp)                             -- Add \spad{lp} to \spad{llp}.
\end{xmpLinesNoReset}

The code consists of both an inner and outer loop.
Each pass through the inner loop adds one list \spad{lp} of points
to the list of lists of points \spad{llp}.
The elements of \spad{lp} are collected in reverse order.

\begin{xmpLinesNoReset}
  makeViewport3D(mesh(llp), "Complex Function")      -- Create a mesh and display.
\end{xmpLinesNoReset}

The operation \spadfun{mesh} then creates an object of type
\spadtype{ThreeSpace(DoubleFloat)} from the list of lists of points.
This is then passed to \spadfun{makeViewport3D} to display the
image.

Now add this function directly to your {\bf vectors.input}
file and re-read the file using \spad{)read vectors}.
We try \userfun{drawComplex} using
a user-defined function \spad{f}.

\begin{xtc}
\begin{xtccomment}
Read the file.
\end{xtccomment}
\begin{spadsrc}
)read vectors 
\end{spadsrc}
\begin{TeXOutput}
\begin{fricasmath}{1}
\STRING{2.827433388230814}%
\end{fricasmath}
\end{TeXOutput}
\formatResultType{DoubleFloat}
\begin{TeXOutput}
\begin{fricasmath}{2}
\STRING{0.2}%
\end{fricasmath}
\end{TeXOutput}
\formatResultType{DoubleFloat}
\begin{TeXOutput}
\begin{fricasmath}{4}
\STRING{6.0}%
\end{fricasmath}
\end{TeXOutput}
\formatResultType{DoubleFloat}
\begin{TeXOutput}
\begin{fricasmath}{6}
25%
\end{fricasmath}
\end{TeXOutput}
\formatResultType{Integer}
\begin{TeXOutput}
\begin{fricasmath}{7}
25%
\end{fricasmath}
\end{TeXOutput}
\formatResultType{Integer}
\begin{TeXOutput}
\begin{fricasmath}{8}
\STRING{Complex(DoubleFloat)}%
\end{fricasmath}
\end{TeXOutput}
\formatResultType{Type}
\begin{TeXOutput}
\begin{fricasmath}{9}
\STRING{Segment(DoubleFloat)}%
\end{fricasmath}
\end{TeXOutput}
\formatResultType{Type}
\end{xtc}
\begin{xtc}
\begin{xtccomment}
This one has a pole at \smath{z=0}.
\end{xtccomment}
\begin{spadsrc}
f(z) == exp(1/z)
\end{spadsrc}
\end{xtc}
\begin{psXtc}
\begin{xtccomment}
Draw it with an odd number of steps to avoid the pole.
\end{xtccomment}
\begin{spadsrc}
drawComplex(f,-2..2,-2..2)
\end{spadsrc}
\epsffile[0 0 295 295]{complexExp.ps}
\end{psXtc}
\

% *********************************************************************
\head{section}{Functions Producing Functions}{ugIntProgFunctions}
% *********************************************************************

In \spadref{ugUserMake}, you learned how to use the operation
\spadfun{function} to create a function from symbolic formulas.
Here we introduce a similar operation which not only
creates functions, but functions from functions.

The facility we need is provided by the package
\spadtype{MakeUnaryCompiledFunction(E,S,T)}.
\exptypeindex{MakeUnaryCompiledFunction}
This package produces a unary (one-argument) compiled
function from some symbolic data
generated by a previous computation.\footnote{%
\spadtype{MakeBinaryCompiledFunction} is available for binary
functions.}
\exptypeindex{MakeBinaryCompiledFunction}
The \spad{E} tells where the symbolic data comes from;
the \spad{S} and \spad{T} give \Language{} the
source and target type of the function, respectively.
The compiled function produced  has type
\spad{S -> T}.
To produce a compiled function with definition \spad{p(x) == expr}, call
\spad{compiledFunction(expr, x)} from this package.
The function you get has no name.
You must to assign the function to the variable \spad{p} to give it that name.
%
\begin{xtc}
\begin{xtccomment}
Do some computation.
\end{xtccomment}
\begin{spadsrc}
(x+1/3)^5
\end{spadsrc}
\begin{TeXOutput}
\begin{fricasmath}{1}
\SUPER{\SYMBOL{x}}{5}+\frac{5}{3}\TIMES \SUPER{\SYMBOL{x}}{4}+\frac{10}{9}%
\TIMES \SUPER{\SYMBOL{x}}{3}+\frac{10}{27}\TIMES \SUPER{\SYMBOL{x}}{2}+\frac{%
5}{81}\TIMES \SYMBOL{x}+\frac{1}{243}%
\end{fricasmath}
\end{TeXOutput}
\formatResultType{Polynomial(Fraction(Integer))}
\end{xtc}
\begin{xtc}
\begin{xtccomment}
Convert this to an anonymous function of \spad{x}.
Assign it to the variable \spad{p} to give the function a name.
\end{xtccomment}
\begin{spadsrc}
p := compiledFunction(%,x)$MakeUnaryCompiledFunction(POLY FRAC INT,DFLOAT,DFLOAT)
\end{spadsrc}
\begin{MessageOutput}
   Compiling function %A with type DoubleFloat -> DoubleFloat 
\end{MessageOutput}
\begin{TeXOutput}
\begin{fricasmath}{2}
\theMap{unaryFunction}%
\end{fricasmath}
\end{TeXOutput}
\formatResultType{(DoubleFloat -> DoubleFloat)}
\end{xtc}
\begin{xtc}
\begin{xtccomment}
Apply the function.
\end{xtccomment}
\begin{spadsrc}
p(sin(1.3))
\end{spadsrc}
\begin{TeXOutput}
\begin{fricasmath}{3}
\STRING{3.668751115057229}%
\end{fricasmath}
\end{TeXOutput}
\formatResultType{DoubleFloat}
\end{xtc}

For a more sophisticated application, read on.

% *********************************************************************
\head{section}{Automatic Newton Iteration Formulas}{ugIntProgNewton}
% *********************************************************************

%--rhx: TODO: Check this carefully.
% This setting is needed to get Newton's iterations to converge.
% \spadcommand{)set streams calculate 10}

We resume
our continuing saga of arrows and complex functions.
Suppose we want to investigate the behavior of Newton's iteration function
\index{Newton iteration}
in the complex plane.
Given a function \smath{f}, we want to find the complex values
\smath{z} such that \smath{f(z) = 0}.

The first step is to produce a Newton iteration formula for
a given \smath{f}:
$x_{n+1} = x_n - \frac{f(x_n)}{f'(x_n)}.$
We represent this formula by a function \smath{g}
that performs the computation on the right-hand side, that is,
$x_{n+1} = {g}(x_n)$.

The type \spadtype{Expression Integer}
(abbreviated \spadtype{EXPR INT})
is used to represent general symbolic expressions in
\Language{}.
\exptypeindex{Expression}
To make our facility as general as possible, we assume
\smath{f} has this type.
Given \smath{f}, we want
to produce a Newton iteration function \spad{g} which,
given a complex point $x_n$, delivers the next
Newton iteration point $x_{n+1}$.

This time we write an input file called {\bf newton.input}.
We need to import \spadtype{MakeUnaryCompiledFunction} (discussed
in the last section), call it with appropriate types, and then define
the function \spad{newtonStep} which references it.
Here is the function \spad{newtonStep}:

\begin{xmpLines}
C := Complex DoubleFloat                             -- The complex numbers.
E := Expression Integer                              -- The expression domain.
complexFunPack:=MakeUnaryCompiledFunction(E,C,C)     -- Package for making functions.

newtonStep(f) ==                                     -- Newton's iteration function.
  fun  := complexNumericFunction f                   -- Function for $f$.
  deriv := complexDerivativeFunction(f,1)            -- Function for $f'$.
  (x:C):C +->                                        -- Return the iterator function.
    x - fun(x)/deriv(x)

complexNumericFunction f ==                          -- Turn an expression \spad{f} into a
  v := theVariableIn f                               -- \quad{}function.
  compiledFunction(f, v)$complexFunPack

complexDerivativeFunction(f,n) ==                    -- Create an nth derivative
  v := theVariableIn f                               -- \quad{}function.
  df := D(f,v,n)
  compiledFunction(df, v)$complexFunPack

theVariableIn f ==                                   -- Returns the variable in $f$.
  vl := variables f                                  -- The list of variables.
  nv := # vl                                         -- The number of variables.
  nv > 1 => error "Expression is not univariate."
  nv = 0 => 'x                                       -- Return a dummy variable.
  first vl
\end{xmpLines}

Do you see what is going on here?
A formula \spad{f} is passed into the function \userfun{newtonStep}.
First, the function turns \spad{f} into a compiled program mapping
complex numbers into complex numbers.  Next, it does the same thing
for the derivative of \spad{f}.  Finally, it returns a function which
computes a single step of Newton's iteration.

The function \userfun{complexNumericFunction} extracts the variable
from the expression \spad{f} and then turns \spad{f} into a function
which maps complex numbers into complex numbers. The function
\userfun{complexDerivativeFunction} does the same thing for the
derivative of \spad{f}.  The function \userfun{theVariableIn}
extracts the variable from the expression \spad{f}, calling the function
\spadfun{error} if \spad{f} has more than one variable.
It returns the dummy variable \spad{x} if \spad{f} has no variables.

Let's now apply \userfun{newtonStep} to the formula for computing
cube roots of two.
%
\begin{noOutputXtc}
\begin{xtccomment}
Read the input file with the definitions.
\end{xtccomment}
\begin{spadsrc}
)read newton
\end{spadsrc}
Newton's Iteration function
newtonStep(f) returns a newton's iteration function for the
expression f.
create complex numeric functions from an expression
\begin{TeXOutput}
\begin{fricasmath}{2}
\STRING{%
MakeUnaryCompiledFunction(Expression(Integer),Complex(DoubleFloat),Complex(DoubleFloat))%
}%
\end{fricasmath}
\end{TeXOutput}
\formatResultType{Type}
create a complex numeric function from an expression
create a complex numeric derivatiave function from an expression
return the unique variable in x, or an error if it is multivariate
\end{noOutputXtc}
\begin{noOutputXtc}
\begin{xtccomment}
\end{xtccomment}
\begin{spadsrc}
)read vectors 
\end{spadsrc}
\begin{TeXOutput}
\begin{fricasmath}{6}
\STRING{2.827433388230814}%
\end{fricasmath}
\end{TeXOutput}
\formatResultType{DoubleFloat}
\begin{TeXOutput}
\begin{fricasmath}{7}
\STRING{0.2}%
\end{fricasmath}
\end{TeXOutput}
\formatResultType{DoubleFloat}
\begin{TeXOutput}
\begin{fricasmath}{9}
\STRING{6.0}%
\end{fricasmath}
\end{TeXOutput}
\formatResultType{DoubleFloat}
\begin{TeXOutput}
\begin{fricasmath}{11}
25%
\end{fricasmath}
\end{TeXOutput}
\formatResultType{Integer}
\begin{TeXOutput}
\begin{fricasmath}{12}
25%
\end{fricasmath}
\end{TeXOutput}
\formatResultType{Integer}
\begin{TeXOutput}
\begin{fricasmath}{13}
\STRING{Complex(DoubleFloat)}%
\end{fricasmath}
\end{TeXOutput}
\formatResultType{Type}
\begin{TeXOutput}
\begin{fricasmath}{14}
\STRING{Segment(DoubleFloat)}%
\end{fricasmath}
\end{TeXOutput}
\formatResultType{Type}
\end{noOutputXtc}

\begin{xtc}
\begin{xtccomment}
The cube root of two.
\end{xtccomment}
\begin{spadsrc}
f := x^3 - 2
\end{spadsrc}
\begin{TeXOutput}
\begin{fricasmath}{19}
\SUPER{\SYMBOL{x}}{3}-{2}%
\end{fricasmath}
\end{TeXOutput}
\formatResultType{Polynomial(Integer)}
\end{xtc}
\begin{xtc}
\begin{xtccomment}
Get Newton's iteration formula.
\end{xtccomment}
\begin{spadsrc}
g := newtonStep f
\end{spadsrc}
\begin{MessageOutput}
   Compiling function theVariable with type Polynomial(Integer) -> 
      Symbol 
\end{MessageOutput}
\begin{MessageOutput}
   Compiling function complexNumericFunction with type Polynomial(
      Integer) -> (Complex(DoubleFloat) -> Complex(DoubleFloat)) 
\end{MessageOutput}
\begin{MessageOutput}
   Compiling function complexDerivativeFunction with type (Polynomial(
      Integer),PositiveInteger) -> (Complex(DoubleFloat) -> Complex(
      DoubleFloat)) 
\end{MessageOutput}
\begin{MessageOutput}
   Compiling function newtonStep with type Polynomial(Integer) -> (
      Complex(DoubleFloat) -> Complex(DoubleFloat)) 
\end{MessageOutput}
\begin{MessageOutput}
   Compiling function %B with type Complex(DoubleFloat) -> Complex(
      DoubleFloat) 
\end{MessageOutput}
\begin{MessageOutput}
   Compiling function %C with type Complex(DoubleFloat) -> Complex(
      DoubleFloat) 
\end{MessageOutput}
\begin{TeXOutput}
\begin{fricasmath}{20}
\theMap{?}%
\end{fricasmath}
\end{TeXOutput}
\formatResultType{(Complex(DoubleFloat) -> Complex(DoubleFloat))}
\end{xtc}
\begin{xtc}
\begin{xtccomment}
Let \spad{a} denote the result of
applying Newton's iteration once to the complex number \spad{1 + %i}.
\end{xtccomment}
\begin{spadsrc}
a := g(1.0 + %i)
\end{spadsrc}
\begin{TeXOutput}
\begin{fricasmath}{21}
\STRING{0.6666666666666667}+\STRING{0.33333333333333337}\TIMES \ImaginaryI %
\end{fricasmath}
\end{TeXOutput}
\formatResultType{Complex(DoubleFloat)}
\end{xtc}
\begin{xtc}
\begin{xtccomment}
Now apply it repeatedly. How fast does it converge?
\end{xtccomment}
\begin{spadsrc}
[(a := g(a)) for i in 1..]
\end{spadsrc}
\begin{TeXOutput}
\begin{fricasmath}{22}
\BRACKET{\STRING{1.1644444444444444}-{\STRING{0.7377777777777778}\TIMES %
\ImaginaryI }\COMMA \STRING{0.9261400469716478}-{\STRING{0.17463006425584393}%
\TIMES \ImaginaryI }\COMMA \STRING{1.3164444838140228}+\STRING{%
0.15690694583015852}\TIMES \ImaginaryI \COMMA \STRING{1.2462991025761463}+%
\STRING{0.015454763610132094}\TIMES \ImaginaryI \COMMA \STRING{%
1.259872529653208}-{\STRING{3.382716205931127e-4}\TIMES \ImaginaryI }\COMMA %
\STRING{1.259920960928212}+\STRING{2.602353465342268e-8}\TIMES \ImaginaryI %
\COMMA \STRING{1.259921049894879}-{\STRING{3.6751942591616685e-15}\TIMES %
\ImaginaryI }\COMMA \STRING{...}}%
\end{fricasmath}
\end{TeXOutput}
\formatResultType{Stream(Complex(DoubleFloat))}
\end{xtc}
\begin{xtc}
\begin{xtccomment}
Check the accuracy of the last iterate.
\end{xtccomment}
\begin{spadsrc}
a^3
\end{spadsrc}
\begin{TeXOutput}
\begin{fricasmath}{23}
\STRING{2.0000000000000275}-{\STRING{1.7502021699542322e-14}\TIMES %
\ImaginaryI }%
\end{fricasmath}
\end{TeXOutput}
\formatResultType{Complex(DoubleFloat)}
\end{xtc}

In \xmpref{MappingPackage1}, we show how functions can be
manipulated as objects in \Language{}.
A useful operation to consider here is \spadop{*}, which means
composition.
For example \spad{g*g} causes the Newton iteration formula
to be applied twice.
Correspondingly, \spad{g^n} means to apply the iteration formula
\spad{n} times.

%
\begin{xtc}
\begin{xtccomment}
Apply \spad{g} twice to the point \spad{1 + %i}.
\end{xtccomment}
\begin{spadsrc}
(g*g) (1.0 + %i)
\end{spadsrc}
\begin{TeXOutput}
\begin{fricasmath}{24}
\STRING{1.1644444444444444}-{\STRING{0.7377777777777778}\TIMES \ImaginaryI }%
\end{fricasmath}
\end{TeXOutput}
\formatResultType{Complex(DoubleFloat)}
\end{xtc}
\begin{xtc}
\begin{xtccomment}
Apply \spad{g} 11 times.
\end{xtccomment}
\begin{spadsrc}
(g^11) (1.0 + %i)
\end{spadsrc}
\begin{TeXOutput}
\begin{fricasmath}{25}
\STRING{1.2599210498948732}%
\end{fricasmath}
\end{TeXOutput}
\formatResultType{Complex(DoubleFloat)}
\end{xtc}

Look now at the vector field and surface generated
after two steps of Newton's formula for the cube root of two.
The poles in these pictures represent bad starting values, and the
flat areas are the regions of convergence to the three roots.
%
\begin{psXtc}
\begin{xtccomment}
The vector field.
\end{xtccomment}
\begin{spadsrc}
drawComplexVectorField(g^3,-3..3,-3..3)
\end{spadsrc}
\epsffile[0 0 295 295]{vectorRoot.ps}
\end{psXtc}
\begin{psXtc}
\begin{xtccomment}
The surface.
\end{xtccomment}
\begin{spadsrc}
drawComplex(g^3,-3..3,-3..3)
\end{spadsrc}
\epsffile[0 0 295 295]{complexRoot.ps}
\end{psXtc}

% !! DO NOT MODIFY THIS FILE BY HAND !! Created by spool2tex.awk.

% Copyright (c) 1991-2002, The Numerical ALgorithms Group Ltd.
% All rights reserved.
%
% Redistribution and use in source and binary forms, with or without
% modification, are permitted provided that the following conditions are
% met:
%
%     - Redistributions of source code must retain the above copyright
%       notice, this list of conditions and the following disclaimer.
%
%     - Redistributions in binary form must reproduce the above copyright
%       notice, this list of conditions and the following disclaimer in
%       the documentation and/or other materials provided with the
%       distribution.
%
%     - Neither the name of The Numerical ALgorithms Group Ltd. nor the
%       names of its contributors may be used to endorse or promote products
%       derived from this software without specific prior written permission.
%
% THIS SOFTWARE IS PROVIDED BY THE COPYRIGHT HOLDERS AND CONTRIBUTORS "AS
% IS" AND ANY EXPRESS OR IMPLIED WARRANTIES, INCLUDING, BUT NOT LIMITED
% TO, THE IMPLIED WARRANTIES OF MERCHANTABILITY AND FITNESS FOR A
% PARTICULAR PURPOSE ARE DISCLAIMED. IN NO EVENT SHALL THE COPYRIGHT OWNER
% OR CONTRIBUTORS BE LIABLE FOR ANY DIRECT, INDIRECT, INCIDENTAL, SPECIAL,
% EXEMPLARY, OR CONSEQUENTIAL DAMAGES (INCLUDING, BUT NOT LIMITED TO,
% PROCUREMENT OF SUBSTITUTE GOODS OR SERVICES-- LOSS OF USE, DATA, OR
% PROFITS-- OR BUSINESS INTERRUPTION) HOWEVER CAUSED AND ON ANY THEORY OF
% LIABILITY, WHETHER IN CONTRACT, STRICT LIABILITY, OR TORT (INCLUDING
% NEGLIGENCE OR OTHERWISE) ARISING IN ANY WAY OUT OF THE USE OF THIS
% SOFTWARE, EVEN IF ADVISED OF THE POSSIBILITY OF SUCH DAMAGE.

% Here and throughout the book we should use the terminology
% "type of a function", rather than talking about source and target.
% This is how the brave new world of SMWATT regards them. A function
% is just an object that has a mapping type.
%
% *********************************************************************
\head{chapter}{Packages}{ugPackages}
% *********************************************************************

Packages provide the bulk of
\index{package}
\Language{}'s algorithmic library, from numeric packages for computing
special functions to symbolic facilities for
\index{constructor!package}
differential equations, symbolic integration, and limits.
\index{package!constructor}

In \chapref{ugIntProg}, we developed several useful functions for drawing
vector fields and complex functions.
We now show you how you can add these functions to the
\Language{} library to make them available for general use.

The way we created the functions in \chapref{ugIntProg} is typical of how
you, as an advanced \Language{} user, may interact with \Language{}.
You have an application.
You go to your editor and create an input file defining some
functions for the application.
Then you run the file and try the functions.
Once you get them all to work, you will often want to extend them,
add new features, perhaps write additional functions.

Eventually, when you have a useful set of functions for your application,
you may want to add them to your local \Language{} library.
To do this, you embed these function definitions in a package and add
that package to the library.

To introduce new packages, categories, and domains into the system,
you need to use the \Language{} compiler to convert the constructors
into executable machine code.
An existing compiler in \Language{} is available on an ``as-is''
basis.
A new, faster compiler will be available in version 2.0
of \Language{}.

\begin{figXmpLines}[caption={The DrawComplex package.},label={fig-pak-cdraw}]
C      ==> Complex DoubleFloat                       -- All constructors used in a file
S      ==> Segment DoubleFloat                       -- \quad{}must be spelled out in full
INT    ==> Integer                                   -- \quad{}unless abbreviated by macros
DFLOAT ==> DoubleFloat                               -- \quad{}like these at the top of
VIEW3D ==> ThreeDimensionalViewport                  -- \quad{}a file.
CURVE  ==> List List Point DFLOAT

)abbrev package DRAWCX DrawComplex                   -- Identify kinds and abbreviations
DrawComplex(): Exports == Implementation where       -- Type definition begins here.

  Exports == with                                    -- Export part begins.
    drawComplex: (C -> C,S,S,Boolean) -> VIEW3D      -- Exported Operations
    drawComplexVectorField: (C -> C,S,S) -> VIEW3D
    setRealSteps: INT -> INT
    setImagSteps: INT -> INT
    setClipValue: DFLOAT-> DFLOAT

  Implementation == add                              -- Implementation part begins.
    arrowScale : DFLOAT := (0.2)::DFLOAT             -- (relative size) Local variable 1.
    arrowAngle : DFLOAT := pi()-pi()/(20::DFLOAT)    -- Local variable 2.
    realSteps  : INT := 11                           -- (real steps) Local variable 3.
    imagSteps  : INT := 11                           -- (imaginary steps) Local variable 4.
    clipValue  : DFLOAT  := 10::DFLOAT               -- (maximum vector length) Local variable 5.

    setRealSteps(n) == realSteps := n                -- Exported function definition 1.
    setImagSteps(n) == imagSteps := n                -- Exported function definition 2.
    setClipValue(c) == clipValue := c                -- Exported function definition 3.

    clipFun: DFLOAT -> DFLOAT                        -- Clip large magnitudes.
    clipFun(x) == min(max(x, -clipValue), clipValue) -- Local function definition 1.

    makeArrow: (Point DFLOAT,Point DFLOAT,DFLOAT,DFLOAT) -> CURVE
    makeArrow(p1, p2, len, arg) == ...               -- Local function definition 2.

    drawComplex(f, realRange, imagRange, arrows?) == ... -- Exported function definition 4.
\end{figXmpLines}

% *********************************************************************
\head{section}{Names, Abbreviations, and File Structure}{ugPackagesNames}
% *********************************************************************
%
Each package has a name and an abbreviation.
For a package of the complex draw functions from \chapref{ugIntProg},
we choose the name \nonLibAxiomType{DrawComplex}
and
\index{abbreviation!constructor}
abbreviation \nonLibAxiomType{DRAWCX}.\footnote{An abbreviation can be any string
of
\index{constructor!abbreviation}
between two and seven capital letters and digits, beginning with a letter.
See \spadref{ugTypesWritingAbbr} for more information.}
To be sure that you have not chosen a name or abbreviation already used by
the system, issue the system command \spadsys{)show} for both the name and
the abbreviation.
\syscmdindex{show}

Once you have named the package and its abbreviation, you can choose any new
filename you like with extension ``{\bf \spadFileExt{}}'' to hold the
definition of your package.
We choose the name {\bf drawpak\spadFileExt{}}.
If your application involves more than one package, you
can put them all in the same file.
\Language{} assumes no relationship between the name of a library file, and
the name or abbreviation of a package.

Near the top of the ``{\bf \spadFileExt{}}'' file, list all the
abbreviations for the packages
using \spadsys{)abbrev}, each command beginning in column one.
Macros giving names to \Language{} expressions can also be placed near the
top of the file.
The macros are only usable from their point of definition until the
end of the file.

Consider the definition of
\nonLibAxiomType{DrawComplex} in Figure \ref{fig-pak-cdraw}.
After the macro
\index{macro}
definition
\begin{verbatim}
S      ==> Segment DoubleFloat
\end{verbatim}
the name
{\tt S} can be used in the file as a
shorthand for \spadtype{Segment DoubleFloat}.\footnote{The interpreter also allows
{\tt macro} for macro definitions.}
The abbreviation command for the package
\begin{verbatim}
)abbrev package DRAWCX DrawComplex
\end{verbatim}
is given after the macros (although it could precede them).

% *********************************************************************
\head{section}{Syntax}{ugPackagesSyntax}
% *********************************************************************
%
The definition of a package has the syntax:
\begin{center}
\frenchspacing{\it PackageForm {\tt :} Exports\quad{\tt ==}\quad Implementation}
\end{center}
The syntax for defining a package constructor is the same as that
\index{syntax}
for defining any function in \Language{}.
In practice, the definition extends over many lines so that this syntax is
not practical.
Also, the type of a package is expressed by the operator \spad{with}
\spadkey{with}
followed by an explicit list of operations.
A preferable way to write the definition of a package is with a \spad{where}
\spadkey{where}
expression:

% ----------------------------------------------------------------------
\beginImportant
The definition of a package usually has the form: \newline
{\tt%
{\it PackageForm} : Exports  ==  Implementation where \newline
\hspace*{.75pc} {\it optional type declarations}\newline
\hspace*{.75pc} Exports  ==   with \newline
\hspace*{2.0pc}   {\it list of exported operations}\newline
\hspace*{.75pc} Implementation == add \newline
\hspace*{2.0pc}   {\it list of function definitions for exported operations}
}
\endImportant
% ----------------------------------------------------------------------

The \spadtype{DrawComplex} package takes no parameters and exports five
operations, each a separate item of a \spadgloss{pile}.
Each operation is described as a \spadgloss{declaration}: a name, followed
by a colon (\spadSyntax{:}), followed by the type of the operation.
All operations have types expressed as \spadglossSee{mappings}{mapping} with
the syntax
\begin{center}
{\it
source\quad{\tt ->}\quad target
}
\end{center}

%e *********************************************************************
\head{section}{Abstract Datatypes}{ugPackagesAbstract}
% *********************************************************************

A constructor as defined in \Language{} is called an \spadgloss{abstract
datatype} in the computer science literature.
Abstract datatypes separate ``specification'' (what operations are
provided) from ``implementation'' (how the operations are implemented).
The {\tt Exports} (specification) part of a constructor is said to be ``public'' (it
provides the user interface to the package) whereas the {\tt Implementation}
part is ``private'' (information here is effectively hidden---programs
cannot take advantage of it).

The {\tt Exports} part specifies what operations the package provides to users.
As an author of a package, you must ensure that
the {\tt Implementation} part provides a function for each
operation in the {\tt Exports} part.\footnote{The \spadtype{DrawComplex}
package enhances the facility
described in  \chapref{ugIntProgCompFuns} by allowing a
complex function to have
arrows emanating from the surface to indicate the direction of the
complex argument.}

An important difference between interactive programming and the
use of packages is in the handling of global variables such as
\spad{realSteps} and \spad{imagSteps}.
In interactive programming, you simply change the values of
variables by \spadgloss{assignment}.
With packages, such variables are local to the package---their
values can only be set using functions exported by the package.
In our example package, we provide two functions
\fakeAxiomFun{setRealSteps} and \fakeAxiomFun{setImagSteps} for
this purpose.

Another local variable is \spad{clipValue} which can be changed using
the exported operation \fakeAxiomFun{setClipValue}.
This value is referenced by the internal function \fakeAxiomFun{clipFun} that
decides whether to use the computed value of the function at a point or,
if the magnitude of that value is too large, the
value assigned to \spad{clipValue} (with the
appropriate sign).

% *********************************************************************
\head{section}{Capsules}{ugPackagesCapsules}
% *********************************************************************
%
The part to the right of {\tt add} in the {\tt Implementation}
\spadkey{add}
part of the definition is called a \spadgloss{capsule}.
The purpose of a capsule is:
\begin{itemize}
\item to define a function for each exported operation, and
\item to define a \spadgloss{local environment} for these functions to run.
\end{itemize}

What is a local environment?
First, what is an environment?
\index{environment}
Think of the capsule as an input file that \Language{} reads from top to
bottom.
Think of the input file as having a \spad{)clear all} at the top
so that initially no variables or functions are defined.
When this file is read, variables such as \spad{realSteps} and
\spad{arrowSize} in \nonLibAxiomType{DrawComplex} are set to initial values.
Also, all the functions defined in the capsule are compiled.
These include those that are exported (like \spad{drawComplex}), and
those that are not (like \spad{makeArrow}).
At the end, you get a set of name-value pairs:
variable names (like \spad{realSteps} and \spad{arrowSize})
are paired with assigned values, while
operation names (like \spad{drawComplex} and \spad{makeArrow})
are paired with function values.

This set of name-value pairs is called an \spadgloss{environment}.
Actually, we call this environment the ``initial environment'' of a package:
it is the environment that exists immediately after the package is
first built.
Afterwards, functions of this capsule can
access or reset a variable in the environment.
The environment is called {\it local} since any changes to the value of a
variable in this environment can be seen {\it only} by these functions.

Only the functions from the package can change the variables in the local
environment.
When two functions are called successively from a package,
any changes caused by the first function called
are seen by the second.

Since the environment is local to the package, its names
don't get mixed
up with others in the system or your workspace.
If you happen to have a variable called \spad{realSteps} in your
workspace, it does not affect what the
\nonLibAxiomType{DrawComplex} functions do in any way.

The functions in a package are compiled into machine code.
Unlike function definitions in input files that may be compiled repeatedly
as you use them with varying argument types,
functions in packages have a unique type (generally parameterized by
the argument parameters of a package) and a unique compilation residing on disk.

The capsule itself is turned into a compiled function.
This so-called {\it capsule function} is what builds the initial environment
spoken of above.
If the package has arguments (see below), then each call to the package
constructor with a distinct pair of arguments
builds a distinct package, each with its own local environment.

% *********************************************************************
\head{section}{Input Files vs. Packages}{ugPackagesInputFiles}
% *********************************************************************
%
A good question at this point would be ``Is writing a package more difficult than
writing an input file?''

The programs in input files are designed for flexibility and ease-of-use.
\Language{} can usually work out all of your types as it reads your program
and does the computations you request.
Let's say that you define a one-argument function without giving its type.
When you first apply the function to a value, this
value is understood by \Language{} as identifying the type for the
argument parameter.
Most of the time \Language{} goes through the body of your function and
figures out the target type that you have in mind.
\Language{} sometimes fails to get it right.
Then---and only then---do you need a declaration to tell \Language{} what
type you want.

Input files are usually written to be read by \Language{}---and by you.
\index{file!input!vs. package}
Without suitable documentation and declarations, your input files
\index{package!vs. input file}
are likely incomprehensible to a colleague---and to you some
months later!

Packages are designed for legibility, as well as
run-time efficiency.
There are few new concepts you need to learn to write
packages. Rather, you just have to be explicit about types
and type conversions.
The types of all functions are pre-declared so that \Language{}---and the reader---
knows precisely what types of arguments can be passed to and from
the functions (certainly you don't want a colleague to guess or to
have to work this out from context!).
The types of local variables are also declared.
Type conversions are explicit, never automatic.\footnote{There
is one exception to this rule: conversions from a subdomain to a
domain are automatic.
After all, the objects both have the domain as a common type.}

In summary, packages are more tedious to write than input files.
When writing input files, you can casually go ahead, giving some
facts now, leaving others for later.
Writing packages requires forethought, care and discipline.

% *********************************************************************
\head{section}{Compiling Packages}{ugPackagesPackages}
% *********************************************************************
%

Once you have defined the package \nonLibAxiomType{DrawComplex},
you need to compile and test it.
To compile the package, issue the system command \spadsys{)compile drawpak}.
\Language{} reads the file {\bf drawpak\spadFileExt{}}
and compiles its contents into machine binary.
If all goes well, the file {\bf DRAWCX.NRLIB} is created in your
local directory for the package.
To test the package, you must load the package before trying an
operation.

\begin{nullXtc}
\begin{xtccomment}
Compile the package.
\end{xtccomment}
\begin{spadsrc}
)compile drawpak
\end{spadsrc}
\end{nullXtc}
\begin{xtc}
\begin{xtccomment}
Expose the package.
\end{xtccomment}
\begin{spadsrc}
)expose DRAWCX 
\end{spadsrc}
\begin{SysCmdOutput}
   DrawComplex is now explicitly exposed in frame initial 
\end{SysCmdOutput}
\end{xtc}
\begin{xtc}
\begin{xtccomment}
Use an odd step size to avoid
a pole at the origin.
\end{xtccomment}
\begin{spadsrc}
setRealSteps 51 
\end{spadsrc}
\begin{TeXOutput}
\begin{fricasmath}{1}
51%
\end{fricasmath}
\end{TeXOutput}
\formatResultType{PositiveInteger}
\end{xtc}
\begin{xtc}
\begin{xtccomment}
\end{xtccomment}
\begin{spadsrc}
setImagSteps 51 
\end{spadsrc}
\begin{TeXOutput}
\begin{fricasmath}{2}
51%
\end{fricasmath}
\end{TeXOutput}
\formatResultType{PositiveInteger}
\end{xtc}
\begin{xtc}
\begin{xtccomment}
Define \userfun{f} to be the Gamma function.
\end{xtccomment}
\begin{spadsrc}
f(z) == Gamma(z) 
\end{spadsrc}
\end{xtc}
\begin{xtc}
\begin{xtccomment}
Clip values of function with magnitude larger than 7.
\end{xtccomment}
\begin{spadsrc}
setClipValue 7
\end{spadsrc}
\begin{TeXOutput}
\begin{fricasmath}{4}
\STRING{7.0}%
\end{fricasmath}
\end{TeXOutput}
\formatResultType{DoubleFloat}
\end{xtc}
\begin{psXtc}
\begin{xtccomment}
Draw the \spadfun{Gamma} function.
\end{xtccomment}
\begin{spadsrc}
drawComplex(f,-%pi..%pi,-%pi..%pi, false) 
\end{spadsrc}
\epsffile[0 0 300 300]{3Dgamma11.ps}
\end{psXtc}

% *********************************************************************
\head{section}{Parameters}{ugPackagesParameters}
% *********************************************************************
%
The power of packages becomes evident when packages have parameters.
Usually these parameters are domains and the exported operations have types
involving these parameters.

In \chapref{ugTypes}, you learned that categories denote classes of
domains.
Although we cover this notion in detail in the next
chapter, we now give you a sneak preview of its usefulness.

In \spadref{ugUserBlocks}, we defined functions \spad{bubbleSort(m)} and
\spad{insertionSort(m)} to sort a list of integers.
If you look at the code for these functions, you see that they may be
used to sort {\it any} structure \spad{m} with the right properties.
Also, the functions can be used to sort lists of {\it any} elements---not
just integers.
Let us now recall the code for \spad{bubbleSort}.

\begin{verbatim}
bubbleSort(m) ==
  n := #m
  for i in 1..(n-1) repeat
    for j in n..(i+1) by -1 repeat
      if m.j < m.(j-1) then swap!(m,j,j-1)
  m
\end{verbatim}

What properties of ``lists of integers'' are assumed by the sorting
algorithm?
In the first line, the operation \spadop{#} computes the maximum index of
the list.
The first obvious property is that \spad{m} must have a finite number of
elements.
In \Language{}, this is done
by your telling \Language{} that \spad{m} has
the ``attribute'' \spadtype{finiteAggregate}.
An \spadgloss{attribute} is a property
that a domain either has or does not have.
As we show later in \spadref{ugCategoriesAttributes},
programs can query domains as to the presence or absence of an attribute.

The operation \spadfunX{swap} swaps elements of \spad{m}.
Using \Browse{}, you find that \spadfunX{swap} requires its
elements to come from a domain of category
\spadtype{IndexedAggregate} with attribute
\spadtype{shallowlyMutable}.
This attribute means that you can change the internal components
of \spad{m} without changing its external structure.
Shallowly-mutable data structures include lists, streams, one- and
two-dimensional arrays, vectors, and matrices.

The category \spadtype{IndexedAggregate} designates the class of
aggregates whose elements can be accessed by the notation
\spad{m.s} for suitable selectors \spad{s}.
The category \spadtype{IndexedAggregate} takes two arguments:
\spad{Index}, a domain of selectors for the aggregate, and
\spad{Entry}, a domain of entries for the aggregate.
Since the sort functions access elements by integers, we must
choose \spad{Index = }\spadtype{Integer}.
The most general class of domains for which \spad{bubbleSort} and
\spad{insertionSort} are defined are those of
category \spadtype{IndexedAggregate(Integer,Entry)} with the two
attributes \spadtype{shallowlyMutable} and
\spadtype{finiteAggregate}.

Using \Browse{}, you can also discover that \Language{} has many kinds of domains
with attribute \spadtype{shallowlyMutable}.
Those of class \spadtype{IndexedAggregate(Integer,Entry)} include
\spadtype{Bits}, \spadtype{FlexibleArray}, \spadtype{OneDimensionalArray},
\spadtype{List}, \spadtype{String}, and \spadtype{Vector}, and also
\spadtype{HashTable} and \spadtype{EqTable} with integer keys.
Although you may never want to sort all such structures, we
nonetheless demonstrate \Language{}'s
ability to do so.

Another requirement is that \nonLibAxiomType{Entry} has an
operation \spadop{<}.
One way to get this operation is to assume that
\nonLibAxiomType{Entry} has category \spadtype{OrderedSet}.
By definition, will then export a \spadop{<} operation.
A more general approach is to allow any comparison function
\spad{f} to be used for sorting.
This function will be passed as an argument to the sorting
functions.

Our sorting package then takes two arguments: a domain \spad{S}
of objects of {\it any} type, and a domain \spad{A}, an aggregate
of type \spadtype{IndexedAggregate(Integer, S)} with the above
two attributes.
Here is its definition using what are close to the original
definitions of \spad{bubbleSort} and \spad{insertionSort} for
sorting lists of integers.
The symbol \spadSyntax{!} is added to the ends of the operation
names.
This uniform naming convention is used for \Language{} operation
names that destructively change one or more of their arguments.

\begin{xmpLines}
SortPackage(S,A) : Exports == Implementation where
  S: Object
  A: IndexedAggregate(Integer,S)
    with (finiteAggregate; shallowlyMutable)

  Exports == with
    bubbleSort!: (A,(S,S) -> Boolean) -> A
    insertionSort!: (A, (S,S) -> Boolean) -> A

  Implementation == add
    bubbleSort!(m,f) ==
      n := #m
      for i in 1..(n-1) repeat
        for j in n..(i+1) by -1 repeat
          if f(m.j,m.(j-1)) then swap!(m,j,j-1)
      m
    insertionSort!(m,f) ==
      for i in 2..#m repeat
        j := i
        while j > 1 and f(m.j,m.(j-1)) repeat
          swap!(m,j,j-1)
          j := (j - 1) pretend PositiveInteger
      m
\end{xmpLines}

% *********************************************************************
\head{section}{Conditionals}{ugPackagesConds}
% *********************************************************************
%
When packages have parameters, you can say that an operation is or is not
\index{conditional}
exported depending on the values of those parameters.
When the domain of objects \spad{S} has an \spadop{<}
operation, we can supply one-argument versions of
\spad{bubbleSort} and \spad{insertionSort} which use this operation
for sorting.
The presence of the
operation \spadop{<} is guaranteed when \spad{S} is an ordered set.

\begin{xmpLines}
Exports == with
    bubbleSort!: (A,(S,S) -> Boolean) -> A
    insertionSort!: (A, (S,S) -> Boolean) -> A

    if S has OrderedSet then
      bubbleSort!: A -> A
      insertionSort!: A -> A
\end{xmpLines}

In addition to exporting the one-argument sort operations
\index{sort!bubble}
conditionally, we must provide conditional definitions for the
\index{sort!insertion}
operations in the {\tt Implementation} part.
This is easy: just have the one-argument functions call the
corresponding two-argument functions with the operation
\spadop{<} from \spad{S}.

\begin{xmpLines}
  Implementation == add
       ...
    if S has OrderedSet then
      bubbleSort!(m) == bubbleSort!(m,<$S)
      insertionSort!(m) == insertionSort!(m,<$S)
\end{xmpLines}

In \spadref{ugUserBlocks}, we give an alternative definition of
\fakeAxiomFun{bubbleSort} using \spadfunFrom{first}{List} and
\spadfunFrom{rest}{List} that is more efficient for a list (for
which access to any element requires traversing the list from its
first node).
To implement a more efficient algorithm for lists, we need the
operation \spadfunX{setelt} which allows us to destructively change
the \spadfun{first} and \spadfun{rest} of a list.
Using \Browse{}, you find that these operations come from category
\spadtype{UnaryRecursiveAggregate}.
Several aggregate types are unary recursive aggregates including
those of \spadtype{List} and \spadtype{AssociationList}.
We provide two different implementations for
\fakeAxiomFun{bubbleSort!} and \fakeAxiomFun{insertionSort!}: one
for list-like structures, another for array-like structures.

\begin{xmpLines}
Implementation == add
        ...
    if A has UnaryRecursiveAggregate(S) then
      bubbleSort!(m,fn) ==
        empty? m => m
        l := m
        while not empty? (r := l.rest) repeat
           r := bubbleSort! r
           x := l.first
           if fn(r.first,x) then
             l.first := r.first
             r.first := x
           l.rest := r
           l := l.rest
         m
       insertionSort!(m,fn) ==
          ...
\end{xmpLines}

The ordering of definitions is important.
The standard definitions come first and
then the predicate
\begin{verbatim}
A has UnaryRecursiveAggregate(S)
\end{verbatim}
is evaluated.
If {\tt true}, the special definitions cover up the standard ones.

Another equivalent way to write the capsule is to use an
\spad{if-then-else} expression:
\spadkey{if}

\begin{xmpLines}
     if A has UnaryRecursiveAggregate(S) then
        ...
     else
        ...
\end{xmpLines}

% *********************************************************************
\head{section}{Testing}{ugPackagesCompiling}
% *********************************************************************
%
Once you have written the package, embed it in a file, for example,
{\bf sortpak\spadFileExt{}}.
\index{testing}
Be sure to include an \spad{)abbrev} command at the top of the file:
\begin{verbatim}
)abbrev package SORTPAK SortPackage
\end{verbatim}
Now compile the file (using \spadsys{)compile sortpak\spadFileExt{}}).
\begin{xtc}
\begin{xtccomment}
Expose the constructor.
You are then ready to begin testing.
\end{xtccomment}
\begin{spadsrc}
)expose SORTPAK
\end{spadsrc}
\begin{SysCmdOutput}
   SortPackage is now explicitly exposed in frame initial 
\end{SysCmdOutput}
\end{xtc}
\begin{xtc}
\begin{xtccomment}
Define a list.
\end{xtccomment}
\begin{spadsrc}
l := [1,7,4,2,11,-7,3,2]
\end{spadsrc}
\begin{TeXOutput}
\begin{fricasmath}{1}
\BRACKET{1\COMMA 7\COMMA 4\COMMA 2\COMMA 11\COMMA -{7}\COMMA 3\COMMA 2}%
\end{fricasmath}
\end{TeXOutput}
\formatResultType{List(Integer)}
\end{xtc}
\begin{xtc}
\begin{xtccomment}
Since the integers are an ordered set,
a one-argument operation will do.
\end{xtccomment}
\begin{spadsrc}
bubbleSort!(l)
\end{spadsrc}
\begin{TeXOutput}
\begin{fricasmath}{2}
\BRACKET{-{7}\COMMA 1\COMMA 2\COMMA 2\COMMA 3\COMMA 4\COMMA 7\COMMA 11}%
\end{fricasmath}
\end{TeXOutput}
\formatResultType{List(Integer)}
\end{xtc}
\begin{xtc}
\begin{xtccomment}
Re-sort it using ``greater than.''
\end{xtccomment}
\begin{spadsrc}
bubbleSort!(l,(x,y) +-> x > y)
\end{spadsrc}
\begin{TeXOutput}
\begin{fricasmath}{3}
\BRACKET{11\COMMA 7\COMMA 4\COMMA 3\COMMA 2\COMMA 2\COMMA 1\COMMA -{7}}%
\end{fricasmath}
\end{TeXOutput}
\formatResultType{List(Integer)}
\end{xtc}
\begin{xtc}
\begin{xtccomment}
Now sort it again using \spadop{<} on integers.
\end{xtccomment}
\begin{spadsrc}
bubbleSort!(l, <$Integer)
\end{spadsrc}
\begin{TeXOutput}
\begin{fricasmath}{4}
\BRACKET{-{7}\COMMA 1\COMMA 2\COMMA 2\COMMA 3\COMMA 4\COMMA 7\COMMA 11}%
\end{fricasmath}
\end{TeXOutput}
\formatResultType{List(Integer)}
\end{xtc}
\begin{xtc}
\begin{xtccomment}
A string is an aggregate of characters so we can sort them as well.
\end{xtccomment}
\begin{spadsrc}
bubbleSort! "Mathematical Sciences"
\end{spadsrc}
\begin{TeXOutput}
\begin{fricasmath}{5}
\STRING{"\ MSaaaccceeehiilmnstt"}%
\end{fricasmath}
\end{TeXOutput}
\formatResultType{String}
\end{xtc}
\begin{xtc}
\begin{xtccomment}
Is \spadop{<} defined on booleans?
\end{xtccomment}
\begin{spadsrc}
false < true
\end{spadsrc}
\begin{TeXOutput}
\begin{fricasmath}{6}
\STRING{true}%
\end{fricasmath}
\end{TeXOutput}
\formatResultType{Boolean}
\end{xtc}
\begin{xtc}
\begin{xtccomment}
Good! Create a bit string representing ten consecutive
boolean values \spad{true}.
\end{xtccomment}
\begin{spadsrc}
u : Bits := new(10,true)
\end{spadsrc}
\begin{TeXOutput}
\begin{fricasmath}{7}
\STRING{"1111111111"}%
\end{fricasmath}
\end{TeXOutput}
\formatResultType{Bits}
\end{xtc}
\begin{xtc}
\begin{xtccomment}
Set bits 3 through 5 to \spad{false}, then display the result.
\end{xtccomment}
\begin{spadsrc}
u(3..5) := false; u
\end{spadsrc}
\begin{TeXOutput}
\begin{fricasmath}{8}
\STRING{"1100011111"}%
\end{fricasmath}
\end{TeXOutput}
\formatResultType{Bits}
\end{xtc}
\begin{xtc}
\begin{xtccomment}
Now sort these booleans.
\end{xtccomment}
\begin{spadsrc}
bubbleSort! u
\end{spadsrc}
\begin{TeXOutput}
\begin{fricasmath}{9}
\STRING{"0001111111"}%
\end{fricasmath}
\end{TeXOutput}
\formatResultType{Bits}
\end{xtc}
\begin{xtc}
\begin{xtccomment}
Create an ``eq-table'' (see \xmpref{EqTable}), a
table having integers as keys
and strings as values.
\end{xtccomment}
\begin{spadsrc}
t : EqTable(Integer,String) := table()
\end{spadsrc}
\begin{TeXOutput}
\begin{fricasmath}{10}
\STRING{table}\PAREN{}%
\end{fricasmath}
\end{TeXOutput}
\formatResultType{EqTable(Integer, String)}
\end{xtc}
\begin{xtc}
\begin{xtccomment}
Give the table a first entry.
\end{xtccomment}
\begin{spadsrc}
t.1 := "robert"
\end{spadsrc}
\begin{TeXOutput}
\begin{fricasmath}{11}
\STRING{"robert"}%
\end{fricasmath}
\end{TeXOutput}
\formatResultType{String}
\end{xtc}
\begin{xtc}
\begin{xtccomment}
And a second.
\end{xtccomment}
\begin{spadsrc}
t.2 := "richard"
\end{spadsrc}
\begin{TeXOutput}
\begin{fricasmath}{12}
\STRING{"richard"}%
\end{fricasmath}
\end{TeXOutput}
\formatResultType{String}
\end{xtc}
\begin{xtc}
\begin{xtccomment}
What does the table look like?
\end{xtccomment}
\begin{spadsrc}
t
\end{spadsrc}
\begin{TeXOutput}
\begin{fricasmath}{13}
\STRING{table}\PAREN{2=\STRING{"richard"},1=\STRING{"robert"}}%
\end{fricasmath}
\end{TeXOutput}
\formatResultType{EqTable(Integer, String)}
\end{xtc}
\begin{xtc}
\begin{xtccomment}
Now sort it.
\end{xtccomment}
\begin{spadsrc}
bubbleSort! t
\end{spadsrc}
\begin{TeXOutput}
\begin{fricasmath}{14}
\STRING{table}\PAREN{2=\STRING{"robert"},1=\STRING{"richard"}}%
\end{fricasmath}
\end{TeXOutput}
\formatResultType{EqTable(Integer, String)}
\end{xtc}

% *********************************************************************
\head{section}{How Packages Work}{ugPackagesHow}
% *********************************************************************
%
Recall that packages as abstract datatypes are compiled independently
and put into the library.
The curious reader may ask: ``How is the interpreter able to find an
operation such as \fakeAxiomFun{bubbleSort!}?
Also, how is a single compiled function such as \fakeAxiomFun{bubbleSort!} able
to sort data of different types?''

After the interpreter loads the package \nonLibAxiomType{SortPackage}, the four
operations from the package become known to the interpreter.
Each of these operations is expressed as a {\it modemap} in which the type
\index{modemap}
of the operation is written in terms of symbolic domains.
\begin{noOutputXtc}
\begin{xtccomment}
\end{xtccomment}
\begin{spadsrc}
)expose SORTPAK
\end{spadsrc}
\begin{SysCmdOutput}
   SortPackage is already explicitly exposed in frame initial 
\end{SysCmdOutput}
\end{noOutputXtc}

\begin{xtc}
\begin{xtccomment}
See the modemaps for \fakeAxiomFun{bubbleSort!}.
\end{xtccomment}
\begin{spadsrc}
)display op bubbleSort!
\end{spadsrc}
\begin{SysCmdOutput}

There are 2 exposed functions called bubbleSort! :
   [1] (D1,((D3,D3) -> Boolean)) -> D1 from SortPackage(D3,D1)
            if D3 has TYPE and D1 has Join(IXAGG(INT,D3),ATFINAG,
            ATSHMUT)
   [2] D1 -> D1 from SortPackage(D2,D1)
            if D2 has ORDSET and D2 has TYPE and D1 has Join(IXAGG(INT,
            D2),ATFINAG,ATSHMUT)
\end{SysCmdOutput}
\end{xtc}

What happens if you ask for \spad{bubbleSort!([1,-5,3])}?
There is a unique modemap for an operation named
\fakeAxiomFun{bubbleSort!} with one argument.
Since \spad{[1,-5,3]} is a list of integers, the symbolic domain
\spad{D1} is defined as \spadtype{List(Integer)}.
For some operation to apply, it must satisfy the predicate for
some \spad{D2}.
What \spad{D2}?
The third expression of the \spad{and} requires {\tt D1 has
IndexedAggregate(Integer, D2) with} two attributes.
So the interpreter searches for an \spadtype{IndexedAggregate}
among the ancestors of \spadtype{List (Integer)} (see
\spadref{ugCategoriesHier}).
It finds one: \spadtype{IndexedAggregate(Integer, Integer)}.
The interpreter tries defining \spad{D2} as \spadtype{Integer}.
After substituting for \spad{D1} and \spad{D2}, the predicate
evaluates to \spad{true}.
An applicable operation has been found!

Now \Language{} builds the package
\spadtype{SortPackage(List(Integer), Integer)}.
According to its definition, this package exports the required
operation:
\fakeAxiomFun{bubbleSort!}: \spad{List Integer->List Integer}.
The interpreter then asks the package for a function implementing
this operation.
The package gets all the functions it needs (for example,
\spadfun{rest} and \spadfunX{swap}) from the appropriate
domains and then it
returns a \fakeAxiomFun{bubbleSort!} to the interpreter together with
the local environment for \fakeAxiomFun{bubbleSort!}.
The interpreter applies the function to the argument \spad{[1,-5,3]}.
The \fakeAxiomFun{bubbleSort!} function is executed in its local
environment and produces the result.

% !! DO NOT MODIFY THIS FILE BY HAND !! Created by spool2tex.awk.

% Copyright (c) 1991-2002, The Numerical ALgorithms Group Ltd.
% All rights reserved.
%
% Redistribution and use in source and binary forms, with or without
% modification, are permitted provided that the following conditions are
% met:
%
%     - Redistributions of source code must retain the above copyright
%       notice, this list of conditions and the following disclaimer.
%
%     - Redistributions in binary form must reproduce the above copyright
%       notice, this list of conditions and the following disclaimer in
%       the documentation and/or other materials provided with the
%       distribution.
%
%     - Neither the name of The Numerical ALgorithms Group Ltd. nor the
%       names of its contributors may be used to endorse or promote products
%       derived from this software without specific prior written permission.
%
% THIS SOFTWARE IS PROVIDED BY THE COPYRIGHT HOLDERS AND CONTRIBUTORS "AS
% IS" AND ANY EXPRESS OR IMPLIED WARRANTIES, INCLUDING, BUT NOT LIMITED
% TO, THE IMPLIED WARRANTIES OF MERCHANTABILITY AND FITNESS FOR A
% PARTICULAR PURPOSE ARE DISCLAIMED. IN NO EVENT SHALL THE COPYRIGHT OWNER
% OR CONTRIBUTORS BE LIABLE FOR ANY DIRECT, INDIRECT, INCIDENTAL, SPECIAL,
% EXEMPLARY, OR CONSEQUENTIAL DAMAGES (INCLUDING, BUT NOT LIMITED TO,
% PROCUREMENT OF SUBSTITUTE GOODS OR SERVICES-- LOSS OF USE, DATA, OR
% PROFITS-- OR BUSINESS INTERRUPTION) HOWEVER CAUSED AND ON ANY THEORY OF
% LIABILITY, WHETHER IN CONTRACT, STRICT LIABILITY, OR TORT (INCLUDING
% NEGLIGENCE OR OTHERWISE) ARISING IN ANY WAY OUT OF THE USE OF THIS
% SOFTWARE, EVEN IF ADVISED OF THE POSSIBILITY OF SUCH DAMAGE.


% *********************************************************************
\head{chapter}{Categories}{ugCategories}
% *********************************************************************

This chapter unravels the mysteries of categories---what
\index{category}
they are, how they are related to domains and packages,
\index{category!constructor}
how they are defined in \Language{}, and how you can extend the
\index{constructor!category}
system to include new categories of your own.

We assume that you have read the introductory material on domains
and categories in \spadref{ugTypesBasicDomainCons}.
There you learned that the notion of packages covered in the
previous chapter are special cases of domains.
While this is in fact the case, it is useful here to regard domains
as distinct from packages.

Think of a domain as a datatype, a collection of objects (the
objects of the domain).
From your ``sneak preview'' in the previous chapter, you might
conclude that categories are simply named clusters of operations
exported by domains.
As it turns out, categories have a much deeper meaning.
Categories are fundamental to the design of \Language{}.
They control the interactions between domains and algorithmic
packages, and, in fact, between all the components of \Language{}.

Categories form hierarchies as shown on the inside cover pages of
this book.
The inside front-cover pages illustrate the basic
algebraic hierarchy of the \Language{} programming language.
The inside back-cover pages show the hierarchy for data
structures.

Think of the category structures of \Language{} as a foundation
for a city on which superstructures (domains) are built.
The algebraic hierarchy, for example, serves as a foundation for
constructive mathematical algorithms embedded in the domains of
\Language{}.
Once in place, domains can be constructed, either independently or
from one another.

Superstructures are built for quality---domains are compiled into
machine code for run-time efficiency.
You can extend the foundation in directions beyond the space
directly beneath the superstructures, then extend selected
superstructures to cover the space.
Because of the compilation strategy, changing components of the
foundation generally means that the existing superstructures
(domains) built on the changed parts of the foundation
(categories) have to be rebuilt---that is, recompiled.

Before delving into some of the interesting facts about categories, let's see
how you define them in \Language{}.

% *********************************************************************
\head{section}{Definitions}{ugCategoriesDefs}
% *********************************************************************

A category is defined by a function with exactly the same format as
\index{category!definition}
any other function in \Language{}.

% ----------------------------------------------------------------------
\beginImportant
The definition of a category has the syntax:
\begin{center}
{\it CategoryForm} : {\tt Category\quad{}==\quad{}} {\it Extensions} {\tt [ with} {\it Exports} {\tt ]}
\end{center}

The brackets {\tt [ ]} here indicate optionality.
\endImportant
% ----------------------------------------------------------------------


The first example of a category definition is
\spadtype{SetCategory},
the most basic of the algebraic categories in \Language{}.
\exptypeindex{SetCategory}

\begin{xmpLines}
SetCategory(): Category ==
   Join(Type,CoercibleTo OutputForm) with
      "=" : (%, %) -> Boolean
\end{xmpLines}

The definition starts off with the name of the
category (\spadtype{SetCategory}); this is
always in column one in the source file.
%% maybe talk about naming conventions for source files? .spad or .ax?
All parts of a category definition are then indented with respect to this
\index{indentation}
first line.

In \chapref{ugTypes}, we talked about \spadtype{Ring} as denoting the
class of all domains that are rings, in short, the class of all
rings.
While this is the usual naming convention in \Language{}, it is also
common to use the word ``Category'' at the end of a category name for clarity.
The interpretation of the name \spadtype{SetCategory} is, then, ``the
category of all domains that are (mathematical) sets.''

The name \spadtype{SetCategory} is followed in the definition by its
formal parameters enclosed in parentheses \spadSyntax{()}.
Here there are no parameters.
As required, the type of the result of this category function is the
distinguished name {\sf Category}.

Then comes the \spadSyntax{==}.
As usual, what appears to the right of the \spadSyntax{==} is a
definition, here, a category definition.
A category definition always has two parts separated by the reserved word
\spadkey{with}
\spad{with}.
%\footnote{Debugging hint: it is very easy to forget
%the \spad{with}!}

The first part tells what categories the category extends.
Here, the category extends two categories: \spadtype{Type}, the
category of all domains, and
\spadtype{CoercibleTo(OutputForm)}.
%\footnote{\spadtype{CoercibleTo(OutputForm)}
%can also be written (and is written in the definition above) without
%parentheses.}
The operation \spad{Join} is a system-defined operation that
\spadkey{Join}
forms a single category from two or more other categories.

Every category other than \spadtype{Type} is an extension of some other
category.
If, for example, \spadtype{SetCategory} extended only the category
\spadtype{Type}, the definition here would read ``{\tt Type with
...}''.
In fact, the {\tt Type} is optional in this line; ``{\tt with
...}'' suffices.

% *********************************************************************
\head{section}{Exports}{ugCategoriesExports}
% *********************************************************************

To the right of the \spad{with} is a list of
\spadkey{with}
all the \spadglossSee{exports}{export} of the category.
Each exported operation has a name and a type expressed by a
\spadgloss{declaration} of the form
``{\frenchspacing\tt {\it name}: {\it type}}''.

Categories can export symbols, as well as
{\tt 0} and {\tt 1} which denote
domain constants.\footnote{The
numbers {\tt 0} and {\tt 1} are operation names in \Language{}.}
In the current implementation, all other exports are operations with
types expressed as \spadglossSee{mappings}{mapping} with the syntax
\begin{center}
{\it
source\quad{\tt ->}\quad target
}
\end{center}

The category \spadtype{SetCategory} has a single export: the operation
\spadop{=} whose type is given by the mapping \spad{(%, %) -> Boolean}.
The \spadSyntax{%} in a mapping type always means ``the domain.'' Thus
the operation \spadop{=} takes two arguments from the domain and
returns a value of type \spadtype{Boolean}.

The source part of the mapping here is given by a {\it tuple}
\index{tuple}
consisting of two or more types separated by commas and enclosed in
parentheses.
If an operation takes only one argument, you can drop the parentheses
around the source type.
If the mapping has no arguments, the source part of the mapping is either
left blank or written as \spadSyntax{()}.
Here are examples of formats of various operations with some
contrived names.

\begin{verbatim}
someIntegerConstant  :    %
aZeroArgumentOperation:   () -> Integer
aOneArgumentOperation:    Integer -> %
aTwoArgumentOperation:    (Integer,%) -> Void
aThreeArgumentOperation:  (%,Integer,%) -> Fraction(%)
\end{verbatim}

% *********************************************************************
\head{section}{Documentation}{ugCategoriesDoc}
% *********************************************************************

The definition of \spadtype{SetCategory} above is  missing
an important component: its library documentation.
\index{documentation}
Here is its definition, complete with documentation.

\begin{xmpLines}
++ Description:
++ \spadtype{SetCategory} is the basic category
++ for describing a collection of elements with
++ \spadop{=} (equality) and a \spadfun{coerce}
++ to \spadtype{OutputForm}.

SetCategory(): Category ==
  Join(Type, CoercibleTo OutputForm) with
    "=": (%, %) -> Boolean
      ++ \spad{x = y} tests if \spad{x} and
      ++ \spad{y} are equal.
\end{xmpLines}

Documentary comments are an important part of constructor definitions.
Documentation is given both for the category itself and for
each export.
A description for the category precedes the code.
Each line of the description begins in column one with \spadSyntax{++}.
The description starts with the word {\tt Description:}.\footnote{Other
information such as the author's name, date of creation, and so on,
can go in this
area as well but are currently ignored by \Language{}.}
All lines of the description following the initial line are
indented by the same amount.

{\sloppy
Mark the name of any constructor (with or without parameters) with
\cs{spadtype} like this
\begin{verbatim}
\spadtype{Polynomial(Integer)}
\end{verbatim}
Similarly, mark an
operator name with \cs{spadop},
a \Language{} operation (function) with \cs{spadfun}, and a
variable or \Language{} expression with
\cs{spad}.
Library documentation is given in a \TeX{}-like language so that
it can be used both for hard-copy and for \Browse{}.
These different wrappings cause operations and types to have
mouse-active buttons in \Browse{}.
For hard-copy output, wrapped expressions appear in a different font.
The above documentation appears in hard-copy as:

}
%
\begin{quotation}
%
\spadtype{SetCategory} is the basic category
for describing a collection of elements with \spadop{=}
(equality) and a \spadfun{coerce} to \spadtype{OutputForm}.
%
\end{quotation}
%
and
%
\begin{quotation}
%
\spad{x = y} tests if \spad{x} and \spad{y} are equal.
%
\end{quotation}
%

For our purposes in this chapter, we omit the documentation from further
category descriptions.

% *********************************************************************
\head{section}{Hierarchies}{ugCategoriesHier}
% *********************************************************************

A second example of a category is
\spadtype{SemiGroup}, defined by:
\exptypeindex{SemiGroup}

\begin{xmpLines}
SemiGroup(): Category == SetCategory with
      "*":  (%,%) -> %
      "^": (%, PositiveInteger) -> %
\end{xmpLines}

This definition is as simple as that for \spadtype{SetCategory},
except that there are two exported operations.
Multiple exported operations are written as a \spadgloss{pile},
that is, they all begin in the same column.
Here you see that the category mentions another type,
\spadtype{PositiveInteger}, in a signature.
Any domain can be used in a signature.

Since categories extend one another, they form hierarchies.
Each category other than \spadtype{Type} has one or more parents given
by the one or more categories mentioned before the \spad{with} part of
the definition.
\spadtype{SemiGroup} extends \spadtype{SetCategory} and
\spadtype{SetCategory} extends both \spadtype{Type} and
\spadtype{CoercibleTo (OutputForm)}.
Since \spadtype{CoercibleTo (OutputForm)} also extends \spadtype{Type},
the mention of \spadtype{Type} in the definition is unnecessary but
included for emphasis.

% *********************************************************************
\head{section}{Membership}{ugCategoriesMembership}
% *********************************************************************

We say a category designates a class of domains.
What class of domains?
\index{category!membership}
That is, how does \Language{} know what domains belong to what categories?
The simple answer to this basic question is key to the design of
\Language{}:

\beginImportant
\begin{center}
{\bf Domains belong to categories by assertion.}
\end{center}
\endImportant

When a domain is defined, it is asserted to belong to one or more
categories.
Suppose, for example, that an author of domain \spadtype{String} wishes to
use the binary operator \spadop{*} to denote concatenation.
Thus \spad{"hello " * "there"} would produce the string
\spad{"hello there"}\footnote{Actually, concatenation of strings in
\Language{} is done by juxtaposition or by using the operation
\spadfunFrom{concat}{String}.
The expression \spad{"hello " "there"} produces the string
\spad{"hello there"}.}.
The author of \spadtype{String} could then assert that \spadtype{String}
is a member of \spadtype{SemiGroup}.
According to our definition of \spadtype{SemiGroup}, strings
would then also have the operation \spadop{^} defined automatically.
Then \spad{"--" ^ 4} would produce a string of eight dashes
\spad{"--------"}.
Since \spadtype{String} is a member of \spadtype{SemiGroup}, it also is
a member of \spadtype{SetCategory} and thus has an operation
\spadop{=} for testing that two strings are equal.

  Now turn to the algebraic category hierarchy inside the front cover
  of this book.
Any domain that is a member of a
category extending \spadtype{SemiGroup} is a member of
\spadtype{SemiGroup} (that is, it {\it is} a semigroup).
In particular, any domain asserted to be a \spadtype{Ring} is a
semigroup since \spadtype{Ring} extends \spadtype{Monoid}, that,
in turn, extends \spadtype{SemiGroup}.
The definition of \spadtype{Integer} in \Language{} asserts that
\spadtype{Integer} is a member of category
\spadtype{IntegerNumberSystem}, that, in turn, asserts that it is
a member of \spadtype{EuclideanDomain}.
Now \spadtype{EuclideanDomain} extends
\spadtype{PrincipalIdealDomain} and so on.
If you trace up the hierarchy, you see that
\spadtype{EuclideanDomain} extends \spadtype{Ring}, and,
therefore, \spadtype{SemiGroup}.
Thus \spadtype{Integer} is a semigroup and also exports the
operations \spadop{*} and \spadop{^}.

% *********************************************************************
\head{section}{Defaults}{ugCategoriesDefaults}
% *********************************************************************

We actually omitted the last
\index{category!defaults}
part of the definition of
\index{default definitions}
\spadtype{SemiGroup} in
\spadref{ugCategoriesHier}.
Here now is its complete \Language{} definition.

\begin{xmpLines}
SemiGroup(): Category == SetCategory with
      "*": (%, %) -> %
      "^": (%, PositiveInteger) -> %
    add
      import RepeatedSquaring(%)
      x: % ^ n: PositiveInteger == expt(x,n)
\end{xmpLines}

The \spad{add} part at the end is used to give ``default definitions'' for
\spadkey{add}
exported operations.
Once you have a multiplication operation \spadop{*}, you can
define exponentiation
for positive integer exponents
using repeated multiplication:
\begin{displaymath}
x^n = {\underbrace{x \, x \, x \, \cdots \, x}_{\displaystyle n \hbox{\ times}}}
\end{displaymath}
This definition for \spadop{^} is called a {\it default} definition.
In general, a category can give default definitions for any
operation it exports.
Since \spadtype{SemiGroup} and all its category descendants in the hierarchy
export \spadop{^}, any descendant category may redefine \spadop{^} as well.

A domain of category \spadtype{SemiGroup}
(such as \spadtype{Integer}) may or may not choose to
define its own \spadop{^} operation.
If it does not, a default definition that is closest (in a ``tree-distance''
sense of the hierarchy) to the domain is chosen.

The part of the category definition following an \spadSyntax{add} operation
is a \spadgloss{capsule}, as discussed in
the previous chapter.
The line
\begin{verbatim}
import RepeatedSquaring(%)
\end{verbatim}
references the package
\spadtype{RepeatedSquaring(%)}, that is, the package
\spadtype{RepeatedSquaring} that takes ``this domain'' as its
parameter.
For example, if the semigroup \spadtype{Polynomial (Integer)}
does not define its own exponentiation operation, the
definition used may come from the package
\spadtype{RepeatedSquaring (Polynomial (Integer))}.
The next line gives the definition in terms of \spadfun{expt} from that
package.

The default definitions are collected to form a ``default
package'' for the category.
The name of the package is the same as  the category but with an
ampersand (\spadSyntax{&}) added at the end.
A default package always takes an additional argument relative to the
category.
Here is the definition of the default package \pspadtype{SemiGroup&} as
automatically generated by \Language{} from the above definition of
\spadtype{SemiGroup}.

\begin{xmpLines}
SemiGroup_&(%): Exports == Implementation where
  %: SemiGroup
  Exports == with
    "^": (%, PositiveInteger) -> %
  Implementation == add
    import RepeatedSquaring(%)
    x:% ^ n:PositiveInteger == expt(x,n)
\end{xmpLines}

% *********************************************************************
\head{section}{Axioms}{ugCategoriesAxioms}
% *********************************************************************

In the previous section you saw the
complete \Language{} program defining \index{axiom}
\spadtype{SemiGroup}.
According to this definition, semigroups (that is, are sets with
the operations \spadopFrom{*}{SemiGroup} and
\spadopFrom{^}{SemiGroup}.
\exptypeindex{SemiGroup}

You might ask: ``Aside from the notion of default packages, isn't
a category just a \spadgloss{macro}, that is, a shorthand
equivalent to the two operations \spadop{*} and \spadop{^} with
their types?'' If a category were a macro, every time you saw the
word \spadtype{SemiGroup}, you would rewrite it by its list of
exported operations.
Furthermore, every time you saw the exported operations of
\spadtype{SemiGroup} among the exports of a constructor, you could
conclude that the constructor exported \spadtype{SemiGroup}.

A category is {\it not} a macro and here is why.
The definition for \spadtype{SemiGroup} has documentation that states:

\begin{quotation}
    Category \spadtype{SemiGroup} denotes the class of all multiplicative
    semigroups, that is, a set with an associative operation \spadop{*}.

    \vskip .5\baselineskip
    {Axioms:}

    {\small\spad{associative("*" : (%,%)->%)} \quad\quad\quad \spad{(x*y)*z = x*(y*z)}}
\end{quotation}

According to the author's remarks, the mere
exporting of an operation named \spadop{*} and \spadop{^} is not
enough to qualify the domain as a \spadtype{SemiGroup}.
In fact, a domain can be a semigroup only if it explicitly
exports a \spadop{^} and
a \spadop{*} satisfying the associativity axiom.

In general, a category name implies a set of axioms, even mathematical
theorems.
There are numerous axioms from \spadtype{Ring}, for example,
that are well-understood from the literature.
No attempt is made to list them all.
Nonetheless, all such mathematical facts are implicit by the use of the
name \spadtype{Ring}.

% *********************************************************************
\head{section}{Correctness}{ugCategoriesCorrectness}
% *********************************************************************

While such statements are only comments,
\index{correctness}
\Language{} can enforce their intention simply by shifting the burden of
responsibility onto the author of a domain.
A domain belongs to category \spad{Ring} only if the
author asserts that the domain  belongs to \spadtype{Ring} or
to a category that extends \spadtype{Ring}.

This principle of assertion is important for large user-extendable
systems.
\Language{} has a large library of operations offering facilities in
many areas.
Names such as \spadfun{norm} and \spadfun{product}, for example, have
diverse meanings in diverse contexts.
An inescapable hindrance to users would be to force those who wish to
extend \Language{} to always invent new names for operations.
%>> I don't think disambiguate is really a word, though I like it
\Language{} allows you to reuse names, and then use context to disambiguate one
from another.

Here is another example of why this is important.
Some languages, such as {\bf APL},
\index{APL}
denote the \spadtype{Boolean} constants \spad{true} and
\spad{false} by the integers \spad{1} and \spad{0}.
You may want to let infix operators \spadop{+} and \spadop{*} serve as the logical
operators \spadfun{or} and \spadfun{and}, respectively.
But note this: \spadtype{Boolean} is not a ring.
The {\it inverse axiom} for \spadtype{Ring} states:
%
\begin{center}
Every element \spad{x} has an additive inverse \spad{y} such that
\spad{x + y = 0}.
\end{center}
%
\spadtype{Boolean} is not a ring since \spad{true} has
no inverse---there is no inverse element \spad{a} such that
\spad{1 + a = 0} (in terms of booleans, \spad{(true or a) = false}).
Nonetheless, \Language{} {\it could} easily and correctly implement
\spadtype{Boolean} this way.
\spadtype{Boolean} simply would not assert that it is of category
\spadtype{Ring}.
Thus the \spadop{+} for \spadtype{Boolean} values
is not confused with the one for \spadtype{Ring}.
Since the \spadtype{Polynomial} constructor requires its argument
to be a ring, \Language{} would then refuse to build the
domain \spadtype{Polynomial(Boolean)}. Also, \Language{} would refuse to
wrongfully apply algorithms to \spadtype{Boolean} elements that  presume that the
ring axioms for \spadop{+} hold.

% *********************************************************************
\head{section}{Attributes}{ugCategoriesAttributes}
% *********************************************************************

Most axioms are not computationally useful.
Those that are can be explicitly expressed by what \Language{} calls an
\spadgloss{attribute}.
The attribute \spadtype{CommutativeStar}, for example, is used to assert
that a domain has commutative multiplication.
Its definition is given by its documentation:

\begingroup \parindent=1pc \narrower\noindent%
    A domain \spad{R} has \spadtype{CommutativeStar}
    if it has an operation "*": \spad{(R,R)->R} such that \spad{x * y = y * x}.
\par\endgroup

Just as you can test whether a domain has the category \spadtype{Ring}, you
can test that a domain has a given attribute.

\begin{xtc}
\begin{xtccomment}
Do polynomials over the integers
have commutative multiplication?
\end{xtccomment}
\begin{spadsrc}
Polynomial Integer has CommutativeStar
\end{spadsrc}
\begin{TeXOutput}
\begin{fricasmath}{1}
\STRING{true}%
\end{fricasmath}
\end{TeXOutput}
\formatResultType{Boolean}
\end{xtc}
\begin{xtc}
\begin{xtccomment}
Do matrices over the integers
have commutative multiplication?
\end{xtccomment}
\begin{spadsrc}
Matrix Integer has CommutativeStar
\end{spadsrc}
\begin{TeXOutput}
\begin{fricasmath}{2}
\STRING{false}%
\end{fricasmath}
\end{TeXOutput}
\formatResultType{Boolean}
\end{xtc}

Attributes are used to conditionally export and define
operations for a domain (see \spadref{ugDomainsAssertions}).
Attributes can also be asserted in a category definition.

After mentioning category \spadtype{Ring} many times in this book,
it is high time that we show you its definition:
\exptypeindex{Ring}

\begin{xmpLines}
Ring(): Category ==
  Join(Rng,Monoid,LeftModule(%: Rng)) with
      characteristic: -> NonNegativeInteger
      coerce: Integer -> %
      unitsKnown
    add
      n:Integer
      coerce(n) == n * 1$%
\end{xmpLines}

There are only two new things here.
First, look at the \spadSyntax{$%} on the last line.
This is not a typographic error!
The \spadSyntax{$} says that the \spad{1} is to come from some
domain.
The \spadSyntax is \spadtype{Fraction(Integer)}, this line reads
\spad{coerce(n) == n * 1$Fraction(Integer)}.

The second new thing is the presence of attribute ``\spad{unitsKnown}''.
\Language{} can always distinguish an attribute from an operation.
An operation has a name and a type. An attribute has no type.
The attribute \spadtype{unitsKnown} asserts a rather subtle mathematical
fact that is normally taken for granted when working with
rings.\footnote{With this axiom, the units of a domain are the set of
elements \spad{x} that each have a multiplicative
inverse \spad{y} in the domain.
Thus \spad{1} and \spad{-1} are units in domain \spadtype{Integer}.
Also, for \spadtype{Fraction Integer}, the domain of rational numbers,
all non-zero elements are units.}
Because programs can test for this attribute, \Language{} can
correctly handle rather more complicated mathematical structures (ones
that are similar to rings but do not have this attribute).

% *********************************************************************
\head{section}{Parameters}{ugCategoriesParameters}
% *********************************************************************

Like domain constructors, category constructors can also have
parameters.
For example, category \spadtype{MatrixCategory} is a parameterized
category for defining matrices over a ring \spad{R} so that the
matrix domains can have
different representations and indexing schemes.
Its definition has the form:

\begin{xmpLines}
MatrixCategory(R,Row,Col): Category ==
    TwoDimensionalArrayCategory(R,Row,Col) with ...
\end{xmpLines}

The category extends \spadtype{TwoDimensionalArrayCategory} with
the same arguments.
You cannot find \spadtype{TwoDimensionalArrayCategory} in the
algebraic hierarchy listing.
Rather, it is a member of the data structure hierarchy,
given inside the back cover of this book.
In particular, \spadtype{TwoDimensionalArrayCategory} is an extension of
\spadtype{HomogeneousAggregate} since its elements are all one type.

The domain \spadtype{Matrix(R)}, the class of matrices with coefficients
from domain \spad{R}, asserts that it is a member of category
\spadtype{MatrixCategory(R, Vector(R), Vector(R))}.
The parameters of a category must also have types.
The first parameter to \spadtype{MatrixCategory}
\spad{R} is required to be a ring.
The second and third are required to be domains of category
\spadtype{FiniteLinearAggregate(R)}.\footnote{%
This is another extension of
\spadtype{HomogeneousAggregate} that you can see in
the data structure hierarchy.}
In practice, examples of categories having parameters other than
domains are rare.

Adding the declarations for parameters to the definition for
\spadtype{MatrixCategory}, we have:

\begin{xmpLines}
R: Ring
(Row, Col): FiniteLinearAggregate(R)

MatrixCategory(R, Row, Col): Category ==
    TwoDimensionalArrayCategory(R, Row, Col) with ...
\end{xmpLines}

% *********************************************************************
\head{section}{Conditionals}{ugCategoriesConditionals}
% *********************************************************************

As categories have parameters, the actual operations exported by a
\index{conditional}
category can depend on these parameters.
As an example, the operation \spadfunFrom{determinant}{MatrixCategory}
from category \spadtype{MatrixCategory} is only exported when the
underlying domain \spad{R} has commutative multiplication:

\begin{verbatim}
if R has CommutativeRing then
   determinant: % -> R
\end{verbatim}

Conditionals can also define conditional extensions of a category.
Here is a portion of the definition of \spadtype{QuotientFieldCategory}:
\exptypeindex{QuotientFieldCategory}

\begin{xmpLines}
QuotientFieldCategory(R) : Category == ... with ...
     if R has OrderedSet then OrderedSet
     if R has IntegerNumberSystem then
       ceiling: % -> R
         ...
\end{xmpLines}

Think of category \spadtype{QuotientFieldCategory(R)} as
denoting the domain \spadtype{Fraction(R)}, the
class of all fractions of the form \smath{a/b} for elements of \spad{R}.
The first conditional means in English:
``If the elements of \spad{R} are totally ordered (\spad{R}
is an \spadtype{OrderedSet}), then so are the fractions \smath{a/b}''.
\exptypeindex{Fraction}

The second conditional is used to conditionally export an
operation \spadfun{ceiling} which returns the smallest integer
greater than or equal to its argument.
Clearly, ``ceiling'' makes sense for integers but not for
polynomials and other algebraic structures.
Because of this conditional,
the domain \spadtype{Fraction(Integer)} exports
an operation
\spadfun{ceiling}: \spad{Fraction Integer->Integer}, but
\spadtype{Fraction Polynomial Integer} does not.

Conditionals can also appear in the default definitions for the
operations of a category.
For example, a default definition for \spadfunFrom{ceiling}{Field}
within the part following the \spadSyntax{add} reads:

\begin{verbatim}
if R has IntegerNumberSystem then
    ceiling x == ...
\end{verbatim}

Here the predicate used is identical to the predicate
in the {\tt Exports} part.
This need not be the case.
See \spadref{ugPackagesConds} for a more complicated example.

% *********************************************************************
\head{section}{Anonymous Categories}{ugCategoriesAndPackages}
% *********************************************************************

The part of a category to the right of a {\tt with} is also
regarded as a category---an ``anonymous category.''
Thus you have already seen a   category definition
\index{category!anonymous}  in \chapref{ugPackages}.
The {\tt Exports} part of the package \pspadtype{DrawComplex}
(\spadref{ugPackagesAbstract}) is an anonymous category.
This is not necessary.
We could, instead, give this category a name:

%
\begin{xmpLines}
DrawComplexCategory(): Category == with
   drawComplex: (C -> C,S,S,Boolean) -> VIEW3D
   drawComplexVectorField: (C -> C,S,S) -> VIEW3D
   setRealSteps: INT -> INT
   setImagSteps: INT -> INT
   setClipValue: DFLOAT-> DFLOAT
\end{xmpLines}
%
and then define \spadtype{DrawComplex} by:
%
\begin{xmpLines}
DrawComplex(): DrawComplexCategory == Implementation
   where
      ...
\end{xmpLines}
%

There is no reason, however, to give this list of exports a name
since no other domain or package exports it.
In fact, it is rare for a package to export a named category.
As you will see in the next chapter, however, it is very common
for the definition of domains to mention one or more category
before the {\tt with}.
\spadkey{with}

% !! DO NOT MODIFY THIS FILE BY HAND !! Created by spool2tex.awk.

% Copyright (c) 1991-2002, The Numerical ALgorithms Group Ltd.
% All rights reserved.
%
% Redistribution and use in source and binary forms, with or without
% modification, are permitted provided that the following conditions are
% met:
%
%     - Redistributions of source code must retain the above copyright
%       notice, this list of conditions and the following disclaimer.
%
%     - Redistributions in binary form must reproduce the above copyright
%       notice, this list of conditions and the following disclaimer in
%       the documentation and/or other materials provided with the
%       distribution.
%
%     - Neither the name of The Numerical ALgorithms Group Ltd. nor the
%       names of its contributors may be used to endorse or promote products
%       derived from this software without specific prior written permission.
%
% THIS SOFTWARE IS PROVIDED BY THE COPYRIGHT HOLDERS AND CONTRIBUTORS "AS
% IS" AND ANY EXPRESS OR IMPLIED WARRANTIES, INCLUDING, BUT NOT LIMITED
% TO, THE IMPLIED WARRANTIES OF MERCHANTABILITY AND FITNESS FOR A
% PARTICULAR PURPOSE ARE DISCLAIMED. IN NO EVENT SHALL THE COPYRIGHT OWNER
% OR CONTRIBUTORS BE LIABLE FOR ANY DIRECT, INDIRECT, INCIDENTAL, SPECIAL,
% EXEMPLARY, OR CONSEQUENTIAL DAMAGES (INCLUDING, BUT NOT LIMITED TO,
% PROCUREMENT OF SUBSTITUTE GOODS OR SERVICES-- LOSS OF USE, DATA, OR
% PROFITS-- OR BUSINESS INTERRUPTION) HOWEVER CAUSED AND ON ANY THEORY OF
% LIABILITY, WHETHER IN CONTRACT, STRICT LIABILITY, OR TORT (INCLUDING
% NEGLIGENCE OR OTHERWISE) ARISING IN ANY WAY OUT OF THE USE OF THIS
% SOFTWARE, EVEN IF ADVISED OF THE POSSIBILITY OF SUCH DAMAGE.

% *********************************************************************
\head{chapter}{Domains}{ugDomains}
% *********************************************************************

We finally come to the \spadgloss{domain constructor}.
A few subtle differences between packages and
domains turn up some interesting issues.
We first discuss these differences then
describe the resulting issues by illustrating a program
for the \spadtype{QuadraticForm} constructor.
After a short example of an algebraic constructor,
\spadtype{CliffordAlgebra}, we show how you use domain constructors to build
a database query facility.

% *********************************************************************
\head{section}{Domains vs. Packages}{ugPackagesDoms}
% *********************************************************************
%
Packages are special cases of domains.
What is the difference between a package and a domain that is not a
package?  Internally, \Language{} makes no distinction.  However,
humans think differently about them, so we make the
following definition: a domain that is not a package
has the symbol \spadSyntax
denotes ``this domain.'' If the \spadSyntax, then
evidently there is nothing of interest to do with the objects of the
domain.
You might then say that a package is a ``boring'' domain!
But, as you saw in \chapref{ugPackages}, packages are a very useful
notion indeed.
The exported operations of a package depend solely on the parameters
to the package constructor and other explicit domains.

To summarize, domain constructors are versatile structures that serve two
distinct practical purposes:
Those like \spadtype{Polynomial} and \spadtype{List}
describe classes of computational objects;
others, like \pspadtype{SortPackage}, describe packages of useful
operations.
As in the last chapter, we focus here on the first kind.

% *********************************************************************
\head{section}{Definitions}{ugDomainsDefs}
% *********************************************************************
%

The syntax for defining a domain constructor is the same as for any
function in \Language{}:
\begin{center}
\frenchspacing{\tt {\it DomainForm} : {\it Exports} == {\it Implementation}}
\end{center}
As this definition usually extends over many lines, a
\spad{where} expression is generally used instead.
\spadkey{where}

% ----------------------------------------------------------------------
\beginImportant
A recommended format for the definition of a domain is:\newline
{\tt%
{\it DomainForm} : Exports  ==  Implementation where \newline
\hspace*{.75pc} {\it optional type declarations} \newline
\hspace*{.75pc} Exports  ==  [{\it Category Assertions}] with \newline
\hspace*{2.0pc}   {\it list of exported operations} \newline
\hspace*{.75pc} Implementation  ==  [{\it Add Domain}] add \newline
\hspace*{2.0pc}   [Rep := {\it Representation}] \newline
\hspace*{2.0pc}   {\it list of function definitions for exported operations}
}

Note: The brackets {\tt [ ]} here denote optionality.
\endImportant
% ----------------------------------------------------------------------

A complete domain constructor definition for
\spadtype{QuadraticForm} is shown in Figure \ref{fig-quadform}.
Interestingly, this little domain illustrates all the new
concepts you need to learn.

\begin{figXmpLines}[caption={The \textspadtype{QuadraticForm} domain.},label={fig-quadform}]
)abbrev domain QFORM QuadraticForm

++ Description:
++   This domain provides modest support for
++   quadratic forms.
QuadraticForm(n, K): Exports == Implementation where
    n: PositiveInteger
    K: Field

    Exports == AbelianGroup with                         -- The exports.
      quadraticForm: SquareMatrix(n,K) -> %              -- The export \spadfun{quadraticForm}.
        ++ \spad{quadraticForm(m)} creates a quadratic
        ++ quadratic form from a symmetric,
        ++ square matrix \spad{m}.
      matrix: % -> SquareMatrix(n,K)                     -- The export \spadfun{matrix}.
        ++ \spad{matrix(qf)} creates a square matrix
        ++ from the quadratic form \spad{qf}.
      elt: (%, DirectProduct(n,K)) -> K                  -- The export \spadfun{elt}.
        ++ \spad{qf(v)} evaluates the quadratic form
        ++ \spad{qf} on the vector \spad{v},
        ++ producing a scalar.

    Implementation == SquareMatrix(n,K) add              -- The definitions of the exports
      Rep := SquareMatrix(n,K)                           -- The ``representation.''
      quadraticForm m ==                                 -- The definition of
        not symmetric? m => error                        -- \spadfun{quadraticForm}.
          "quadraticForm requires a symmetric matrix"
        m :: %
      matrix q == q :: Rep                               -- The definition of \spadfun{matrix}.
      elt(q,v) == dot(v, (matrix q * v))                 -- The definition of \spadfun{elt}.
\end{figXmpLines}

A domain constructor can take any number and type of parameters.
\spadtype{QuadraticForm} takes a positive integer \spad{n} and a field
\spad{K} as arguments.
Like a package, a domain has a set of explicit exports and an
implementation described by a capsule.
Domain constructors are documented in the same way as package constructors.

Domain \spadtype{QuadraticForm(n, K)}, for a given positive integer
\spad{n} and domain \spad{K}, explicitly exports three operations:
%
\begin{itemize}
\item\spad{quadraticForm(A)} creates a quadratic form from a matrix
\spad{A}.
\item\spad{matrix(q)} returns the matrix \spad{A} used to create
the quadratic form \spad{q}.
\item\spad{q.v} computes the scalar $v^TAv$
for a given vector \spad{v}.
\end{itemize}

Compared with the corresponding syntax given for the definition of a
package, you see that a domain constructor has three optional parts to
its definition: {\it Category Assertions}, {\it Add Domain}, and
{\it Representation}.

% *********************************************************************
\head{section}{Category Assertions}{ugDomainsAssertions}
% *********************************************************************
%

The {\it Category Assertions} part of your domain constructor
definition lists those categories of which all domains created by
the constructor are unconditionally members.
The word ``unconditionally'' means that membership in a category
does not depend on the values of the parameters to the domain
constructor.
This part thus defines the link between the domains and the
category hierarchies given on the inside covers of this book.
As described in \spadref{ugCategoriesCorrectness}, it is this link
that makes it possible for you to pass objects of the domains as
arguments to other operations in \Language{}.

Every \spadtype{QuadraticForm} domain is declared
to be unconditionally a member of category \spadtype{AbelianGroup}.
An abelian group is a collection of elements closed under
addition.
Every object {\it x} of an abelian group has an additive inverse
{\it y} such that $x + y = 0$.
The exports of an abelian group include \spad{0},
\spadop{+}, \spadop{-}, and scalar multiplication by an integer.
After asserting that \spadtype{QuadraticForm} domains are abelian
groups, it is possible to pass quadratic forms to algorithms that
only assume arguments to have these abelian group
properties.

In \spadref{ugCategoriesConditionals}, you saw that
\spadtype{Fraction(R)}, a member of
\spadtype{QuotientFieldCategory(R)},
is a member of \spadtype{OrderedSet} if \spad{R}
is a member of \spadtype{OrderedSet}.
Likewise, from the {\tt Exports} part of the definition of
\spadtype{ModMonic(R, S)},
\begin{verbatim}
UnivariatePolynomialCategory(R) with
  if R has Finite then Finite
     ...
\end{verbatim}
you see that \spadtype{ModMonic(R, S)} is a member of
\spadtype{Finite} is \spad{R} is.

The {\tt Exports} part of a domain definition is
the same kind of
expression that can appear to the right of an
\spadSyntax{==} in a category definition.
If a domain constructor is unconditionally a member of two or more
categories, a \spad{Join} form is used.
\spadkey{Join}
The {\tt Exports} part of the definition of
\spadtype{FlexibleArray(S)} reads, for example:
\begin{verbatim}
Join(ExtensibleLinearAggregate(S),
     OneDimensionalArrayAggregate(S)) with...
\end{verbatim}

% *********************************************************************
\head{section}{A Demo}{ugDomainsDemo}
% *********************************************************************
%
Before looking at the {\it Implementation} part of \spadtype{QuadraticForm},
let's try some examples.

\vskip 2pc
\begin{xtc}
\begin{xtccomment}
Build a domain \spad{QF}.
\end{xtccomment}
\begin{spadsrc}
QF := QuadraticForm(2,Fraction Integer)
\end{spadsrc}
\begin{TeXOutput}
\begin{fricasmath}{1}
\STRING{QuadraticForm(2,Fraction(Integer))}%
\end{fricasmath}
\end{TeXOutput}
\formatResultType{Type}
\end{xtc}
\begin{xtc}
\begin{xtccomment}
Define a matrix to be used to construct
a quadratic form.
\end{xtccomment}
\begin{spadsrc}
A := matrix [[-1,1/2],[1/2,1]]
\end{spadsrc}
\begin{TeXOutput}
\begin{fricasmath}{2}
\begin{MATRIX}{2}-{1}&\frac{1}{2}\\\frac{1}{2}&1\end{MATRIX}%
\end{fricasmath}
\end{TeXOutput}
\formatResultType{Matrix(Fraction(Integer))}
\end{xtc}
\begin{xtc}
\begin{xtccomment}
Construct the quadratic form.
A package call \spad{$QF} is necessary since there
are other \spadtype{QuadraticForm} domains.
\end{xtccomment}
\begin{spadsrc}
q : QF := quadraticForm(A)
\end{spadsrc}
\begin{TeXOutput}
\begin{fricasmath}{3}
\begin{MATRIX}{2}-{1}&\frac{1}{2}\\\frac{1}{2}&1\end{MATRIX}%
\end{fricasmath}
\end{TeXOutput}
\formatResultType{QuadraticForm(2, Fraction(Integer))}
\end{xtc}
\begin{xtc}
\begin{xtccomment}
Looks like a matrix. Try computing
the number of rows.
\Language{} won't let you.
\end{xtccomment}
\begin{spadsrc}
nrows q
\end{spadsrc}
\begin{MessageOutput}
   There are 2 exposed and 1 unexposed library operations named nrows 
      having 1 argument(s) but none was determined to be applicable. 
      Use HyperDoc Browse, or issue
                              )display op nrows
      to learn more about the available operations. Perhaps 
      package-calling the operation or using coercions on the arguments
      will allow you to apply the operation.
\end{MessageOutput}
\begin{MessageOutput}
   Cannot find a definition or applicable library operation named nrows
      with argument type(s) 
                     QuadraticForm(2,Fraction(Integer))
      
      Perhaps you should use "@" to indicate the required return type, 
      or "$" to specify which version of the function you need.
\end{MessageOutput}
\end{xtc}
\begin{xtc}
\begin{xtccomment}
Create a direct product element \spad{v}.
A package call is again necessary, but \Language{}
understands your list as denoting a vector.
\end{xtccomment}
\begin{spadsrc}
v := directProduct([2,-1])$DirectProduct(2,Fraction Integer)
\end{spadsrc}
\begin{TeXOutput}
\begin{fricasmath}{4}
\BRACKET{2\COMMA -{1}}%
\end{fricasmath}
\end{TeXOutput}
\formatResultType{DirectProduct(2, Fraction(Integer))}
\end{xtc}
\begin{xtc}
\begin{xtccomment}
Compute the product $v^TAv$.
\end{xtccomment}
\begin{spadsrc}
q.v
\end{spadsrc}
\begin{TeXOutput}
\begin{fricasmath}{5}
-{5}%
\end{fricasmath}
\end{TeXOutput}
\formatResultType{Fraction(Integer)}
\end{xtc}
\begin{xtc}
\begin{xtccomment}
What is 3 times \spad{q} minus \spad{q} plus \spad{q}?
\end{xtccomment}
\begin{spadsrc}
3*q-q+q
\end{spadsrc}
\begin{TeXOutput}
\begin{fricasmath}{6}
\begin{MATRIX}{2}-{3}&\frac{3}{2}\\\frac{3}{2}&3\end{MATRIX}%
\end{fricasmath}
\end{TeXOutput}
\formatResultType{QuadraticForm(2, Fraction(Integer))}
\end{xtc}

% *********************************************************************
\head{section}{Browse}{ugDomainsBrowse}
% *********************************************************************

The \Browse{} facility of \HyperName{} is useful for
investigating
the properties of domains, packages, and categories.
From the main \HyperName{} menu, move your mouse to {\bf Browse} and
click on the left mouse button.
This brings up the \Browse{} first page.
Now, with your mouse pointer somewhere in this window, enter the
string ``quadraticform'' into the input area (all lower case
letters will do).
Move your mouse to {\bf Constructors} and click.
Up comes a page describing \spadtype{QuadraticForm}
that includes a part labeled by ``{\it
Description:}''.
You also see the types for arguments \spad{n} and \spad{K}
displayed as well as the fact that \spadtype{QuadraticForm}
returns an \spadtype{AbelianGroup}.

Select {\bf Operations} to get a list of operations for
\spadtype{QuadraticForm}.
You can select an operation by clicking on it
to get an individual page with information about that operation.
Or you can select the buttons along the bottom to see alternative
views or get additional information on the operations.
Eventually, use
\UpButton{}
to return to the first page on
\spadtype{QuadraticForm}.

You can go and experiment a bit by selecting {\bf Field} and {\bf n}
with your mouse.  Going back to {\bf Operations} you will see
that {\bf Implementations} view now works (it is disabled if
some domain parameter is unspecified).
Then return to the page on \spadtype{QuadraticForm}.

At the bottom the \spadtype{QuadraticForm} page has buttons
for {\bf Parents}, {\bf Ancestors}, and others.
Clicking on {\bf Parents}, you see that \spadtype{QuadraticForm}
has \spadtype{AbelianGroup} and \spadtype{ConvertibleTo} as
parents (note that \spadtype{QuadraticForm} distributed
with \Language{} is richer then the demo version presented
before).

% *********************************************************************
\head{section}{Representation}{ugDomainsRep}
% *********************************************************************
%
The {\tt Implementation} part of an \Language{} capsule for a
domain constructor uses the special variable \spad{Rep} to
\index{Rep @ {\tt Rep}}
identify the lower level data type used to represent the objects
\index{representation!of a domain}
of the domain.
\index{domain!representation}
The \spad{Rep} for quadratic forms is \spadtype{SquareMatrix(n, K)}.
This means that all objects of the domain are required to be
\spad{n} by \spad{n} matrices with elements from \spadtype{K}.

The code for \spadfun{quadraticForm} in Figure \ref{fig-quadform}
on page \pageref{fig-quadform}
checks that the matrix is symmetric and then converts it to
\spadSyntax on line 28 coerces \spad{m} to a
quadratic form.
In fact, the quadratic form you created in step (3) of
\spadref{ugDomainsDemo} is just the matrix you passed it in
disguise!
Without seeing this definition, you would not know that.
Nor can you take advantage of this fact now that you do know!
When we try in the next step of \spadref{ugDomainsDemo} to regard
\spad{q} as a matrix by asking for \spadfun{nrows}, the number of
its rows, \Language{} gives you an error message saying, in
effect, ``Good try, but this won't work!''

The definition for the \spadfunFrom{matrix}{QuadraticForm}
function could hardly be simpler:
it just returns its argument after explicitly
\spadglossSee{coercing}{coerce} its argument to a matrix.
Since the argument is already a matrix, this coercion does no computation.

Within the context of a capsule, an object of \spadSyntax and \spad{Rep}
have the same named operation available,
the one from \textspadsyntax{\%} takes precedence.
Thus, if you want the one from \textspadsyntax{Rep}, you must
package call it using a \textspadsyntax{\$Rep} suffix.}
This makes the definition of \spad{q.v} easy---it
just calls the \spadfunFrom{dot}{DirectProduct} product from
\spadtype{DirectProduct} to perform the indicated operation.

% *********************************************************************
\head{section}{Multiple Representations}{ugDomainsMultipleReps}
% *********************************************************************
%

To write functions that implement the operations of a domain, you
want to choose the most computationally efficient
data structure to represent the elements of your domain.

A classic problem in computer algebra is the optimal choice for an
internal representation of polynomials.
If you create a polynomial, say \mathOrSpad{3x^2+ 5}, how
does \Language{} hold this value internally?
There are many ways.
\Language{} has nearly a dozen different representations of
polynomials, one to suit almost any purpose.
Algorithms for solving polynomial equations work most
efficiently with polynomials represented one way, whereas those for
factoring polynomials are most efficient using another.
One often-used representation is a list of terms, each term
consisting of exponent-coefficient records written in the order
of decreasing exponents.
For example, the polynomial \mathOrSpad{3x^2+5} is
%>> I changed the k's in next line to e's as I thought that was
%>> clearer.
represented by the list \spad{[[e:2, c:3], [e:0, c:5]]}.

What is the optimal data structure for a matrix?
It depends on the application.
For large sparse matrices, a linked-list structure of records
holding only the non-zero elements may be optimal.
If the elements can be defined by a simple formula
\mathOrSpad{f(i,j)}, then a compiled function for
\spad{f} may be optimal.
Some programmers prefer to represent ordinary matrices as vectors
of vectors.
Others prefer to represent matrices by one big linear array where
elements are accessed with linearly computable indexes.

While all these simultaneous structures tend to be confusing,
\Language{} provides a helpful organizational tool for such a purpose:
categories.
\spadtype{PolynomialCategory}, for example, provides a uniform user
interface across all polynomial types.
Each kind of polynomial implements functions for
all these operations, each in its own way.
If you use only the top-level operations in
\spadtype{PolynomialCategory} you usually do not care what kind
of polynomial implementation is used.

%>> I've often thought, though, that it would be nice to be
%>> be able to use conditionals for representations.
Within a given domain, however, you define (at most) one
representation.\footnote{You can make that representation a
\pspadtype{Union} type, however.
See \spadref{ugTypesUnions} for examples of unions.}
If you want to have multiple representations (that is, several
domains, each with its own representation), use a category to
describe the {\tt Exports}, then define separate domains for each
representation.

% *********************************************************************
\head{section}{Add Domain}{ugDomainsAddDomain}
% *********************************************************************
%

The capsule part of {\tt Implementation} defines functions that
implement the operations exported by the domain---usually only
some of the operations.
In our demo in \spadref{ugDomainsDemo}, we asked for the value of
\spad{3*q-q+q}.
Where do the operations \spadop{*}, \spadop{+}, and
\spadop{-} come from?
There is no definition for them in the capsule!

The {\tt Implementation} part of a definition can
\index{domain!add}
optionally specify an ``add-domain'' to the left of an {\tt add}
\spadkey{add}
(for \spadtype{QuadraticForm}, defines
\spadtype{SquareMatrix(n,K)} is the add-domain).
The meaning of an add-domain is simply this: if the capsule part
of the {\tt Implementation} does not supply a function for an
operation, \Language{} goes to the add-domain to find the
function.
So do \spadopFrom{*}{QuadraticForm}, \spadopFrom{+}{QuadraticForm}
and \spadopFrom{-}{QuadraticForm} come from
\spadtype{SquareMatrix(n,K)}?
%Read on!

%*********************************************************************
\head{section}{Defaults}{ugDomainsDefaults}
% *********************************************************************
%
In \chapref{ugPackages}, we saw that categories can provide
default implementations for their operations.
How and when are they used?
When \Language{} finds that
\spadtype{QuadraticForm(2, Fraction Integer)}
does not implement the operations \spadop{*},
\spadop{+}, and \spadop{-}, it goes to
\spadtype{SquareMatrix(2,Fraction Integer)} to find it.
As it turns out, \spadtype{SquareMatrix(2, Fraction Integer)} does
not implement {\it any} of these operations!

What does \Language{} do then?
Here is its overall strategy.
First, \Language{} looks for a function in the capsule for the domain.
If it is not there, \Language{} looks in the add-domain for the
operation.
If that fails, \Language{} searches the add-domain of the add-domain,
and so on.
If all those fail, it then searches the default packages for the
categories of which the domain is a member.
In the case of \spadtype{QuadraticForm}, it searches
\spadtype{AbelianGroup}, then its parents, grandparents, and
so on.
If this fails, it then searches the default packages of the
add-domain.
Whenever a function is found, the search stops immediately and the
function is returned.
When all fails, the system calls \spadfun{error} to report this
unfortunate news to you.
To find out the actual order of constructors searched for
\spadtype{QuadraticForm}, consult \Browse{}: from the
\spadtype{QuadraticForm}, and click on {\bf Search Path}.

Let's apply this search strategy for our example \spad{3*q-q+q}.
The scalar multiplication comes first.
\Language{} finds a default implementation in
\spadtype{AbelianGroup&}.
Remember from \spadref{ugCategoriesDefaults} that
\spadtype{SemiGroup} provides a default definition for
\mathOrSpad{x^n} by repeated squaring?
\spadtype{AbelianGroup} similarly provides a definition for
\mathOrSpad{n*x} by repeated doubling.

But the search of the defaults for \spadtype{QuadraticForm} fails
to find any \spadop{+} or \spadop{*} in the default packages for
the ancestors of \spadtype{QuadraticForm}.
So it now searches among those for \spadtype{SquareMatrix}.
Category \spadtype{MatrixCategory}, which provides a uniform interface
for all matrix domains,
is a grandparent of \spadtype{SquareMatrix} and
has a capsule defining many functions for matrices, including
matrix addition, subtraction, and scalar multiplication.
The default package \spadtype{MatrixCategory&} is where the
functions for \spadopFrom{+}{QuadraticForm} and
\spadopFrom{-}{QuadraticForm} come from.

You can use \Browse{} to discover where the operations for
\spadtype{QuadraticForm} are implemented.
First, get the page describing \spadtype{QuadraticForm}.
With your mouse somewhere in this window, type a ``2'', press the
\fbox{\bf Tab} key, and then enter ``Fraction
Integer'' to indicate that you want the domain
\spadtype{QuadraticForm(2, Fraction Integer)}.
Now click on {\bf Operations} to get a table of operations and on
\spadop{*} to get a page describing the \spadop{*} operation.
Finally, click on {\bf implementation} at the bottom.

% *********************************************************************
\head{section}{Origins}{ugDomainsOrigins}
% *********************************************************************
%

Aside from the notion of where an operation is implemented,
\index{operation!origin}
a useful notion is  the {\it origin} or ``home'' of an operation.
When an operation (such as
\spadfunFrom{quadraticForm}{QuadraticForm}) is explicitly exported by
a domain (such as \spadtype{QuadraticForm}), you can say that the
origin of that operation is that domain.
If an operation is not explicitly exported from a domain, it is inherited
from, and has as origin, the (closest) category that explicitly exports it.
The operations \spadopFrom{+}{AbelianMonoid} and
\spadopFrom{-}{AbelianMonoid} of \spadtype{QuadraticForm},
for example, are inherited from \spadtype{AbelianMonoid}.
As it turns out, \spadtype{AbelianMonoid} is the origin of virtually every
\spadop{+} operation in \Language{}!

Again, you can use \Browse{} to discover the origins of
operations.
From the \Browse{} page on \spadtype{QuadraticForm}, click on {\bf
Operations}, then on {\bf origins} at the bottom of the page.

The origin of the operation is the {\it only} place where on-line
documentation is given.
However, you can re-export an operation to give it special
documentation.
Suppose you have just invented the world's fastest algorithm for
inverting matrices using a particular internal representation for
matrices.
If your matrix domain just declares that it exports
\spadtype{MatrixCategory}, it exports the \spadfun{inverse}
operation, but the documentation the user gets from \Browse{} is
the standard one from \spadtype{MatrixCategory}.
To give your version of \spadfun{inverse} the attention it
deserves, simply export the operation explicitly with new
documentation.
This redundancy gives \spadfun{inverse} a new origin and tells
\Browse{} to present your new documentation.

% *********************************************************************
\head{section}{Short Forms}{ugDomainsShortForms}
% *********************************************************************
%
In \Language{}, a domain could be defined using only an add-domain
and no capsule.
Although we talk about rational numbers as quotients of integers,
there is no type \pspadtype{RationalNumber} in \Language{}.
To create such a type, you could compile the following
``short-form'' definition:

\begin{xmpLines}
RationalNumber() == Fraction(Integer)
\end{xmpLines}

The {\tt Exports} part of this definition is missing and is taken
to be equivalent to that of \spadtype{Fraction(Integer)}.
Because of the add-domain philosophy, you get precisely
what you want.
The effect is to create a little stub of a domain.
When a user asks to add two rational numbers, \Language{} would
ask \pspadtype{RationalNumber} for a function implementing this
\spadop{+}.
Since the domain has no capsule, the domain then immediately
sends its request to \spadtype{Fraction (Integer)}.

The short form definition for domains is used to
define such domains as \spadtype{MultivariatePolynomial}:
\exptypeindex{MultivariatePolynomial}

\begin{xmpLines}
MultivariatePolynomial(vl: List Symbol, R: Ring) ==
   SparseMultivariatePolynomial(R,
      OrderedVariableList vl)
\end{xmpLines}

%% *********************************************************************
\head{section}{Example 1: Clifford Algebra}{ugDomainsClifford}
% *********************************************************************
%

Now that we have \spadtype{QuadraticForm} available,
let's put it to use.
Given some quadratic form \smath{Q} described by an
\smath{n} by \smath{n} matrix over a field \smath{K}, the domain
\spadtype{CliffordAlgebra(n, K, Q)} defines a vector space of
dimension \mathOrSpad{2^n} over \smath{K}.
This is an interesting domain since complex numbers, quaternions,
exterior algebras and spin algebras are all examples of Clifford
algebras.

The basic idea is this:
the quadratic form \spad{Q} defines a basis
$e_1,e_2\ldots,e_n$ for the
vector space \mathOrSpad{K^n}---the direct product of \spad{K}
with itself \spad{n} times.
From this, the Clifford algebra generates a basis of
\mathOrSpad{2^n} elements given by all the possible products
of the $e_i$ in order without duplicates, that is,
1,
$e_1$,
$e_2$,
$e_1e_2$,
$e_3$,
$e_1e_3$,
$e_2e_3$,
$e_1e_2,e_3$,
and so on.

The algebra is defined by the relations
$$
\begin{array}{lclc}
e_i \  e_i & = & Q(e_i) \\
e_i \  e_j & = & -e_j \  e_i & \hbox{for } i \neq j
\end{array}
$$

Now look at the snapshot of its definition given in Figure
\ref{fig-clifalg}.
Lines 9-10 show part of the definitions of the {\tt Exports}.
A Clifford algebra over a field \spad{K} is asserted to be a ring,
an algebra over \spad{K}, and a vector space over \spad{K}.
Its explicit exports include
\spad{e(n),} which returns the \eth{n} unit element.

\begin{figXmpLines}[caption={Part of the \textspadtype{CliffordAlgebra} domain.},label={fig-clifalg}]
NNI ==> NonNegativeInteger
PI  ==> PositiveInteger

CliffordAlgebra(n,K,q): Exports == Implementation where
    n: PI
    K: Field
    q: QuadraticForm(n, K)

    Exports == Join(Ring,Algebra(K),VectorSpace(K)) with
      e: PI -> %
          ...

    Implementation == add
      Qeelist :=
        [q.unitVector(i::PI) for i in 1..n]
      dim     :=  2^n
      Rep     := PrimitiveArray K
      New ==> new(dim, 0$K)$Rep
      x + y ==
        z := New
        for i in 0..dim-1 repeat z.i := x.i + y.i
        z
      addMonomProd: (K, NNI, K, NNI, %) -> %
      addMonomProd(c1, b1, c2, b2, z) ==  ...
      x * y ==
        z := New
        for ix in 0..dim-1 repeat
          if x.ix ~= 0 then for iy in 0..dim-1 repeat
            if y.iy ~= 0
            then addMonomProd(x.ix,ix,y.iy,iy,z)
          z
           ...
\end{figXmpLines}

The {\tt Implementation} part begins by defining a local variable
\spad{Qeelist} to hold the list of all \spad{q.v} where \spad{v}
runs over the unit vectors from 1 to the dimension \spad{n}.
Another local variable \spad{dim} is set to \mathOrSpad{2^n},
computed once and for all.
The representation for the domain is
\spadtype{PrimitiveArray(K)},
which is a basic array of elements from domain \spad{K}.
Line 18 defines \spad{New} as shorthand for the more lengthy
expression \spad{new(dim, 0$K)$Rep}, which computes a primitive
array of length \mathOrSpad{2^n} filled with \spad{0}'s from
domain \spad{K}.

Lines 19-22 define the sum of two elements \spad{x} and \spad{y}
straightforwardly.
First, a new array of all \spad{0}'s is created, then filled with
the sum of the corresponding elements.
Indexing for primitive arrays starts at 0.
The definition of the product of \spad{x} and \spad{y} first requires
the definition of a local function \userfun{addMonomProd}.
\Language{} knows it is local since it is not an exported function.
The types of all local functions must be declared.

For a demonstration of \spadtype{CliffordAlgebra}, see
\xmpref{CliffordAlgebra}.

% *********************************************************************
\head{section}{Example 2: Building A Query Facility}{ugDomsinsDatabase}
% *********************************************************************
%
We now turn to an entirely different kind of application,
building a query language for a database.

Here is the practical problem to solve.
The \Browse{} facility of \Language{} has a
database for all operations and constructors which is
stored on disk and accessed by \HyperName{}.
For our purposes here, we regard each line of this file as having
eight fields:
{\tt class, name, type, nargs, exposed, kind, origin,} and {\tt condition.}
Here is an example entry:

\begin{verbatim}
o`determinant`$->R`1`x`d`Matrix(R)`has(R,commutative("*"))
\end{verbatim}

In English, the entry means:
\begin{quotation}\raggedright
The operation \spadfun{determinant}: \spad{% -> R}
with {\it 1} argument, is
{\it exposed} and is exported by {\it domain} \spadtype{Matrix(R)}
if {\tt R has commutative("*")}.
\end{quotation}

Our task is to create a little query language that allows us
to get useful information from this database.

% *********************************************************************
\head{subsection}{A Little Query Language}{ugDomainsQueryLanguage}
% *********************************************************************

First we design a simple language for accessing information from
the database.
We have the following simple model in mind for its design.
Think of the database as a box of index cards.
There is only one search operation---it
takes the name of a field and a predicate
\index{predicate}
(a boolean-valued function) defined on the fields of the
index cards.
When applied, the search operation goes through the entire box
selecting only those index cards for which the predicate is true.
The result of a search is a new box of index cards.
This process can be repeated again and again.

The predicates all have a particularly simple form: {\it symbol}
{\tt =} {\it pattern}, where {\it symbol} designates one of the
fields, and {\it pattern} is a ``search string''---a string
that may contain a ``{\tt *}'' as a
wildcard.
Wildcards match any substring, including the empty string.
Thus the pattern {\tt "*ma*t"} matches
{\tt "mat", "doormat"} and {\tt "smart".}

To illustrate how queries are given, we give you a sneak preview
of the facility we are about to create.

\begin{xtc}
\begin{xtccomment}
Extract the database of all \Language{} operations.
\end{xtccomment}
\begin{spadsrc}
ops := getDatabase("o")
\end{spadsrc}
\begin{TeXOutput}
\begin{fricasmath}{1}
8002%
\end{fricasmath}
\end{TeXOutput}
\formatResultType{Database(IndexCard)}
\end{xtc}
\begin{xtc}
\begin{xtccomment}
How many exposed three-argument \spadfun{map} operations involving streams?
\end{xtccomment}
\begin{spadsrc}
ops.(name="map").(nargs="3").(type="*Stream*")
\end{spadsrc}
\begin{TeXOutput}
\begin{fricasmath}{2}
3%
\end{fricasmath}
\end{TeXOutput}
\formatResultType{Database(IndexCard)}
\end{xtc}

As usual, the arguments of \spadfun{elt} (\spadSyntax{.})
associate to the left.
The first \spadfun{elt} produces the set of all operations with
name {\tt map}.
The second \spadfun{elt} produces the set of all map operations
with three arguments.
The third \spadfun{elt} produces the set of all three-argument map
operations having a type mentioning \spadtype{Stream}.

Another thing we'd like to do is to extract one field from each of
the index cards in the box and look at the result.
Here is an example of that kind of request.

\begin{xtc}
\begin{xtccomment}
What constructors explicitly export a \spadfun{determinant} operation?
\end{xtccomment}
\begin{spadsrc}
elt(elt(elt(elt(ops,name="determinant"),origin),sort),unique)
\end{spadsrc}
\begin{TeXOutput}
\begin{fricasmath}{3}
\BRACKET{\STRING{"InnerMatrixLinearAlgebraFunctions"}\COMMA \STRING{%
"MatrixCategory"}\COMMA \STRING{"MatrixLinearAlgebraFunctions"}\COMMA \STRING%
{"SquareMatrixCategory"}}%
\end{fricasmath}
\end{TeXOutput}
\formatResultType{DataList(String)}
\end{xtc}

The first \spadfun{elt} produces the set of all index cards with
name {\tt determinant}.
The second \spadfun{elt} extracts the {\tt origin} component from
each index card. Each origin component
is the name of a constructor which directly
exports the operation represented by the index card.
Extracting a component from each index card produces what we call
a {\it datalist}.
The third \spadfun{elt}, {\tt sort}, causes the datalist of
origins to be sorted in alphabetic
order.
The fourth, {\tt unique}, causes duplicates to be removed.

Before giving you a more extensive demo of this facility,
we now build the necessary domains and packages to implement it.
%We will introduce a few of our minor conveniences.

% *********************************************************************
\head{subsection}{The Database Constructor}{ugDomainsDatabaseConstructor}
% *********************************************************************

We work from the top down. First, we define a database,
our box of index cards, as an abstract datatype.
For sake of illustration and generality,
we assume that an index card is some type \spad{S}, and
that a database is a box of objects of type \spad{S}.
Here is the \Language{} program defining the \pspadtype{Database}
domain.

\begin{xmpLines}
PI ==> PositiveInteger
Database(S): Exports == Implementation where
  S: Object with
    elt: (%, Symbol) -> String
    display: % -> Void
    fullDisplay: % -> Void

  Exports == with
    elt: (%,QueryEquation) -> %                          -- Select by an equation.
    elt: (%, Symbol) -> DataList String                  -- Select by a field name.
    "+": (%,%) -> %                                      -- Combine two databases.
    "-": (%,%) -> %                                      -- Subtract one from another.
    display: % -> Void                                   -- A brief database display.
    fullDisplay: % -> Void                               -- A full database display.
    fullDisplay: (%,PI,PI) -> Void                       -- A selective display.
    coerce: % -> OutputForm                              -- Display a database.
  Implementation == add
      ...
\end{xmpLines}

The domain constructor takes a parameter \spad{S}, which
stands for the class of index cards.
We describe an index card later.
Here think of an index card as a string which has
the eight fields mentioned above.

First, we tell \Language{} what operations we are going to require
from index cards.
We need an \spadfun{elt} to extract the contents of a field
(such as {\tt name} and {\tt type}) as a string.
For example,
\spad{c.name} returns a string that is the content of the
\spad{name} field on the index card \spad{c}.
We need to display an index card in two ways:
\pspadfun{display} shows only the name and type of an
operation;
\pspadfun{fullDisplay} displays all fields.
The display operations return no useful information and thus have
return type \spadtype{Void}.

Next, we tell \Language{} what operations the user can apply
to the database.
This part defines our little query language.
The most important operation is
{\frenchspacing\tt db . field = pattern} which
returns a new database, consisting of all index
cards of {\tt db} such that the \spad{field} part of the index
card is matched by the string pattern called \spad{pattern}.
The expression {\tt field = pattern} is an object of type
\spadtype{QueryEquation} (defined in the next section).

Another \spadfun{elt} is needed to produce a \pspadtype{DataList}
object.
Operation \spadop{+} is to merge two databases together;
\spadop{-} is used to subtract away common entries in a second
database from an initial database.
There are three display functions.
The \pspadfun{fullDisplay} function has two versions: one
that prints all the records, the other that prints only a fixed
number of records.
A \spadfun{coerce} to \spadtype{OutputForm} creates a display
object.

The {\tt Implementation} part of \spadtype{Database} is straightforward.
\begin{xmpLines}
  Implementation == add
    s: Symbol
    Rep := List S
    elt(db,equation) == ...
    elt(db,key) == [x.key for x in db]::DataList(String)
    display(db) ==  for x in db repeat display x
    fullDisplay(db) == for x in db repeat fullDisplay x
    fullDisplay(db, n, m) == for x in db for i in 1..m
      repeat
        if i >= n then fullDisplay x
    x+y == removeDuplicates! merge(x,y)
    x-y == mergeDifference(copy(x::Rep),
                           y::Rep)$MergeThing(S)
    coerce(db): OutputForm == (#db):: OutputForm
\end{xmpLines}

The database is represented by a list of elements of \spad{S} (index cards).
We leave the definition of the first \spadfun{elt} operation
(on line 4) until the next section.
The second \spadfun{elt} collects all the strings with field name
{\it key} into a list.
The \spadfun{display} function and first \spadfun{fullDisplay} function
simply call the corresponding functions from \spad{S}.
The second \spadfun{fullDisplay} function provides an efficient way of
printing out a portion of a large list.
The \spadop{+} is defined by using the existing
\spadfunFrom{merge}{List} operation defined on lists, then
removing duplicates from the result.
The \spadop{-} operation requires writing a corresponding
subtraction operation.
A package \spadtype{MergeThing} (not shown) provides this.

The \spadfun{coerce} function converts the database to an
\spadtype{OutputForm} by computing the number of index cards.
This is a good example of the independence of
the representation of an \Language{} object from how it presents
itself to the user. We usually do not want to look at a database---but
do care how many ``hits'' we get for a given query.
So we define the output representation of a database to be simply
the number of index cards our query finds.
% *********************************************************************
\head{subsection}{Query Equations}{ugDomainsQueryEquations}
% *********************************************************************

The predicate for our search is given by an object of type
\pspadtype{QueryEquation}.
\Language{} does not have such an object yet so we
have to invent it.

\begin{xmpLines}
QueryEquation(): Exports == Implementation where
  Exports == with
    equation: (Symbol, String) -> %
    variable: % -> Symbol
    value:    % -> String

  Implementation == add
    Rep := Record(var:Symbol, val:String)
    equation(x, s) == [x, s]
    variable q == q.var
    value    q == q.val
\end{xmpLines}

\Language{} converts an input expression of the form
\spad{a = b} to \spad{equation(a, b)}.
Our equations always have a symbol on the left and a string
on the right.
The {\tt Exports} part thus specifies an operation
\spadfun{equation} to create a query equation, and
\pspadfun{variable} and \pspadfun{value} to select the left- and
right-hand sides.
The {\tt Implementation} part uses \pspadtype{Record} for a
space-efficient representation of an equation.

Here is the missing definition for the \spadfun{elt} function of
\pspadtype{Database} in the last section:

\begin{xmpLines}
    elt(db,eq) ==
      field  := variable eq
      value := value eq
      [x for x in db | matches?(value,x.field)]
\end{xmpLines}

Recall that a database is represented by a list.
Line 4 simply runs over that list collecting all elements
such that the pattern (that is, \spad{value})
matches the selected field of the element.

% *********************************************************************
\head{subsection}{DataLists}{ugDomainsDataLists}
% *********************************************************************

Type \pspadtype{DataList} is a new type invented to hold the result
of selecting one field from each of the index cards in the box.
It is useful to make datalists extensions of lists---lists that
have special \spadfun{elt} operations defined on them for
sorting and removing duplicates.

\begin{xmpLines}
DataList(S:OrderedSet) : Exports == Implementation where
  Exports == ListAggregate(S) with
    elt: (%,"unique") -> %
    elt: (%,"sort") -> %
    elt: (%,"count") -> NonNegativeInteger
    coerce: List S -> %

  Implementation ==  List(S) add
    Rep := List S
    elt(x,"unique") == removeDuplicates(x)
    elt(x,"sort") == sort(x)
    elt(x,"count") == #x
    coerce(x:List S) == x :: %
\end{xmpLines}

The {\tt Exports} part asserts that datalists belong to the
category \spadtype{ListAggregate}.
Therefore, you can use all the usual list operations on datalists,
such as \spadfunFrom{first}{List}, \spadfunFrom{rest}{List}, and
\spadfunFrom{concat}{List}.
In addition, datalists have four explicit operations.
Besides the three \spadfun{elt} operations, there is a
\spadfun{coerce} operation that creates datalists from lists.

The {\tt Implementation} part needs only to define four functions.
All the rest are obtained from \spadtype{List(S)}.

% *********************************************************************
\head{subsection}{Index Cards}{ugDomainsDatabase}
% *********************************************************************

An index card comes from a file as one long string.
We define functions that extract substrings from the long
string.
Each field has a name that
is passed as a second argument to \spadfun{elt}.

\begin{xmpLines}
IndexCard() == Implementation where
  Exports == with
    elt: (%, Symbol) -> String
    display: % -> Void
    fullDisplay: % -> Void
    coerce: String -> %
  Implementation == String add ...
\end{xmpLines}

We leave the {\tt Implementation} part to the reader.
All operations involve straightforward string manipulations.

% *********************************************************************
\head{subsection}{Creating a Database}{ugDomainsCreating}
% *********************************************************************

We must not forget one important operation: one that builds the database in the
first place!
We'll name it \pspadfun{getDatabase} and put it in a package.
This function is implemented by calling the \Lisp{} function
\spad{getBrowseDatabase(s)} to get appropriate information from
\Browse{}.
This operation takes a string indicating which lines you
want from the database: \spad{"o"} gives you all operation
lines, and \spad{"k"}, all constructor lines.
Similarly, \spad{"c"}, \spad{"d"}, and \spad{"p"} give
you all category, domain and package lines respectively.
%
\begin{xmpLines}
OperationsQuery(): Exports == Implementation where
  Exports == with
    getDatabase: String -> Database(IndexCard)

  Implementation == add
    getDatabase(s) == getBrowseDatabase(s)$Lisp
\end{xmpLines}

We do not bother creating a special name for databases of index
cards.
\pspadtype{Database (IndexCard)} will do.
Notice that we used the package \pspadtype{OperationsQuery} to
create, in effect,
a new kind of domain: \pspadtype{Database(IndexCard)}.

% *********************************************************************
\head{subsection}{Putting It All Together}{ugDomainsPutting}
% *********************************************************************

To create the database facility, you put all these constructors
into one file.\footnote{You could use separate files, but we
are putting them all together because, organizationally, that is
the logical thing to do.}
At the top of the file put \spadsys{)abbrev} commands, giving the
constructor abbreviations you created.

\begin{xmpLines}
)abbrev domain  ICARD   IndexCard
)abbrev domain  QEQUAT  QueryEquation
)abbrev domain  MTHING  MergeThing
)abbrev domain  DLIST   DataList
)abbrev domain  DBASE   Database
)abbrev package OPQUERY OperationsQuery
\end{xmpLines}

With all this in {\bf alql.spad}, for example, compile it using
\syscmdindex{compile}
\begin{verbatim}
)compile alql
\end{verbatim}
and then load each of the constructors:
\begin{verbatim}
)load ICARD QEQUAT MTHING DLIST DBASE OPQUERY
\end{verbatim}
\syscmdindex{load}
You are ready to try some sample queries.

% *********************************************************************
\head{subsection}{Example Queries}{ugDomainsExamples}
% *********************************************************************

Our first set of queries give some statistics on constructors in
the current \Language{} system.

\begin{xtc}
\begin{xtccomment}
How many constructors does \Language{} have?
\end{xtccomment}
\begin{spadsrc}
ks := getDatabase "k"
\end{spadsrc}
\begin{TeXOutput}
\begin{fricasmath}{1}
1212%
\end{fricasmath}
\end{TeXOutput}
\formatResultType{Database(IndexCard)}
\end{xtc}
\begin{xtc}
\begin{xtccomment}
Break this down into the number of categories, domains, and packages.
\end{xtccomment}
\begin{spadsrc}
[ks.(kind=k) for k in ["c","d","p"]]
\end{spadsrc}
\begin{TeXOutput}
\begin{fricasmath}{2}
\BRACKET{264\COMMA 413\COMMA 535}%
\end{fricasmath}
\end{TeXOutput}
\formatResultType{List(Database(IndexCard))}
\end{xtc}
\begin{xtc}
\begin{xtccomment}
What are all the domain constructors that take 5 parameters?
\end{xtccomment}
\begin{spadsrc}
elt(ks.(kind="d").(nargs="5"),name)
\end{spadsrc}
\begin{TeXOutput}
\begin{fricasmath}{3}
\BRACKET{\STRING{"FractionalIdealAsModule"}\COMMA \STRING{"IndexedJetBundle"}%
\COMMA \STRING{"InnerIndexedTwoDimensionalArray"}\COMMA \STRING{%
"ModularField"}\COMMA \STRING{"ModularRing"}\COMMA \STRING{%
"RadicalFunctionField"}\COMMA \STRING{"ResidueRing"}\COMMA \STRING{%
"TensorProduct"}}%
\end{fricasmath}
\end{TeXOutput}
\formatResultType{DataList(String)}
\end{xtc}
\begin{xtc}
\begin{xtccomment}
How many constructors have ``Matrix'' in their name?
\end{xtccomment}
\begin{spadsrc}
mk := ks.(name="*Matrix*")
\end{spadsrc}
\begin{TeXOutput}
\begin{fricasmath}{4}
34%
\end{fricasmath}
\end{TeXOutput}
\formatResultType{Database(IndexCard)}
\end{xtc}
\begin{xtc}
\begin{xtccomment}
What are the names of those that are domains?
\end{xtccomment}
\begin{spadsrc}
elt(mk.(kind="d"),name)
\end{spadsrc}
\begin{TeXOutput}
\begin{fricasmath}{5}
\BRACKET{\STRING{"ComplexDoubleFloatMatrix"}\COMMA \STRING{%
"DenavitHartenbergMatrix"}\COMMA \STRING{"DirectProductMatrixModule"}\COMMA %
\STRING{"DoubleFloatMatrix"}\COMMA \STRING{"IndexedMatrix"}\COMMA \STRING{%
"LieSquareMatrix"}\COMMA \STRING{"Matrix"}\COMMA \STRING{"RectangularMatrix"}%
\COMMA \STRING{"SparseEchelonMatrix"}\COMMA \STRING{"SquareMatrix"}\COMMA %
\STRING{"ThreeDimensionalMatrix"}\COMMA \STRING{"U16Matrix"}\COMMA \STRING{%
"U32Matrix"}\COMMA \STRING{"U8Matrix"}}%
\end{fricasmath}
\end{TeXOutput}
\formatResultType{DataList(String)}
\end{xtc}
\begin{xtc}
\begin{xtccomment}
How many operations are there in the library?
\end{xtccomment}
\begin{spadsrc}
o := getDatabase "o"
\end{spadsrc}
\begin{TeXOutput}
\begin{fricasmath}{6}
8002%
\end{fricasmath}
\end{TeXOutput}
\formatResultType{Database(IndexCard)}
\end{xtc}
\begin{xtc}
\begin{xtccomment}
Break this down into categories, domains, and packages.
\end{xtccomment}
\begin{spadsrc}
[o.(kind=k) for k in ["c","d","p"]]
\end{spadsrc}
\begin{TeXOutput}
\begin{fricasmath}{7}
\BRACKET{2011\COMMA 2791\COMMA 3200}%
\end{fricasmath}
\end{TeXOutput}
\formatResultType{List(Database(IndexCard))}
\end{xtc}

The query language is helpful in getting information about a
particular operation you might like to apply.
While this information can be obtained with
\Browse{}, the use of the query database gives you data that you
can manipulate in the workspace.

\begin{xtc}
\begin{xtccomment}
How many operations have ``eigen'' in the name?
\end{xtccomment}
\begin{spadsrc}
eigens := o.(name="*eigen*")
\end{spadsrc}
\begin{TeXOutput}
\begin{fricasmath}{8}
9%
\end{fricasmath}
\end{TeXOutput}
\formatResultType{Database(IndexCard)}
\end{xtc}
\begin{xtc}
\begin{xtccomment}
What are their names?
\end{xtccomment}
\begin{spadsrc}
elt(eigens,name)
\end{spadsrc}
\begin{TeXOutput}
\begin{fricasmath}{9}
\BRACKET{\STRING{"eigenMatrix"}\COMMA \STRING{"eigenvalues"}\COMMA \STRING{%
"eigenvalues"}\COMMA \STRING{"eigenvalues"}\COMMA \STRING{"eigenvector"}%
\COMMA \STRING{"eigenvector"}\COMMA \STRING{"eigenvectors"}\COMMA \STRING{%
"eigenvectors"}\COMMA \STRING{"eigenvectors"}}%
\end{fricasmath}
\end{TeXOutput}
\formatResultType{DataList(String)}
\end{xtc}
\begin{xtc}
\begin{xtccomment}
Where do they come from?
\end{xtccomment}
\begin{spadsrc}
elt(elt(elt(eigens,origin),sort),unique) 
\end{spadsrc}
\begin{TeXOutput}
\begin{fricasmath}{10}
\BRACKET{\STRING{"EigenPackage"}\COMMA \STRING{"InnerEigenPackage"}\COMMA %
\STRING{"RadicalEigenPackage"}}%
\end{fricasmath}
\end{TeXOutput}
\formatResultType{DataList(String)}
\end{xtc}

The operations \spadop{+} and \spadop{-} are useful for
constructing small databases and combining them.
However, remember that the only matching you can do is string
matching.
Thus a pattern such as {\tt "*Matrix*"} on the type field
matches
any type containing \spadtype{Matrix}, \spadtype{MatrixCategory},
\spadtype{SquareMatrix}, and so on.

\begin{xtc}
\begin{xtccomment}
How many operations mention ``Matrix'' in their type?
\end{xtccomment}
\begin{spadsrc}
tm := o.(type="*Matrix*")
\end{spadsrc}
\begin{TeXOutput}
\begin{fricasmath}{11}
329%
\end{fricasmath}
\end{TeXOutput}
\formatResultType{Database(IndexCard)}
\end{xtc}
\begin{xtc}
\begin{xtccomment}
How many operations come from constructors with ``Matrix'' in
their name?
\end{xtccomment}
\begin{spadsrc}
fm := o.(origin="*Matrix*")
\end{spadsrc}
\begin{TeXOutput}
\begin{fricasmath}{12}
245%
\end{fricasmath}
\end{TeXOutput}
\formatResultType{Database(IndexCard)}
\end{xtc}
\begin{xtc}
\begin{xtccomment}
How many operations are in \spad{fm} but not in \spad{tm}?
\end{xtccomment}
\begin{spadsrc}
fm-tm 
\end{spadsrc}
\begin{TeXOutput}
\begin{fricasmath}{13}
204%
\end{fricasmath}
\end{TeXOutput}
\formatResultType{Database(IndexCard)}
\end{xtc}
\begin{xtc}
\begin{xtccomment}
Display the operations that both mention ``Matrix'' in their type
and come from a constructor having ``Matrix'' in their name.
\end{xtccomment}
\begin{spadsrc}
fullDisplay(fm-%) 
\end{spadsrc}
\end{xtc}
\begin{xtc}
\begin{xtccomment}
How many operations involve matrices?
\end{xtccomment}
\begin{spadsrc}
m := tm+fm 
\end{spadsrc}
\begin{TeXOutput}
\begin{fricasmath}{15}
525%
\end{fricasmath}
\end{TeXOutput}
\formatResultType{Database(IndexCard)}
\end{xtc}
\begin{xtc}
\begin{xtccomment}
Display 4 of them.
\end{xtccomment}
\begin{spadsrc}
fullDisplay(m, 202, 205) 
\end{spadsrc}
\end{xtc}
\begin{xtc}
\begin{xtccomment}
How many distinct names of operations involving matrices are there?
\end{xtccomment}
\begin{spadsrc}
elt(elt(elt(m,name),unique),count) 
\end{spadsrc}
\begin{TeXOutput}
\begin{fricasmath}{17}
284%
\end{fricasmath}
\end{TeXOutput}
\formatResultType{PositiveInteger}
\end{xtc}

% !! DO NOT MODIFY THIS FILE BY HAND !! Created by spool2tex.awk.

% Copyright (c) 1991-2002, The Numerical ALgorithms Group Ltd.
% All rights reserved.
%
% Redistribution and use in source and binary forms, with or without
% modification, are permitted provided that the following conditions are
% met:
%
%     - Redistributions of source code must retain the above copyright
%       notice, this list of conditions and the following disclaimer.
%
%     - Redistributions in binary form must reproduce the above copyright
%       notice, this list of conditions and the following disclaimer in
%       the documentation and/or other materials provided with the
%       distribution.
%
%     - Neither the name of The Numerical ALgorithms Group Ltd. nor the
%       names of its contributors may be used to endorse or promote products
%       derived from this software without specific prior written permission.
%
% THIS SOFTWARE IS PROVIDED BY THE COPYRIGHT HOLDERS AND CONTRIBUTORS "AS
% IS" AND ANY EXPRESS OR IMPLIED WARRANTIES, INCLUDING, BUT NOT LIMITED
% TO, THE IMPLIED WARRANTIES OF MERCHANTABILITY AND FITNESS FOR A
% PARTICULAR PURPOSE ARE DISCLAIMED. IN NO EVENT SHALL THE COPYRIGHT OWNER
% OR CONTRIBUTORS BE LIABLE FOR ANY DIRECT, INDIRECT, INCIDENTAL, SPECIAL,
% EXEMPLARY, OR CONSEQUENTIAL DAMAGES (INCLUDING, BUT NOT LIMITED TO,
% PROCUREMENT OF SUBSTITUTE GOODS OR SERVICES-- LOSS OF USE, DATA, OR
% PROFITS-- OR BUSINESS INTERRUPTION) HOWEVER CAUSED AND ON ANY THEORY OF
% LIABILITY, WHETHER IN CONTRACT, STRICT LIABILITY, OR TORT (INCLUDING
% NEGLIGENCE OR OTHERWISE) ARISING IN ANY WAY OUT OF THE USE OF THIS
% SOFTWARE, EVEN IF ADVISED OF THE POSSIBILITY OF SUCH DAMAGE.

% *********************************************************************
\head{chapter}{Browse}{ugBrowse}
% *********************************************************************

This chapter discusses the \Browse{}
\index{Browse@\Browse{}}
component of \HyperName{}.
\index{HyperDoc@{\HyperName{}}}
We suggest you invoke \Language{} and work through this
chapter, section by section, following our examples to gain some
familiarity with \Browse{}.

% *********************************************************************
\head{section}{The Front Page: Searching the Library}{ugBrowseStart}
% *********************************************************************
To enter \Browse{}, click on {\bf Browse} on the top level page
of \HyperName{} to get the {\it front page} of \Browse{}.
%
%324pt is 4.5",180pt is 2.5",432pt is 6"=textwidth,54=(432-324)/2
%ps files are 4.5"x2.5" except source 4.5"x2.5"
%
\begin{figure}[htbp]
\begin{picture}(324,180)%(-54,0)
\special{psfile=h-brfront.ps}
\end{picture}
\caption{The Browse front page.}
\end{figure}

To use this page, you first enter a \spadgloss{search string} into
the input area at the top, then click on one of the buttons below.
We show the use of each of the buttons by example.

\subsubsection{Constructors}

First enter the search string {\tt Matrix} into the input area and
click on {\bf Constructors}.
What you get is the {\it constructor page} for \spadtype{Matrix}.
We show and describe this page in detail in
\spadref{ugBrowseDomain}.
By convention, \Language{} does a case-insensitive search for a
match.
Thus {\tt matrix} is just as good as {\tt Matrix}, has the same
effect as {\tt MaTrix}, and so on.
We recommend that you generally use small letters for names
however.
A search string with only capital letters has a special meaning
(see \spadref{ugBrowseCapitalizationConvention}).


Click on \UpBitmap{} to return to the \Browse{} front page.

Use the symbol ``{\tt *}'' in search strings as a \spadgloss{wild
card}.
A wild card matches any substring, including the empty string.
For example, enter the search string {\tt *matrix*} into the input
area and click on {\bf Constructors}.\footnote{To get only
categories, domains, or packages, rather than all constructors,
you can click on the corresponding button to the right of {\bf
Constructors}.}
What you get is a table of all constructors whose names contain
the string ``{\tt matrix}.''

\begin{figure}[htbp]
\begin{picture}(324,180)%(-54,0)
\special{psfile=h-consearch.ps}
\end{picture}
\caption{Table of exposed constructors matching \texttt{*matrix*}.}
\end{figure}

%% Following para replaced 1995oct30 MGR
%These are all the \spadglossSee{exposed}{expose} constructors in
%\Language{}.
%To see how to get all exposed and unexposed constructors in
%\Language{}, skip to the section entitled {\bf Exposure} in
%\spadref{ugBrowseOptions}.
All constructors containing the string are listed, whether
\spadglossSee{exposed}{expose} or \spadglossSee{unexposed}{expose}.
You can hide the names of the unexposed constructors by clicking
on the {\it *=}{\bf unexposed} button in the {\it Views} panel at
the bottom of the window.
(The button will change to {\bf exposed} {\it only}.)

One of the names in this table is \spadtype{Matrix}.
Click on \spadtype{Matrix}.
What you get is again the constructor page for \spadtype{Matrix}.
As you see, \Browse{} gives you a large network of
information in which there are many ways to reach the same
pages.
\exptypeindex{Matrix}

Again click on the \UpBitmap{} to return to the table of constructors
whose names contain {\tt matrix}.
%Below the table is a {\bf Views} panel. % here & globally MGR 1995oct30
Below the table is a {\it Views} panel.
This panel contains buttons that let you view constructors in different
ways.
To learn about views of constructors, skip to
\spadref{ugBrowseViewsOfConstructors}.

Click on \UpBitmap{} to return to the \Browse{} front page.

\subsubsection{Operations}

Enter {\tt *matrix} into the input area and click on {\bf
Operations}.
This time you get a table of {\it operations} whose names end with {\tt
matrix} or {\tt Matrix}.

\begin{figure}[htbp]
\begin{picture}(324,180)%(-54,0)
\special{psfile=h-matrixops.ps}
\end{picture}
\caption{Table of operations matching \texttt{*matrix}.}
\end{figure}

If you select an operation name, you go to a page describing all
the operations in \Language{} of that name.
At the bottom of an operation page is another kind of {\it Views} panel,
one for operation pages.
To learn more about these views, skip to
\spadref{ugBrowseViewsOfOperations}.

Click on \UpBitmap{} to return to the \Browse{} front page.

\subsubsection{Attributes}

This button gives you a table of attribute names that match the
search string. Enter the search string {\tt *} and click on
{\bf Attributes} to get a list
of all system attributes.

Click on \UpBitmap{} to return to the \Browse{} front page.


\begin{figure}[htbp]
\begin{picture}(324,180)%(-54,0)
\special{psfile=h-atsearch.ps}
\end{picture}
\caption{Table of \Language{} attributes.}
\end{figure}

Again there is a {\it Views} panel at the bottom with buttons that let
you view the attributes in different ways.

\subsubsection{General}

This button does a general search for all constructor, operation, and
attribute names matching the search string.
Enter the search string \allowbreak
{\tt *matrix*} into the input area.
Click on {\bf General} to find all constructs that have {\tt
matrix} as a part of their name.

\begin{figure}[htbp]
\begin{picture}(324,180)%(-54,0)
\special{psfile=h-gensearch.ps}
\end{picture}
\caption{Table of all constructs matching \texttt{*matrix*}.}
\end{figure}

The summary gives you all the names under a heading when the number of
entries is less than 10. % "less than 10." replaces the following:
                         % sufficiently small%\footnote{See
%\spadref{ugBrowseOptions} to see how you can change this.}.
%% MGR 1995oct31

Click on \UpBitmap{} to return to the \Browse{} front page.

\subsubsection{Documentation}

Again enter the search key {\tt *matrix*} and this time click on
{\bf Documentation}.
This search matches any constructor, operation, or attribute
name whose documentation contains a substring matching {\tt
matrix}.

\begin{figure}[htbp]
\begin{picture}(324,180)%(-54,0)
\special{psfile=h-docsearch.ps}
\end{picture}
\caption{Table of constructs with documentation matching \texttt{*matrix*}.}
\end{figure}

Click on \UpBitmap{} to return to the \Browse{} front page.

\subsubsection{Complete}

This search combines both {\bf General} and {\bf Documentation}.

\begin{figure}[htbp]
\begin{picture}(324,180)%(-54,0)
\special{psfile=h-comsearch.ps}
\end{picture}
\caption{Table summarizing complete search for pattern \texttt{*matrix*}.}
\end{figure}

% *********************************************************************
\head{section}{The Constructor Page}{ugBrowseDomain}
% *********************************************************************

In this section we look in detail at a constructor page for domain
\spadtype{Matrix}.
Enter {\tt matrix} into the input area on the main \Browse{} page
and click on {\bf Constructors}.

\begin{figure}[htbp]
\begin{picture}(324,180)%(-54,0)
\special{psfile=h-matpage.ps}
\end{picture}
\caption{Constructor page for \protect\spadtype{Matrix}.}
\end{figure}


The header part tells you that \spadtype{Matrix} has abbreviation
\spadtype{MATRIX} and one argument called {\tt R} that must be a
domain of category \spadtype{Ring}.
Just what domains can be arguments of \spadtype{Matrix}?
To find this out, click on the {\tt R} on the second line of the
heading.
What you get is a table of all acceptable domain parameter values
of {\tt R}, or a table of \spadgloss{rings} in \Language{}.

\begin{figure}[htbp]
\begin{picture}(324,180)%(-54,0)
\special{psfile=h-matargs.ps}
\end{picture}
\caption{Table of acceptable domain parameters to \protect\spadtype{Matrix}.}
\end{figure}

Click on \UpBitmap{} to return to the constructor page for
\spadtype{Matrix}.
\newpage

If you have access to the source code of \Language{}, the third
\index{source code}
line of the heading gives you the name of the source file
containing the definition of \spadtype{Matrix}.
Click on it to pop up an editor window containing the source code
of \spadtype{Matrix}.

\begin{figure}[htbp]
\begin{picture}(324,168)%(-54,0)
\special{psfile=h-matsource.ps}
\end{picture}
\caption{Source code for \protect\spadtype{Matrix}.}
\end{figure}

We recommend that you leave the editor window up while working
through this chapter as you occasionally may want to refer to it.
\newpage

% *********************************************************************
\head{subsection}{Constructor Page Buttons}{ugBrowseDomainButtons}
% *********************************************************************

We examine each button on this page in order.

\subsubsection{Description}

Click here to bring up a page with a brief description of
constructor \spadtype{Matrix}.
If you have access to system source code, note that these comments
can be found directly over the constructor definition.

\begin{figure}[htbp]
\begin{picture}(324,180)%(-54,0)
\special{psfile=h-matdesc.ps}
\end{picture}
\caption{Description page for \protect\spadtype{Matrix}.}
\end{figure}

\subsubsection{Operations}

Click here to get a table of operations exported by
\spadtype{Matrix}.
You may wish to widen the window to have multiple columns as
below.

\begin{figure}[htbp]
\begin{picture}(324,180)%(-54,0)
\special{psfile=h-matops.ps}
\end{picture}
\caption{Table of operations from \protect\spadtype{Matrix}.}
\end{figure}

If you click on an operation name, you bring up a description
page for the operations.
For a detailed description of these pages, skip to
\spadref{ugBrowseViewsOfOperations}.

\subsubsection{Attributes}

Click here to get a table of the two attributes exported by
\spadtype{Matrix}:
\index{attribute}
\spadtype{finiteAggregate} and \spadtype{shallowlyMutable}.
These are two computational properties that result from
\spadtype{Matrix} being regarded as a data structure.

\begin{figure}[htbp]
\begin{picture}(324,180)%(-54,0)
\special{psfile=h-matats.ps}
\end{picture}
\caption{Attributes from \protect\spadtype{Matrix}.}
\end{figure}

\subsubsection{Examples}

Click here to get an {\it examples page} with examples of operations to
create and manipulate matrices.

\begin{figure}[htbp]
\begin{picture}(324,180)%(-54,0)
\special{psfile=h-matexamp.ps}
\end{picture}
\caption{Example page for \protect\spadtype{Matrix}.}
\end{figure}

Read through this section.
Try selecting the various buttons.
Notice that if you click on an operation name, such as
\spadfunFrom{new}{Matrix}, you bring up a description page for that
operation from \spadtype{Matrix}.

Example pages have several examples of \Language{} commands.
Each example has an active button to its left.
Click on it!
A pre-computed answer is pasted into the page immediately following the
command.
If you click on the button a second time, the answer disappears.
This button thus acts as a toggle:
``now you see it; now you don't.''

Note also that the \Language{} commands themselves are active.
If you want to see \Language{} execute the command, then click on it!
A new \Language{} window appears on your screen and the command is
executed.


\subsubsection{Exports}

Click here to see a page describing the exports of \spadtype{Matrix}
exactly as described by the source code.

\begin{figure}[htbp]
\begin{picture}(324,180)%(-54,0)
\special{psfile=h-matexports.ps}
\end{picture}
\caption{Exports of \protect\spadtype{Matrix}.}
\end{figure}

As you see, \spadtype{Matrix} declares that it exports all the operations
and attributes exported by category
\spadtype{MatrixCategory(R, Row, Col)}.
In addition, two operations, \spadfun{diagonalMatrix} and
\spadfun{inverse}, are explicitly exported.

To learn a little about the structure of \Language{}, we suggest you do
the following exercise.
Otherwise, go on to the next section.
\spadtype{Matrix} explicitly exports only two operations.
The other operations are thus exports of \spadtype{MatrixCategory}.
In general, operations are usually not explicitly exported by a domain.
Typically they are \spadglossSee{inherited}{inherit} from several
different categories.
Let's find out from where the operations of \spadtype{Matrix} come.

\begin{enumerate}
\item Click on {\bf MatrixCategory}, then on {\bf Exports}.
Here you see that {\bf MatrixCategory} explicitly exports many matrix
operations.
Also, it inherits its operations from
\spadtype{TwoDimensionalArrayCategory}.

\item Click on {\bf TwoDimensionalArrayCategory}, then on {\bf Exports}.
Here you see explicit operations dealing with rows and columns.
In addition, it inherits operations from
\spadtype{HomogeneousAggregate}.

%\item Click on {\bf HomogeneousAggregate}, then on {\bf Exports}.
%And so on.
%If you continue doing this, eventually you will

\item Click on \UpBitmap{} and then
click on {\bf Object}, then on {\bf Exports}, where you see
there are no exports.

\item Click on \UpBitmap{} repeatedly to return to the constructor page
for \spadtype{Matrix}.

\end{enumerate}

\subsubsection{Related Operations}

Click here bringing up a table of operations that are exported by
\spadglossSee{packages}{package} but not by \spadtype{Matrix} itself.

\begin{figure}[htbp]
\begin{picture}(324,180)%(-54,0)
\special{psfile=h-matrelops.ps}
\end{picture}
\caption{Related operations of \protect\spadtype{Matrix}.}
\end{figure}

To see a table of such packages, use the {\bf Relatives} button on the
{\bf Cross Reference} page described next.


% *********************************************************************
\head{subsection}{Cross Reference}{ugBrowseCrossReference}
% *********************************************************************
Click on the {\bf Cross Reference} button on the main constructor page
for \spadtype{Matrix}.
This gives you a page having various cross reference information stored
under the respective buttons.

\begin{figure}[htbp]
\begin{picture}(324,180)%(-54,0)
\special{psfile=h-matxref.ps}
\end{picture}
\caption{Cross-reference page for \protect\spadtype{Matrix}.}
\end{figure}

\subsubsection{Parents}

The parents of a domain are the same as the categories mentioned under
the {\bf Exports} button on the first page.
Domain \spadtype{Matrix} has only one parent but in general a domain can
have any number.

\subsubsection{Ancestors}

The \spadglossSee{ancestors}{ancestor} of a constructor consist of its parents, the
parents of its parents, and so on.
Did you perform the exercise in the last section under {\bf Exports}?
If so, you  see here all the categories you found while ascending the
{\bf Exports} chain for \spadtype{Matrix}.

\subsubsection{Relatives}

The \spadglossSee{relatives}{relative} of a domain constructor are package
constructors that provide operations in addition to those
\spadglossSee{exported}{export} by the domain.

Try this exercise.
\begin{enumerate}
\item Click on {\bf Relatives}, bringing up a list of
\spadglossSee{packages}{package}.

\item Click on {\bf LinearSystemMatrixPackage} bringing up its
constructor page.\footnote{You may want to widen your \HyperName{}
window to make what follows more legible.}

\item Click on {\bf Operations}.
Here you see \spadfun{rank}, an operation also exported by
\spadtype{Matrix} itself.

\item Click on {\bf rank}.
This \spadfunFrom{rank}{LinearSystemMatrixPackage} has two arguments and
thus is different from the \spadfunFrom{rank}{Matrix} from
\spadtype{Matrix}.

\item Click on \UpBitmap{} to return to the list of operations for the
package \spadtype{LinearSystemMatrixPackage}.

\item Click on {\bf solve} to bring up a
\spadfunFrom{solve}{LinearSystemMatrixPackage} for linear systems of
equations.

\item Click on \UpBitmap{} several times to return to the cross
reference page for \spadtype{Matrix}.
\end{enumerate}

\subsubsection{Dependents}

The \spadglossSee{dependents}{dependent} of a constructor are those
\spadglossSee{domains}{domain} or \spadglossSee{packages}{package}
that mention that
constructor either as an argument or in its \spadglossSee{exports}{export}.

If you click on {\bf Dependents} two entries may surprise you:
\spadtype{RectangularMatrix} and \spadtype{SquareMatrix}.
This happens because \spadtype{Matrix}, as it turns out, appears in
signatures of operations exported by these domains.

\subsubsection{Search Path}

The term \spadgloss{search path} refers to the {\it search order} for
functions.
If you are an expert user or curious about how the \Language{} system
works, try the following exercise.
Otherwise, you best skip this button and go on to {\bf Users}.

Clicking on {\bf Search Path} gives you a
list of domain constructors:
\spadtype{InnerIndexedTwoDimensionalArray},
\aliascon{MatrixCategory&}{MATCAT-},
\aliascon{TwoDimensionalArrayCategory&}{ARR2CAT-},
\aliascon{HomogeneousAggregate&}{HOAGG-},
\aliascon{Aggregate&}{AGG-},
\aliascon{Evalable&}{EVALAB-},
\aliascon{SetCategory&}{SETCAT-},
\aliascon{InnerEvalable&}{IEVALAB-},
\aliascon{BasicType&}{BASTYPE-}.
What are these constructors and how are they used?

We explain by an example.
Suppose you create a matrix using the interpreter, then ask for its
\spadfun{rank}.
\Language{} must then find a function implementing the \spadfun{rank}
operation for matrices.
The first place \Language{} looks for \spadfun{rank} is in the \spadtype{Matrix}
domain.

If not there, the search path of \spadtype{Matrix} tells \Language{} where
else to look.
Associated with the matrix domain are eight other search path domains.
Their order is important.
\Language{} first searches the first one,
\spadtype{InnerIndexedTwoDimensionalArray}.
If not there, it searches the second \aliascon{MatrixCategory&}{MATCAT-}.
And so on.

Where do these {\it search path constructors} come from?
The source code for \spadtype{Matrix} contains this syntax for the
\spadgloss{function body} of
\spadtype{Matrix}:\footnote{\spadtype{InnerIndexedTwoDimensionalArray}
is a special domain implemented for matrix-like domains to provide
efficient implementations of \twodim{} arrays.
For example, domains of category \spadtype{TwoDimensionalArrayCategory}
can have any integer as their \spad{minIndex}.
Matrices and other members of this special ``inner'' array have their
\spad{minIndex} defined as \spad{1}.}
\begin{verbatim}
InnerIndexedTwoDimensionalArray(R,mnRow,mnCol,Row,Col)
   add ...
\end{verbatim}
where the ``{\tt ...}'' denotes all the code that follows.
In English, this means:
``The functions for matrices are defined as those from
\spadtype{InnerIndexedTwoDimensionalArray} domain augmented by those
defined in `{\tt ...}','' where the latter take precedence.

This explains \spadtype{InnerIndexedTwoDimensionalArray}.
The other names, those with names ending with an ampersand \spadSyntax{&} are
\spadglossSee{default packages}{default package}
for categories to which \spadtype{Matrix} belongs.
Default packages are ordered by the notion of ``closest ancestor.''

\subsubsection{Users}

A user of \spadtype{Matrix} is any constructor that uses
\spadtype{Matrix} in its implementation.
For example, \spadtype{Complex} is a user of \spadtype{Matrix}; it
exports several operations that take matrices as arguments or return
matrices as values.\footnote{A constructor is a user of
\spadtype{Matrix} if it handles any matrix.
For example, a constructor having internal (unexported) operations
dealing with matrices is also a user.}

\subsubsection{Uses}

A \spadgloss{benefactor} of \spadtype{Matrix} is any constructor that
\spadtype{Matrix} uses in its implementation.
This information, like that for clients, is gathered from run-time
structures.\footnote{The benefactors exclude constructors such as
\spadtype{PrimitiveArray} whose operations macro-expand and so vanish
from sight!}

Cross reference pages for categories have some different buttons on
them.
Starting with the constructor page of \spadtype{Matrix}, click on
\spadtype{Ring} producing its constructor page.
Click on {\bf Cross Reference},
producing the cross-reference page for \spadtype{Ring}.
Here are buttons {\bf Parents} and {\bf Ancestors} similar to the notion
for domains, except for categories the relationship between parent and
child is defined through \spadgloss{category extension}.

\subsubsection{Children}

Category hierarchies go both ways.
There are children as well as parents.
A child can have any number of parents, but always at least one.
Every category is therefore a descendant of exactly one category:
\spadtype{Object}.

\subsubsection{Descendants}

These are children, children of children, and so on.

Category hierarchies are complicated by the fact that categories take
parameters.
Where a parameterized category fits into a hierarchy {\it may} depend on
values of its parameters.
In general, the set of categories in \Language{} forms a {\it directed
acyclic graph}, that is, a graph with directed arcs and no cycles.

\subsubsection{Domains}

This produces a table of all domain constructors that can possibly be
rings (members of category \spadtype{Ring}).
Some domains are unconditional rings.
Others are rings for some parameters and not for others.
To find out which, select the {\bf conditions} button in the views
panel.
For example, \spadtype{DirectProduct(n, R)} is a ring if {\tt R} is a
ring.



% *********************************************************************
\head{subsection}{Views Of Constructors}{ugBrowseViewsOfConstructors}
% *********************************************************************

Below every constructor table page is a {\it Views} panel.
As an example, click on {\bf Cross Reference} from
the constructor page of \spadtype{Matrix},
then on {\bf Benefactors} to produce a
short table of constructor names.

The {\it Views} panel is at the bottom of the page.
Two items, {\it names} and {\it conditions,} are in italics.
Others are active buttons.
The active buttons are those that give you useful alternative views
on this table of constructors.
Once you select a view, you notice that the button turns
off (becomes italicized) so that you cannot reselect it.

\subsubsection{names}

This view gives you a table of names.
Selecting any of these names brings up the constructor page for that
constructor.

\subsubsection{abbrs}

This view gives you a table of abbreviations, in the same order as the
original constructor names.
Abbreviations are in capitals and are limited to 7 characters.
They can be used interchangeably with constructor names in input areas.

\subsubsection{kinds}

This view organizes constructor names into
the three kinds: categories, domains and packages.

\subsubsection{files}

This view gives a table of file names for the source
code of the constructors in alphabetic order after removing
duplicates.

\subsubsection{parameters}

This view presents constructors with the arguments.
This view of the benefactors of \spadtype{Matrix} shows that
\spadtype{Matrix} uses as many as five different \spadtype{List} domains
in its implementation.

\subsubsection{filter}

This button is used to refine the list of names or abbreviations.
Starting with the {\it names} view, enter {\tt m*} into the input area
and click on {\bf filter}.
You then get a shorter table with only the names beginning with {\tt m}.

\subsubsection{documentation}

This gives you documentation for each of the constructors.

\subsubsection{conditions}

This page organizes the constructors according to predicates.
The view is not available for your example page since all constructors
are unconditional.
For a table with conditions, return to the {\bf Cross Reference} page
for \spadtype{Matrix}, click on {\bf Ancestors}, then on {\bf
conditions} in the view panel.
This page shows you that \spadtype{CoercibleTo(OutputForm)} and
\spadtype{SetCategory} are ancestors of \spadtype{Matrix(R)} only if {\tt R}
belongs to category \spadtype{SetCategory}.

%*********************************************************************
\head{subsection}{Giving Parameters to Constructors}{ugBrowseGivingParameters}
%*********************************************************************

Notice the input area at the bottom of the constructor page.
If you leave this blank, then the information you get is for the
domain constructor \spadtype{Matrix(R)}, that is, \spadtype{Matrix} for an
arbitrary underlying domain {\tt R}.

In general, however, the exports and other information {\it do} usually
depend on the actual value of {\tt R}.
For example, \spadtype{Matrix} exports the \spadfun{inverse} operation
only if the domain {\tt R} is a \spadtype{Field}.
To see this, try this from the main constructor page:

\begin{enumerate}
\item Enter {\tt Integer} into the input area at the bottom of the page.

\item Click on {\bf Operations}, producing a table of operations.
Note the number of operation names that appear at the top of the
page.

\item Click on \UpBitmap{} to return to the constructor page.

\item Use the
\fbox{\bf Delete}
or
\fbox{\bf Backspace}
keys to erase {\tt Integer} from the input area.

\item Click on {\bf Operations} to produce a new table of operations.
Look at the number of operations you get.
This number is greater than what you had before.
Find, for example, the operation \spadfun{inverse}.

\item Click on {\bf inverse} to produce a page describing the operation
\spadfun{inverse}.
At the bottom of the description, you notice that the {\bf
Conditions} line says ``{\tt R} has \spadtype{Field}.''
This operation is {\it not} exported by \spadtype{Matrix(Integer)} since
\spadtype{Integer} is not a \spadgloss{field}.

Try putting the name of a domain such as \spadtype{Fraction Integer}
(which is a field) into the input area, then clicking on {\bf Operations}.
As you see, the operation \spadfun{inverse} is exported.
\end{enumerate}

% *********************************************************************
\head{section}{Miscellaneous Features of Browse}{ugBrowseMiscellaneousFeatures}
% *********************************************************************
% *********************************************************************
\head{subsection}{The Description Page for Operations}{ugBrowseDescriptionPage}
% *********************************************************************
From the constructor page of \spadtype{Matrix},
click on {\bf Operations} to bring up the table of operations
for \spadtype{Matrix}.

Find the operation {\bf inverse} in the table and click on it.
This takes you to a page showing the documentation for this operation.

\begin{figure}[htbp]
\begin{picture}(324,180)%(-54,0)
\special{psfile=h-matinv.ps}
\end{picture}
\caption{Operation \protect\spadfunFrom{inverse}{Matrix} from \protect\spadtype{Matrix}.}
\end{figure}

Here is the significance of the headings you see.

\subsubsection{Arguments}

This lists each of the arguments of the operation in turn, paraphrasing
the \spadgloss{signature} of the operation.
As for signatures, a \spadSyntax{%} is used to designate {\em this domain},
that is, \spadtype{Matrix(R)}.

\subsubsection{Returns}

This describes the return value for the operation, analogous to the {\bf
Arguments} part.

\subsubsection{Origin}

This tells you which domain or category explicitly exports the
operation.
In this example, the domain itself is the {\it Origin}.


\subsubsection{Conditions}

This tells you that the operation is exported by \spadtype{Matrix(R)} only if
``{\tt R} has \spadtype{Field},'' that is, ``{\tt R} is a member of
category \spadtype{Field}.''
When no {\bf Conditions} part is given, the operation is exported for
all values of {\tt R}.

\subsubsection{Description}

Here are the \spadSyntax{++} comments
that appear in the source code of its {\it Origin}, here \spadtype{Matrix}.
You find these comments in the source code for \spadtype{Matrix}.

\begin{figure}[htbp]
\begin{picture}(324,180)%(-54,0)
\special{psfile=h-matmap.ps}
\end{picture}
\caption{Operations \protect\spadfun{map} from \protect\spadtype{Matrix}.}
\end{figure}

Click on \UpBitmap{} to return to the table of operations.
Click on {\bf map}.
Here you find three different operations named \spadfun{map}.
This should not surprise you.
Operations are identified by name and \spadgloss{signature}.
There are three operations named \spadfun{map}, each with
different signatures.
What you see is the {\it descriptions} view of the operations.
If you like, select the button in the heading of one of these
descriptions to get {\it only} that operation.

\subsubsection{Where}

This part qualifies domain parameters mentioned in the arguments to the
operation.

% *********************************************************************
\head{subsection}{Views of Operations}{ugBrowseViewsOfOperations}
% *********************************************************************

We suggest that you go to the constructor page for \spadtype{Matrix}
and click on {\bf Operations} to bring up a table of operations
with a {\it Views} panel at the bottom.

\subsubsection{names}

This view lists the names of the operations.
Unlike constructors, however, there may be several operations with the
same name.
The heading for the page tells you the number of unique names and the
number of distinct operations when these numbers are different.

\subsubsection{filter}

As for constructors, you can use this button to cut down the list of
operations you are looking at.
Enter, for example, {\tt m*} into the input area to the right of {\bf
filter} then click on {\bf filter}.
As usual, any logical expression is permitted.
For example, use
\begin{verbatim}
*! or *?
\end{verbatim}
to get a list of destructive operations and predicates.

\subsubsection{documentation}

This gives you the most information:
a detailed description of all the operations in the form you have seen
before.
Every other button summarizes these operations in some form.

\subsubsection{signatures}

This views the operations by showing their signatures.

\subsubsection{parameters}

This views the operations by their distinct syntactic forms with
parameters.

\subsubsection{origins}

This organizes the operations according to the constructor that
explicitly exports them.

\subsubsection{conditions}

This view organizes the operations into conditional and unconditional
operations.

\subsubsection{usage}

This button is only available if your user-level is set to {\it
\index{user-level}
development}.
The {\bf usage} button produces a table of constructors that reference this
operation.\footnote{\Language{} requires an especially long time to
produce this table, so anticipate this when requesting this
information.}

\subsubsection{implementation}

This button is only available if your user-level is set to {\it
development}.
\index{user-level}
If you enter values for all domain parameters on the constructor page,
then the {\bf implementation} button appears in place of the {\bf
conditions} button.
This button tells you what domains or packages actually implement the
various operations.\footnote{This button often takes a long time; expect
a delay while you wait for an answer.}

With your user-level set to {\it development}, we suggest you try this
exercise.
Return to the main constructor page for \spadtype{Matrix}, then enter
{\tt Integer} into the input area at the bottom as the value of {\tt R}.
Then click on {\bf Operations} to produce a table of operations.
Note that the {\bf conditions} part of the {\it Views} table is
replaced by {\bf implementation}.
Click on {\bf implementation}.
After some delay, you get a page describing what implements each of
the matrix operations, organized by the various domains and packages.

\begin{figure}[htbp]
\begin{picture}(324,180)%(-54,0)
\special{psfile=h-matimp.ps}
\end{picture}
\caption{Implementation domains for \protect\spadtype{Matrix}.}
\end{figure}

\subsubsection{generalize}

This button only appears for an operation page of a constructor
involving a unique operation name.

From an operations page for \spadtype{Matrix}, select any
operation name, say {\bf rank}.
In the views panel, the {\bf filter} button is  replaced by
{\bf generalize}.
Click on it!
%% Replaced {\bf threshold} with 10 below.  MGR 1995oct31
What you get is a description of all \Language{} operations
named \spadfun{rank}.\footnote{If there were more than 10
operations of the name, you get instead a page
with a {\it Views} panel at the bottom and the message to {\bf
Select a view below}.
To get the descriptions of all these operations as mentioned
above, select the {\bf description} button.}
%See the discussion of {\bf threshold} in
%\spadref{ugBrowseOptions}.} %% Removed MGR 1995oct31

\begin{figure}[htbp]
\begin{picture}(324,180)%(-54,0)
\special{psfile=h-allrank.ps}
\end{picture}
\caption{All operations named \protect\spadfun{rank} in \Language{}.}
\end{figure}

\subsubsection{all domains}

This button only appears on an operation page resulting from a
search from the front page of \Browse{} or from selecting
{\bf generalize} from an operation page for a constructor.

Note that the {\bf filter} button in the {\it Views} panel is
replaced by {\bf all domains}.
Click on it to produce a table of {\it all} domains or packages that
export a \spadfun{rank} operation.

\begin{figure}[htbp]
\begin{picture}(324,180)%(-54,0)
\special{psfile=h-alldoms.ps}
\end{picture}
\caption{Table of all domains that export \spadfun{rank}.}
\end{figure}

We note that this table specifically refers to all the \spadfun{rank}
operations shown in the preceding page.
Return to the descriptions of all the \spadfun{rank} operations and
select one of them by clicking on the button in its heading.
Select {\bf all domains}.
As you see, you have a smaller table of constructors.
When there is only one constructor, you get the
constructor page for that constructor.
\newpage

% *********************************************************************
\head{subsection}{Capitalization Convention}{ugBrowseCapitalizationConvention}
% *********************************************************************

When entering search keys for constructors, you can use capital
letters to search for abbreviations.
For example, enter {\tt UTS} into the input area and click on {\bf
Constructors}.
Up comes a page describing \spadtype{UnivariateTaylorSeries}
whose abbreviation is \spadtype{UTS}.

Constructor abbreviations always have three or more capital
letters.
For short constructor names (six letters or less), abbreviations
are not generally helpful as their abbreviation is typically the
constructor name in capitals.
For example, the abbreviation for \spadtype{Matrix} is
\spadtype{MATRIX}.

Abbreviations can also contain numbers.
For example, \spadtype{POLY2} is the abbreviation for constructor
\spadtype{PolynomialFunctions2}.
For default packages, the abbreviation is the same as the
abbreviation for the corresponding category with the ``\&''
replaced by ``-''.
For example, for the category default package
\aliascon{MatrixCategory&}{MATCAT-} the abbreviation is
\spadtype{MATCAT-} since the corresponding category
\spadtype{MatrixCategory} has abbreviation \spadtype{MATCAT}.

%% *********************************************************************
%\head{subsection}{Browse Options}{ugBrowseOptions}
%% *********************************************************************
%
%You can set two options for using \Browse{}: exposure and threshold.
%
%% *********************************************************************
%\subsubsection{Exposure}
%% *********************************************************************
%
%By default, the only constructors, operations, and attributes
%shown by \Browse{} are those from \spadglossSee{exposed constructors}{expose}.
%To change this, you can issue
%\syscmdindex{set hyperdoc browse exposure}
%\begin{verbatim}
%)set hyperdoc browse exposure on
%\end{verbatim}
%After you make this setting, you will see
%both exposed and unexposed constructs.
%By definition, an operation or attribute is exposed only if it is
%exported from an exposed constructor.
%Unexposed items are generally marked by \Browse{} with an asterisk.
%For more information on exposure, see \spadref{ugTypesExpose}.
%
%With this setting, try the following experiment.
%Starting with the main \Browse{} page, enter {\tt *matrix*} into the
%input area and click on {\bf Constructors}.
%The result is the following table. %% This line  should be texonly. MGR
%
%\begin{texonly}
%\begin{figure}[htbp]
%\begin{picture}(324,180)%(-54,0)
%\hspace*{\baseLeftSkip}\special{psfile=h-consearch2.ps}
%\end{picture}
%\caption{Table of all constructors matching {\tt *matrix*} .}
%\end{figure}
%\end{texonly}
%
%
%% *********************************************************************
%\subsubsection{Threshold}
%% *********************************************************************
%
%For General, Documentation or Complete searches, a summary is presented
%of all matches.
%When the number of items of a given kind is less than a number called
%{\bf threshold}, \Language{} presents a table of names with the heading
%for that kind.
%
%Also, when an operation name is chosen and there are less than {\bf
%threshold} distinct operations, the operations are initially shown in
%{\bf description} mode.
%
%The default value of {\bf threshold} is 10.
%To change its value to say 5, issue
%\syscmdindex{set hyperdoc browse threshold}
%\begin{verbatim}
%)set hyperdoc browse threshold 5
%\end{verbatim}
%Notice that the headings in
%the summary are active.
%If you click on a heading, you bring up a separate page for those
%entries.
%%
%% Above section removed by MGR, 1995oct30, as these two options do
%% not exist.
\begin{SysCmdOutput}
\end{SysCmdOutput}

% !! DO NOT MODIFY THIS FILE BY HAND !! Created by spool2tex.awk.
% Copyright (c) 1991-2002, The Numerical ALgorithms Group Ltd.
% All rights reserved.
%
% Redistribution and use in source and binary forms, with or without
% modification, are permitted provided that the following conditions are
% met:
%
%     - Redistributions of source code must retain the above copyright
%       notice, this list of conditions and the following disclaimer.
%
%     - Redistributions in binary form must reproduce the above copyright
%       notice, this list of conditions and the following disclaimer in
%       the documentation and/or other materials provided with the
%       distribution.
%
%     - Neither the name of The Numerical ALgorithms Group Ltd. nor the
%       names of its contributors may be used to endorse or promote products
%       derived from this software without specific prior written permission.
%
% THIS SOFTWARE IS PROVIDED BY THE COPYRIGHT HOLDERS AND CONTRIBUTORS "AS
% IS" AND ANY EXPRESS OR IMPLIED WARRANTIES, INCLUDING, BUT NOT LIMITED
% TO, THE IMPLIED WARRANTIES OF MERCHANTABILITY AND FITNESS FOR A
% PARTICULAR PURPOSE ARE DISCLAIMED. IN NO EVENT SHALL THE COPYRIGHT OWNER
% OR CONTRIBUTORS BE LIABLE FOR ANY DIRECT, INDIRECT, INCIDENTAL, SPECIAL,
% EXEMPLARY, OR CONSEQUENTIAL DAMAGES (INCLUDING, BUT NOT LIMITED TO,
% PROCUREMENT OF SUBSTITUTE GOODS OR SERVICES-- LOSS OF USE, DATA, OR
% PROFITS-- OR BUSINESS INTERRUPTION) HOWEVER CAUSED AND ON ANY THEORY OF
% LIABILITY, WHETHER IN CONTRACT, STRICT LIABILITY, OR TORT (INCLUDING
% NEGLIGENCE OR OTHERWISE) ARISING IN ANY WAY OUT OF THE USE OF THIS
% SOFTWARE, EVEN IF ADVISED OF THE POSSIBILITY OF SUCH DAMAGE.



% *********************************************************************
\head{chapter}{What's New in FriCAS}{ugWhatsNew}
% *********************************************************************

%------------------------------------------------------------------
\head{section}{Release Notes}{releaseNotes}
%------------------------------------------------------------------
\Language{} information can be found online at
{http://fricas.sourceforge.net}
%------------------------------------------------------------------
\subsubsection{FriCAS 1.3.1}
%------------------------------------------------------------------
\begin{itemize}

\item Categories with associative multiplication are now subcategories
      of categories with nonassociative multiplication.

\item Inlining optimization in now effective also in command line
      (interpreter) compiler.

\item Added conversions between finitely presented groups
      and permutation groups (Todd-Coxeter algorithm) and
      back.

\item Removed special handling of coercion of String to OutputForm
      from Spad compiler.

\item Former FramedModule is renamed to FractionalIdealAsModule.
      Added new FramedModule.

\item Whole interpreter is now included in executable (no need
      to load parts before use).

\end{itemize}

Bug fixes, in particular:

\begin{itemize}

\item Fixed build with sbcl-1.3.13.

\item Limits using the name of variable in limit point work
      now.

\item A few output fixes.

\item Several integrator fixes.

\item Removed wrond interpreter transform of '~='.

\item Fixed compilation of type parameters containing non-type values.

\item Plots sometimes used single precision.  Now they should
      always use double precision.

\end{itemize}
%------------------------------------------------------------------
\subsubsection{FriCAS 1.3.0}
%------------------------------------------------------------------
\begin{itemize}

\item Several domains and categories are more general,
      in particular matrices, indexed products and
      direct product.

\item ')show' now evaluates predicates.

\item Improved integrator, handles few more 'erf' cases and
      more algebraic functions.  Result should be
      simpler.

\item Added support for using FriCAS as ECL shared library.

\item Polynomial factorization uses Kaltofen-Shoup method when
      applicable.

\item '\$createLocalLibDb' defaults to false.

\item Simpler, more predictable equality for algebraic
      numbers (no longer uses 'trueEqual').

\item Renamed LinearlyExplicitRingOver to LinearlyExplicitOver.

\item Renamed 'length' in Tuple to '\#'.

\item Removed argumentless 'random'.

\end{itemize}

Bug fixes, in particular:

\begin{itemize}

\item Fixed several build problems.

\item Handle scripted symbols in DeRhamComplex.

\item Handle empty matrices in more places.

\item Fixed unparse of negative integers.

\item No longer crashes on quoted expressions in types.

\end{itemize}

%------------------------------------------------------------------
\subsubsection{FriCAS 1.2.7}
%------------------------------------------------------------------
\begin{itemize}

\item New package implementing van Hoej factorization algorithm
      for LODO-s.

\item Gcd over Expression(Integer) now uses modular method.

\item Improvements to integrator, in particular trigonometric
      functions are consistently integrated via transformation
      to complex exponentials.

\item Some categories and domains are more general.  In particular
      OrderedFreeMonoid is removed, as ordered case is handled
      by FreeMonoid.

\item Category Monad in renamed to Magma.  Domain Magma is
      renamed to FreeMagma.

\end{itemize}

Bug fixes, in particular:

\begin{itemize}

\item Coercion of square matrices to polynomials is fixed.

\item Problem with division by 0 in derivative of 'ellipticPi'
      is fixed.

\item Division in Ore algebras used to cause infinite loop
      when coefficients were power series.

\end{itemize}

%------------------------------------------------------------------
\subsubsection{FriCAS 1.2.6}
%------------------------------------------------------------------
\begin{itemize}

\item Polynomial factorization is available for larger class
      of base rings.

\item Improvements to integrator.

\item 'normalize' can be applied to list of expressions.

\item Eigenvalues can be computed over larger range of base fields.

\item Common denominator package handles now multivariate polynomials.

\item More uniform break (error) handling.

\end{itemize}

Bug fixes, in particular:

\begin{itemize}

\item 'distribute' handles 'box' operator.

\item Fixed problem with guessing over multivariate polynomials.

\item Fixed hashcode handling for Void in Aldor.

\end{itemize}

%------------------------------------------------------------------
\subsubsection{FriCAS 1.2.5}
%------------------------------------------------------------------
\begin{itemize}

\item Several improvements to integrator.

\item Improvements to handling of series, in particular new
      function 'prodiag' to compute infinite products, 'series' and
      'coefficients' for multivariate Taylor series, new 'laurent'
      function which builds Laurent series from order and stream of
      coefficients.

\item GMP should now work with sbcl on all platforms and with Closure CL
      on all platforms except for Power PC.

\item Added a few domains for discrete groups.

\item Extended GCD in Ore algebras can now return coefficients of
      both GCD and LCM.

\item New function for computing integrals of solutions of linear
      differential operators.

\item ')savesystem' command is now removed.

\item Continuation lines which begins like commands are no longer
      treated as commands.

\end{itemize}

Bug fixes, in particular:

\begin{itemize}

\item Fixed printing of scripted symbols.

\item Fixed 'totalDegreeSorted' (affected Groebner bases).

\item Fixed few problems with Hensel lifting (including SF bug 47).

\item Fixed 'series' in UnivariateLaurentSeriesConstructor.

\item Fixed 'order' in SparseUnivariatePowerSeries.

\item Printing of series now respect 'showall' setting, cyclic
      series are detected.

\item Fixed problem with interpreter preferring Union to base type.

\end{itemize}

%------------------------------------------------------------------
\subsubsection{FriCAS 1.2.4}
%------------------------------------------------------------------
\begin{itemize}

\item New cylindrical decomposition package.

\item New GnuDraw package for plotting via gnuplot.

\item Texmacs interface now handles Cork symbols.

\item Added double precision versions of several special functions
      (needed for plotting).

\item Nopile mode for Spad is changed to be more convenient.

\item 'stringMatch' is removed (was broken beyond repair).

\end{itemize}

Bug fixes, in particular:

\begin{itemize}

\item Fixed interpreter assignment to parts of nested aggregates
      (issue 376).

\item Fixed interpreter coercion from Equation to Boolean (issue 359).

\item Fix printing of '\%i' in types (issue 132).

\item Disabled incorrect shortcut during coercion (issue 29).

\item Difference of intervals now agrees with definition as interval
      operation.

\item Avoid overwriting loop limit and increment.

\item Fix a polynomial gcd failure due to bad reduction.

\item Avoid mangling unevaluated algebraic integrals.

\item Fix integration of unevaluated derivatives.

\item Restore parser handling of '\\/' and '/\\'.

\item Properly escape strings and symbols in TeXFormat.

\item Fix toplevel multiparameter macros.

\item Fix problem with missing parentheses around plexes.

\item Avoid crash when printing error message from '-eval'.

\item Redirect I/O when running programs from Closure CL.

\end{itemize}

%------------------------------------------------------------------
\subsubsection{FriCAS 1.2.3}
%------------------------------------------------------------------
\begin{itemize}

\item Improved integration in terms of 'Ei' and 'erf'.

\item Classical orthogonal polynomials may be used as expressions.

\item More cases of generalized indexing for two dimensional arrays.

\item Value of 'lambertW' at '-1/e' is now simplified.

\item FriCAS now knows that formal derivatives are commutative.

\item 'setelt' is renamed to 'setelt!'.

\item ')read' now creates intermediate files in current directory.

\item Continuation characters in comments are now respected.

\item In Spad '\$Lisp' calls now must have a type.

\item In Spad \spad{error} did only minimal checking of its argument.
  Now argument to \spad{error} must be a \spad{String} or \spad{OutputForm}
  or a literal list of \spad{OutputForm}-s.

\end{itemize}

Bug fixes, in particular:

\begin{itemize}

\item Input lines with empty continuation are no longer lost.

\item Types like "failed" now consistently use string quotes in output form.

\item Fixed pattern matching using \spad{%i} in patterns.

\item Fixed ')display op coerce'.

\item Fixed ')version' command.

\item Fixed crash when printing '\%'.

\item Fix a buffer overflow in HyperDoc.

\item Fixed HyperDoc errors in 'Dependants' and 'Users'.

\item HyperDoc browser better handles constructors with parameters.

\end{itemize}

%------------------------------------------------------------------
\subsubsection{FriCAS 1.2.2}
%------------------------------------------------------------------
\begin{itemize}
\item Improvements to 'integrate': better handling of algebraic
  integrals, new routine which handles some integrals containing
  'lambertW'.

\item Improvements to 'limit', now Gruntz algorithm knows about
  a few tractable functions.

\item Smith form of sparse integer matrices is now much more
  efficient.

\item Generalized indexing for two dimensional arrays.

\item Pile/nopile mode is now restored after ')read' or ')compile'.
  Piling rules now accept some forms of multiline lists.

\item Eliminated version checking in generated code. Note: this
  change means that Spad code compiled by earlier FriCAS versions
  will not run in FriCAS 1.2.2.

\item Updated Aldor interface to work with free Aldor.
\end{itemize}

Bug fixes, in particular:

\begin{itemize}
\item Interpreter can now handle complicated mutually recursive
  functions.

\item Spad compiler should now correctly handle 'has' inside a function.

\item Fixed derivatives of Whittaker functions.
\end{itemize}

%------------------------------------------------------------------
\subsubsection{FriCAS 1.2.1}
%------------------------------------------------------------------
\begin{itemize}
\item Improvements to 'integrate': a new routine for integration in
  terms of Ei, better handling of algebraic integrals.

\item Implemented 'erfi'.

\item Derivatives of 'asec', 'asech', 'acsc' and 'acsch' use different
  formula so that numeric evaluation of derivative will take correct
  branch on real axis.

\item Linear dependence package is changed to be consistent with
  linear solvers.

\item It is now possible to extract empty submatrices.

\item Changed default style of 3D graphics.

\item Support for building Mac OS application bundle.
\end{itemize}

Bug fixes, in particular:

\begin{itemize}
\item fixed few cases of wrong or unevaluated integrals.

\item better zero test during limit computation avoids division by
  zero.

\item fixed buffer overflow problems in view3D.

\item 'reducedSystem' on empty input returns basis of correct size.
\end{itemize}

%------------------------------------------------------------------
\subsubsection{FriCAS 1.2.0}
%------------------------------------------------------------------
\begin{itemize}
\item New MatrixManipulation package.

\item New ParallelIntegrationTools package.

\item Gruntz algorithm is now used also for finite one-sided limits.

\item FriCAS has now true 2-dimensional arrays (previously they were
emulated using vectors of vectors).

\item Speedups in some matrix operations and in arithmetic with
algebraic expressions.

\item FreeModule is now more general, it allows Comparable as second
argument.

\item Changed Spad parser, it now uses common scanner with
interpreter. Spad language is now closer to interpreter language and
Aldor. 'leave' is removed, 'free', 'generate' and 'goto' are now
keywords. Pile rules changed slightly, they should be more intuitive
now. Error messages from Spad parser should be slightly better.
\end{itemize}

Bug fixes, in particular:

\begin{itemize}
\item Fixed a few build problems.

\item Eliminated division by 0 during 'normalize'.

\item 'nthRootIfCan' removes leading zeros from generalized series
  (this avoids problems with power series expanders).

\item Fixed corruption of formal derivatives.

\item Fixed two problems with Fortran output.

\item Fixed ')untrace' and ')undo'. Fixed ')trace' with ECL.

\item Fixed problem with calling efricas if user's default shell is
  (t)csh.
\end{itemize}

%------------------------------------------------------------------
\subsubsection{FriCAS 1.1.8}
%------------------------------------------------------------------

\begin{itemize}
\item Improvements of pattern matching integrator, it can now
  integrate in terms of Fresnel integrals and better handles integrals
  in terms of Si and Ci.

\item Better integration of symbolic derivatives.

\item Better normalization of Liouvillian functions.

\item New package for computing limits using Gruntz algorithm.

\item Faster removal of roots from denominators.

\item New domains for multivariate Ore algebras and partial
  differential operators.

\item New package for noncommutative Groebner bases.

\item New domain for univariate power series with arbitrary exponents.

\item New special functions: Shi and Chi.

\item Several aggregates (in particular tables) allow more general
  parameter types.

\item New domain for hash tables using equality from underlying
  domain.
\end{itemize}

Bug fixes, in particular:

\begin{itemize}
\item Fixed problem with gcd failing due to bad reduction.

\item Fixed series of 'acot' and Puiseux series of several special
  functions.

\item Fixed wrong factorization of differential operators.

\item Fixed build problem on recent Mac OS X.
\end{itemize}

%------------------------------------------------------------------
\subsubsection{FriCAS 1.1.7}
%------------------------------------------------------------------

\begin{itemize}
\item Improved integration in terms of special functions.

\item Updated new graphics framework and graph theory package.

\item Added routines for numerical evaluation of several special
  functions.

\item Added modular method for computing polynomial gcd over algebraic
  extensions.

\item Derivatives of fresnelC and fresnelS are changed to agree with
  established convention.

\item When printing floats groups of digits are now separated by
  underscores (previously were separated by spaces).

\item Added C code for removing directories, this speeds up full build
  and should avoid build problems on Mac OSX.
\end{itemize}

Bug fixes, in particular:

\begin{itemize}
\item Series expansion now handle poles of Gamma.

\item Fixed derivatives of meijerG.
\end{itemize}

%------------------------------------------------------------------
\subsubsection{FriCAS 1.1.6}
%------------------------------------------------------------------

\begin{itemize}
\item Added experimental graph theory package.

\item Added power series expanders for Weierstrass elliptic functions
  at 0.

\item New functions: kroneckerProduct and kroneckerSum for matrices,
  numeric weierstrassInvariants and modularInvariantJ, symbolic Jacobi
  Zeta, double float numeric elliptic integrals.

\item New domains for vectors and matrices of unsigned 8 and 16 bit
  integers.

\item Changes to Spad compiler: underscores which are not needed as
  escape are now significant in Spad names and strings, macros with
  parameters are supported, added partial support for exceptions,
  braces can be used for grouping.

\item A few speedups.

\item Reduced disc space usage during build.
\end{itemize}

Bug fixes, in particular:

\begin{itemize}
\item Fixed eval of hypergeometricF at 0

\item Fixed problem with scope of macros.

\item Worked around problems with opening named pipes in several Lisp
  implementations.

\item Fixed a problem with searching documentation via HyperDoc.

\item Fixed build problem on Mac OSX.
\end{itemize}

%------------------------------------------------------------------
\subsubsection{FriCAS 1.1.5}
%------------------------------------------------------------------

\begin{itemize}
\item Added numeric version of lambertW.

\item New function 'rootFactor' which tries to write roots of products
  as products of roots.

\item 'try', 'catch' and 'finally' are now Spad keywords.

\item Experimental support for using gmp with Closure CL (64-bit
  Intel/Amd only).

\item New categories CoercibleFrom and ConvertibleFrom. New domain for
  ordinals up to epsilon0. New domain for matrices of machine
  integers. New package for solving linear equations written as
  expressions (faster then general expression solver).

\item Functions exported by Product() are now called 'construct',
  'first' and 'second' (instead of 'makeprod', 'selectfirst' and
  'selectsecond' respectively).

\item Some functions are now much faster, in particular bivariate
  factorization over small finite fields.

\item When using sbcl FriCAS now tries to preload statistical
  profiler.
\end{itemize}

Bug fixes, in particular:

\begin{itemize}
\item Fixed handling of Control-C in FriCAS compiled by recent sbcl.

\item Fixed HyperDoc crash due to bad handling of '\#'.

\item Fixed power series expanders for elliptic integrals.

\item Fixed 'possible wild ramification' problem with algebraic
  integrals.

\item 'has' in interpreter now correctly handles \spad{%}.

\item Spad compiler can now handle single \spad{=>} at top level of a
  function.

\item Fixed few problems with conditional types in Spad compiler.
\end{itemize}

%------------------------------------------------------------------
\subsubsection{FriCAS 1.1.4}
%------------------------------------------------------------------

\begin{itemize}
\item New domains for combinatorial probability theory by Franz
  Lehner.

\item Improved integration of algebraic functions.

\item Initial support for semirings.

\item Updated framework for theory of computations.

\item In Spad parser \spad{**, ^'} and \spad{->} are now
  right-associative.

\item Spad parser no longer transforms relational operators.

\item Join of categories is faster which speeds up Spad compiler.
\end{itemize}

Bug fixes, in particular:

\begin{itemize}
\item Retraction of 'rootOf' from Expression(Integer) to
  AlgebraicNumber works now.

\item Attempt to print error message about invalid type no longer
  crash (SF 2977357).

\item Fixed few problems in Spad compiler dealing with conditional
  exports.

\item HyperDoc now should find all function descriptions (previously
  it missed several).
\end{itemize}

%------------------------------------------------------------------
\subsubsection{FriCAS 1.1.3}
%------------------------------------------------------------------

\begin{itemize}
\item Added "jet bundle" framework by Werner Seiler and Joachim Schue,
  which includes completion procedure and symmetry analysis for PDE.

\item Better splitting of group representations (added Holt-Rees
  improvement to meatAxe).

\item Added numeric versions of some elliptic integrals and few more
  elliptic functions.

\item Speeded up FFCGP (finite fields via Zech logarithms).

\item New experimental flag (off by default, set via
  setSimplifyDenomsFlag) which if on causes removal of irrationalities
  from denominators. Usually it causes slowdown, but on some examples
  gives huge speedup. It may go away in future (when no longer
  needed).

\item Added experimental framework for theory of computations.
\end{itemize}

Bug fixes, in particular:

\begin{itemize}
\item Numerical solutions of polynomial systems have now required
  accuracy (SF 2418832).

\item Fixed problem with crashes during tracing.

\item Fixed a problem with nested iteration (SF 3016806).

\item Eliminated stack overflow when concatenating long lists.
\end{itemize}

%------------------------------------------------------------------
\subsubsection{FriCAS 1.1.2}
%------------------------------------------------------------------

\begin{itemize}
\item Experimental Texmacs interface and Texmacs format output.

\item Guessing package can now guess algebraic dependencies.

\item Expansion into Taylor series and limits now work for most
  special functions.

\item Spad to Aldor translator is removed.

\item Spad compiler no longer allows to denote sets using braces.
\end{itemize}

Bug fixes, in particular:

\begin{itemize}
\item Fixed few cases where elementary integrals were returned
  unevaluated or produced wrong results.

\item Unwanted numerical evaluation should be no longer a problem
  (FriCAS interpreter now very strongly prefers symbolic evaluation
  over numerical evaluation).

\item Fixed a truncation bug in guessing package which caused loss of
  some correct solutions.

\item TeX and MathML format should correctly put parentheses around
  and inside sums and products.

\item Fixed few problems with handling of Unicode.
\end{itemize}

%------------------------------------------------------------------
\subsubsection{FriCAS 1.1.1}
%------------------------------------------------------------------

\begin{itemize}
\item New graphics framework.

\item Support for using GMP with sbcl on 32/64 bit AMD/Intel
  processors (to activate it one must use '--with-gmp' option to
  configure).

\item Improvements to integration and normalization. In particular
  integrals containing multiple non-nested roots should now work much
  faster. Also FriCAS now can compute more integrals of Liouvillian
  functions.

\item Several new special functions.

\item Improvements to efricas.

\item Looking for default init file FriCAS now first tries to use
  '.fricas.input' and only if that fails it looks for '.axiom.input'.
\end{itemize}

Bug fixes, in particular:

\begin{itemize}
\item Numeric atan, asin and acos took wrong branch.

\item WeierstrassPreparation package did not work.

\item Saving and restoring history should be now more reliable.

\item Fixed two bugs in Spad compiler related to conditional
  compilation.

\item Fixed a problem with rational reconstruction which affected
  guessing package.
\end{itemize}

%------------------------------------------------------------------
\subsubsection{FriCAS 1.1.0}
%------------------------------------------------------------------

\begin{itemize}
\item New domains and packages: VectorSpaceBasis domain, DirichletRing
  domain, 3D graphic output in Wavefront .obj format, specialized
  machine precision numeric vectors and matrices (faster then general
  vectors and matrices), Html output.

\item Support Clifford algebras corresponding to non-diagonal matrix,
  added new operations.

\item 'normalize' now tries to simplify logarithms of algebraic
  constants.

\item New functions: Fresnel integrals, carmichaelLambda.

\item Speed improvements: several polynomial operations are faster,
  faster multiplication in Ore algebras, faster computation of strong
  generating set for permutation groups, faster coercions.

\item Several improvements to the guessing package (in particular new
  option Somos for restricting attention to Somos-like sequences
\end{itemize}

Bug fixes, in particular:

\begin{itemize}
\item FriCAS can now compute multiplicative inverse of a power series
  with constant term not equal to 1.

\item Fixed a problem with passing interpreter functions to algebra.

\item Two bugs causing crashes in HyperDoc interface are fixed.

\item FriCAS now ignores sign when deciding if number is prime.

\item A failing coercion that used to crash FriCAS is now detected.

\item 'has' test sometimes gave wrong result.

\item Plotting fixes.
\end{itemize}

%------------------------------------------------------------------
\subsubsection{FriCAS 1.0.9}
%------------------------------------------------------------------

\begin{itemize}
\item Speed improvements to polynomial multiplication, power series
  multiplication, guessing package and coercion of polynomials to
  expressions.

\item Domains for tensor products.

\item 'Complex(Integer)' is now UniqueFactorizationDomain.

\item Types in interpreter are now of type 'Type' (instead of
  'Domain') and categories in interpreter are of type 'Category'
  (instead of 'Subdomain(Domain)').

\item Interpreter functions can now return 'Type'.

\item New function for files: 'flush'.

\item Spad compiler: return in nested functions and nested functions
  returning functions.
\end{itemize}

Bug fixes, in particular:

\begin{itemize}
\item Several fixes to guessing package.

\item Avoid crash when unparsing equations.

\item Equation solver accepts more solutions.

\item Fixed handling of 'Tuple' in Spad parser.

\item Fixed miscompilation of record constructor by Spad compiler.
\end{itemize}

%------------------------------------------------------------------
\subsubsection{FriCAS 1.0.8}
%------------------------------------------------------------------

\begin{itemize}
\item Improved version of guessing package. It can now handle much
  larger problems than before. Added ability to guess functional
  substitution equations.

\item Experimental support for build using CMU CL

\item Various speed improvements including faster indexing for two
  dimensional arrays

\item By default FriCAS build tries to use sbcl.

\item Building no longer require patch.
\end{itemize}

Bug fixes, in particular:

\begin{itemize}
\item correct definition of random() for matrices

\item conditionals in .input files work again

\item Spad compiler now recognizes more types as equal

\item fixed problem with pattern-matching quote
\end{itemize}

%------------------------------------------------------------------
\subsubsection{FriCAS 1.0.7}
%------------------------------------------------------------------

\begin{itemize}
\item Comparisons between elements of the Expression domain are
  undefined. Earlier versions gave confusing results for expressions
  like '\%e < \%pi' -- now FriCAS will complain about '<' being
  undefined.

\item A domain for general quaternions was added.

\item Equality in Any is now more reasonable -- it uses equality from
  underlying domain if available.

\item Messages about loading of components are switched off by
  default.

\item Release build benefits from parallel make.

\item In Spad code a single quote now means that the following token
  is a symbol.

\item Reorganization of algebra sources, in particular several types
  have changed (this may affect users Spad code).
\end{itemize}

Bug fixes, in particular:

\begin{itemize}
\item Categories with default package can be used just after
  definition (fixes 1.0.6 regression).

\item Plots involving 0 or 1 work now.

\item Numbers in radix bigger than 10 appear correctly in TeX output.

\item Fixed browser crashes when displaying some domains.

\item Fix horizontal display of fractions.

\item Allow local domains in conditionals (in Spad code).

\item Fixed problem with splitting polynomials and nested extensions.
\end{itemize}

%------------------------------------------------------------------
\subsubsection{FriCAS 1.0.6}
%------------------------------------------------------------------

\begin{itemize}
\item the axiom script is no longer installed (use fricas script
  instead)

\item some undesirable simplification are no longer done by default,
  for example now asin(sin(3)) is left unevaluated

\item support lambda expressions using '+->' syntax and nested
  functions in Spad

\item better configure, support for Dragonfly BSD

\item faster bootstrap, also parallel (this does not affect speed of
  release build)
\end{itemize}

Several bug fixes, in particular:

\begin{itemize}
\item fixed a regression introduced in 1.0.4 which caused equality for
  nested products to sometimes give wrong result

\item corrected fixed output of floating point numbers,

\item operations on differential operators like symmetric power work
  now

\item fixed crashes related to coercing power series

\item functions returning Void can be traced
\end{itemize}

%------------------------------------------------------------------
\subsubsection{FriCAS 1.0.5}
%------------------------------------------------------------------

\begin{itemize}
\item improvement to normalize function, it performs now much stronger
  simplifications than before

\item better integration: due to improved normalize FriCAS can now
  integrate many functions that it previously considered unintegrable

\item improvement to Martin Rubey guessing package, for example it can
  now guess differential equation for the generating function of
  integer partitions

\item better support for using type valued functions

\item several bug fixes
\end{itemize}

%------------------------------------------------------------------
\subsubsection{FriCAS 1.0.4}
%------------------------------------------------------------------

\begin{itemize}
\item significant speedups for some operations (for example definite
  integration)

\item support for building algebra using user-defined optimization
  settings

\item support for mouse wheel in HyperDoc browser

\item included support for interfacing with Aldor

\item new optional Emacs mode and efricas script to run FriCAS inside
  emacs

\item better unparse

\item removed support for attributes (replaced by empty categories)
  and use of colon for type conversions in Spad code

\item a few bug fixes
\end{itemize}

%------------------------------------------------------------------
\subsubsection{FriCAS 1.0.3}
%------------------------------------------------------------------

\begin{itemize}
\item added multiple precision Gamma and logGamma functions

\item better line editing

\item removed some undocumented and confusing constructs from Spad
  language

\item added new categories for semiring and ordered semigroup, direct
  product of monoids is now a monoid

\item internal cleanups and restructurings

\item a few bug fixes
\end{itemize}

%------------------------------------------------------------------
\subsubsection{FriCAS 1.0.2}
%------------------------------------------------------------------

\begin{itemize}
\item ')nopiles' command gives conventional syntax

\item added pfaffian function

\item ECL support

\item Graphics and Hyperdoc work using openmcl or ECL

\item Output may be now delimited by user defined markers

\item Experimental support for using as a Lisp library

\item Spad compiler is now significantly faster

\item Several bug fixes
\end{itemize}

%------------------------------------------------------------------
\subsubsection{FriCAS 1.0.1}
%------------------------------------------------------------------

\begin{itemize}
\item Graphics and Hyperdoc work using sbcl or clisp

\item Builds under Cygwin (using Cygwin clisp)

\item MathML support contributed by Arthur C. Ralfs

\item Help files created by Tim Daly

\item Added SPADEDIT script

\item Full release caches all generated HyperDoc pages

\item Bug fixes, including implementing some missing functions and
  build fixes
\end{itemize}

%------------------------------------------------------------------
\subsubsection{FriCAS 1.0.0}
%------------------------------------------------------------------

The 1.0 release is the first release of FriCAS. Below we list main
differences compared to AXIOM September 2006.

Numerous bug fixes (in particular HyperDoc is now fully functional on
Unix systems).

FriCAS includes guessing package written by Martin Rubey. This package
provides unique ability to guess formulas for sequences of numbers or
polynomials.

Some computation, in particular involving Expression domain, should be
much faster. FriCAS to go trough its testsuite needs only half of the
time needed by AXIOM September 2006.

Spad compilation is faster (in some cases 2 times faster).

FriCAS is much more portable than AXIOM September 2006. It can be
build on Linux, many Unix systems (for example Mac OSX and Solaris 10)
and Windows. It can be build on top of gcl, sbcl, clisp or openmcl
(gcl and sbcl based FriCAS is fully functional, clisp or openmcl based
one lacks graphic support).

Many unused or non-working parts are removed from FriCAS. In
particular FriCAS does not contain support for NAG numerical library.

FriCAS can be build from sources using only a few pre-generated Lisp
files for bootstrap -- only to bootstrap Shoe translator. This means
that modifying FriCAS algebra is now much easier.

%------------------------------------------------------------------------
\head{section}{Changes to Spad language}{ugSpadChanges}
%------------------------------------------------------------------------
\begin{enumerate}
\item \spad{$} as name of current domain is no longer supported, use
  \spad{%} instead.

\item Attributes are no longer supported, use niladic categories with
  no exports instead.

\item Floating point numbers without leading zero are no longer
  supported, so instead of \spad{.01} use \spad{0.01}

\item Anonymous functions using \spad{#1}, \spad{#2}, etc. are no
  longer supported, to define anonymous functions use \textspadexpr{+->}.

\item Braces no longer construct sets. So instead of
  \textspadexpr{\{'sin, 'cos\}::Set(Symbol)} use
  \spad{set(['sin, 'cos])$Set(Symbol)}. %$

\item Old Spad used colon (\spad{:}) to denote conversion, like
  \spad{pretend} but performing even less checking. This is no longer
  supported, use \spad{::} or \spad{pretend} instead.

\item There was an alternative spelling for brackets and braces, in
  FriCAS this is no longer supported, so one has to write brackets and
  braces as is.

\item \spad{SubsetCategory} was handled in special way by the
  compiler. This is no longer supported.

\item Old Spad compiler used to transform relational operators
  \spad{~=,<=,>,>=} in ways which are correct for linear order, but
  may conflict with other uses (as partial order or when generating
  \spad{OutputForm}). FriCAS no longer performs this transformation.
  Similarely, Spad parser no longer treats \spadop{^} and \spadop{^=} in
  special way.

\item Quote in old Spad allowed to insert arbitrary literal Lisp data,
  FriCAS only allows symbols after quote. Code using old behavior
  needs to be rewritten, however it seems that this feature was almost
  unused, so this should be no problem.

\item Old Spad treated statement consisting just of constructor name
  (with arguments if needed) as request to import the constructor.
  FriCAS requires \spad{import} keyword.

\item In FriCAS \spad{**, ^, ->} are right associative. Also, right
  binding power of \textspadexpr{+->} is increased, which allows more natural
  writing of code.

\item Few non-working experimental features are removed, in particular
  partial support for APL-like syntax.

\item FriCAS implemented parametric macros in the Spad compiler.

\item FriCAS allows simplified form for exporting constants (without
  \spad{constant} keyword).

\item FriCAS added partial support for exception handling (currently
  only \spad{finally} part).

\item The \spad{leave} construct is removed from FriCAS. Use
  \spad{break} instead.

\item \spad{div} is no longer a keyword. \spad{free}, \spad{generate},
  \spad{goto} are FriCAS keywords.

\item '\$Lisp' calls now must have a type

\item \spad{error} did only minimal checking of its argument.  Now
  argument to \spad{error} must be a \spad{String} or \spad{OutputForm}
  or a literal list of \spad{OutputForm}-s.

\end{enumerate}

There are also library changes that affect user code:

\begin{enumerate}
\item \spad{**} lost its definition as exponentiation, use \spadop{^}
  instead.

\item \spadop{^} is no longer used as negation (it means exponentiation
  now) and \spadop{^=} no longer means inequality, use \spad{not} and
  \spad{~=} instead.

\item \spad{setelt} is renamed to \spad{setelt!}.

\item Operator properties are now symbols and not strings, so instead
  of \spad{has?(op, "even")} use \spad{has?(op, 'even)}

\item There is new category \spad{Comparable}, several constructors
  that asserted \spad{OrderedSet} now only assert \spad{Comparable}.
\end{enumerate}


%------------------------------------------------------------------
\head{section}{Online Information}{onlineInformation}
%------------------------------------------------------------------
\Language{} information can be found online at
\begin{itemize}
\item {http://fricas.sourceforge.net} -- The official homepage of
  FriCAS.
\item {http://axiom-wiki.newsynthesis.org} -- A wiki site related to
  FriCAS.
\item {http://sourceforge.net/p/fricas/code/HEAD/tree/} -- The
  official source code repository.
\item {https://github.com/fricas/fricas} -- A live git mirror of
  the official SVN repository.
\item {http://fricas.github.io} -- Documentation of FriCAS including
  the API of the FriCAS library.
\end{itemize}

% ------------------------------------------------------------------------
\head{section}{Old News about AXIOM Version 2.x}{ugWhatsNewNAG}
%------------------------------------------------------------------------

Many things have changed in this version of AXIOM and
we describe many of the more important topics here.

%------------------------------------------------------------------------
%\head{section}{The New \Language{} Library Compiler}{ugWhatsNewAsharp}
%------------------------------------------------------------------------
%
% A new compiler is now available for \Language{}.
% The programming language is referred to as the \Language{} Extension Language
% (or \aldor{} for short), and
% improves upon the old \Language{} language in many ways.
% The \spadsys{)compile} command has been upgraded to be able to
% invoke the new or old compilers.
% The language and the compiler are described in the hard-copy
% documentation which came with your \Language{} system.
%
% To ease the chore of upgrading your {\it .spad} files (old
% compiler) to {\it .as} files (new compiler), the
% \spadsys{)compile} command has been given a {\tt )translate}
% option. This invokes a special version of the old compiler which
% parses and analyzes your old code and produces augmented code
% using the new syntax.
% Please be aware that the translation is not necessarily one
% hundred percent complete or correct.
% You should attempt to compile the output with the \aldor{} compiler
% and make any necessary corrections.


% ----------------------------------------------------------------------
\head{subsection}{The NAG Library Link}{nagLinkIntro}
% ----------------------------------------------------------------------

Content removed, NAGLink is no longer included in FriCAS.


%------------------------------------------------------------------------
\head{subsection}{Interactive Front-end and Language}{ugWhatsNewLanguage}
%------------------------------------------------------------------------

The \spad{leave} keyword has been replaced by the
\spad{break} keyword for compatibility with the new AXIOM
extension language.
See section \spadref{ugLangLoopsBreak}
for more information.

Curly braces are no longer used to create sets. Instead, use
\spadfun{set} followed by a bracketed expression. For example,
\begin{xtc}
\begin{xtccomment}
\end{xtccomment}
\begin{spadsrc}
set [1,2,3,4]
\end{spadsrc}
\begin{TeXOutput}
\begin{fricasmath}{1}
\BRACE{1\COMMA 2\COMMA 3\COMMA 4}%
\end{fricasmath}
\end{TeXOutput}
\formatResultType{Set(PositiveInteger)}
\end{xtc}

Curly braces are now used to enclose a block (see section
\spadref{ugLangBlocks}
for more information). For compatibility, a block can still be
enclosed by parentheses as well.

% ``Free functions'' created by the \aldor{} compiler can now be
% loaded and used within the AXIOM interpreter. A {\it free
% function} is a library function that is implemented outside a
% domain or category constructor.

New coercions to and from type \spadtype{Expression} have been
added. For example, it is now possible to map a polynomial
represented as an expression to an appropriate polynomial type.

Various messages have been added or rewritten for clarity.

%------------------------------------------------------------------------
\head{subsection}{Library}{ugWhatsNewLibrary}
%------------------------------------------------------------------------

The \spadtype{FullPartialFractionExpansion}
domain has been added. This domain computes factor-free full
partial fraction expansions.
See section
\xmpref{FullPartialFractionExpansion}
for examples.

We have implemented the Bertrand/Cantor algorithm for integrals of
hyperelliptic functions. This brings a major speedup for some
classes of algebraic integrals.

We have implemented a new (direct) algorithm for integrating trigonometric
functions. This brings a speedup and an improvement in the answer
quality.

The {\sf SmallFloat} domain has been renamed
\spadtype{DoubleFloat} and {\sf SmallInteger} has been renamed
\spadtype{SingleInteger}. The new abbreviations as
\spadtype{DFLOAT} and \spadtype{SINT}, respectively.
We have defined the macro {\sf SF}, the old abbreviation for {\sf
SmallFloat}, to expand to \spadtype{DoubleFloat} and modified
the documentation and input file examples to use the new names
and abbreviations. You should do the same in any private \Language{}
files you have.

We have made improvements to the differential equation solvers
and there is a new facility for solving systems of first-order
linear differential equations.
In particular, an important fix was made to the solver for
inhomogeneous linear ordinary differential equations that
corrected the calculation of particular solutions.
We also made improvements to the polynomial
and transcendental equation solvers including the
ability to solve some classes of systems of transcendental
equations.

The efficiency of power series have been improved and left and right
expansions of \spad{tan(f(x))} at \spad{x =} a pole of \spad{f(x)}
can now be computed.
A number of power series bugs were fixed and the \spadtype{GeneralUnivariatePowerSeries}
domain was added.
The power series variable can appear in the coefficients and when this
happens, you cannot differentiate or integrate the series.  Differentiation
and integration with respect to other variables is supported.

A domain was added for representing asymptotic expansions of a
function at an exponential singularity.

For limits, the main new feature is the exponential expansion domain used
to treat certain exponential singularities.  Previously, such singularities
were treated in an {\it ad hoc} way and only a few cases were covered.  Now
AXIOM can do things like

\begin{verbatim}
limit( (x+1)^(x+1)/x^x - x^x/(x-1)^(x-1), x = %plusInfinity)
\end{verbatim}

in a systematic way.  It only does one level of nesting, though.  In other
words, we can handle \spad{exp(} some function with a pole \spad{)}, but not
\linebreak \spad{exp(exp(} some function with a pole \spad{)).}

The computation of integral bases has been improved through careful
use of Hermite row reduction. A P-adic algorithm
for function fields of algebraic curves in finite characteristic has also
been developed.

Miscellaneous:
There is improved conversion of definite and indefinite integrals to
\spadtype{InputForm};
binomial coefficients are displayed in a new way;
some new simplifications of radicals have been implemented;
the operation \spadfun{complexForm} for converting to rectangular coordinates
has been added;
symmetric product operations have been added to \spadtype{LinearOrdinaryDifferentialOperator}.

%------------------------------------------------------------------------
\head{subsection}{\HyperName}{ugWhatsNewHyperDoc}
%------------------------------------------------------------------------

The buttons on the titlebar and scrollbar have been replaced
with ones which have a 3D effect. You can change the foreground and
background colors of these ``controls'' by including and modifying
the following lines in your {\bf .Xdefaults} file.
\begin{verbatim}
Axiom.hyperdoc.ControlBackground: White
Axiom.hyperdoc.ControlForeground: Black
\end{verbatim}

For various reasons, \HyperName{} sometimes displays a
secondary window. You can control the size and placement of this
window by including and modifying
the following line in your {\bf .Xdefaults} file.
%
\begin{verbatim}
Axiom.hyperdoc.FormGeometry: =950x450+100+0
\end{verbatim}
%
This setting is a standard X Window System geometry specification:
you are requesting a window 950 pixels wide by 450 deep and placed in
the upper left corner.

Some key definitions have been changed to conform more closely
with the CUA guidelines. Press
F9
to see the current definitions.

Input boxes (for example, in the Browser) now accept paste-ins from
the X Window System. Use the second button to paste in something
you have previously copied or cut. An example of how you can use this
is that you can paste the type from an \Language{} computation
into the main Browser input box.


%------------------------------------------------------------------------
\head{subsection}{Documentation}{ugWhatsNewDocumentation}
%------------------------------------------------------------------------
We describe here a few additions to the on-line
version of the AXIOM book which you can read with
HyperDoc.

A section has been added to the graphics chapter, describing
how to build \twodim{} graphs from lists of points. An example is
given showing how to read the points from a file.
See section \spadref{ugGraphTwoDbuild}
for details.

A further section has been added to that same chapter, describing
how to add a \twodim{} graph to a viewport which already
contains other graphs.
See section
\spadref{ugGraphTwoDappend}
for details.

Chapter 3
and the on-line \HyperName{} help have been unified.

An explanation of operation names ending in ``?'' and ``!'' has
been added to the first chapter.
See the
end of the section
\spadref{ugIntroCallFun}
for details.

An expanded explanation of using predicates has
been added to the sixth chapter. See the
example involving \userfun{evenRule} in the middle of the section
\spadref{ugUserRules}
for details.

Documentation for the \spadsys{)compile}, \spadsys{)library} and
\spadsys{)load} commands has been greatly changed. This reflects
the ability of the \spadsys{)compile} to now invoke the \aldor{}
compiler, the impending deletion of the \spadsys{)load} command
and the new \spadsys{)library} command.
The \spadsys{)library} command replaces \spadsys{)load} and is
compatible with the compiled output from both the old and new
compilers.

%------------------------------------------------------------------------
\head{subsection}{\aldor{} compiler - Enhancements and Additions}{ugTwoTwoAldor}
%------------------------------------------------------------------------
Content removed - \aldor{} (now using name {\it Aldor}) is a separate
project.
%
%------------------------------------------------------------------------
\head{subsection}{New polynomial domains and algorithms}{ugTwoTwoPolynomials}
%------------------------------------------------------------------------
Univariate polynomial factorization over the integers has been
enhanced by updates to the \spadtype{GaloisGroupFactorizer} type
and friends from Frederic Lehobey (Frederic.Lehobey@lifl.fr, University of
Lille I, France).

The package constructor \spadtype{PseudoRemainderSequence}
provides efficient algorithms by Lionel Ducos
(Lionel.Ducos@mathlabo.univ-poitiers.fr, University of Poitiers, France)
for computing sub-resultants.
This leads to a speed up in many places in \Language{} where
sub-resultants are computed (polynomial system solving,
algebraic factorization, integration).

Based on this package, the domain constructor
\spadtype{NewSparseUnivariatePolynomial}
extends the constructor \spadtype{SparseUnivariatePolynomial}.
In a similar way, the \spadtype{NewSparseMultivariatePolynomial} extends
the constructor \spadtype{SparseUnivariatePolynomial};
it also provides some additional operations related
to polynomial system solving by means of triangular sets.

Several domain constructors implement
regular triangular sets (or regular chains).
Among them \spadtype{RegularTriangularSet}
and \spadtype{SquareFreeRegularTriangularSet}.
They also implement an algorithm by Marc Moreno Maza (marc@nag.co.uk, NAG)
for computing triangular decompositions of polynomial systems.
This method is refined in the package \spadtype{LazardSetSolvingPackage}
in order to produce decompositions by means of Lazard triangular sets.
For the case of polynomial systems with finitely many solutions,
these decompositions can also be computed by
the package \spadtype{LexTriangularPackage}.

The domain constructor \spadtype{RealClosure} by Renaud Rioboo
(Renaud.Rioboo@lip6.fr, University of Paris 6, France)
provides the real closure of an ordered field.
The implementation is based on interval arithmetic.
Moreover, the design of this constructor and its related
packages allows an easy use of other codings for real algebraic numbers.

Based on triangular decompositions and the \spadtype{RealClosure} constructor,
the package \spadtype{ZeroDimensionalSolvePackage}
provides operations for computing symbolically the real or complex roots
of polynomial systems with finitely many solutions.

Polynomial arithmetic with non-commutative variables
has been improved too by a contribution of Michel Petitot
(Michel.Petitot@lifl.fr, University of Lille I, France).
The domain constructors \spadtype{XRecursivePolynomial}
and \spadtype{XDistributedPolynomial} provide
recursive and distributed representations for these polynomials.
They are the non-commutative equivalents for
the \spadtype{SparseMultivariatePolynomial}
and \spadtype{DistributedMultivariatePolynomial} constructors.
The constructor \spadtype{LiePolynomial} implement Lie
polynomials in the Lyndon basis.
The constructor \spadtype{XPBWPolynomial} manage polynomials
with non-commutative variables in
the Poincar\'e-Birkhoff-Witt basis from the Lyndon basis.
This allows to compute in the Lie Group associated with a
free nilpotent Lie algebra by using the \spadtype{LieExponentials}
domain constructor.
%
%------------------------------------------------------------------------
\head{subsection}{Enhancements to HyperDoc and Graphics}{ugTwoTwoHyperdoc}
%------------------------------------------------------------------------
From this version of AXIOM onwards, the pixmap format used to save graphics
images in color and to display them in HyperDoc has been changed to the
industry-standard XPM format. See {\tt ftp://koala.inria.fr/pub/xpm}.
%
%------------------------------------------------------------------------
\head{subsection}{Enhancements to NAGLink}{ugTwoTwoNAGLink}
%------------------------------------------------------------------------
Content removed -  NAGLink is no longer included in FriCAS.
%
%------------------------------------------------------------------------
\head{subsection}{Enhancements to the Lisp system}{ugTwoTwoCCL}
%------------------------------------------------------------------------
Content removed - no longer relevant since FriCAS runs on different
Lisp systems.

%
\appendix

% !! DO NOT MODIFY THIS FILE BY HAND !! Created by spool2tex.awk.

% Copyright (c) 1991-2002, The Numerical ALgorithms Group Ltd.
% All rights reserved.
%
% Redistribution and use in source and binary forms, with or without
% modification, are permitted provided that the following conditions are
% met:
%
%     - Redistributions of source code must retain the above copyright
%       notice, this list of conditions and the following disclaimer.
%
%     - Redistributions in binary form must reproduce the above copyright
%       notice, this list of conditions and the following disclaimer in
%       the documentation and/or other materials provided with the
%       distribution.
%
%     - Neither the name of The Numerical ALgorithms Group Ltd. nor the
%       names of its contributors may be used to endorse or promote products
%       derived from this software without specific prior written permission.
%
% THIS SOFTWARE IS PROVIDED BY THE COPYRIGHT HOLDERS AND CONTRIBUTORS "AS
% IS" AND ANY EXPRESS OR IMPLIED WARRANTIES, INCLUDING, BUT NOT LIMITED
% TO, THE IMPLIED WARRANTIES OF MERCHANTABILITY AND FITNESS FOR A
% PARTICULAR PURPOSE ARE DISCLAIMED. IN NO EVENT SHALL THE COPYRIGHT OWNER
% OR CONTRIBUTORS BE LIABLE FOR ANY DIRECT, INDIRECT, INCIDENTAL, SPECIAL,
% EXEMPLARY, OR CONSEQUENTIAL DAMAGES (INCLUDING, BUT NOT LIMITED TO,
% PROCUREMENT OF SUBSTITUTE GOODS OR SERVICES-- LOSS OF USE, DATA, OR
% PROFITS-- OR BUSINESS INTERRUPTION) HOWEVER CAUSED AND ON ANY THEORY OF
% LIABILITY, WHETHER IN CONTRACT, STRICT LIABILITY, OR TORT (INCLUDING
% NEGLIGENCE OR OTHERWISE) ARISING IN ANY WAY OUT OF THE USE OF THIS
% SOFTWARE, EVEN IF ADVISED OF THE POSSIBILITY OF SUCH DAMAGE.

% *********************************************************************
\head{chapter}{\Language{} System Commands}{ugSysCmd}
% *********************************************************************

This chapter describes system commands, the command-line
facilities used to control the \Language{} environment.
The first section is an introduction and discusses the common
syntax of the commands available.

% *********************************************************************
\head{section}{Introduction}{ugSysCmdOverview}
% *********************************************************************

System commands are used to perform \Language{} environment
management.
Among the commands are those that display what has been defined or
computed, set up multiple logical \Language{} environments
(frames), clear definitions, read files of expressions and
commands, show what functions are available, and terminate
\Language{}.

Some commands are restricted: the commands
\syscmdindex{set userlevel interpreter}
\syscmdindex{set userlevel compiler}
\syscmdindex{set userlevel development}
\begin{verbatim}
)set userlevel interpreter
)set userlevel compiler
)set userlevel development
\end{verbatim}
set the user-access level to the three possible choices.
All commands are available at {\tt development} level and the fewest
are available at {\tt interpreter} level.
The default user-level is {\tt interpreter}.
\index{user-level}
In addition to the \spadsys{)set} command (discussed in \spadref{ugSysCmdset})
you can use the \HyperName{} settings facility to change the {\it user-level.}


Each command listing begins with one or more syntax pattern descriptions
plus examples of related commands.
The syntax descriptions are intended to be easy to read and do not
necessarily represent the most compact way of specifying all
possible arguments and options; the descriptions may occasionally
be redundant.

All system commands begin with a right parenthesis which should be in
the first available column of the input line (that is, immediately
after the input prompt, if any).
System commands may be issued directly to \Language{} or be
included in {\bf .input} files.
\index{file!input}

A system command {\it argument} is a word that directly
follows the command name and is not followed or preceded by a
right parenthesis.
A system command {\it option} follows the system command and
is directly preceded by a right parenthesis.
Options may have arguments: they directly follow the option.
This example may make it easier to remember what is an option and
what is an argument:

\begin{center}
{\tt )syscmd {\it arg1 arg2} )opt1 {\it opt1arg1 opt1arg2} )opt2 {\it opt2arg1} ...}
\end{center}

In the system command descriptions, optional arguments and options are
enclosed in brackets (``\lanb'' and ``\ranb'').
If an argument or option name is in italics, it is
meant to be a variable and must have some actual value substituted
for it when the system command call is made.
For example, the syntax pattern description

\noindent
{\tt )read} {\it fileName} {\tt \lanb{})quietly\ranb{}}

\noindent
would imply that you must provide an actual file name for
{\it fileName} but need not use the {\tt )quietly} option.
Thus
\begin{verbatim}
)read matrix.input
\end{verbatim}
is a valid instance of the above pattern.

System command names and options may be abbreviated and may be in
upper or lower case.
The case of actual arguments may be significant, depending on the
particular situation (such as in file names).
System command names and options may be abbreviated to the minimum
number of starting letters so that the name or option is unique.
Thus
\begin{verbatim}
)s Integer
\end{verbatim}
is not a valid abbreviation for the {\tt )set} command,
because both {\tt )set} and {\tt )show}
begin with the letter ``s''.
Typically, two or three letters are sufficient for disambiguating names.
In our descriptions of the commands, we have used no abbreviations for
either command names or options.

In some syntax descriptions we use a vertical line ``{\tt |}''
to indicate that you must specify one of the listed choices.
For example, in
\begin{verbatim}
)set output fortran on | off
\end{verbatim}
only {\tt on} and {\tt off} are acceptable words for following
{\tt boot}.
We also sometimes use ``...'' to indicate that additional arguments
or options of the listed form are allowed.
Finally, in the syntax descriptions we may also list the syntax of
related commands.

% *********************************************************************
\head{section}{)abbreviation}{ugSysCmdabbreviation}
% *********************************************************************
\syscmdindex{abbreviation}


\par\noindent{\bf User Level Required:} compiler

\par\noindent{\bf Command Syntax:}
\begin{simpleList}
\item {\tt )abbreviation query  \lanb{}{\it nameOrAbbrev}\ranb{}}
\item {\tt )abbreviation category  {\it abbrev  fullname} \lanb{})quiet\ranb{}}
\item {\tt )abbreviation domain  {\it abbrev  fullname}   \lanb{})quiet\ranb{}}
\item {\tt )abbreviation package  {\it abbrev  fullname}  \lanb{})quiet\ranb{}}
\item {\tt )abbreviation remove  {\it nameOrAbbrev}}
\end{simpleList}

\par\noindent{\bf Command Description:}

This command is used to query, set and remove abbreviations for category,
domain and package constructors.
Every constructor must have a unique abbreviation.
This abbreviation is part of the name of the subdirectory
under which the components of the compiled constructor are
stored.
%% BEGIN OBSOLETE
% It is this abbreviation that is used to bring compiled code into
% \Language{} with the {\tt )load} command.
%% END OBSOLETE
Furthermore, by issuing this command you
let the system know what file to load automatically if you use a new
constructor.
Abbreviations must start with a letter and then be followed by
up to seven letters or digits.
Any letters appearing in the abbreviation must be in uppercase.

When used with the {\tt query} argument,
\syscmdindex{abbreviation query}
this command may be used to list the name
associated with a  particular abbreviation or the  abbreviation for a
constructor.
If no abbreviation or name is given, the names and corresponding
abbreviations for {\it all} constructors are listed.

The following shows the abbreviation for the constructor \spadtype{List}:
\begin{verbatim}
)abbreviation query List
\end{verbatim}
The following shows the constructor name corresponding to the
abbreviation \spadtype{NNI}:
\begin{verbatim}
)abbreviation query NNI
\end{verbatim}
The following lists all constructor names and their abbreviations.
\begin{verbatim}
)abbreviation query
\end{verbatim}

To add an abbreviation for a constructor, use this command with
{\tt category}, {\tt domain} or {\tt package}.
\syscmdindex{abbreviation package}
\syscmdindex{abbreviation domain}
\syscmdindex{abbreviation category}
The following add abbreviations to the system for a
category, domain and package, respectively:
\begin{verbatim}
)abbreviation domain   SET Set
)abbreviation category COMPCAT  ComplexCategory
)abbreviation package  LIST2MAP ListToMap
\end{verbatim}
If the {\tt )quiet} option is used,
no output is displayed from this command.
You would normally only define an abbreviation in a library source file.
If this command is issued for a constructor that has already been loaded, the
constructor will be reloaded next time it is referenced.  In particular, you
can use this command to force the automatic reloading of constructors.

To remove an abbreviation, the {\tt remove} argument is used.
\syscmdindex{abbreviation remove}
This is usually
only used to correct a previous command that set an abbreviation for a
constructor name.
If, in fact, the abbreviation does exist, you are prompted
for confirmation of the removal request.
Either of the following commands
will remove the abbreviation \spadtype{VECTOR2} and the
constructor name \spadtype{VectorFunctions2} from the system:
\begin{verbatim}
)abbreviation remove VECTOR2
)abbreviation remove VectorFunctions2
\end{verbatim}

\par\noindent{\bf Also See:}
\titledspadref{{\tt )compile}}{ugSysCmdcompile} and
% \titledspadref{{\tt )load}}{ugSysCmdload}.


% *********************************************************************
\head{section}{)boot}{ugSysCmdboot}
% *********************************************************************
\syscmdindex{boot}


\par\noindent{\bf User Level Required:} development

\par\noindent{\bf Command Syntax:}
\begin{simpleList}
\item {\tt )boot} {\it bootExpression}
\end{simpleList}

\par\noindent{\bf Command Description:}

This command is used by \Language{} system developers to execute
expressions written in the BOOT language.
For example,
\begin{verbatim}
)boot times3(x) == 3*x
\end{verbatim}
creates and compiles the \Lisp{} function ``times3''
obtained by translating the BOOT code.

\par\noindent{\bf Also See:}
\titledspadref{{\tt )fin}}{ugSysCmdfin},
\titledspadref{{\tt )lisp}}{ugSysCmdlisp},
\titledspadref{{\tt )set}}{ugSysCmdset}, and
\titledspadref{{\tt )system}}{ugSysCmdsystem}.


% *********************************************************************
\head{section}{)cd}{ugSysCmdcd}
% *********************************************************************
\syscmdindex{cd}


\par\noindent{\bf User Level Required:} interpreter

\par\noindent{\bf Command Syntax:}
\begin{simpleList}
\item {\tt )cd} {\it directory}
\end{simpleList}

\par\noindent{\bf Command Description:}

This command sets the \Language{} working current directory.
The current directory is used for looking for
input files (for {\tt )read}),
\Language{} library source files (for {\tt )compile}),
saved history environment files (for {\tt )history )restore}),
compiled \Language{} library files (for \spadsys{)library}), and
files to edit (for {\tt )edit}).
It is also used for writing
spool files (via {\tt )spool}),
writing history input files (via {\tt )history )write}) and
history environment files (via {\tt )history )save}),and
compiled \Language{} library files (via {\tt )compile}).
\syscmdindex{read}
\syscmdindex{compile}
\syscmdindex{history )restore}
\syscmdindex{edit}
\syscmdindex{spool}
\syscmdindex{history )write}
\syscmdindex{history )save}

If issued with no argument, this command sets the \Language{}
current directory to your home directory.
If an argument is used, it must be a valid directory name.
Except for the ``{\tt )}'' at the beginning of the command,
this has the same syntax as the operating system {\tt cd} command.

\par\noindent{\bf Also See:}
\titledspadref{{\tt )compile}}{ugSysCmdcompile},
\titledspadref{{\tt )edit}}{ugSysCmdedit},
\titledspadref{{\tt )history}}{ugSysCmdhistory},
\titledspadref{{\tt )library}}{ugSysCmdlibrary},
\titledspadref{{\tt )read}}{ugSysCmdread}, and
\titledspadref{{\tt )spool}}{ugSysCmdspool}.

% *********************************************************************
\head{section}{)close}{ugSysCmdclose}
% *********************************************************************
\syscmdindex{close}


\par\noindent{\bf User Level Required:} interpreter

\par\noindent{\bf Command Syntax:}
\begin{simpleList}
\item{\tt )close}
\item{\tt )close )quietly}
\end{simpleList}
\par\noindent{\bf Command Description:}

This command is used to close down interpreter client processes.
Such processes are started by \HyperName{} to run \Language{} examples
when you click on their text. When you have finished examining or modifying the
example and you do not want the extra window around anymore, issue
\begin{verbatim}
)close
\end{verbatim}
to the \Language{} prompt in the window.

If you try to close down the last remaining interpreter client
process, \Language{} will offer to close down the entire \Language{}
session and return you to the operating system by displaying something
like
\begin{verbatim}
   This is the last FriCAS session. Do you want to kill FriCAS?
\end{verbatim}
Type "y" (followed by the Return key) if this is what you had in mind.
Type "n" (followed by the Return key) to cancel the command.

You can use the {\tt )quietly} option to force \Language{} to
close down the interpreter client process without closing down
the entire \Language{} session.

\par\noindent{\bf Also See:}
\titledspadref{{\tt )quit}}{ugSysCmdquit} and
\titledspadref{{\tt )pquit}}{ugSysCmdpquit}.



% *********************************************************************
\head{section}{)clear}{ugSysCmdclear}
% *********************************************************************
\syscmdindex{clear}


\par\noindent{\bf User Level Required:} interpreter

\par\noindent{\bf Command Syntax:}
\begin{simpleList}
\item{\tt )clear all}
\item{\tt )clear completely}
\item{\tt )clear properties all}
\item{\tt )clear properties}  {\it obj1 \lanb{}obj2 ...\ranb{}}
\item{\tt )clear value      all}
\item{\tt )clear value}     {\it obj1 \lanb{}obj2 ...\ranb{}}
\item{\tt )clear mode       all}
\item{\tt )clear mode}      {\it obj1 \lanb{}obj2 ...\ranb{}}
\end{simpleList}
\par\noindent{\bf Command Description:}

This command is used to remove function and variable declarations, definitions
and values  from the workspace.
To  empty the entire workspace  and reset the
step counter to 1, issue
\begin{verbatim}
)clear all
\end{verbatim}
To remove everything in the workspace but not reset the step counter, issue
\begin{verbatim}
)clear properties all
\end{verbatim}
To remove everything about the object {\tt x}, issue
\begin{verbatim}
)clear properties x
\end{verbatim}
To remove everything about the objects {\tt x, y} and {\tt f}, issue
\begin{verbatim}
)clear properties x y f
\end{verbatim}

The word {\tt properties} may be abbreviated to the single letter
``{\tt p}''.
\begin{verbatim}
)clear p all
)clear p x
)clear p x y f
\end{verbatim}
All definitions of functions and values of variables may be removed by either
\begin{verbatim}
)clear value all
)clear v all
\end{verbatim}
This retains whatever declarations the objects had.  To remove definitions and
values for the specific objects {\tt x, y} and {\tt f}, issue
\begin{verbatim}
)clear value x y f
)clear v x y f
\end{verbatim}
To remove  the declarations  of everything while  leaving the  definitions and
values, issue
\begin{verbatim}
)clear mode  all
)clear m all
\end{verbatim}
To remove declarations for the specific objects {\tt x, y} and {\tt f}, issue
\begin{verbatim}
)clear mode x y f
)clear m x y f
\end{verbatim}
The {\tt )display names} and {\tt )display properties} commands  may be used
to see what is currently in the workspace.

The command
\begin{verbatim}
)clear completely
\end{verbatim}
does everything that {\tt )clear all} does, and also clears the internal
system function and constructor caches.

\par\noindent{\bf Also See:}
\titledspadref{{\tt )display}}{ugSysCmddisplay},
\titledspadref{{\tt )history}}{ugSysCmdhistory}, and
\titledspadref{{\tt )undo}}{ugSysCmdundo}.


% *********************************************************************
\head{section}{)compile}{ugSysCmdcompile}
% *********************************************************************
\syscmdindex{compile}


\par\noindent{\bf User Level Required:} compiler

\par\noindent{\bf Command Syntax:}

\begin{simpleList}
\item {\tt )compile}
\item {\tt )compile {\it fileName}}
\item {\tt )compile {\it fileName}.as}
\item {\tt )compile {\it directory/fileName}.as}
\item {\tt )compile {\it fileName}.ao}
\item {\tt )compile {\it directory/fileName}.ao}
\item {\tt )compile {\it fileName}.al}
\item {\tt )compile {\it directory/fileName}.al}
\item {\tt )compile {\it fileName}.lsp}
\item {\tt )compile {\it directory/fileName}.lsp}
\item {\tt )compile {\it fileName}.spad}
\item {\tt )compile {\it directory/fileName}.spad}
\item {\tt )compile {\it fileName} )new}
\item {\tt )compile {\it fileName} )old}
\item {\tt )compile {\it fileName} )translate}
\item {\tt )compile {\it fileName} )quiet}
\item {\tt )compile {\it fileName} )noquiet}
\item {\tt )compile {\it fileName} )moreargs}
\item {\tt )compile {\it fileName} )onlyargs}
\item {\tt )compile {\it fileName} )break}
\item {\tt )compile {\it fileName} )nobreak}
\item {\tt )compile {\it fileName} )library}
\item {\tt )compile {\it fileName} )nolibrary}
\item {\tt )compile {\it fileName} )vartrace}
\item {\tt )compile {\it fileName} )constructor} {\it nameOrAbbrev}
\end{simpleList}

\par\noindent{\bf Command Description:}

You use this command to invoke the \aldor{} library compiler or
the \Language{} system compiler.
The {\tt )compile} system command is actually a combination of
\Language{} processing and a call to the \aldor{} compiler.
It is performing double-duty, acting as a front-end to
both the \aldor{} compiler and the \Language{} system
compiler.
(The \Language{} system compiler is written in Boot and is
an integral part of the \Language{} environment.
The \aldor{} compiler is written in C and executed by the operating system
when called from within \Language{}.)

The command compiles files with file extensions {\it .as, .ao}
and {\it .al} with the
\aldor{} compiler and files with file extension {\it .spad} with the
\Language{} system compiler.
It also can compile files with file extension {\it .lsp}. These
are assumed to be Lisp files generated by the \aldor{}
compiler.
If you omit the file extension, the command looks to see if you
have specified the {\tt )new} or {\tt )old} option.
If you have given one of these options, the corresponding compiler
is used.
Otherwise, the command first looks in the standard system
directories for files with extension {\it .as, .ao} and {\it
.al} and then files with extension {\it .spad}.
The first file found has the appropriate compiler invoked on it.
If the command cannot find a matching file, an error message is
displayed and the command terminates.

The {\tt )translate} option is used to invoke a special version
of the \Language{} system compiler that will translate a {\it .spad} file
to a {\it .as} file. That is, the {\it .spad} file will be parsed and
analyzed and a file using the new syntax will be created. By default,
the {\it .as} file is created in the same directory as the
{\it .spad} file. If that directory is not writable, the current
directory is used. If the current directory is not writable, an
error message is given and the command terminates.
Note that {\tt )translate} implies the {\tt )old} option so the
file extension can safely be omitted. If {\tt )translate} is
given, all other options are ignored.
Please be aware that the translation is not necessarily one
hundred percent complete or correct.
You should attempt to compile the output with the \aldor{} compiler
and make any necessary corrections.

We now describe the options for the \aldor{} compiler.

The first thing {\tt )compile} does is look for a source code
filename among its arguments.
Thus
\begin{verbatim}
)compile mycode.as
)compile /u/jones/as/mycode.as
)compile mycode
\end{verbatim}
all invoke {\tt )compiler} on the file {\tt
/u/jones/as/mycode.as} if the current \Language{} working
directory is {\tt /u/jones/as.} (Recall that you can set the
working directory via the {\tt )cd} command. If you don't set it
explicitly, it is the directory from which you started
\Language{}.)

This is frequently all you need to compile your file.
This simple command:
\begin{enumerate}
\item invokes the \aldor{} compiler and produces Lisp output,
\item calls the Lisp compiler if the \aldor{} compilation was
successful,
\item uses the {\tt )library} command to tell \Language{} about
the contents of your compiled file and arrange to have those
contents loaded on demand.
\end{enumerate}

Should you not want the {\tt )library} command automatically
invoked, call {\tt )compile} with the {\tt )nolibrary} option.
For example,
\begin{verbatim}
)compile mycode.as )nolibrary
\end{verbatim}

The general description of \aldor{} command line arguments is in
the \aldor{} documentation.
The default options used by the {\tt )compile} command can be
viewed and set using the {\tt )set compiler args} \Language{}
system command.
The current defaults are
\begin{verbatim}
-O -Fasy -Fao -Flsp -laxiom -Mno-ALDOR_W_WillObsolete -DAxiom
-Y $AXIOM/algebra -I $AXIOM/algebra
\end{verbatim}
These options mean:
\begin{itemize}
\item {\tt -O}: perform all optimizations,
\item {\tt -Fasy}: generate a {\tt .asy} file,
\item {\tt -Fao}: generate a {\tt .ao} file,
\item {\tt -Flsp}: generate a {\tt .lsp} (Lisp)
file,
\index{Lisp!code generation}
\item {\tt -laxiom}: use the {\tt axiom} library {\tt libaxiom.al},
\item {\tt -Mno-ALDOR\_W\_WillObsolete}: do not display messages
about older generated files becoming obsolete, and
\item {\tt -DAxiom}: define the global assertion {\tt Axiom} so that the
\aldor{} libraries for generating stand-alone code
are not accidentally used with \Language{}.
\end{itemize}

To supplement these default arguments, use the {\tt )moreargs} option on
{\tt )compile.}
For example,
\begin{verbatim}
)compile mycode.as )moreargs "-v"
\end{verbatim}
uses the default arguments and appends the {\tt -v} (verbose)
argument flag.
The additional argument specification {\bf must be enclosed in
double quotes.}

To completely replace these default arguments for a particular
use of {\tt )compile}, use the {\tt )onlyargs} option.
For example,
\begin{verbatim}
)compile mycode.as )onlyargs "-v -O"
\end{verbatim}
only uses the {\tt -v} (verbose) and {\tt -O} (optimize)
arguments.
The argument specification {\bf must be enclosed in double quotes.}
In this example, Lisp code is not produced and so the compilation
output will not be available to \Language{}.

To completely replace the default arguments for all calls to {\tt
)compile} within your \Language{} session, use {\tt )set compiler args.}
For example, to use the above arguments for all compilations, issue
\begin{verbatim}
)set compiler args "-v -O"
\end{verbatim}
Make sure you include the necessary {\tt -l} and {\tt -Y}
arguments along with those needed for Lisp file creation.
As above, {\bf the argument specification must be enclosed in double
quotes.}

By default, the {\tt )library} system command {\it exposes} all
domains and categories it processes.
This means that the \Language{} interpreter will consider those
domains and categories when it is trying to resolve a reference
to a function.
Sometimes domains and categories should not be exposed.
For example, a domain may just be used privately by another
domain and may not be meant for top-level use.
The {\tt )library} command should still be used, though, so that
the code will be loaded on demand.
In this case, you should use the {\tt )nolibrary} option on {\tt
)compile} and the {\tt )noexpose} option in the {\tt )library}
command. For example,
\begin{verbatim}
)compile mycode.as )nolibrary
)library mycode )noexpose
\end{verbatim}

Once you have established your own collection of compiled code,
you may find it handy to use the {\tt )dir} option on the
{\tt )library} command.
This causes {\tt )library} to process all compiled code in the
specified directory. For example,
\begin{verbatim}
)library )dir /u/jones/as/quantum
\end{verbatim}
You must give an explicit directory after {\tt )dir}, even if you
want all compiled code in the current working directory
processed, e.g.
\begin{verbatim}
)library )dir .
\end{verbatim}

The {\tt )compile} command works with several file extensions. We saw
above what happens when it is invoked on a file with extension {\tt
.as.} A {\tt .ao} file is a portable binary compiled version of a
{\tt .as} file, and {\tt )compile} simply passes the {\tt .ao} file
onto \aldor{}. The generated Lisp file is compiled and {\tt )library}
is automatically called, just as if you had specified a {\tt .as} file.

A {\tt .al} file is an archive file containing {\tt .ao} files. The
archive is created (on Unix systems) with the {\tt ar} program. When
{\tt )compile} is given a {\tt .al} file, it creates a directory whose
name is based on that of the archive. For example, if you issue
\begin{verbatim}
)compile mylib.al
\end{verbatim}
the directory {\tt mylib.axldir} is created. All
members of the archive are unarchived into the
directory and {\tt )compile} is called on each {\tt .ao} file found. It
is your responsibility to remove the directory and its contents, if you
choose to do so.

A {\tt .lsp} file is a Lisp source file, presumably, in our context,
generated by \aldor{} when called with the {\tt -Flsp} option. When
{\tt )compile} is used with a {\tt .lsp} file, the Lisp file is
compiled and {\tt )library} is called. You must also have present a
{\tt .asy} generated from the same source file.

The following are descriptions of options for the \Language{} system compiler.

You can compile category, domain, and package constructors
contained in files with file extension {\it .spad}.
You can compile individual constructors or every constructor
in a file.

The full filename is remembered between invocations of this command and
{\tt )edit} commands.
The sequence of commands
\begin{verbatim}
)compile matrix.spad
)edit
)compile
\end{verbatim}
will call the compiler, edit, and then call the compiler again
on the file {\bf matrix.spad.}
If you do not specify a {\it directory,} the working current
directory (see \spadref{ugSysCmdcd})
is searched for the file.
If the file is not found, the standard system directories are searched.

If you do not give any options, all constructors within a file are
compiled.
Each constructor should have an {\tt )abbreviation} command in
the file in which it is defined.
We suggest that you place the {\tt )abbreviation} commands at the
top of the file in the order in which the constructors are
defined.
The list of commands serves as a table of contents for the file.
\syscmdindex{abbreviation}

The {\tt )library} option causes directories containing the
compiled code for each constructor
to be created in the working current directory.
The name of such a directory consists of the constructor
abbreviation and the {\bf .NRLIB} file extension.
For example, the directory containing the compiled code for
the \spadtype{MATRIX} constructor is called {\bf MATRIX.NRLIB.}
The {\tt )nolibrary} option says that such files should not
be created.
The default is {\tt )library.}
Note that the semantics of {\tt )library} and {\tt )nolibrary}
for the \aldor{} compiler and for the \Language{} system compiler are
completely different.

The {\tt )vartrace} option causes the compiler to generate
extra code for the constructor to support conditional tracing of
variable assignments. (see \spadref{ugSysCmdtrace}). Without
this option, this code is suppressed and one cannot use
the {\tt )vars} option for the trace command.

The {\tt )constructor} option is used to
specify a particular constructor to compile.
All other constructors in the file are ignored.
The constructor name or abbreviation follows {\tt )constructor.}
Thus either
\begin{verbatim}
)compile matrix.spad )constructor RectangularMatrix
\end{verbatim}
or
\begin{verbatim}
)compile matrix.spad )constructor RMATRIX
\end{verbatim}
compiles  the \spadtype{RectangularMatrix} constructor
defined in {\bf matrix.spad.}

The {\tt )break} and {\tt )nobreak} options determine what
the \Language system compiler does when it encounters an error.
{\tt )break} is the default and it indicates that processing
should stop at the first error.
The value of the {\tt )set break} variable then controls what happens.


%% BEGIN OBSOLTE
% It is important for you to realize that it does not suffice to compile a
% constructor to use the new code in the interpreter.
% After compilation, the {\tt )load} command with the
% {\tt )update} option should be used to bring in the new code
% and update internal system tables with information about the
% constructor.
%% END OBSOLTE

\par\noindent{\bf Also See:}
\titledspadref{{\tt )abbreviation}}{ugSysCmdabbreviation},
\titledspadref{{\tt )edit}}{ugSysCmdedit}, and
\titledspadref{{\tt )library}}{ugSysCmdlibrary}.


% *********************************************************************
\head{section}{)display}{ugSysCmddisplay}
% *********************************************************************
\syscmdindex{display}


\par\noindent{\bf User Level Required:} interpreter

\par\noindent{\bf Command Syntax:}
\begin{simpleList}
\item {\tt )display all}
\item {\tt )display properties}
\item {\tt )display properties all}
\item {\tt )display properties} {\it \lanb{}obj1 \lanb{}obj2 ...\ranb{}\ranb{}}
\item {\tt )display value all}
\item {\tt )display value} {\it \lanb{}obj1 \lanb{}obj2 ...\ranb{}\ranb{}}
\item {\tt )display mode all}
\item {\tt )display mode} {\it \lanb{}obj1 \lanb{}obj2 ...\ranb{}\ranb{}}
\item {\tt )display names}
\item {\tt )display operations} {\it opName}
\end{simpleList}
\par\noindent{\bf Command Description:}

This command is  used to display the contents of  the workspace and
signatures of functions  with a  given  name.\footnote{A
\spadgloss{signature} gives the argument and return types of a
function.}

The command
\begin{verbatim}
)display names
\end{verbatim}
lists the names of all user-defined  objects in the workspace.  This is useful
if you do  not wish to see everything  about the objects and need  only be
reminded of their names.

The commands
\begin{verbatim}
)display all
)display properties
)display properties all
\end{verbatim}
all do  the same thing: show  the values and  types and declared modes  of all
variables in the  workspace.  If you have defined  functions, their signatures
and definitions will also be displayed.

To show all information about a  particular variable or user functions,
for example, something named {\tt d}, issue
\begin{verbatim}
)display properties d
\end{verbatim}
To just show the value (and the type) of {\tt d}, issue
\begin{verbatim}
)display value d
\end{verbatim}
To just show the declared mode of {\tt d}, issue
\begin{verbatim}
)display mode d
\end{verbatim}

All modemaps for a given operation  may be
displayed by using {\tt )display operations}.
A \spadgloss{modemap} is a collection of information about  a particular
reference
to an  operation.  This  includes the  types of the  arguments and  the return
value, the  location of the  implementation and  any conditions on  the types.
The modemap may contain patterns.  The following displays the modemaps for the
operation \spadfunFrom{complex}{ComplexCategory}:
\begin{verbatim}
)d op complex
\end{verbatim}

\par\noindent{\bf Also See:}
\titledspadref{{\tt )clear}}{ugSysCmdclear},
\titledspadref{{\tt )history}}{ugSysCmdhistory},
\titledspadref{{\tt )set}}{ugSysCmdset},
\titledspadref{{\tt )show}}{ugSysCmdshow}, and
\titledspadref{{\tt )what}}{ugSysCmdwhat}.


% *********************************************************************
\head{section}{)edit}{ugSysCmdedit}
% *********************************************************************
\syscmdindex{edit}


\par\noindent{\bf User Level Required:} interpreter

\par\noindent{\bf Command Syntax:}
\begin{simpleList}
\item{\tt )edit} \lanb{}{\it filename}\ranb{}
\end{simpleList}
\par\noindent{\bf Command Description:}

This command is  used to edit files.
It works in conjunction  with the {\tt )read}
and {\tt )compile} commands to remember the name
of the file on which you are working.
By specifying the name fully, you  can edit any file you wish.
Thus
\begin{verbatim}
)edit /u/julius/matrix.input
\end{verbatim}
will place  you in an editor looking at the  file
{\tt /u/julius/matrix.input}.
\index{editing files}
By default, the editor is {\tt vi},
\index{vi}
but if you have an EDITOR shell environment variable defined, that editor
will be used.
When \Language{} is running under the X Window System,
it will try to open a separate {\tt xterm} running your editor if
it thinks one is necessary.
\index{Korn shell}
For example, under the Korn shell, if you issue
\begin{verbatim}
export EDITOR=emacs
\end{verbatim}
then the emacs
\index{emacs}
editor will be used by \spadsys{)edit}.

If you do not specify a file name, the last file you edited,
read or compiled will be used.
If there is no ``last file'' you will be placed in the editor editing
an empty unnamed file.

It is possible to use the {\tt )system} command to edit a file directly.
For example,
\begin{verbatim}
)system emacs /etc/rc.tcpip
\end{verbatim}
calls {\tt emacs} to edit the file.
\index{emacs}

\par\noindent{\bf Also See:}
\titledspadref{{\tt )system}}{ugSysCmdsystem},
\titledspadref{{\tt )compile}}{ugSysCmdcompile}, and
\titledspadref{{\tt )read}}{ugSysCmdread}.


% *********************************************************************
\head{section}{)fin}{ugSysCmdfin}
% *********************************************************************
\syscmdindex{fin}


\par\noindent{\bf User Level Required:} development

\par\noindent{\bf Command Syntax:}
\begin{simpleList}
\item {\tt )fin}
\end{simpleList}
\par\noindent{\bf Command Description:}

This command is used by \Language{}
developers to leave the \Language{} system and return
to the underlying \Lisp{} system.
To return to \Language{}, issue the
``{\tt (|spad|)}''
function call to \Lisp{}.

\par\noindent{\bf Also See:}
\titledspadref{{\tt )pquit}}{ugSysCmdpquit} and
\titledspadref{{\tt )quit}}{ugSysCmdquit}.


% *********************************************************************
\head{section}{)frame}{ugSysCmdframe}
% *********************************************************************
\syscmdindex{frame}


\par\noindent{\bf User Level Required:} interpreter

\par\noindent{\bf Command Syntax:}
\begin{simpleList}
\item{\tt )frame  new  {\it frameName}}
\item{\tt )frame  drop  {\it \lanb{}frameName\ranb{}}}
\item{\tt )frame  next}
\item{\tt )frame  last}
\item{\tt )frame  names}
\item{\tt )frame  import {\it frameName} {\it \lanb{}objectName1 \lanb{}objectName2 ...\ranb{}\ranb{}}}
\item{\tt )set message frame on | off}
\item{\tt )set message prompt frame}
\end{simpleList}

\par\noindent{\bf Command Description:}

A {\it frame} can be thought of as a logical session within the
physical session that you get when you start the system.  You can
have as many frames as you want, within the limits of your computer's
storage, paging space, and so on.
Each frame has its own {\it step number}, {\it environment} and {\it history.}
You can have a variable named {\tt a} in one frame and it will
have nothing to do with anything that might be called {\tt a} in
any other frame.

Some frames are created by the \HyperName{} program and these can
have pretty strange names, since they are generated automatically.
\syscmdindex{frame names}
To find out the names
of all frames, issue
\begin{verbatim}
)frame names
\end{verbatim}
It will indicate the name of the current frame.

You create a new frame
\syscmdindex{frame new}
``{\bf quark}'' by issuing
\begin{verbatim}
)frame new quark
\end{verbatim}
The history facility can be turned on by issuing either
{\tt )set history on} or {\tt )history )on}.
If the history facility is on and you are saving history information
in a file rather than in the \Language{} environment
then a history file with filename {\bf quark.axh} will
be created as you enter commands.
If you wish to go back to what
you were doing in the
\syscmdindex{frame next}
``{\bf initial}'' frame, use
\syscmdindex{frame last}
\begin{verbatim}
)frame next
\end{verbatim}
or
\begin{verbatim}
)frame last
\end{verbatim}
to cycle through the ring of available frames to get back to
``{\bf initial}''.

If you want to throw
away a frame (say ``{\bf quark}''), issue
\begin{verbatim}
)frame drop quark
\end{verbatim}
If you omit the name, the current frame is dropped.
\syscmdindex{frame drop}

If you do use frames with the history facility on and writing to a file,
you may want to delete some of the older history files.
\index{file!history}
These are directories, so you may want to issue a command like
{\tt rm -r quark.axh} to the operating system.

You can bring things from another frame by using
\syscmdindex{frame import}
{\tt )frame import}.
For example, to bring the {\tt f} and {\tt g} from the frame ``{\bf quark}''
to the current frame, issue
\begin{verbatim}
)frame import quark f g
\end{verbatim}
If you want everything from the frame ``{\bf quark}'', issue
\begin{verbatim}
)frame import quark
\end{verbatim}
You will be asked to verify that you really want everything.

There are two {\tt )set} flags
\syscmdindex{set message frame}
to make it easier to tell where you are.
\begin{verbatim}
)set message frame on | off
\end{verbatim}
will print more messages about frames when it is set on.
By default, it is off.
\begin{verbatim}
)set message prompt frame
\end{verbatim}
will give a prompt
\syscmdindex{set message prompt frame}
that looks like
\begin{verbatim}
initial (1) ->
\end{verbatim}
\index{prompt!with frame name}
when you start up. In this case, the frame name and step make up the
prompt.

\par\noindent{\bf Also See:}
\titledspadref{{\tt )history}}{ugSysCmdhistory} and
\titledspadref{{\tt )set}}{ugSysCmdset}.


% *********************************************************************
\head{section}{)help}{ugSysCmdhelp}
% *********************************************************************
\syscmdindex{help}


\par\noindent{\bf User Level Required:} interpreter

\par\noindent{\bf Command Syntax:}
\begin{simpleList}
\item{\tt )help}
\item{\tt )help} {\it commandName}
\item{\tt )help} {\tt syntax}
\end{simpleList}

\par\noindent{\bf Command Description:}

This command displays help information about system commands.
If you issue
\begin{verbatim}
)help help
\end{verbatim}
then this very text will be shown.
You can also give the name of a system command
to display information about it.
For example,
\begin{verbatim}
)help clear
\end{verbatim}
will display the description of the {\tt )clear} system command.

The command
\begin{verbatim}
)help syntax
\end{verbatim}
will give further information about the FriCAS language syntax.

All this material is available in the \Language{} User Guide
and in \HyperName{}.
In \HyperName{}, choose the {\bf Commands} item from the
{\bf Reference} menu.



% *********************************************************************
\head{section}{)history}{ugSysCmdhistory}
% *********************************************************************
\syscmdindex{history}


\par\noindent{\bf User Level Required:} interpreter

\par\noindent{\bf Command Syntax:}
\begin{simpleList}
\item{\tt )history )on}
\item{\tt )history )off}
\item{\tt )history )write} {\it historyInputFileName}
\item{\tt )history )show \lanb{}{\it n}\ranb{} \lanb{}both\ranb{}}
\item{\tt )history )save} {\it savedHistoryName}
\item{\tt )history )restore} \lanb{}{\it savedHistoryName}\ranb{}
\item{\tt )history )reset}
\item{\tt )history )change} {\it n}
\item{\tt )history )memory}
\item{\tt )history )file}
\item{\tt \%}
\item{\tt \%\%({\it n})}
\item{\tt )set history on | off}
\end{simpleList}

\par\noindent{\bf Command Description:}

The {\it history} facility within \Language{} allows you to restore your
environment to that of another session and recall previous
computational results.
Additional commands allow you to review previous
input lines and to create an {\bf .input} file of the lines typed to
\index{file!input}
\Language{}.

\Language{} saves your input and output if the history facility is
turned on (which is the default).
This information is saved if either of
\begin{verbatim}
)set history on
)history )on
\end{verbatim}
has been issued.
Issuing either
\begin{verbatim}
)set history off
)history )off
\end{verbatim}
will discontinue the recording of information.
\syscmdindex{history )on}
\syscmdindex{set history on}
\syscmdindex{set history off}
\syscmdindex{history )off}

Whether the facility is disabled or not,
the value of \spadSyntax evaluates to an object of type
\spadtype{Variable('%)}.
The function \spadSyntax may be  used to refer
to other previous results if the history facility is enabled.
In that case,
\spad{%%(n)} is  the output from step \spad{n} if \spad{n > 0}.
If \spad{n < 0}, the step is computed relative to the current step.
Thus \spad{%%(-1)} is also the previous step,
\spad{%%(-2)}, is the  step before that, and so on.
If an invalid step number is given, \Language{} will signal an error.

The {\it environment} information can either be saved in a file or entirely in
memory (the default).
Each frame (\spadref{ugSysCmdframe})
has its own history database.
When it is kept in a file, some of it may also be kept in memory for
efficiency.
When the information is saved in a file, the name of the file is
of the form {\bf FRAME.axh} where ``{\bf FRAME}'' is the name of the
current frame.
The history file is placed in the current working directory
(see \spadref{ugSysCmdcd}).
Note that these history database files are not text files (in fact,
they are directories themselves), and so are not in human-readable
format.

The options to the {\tt )history} command are as follows:

\begin{description}
\item[{\tt )change} {\it n}]
will set the number of steps that are saved in memory to {\it n}.
This option only has effect when the history data is maintained in a
file.
If you have issued {\tt )history )memory} (or not changed the default)
there is no need to use {\tt )history )change}.
\syscmdindex{history )change}

\item[{\tt )on}]
will start the recording of information.
If the workspace is not empty, you will be asked to confirm this
request.
If you do so, the workspace will be cleared and history data will begin
being saved.
You can also turn the facility on by issuing {\tt )set history on}.

\item[{\tt )off}]
will stop the recording of information.
The {\tt )history )show} command will not work after issuing this
command.
Note that this command may be issued to save time, as there is some
performance penalty paid for saving the environment data.
You can also turn the facility off by issuing {\tt )set history off}.

\item[{\tt )file}]
indicates that history data should be saved in an external file on disk.

\item[{\tt )memory}]
indicates that all history data should be kept in memory rather than
saved in a file.
Note that if you are computing with very large objects it may not be
practical to kept this data in memory.

\item[{\tt )reset}]
will flush the internal list of the most recent workspace calculations
so that the data structures may be garbage collected by the underlying
\Lisp{} system.
Like {\tt )history )change}, this option only has real effect when
history data is being saved in a file.

\item[{\tt )restore} \lanb{}{\it savedHistoryName}\ranb{}]
completely clears the environment and restores it to a saved session, if
possible.
The {\tt )save} option below allows you to save a session to a file
with a given name. If you had issued
{\tt )history )save jacobi}
the command
{\tt )history )restore jacobi}
would clear the current workspace and load the contents of the named
saved session. If no saved session name is specified, the system looks
for a file called {\bf last.axh}.

\item[{\tt )save} {\it savedHistoryName}]
is used to save  a snapshot of the environment in a file.
This file is placed in the current working directory
(see \spadref{ugSysCmdcd}).
Use {\tt )history )restore} to restore the environment to the state
preserved in the file.
This option also creates an input file containing all the lines of input
since you created the workspace frame (for example, by starting your
\Language{} session) or last did a \spadsys{)clear all} or
\spadsys{)clear completely}.

\item[{\tt )show} \lanb{}{\it n}\ranb{} \lanb{}{\tt both}\ranb{}]
can show previous input lines and output results.
{\tt )show} will display up to twenty of the last input lines
(fewer if you haven't typed in twenty lines).
{\tt )show} {\it n} will display up to {\it n} of the last input lines.
{\tt )show both} will display up to five of the last input lines and
output results.
{\tt )show} {\it n} {\tt both} will display up to {\it n} of the last
input lines and output results.

\item[{\tt )write} {\it historyInputFile}]
creates an {\bf .input} file with the input lines typed since the start
of the session/frame or the last {\tt )clear all} or {\tt )clear
completely}.
If {\it historyInputFileName} does not contain a period (``.'') in the filename,
{\bf .input} is appended to it.
For example,
{\tt )history )write chaos}
and
{\tt )history )write chaos.input}
both write the input lines to a file called {\bf chaos.input} in your
current working directory.
If you issued one or more {\tt )undo} commands,
{\tt )history )write}
eliminates all
input lines backtracked over as a result of {\tt )undo}.
You can edit this file and then use {\tt )read} to have \Language{} process
the contents.
\end{description}

\par\noindent{\bf Also See:}
\titledspadref{{\tt )frame}}{ugSysCmdframe},
\titledspadref{{\tt )read}}{ugSysCmdread},
\titledspadref{{\tt )set}}{ugSysCmdset}, and
\titledspadref{{\tt )undo}}{ugSysCmdundo}.


% *********************************************************************
\head{section}{)library}{ugSysCmdlibrary}
% *********************************************************************
\syscmdindex{library}


\par\noindent{\bf User Level Required:} interpreter

\par\noindent{\bf Command Syntax:}
\begin{simpleList}
\item{\tt )library {\it libName1  \lanb{}libName2 ...\ranb{}}}
\item{\tt )library )dir {\it dirName}}
\item{\tt )library )only {\it objName1  \lanb{}objlib2 ...\ranb{}}}
\item{\tt )library )noexpose}
\end{simpleList}

\par\noindent{\bf Command Description:}

This command replaces the {\tt )load} system command that
was available in \Language{} releases before version 2.0.
The \spadsys{)library} command makes available to \Language{} the compiled
objects in the libraries listed.

For example, if you {\tt )compile dopler.as} in your home
directory, issue {\tt )library dopler} to have \Language{} look
at the library, determine the category and domain constructors present,
update the internal database with various properties of the
constructors, and arrange for the constructors to be
automatically loaded when needed.
If the {\tt )noexpose} option has not been given, the
constructors will be exposed (that is, available) in the current
frame.

If you compiled a file with the \Language{} system compiler, you will
have an {\it NRLIB} present, for example, {\it DOPLER.NRLIB,}
where {\tt DOPLER} is a constructor abbreviation.
The command {\tt )library DOPLER} will then do the analysis and
database updates as above.

To tell the system about all libraries in a directory, use
{\tt )library )dir dirName} where {\tt dirName} is an explicit
directory.
You may specify ``.'' as the directory, which means the current
directory from which you started the system or the one you set
via the \spadsys{)cd} command. The directory name is required.

You may only want to tell the system about particular
constructors within a library. In this case, use the {\tt )only}
option. The command {\tt )library dopler )only Test1} will only
cause the {\sf Test1} constructor to be analyzed, autoloaded,
etc..

Finally, each constructor in a library  are usually automatically exposed when the
\spadsys{)library} command is used. Use the {\tt )noexpose}
option if you not want them exposed. At a later time you can use
{\tt )set expose add constructor} to expose any hidden
constructors.

\par\noindent{\bf Also See:}
\titledspadref{{\tt )cd}}{ugSysCmdcd},
\titledspadref{{\tt )compile}}{ugSysCmdcompile},
\titledspadref{{\tt )frame}}{ugSysCmdframe}, and
\titledspadref{{\tt )set}}{ugSysCmdset}.

% *********************************************************************
\head{section}{)lisp}{ugSysCmdlisp}
% *********************************************************************
\syscmdindex{lisp}


\par\noindent{\bf User Level Required:} development

\par\noindent{\bf Command Syntax:}
\begin{simpleList}
\item {\tt )lisp} {\it\lanb{}lispExpression\ranb{}}
\end{simpleList}

\par\noindent{\bf Command Description:}

This command is used by \Language{} system developers to have single
expressions evaluated by the \Lisp{} system on which
\Language{} is built.
The {\it lispExpression} is read by the \Lisp{} reader and
evaluated.
If this expression is not complete (unbalanced parentheses, say), the reader
will wait until a complete expression is entered.

Since this command is only useful  for evaluating single expressions, the
{\tt )fin}
command may be used to  drop out  of \Language{}  into \Lisp{}.

\par\noindent{\bf Also See:}
\titledspadref{{\tt )system}}{ugSysCmdsystem},
\titledspadref{{\tt )boot}}{ugSysCmdboot}, and
\titledspadref{{\tt )fin}}{ugSysCmdfin}.



% *********************************************************************
\head{section}{)load}{ugSysCmdload}
% *********************************************************************
\syscmdindex{load}


\par\noindent{\bf User Level Required:} interpreter

%% BEGIN OBSOLETE
% \par\noindent{\bf Command Syntax:}
% \begin{simpleList}
% \item{\tt )load {\it libName1  \lanb{}libName2 ...\ranb{}} \lanb{})update\ranb{}}
% \item{\tt )load {\it libName1  \lanb{}libName2 ...\ranb{}} )cond \lanb{})update\ranb{}}
% \item{\tt )load {\it libName1  \lanb{}libName2 ...\ranb{}} )query}
% \item{\tt )load {\it libName1  \lanb{}libName2 ...\ranb{}} )noexpose}
% \end{simpleList}
%% END OBSOLETE

\par\noindent{\bf Command Description:}

This command is obsolete. Use \spadsys{)library} instead.

%% BEGIN OBSOLETE

% The {\tt )load} command is used to bring in the compiled library code
% for constructors and update internal system tables with information
% about the constructors.
% This command is usually only used by \Language{} library developers.
%
% The abbreviation of a constructor serves as part of the name of the
% directory in which the compiled code is stored (see
% \spadref{ugSysCmdabbreviation} for a discussion of defining and querying
% abbreviations).
% The abbreviation is used in the {\tt )load} command.
% For example, to load the constructors \spadtype{Integer},
% \spadtype{NonNegativeInteger} and \spadtype{List} which have
% abbreviations \spadtype{INT}, \spadtype{NNI} and \spadtype{LIST},
% respectively, issue the command
% \begin{verbatim}
% )load INT NNI LIST
% \end{verbatim}
% To load constructors only if they have not already been
% loaded (that is., load {\it conditionally}), use the {\tt )cond}
% option:
% \begin{verbatim}
% )load INT NNI LIST )cond
% \end{verbatim}
% To query whether particular constructors have been loaded, use the
% {\tt )query} option:
% \begin{verbatim}
% )load I NNI L )query
% \end{verbatim}
% When constructors are loaded from \Language{} system directories, some
% checks and updates are not performed because it is assumed that the system
% knows about these constructors.
% To force these checks and updates to occur, add the {\tt )update}
% option to the command:
% \begin{verbatim}
% )load INT NNI LIST )update
% )load INT NNI LIST )cond )update
% \end{verbatim}
% The only time it is really necessary to use the {\tt )load} command is
% when a new constructor has been compiled or an existing constructor has
% been modified and then compiled.
% If an {\tt )abbreviate} command has been issued for a constructor, it
% will be automatically loaded when needed.
% In particular, any constructor that comes with the \Language{} system
% will be automatically loaded.
%
% If you write several interdependent constructors it is important that
% they all get loaded when needed.
% To accomplish this, either load them manually or issue
% {\tt )abbreviate} commands for each of the constructors so that they
% will be automatically loaded when needed.
%
% Constructors are automatically exposed in the frame in which you load
% them unless you use the {\tt )noexpose} option.
% \begin{verbatim}
% )load MATCAT- )noexpose
% \end{verbatim}
% See \spadref{ugTypesExpose}
% for more information about constructor exposure.
%
% \par\noindent{\bf Also See:}
% \titledspadref{{\tt )abbreviation}}{ugSysCmdabbreviation} and
% \titledspadref{{\tt )compile}}{ugSysCmdcompile}.

%% END OBSOLETE


% *********************************************************************
\head{section}{)ltrace}{ugSysCmdltrace}
% *********************************************************************
\syscmdindex{ltrace}


\par\noindent{\bf User Level Required:} development

\par\noindent{\bf Command Syntax:}

This command has the same arguments as options as the
\spadsys{)trace} command.

\par\noindent{\bf Command Description:}

This command is used by \Language{} system developers to trace
\Lisp{} or
BOOT functions.
It is not supported for general use.

\par\noindent{\bf Also See:}
\titledspadref{{\tt )boot}}{ugSysCmdboot},
\titledspadref{{\tt )lisp}}{ugSysCmdlisp}, and
\titledspadref{{\tt )trace}}{ugSysCmdtrace}.


% *********************************************************************
\head{section}{)pquit}{ugSysCmdpquit}
% *********************************************************************
\syscmdindex{pquit}


\par\noindent{\bf User Level Required:} interpreter

\par\noindent{\bf Command Syntax:}
\begin{simpleList}
\item{\tt )pquit}
\end{simpleList}

\par\noindent{\bf Command Description:}

This command is used to terminate \Language{}  and return to the
operating system.
Other than by redoing all your computations or by
using the {\tt )history )restore}
command to try to restore your working environment,
you cannot return to \Language{} in the same state.

{\tt )pquit} differs from the {\tt )quit} in that it always asks for
confirmation that you want to terminate \Language{} (the ``p'' is for
``protected'').
\syscmdindex{quit}
When you enter the {\tt )pquit} command, \Language{} responds
%
\begin{center}
Please enter {\bf y} or {\bf yes} if you really want to leave the interactive \\
environment and return to the operating system:
\end{center}
%
If you respond with {\tt y} or {\tt yes}, you will see the message
%
\begin{center}
You are now leaving the \Language{} interactive environment. \\
Issue the command {\bf axiom} to the operating system to start a new session.
\end{center}
%
and \Language{} will terminate and return you to the operating
system (or the environment from which you invoked the system).
If you responded with something other than {\tt y} or {\tt yes}, then
the message
%
\begin{center}
You have chosen to remain in the \Language{} interactive environment.
\end{center}
%
will be displayed and, indeed, \Language{} would still be running.

\par\noindent{\bf Also See:}
\titledspadref{{\tt )fin}}{ugSysCmdfin},
\titledspadref{{\tt )history}}{ugSysCmdhistory},
\titledspadref{{\tt )close}}{ugSysCmdclose},
\titledspadref{{\tt )quit}}{ugSysCmdquit}, and
\titledspadref{{\tt )system}}{ugSysCmdsystem}.


% *********************************************************************
\head{section}{)quit}{ugSysCmdquit}
% *********************************************************************
\syscmdindex{quit}


\par\noindent{\bf User Level Required:} interpreter

\par\noindent{\bf Command Syntax:}
\begin{simpleList}
\item{\tt )quit}
\item{\tt )set quit protected | unprotected}
\end{simpleList}

\par\noindent{\bf Command Description:}

This command is used to terminate \Language{}  and return to the
operating system.
Other than by redoing all your computations or by
using the {\tt )history )restore}
command to try to restore your working environment,
you cannot return to \Language{} in the same state.

{\tt )quit} differs from the {\tt )pquit} in that it asks for
\syscmdindex{pquit}
confirmation only if the command
\begin{verbatim}
)set quit protected
\end{verbatim}
has been issued.
\syscmdindex{set quit protected}
Otherwise, {\tt )quit} will make \Language{} terminate and return you
to the operating system (or the environment from which you invoked the
system).

The default setting is {\tt )set quit unprotected}.  We
\syscmdindex{set quit unprotected}
suggest that you do not (somehow) assign {\tt )quit} to be
executed when you press, say, a function key.

\par\noindent{\bf Also See:}
\titledspadref{{\tt )fin}}{ugSysCmdfin},
\titledspadref{{\tt )history}}{ugSysCmdhistory},
\titledspadref{{\tt )close}}{ugSysCmdclose},
\titledspadref{{\tt )pquit}}{ugSysCmdpquit}, and
\titledspadref{{\tt )system}}{ugSysCmdsystem}.


% *********************************************************************
\head{section}{)read}{ugSysCmdread}
% *********************************************************************
\syscmdindex{read}


\par\noindent{\bf User Level Required:} interpreter

\par\noindent{\bf Command Syntax:}
\begin{simpleList}
\item {\tt )read} {\it \lanb{}fileName\ranb{}}
\item {\tt )read} {\it \lanb{}fileName\ranb{}} \lanb{}{\tt )quiet}\ranb{} \lanb{}{\tt )ifthere}\ranb{}
\end{simpleList}
\par\noindent{\bf Command Description:}

This command is used to read {\bf .input} files into \Language{}.
\index{file!input}
The command
\begin{verbatim}
)read matrix.input
\end{verbatim}
will read the contents of the file {\bf matrix.input} into
\Language{}.
The ``.input'' file extension is optional.
See \spadref{ugInOutIn} for more information about {\bf .input} files.

This command remembers the previous file you edited, read or compiled.
If you do not specify a file name, the previous file will be read.

The {\tt )ifthere} option checks to see whether the {\bf .input} file
exists.
If it does not, the  {\tt )read} command does nothing.
If you do not use this option and the file does not exist,
you are asked to give the name of an existing {\bf .input} file.

The {\tt )quiet} option suppresses output while the file is being read.

\par\noindent{\bf Also See:}
\titledspadref{{\tt )compile}}{ugSysCmdcompile},
\titledspadref{{\tt )edit}}{ugSysCmdedit}, and
\titledspadref{{\tt )history}}{ugSysCmdhistory}.


% *********************************************************************
\head{section}{)set}{ugSysCmdset}
% *********************************************************************
\syscmdindex{set}


\par\noindent{\bf User Level Required:} interpreter

\par\noindent{\bf Command Syntax:}
\begin{simpleList}
\item {\tt )set}
\item {\tt )set} {\it label1 \lanb{}... labelN\ranb{}}
\item {\tt )set} {\it label1 \lanb{}... labelN\ranb{} newValue}
\end{simpleList}
\par\noindent{\bf Command Description:}

The {\tt )set} command is used to view or set system variables that
control what messages are displayed, the type of output desired, the
status of the history facility, the way \Language{} user functions are
cached, and so on.
Since this collection is very large, we will not discuss them here.
Rather, we will show how the facility is used.
We urge you to explore the {\tt )set} options to familiarize yourself
with how you can modify your \Language{} working environment.
There is a \HyperName{} version of this same facility available from the
main \HyperName{} menu.


The {\tt )set} command is command-driven with a menu display.
It is tree-structured.
To see all top-level nodes, issue {\tt )set} by itself.
\begin{verbatim}
)set
\end{verbatim}
Variables with values have them displayed near the right margin.
Subtrees of selections have ``{\tt ...}''
displayed in the value field.
For example, there are many kinds of messages, so issue
{\tt )set message} to see the choices.
\begin{verbatim}
)set message
\end{verbatim}
The current setting  for the variable that displays
\index{computation timings!displaying}
whether computation times
\index{timings!displaying}
are displayed is visible in the menu displayed by the last command.
To see more information, issue
\begin{verbatim}
)set message time
\end{verbatim}
This shows that time printing is on now.
To turn it off, issue
\begin{verbatim}
)set message time off
\end{verbatim}
\syscmdindex{set message time}

As noted above, not all settings have so many qualifiers.
For example, to change the {\tt )quit} command to being unprotected
(that is, you will not be prompted for verification), you need only issue
\begin{verbatim}
)set quit unprotected
\end{verbatim}
\syscmdindex{set quit unprotected}

\par\noindent{\bf Also See:}
\titledspadref{{\tt )quit}}{ugSysCmdquit}.


% *********************************************************************
\head{section}{)show}{ugSysCmdshow}
% *********************************************************************
\syscmdindex{show}


\par\noindent{\bf User Level Required:} interpreter

\par\noindent{\bf Command Syntax:}
\begin{simpleList}
\item{\tt )show {\it nameOrAbbrev}}
\item{\tt )show {\it nameOrAbbrev} )operations}
\item{\tt )show {\it nameOrAbbrev} )attributes}
\end{simpleList}

\par\noindent{\bf Command Description:}
This command displays information about \Language{}
domain, package and category {\it constructors}.
If no options are given, the {\tt )operations} option is assumed.
For example,
\begin{verbatim}
)show POLY
)show POLY )operations
)show Polynomial
)show Polynomial )operations
\end{verbatim}
each display basic information about the
\spadtype{Polynomial} domain constructor and then provide a
listing of operations.
Since \spadtype{Polynomial} requires a \spadtype{Ring} (for example,
\spadtype{Integer}) as argument, the above commands all refer
to a unspecified ring {\tt R}.
In the list of operations, \spadSyntax{$} means
\spadtype{Polynomial(R)}.

The basic information displayed includes the {\it signature}
of the constructor (the name and arguments), the constructor
{\it abbreviation}, the {\it exposure status} of the constructor, and the
name of the {\it library source file} for the constructor.

If operation information about a specific domain is wanted,
the full or abbreviated domain name may be used.
For example,
\begin{verbatim}
)show POLY INT
)show POLY INT )operations
)show Polynomial Integer
)show Polynomial Integer )operations
\end{verbatim}
are among  the combinations that will
display the operations exported  by the
domain \spadtype{Polynomial(Integer)} (as opposed to the general
{\it domain constructor} \spadtype{Polynomial}).
Attributes may be listed by using the {\tt )attributes} option.

\par\noindent{\bf Also See:}
\titledspadref{{\tt )display}}{ugSysCmddisplay},
\titledspadref{{\tt )set}}{ugSysCmdset}, and
\titledspadref{{\tt )what}}{ugSysCmdwhat}.


% *********************************************************************
\head{section}{)spool}{ugSysCmdspool}
% *********************************************************************
\syscmdindex{spool}


\par\noindent{\bf User Level Required:} interpreter

\par\noindent{\bf Command Syntax:}
\begin{simpleList}
\item{\tt )spool} \lanb{}{\it fileName}\ranb{}
\item{\tt )spool}
\end{simpleList}

\par\noindent{\bf Command Description:}

This command is used to save {\it (spool)} all \Language{} input and output
\index{file!spool}
into a file, called a {\it spool file.}
You can only have one spool file active at a time.
To start spool, issue this command with a filename. For example,
\begin{verbatim}
)spool integrate.out
\end{verbatim}
To stop spooling, issue {\tt )spool} with no filename.

If the filename is qualified with a directory, then the output will
be placed in that directory.
If no directory information is given, the spool file will be placed in the
\index{directory!for spool files}
{\it current directory.}
The current directory is the directory from which you started
\Language{} or is the directory you specified using the
{\tt )cd} command.
\syscmdindex{cd}

\par\noindent{\bf Also See:}
\titledspadref{{\tt )cd}}{ugSysCmdcd}.


% *********************************************************************
\head{section}{)synonym}{ugSysCmdsynonym}
% *********************************************************************
\syscmdindex{synonym}


\par\noindent{\bf User Level Required:} interpreter

\par\noindent{\bf Command Syntax:}
\begin{simpleList}
\item{\tt )synonym}
\item{\tt )synonym} {\it synonym fullCommand}
\item{\tt )what synonyms}
\end{simpleList}

\par\noindent{\bf Command Description:}

This command is used to create short synonyms for system command expressions.
For example, the following synonyms  might simplify commands you often
use.
\begin{verbatim}
)synonym save         history )save
)synonym restore      history )restore
)synonym mail         system mail
)synonym ls           system ls
)synonym fortran      set output fortran
\end{verbatim}
Once defined, synonyms can be
used in place of the longer  command expressions.
Thus
\begin{verbatim}
)fortran on
\end{verbatim}
is the same as the longer
\begin{verbatim}
)set fortran output on
\end{verbatim}
To list all defined synonyms, issue either of
\begin{verbatim}
)synonyms
)what synonyms
\end{verbatim}
To list, say, all synonyms that contain the substring
``{\tt ap}'', issue
\begin{verbatim}
)what synonyms ap
\end{verbatim}

\par\noindent{\bf Also See:}
\titledspadref{{\tt )set}}{ugSysCmdset} and
\titledspadref{{\tt )what}}{ugSysCmdwhat}.


% *********************************************************************
\head{section}{)system}{ugSysCmdsystem}
% *********************************************************************
\syscmdindex{system}

\par\noindent{\bf User Level Required:} interpreter

\par\noindent{\bf Command Syntax:}
\begin{simpleList}
\item{\tt )system} {\it cmdExpression}
\end{simpleList}

\par\noindent{\bf Command Description:}

This command may be used to issue commands to the operating system while
remaining in \Language{}.
The {\it cmdExpression} is passed to the operating system for
execution.

To get an operating system shell, issue, for example,
\spadsys{)system sh}.
When you enter the key combination,
\fbox{\bf Ctrl}--\fbox{\bf D}
(pressing and holding the
\fbox{\bf Ctrl} key and then pressing the
\fbox{\bf D} key)
the shell will terminate and you will return to \Language{}.
We do not recommend this way of creating a shell because
\Lisp{} may field some interrupts instead of the shell.
If possible, use a shell running in another window.

If you execute programs that misbehave you may not be able to return to
\Language{}.
If this happens, you may have no other choice than to restart
\Language{} and restore the environment via {\tt )history )restore}, if
possible.

\par\noindent{\bf Also See:}
\titledspadref{{\tt )boot}}{ugSysCmdboot},
\titledspadref{{\tt )fin}}{ugSysCmdfin},
\titledspadref{{\tt )lisp}}{ugSysCmdlisp},
\titledspadref{{\tt )pquit}}{ugSysCmdpquit}, and
\titledspadref{{\tt )quit}}{ugSysCmdquit}.


% *********************************************************************
\head{section}{)trace}{ugSysCmdtrace}
% *********************************************************************
\syscmdindex{trace}


\par\noindent{\bf User Level Required:} interpreter

\par\noindent{\bf Command Syntax:}
\begin{simpleList}
\item{\tt )trace}
\item{\tt )trace )off}

\item{\tt )trace} {\it function \lanb{}options\ranb{}}
\item{\tt )trace} {\it constructor \lanb{}options\ranb{}}
\item{\tt )trace} {\it domainOrPackage \lanb{}options\ranb{}}
\end{simpleList}
%
where options can be one or more of
%
\begin{simpleList}
\item{\tt )after} {\it S-expression}
\item{\tt )before} {\it S-expression}
\item{\tt )break after}
\item{\tt )break before}
\item{\tt )cond} {\it S-expression}
\item{\tt )count}
\item{\tt )count} {\it n}
\item{\tt )depth} {\it n}
\item{\tt )local} {\it op1 \lanb{}... opN\ranb{}}
\item{\tt )nonquietly}
\item{\tt )nt}
\item{\tt )off}
\item{\tt )only} {\it listOfDataToDisplay}
\item{\tt )ops}
\item{\tt )ops} {\it op1 \lanb{}... opN \ranb{}}
\item{\tt )restore}
\item{\tt )stats}
\item{\tt )stats reset}
\item{\tt )timer}
\item{\tt )varbreak}
\item{\tt )varbreak} {\it var1 \lanb{}... varN \ranb{}}
\item{\tt )vars}
\item{\tt )vars} {\it var1 \lanb{}... varN \ranb{}}
\item{\tt )within} {\it executingFunction}
\end{simpleList}

\par\noindent{\bf Command Description:}

This command is used to trace the execution of functions that make
up the \Language{} system, functions defined by users,
and functions from the system library.
Almost all options are available for each type of function but
exceptions will be noted below.

To list all functions, constructors, domains and packages that are
traced, simply issue
\begin{verbatim}
)trace
\end{verbatim}
To untrace everything that is traced, issue
\begin{verbatim}
)trace )off
\end{verbatim}
When a function is traced, the default system action is to display
the arguments to the function and the return value when the
function is exited.
Note that if a function is left via an action such as a {\tt THROW}, no
return value will be displayed.
Also, optimization of tail recursion may decrease the number of
times a function is actually invoked and so may cause less trace
information to be displayed.
Other information can be displayed or collected when a function is
traced and this is controlled by the various options.
Most options will be of interest only to \Language{} system
developers.
If a domain or package is traced, the default action is to trace
all functions exported.

Individual interpreter, lisp or boot
functions can be traced by listing their names after
{\tt )trace}.
Any options that are present must follow the functions to be
traced.
\begin{verbatim}
)trace f
\end{verbatim}
traces the function {\tt f}.
To untrace {\tt f}, issue
\begin{verbatim}
)trace f )off
\end{verbatim}
Note that if a function name contains a special character, it will
be necessary to escape the character with an underscore
%
\begin{verbatim}
)trace _/D_,1
\end{verbatim}
%
To trace all domains or packages that are or will be created from a particular
constructor, give the constructor name or abbreviation after
{\tt )trace}.
%
\begin{verbatim}
)trace MATRIX
)trace List Integer
\end{verbatim}
%
The first command traces all domains currently instantiated with
\spadtype{Matrix}.
If additional domains are instantiated with this constructor
(for example, if you have used \spadtype{Matrix(Integer)} and
\spadtype{Matrix(Float)}), they will be automatically traced.
The second command traces \spadtype{List(Integer)}.
It is possible to trace individual functions in a domain or
package.
See the {\tt )ops} option below.

The following are the general options for the {\tt )trace}
command.

%!! system command parser doesn't treat general s-expressions correctly,
%!! I recommand not documenting )after )before and )cond
\begin{description}
%\item[{\tt )after} {\it S-expression}]
%causes the given \Lisp{} {\it S-expression} to be
%executed after exiting the traced function.

%\item[{\tt )before} {\it S-expression}]
%causes the given \Lisp{} {\it S-expression} to be
%executed before entering the traced function.

\item[{\tt )break after}]
causes a \Lisp{} break loop to be entered after
exiting the traced function.

\item[{\tt )break before}]
causes a \Lisp{} break loop to be entered before
entering the traced function.

\item[{\tt )break}]
is the same as \spadsys{)break before}.

%\item[{\tt )cond} {\it S-expression}]
%causes trace information to be shown only if the given
%\Lisp{} {\it S-expression} evaluates to non-NIL.  For
%example, the following command causes the system function
%{\tt resolveTT} to be traced but to have the information
%displayed only if the value of the variable
%{\tt \$reportBottomUpFlag} is non-NIL.
%\begin{verbatim}
%)trace resolveTT )cond \_\$reportBottomUpFlag}
%\end{verbatim}

\item[{\tt )count}]
causes the system to keep a count of the number of times the
traced function is entered.  The total can be displayed with
{\tt )trace )stats} and cleared with {\tt )trace )stats reset}.

\item[{\tt )count} {\it n}]
causes information about the traced function to be displayed for
the first {\it n} executions.  After the \eth{\it n} execution, the
function is untraced.

\item[{\tt )depth} {\it n}]
causes trace information to be shown for only {\it n} levels of
recursion of the traced function.  The command
\begin{verbatim}
)trace fib )depth 10
\end{verbatim}
will cause the display of only 10 levels of trace information for
the recursive execution of a user function \userfun{fib}.

\item[{\tt )math}]
causes the function arguments and return value to be displayed in the
\Language{} monospace two-dimensional math format.

\item[{\tt )nonquietly}]
causes the display of additional messages when a function is
traced.

\item[{\tt )nt}]
This suppresses all normal trace information.  This option is
useful if the {\tt )count} or {\tt )timer} options are used and
you are interested in the statistics but not the function calling
information.

\item[{\tt )off}]
causes untracing of all or specific functions.  Without an
argument, all functions, constructors, domains and packages are
untraced.  Otherwise, the given functions and other objects
are untraced.  To
immediately retrace the untraced functions, issue {\tt )trace
)restore}.

\item[{\tt )only} {\it listOfDataToDisplay}]
causes only specific trace information to be shown.  The items are
listed by using the following abbreviations:
\begin{description}
\item[a]        display all arguments
\item[v]        display return value
\item[1]        display first argument
\item[2]        display second argument
\item[15]       display the 15th argument, and so on
\end{description}
\end{description}
\begin{description}

\item[{\tt )restore}]
causes the last untraced functions to be retraced.  If additional
options are present, they are added to those previously in effect.

\item[{\tt )stats}]
causes the display of statistics collected by the use of the
{\tt )count} and {\tt )timer} options.

\item[{\tt )stats reset}]
resets to 0 the statistics collected by the use of the
{\tt )count} and {\tt )timer} options.

\item[{\tt )timer}]
causes the system to keep a count of execution times for the
traced function.  The total can be displayed with {\tt )trace
)stats} and cleared with {\tt )trace )stats reset}.

%!! only for lisp, boot, may not work in any case, recommend removing
%\item[{\tt )varbreak}]
%causes a \Lisp{} break loop to be entered after
%the assignment to any variable in the traced function.

\item[{\tt )varbreak} {\it var1 \lanb{}... varN\ranb{}}]
causes a \Lisp{} break loop to be entered after
the assignment to any of the listed variables in the traced
function.

\item[{\tt )vars}]
causes the display of the value of any variable after it is
assigned in the traced function.
Note that library code must
have been compiled (see \spadref{ugSysCmdcompile})
using the {\tt )vartrace} option in order
to support this option.

\item[{\tt )vars} {\it var1 \lanb{}... varN\ranb{}}]
causes the display of the value of any of the specified variables
after they are assigned in the traced function.
Note that library code must
have been compiled (see \spadref{ugSysCmdcompile})
using the {\tt )vartrace} option in order
to support this option.

\item[{\tt )within} {\it executingFunction}]
causes the display of trace information only if the traced
function is called when the given {\it executingFunction} is running.
\end{description}

The following are the options for tracing constructors, domains
and packages.

\begin{description}
\item[{\tt )local} {\it \lanb{}op1 \lanb{}... opN\ranb{}\ranb{}}]
causes local functions of the constructor to be traced.  Note that
to untrace an individual local function, you must use the fully
qualified internal name, using the escape character
\spadSyntax{_} before the semicolon.
\begin{verbatim}
)trace FRAC )local
)trace FRAC_;cancelGcd )off
\end{verbatim}

\item[{\tt )ops} {\it op1 \lanb{}... opN\ranb{}}]
By default, all operations from a domain or package are traced
when the domain or package is traced.  This option allows you to
specify that only particular operations should be traced.  The
command
%
\begin{verbatim}
)trace Integer )ops min max _+ _-
\end{verbatim}
%
traces four operations from the domain \spadtype{Integer}.  Since
{\tt +} and {\tt -} are special
characters, it is necessary
to escape them with an underscore.
\end{description}

\par\noindent{\bf Also See:}
\titledspadref{{\tt )boot}}{ugSysCmdboot},
\titledspadref{{\tt )lisp}}{ugSysCmdlisp}, and
\titledspadref{{\tt )ltrace}}{ugSysCmdltrace}.

% *********************************************************************
\head{section}{)undo}{ugSysCmdundo}
% *********************************************************************
\syscmdindex{undo}


\par\noindent{\bf User Level Required:} interpreter

\par\noindent{\bf Command Syntax:}
\begin{simpleList}
\item{\tt )undo}
\item{\tt )undo} {\it integer}
\item{\tt )undo} {\it integer \lanb{}option\ranb{}}
\item{\tt )undo} {\tt )redo}
\end{simpleList}
%
where {\it option} is one of
%
\begin{simpleList}
\item{\tt )after}
\item{\tt )before}
\end{simpleList}

\par\noindent{\bf Command Description:}

This command is used to
restore the state of the user environment to an earlier
point in the interactive session.
The argument of an {\tt )undo} is an integer which must designate some
step number in the interactive session.

\begin{verbatim}
)undo n
)undo n )after
\end{verbatim}
These commands return the state of the interactive
environment to that immediately after step {\tt n}.
If {\tt n} is a positive number, then {\tt n} refers to step number
{\tt n}. If {\tt n} is a negative number, it refers to the \eth{\tt n}
previous command (that is, undoes the effects of the last \smath{-n}
commands).

A {\tt )clear all} resets the {\tt )undo} facility.
Otherwise, an {\tt )undo} undoes the effect of {\tt )clear} with
options {\tt properties}, {\tt value}, and {\tt mode}, and
that of a previous {\tt undo}.
If any such system commands are given between steps \smath{n} and
\smath{n + 1} (\smath{n > 0}), their effect is undone
for {\tt )undo m} for any \smath{0 < m \leq n}..

The command {\tt )undo} is equivalent to {\tt )undo -1} (it undoes
the effect of the previous user expression).
The command {\tt )undo 0} undoes any of the above system commands
issued since the last user expression.

\begin{verbatim}
)undo n )before
\end{verbatim}
This command returns the state of the interactive
environment to that immediately before step {\tt n}.
Any {\tt )undo} or {\tt )clear} system commands
given before step {\tt n} will not be undone.

\begin{verbatim}
)undo )redo
\end{verbatim}
This command reads the file {\tt redo.input}.
created by the last {\tt )undo} command.
This file consists of all user input lines, excluding those
backtracked over due to a previous {\tt )undo}.

The command {\tt )history )write} will eliminate the ``undone'' command
lines of your program.

\par\noindent{\bf Also See:}
\titledspadref{{\tt )history}}{ugSysCmdhistory}.

% *********************************************************************
\head{section}{)what}{ugSysCmdwhat}
% *********************************************************************
\syscmdindex{what}


\par\noindent{\bf User Level Required:} interpreter

\par\noindent{\bf Command Syntax:}
\begin{simpleList}
\item{\tt )what categories} {\it pattern1} \lanb{}{\it pattern2 ...\ranb{}}
\item{\tt )what commands  } {\it pattern1} \lanb{}{\it pattern2 ...\ranb{}}
\item{\tt )what domains   } {\it pattern1} \lanb{}{\it pattern2 ...\ranb{}}
\item{\tt )what operations} {\it pattern1} \lanb{}{\it pattern2 ...\ranb{}}
\item{\tt )what packages  } {\it pattern1} \lanb{}{\it pattern2 ...\ranb{}}
\item{\tt )what synonym   } {\it pattern1} \lanb{}{\it pattern2 ...\ranb{}}
\item{\tt )what things    } {\it pattern1} \lanb{}{\it pattern2 ...\ranb{}}
\item{\tt )apropos        } {\it pattern1} \lanb{}{\it pattern2 ...\ranb{}}
\end{simpleList}

\par\noindent{\bf Command Description:}

This command is used to display lists of things in the system.  The
patterns are all strings and, if present, restrict the contents of the
lists.  Only those items that contain one or more of the strings as
substrings are displayed.  For example,
\begin{verbatim}
)what synonym
\end{verbatim}
displays all command synonyms,
\begin{verbatim}
)what synonym ver
\end{verbatim}
displays all command synonyms containing the substring ``{\tt ver}'',
\begin{verbatim}
)what synonym ver pr
\end{verbatim}
displays all command synonyms
containing the substring  ``{\tt ver}'' or  the substring
``{\tt pr}''.
Output similar to the following will be displayed
\begin{verbatim}
---------------- System Command Synonyms -----------------

user-defined synonyms satisfying patterns:
      ver pr

  )apr ........................... )what things
  )apropos ....................... )what things
  )prompt ........................ )set message prompt
  )version ....................... )lisp *yearweek*
\end{verbatim}

Several other things can be listed with the {\tt )what} command:

\begin{description}
\item[{\tt categories}] displays a list of category constructors.
\syscmdindex{what categories}
\item[{\tt commands}]  displays a list of  system commands available  at your
user-level.
\syscmdindex{what commands}
Your user-level
\index{user-level}
is set via the  {\tt )set userlevel} command.
\syscmdindex{set userlevel}
To get a description of a particular command, such as ``{\tt )what}'', issue
{\tt )help what}.
\item[{\tt domains}]   displays a list of domain constructors.
\syscmdindex{what domains}
\item[{\tt operations}] displays a list of operations in  the system library.
\syscmdindex{what operations}
It  is recommended that you  qualify this command with one or
more patterns, as there are thousands of operations available.  For
example, say you are looking for functions that involve computation of
eigenvalues.  To find their names, try {\tt )what operations eig}.
A rather large list of operations  is loaded into the workspace when
this command  is first issued.  This  list will be deleted  when you
clear the workspace  via {\tt )clear all} or {\tt )clear completely}.
It will be re-created if it is needed again.
\item[{\tt packages}]  displays a list of package constructors.
\syscmdindex{what packages}
\item[{\tt synonym}]  lists system command synonyms.
\syscmdindex{what synonym}
\item[{\tt things}]    displays all  of the  above types for  items containing
\syscmdindex{what things}
the pattern strings as  substrings.
The command synonym  {\tt )apropos} is equivalent to
\syscmdindex{apropos}
{\tt )what things}.
\end{description}

\par\noindent{\bf Also See:}
\titledspadref{{\tt )display}}{ugSysCmddisplay},
\titledspadref{{\tt )set}}{ugSysCmdset}, and
\titledspadref{{\tt )show}}{ugSysCmdshow}.
\begin{SysCmdOutput}
\end{SysCmdOutput}

%\input{ug17}
%\input{ug18}
%\input{ug19}
%\input{ug20}
% !! DO NOT MODIFY THIS FILE BY HAND !! Created by spool2tex.awk.

% Copyright (c) 1991-2002, The Numerical ALgorithms Group Ltd.
% All rights reserved.
%
% Redistribution and use in source and binary forms, with or without
% modification, are permitted provided that the following conditions are
% met:
%
%     - Redistributions of source code must retain the above copyright
%       notice, this list of conditions and the following disclaimer.
%
%     - Redistributions in binary form must reproduce the above copyright
%       notice, this list of conditions and the following disclaimer in
%       the documentation and/or other materials provided with the
%       distribution.
%
%     - Neither the name of The Numerical ALgorithms Group Ltd. nor the
%       names of its contributors may be used to endorse or promote products
%       derived from this software without specific prior written permission.
%
% THIS SOFTWARE IS PROVIDED BY THE COPYRIGHT HOLDERS AND CONTRIBUTORS "AS
% IS" AND ANY EXPRESS OR IMPLIED WARRANTIES, INCLUDING, BUT NOT LIMITED
% TO, THE IMPLIED WARRANTIES OF MERCHANTABILITY AND FITNESS FOR A
% PARTICULAR PURPOSE ARE DISCLAIMED. IN NO EVENT SHALL THE COPYRIGHT OWNER
% OR CONTRIBUTORS BE LIABLE FOR ANY DIRECT, INDIRECT, INCIDENTAL, SPECIAL,
% EXEMPLARY, OR CONSEQUENTIAL DAMAGES (INCLUDING, BUT NOT LIMITED TO,
% PROCUREMENT OF SUBSTITUTE GOODS OR SERVICES-- LOSS OF USE, DATA, OR
% PROFITS-- OR BUSINESS INTERRUPTION) HOWEVER CAUSED AND ON ANY THEORY OF
% LIABILITY, WHETHER IN CONTRACT, STRICT LIABILITY, OR TORT (INCLUDING
% NEGLIGENCE OR OTHERWISE) ARISING IN ANY WAY OUT OF THE USE OF THIS
% SOFTWARE, EVEN IF ADVISED OF THE POSSIBILITY OF SUCH DAMAGE.

% *********************************************************************
\head{chapter}{Programs for FriCAS Images}{ugAppGraphics}
% *********************************************************************
%
This appendix contains the \Language{} programs used to generate
the images in the \Gallery{} color insert of this book.
All these input files are included
with the \Language{} system.
To produce the images
on page 6 of the \Gallery{} insert, for example, issue the command:
\begin{verbatim}
)read images6
\end{verbatim}

These images were produced on an IBM RS/6000 model 530 with a
standard color graphics adapter.  The smooth shaded images
were made from X Window System screen dumps.
The remaining images were produced with \Language{}-generated
PostScript output.  The images were reproduced from slides made on an Agfa
ChromaScript PostScript interpreter with a Matrix Instruments QCR camera.

% *********************************************************************
\head{section}{images1.input}{ugFimagesOne}
% *********************************************************************

\begin{xmpLinesPlain}
)read tknot                                              -- Read torus knot program.

torusKnot(15,17, 0.1, 6, 700)                            -- {\bf A (15,17) torus knot.}
\end{xmpLinesPlain}
\index{torus knot}

\newpage
% *********************************************************************
\head{section}{images2.input}{ugFimagesTwo}
% *********************************************************************

These images illustrate how Newton's method converges when computing the
\index{Newton iteration}
complex cube roots of 2.   Each point in the \smath{(x,y)}-plane represents the
complex number \smath{x + iy,} which is given as a starting point for Newton's
method.  The poles in these images represent bad starting values.
The flat areas are the regions of convergence to the three roots.

\begin{xmpLinesPlain}
)read newton                                             -- Read the programs from
)read vectors                                            -- \quad{}Chapter 10.
f := newtonStep(x^3 - 2)                                 -- Create a Newton's iteration
                                                         -- \quad{}function for $x^3 = 2$.
\end{xmpLinesPlain}

The function $f^n$ computes $n$ steps of Newton's method.

\begin{xmpLinesNoResetPlain}
clipValue := 4                                           -- Clip values with magnitude $> 4$.
drawComplexVectorField(f^3, -3..3, -3..3)                -- The vector field for $f^3$
drawComplex(f^3, -3..3, -3..3)                           -- The surface for $f^3$
drawComplex(f^4, -3..3, -3..3)                           -- The surface for $f^4$
\end{xmpLinesNoResetPlain}

% *********************************************************************
\head{section}{images3.input}{ugFimagesThree}
% *********************************************************************

\begin{xmpLinesPlain}
)r tknot
for i in 0..4 repeat torusKnot(2, 2 + i/4, 0.5, 25, 250)
\end{xmpLinesPlain}

% *********************************************************************
\head{section}{images5.input}{ugFimagesFive}
% *********************************************************************

The parameterization of the Etruscan Venus is due to George Frances.
\index{Etruscan Venus}

\begin{xmpLinesPlain}
venus(a,r,steps) ==
  surf := (u:DFLOAT, v:DFLOAT): Point DFLOAT +->
    cv := cos(v)
    sv := sin(v)
    cu := cos(u)
    su := sin(u)
    x := r * cos(2*u) * cv + sv * cu
    y := r * sin(2*u) * cv - sv * su
    z := a * cv
    point [x,y,z]
  draw(surf, 0..%pi, -%pi..%pi, var1Steps==steps,
       var2Steps==steps, title == "Etruscan Venus")

venus(5/2, 13/10, 50)                                    -- \textbf{The Etruscan Venus}
\end{xmpLinesPlain}

The Figure-8 Klein Bottle
\index{Klein bottle}
parameterization is from
``Differential Geometry and Computer Graphics'' by Thomas Banchoff,
in {\it Perspectives in Mathematics,} Anniversary of Oberwolfasch 1984,
Birkh\"{a}user-Verlag, Basel, pp. 43-60.

\begin{xmpLinesNoResetPlain}
klein(x,y) ==
  cx := cos(x)
  cy := cos(y)
  sx := sin(x)
  sy := sin(y)
  sx2 := sin(x/2)
  cx2 := cos(x/2)
  sq2 := sqrt(2.0@DFLOAT)
  point [cx * (cx2 * (sq2 + cy) + (sx2 * sy * cy)), _
         sx * (cx2 * (sq2 + cy) + (sx2 * sy * cy)), _
         -sx2 * (sq2 + cy) + cx2 * sy * cy]

draw(klein, 0..4*%pi, 0..2*%pi, var1Steps==50,           -- \textbf{Figure-8 Klein bottle}
     var2Steps==50,title=="Figure Eight Klein Bottle")
\end{xmpLinesNoResetPlain}

The next two images are examples of generalized tubes.

\begin{xmpLinesNoResetPlain}
)read ntube
rotateBy(p, theta) ==                                    -- Rotate a point $p$ by
  c := cos(theta)                                        -- \quad{}$\theta$ around the origin.
  s := sin(theta)
  point [p.1*c - p.2*s, p.1*s + p.2*c]

bcircle t ==                                             -- A circle in three-space.
  point [3*cos t, 3*sin t, 0]

twist(u, t) ==                                           -- An ellipse that twists
  theta := 4*t                                           -- \quad{}around four times as
  p := point [sin u, cos(u)/2]                           -- \quad{}\spad{t} revolves once.
  rotateBy(p, theta)

ntubeDrawOpt(bcircle, twist, 0..2*%pi, 0..2*%pi,         -- \bf{Twisted Torus}
             var1Steps == 70, var2Steps == 250)

twist2(u, t) ==                                          -- Create a twisting circle.
  theta := t
  p := point [sin u, cos(u)]
  rotateBy(p, theta)

cf(u,v) == sin(21*u)                                     -- Color function with \spad{21} stripes.

ntubeDrawOpt(bcircle, twist2, 0..2*%pi, 0..2*%pi,        -- \textbf{Striped Torus}
  colorFunction == cf, var1Steps == 168,
  var2Steps == 126)
\end{xmpLinesNoResetPlain}

% *********************************************************************
\head{section}{images6.input}{ugFimagesSix}
% *********************************************************************

\begin{xmpLinesPlain}
gam(x,y) ==                                              -- The height and color are the
  g := Gamma complex(x,y)                                -- \quad{}real and argument parts
  point [x,y,max(min(real g, 4), -4), argument g]        -- \quad{}of the Gamma function,
                                                         -- \quad{}respectively.

draw(gam, -%pi..%pi, -%pi..%pi,                          -- \textbf{The Gamma Function}
     title == "Gamma(x + %i*y)", _
     var1Steps == 100, var2Steps == 100)

b(x,y) == Beta(x,y)

draw(b, -3.1..3, -3.1 .. 3, title == "Beta(x,y)")        -- \textbf{The Beta Function}

atf(x,y) ==
  a := atan complex(x,y)
  point [x,y,real a, argument a]

draw(atf, -3.0..%pi, -3.0..%pi)                          -- \textbf{The Arctangent function}
\end{xmpLinesPlain}
\index{function!Gamma}
\index{function!Euler Beta}
\index{Euler!Beta function}


% *********************************************************************
\head{section}{images7.input}{ugFimagesSeven}
% *********************************************************************

First we look at the conformal
\index{conformal map}
map $z \mapsto z + 1/z$.

\begin{xmpLinesPlain}
)read conformal                                          -- Read program for drawing
                                                         -- \quad{}conformal maps.

f z == z                                                 -- The coordinate grid for the
                                                         -- \quad{}complex plane.
conformalDraw(f, -2..2, -2..2, 9, 9, "cartesian")        -- \textbf{Mapping 1: Source}

f z == z + 1/z                                           -- The map $z \mapsto z + 1/z$

conformalDraw(f, -2..2, -2..2, 9, 9, "cartesian")        -- \bf{Mapping 1: Target}
\end{xmpLinesPlain}

The map $z \mapsto -(z+1)/(z-1)$ maps
the unit disk to the right half-plane, as shown
\index{Riemann!sphere}
on the Riemann sphere.

\begin{xmpLinesNoResetPlain}
f z == z                                                 -- The unit disk.

riemannConformalDraw(f,0.1..0.99,0..2*%pi,7,11,"polar")  -- \textbf{Mapping 2: Source}

f z == -(z+1)/(z-1)                                      -- The map $x\mapsto -(z+1)/(z-1)$.
riemannConformalDraw(f,0.1..0.99,0..2*%pi,7,11,"polar")  -- \textbf{Mapping 2: Target}

riemannSphereDraw(-4..4, -4..4, 7, 7, "cartesian")       -- \textbf{Riemann Sphere Mapping}
\end{xmpLinesNoResetPlain}

% *********************************************************************
\head{section}{images8.input}{ugFimagesEight}
% *********************************************************************

\begin{xmpLinesPlain}
)read dhtri
)read tetra
drawPyramid 4                                            -- \textbf{Sierpinsky's Tetrahedron}

\index{Sierpinsky's Tetrahedron}
)read antoine
drawRings 2                                              -- \textbf{Antoine's Necklace}

\index{Antoine's Necklace}
)read scherk
drawScherk(3,3)                                          -- \textbf{Scherk's Minimal Surface}

\index{Scherk's minimal surface}
)read ribbonsNew
drawRibbons([x^i for i in 1..5], x=-1..1, y=0..2)        -- \bf{Ribbon Plot}
\end{xmpLinesPlain}


%\input{gallery/conformal.htex}
% *********************************************************************
\head{section}{conformal.input}{ugFconformal}
% *********************************************************************
%
The functions in this section draw conformal maps both on the
\index{conformal map}
plane and on the Riemann sphere.
\index{Riemann!sphere}

%   -- Compile, don't interpret functions.
%\xmpLine{)set fun comp on}{}
\begin{xmpLinesPlain}
C := Complex DoubleFloat                                 -- Complex Numbers
S := Segment DoubleFloat                                 -- Draw ranges
R3 := Point DFLOAT                                       -- Points in 3-space

\end{xmpLinesPlain}

\userfun{conformalDraw}{\it (f, rRange, tRange, rSteps, tSteps, coord)}
draws the image of the coordinate grid under {\it f} in the complex plane.
The grid may be given in either polar or Cartesian coordinates.
Argument {\it f} is the function to draw;
{\it rRange} is the range of the radius (in polar) or real (in Cartesian);
{\it tRange} is the range of $\theta$ (in polar) or imaginary (in Cartesian);
{\it tSteps, rSteps}, are the number of intervals in the {\it r} and
$\theta$ directions; and
{\it coord} is the coordinate system to use (either {\tt "polar"} or
{\tt "cartesian"}).

\begin{xmpLinesNoResetPlain}
conformalDraw: (C -> C, S, S, PI, PI, String) -> VIEW3D
conformalDraw(f,rRange,tRange,rSteps,tSteps,coord) ==
  transformC :=                                          -- Function for changing an \smath{(x,y)}
    coord = "polar" => polar2Complex                     -- \quad{}pair into a complex number.
    cartesian2Complex
  cm := makeConformalMap(f, transformC)
  sp := createThreeSpace()                               -- Create a fresh space.
  adaptGrid(sp, cm, rRange, tRange, rSteps, tSteps)      -- Plot the coordinate lines.
  makeViewport3D(sp, "Conformal Map")                    -- Draw the image.
\end{xmpLinesNoResetPlain}

\userfun{riemannConformalDraw}{\it (f, rRange, tRange, rSteps, tSteps, coord)}
draws the image of the coordinate grid under {\it f} on the Riemann sphere.
The grid may be given in either polar or Cartesian coordinates.
Its arguments are the same as those for \userfun{conformalDraw}.
\begin{xmpLinesNoResetPlain}
riemannConformalDraw:(C->C,S,S,PI,PI,String)->VIEW3D
riemannConformalDraw(f, rRange, tRange,
                     rSteps, tSteps, coord) ==
  transformC :=                                          -- Function for changing an \smath{(x,y)}
    coord = "polar" => polar2Complex                     -- \quad{}pair into a complex number.
    cartesian2Complex
  sp := createThreeSpace()                               -- Create a fresh space.
  cm := makeRiemannConformalMap(f, transformC)
  adaptGrid(sp, cm, rRange, tRange, rSteps, tSteps)      -- Plot the coordinate lines.
  curve(sp,[point [0,0,2.0@DFLOAT,0],point [0,0,2.0@DFLOAT,0]]) -- Add an invisible point at
  makeViewport3D(sp,"Map on the Riemann Sphere")         -- \quad{}the north pole for scaling.

adaptGrid(sp, f, uRange, vRange,  uSteps, vSteps) ==     -- Plot the coordinate grid
  delU := (high(uRange) - low(uRange))/uSteps            -- \quad{}using adaptive plotting for
  delV := (high(vRange) - low(vRange))/vSteps            -- \quad{}coordinate lines, and draw
  uSteps := uSteps + 1; vSteps := vSteps + 1             -- \quad{}tubes around the lines.
  u := low uRange
  for i in 1..uSteps repeat                              -- Draw coordinate lines in the \spad{v}
    c := curryLeft(f,u)                                  -- \quad{}direction; curve \spad{c} fixes the
    cf := (t:DFLOAT):DFLOAT +-> 0                        -- \quad{}current value of \spad{u}.
    makeObject(c,vRange::SEG Float,colorFunction==cf,    -- Draw the \spad{v} coordinate line.
      space == sp, tubeRadius == .02, tubePoints == 6)
    u := u + delU
  v := low vRange
  for i in 1..vSteps repeat                              -- Draw coordinate lines in the \spad{u}
    c := curryRight(f,v)                                 -- \quad{}direction; curve \spad{c} fixes the
    cf := (t:DFLOAT):DFLOAT +-> 1                        -- \quad{}current value of \spad{v}.
    makeObject(c,uRange::SEG Float,colorFunction==cf,    -- Draw the \spad{u} coordinate line.
      space == sp, tubeRadius == .02, tubePoints == 6)
    v := v + delV
  void()

riemannTransform(z) ==                                   -- Map a point in the complex
  r := sqrt norm z                                       -- \quad{}plane to the Riemann sphere.
  cosTheta := (real z)/r
  sinTheta := (imag z)/r
  cp := 4*r/(4+r^2)
  sp := sqrt(1-cp*cp)
  if r>2 then sp := -sp
  point [cosTheta*cp, sinTheta*cp, -sp + 1]

cartesian2Complex(r:DFLOAT, i:DFLOAT):C ==               -- Convert Cartesian coordinates to
  complex(r, i)                                          -- \quad{}complex Cartesian form.

polar2Complex(r:DFLOAT, th:DFLOAT):C ==                  -- Convert polar coordinates to
  complex(r*cos(th), r*sin(th))                          -- \quad{}complex Cartesian form.

makeConformalMap(f, transformC) ==                       -- Convert complex function \spad{f} to a
  (u:DFLOAT,v:DFLOAT):R3 +->                             -- \quad{}mapping: (\spadtype{DFLOAT},\spadtype{DFLOAT}) $\to$ \pspadtype{R3}
    z := f transformC(u, v)                              -- \quad{}in the complex plane.
    point [real z, imag z, 0.0@DFLOAT]

makeRiemannConformalMap(f, transformC) ==                -- Convert a complex function \spad{f} to a
  (u:DFLOAT, v:DFLOAT):R3 +->                            -- \quad{}mapping: (\spadtype{DFLOAT},\spadtype{DFLOAT}) $\to$ \spadtype{R3}
    riemannTransform f transformC(u, v)                  -- \quad{}on the Riemann sphere.

riemannSphereDraw: (S, S, PI, PI, String) -> VIEW3D      -- Draw a picture of the mapping
riemannSphereDraw(rRange,tRange,rSteps,tSteps,coord) ==  -- \quad{}of the complex plane to
  transformC :=                                          -- \quad{}the Riemann sphere.
    coord = "polar" => polar2Complex
    cartesian2Complex
  grid := (u:DFLOAT, v:DFLOAT): R3 +->                   -- Coordinate grid function.
    z1 := transformC(u, v)
    point [real z1, imag z1, 0]
  sp := createThreeSpace()                               -- Create a fresh space.
  adaptGrid(sp, grid, rRange, tRange, rSteps, tSteps)    -- Draw the flat grid.
  connectingLines(sp,grid,rRange,tRange,rSteps,tSteps)
  makeObject(riemannSphere,0..2*%pi,0..%pi,space==sp)    -- Draw the sphere.
  f := (z:C):C +-> z
  cm := makeRiemannConformalMap(f, transformC)
  adaptGrid(sp, cm, rRange, tRange, rSteps, tSteps)      -- Draw the sphere grid.
  makeViewport3D(sp, "Riemann Sphere")

connectingLines(sp,f,uRange,vRange,uSteps,vSteps) ==     -- Draw the lines that connect
  delU := (high(uRange) - low(uRange))/uSteps            -- \quad{}the points in the complex
  delV := (high(vRange) - low(vRange))/vSteps            -- \quad{}plane to the north pole
  uSteps := uSteps + 1; vSteps := vSteps + 1             -- \quad{}of the Riemann sphere.
  u := low uRange
  for i in 1..uSteps repeat                              -- For each u.
    v := low vRange
    for j in 1..vSteps repeat                            -- For each v.
      p1 := f(u,v)
      p2 := riemannTransform complex(p1.1, p1.2)         -- Project p1 onto the sphere.
      fun := lineFromTo(p1,p2)                           -- Create a line function.
      cf := (t:DFLOAT):DFLOAT +-> 3
      makeObject(fun, 0..1,space==sp,tubePoints==4,      -- Draw the connecting line.
                 tubeRadius==0.01,colorFunction==cf)
      v := v + delV
    u := u + delU
  void()

riemannSphere(u,v) ==                                    -- A sphere sitting on the
  sv := sin(v)                                           -- \quad{}complex plane, with radius 1.
  0.99@DFLOAT*(point [cos(u)*sv,sin(u)*sv,cos(v),0.0@DFLOAT])+
    point [0.0@DFLOAT, 0.0@DFLOAT, 1.0@DFLOAT, 4.0@DFLOAT]

lineFromTo(p1, p2) ==                                    -- Create a line function
  d := p2 - p1                                           -- \quad{}that goes from p1 to p2
  (t:DFLOAT):Point DFLOAT +->
    p1 + t*d
\end{xmpLinesNoResetPlain}

%\input{gallery/tknot.htex}
% *********************************************************************
\head{section}{tknot.input}{ugFtknot}
% *********************************************************************
%
Create a $(p,q)$ torus-knot with radius $r$ around the curve.
The formula was derived by Larry Lambe.

\begin{xmpLinesPlain}
)read ntube
torusKnot: (DFLOAT, DFLOAT, DFLOAT, PI, PI) -> VIEW3D
torusKnot(p, q ,r, uSteps, tSteps) ==
  knot := (t:DFLOAT):Point DFLOAT +->                    -- Function for the torus knot.
    fac := 4/(2.2@DFLOAT-sin(q*t))
    fac * point [cos(p*t), sin(p*t), cos(q*t)]
  circle := (u:DFLOAT, t:DFLOAT): Point DFLOAT +->       -- The cross section.
    r * point [cos u, sin u]
  ntubeDrawOpt(knot, circle, 0..2*%pi, 0..2*%pi,         -- Draw the circle around the knot.
               var1Steps == uSteps, var2Steps == tSteps)

\end{xmpLinesPlain}

%\input{gallery/ntube.htex}
% *********************************************************************
\head{section}{ntube.input}{ugFntube}
% *********************************************************************
%
The functions in this file create generalized tubes (also known as generalized
cylinders).
These functions draw a 2-d curve in the normal
planes around a 3-d curve.

\begin{xmpLinesPlain}
R3 := Point DFLOAT                                       -- Points in 3-Space
R2 := Point DFLOAT                                       -- Points in 2-Space
S := Segment Float                                       -- Draw ranges
                                                         -- Introduce types for functions for:
ThreeCurve := DFLOAT -> R3                               -- \quad{}---the space curve function
TwoCurve := (DFLOAT, DFLOAT) -> R2                       -- \quad{}---the plane curve function
Surface := (DFLOAT, DFLOAT) -> R3                        -- \quad{}---the surface function
                                                         -- Frenet frames define a
FrenetFrame :=                                           -- \quad{}coordinate system around a
   Record(value:R3,tangent:R3,normal:R3,binormal:R3)     -- \quad{}point on a space curve.
frame: FrenetFrame                                       -- The current Frenet frame
                                                         -- \quad{}for a point on a curve.
\end{xmpLinesPlain}

\userfun{ntubeDraw}{\it (spaceCurve, planeCurve,}
$u_0 .. u_1,$ $t_0 .. t_1)$
draws {\it planeCurve} in the normal planes of {\it spaceCurve.}
The parameter $u_0 .. u_1$ specifies
the parameter range for {\it planeCurve}
and $t_0 .. t_1$ specifies the parameter range for {\it spaceCurve}.
Additionally, the plane curve function takes
a second parameter: the current parameter of {\it spaceCurve}.
This allows the plane curve to change shape
as it goes around the space curve.
See \spadref{ugFimagesFive} for an example of this.
%
\begin{xmpLinesNoResetPlain}
ntubeDraw: (ThreeCurve,TwoCurve,S,S) -> VIEW3D
ntubeDraw(spaceCurve,planeCurve,uRange,tRange) ==
  ntubeDrawOpt(spaceCurve, planeCurve, uRange, _
               tRange, []$List DROPT)

ntubeDrawOpt: (ThreeCurve,TwoCurve,S,S,List DROPT)
    -> VIEW3D
ntubeDrawOpt(spaceCurve,planeCurve,uRange,tRange,l) ==   -- This function is similar
                                                         -- \quad{}to \userfun{ntubeDraw}, but takes
  delT:DFLOAT := (high(tRange) - low(tRange))/10000      -- \quad{}optional parameters that it
  oldT:DFLOAT := low(tRange) - 1                         -- \quad{}passes to the \spadfun{draw} command.
  fun := ngeneralTube(spaceCurve,planeCurve,delT,oldT)
  draw(fun, uRange, tRange, l)

\end{xmpLinesNoResetPlain}

\userfun{nfrenetFrame}{\it (c, t, delT)}
numerically computes the Frenet frame
about the curve {\it c} at {\it t}.
Parameter {\it delT} is a small number used to
compute derivatives.
\begin{xmpLinesNoResetPlain}
nfrenetFrame(c, t, delT) ==
  f0 := c(t)
  f1 := c(t+delT)
  t0 := f1 - f0                                          -- The tangent.
  n0 := f1 + f0
  b := cross(t0, n0)                                     -- The binormal.
  n := cross(b,t0)                                       -- The normal.
  ln := length n
  lb := length b
  ln = 0 or lb = 0 =>
      error "Frenet Frame not well defined"
  n := (1/ln)*n                                          -- Make into unit length vectors.
  b := (1/lb)*b
  [f0, t0, n, b]$FrenetFrame
\end{xmpLinesNoResetPlain}

\userfun{ngeneralTube}{\it (spaceCurve, planeCurve,}{\it  delT, oltT)}
creates a function that can be passed to the system \spadfun{draw} command.
The function is a parameterized surface for the general tube
around {\it spaceCurve}.  {\it delT} is a small number used to compute
derivatives. {\it oldT} is used to hold the current value of the
{\it t} parameter for {\it spaceCurve.}  This is an efficiency measure
to ensure that frames are only computed once for each value of {\it t}.
\begin{xmpLinesNoResetPlain}
ngeneralTube: (ThreeCurve, TwoCurve, DFLOAT, DFLOAT) -> Surface
ngeneralTube(spaceCurve, planeCurve, delT, oldT) ==
  free frame                                             -- Indicate that \spad{frame} is global.
  (v:DFLOAT, t: DFLOAT): R3 +->
    if (t ~= oldT) then                                  -- If not already computed,
      frame := nfrenetFrame(spaceCurve, t, delT)         -- \quad{}compute new frame.
      oldT := t
    p := planeCurve(v, t)
    frame.value + p.1*frame.normal + p.2*frame.binormal  -- Project \spad{p} into the normal plane.
\end{xmpLinesNoResetPlain}

%\input{gallery/dhtri.htex}
% *********************************************************************
\head{section}{dhtri.input}{ugFdhtri}
% *********************************************************************
%
Create affine transformations (DH matrices) that transform
a given triangle into another.

\begin{xmpLinesPlain}
tri2tri: (List Point DFLOAT, List Point DFLOAT) -> DHMATRIX(DFLOAT)
                                                         -- Compute a \spadtype{DHMATRIX} that
tri2tri(t1, t2) ==                                       -- \quad{}transforms \spad{t1} to \spad{t2,} where
  n1 := triangleNormal(t1)                               -- \quad{}\spad{t1} and \spad{t2} are the vertices
  n2 := triangleNormal(t2)                               -- \quad{}of two triangles in 3-space.
  tet2tet(concat(t1, n1), concat(t2, n2))

tet2tet: (List Point DFLOAT, List Point DFLOAT) -> DHMATRIX(DFLOAT)
                                                         -- Compute a \spadtype{DHMATRIX} that
tet2tet(t1, t2) ==                                       -- \quad{}transforms \spad{t1} to \spad{t2,}
  m1 := makeColumnMatrix t1                              -- \quad{}where \spad{t1} and \spad{t2} are the
  m2 := makeColumnMatrix t2                              -- \quad{}vertices of two tetrahedrons
  m2 * inverse(m1)                                       -- \quad{}in 3-space.

makeColumnMatrix(t) ==                                   -- Put the vertices of a tetra-
  m := new(4,4,0)$DHMATRIX(DFLOAT)                       -- \quad{}hedron into matrix form.
  for x in t for i in 1..repeat
    for j in 1..3 repeat
      m(j,i) := x.j
    m(4,i) := 1
  m

triangleNormal(t) ==                                     -- Compute a vector normal to
  a := triangleArea t                                    -- \quad{}the given triangle, whose
  p1 := t.2 - t.1                                        -- \quad{}length is the square root
  p2 := t.3 - t.2                                        -- \quad{}of the area of the triangle.
  c := cross(p1, p2)
  len := length(c)
  len = 0 => error "degenerate triangle!"
  c := (1/len)*c
  t.1 + sqrt(a) * c

triangleArea t ==                                        -- Compute the area of a
  a := length(t.2 - t.1)                                 -- \quad{}triangle using Heron's
  b := length(t.3 - t.2)                                 -- \quad{}formula.
  c := length(t.1 - t.3)
  s := (a+b+c)/2
  sqrt(s*(s-a)*(s-b)*(s-c))
\end{xmpLinesPlain}


% *********************************************************************
\head{section}{tetra.input}{ugFtetra}
% *********************************************************************
%
%\input{gallery/tetra.htex}
%\outdent{Sierpinsky's Tetrahedron}

\begin{xmpLinesPlain}
)set expose add con DenavitHartenbergMatrix              -- Bring DH matrices into the
                                                         -- \quad{}environment.
x1:DFLOAT := sqrt(2.0@DFLOAT/3.0@DFLOAT)                 -- Set up the coordinates of the
x2:DFLOAT := sqrt(3.0@DFLOAT)/6                          -- \quad{}corners of the tetrahedron.

z := 0.0@DFLOAT
h := 0.5@DFLOAT

p1 := point [-h, -x2, z]                                 -- Some needed points.
p2 := point [h, -x2, z]
p3 := point [z, 2*x2, z]
p4 := point [z, z, x1]

baseTriangle  := [p2, p1, p3]                            -- The base of the tetrahedron.

mt := [h*(p2+p1), h*(p1+p3), h*(p3+p2)]                  -- The ``middle triangle'' inscribed
                                                         -- \quad{}in the base of the tetrahedron.
bt1 := [mt.1, p1, mt.2]                                  -- The bases of the triangles of
bt2 := [p2, mt.1, mt.3]                                  -- \quad{}the subdivided tetrahedron.
bt3 := [mt.2, p3, mt.3]
bt4 := [h*(p2+p4), h*(p1+p4), h*(p3+p4)]

tt1 := tri2tri(baseTriangle, bt1)                        -- Create the transformations
tt2 := tri2tri(baseTriangle, bt2)                        -- \quad{}that bring the base of the
tt3 := tri2tri(baseTriangle, bt3)                        -- \quad{}tetrahedron to the bases of
tt4 := tri2tri(baseTriangle, bt4)                        -- \quad{}the subdivided tetrahedron.

drawPyramid(n) ==                                        -- Draw a Sierpinsky tetrahedron
  s := createThreeSpace()                                -- \quad{}with \spad{n} levels of recursive
  dh := rotatex(0.0@DFLOAT)                              -- \quad{}subdivision.
  drawPyramidInner(s, n, dh)
  makeViewport3D(s, "Sierpinsky Tetrahedron")

drawPyramidInner(s, n, dh) ==                            -- Recursively draw a Sierpinsky
  n = 0 => makeTetrahedron(s, dh, n)                     -- \quad{}tetrahedron.
  drawPyramidInner(s, n-1, dh * tt1)                     -- Draw the 4 recursive pyramids.
  drawPyramidInner(s, n-1, dh * tt2)
  drawPyramidInner(s, n-1, dh * tt3)
  drawPyramidInner(s, n-1, dh * tt4)

makeTetrahedron(sp, dh, color) ==                        -- Draw a tetrahedron into the
  w1 := dh*p1                                            -- \quad{}given space with the given
  w2 := dh*p2                                            -- \quad{}color, transforming it by
  w3 := dh*p3                                            -- \quad{}the given DH matrix.
  w4 := dh*p4
  polygon(sp, [w1, w2, w4])
  polygon(sp, [w1, w3, w4])
  polygon(sp, [w2, w3, w4])
  void()
\end{xmpLinesPlain}
\index{Sierpinsky's Tetrahedron}


%\input{gallery/antoine.htex}
% *********************************************************************
\head{section}{antoine.input}{ugFantoine}
% *********************************************************************
%
Draw Antoine's Necklace.
\index{Antoine's Necklace}
Thank you to Matthew Grayson at IBM's T.J Watson Research Center for the idea.

\begin{xmpLinesPlain}
)set expose add con DenavitHartenbergMatrix              -- Bring DH matrices into
                                                         -- \quad{}the environment.
torusRot: DHMATRIX(DFLOAT)                               -- The current transformation for
                                                         -- \quad{}drawing a sub ring.

drawRings(n) ==                                          -- Draw Antoine's Necklace with \spad{n}
  s := createThreeSpace()                                -- \quad{}levels of recursive subdivision.
  dh:DHMATRIX(DFLOAT) := identity()                      -- \quad{}The number of subrings is $10^n$.
  drawRingsInner(s, n, dh)                               -- Do the real work.
  makeViewport3D(s, "Antoine's Necklace")

\end{xmpLinesPlain}

In order to draw Antoine rings, we take one ring, scale it down to
a smaller size, rotate it around its central axis, translate it
to the edge of the larger ring and rotate it around the edge to
a point corresponding to its count (there are 10 positions around
the edge of the larger ring). For each of these new rings we
recursively perform the operations, each ring becoming 10 smaller
rings. Notice how the \spadtype{DHMATRIX} operations are used to build up
the proper matrix composing all these transformations.

\begin{xmpLinesNoResetPlain}
F ==> DFLOAT
drawRingsInner(s, n, dh) ==                              -- Recursively draw Antoine's
  n = 0 =>                                               -- \quad{}Necklace.
    drawRing(s, dh)
    void()
  t := 0.0@F                                             -- Angle around ring.
  p := 0.0@F                                             -- Angle of subring from plane.
  tr := 1.0@F                                            -- Amount to translate subring.
  inc := 0.1@F                                           -- The translation increment.
  for i in 1..10 repeat                                  -- Subdivide into 10 linked rings.
    tr := tr + inc
    inc := -inc
    dh' := dh*rotatez(t)*translate(tr,0.0@F,0.0@F)*      -- Transform ring in center
           rotatey(p)*scale(0.35@F, 0.48@F, 0.4@F)       -- \quad{}to a link.
    drawRingsInner(s, n-1, dh')
    t := t + 36.0@F
    p := p + 90.0@F
  void()

drawRing(s, dh) ==                                       -- Draw a single ring into
  free torusRot                                          -- \quad{}the given subspace,
  torusRot := dh                                         -- \quad{}transformed by the given
  makeObject(torus, 0..2*%pi, 0..2*%pi, var1Steps == 6,  -- \quad{}\spadtype{DHMATRIX}.
             space == s, var2Steps == 15)

torus(u ,v) ==                                           -- Parameterization of a torus,
  cu := cos(u)/6                                         -- \quad{}transformed by the
  torusRot*point [(1+cu)*cos(v),(1+cu)*sin(v),(sin u)/6] -- \quad{}\spadtype{DHMATRIX} in \spad{torusRot.}
\end{xmpLinesNoResetPlain}

%\input{gallery/scherk.htex}
% *********************************************************************
\head{section}{scherk.input}{ugFscherk}
% *********************************************************************
%

Scherk's minimal surface, defined by:
\index{Scherk's minimal surface}
$e^z \cos(x) = \cos(y)$.
See: {\it A Comprehensive Introduction to Differential Geometry,} Vol. 3,
by Michael Spivak, Publish Or Perish, Berkeley, 1979, pp. 249-252.

\begin{xmpLinesPlain}
(xOffset, yOffset):DFLOAT                                -- Offsets for a single piece
                                                         -- \quad{}of Scherk's minimal surface.

drawScherk(m,n) ==                                       -- Draw Scherk's minimal surface
  free xOffset, yOffset                                  -- \quad{}on an \spad{m} by \spad{n} patch.
  space := createThreeSpace()
  for i in 0..m-1 repeat
    xOffset := i*%pi
    for j in 0 .. n-1 repeat
      rem(i+j, 2) = 0 => 'iter                           -- Draw only odd patches.
      yOffset := j*%pi
      drawOneScherk(space)                               -- Draw a patch.
  makeViewport3D(space, "Scherk's Minimal Surface")

scherk1(u,v) ==                                          -- The first patch that makes
  x := cos(u)/exp(v)                                     -- \quad{}up a single piece of
  point [xOffset + acos(x), yOffset + u, v, abs(v)]      -- \quad{}Scherk's minimal surface.

scherk2(u,v) ==                                          -- The second patch.
  x := cos(u)/exp(v)
  point [xOffset - acos(x), yOffset + u, v, abs(v)]

scherk3(u,v) ==                                          -- The third patch.
  x := exp(v) * cos(u)
  point [xOffset + u, yOffset + acos(x), v, abs(v)]

scherk4(u,v) ==                                          -- The fourth patch.
  x := exp(v) * cos(u)
  point [xOffset + u, yOffset - acos(x), v, abs(v)]

drawOneScherk(s) ==                                      -- Draw the surface by
  makeObject(scherk1,-%pi/2..%pi/2,0..%pi/2,space==s,    -- \quad{}breaking it into four
             var1Steps == 28, var2Steps == 28)           -- \quad{}patches and then drawing
  makeObject(scherk2,-%pi/2..%pi/2,0..%pi/2,space==s,    -- \quad{}the patches.
             var1Steps == 28, var2Steps == 28)
  makeObject(scherk3,-%pi/2..%pi/2,-%pi/2..0,space==s,
             var1Steps == 28, var2Steps == 28)
  makeObject(scherk4,-%pi/2..%pi/2,-%pi/2..0,space==s,
             var1Steps == 28, var2Steps == 28)
  void()
\end{xmpLinesPlain}
\begin{SysCmdOutput}
\end{SysCmdOutput}

%
\printindex
\end{document}
